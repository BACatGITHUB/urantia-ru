\upaper{143}{Путь через Самарию}
\author{Комиссия срединников}
\vs p143 0:1 В конце июня 27 года н.э. из\hyp{}за возросшей враждебности еврейских религиозных правителей Иисус и двенадцать апостолов отбыли из Иерусалима, перед этим оставив свои палатки и немногочисленные личные вещи на хранение в дом Лазаря в Вифании. Идя на север вглубь Самарии, они остановились в Вефиле, чтобы провести там субботу. Здесь в течение нескольких дней они проповедовали народу, приходившему из Гофны и Ефреима. Пригласить Иисуса посетить их селения пришла группа жителей Аримафеи и Фамны. Учитель и его апостолы провели более двух недель, уча евреев и самарян этой местности, многие из которых приходили даже из Антипатриды, чтобы услышать благую весть царства.
\vs p143 0:2 Жители южной Самарии с радостью слушали Иисуса, и апостолы, за исключением Иуды Искариота, сумели преодолеть значительную часть своего предубеждения против самарян. Иуде же было очень трудно олюбить этих самарян. В последнюю неделю июля Иисус и его сподвижники приготовились отправиться в новые греческие города Фазалис и Архелай, расположенные вблизи Иордана.
\usection{1. Проповедь в Архелае}
\vs p143 1:1 Первую половину августа апостолы обосновались в греческих городах Архелае и Фазалисе, где они обрели свой первый опыт проповеди в основном среди неевреев --- греков, римлян и сирийцев, --- ибо евреев в этих двух греческих городах было мало. В общении с этими римскими гражданами апостолы столкнулись с новыми трудностями, возвещая весть о грядущем царстве, и встретились с новыми возражениями против учений Иисуса. Во время одного из многочисленных вечерних собраний со своими апостолами Иисус внимательно выслушал эти возражения против евангелия царства, когда двенадцать апостолов рассказывали о переживаниях, испытанных ими в процессе индивидуальной работы.
\vs p143 1:2 Вопрос, заданный Филиппом, был типичен, ибо в нем отразились затруднения с которыми они столкнулись. Филипп сказал: «Учитель, эти греки и римляне несерьезно относятся к нашей вести, говоря, что подобные учения годятся только для слабых и рабов. Они утверждают, что религия язычников выше наших учений, поскольку она побуждает к воспитанию сильного, крепкого и активного характера. Они утверждают, что мы превратили бы всех людей в слабых пассивных непротивленцев, которые вскоре исчезнут с лица земли. Ты нравишься им, Учитель, и они открыто признают, что твое учение божественно и идеально, но они не примут нас всерьез. Они утверждают, будто твоя религия не для мира сего; что люди не могут жить, как ты учишь. Учитель, что же нам говорить этим неевреям?»
\vs p143 1:3 Выслушав подобные же возражения против евангелия царства и от Фомы, Нафанаила, Симона Зилота и Матфея, Иисус сказал двенадцати:
\vs p143 1:4 «Я пришел в этот мир, чтобы исполнить волю моего Отца и открыть всему человечеству его милосердную сущность. Такова моя миссия, братья мои. И это единственное, что я буду делать, невзирая на непонимание моих учений евреями и неевреями этого или другого времени. Однако вы не должны пренебрегать тем, что даже божественная любовь требует строгой дисциплины. Любовь отца к своему сыну зачастую вынуждает отца удерживать свое беспечное потомство от неблагоразумных поступков. Ребенок не всегда понимает мудрые и вызванные любовью мотивы предупредительных дисциплинарных мер отца. Но я объявляю вам, что Отец мой в Раю правит вселенной вселенных неодолимой силой своей любви. Любовь есть величайшая из всех духовных реальностей. Истина --- это освобождающее откровение, любовь же --- высшая форма отношений. И неважно, какие грубые ошибки делают ваши ближние, управляя миром сегодня, в грядущей эре этим самым миром будет управлять евангелие, которое я возвещаю вам. Конечной целью человеческого прогресса является благоговейное признание отцовства Бога и материализация человеческого братства в любви.
\vs p143 1:5 Но кто сказал вам, что мое евангелие предназначено лишь для рабов и слабых? Разве вы, избранные мною апостолы, похожи на слабых? Разве выглядел Иоанн как слабое существо? Разве можно сказать, что я порабощен страхом? Да, евангелие проповедано бедным и угнетенным этого поколения. Религии мира сего пренебрегали бедными, но Отец мой не взирает на лица. А кроме того, бедные этого времени --- первые, кто внял призыву к покаянию и признанию сыновства. Евангелие царства должно проповедоваться всем людям --- евреям и неевреям, грекам и римлянам, богатым и бедным, свободным и несвободным --- одинаково молодым и старым, мужчинам и женщинам.
\vs p143 1:6 Не думайте, что служение царству должно быть неизменно легким, потому что Отец мой --- Бог любви и любит миловать. Восхождение к Раю есть верховный путь всех времен, трудное достижение вечности. Служение царству на земле потребует всего мужества и отваги, на которые вы и ваши соратники только способны. Многих из вас предадут смерти за вашу верность евангелию этого царства. Легко умереть в строю на поле физической брани, когда ваша смелость укрепляется присутствием товарищей, сражающихся вместе с вами, но для того, чтобы спокойно и в полном одиночестве положить жизнь за любовь к истине, хранимой в вашем смертном сердце, потребуются более высокие и более сильные человеческие отвага и преданность.
\vs p143 1:7 Сегодня неверующие могут насмехаться над вами за проповедь евангелия непротивления и за жизнь без насилия, но вы --- первые добровольцы из длинной череды искренне верующих в евангелие этого царства, которые поразят все человечество своей героической преданностью этим учениям. Никакие армии мира никогда не проявляли большей смелости и отваги, чем те, что будут проявлены вами и вашими верными последователями, которые пойдут по всему миру, провозглашая благую весть об отцовстве Бога и братстве людей. Отвага плоти --- низшая форма храбрости. Смелость ума --- более высокий тип человеческой храбрости, однако высочайшей и верховной ее формой является бескомпромиссная верность просвященным убеждениям в глубочайших духовных реальностях. Подобная отвага и есть героизм человека, знающего Бога. Вы же все есть люди, знающие Бога; вы воистину личные сподвижники Сына Человеческого».
\vs p143 1:8 \pc Это было далеко не все, что сказал Иисус по этому поводу, но таковы были первые слова его обращения, и он еще долго продолжал говорить, усиливая и поясняя это высказывание. Это была одна из наиболее страстных речей, с которыми Иисус когда\hyp{}либо обращался к двенадцати. Учитель редко говорил со своими апостолами столь эмоционально, но это был один из тех немногочисленных примеров, когда он говорил с явной серьезностью, сопровождаемой заметными эмоциями.
\vs p143 1:9 \pc Влияние его слов на публичную проповедь и личное служение апостолов не замедлило сказаться; с того самого дня их послания стали по\hyp{}новому смелы. Двенадцать апостолов продолжали черпать дух положительной настойчивости в новом евангелии царства. Впредь с этого дня они уже не тратили столько времени на проповедь отрицающих добродетелей и пассивных предписаний многостороннего учения своего Учителя.
\usection{2. Урок об умении владеть собой}
\vs p143 2:1 Учитель являл собой совершенный образец человеческого самообладания. Будучи объектом злословия, он не злословил; страдая, он своим мучителям не угрожал; поносимый своими врагами, он просто предавал себя праведному суду Отца Небесного.
\vs p143 2:2 \pc Во время одной из вечерних бесед Андрей спросил Иисуса: «Учитель, должны ли мы предаться самоотречению, как учил нас Иоанн, или нам следует стремиться к самообладанию, согласно твоему учению? В чем отличается твое учение от учения Иоанна?» Иисус ответил: «Иоанн, действительно, учил вас пути праведности в соответствии с учениями и законом своих отцов, и это была религия самоанализа и самоотречения. Однако я пришел с новым посланием о самозабвении и самообладании. Я показываю вам путь жизни, каким мне открыл его Отец мой Небесный.
\vs p143 2:3 Истинно, истинно говорю вам, владеющий собою сильнее завоевателя города. Умение владеть собой есть мера нравственной природы человека и показатель его духовного развития. Согласно прежнему порядку вы постились и молились; как новые же существа, заново рожденные от духа, вы должны верить и радоваться. В царстве Отца вы должны стать новыми созданиями; старое должно пройти; вот я показываю вам, как все должно стать новым. И своей любовью друг к другу вы должны убедить мир в том, что вы перешли из рабства в свободу, из смерти в жизнь вечную.
\vs p143 2:4 Стоя на старом пути, вы пытаетесь сдерживаться, повиноваться и подчиняться обыденным правилам жизни; встав же на новую стезю, вы прежде всего \bibemph{преобразуетесь} Духом Истины и, таким образом, постоянным духовным обновлением вашего ума укрепляетесь внутренне так, что наделяетесь силой уверенно и с радостью исполнять волю Бога, благую, угодную и совершенную. Не забывайте --- это личная вера ваша в великие и бесценные обетования Бога делает все, чтобы вы стали причастны к божественному естеству. Таким образом, благодаря вере вашей и преобразованию духа, вы поистине становитесь храмами Бога, и дух его воистину живет в вас. Если же дух живет в вас, значит, вы более не рабы плоти, но свободные и вольные сыновья духа. Новый закон духа дарует вам свободу самообладания вместо прежнего закона страха самообуздания и рабства самоотречения.
\vs p143 2:5 Множество раз, совершив зло, вы пытались объяснить ваши деяния кознями нечистого, тогда как в действительности являлись жертвами своих же собственных естественных наклонностей. Разве не сказал вам давным\hyp{}давно пророк Иеремия, что лукаво сердце человеческое более всего и порою даже крайне испорчено? Как же легко вы поддаетесь самообману, а потому предаетесь глупым страхам, разным страстям, порабощающим удовольствиям, злобе, зависти и даже мстительной ненависти!
\vs p143 2:6 Спасение --- в возрождении духа, а не в самодовольных поступках плоти. Вы оправданы верой и усыновлены благодатью, а не страхом и самоотречением плоти, тем не менее, дети Отца, рожденные от духа, всегда и вечно \bibemph{владеют} собой и всем, что порождено желаниями плоти. Зная, что вы спасены верой, вы пребываете в истинном мире с Богом. И всем, кто идет по пути этого небесного мира, суждено быть посвященным в вечное служение постоянно совершенствующихся сынов вечного Бога. Впредь очищать себя от всякого зла разума и плоти --- отнюдь не долг, а ваша возвышенная привилегия, поскольку вы стремитесь к совершенству в любви Бога.
\vs p143 2:7 Ваше сыновство основано на вере, и вы должны быть недоступны страху. Ваша радость рождена упованием на божественное слово, и вы, следовательно, не должны поддаваться сомнениям в реальности любви и милосердия Отца. Сама благость Божия ведет людей к истинному и подлинному покаянию. Секрет вашего умения владеть собой связан с вашей верой в дух, пребывающий в вас, который всегда движим любовью. Даже эта спасительная вера, которая есть у вас, --- не от вас; она --- тоже дар Бога. Если же вы дети сей живой веры, значит, вы более не зависимые рабы, а просветленные хозяева самих себя, вольные сыны Бога.
\vs p143 2:8 Если вы, дети мои, рождены от духа, значит, вы навсегда избавлены от покорной жизни в самоотречении и заботе о желаниях плоти и перенесены в счастливое царство духа, где без всякого принуждения будете являть плоды духа в своей ежедневной жизни; плоды же духа есть признак приносящего наслаждение и облагораживающего самообладания, и даже сущность наивысшего уровня достижений смертного на земле --- истинного умения владеть самим собой».
\usection{3. Развлечение и отдых}
\vs p143 3:1 Приблизительно в это время апостолами и их ближайшими последователями овладело состояние сильной нервозности и эмоционального напряжения. Они еще не вполне привыкли жить и работать вместе. Они испытывали все большие трудности, пытаясь поддерживать согласие с последователями Иоанна. Общение с неевреями и самарянами было для этих евреев великим испытанием. А кроме всего этого, последние высказывания Иисуса еще больше углубили беспокойное состояние их умов. Андрей был почти на грани срыва. Он не знал, что делать дальше, и поэтому со своими проблемами и смятением пошел к Учителю. Выслушав рассказ главы апостолов о его проблемах, Иисус сказал: «Андрей, людей разговором от смятения не избавишь, тем более, когда люди столь обеспокоены, и когда речь идет о стольких личностях со столь сильными чувствами. То, о чем ты просишь меня, я сделать не могу --- я не буду участвовать в устранении этих лично\hyp{}общественных затруднений, --- но разделю с вами радости покоя и отдыха в течение трех дней. Пойди к своим братьям и объяви, что все вы должны пойти со мной на гору Сартаба, где я хотел бы отдохнуть денек\hyp{}другой.
\vs p143 3:2 Сейчас ты пойдешь к каждому из одиннадцати своих братьев и, говоря с ним наедине, скажешь: „Учитель желает, чтобы мы на какое\hyp{}то время ушли вместе с ним и предались покою и отдыху. Поскольку все мы недавно пережили большое томление духа и напряжение ума, я предлагаю во время этого отдыха не упоминать о наших испытаниях и тревогах. Могу я положиться на твою помощь в этом деле?“ С этим предложением лично и наедине поговори с каждым из твоих собратьев». И Андрей все сделал так, как велел ему Учитель.
\vs p143 3:3 \pc Это было чудесным событием в опыте каждого из них; они никогда не забывали это восхождение на гору, продолжавшееся целый день. На протяжении всего пути ни слова не было сказано об их тревогах. Поднявшись на вершину горы, Иисус усадил их вокруг себя и сказал: «Братья мои, вы все должны осознать важность и значение покоя и пользу отдыха. Вы должны понять, что лучший способ решения некоторых запутанных проблем --- забыть о них на какое\hyp{}то время. Вернувшись свежими после отдыха или молитвы, вы сможете заняться своими бедами с более ясной головой и более твердой рукой, не говоря уже об исполненном большей решимости сердце. Кроме того, вы обнаружите, что, пока вы отдыхали умом и телом, ваша проблема во многом сократилась и потеряла остроту».
\vs p143 3:4 На следующий день Иисус дал каждому из двенадцати апостолов тему для беседы. Весь день был посвящен воспоминаниям и обсуждению дел, не связанных с их религиозной работой. Они на какой\hyp{}то миг были потрясены, когда Иисус, преломив хлеб перед полуденной трапезой, не стал даже --- вслух --- произносить молитву. Это был первый случай, когда они увидели, что Иисус пренебрег подобными формальностями.
\vs p143 3:5 Когда они поднимались на гору, Андрей был погружен в свои проблемы. Сердце Иоанна пребывало в невероятном смятении. Иаков был горько опечален в душе. Матфей испытывал затруднения в средствах, так как они жили среди неевреев. Петр находился в нервном состоянии и последнее время был более неуравновешен, чем обычно. Иуда страдал от периодических приступов обидчивости и себялюбия. Симон был очень расстроен, пытаясь примирить свой патриотизм с любовью к братству людей. Филипп пребывал во все большем и большем замешательстве от того, как развивались события. Нафанаил несколько утратил свой юмор, с тех пор как они общались с неевреями, а Фома погрузился в глубокую депрессию. Только близнецы оставались естественными и невозмутимыми. Все они были крайне озабочены тем, как мирно наладить отношения с последователями Иоанна.
\vs p143 3:6 На третий день, когда они тронулись в обратный путь к своему лагерю и стали спускаться с горы, в них произошла великая перемена. Они сделали важное открытие о том, что многих человеческих проблем в действительности не существует, что многие гнетущие тревоги являются порождением преувеличенного страха и следствием гипертрофированных опасений. Они узнали, что со всеми проблемами подобного рода лучше всего можно справиться, забыв о них; уйдя от этих проблем, они позволили им разрешиться сами по себе.
\vs p143 3:7 Их возвращение с этого отдыха отметило собой начало периода значительного улучшения отношений с последователями Иоанна. Многие из двенадцати апостолов, заметив у всех перемену в состоянии ума и почувствовав свободу от нервной раздражительности, которую они обрели благодаря трем дням отдыха от рутинных жизненных обязанностей, предались настоящему веселью. Всегда есть опасность, что однообразие человеческого общения может значительно приумножить проблемы и увеличить трудности.
\vs p143 3:8 \pc Немногие из неевреев, живших в двух греческих городах --- Архелае и Фазалисе, поверили в евангелие, но двенадцать апостолов обрели ценный опыт в этой своей первой обширной работе с исключительно нееврейским населением. Приблизительно в середине месяца в понедельник утром Иисус сказал Андрею: «Мы идем в Самарию». И они сразу направились в город Сихарь неподалеку от колодезя Иаковлева.
\usection{4. Евреи и самаряне}
\vs p143 4:1 Более шестисот лет евреи из Иудеи, а позднее и Галилеи, враждовали с самарянами. Эта враждебность между евреями и самарянами возникла так: приблизительно за семьсот лет до н.э. ассирийский царь Саргон, подавив восстание в центральной Палестине, увел в рабство более двадцати пяти тысяч евреев из Северного Израильского царства, а на их место поселил почти такое же число потомков кутитов, сефарвитов и хамафитов. Позднее Ассурбанипал также послал в Самарию колонистов для создания новых поселений.
\vs p143 4:2 Религиозная вражда между евреями и самарянами восходит ко времени возвращения евреев из вавилонского плена, когда самаряне старались помешать восстановлению Иерусалима. Позднее они нанесли обиду евреям, оказав дружеское содействие армиям Александра. В ответ за их дружбу Александр дал самарянам разрешение построить храм на горе Гаризим, где они поклонялись Яхве и богам своего племени, принося жертвы почти так же, как это делалось на службах в Иерусалимском храме. Этот культ они продолжали соблюдать по крайней мере до времен Маккавеев, когда Иоанн Гиркан разрушил их храм на горе Гаризим. Трудясь на благо самарян после смерти Иисуса, апостол Филипп провел множество встреч на месте этого древнего храма самарян.
\vs p143 4:3 Вражда между евреями и самарянами была исторической и освященной веками; со времен Александра они все меньше и меньше общались друг с другом. Двенадцать апостолов были согласны проповедовать в греческих и других нееврейских городах Десятиградия и Сирии, но когда Учитель сказал: «Пойдем в Самарию», их преданность ему подверглась суровому испытанию. Однако более чем годичное пребывание их с Иисусом воспитало в них такую личную преданность, которая превзошла даже их веру в его учение и их предрассудки против самарян.
\usection{5. Женщина из Сихаря}
\vs p143 5:1 Когда Учитель и двенадцать апостолов добрались до колодезя Иаковлева, Иисус, устав в пути, остановился у колодезя; Филипп же взял с собой апостолов, чтобы те помогли принести из Сихаря пищу и палатки, ибо они намеревались провести какое\hyp{}то время в этой местности. Петр и сыновья Зеведеевы хотели остаться с Иисусом, но он попросил их пойти вместе с собратьями, говоря: «Не бойтесь за меня; эти самаряне отнесутся к нам дружественно; только наши братья\hyp{}евреи пытаются причинить нам зло». Было же в тот летний вечер, когда Иисус сел возле колодезя и стал ожидать возвращения апостолов, около шестого часа.
\vs p143 5:2 Вода в колодезе Иаковлевом была более пресная, чем в колодцах Сихаря и поэтому была особенно хороша для питья. Иисуса мучила жажда, но достать воды из колодезя было нечем. Поэтому когда пришла женщина из Сихаря со своим кувшином и приготовилась почерпнуть из колодца, Иисус сказал ей: «Дай мне пить». По внешнему виду и одеянию Иисуса эта самарянка определила, что он --- еврей, а по его выговору догадалась, что он из Галилеи. Звали ее Нальда, и она была миловидным созданием. Она очень удивилась, что мужчина\hyp{}еврей вот так заговорил с ней у колодезя и попросил воды, ибо в те дни для уважающего себя мужчины не считалось приличным говорить с женщиной на людях, тем более не подобало еврею общаться с самарянкой. Поэтому Нальда спросила Иисуса: «Как ты, будучи евреем, просишь пить у меня, самарянки?» Иисус ответил: «Я действительно попросил у тебя пить, но если бы ты только понимала, то попросила бы у меня глоток воды живой». Тогда Нальда сказала: «Но, господин, тебе и почерпнуть нечем, а колодезь глубок: откуда же у тебя эта вода живая? Неужели ты больше отца нашего Иакова, который дал нам этот колодезь, и сам из него пил, и сыновья его, и скот его тоже?»
\vs p143 5:3 Иисус возразил: «Всякий, пьющий воду сию, возжаждет опять; а кто будет пить воду духа живого, не будет жаждать вовек. И вода эта живая сделается в нем источником освежающим, текущим в жизнь вечную». Тогда Нальда сказала: «Дай мне этой воды, чтобы мне не иметь жажды и не приходить сюда черпать. А кроме того, все, чего бы не получила женщина\hyp{}самарянка от такого достойного еврея, принесет удовольствие».
\vs p143 5:4 Нальда не знала, как отнестись к готовности Иисуса говорить с ней. Она видела, что лицо Учителя выражает праведность и святость, но приняла его дружелюбие за банальную фамильярность и неверно истолковала его метафору как попытку заигрывать с ней. Будучи женщиной легкого нрава, она была откровенно готова перейти к флирту, когда Иисус, посмотрев ей прямо в глаза, повелительным голосом сказал: «Женщина, позови мужа и приведи его сюда». Это приказание привело Нальду в чувство. Она увидела, что составила себе неправильное представление о доброте Учителя; она поняла, что неверно истолковала его манеру говорить. Она испугалась; она начала понимать, что стоит рядом с необыкновенной личностью и, пытаясь отыскать в уме подходящий ответ, в великом смущении сказала: «Но, государь, я не могу позвать мужа, ибо у меня мужа нет». Тогда Иисус сказал: «Ты сказала правду, хотя у тебя и был когда\hyp{}то муж, тот, с кем ты живешь сейчас, не муж тебе. Будет лучше, если ты перестанешь заигрывать со мной и попросишь воды живой, которой я в этот день тебе предложил».
\vs p143 5:5 `К этому времени Нальда обрела благоразумие, и все лучшее, что было в ней, пробудилось. Она не была безнравственной женщиной исключительно по собственной воле. Ее безжалостно и несправедливо бросил муж, и она, оказавшись в бедственном положении, согласилась жить с одним греком как его жена, но без заключения брака. Теперь Нальда почувствовала великий стыд за то, что столь бездумно говорила с Иисусом, и со всем раскаянием, на какое была только способна, обратилась к Учителю, говоря: «Господин мой, я раскаиваюсь в том, что так говорила с тобой, ибо я понимаю, что ты святой или, быть может, пророк». Она была почти готова просить прямой и личной помощи у Учителя, когда сделала то, что делали столь многие до нее и после нее, --- уклонилась от темы личного спасения, переведя беседу на обсуждение теологических и философских вопросов. Она быстро перевела разговор со своих собственных нужд на теологический спор. Показав на гору Гаризим, она продолжала: «Отцы наши поклонялись на этой горе, а \bibemph{вы} говорите, что место, где должно поклоняться, находится в Иерусалиме; так где же место, где нужно поклоняться Богу?»
\vs p143 5:6 Иисус понял попытку души женщины избежать прямого и откровенного общения со своим Творцом, но он также уже увидел в ее душе желание узнать лучший путь в жизни. В конце концов сердце Нальды истинно жаждало воды живой; поэтому он проявил к ней терпение и сказал: «Позволь мне сказать, женщина, что уже наступает время, когда не на горе сей, и не в Иерусалиме будете поклоняться Отцу. Однако ныне вы поклоняетесь тому, чего не знаете, смеси религии множества языческих богов и нееврейских философий. Евреи по крайней мере знают, кому поклоняются; они избавились от всей путаницы, сосредоточив свое поклонение на едином Боге, Яхве. Но ты должна верить мне, когда я говорю, что настанет час, и настал уже, когда все кто истинно почитают его будут поклоняться Отцу в духе и истине, ибо именно таких почитателей ищет Отец. Бог есть дух, и почитающие должны почитать его в духе и истине. Твое спасение придет не от знания, как или где должны молиться другие, а от того, что сердце твое примет эту воду живую, которую я предлагаю тебе уже сейчас».
\vs p143 5:7 Но Нальда сделала еще одну попытку уклониться от обсуждения смущавшего ее вопроса, касавшегося личной жизни на земле и положения ее души перед Богом. Она еще раз обратилась к общим религиозным вопросам, говоря: «Да, государь, я знаю, что Иоанн проповедовал о пришествии Обращающего, того, кого будут называть Избавителем, и что когда он придет, то возвестит нам все», --- но Иисус прервал Нальду и с поразительной уверенностью сказал: «Это я, который говорю с тобою».
\vs p143 5:8 Это было первое прямое, позитивное и открытое высказывание Иисуса о своей божественной природе и сыновстве, которое он сделал на земле; и сделано оно было женщине, самарянке, женщине, до этих пор имевшей в глазах мужчин сомнительную репутацию, но женщине, глядя на которую, божественное око увидело, что против нее грешили больше, нежели грешила она сама по собственному желанию, и что \bibemph{теперь} она стала человеческой душой, жаждущей спасения, жаждущей истинно и всем сердцем, и этого было достаточно.
\vs p143 5:9 Когда Нальда собралась выразить свое настоящее и личное стремление к лучшему и к более благородному образу жизни, как только она приготовилась произнести истинное желание своего сердца, из Сихаря возвратились двенадцать апостолов; увидев, что Иисус столь доверительно беседует с этой женщиной --- с самарянкой и притом наедине, --- они были более чем изумлены. Быстро сложив свои припасы, они отошли в сторону, и никто не посмел упрекнуть Иисуса, когда тот сказал Нальде: «Ступай себе, женщина; Бог простил тебя. Впредь ты будешь жить новой жизнью. Ты приняла воды живой; в твоей душе забьет ключом новая радость, и ты станешь дочерью Всевышнего». Женщина же, увидев неодобрение апостолов, бросила свой кувшин и побежала в город.
\vs p143 5:10 Придя в город, она говорила всем встречным: «Идите к колодезю Иаковлеву, идите скорее, ибо там вы увидите человека, который рассказал мне все, что я сделала. Не он ли Обращающий?» И еще до захода солнца у колодезя Иаковлева собралась великая толпа, чтобы послушать Иисуса. Учитель рассказал им еще больше о воде живой, даре пребывающего в них духа.
\vs p143 5:11 Апостолы не переставали изумляться готовности Иисуса говорить с женщинами, женщинами сомнительной репутации, даже безнравственными женщинами. Иисусу было очень трудно объяснить своим апостолам, что у женщин, даже так называемых безнравственных женщин, есть души, которые могут избрать Бога своим Отцом и, таким образом, они могут стать дочерьми Бога, кандидатами на вечную жизнь. Даже девятнадцать веков спустя многие проявляют такое нежелание понять учение Учителя. Даже христианская религия упорно строилась вокруг факта смерти Христа, а не вокруг истины его жизни. Мир должен больше интересоваться его счастливой и открывающей Бога жизнью, чем его трагической и печальной смертью.
\vs p143 5:12 На следующий день Нальда рассказала всю эту историю апостолу Иоанну, но он так и не открыл ее до конца другим апостолам, и Иисус не рассказывал о ней подробно двенадцати.
\vs p143 5:13 Нальда рассказала Иоанну, что Иисус поведал ей «все, что я когда\hyp{}либо сделала». Иоанн много раз хотел спросить Иисуса об этой встрече с Нальдой, но так и не сделал этого. Иисус рассказал Нальде о ней самой только одну вещь, но взгляд его, видящий ее насквозь, и то, как он обращался с ней, в одно мгновение так развернули у нее в уме всю панораму ее пестрой жизни, что все это самооткровение своей прежней жизни она связала со взглядом и словом Учителя. Иисус отнюдь не говорил ей о том, что у нее было пять мужей. После того, как ее бросил муж, она жила с четырьмя разными мужчинами, и в тот момент, когда она поняла, что Иисус --- человек от Бога, это вместе со всем ее прошлым так живо возникло у нее в уме, что впоследствии она повторила Иоанну, что Иисус действительно рассказал ей все о ней самой.
\usection{6. Возрождение самарян}
\vs p143 6:1 В тот вечер, когда Нальда привела из Сихаря толпу желавших увидеть Иисуса, двенадцать апостолов как раз вернулись с пищей; они просили Иисуса поесть с ними вместо того, чтобы говорить с народом, ибо они провели весь день без еды и были голодны. Но Иисус знал, что вскоре наступит темнота; поэтому он упорствовал в своей решимости поговорить с людьми прежде, чем отослать их. Когда же Андрей попытался убедить его перекусить перед тем, как говорить с толпой, Иисус сказал: «У меня есть пища, о которой вы не знаете». Услышав это, апостолы говорили между собою: «Разве кто принес ему есть? Неужели эта женщина дала ему не только пить, но и есть?» Услышав, как они говорят между собою, Иисус, перед тем как обратиться к народу, отошел в сторону и сказал двенадцати апостолам: «Моя пища есть творить волю Того, кто послал меня и совершить дело Его. Вы не должны больше говорить, что столько\hyp{}то или столько\hyp{}то до жатвы. Посмотрите на этих людей, пришедших из самарянского города послушать нас; я говорю вам, нивы уже созрели для жатвы. Жнущий получает плату и собирает сей плод в жизнь вечную, так что и сеющий, и жнущий вместе радоваться будут. Ибо в этом случае справедливо изречение: „один сеет, а другой жнет“. Я посылаю вас жать то, над чем вы не трудились: другие трудились, а вы вскоре разделите с ними их труд». Это он сказал о проповеди Иоанна Крестителя.
\vs p143 6:2 Перед тем, как расположиться лагерем на горе Гаризим, Иисус и апостолы пошли в Сихарь и два дня проповедовали там. И многие из жителей Сихаря поверили евангелию и просили о крещении, но апостолы Иисуса еще не крестили.
\vs p143 6:3 \pc В первую ночь в лагере на горе Гаризим апостолы ожидали, что Иисус будет упрекать их за отношение к женщине у колодезя Иаковлева, но он ни разу не упомянул о случившемся. Вместо этого он провел с ними памятную беседу «о главных реалиях в царстве Бога». В любой религии очень легко допустить возникновение несоразмерных ценностей и позволить фактам занять место истины в теологии. Крест впоследствии занял в христианстве центральное место; однако он --- отнюдь не главная истина в религии, которую можно извлечь из жизни и учений Иисуса из Назарета.
\vs p143 6:4 Тема учения Иисуса, данного на горе Гаризим, была такова: он хочет, чтобы все люди смотрели на Бога как на Отца и друга, так же как он (Иисус) им брат и друг. Снова и снова он внушал им, что любовь --- величайшая форма отношений в мире --- во вселенной --- так же, как истина --- величайшее провозглашенное выражение этих божественных отношений.
\vs p143 6:5 Иисус столь полно явил себя самарянам, и потому что мог не опасаясь, сделать это, и потому что знал, что вновь не посетит центр Самарии и не сможет проповедовать здесь евангелие царства.
\vs p143 6:6 Иисус и двенадцать апостолов жили в лагере на горе Гаризим до конца августа. Они проповедовали благую весть царства --- об отцовстве Бога --- самарянам в городах днем, а ночи проводили в лагере. Работа, которую Иисус и двенадцать апостолов совершили в этих самарянских городах, принесла царству множество душ и многое сделала, что подготовило путь замечательной деятельности Филиппа в этих местах после смерти и воскресения Иисуса, вслед за рассеянием апостолов по разным местам земли, которое было вызвано жестоким гонением на верующих в Иерусалиме.
\usection{7. Учения о молитве и богопочитании}
\vs p143 7:1 Во время вечернего собрания на горе Гаризим Иисус учил многим великим истинам и особое значение придавал, в частности, следующим:
\vs p143 7:2 \pc Истинная религия есть акт отдельно взятой души в ее сознательных отношениях с Творцом; официальная религия --- это попытка человека \bibemph{обобществить} богопочитание отдельных религиозных людей.
\vs p143 7:3 \pc Почитание --- созерцание духовного --- должно чередоваться со служением, взаимодействием с материальной действительностью. Труд должен чередоваться с отдыхом; религия должна быть уравновешена юмором. Глубокая философия должна разнообразиться ритмической поэзией. Трудности жизни --- напряженное состояние личности во времени --- должно ослабляться умиротворенностью, которое дает богопочитание. Противоядием от чувства неуверенности, вызванного страхом перед одиночеством личности во вселенной, должны быть постижение Отца через веру и попытки достичь Верховного.
\vs p143 7:4 \pc Молитва предназначена для того, чтобы сделать человека менее думающим, но более \bibemph{осознающим;} она служит не для увеличения знания, а для расширения понимания.
\vs p143 7:5 \pc Богопочитание служит для того, чтобы предвидеть лучшую жизнь в грядущем, а затем переносить эти новые духовные ценности на жизнь, которая есть сейчас. Молитва духовно поддерживает, почитание же божественно созидающе.
\vs p143 7:6 \pc Богопочитание есть способ созерцания \bibemph{Одного} для вдохновения в служении \bibemph{многих.} Почитание есть мерило, которым измеряют степень отрешенности души от материальной вселенной и одновременно ее надежную приверженность духовным реальностям всего творения.
\vs p143 7:7 \pc Молитва есть самонапоминание --- возвышенное мышление; поклонение есть забвение себя --- сверхмышление. Почитание есть не требующее усилий внимание, истинный и идеальный покой души, форма спокойного духовного усердия.
\vs p143 7:8 \pc Богопочитание есть действие части, отождествляющей себя с Целым; конечного --- с бесконечным; сына --- с Отцом; времени, шагающего в ногу --- с вечностью. Почитание есть акт личного общения сына с божественным Отцом, принятия живительной, творческой, братской и романтической позиции человеческой душой\hyp{}духом.
\vs p143 7:9 \pc Хотя апостолы сумели усвоить лишь немногие из его наставлений в лагере, их поняли иные миры и поймут следующие поколения живущих на земле.
