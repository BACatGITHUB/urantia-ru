\upaper{34}{Дух\hyp{}Мать локальной вселенной}
\author{Могучий Вестник}
\vs p034 0:1 Когда Сын\hyp{}Творец персонализируется Отцом Всего Сущего и Вечным Сыном, тогда Бесконечный Дух индивидуализирует новую и уникальную репрезентацию себя, чтобы сопровождать этого Сына\hyp{}Творца в сферы пространства, быть там его компаньоном сначала при физической организации, а позже при творении и служении созданиям новой спроектированной вселенной.
\vs p034 0:2 Творческий Дух реагирует и на физические, и на духовные реальности; точно так же --- и Сын\hyp{}Творец; и, таким образом, они равноправно соучаствуют в управлении локальной вселенной со временем и пространством.
\vs p034 0:3 Эти Духи\hyp{}Дочери происходят от сущности Бесконечного Духа, но они не могут заниматься физическим творением и духовным служением одновременно. При физическом творении Сын Вселенной дает паттерн, а Дух Вселенной начинает материализацию физических реальностей. Сын оперирует силовыми конструкциями, Дух же преобразует эти энергетические творения в физические субстанции. Хотя довольно трудно описать это присутствие Бесконечного Духа как личности во вселенной на раннем этапе, тем не менее, для Сына\hyp{}Творца его сподвижник Дух является личностным и всегда функционирует как отдельный индивидуум.
\usection{1. Персонализация Творческого Духа}
\vs p034 1:1 После завершения физического формирования скоплений звезд и планет и установления центрами мощи сверхвселенной контуров энергии, вслед за этой предварительной творческой деятельностью сил Бесконечного Духа, действующих через его творческое средоточие в локальной вселенной и под его руководством, возвещается послание Сына\hyp{}Михаила о том, что далее во вновь сформированной вселенной должна быть спроектирована жизнь. После признания этого заявления о намерениях в Раю следуют одобрительная реакция Райской Троицы, затем --- исчезновение в духовном сиянии Божеств Духа\hyp{}Мастера, в сверхвселенной которого это новое творение организуется. Тем временем к этой центральной обители Райских Божеств стягиваются другие Духи\hyp{}Мастера, и затем, когда появляется и признается своими товарищами объятый Божеством Дух\hyp{}Мастер, происходит то, что известно как «первичное извержение». Это грандиозная ослепительная духовная вспышка, явление, ясно видимое из центра соответствующей сверхвселенной; и одновременно с этим малопонятным проявлением Троицы происходит заметное изменение в природе пребывающего в данной локальной вселенной творческого духовного присутствия и мощи Бесконечного Духа. В ответ на это Райское явление прямо в присутствии Сына\hyp{}Творца немедленно персонализируется новая личностная репрезентация Бесконечного Духа. Это --- Божественная Служительница. Творческий Дух --- индивидуализированная помощница Сына\hyp{}Творца --- стала его личным творческим сподвижником, Духом\hyp{}Матерью локальной вселенной.
\vs p034 1:2 Из этого нового личностного отделения Объединенного Творца и через нее идут установленные потоки и предписанные контуры духовной мощи и духовного влияния, чтобы заполнить все миры и всех существ этой локальной вселенной. На самом деле это новое и личностное присутствие является лишь преобразованием предсущего и менее личностного сподвижника Сына в его ранней деятельности по формированию материальной вселенной.
\vs p034 1:3 \pc Это краткий рассказ о грандиозной драме, но в нем показано почти все, что можно сообщить об этих важнейших событиях. Они мгновенны, непроницаемы и непостижимы; тайна метода и процедуры пребывает в недрах Райской Троицы. Лишь в одном мы уверены: присутствие Духа в локальной вселенной во время чисто физического творения или формирования не было полностью обособленным от духа Райского Бесконечного Духа; тогда как, после появления руководящего Духа\hyp{}Мастера из тайного объятия Богов и вслед за вспышкой духовной энергии проявление Бесконечного Духа в локальной вселенной внезапно и полностью меняется на личностное подобие того Духа\hyp{}Мастера, который был в преобразовательной связи с Бесконечным Духом. Таким образом, Дух\hyp{}Мать локальной вселенной обретает личностную природу, окрашенную личностной природой Духа\hyp{}Мастера той сверхвселенной, в юрисдикцию которой входит данная вселенная.
\vs p034 1:4 Это персонализированное присутствие Бесконечного Духа, Творческой Духа\hyp{}Матери локальной вселенной называется в Сатании Божественной Служительницей. Для всех практических намерений и духовных целей это проявление Божества является божественным индивидуумом, духовной персоналией. И она признается и считается таковой Сыном\hyp{}Творцом. Именно через эту локализацию и персонализацию Третьего Источника и Центра в нашей локальной вселенной Дух смог впоследствии так полно подчиниться Сыну\hyp{}Творцу, что об этом Сыне истинно было сказано: «Вся мощь в небе и на земле вверена ему».
\usection{2. Природа Божественной Служительницы}
\vs p034 2:1 Претерпев заметную метаморфозу личности во время творения жизни, Божественная Служительница функционирует как персоналия и очень тесно сотрудничает с Сыном\hyp{}Творцом в планировании и управлении обширными делами их локального творения. Многим вселенским типам существ даже эта репрезентация Бесконечного Духа может показаться не полностью личностной в течение эпох, предшествующих последнему пришествию Михаила; но после возвышения Сына\hyp{}Творца до верховной власти Сына\hyp{}Мастера личностные черты Творческой Духа\hyp{}Матери настолько усиливаются, что начинают осознаваться всеми соприкасающимися с ней индивидуумами.
\vs p034 2:2 С самого начального союза с Сыном\hyp{}Творцом Вселенский Дух обладает всеми атрибутами физического контроля Бесконечного Духа, включая полный дар антигравитации. По достижении личностного статуса Вселенский Дух осуществляет такой же полный и всецелый контроль над гравитацией разума в локальной вселенной, какой осуществлял бы Бесконечный Дух, если бы он лично присутствовал.
\vs p034 2:3 \pc В каждой локальной вселенной Божественная Служительница функционирует в соответствии с природой и свойствами, присущими Бесконечному Духу, воплощенными в одном из Семи Духов\hyp{}Мастеров Рая. Хотя, в основном, характер всех Вселенских Духов единообразен, однако есть и функциональное разнообразие, обусловленное их происхождением от одного из Семи Духов\hyp{}Мастеров. Этой разницей происхождения объясняются различия способов функционирования Духов\hyp{}Матерей в разных сверхвселенных. Но по всем существенным духовным атрибутам эти Духи идентичны, в равной степени духовны и полностью божественны, несмотря на их различия в разных сверхвселенных.
\vs p034 2:4 \pc Творческий Дух разделяет с Сыном\hyp{}Творцом ответственность за сотворение созданий в мирах, и она никогда не разочаровывает Сына в стремлении поддерживать и сохранять эти творения. Жизнь поддерживается и сохраняется через посредство Творческого Духа. «Ты посылаешь своего Духа, и они создаются. Ты обновляешь лик земли».
\vs p034 2:5 При сотворении вселенной с разумными существами Творческий Дух\hyp{}Мать действует сначала в сфере вселенского совершенства и совместно с Сыном производит Яркую и Утреннюю Звезду. Затем потомки Духа все больше приближаются к чину сотворенных на планетах существ, точно так же, как Сыны спускаются с уровня Мелхиседеков на уровень Материальных Сынов, действительно соприкасающихся со смертными сфер. При последующей эволюции смертных созданий Сыны\hyp{}Носители Жизни дают физическое тело, сделанное из существующего формированного материала сферы, а Вселенский Дух вкладывает в него «дыхание жизни».
\vs p034 2:6 \pc Хотя во многих отношениях седьмой сегмент великой вселенной может и отставать в развитии, те, кто вдумчиво изучают наши проблемы, в грядущие эпохи предвкушают развитие необыкновенно хорошо сбалансированного творения. Мы предсказываем эту высокую степень симметрии в Орвонтоне потому, что руководящим Духом этой сверхвселенной является глава Духов\hyp{}Мастеров --- духовный разум, воплощающий сбалансированное объединение и совершенную координацию черт и свойств всех трех вечных Божеств. Мы запаздываем и отстаем по сравнению с другими секторами, но когда\hyp{}нибудь, в бесконечные эпохи будущего, нам, несомненно, предстоит превосходное развитие и небывалые достижения.
\usection{3. Сын и Дух во времени и пространстве}
\vs p034 3:1 Ни Вечный Сын, ни Бесконечный Дух, в отличие от большинства их потомков, не ограничены и не обусловлены ни временем, ни пространством.
\vs p034 3:2 Бесконечный Дух заполняет все пространство и пребывает в круге вечности. Тем не менее, личности Бесконечного Духа при своих непосредственных контактах с детьми времени часто должны считаться с временными элементами, и в малой степени --- с пространством. Многие служения разума при согласовании различных уровней вселенской реальности не принимают во внимание пространство, но испытывают задержку во времени. Одиночный Вестник, по существу, независим от пространства, но при перемещении из одного места в другое ему необходимо время; существуют и другие подобные сущности, неизвестные вам.
\vs p034 3:3 \pc По личным прерогативам Творческий Дух целиком и полностью независима от пространства, но не от времени. Определенного личного присутствия такого Вселенского Духа нет ни в центрах созвездий, ни в центрах систем. Она равномерно и в равной степени присутствует повсюду в своей локальной вселенной, и поэтому в одном мире присутствует так же буквально и личностно, как и в любом другом.
\vs p034 3:4 Только в отношении элемента времени Творческий Дух ограничена в своих вселенских служениях. Сын\hyp{}Творец в своей вселенной повсюду действует мгновенно, но Творческий Дух при служении вселенского разума должна считаться с временем, исключая случай, когда она сознательно и преднамеренно использует личностные прерогативы Вселенского Сына. При чисто духовном функционировании, равно как и при своем сотрудничестве с таинственной функцией вселенской отражательности, Творческий Дух также независима от времени.
\vs p034 3:5 \pc Хотя контур духовной гравитации Вечного Сына действует независимо как от времени, так и от пространства, все функции Сына\hyp{}Творца не свободны от пространственных ограничений. Если исключить акции в эволюционных мирах, Сыны\hyp{}Михаилы, видимо, способны действовать относительно независимо от времени. Сын\hyp{}Творец не обременен временем, но он обусловлен пространством; он не может быть лично одновременно в двух местах. Михаил Небадонский действует вне времени в пределах своей собственной вселенной, а с помощью отражательности --- практически так же и в сверхвселенной. Его связь с Вечным Сыном прямая и вневременная.
\vs p034 3:6 Божественная Служительница --- чуткая помощница Сына\hyp{}Творца, позволяющая сгладить и преодолеть присущие ему ограничения, касающиеся пространства, ибо когда эти двое действуют в административном объединении, они в пределах своего локального творения практически независимы от времени \bibemph{и} пространства. Поэтому, как практически повсюду наблюдается в локальной вселенной, Сын\hyp{}Творец и Творческий Дух обычно действуют независимо и от времени, и от пространства, поскольку каждому из них доступно временное и пространственное освобождение другого.
\vs p034 3:7 \pc Только абсолютные существа независимы от времени и пространства в абсолютном смысле. Большинство подчиненных персоналий и Вечного Сына, и Бесконечного Духа зависимы и от времени, и от пространства.
\vs p034 3:8 Когда Творческий Дух\hyp{}Мать становится «осознающей пространство», она готовится воспринимать ограниченную «пространственную сферу» как свою, как сферу, в которой она пространственно свободна в отличие от всего прочего пространства, которым она была бы обусловлена. Всякий волен выбирать и действовать только в пределах сферы своего осознания.
\usection{4. Контуры локальной вселенной}
\vs p034 4:1 В локальной вселенной Небадона есть три отдельных контура духа:
\vs p034 4:2 \ublistelem{1.}\bibnobreakspace Дух пришествия Сына\hyp{}Творца, Утешителя, Духа Истины.
\vs p034 4:3 \ublistelem{2.}\bibnobreakspace Контур духа Божественной Служительницы, Святого Духа.
\vs p034 4:4 \ublistelem{3.}\bibnobreakspace Контур служения разума, включающий более или менее объединенную деятельность, но различное функционирование семи духов\hyp{}помощников разума.
\vs p034 4:5 \pc Сыны\hyp{}Творцы наделены духом вселенского присутствия, во многих отношениях аналогичного присутствию Семи Духов\hyp{}Мастеров Рая. Это Дух Истины, который изливается на мир совершающим пришествие Сыном по получении им духовного права на этот мир. Этот совершающий пришествие Утешитель --- духовная сила, которая вечно притягивает всех искателей истины к Нему, являющемуся персонификацией истины в локальной вселенной. Этот дух --- неотъемлемо присущий дар Сына\hyp{}Творца, вытекающий из его божественной природы подобно тому, как главные контуры великой вселенной проистекают из личностного присутствия Райских Божеств.
\vs p034 4:6 Сын\hyp{}Творец может приходить и уходить; может личностно присутствовать в локальной вселенной или же где\hyp{}то еще; однако Дух Истины функционирует непотревоженным, ибо это божественное присутствие, хотя и вытекает из личности Сына\hyp{}Творца, функционально сосредоточено в персоналии Божественной Служительницы.
\vs p034 4:7 Дух\hyp{}Мать Вселенной, однако, никогда не покидает мир\hyp{}центр локальной вселенной. Дух Сына\hyp{}Творца может функционировать и функционирует независимо от личностного присутствия Сына, но в случае с личностным духом Духа\hyp{}Матери Вселенной это не так. Святой Дух Божественной Служительницы стал бы нефункциональным, если бы ее личностное присутствие было удалено из Спасограда. По\hyp{}видимому, ее духовное присутствие постоянно во вселенском мире\hyp{}центре, и именно этот факт позволяет духу Сына\hyp{}Творца функционировать независимо от местонахождения Сына. Дух\hyp{}Мать Вселенной действует в качестве вселенского средоточия и центра Духа Истины, равно как и ее собственного личностного влияния, Святого Духа.
\vs p034 4:8 \pc И Творческий Сын\hyp{}Отец, и Творческий Дух\hyp{}Мать по\hyp{}разному вносят вклад в дар разума детей своей локальной вселенной. Но Творческий Дух не дарует разум, пока ей не дарованы личностные прерогативы.
\vs p034 4:9 Сверхэволюционным чинам личностей в локальной вселенной дарован локально\hyp{}вселенский тип разума сверхвселенского паттерна. Человеческим и более низким, чем человеческие, чинам эволюционной жизни дарованы духи\hyp{}помощники разума.
\vs p034 4:10 \pc Семь духов\hyp{}помощников разума --- это творение Божественной Служительницы локальной вселенной. Эти духи разума одинаковы по характеру, но разные по мощи, и все они одинаково разделяют природу Вселенского Духа, хотя едва ли считаются личностями отдельно от их Матери\hyp{}Творца. Эти семь помощников называются: дух \bibemph{мудрости,} дух \bibemph{почитания,} дух \bibemph{обсуждения,} дух \bibemph{знания,} дух \bibemph{отваги,} дух \bibemph{понимания,} дух \bibemph{интуиции ---} быстрого постижения.
\vs p034 4:11 \pc Это --- «семь духов Бога», «подобные светильникам, горящим перед троном», которых пророк видел в символах видений. Но он не увидел тронов четырех и двадцати стражей вокруг этих семи духов помощников разума. В этой записи смешаны два представления: одно относится к центру вселенной, а другое --- к столице системы. Троны четырех и двадцати старейшин находятся в Иерусеме, центре вашей локальной системы обитаемых миров.
\vs p034 4:12 Но именно о Спасограде писал Иоанн: «И из трона исходили молнии и раскаты грома и голоса» --- вселенские возвещения локальным системам. Он также видел мысленным взором существ контроля направлений в локальной вселенной --- живые компасы центрального мира. Этот контроль направлений в Небадоне осуществляется четырьмя контролирующими существами Спасограда, которые действуют на вселенских потоках и которым умело помогает первый функционирующий дух разума --- помощник интуиции, дух «быстрого постижения». Но изображение этих четырех созданий, называемых зверями, было прискорбным образом искажено; они отличаются несравненной красотой и изысканной формой.
\vs p034 4:13 Четыре стороны света универсальны и являются неотъемлемой частью жизни Небадона. Все живые создания обладают телесными элементами, которые чувствительны и реагируют на эти направляющие потоки. Эти сотворенные создания дублируются по всей вселенной и ниже --- вплоть до отдельных планет, и вместе с магнитными силами миров так активизируют сонмы микроскопических тел в животных организмах, что их клетки направления всегда указывают на юг и север. Таким образом, в живых существах вселенной навсегда закреплено чувство ориентации. В какой\hyp{}то степени осознанное обладание этим чувством присуще и человечеству. Такие тела впервые были замечены на Урантии примерно во время данного повествования.
\usection{5. Служение духа}
\vs p034 5:1 Божественная Служительница сотрудничает с Сыном\hyp{}Творцом при формировании жизни и творении новых чинов существ вплоть до времени его седьмого пришествия, а затем, после его возвышения до полного владычества над вселенной, продолжает сотрудничать с Сыном и дарованным духом Сына в дальнейшей деятельности по мировому служению и планетарному продвижению.
\vs p034 5:2 В обитаемых мирах Дух начинает деятельность по эволюционному продвижению с безжизненной материи сфер, сначала даруя растительную жизнь, затем животные организмы, потом первые чины человеческого существования; и каждое последующее дарование вносит вклад в дальнейшее становление эволюционного потенциала планетарной жизни от начальных и примитивных стадий до появления обладающих волей созданий. Эта работа Духа в значительной степени осуществляется через семь помощников, духов обетования, объединяющий и координирующий дух\hyp{}разум развивающихся планет, вечно и совместно ведущих расы людей к более высоким идеям и духовным идеалам.
\vs p034 5:3 \pc Смертный человек сначала испытывает служение Духа в соединении с разумом, когда чисто животный разум эволюционных созданий обнаруживает способность воспринимать помощников почитания и мудрости. Это служение шестого и седьмого помощников указывает на эволюцию разума, пересекающую порог духовного служения. И такие разумы с функцией почитания и мудрости немедленно включаются в духовные контуры Божественной Служительницы.
\vs p034 5:4 После того как разуму таким образом даровано служение Святого Духа, он обладает способностью избирать (сознательно или бессознательно) духовное присутствие Отца Всего Сущего --- Настройщика Мысли. Но все нормальные разумы автоматически становятся готовыми к восприятию Настройщиков Мысли не раньше, чем совершивший пришествие Сын высвободит Дух Истины для планетарного служения всем смертным. Дух Истины действует как одно целое с присутствием духа Божественной Служительницы. Эта двуединая духовная связь парит над мирами, стремясь учить истине и духовно просвещать разумы людей, вдохновлять души творений восходящих рас и вечно вести народы, живущие на эволюционных планетах, к их Райской цели божественного предназначения.
\vs p034 5:5 Хотя Дух Истины изливается на всякую плоть, этот дух Сына почти полностью ограничен в функции и мощи личностным восприятием человека того, что составляет самую суть миссии совершившего пришествие Сына. Святой Дух частично независим от человеческого отношения и отчасти обусловлен решениями и сотрудничеством воли человека. Тем не менее, служение Святого Духа становится более эффективным с возрастанием святости и духовности внутренней жизни тех смертных, которые наиболее полно \bibemph{подчиняются} божественному водительству.
\vs p034 5:6 \pc Как индивидуумы вы лично не обладаете отдельной частью или сущностью духа Сына\hyp{}Отца\hyp{}Творца или творческого Духа\hyp{}Матери; эти служения не соприкасаются с думающими центрами разума индивидуумов и не пребывают в них, как это делают Таинственные Помощники. Настройщики Мысли --- конкретные индивидуализации предличностной реальности Отца Всего Сущего, которые действительно пребывают в человеческом разуме именно как часть этого разума, и они всегда гармонично сотрудничают с союзом духов Сына\hyp{}Творца и Творческого Духа.
\vs p034 5:7 Присутствие Святого Духа Вселенской Дочери Бесконечного Духа, Духа Истины Вселенского Сына Вечного Сына и духа\hyp{}Настройщика Райского Отца в эволюционном смертном или с ним означает соразмерность духовного дара и служения и дает такому смертному возможность ощутить веру --- факт сыновства по отношению к Богу.
\usection{6. Дух в человеке}
\vs p034 6:1 В ходе эволюции обитаемой планеты и роста духовности ее обитателей эти зрелые личности могут испытывать дополнительные духовные влияния. По мере того, как смертные совершенствуются в контроле над разумом и в духовном восприятии, функции этих многочисленных духовных служений становятся все более скоординированными; они все более смешиваются со сверхслужением Райской Троицы.
\vs p034 6:2 Хотя Божественность может проявляться во множестве форм, в человеческом опыте Божество единственно, всегда \bibemph{одно.} Не является множественным в человеческом опыте и духовное служение. Независимо от множественности происхождения все духовные влияния функционируют как одно целое. Они --- действительно одно целое, являющееся духовным служением Бога Семеричного в созданиях великой вселенной и для них; и с ростом понимания и восприимчивости созданий к этому объединяющему служению духа оно становится в их опыте служением Бога Верховного.
\vs p034 6:3 \pc С высот вечной славы божественный Дух спускается по длинному ряду ступеней, чтобы встретить тебя таким, какой ты есть, и там, где ты находишься, и при участии веры с любовью объять душу человеческого происхождения и уверенно и надежно пуститься в обратный путь по этим ступеням нисхождения, никогда не останавливаясь до тех пор, пока эволюционная душа не будет благополучно возвышена до тех же высот блаженства, с которых божественный Дух изначально отправился выполнять эту миссию милосердия и служения.
\vs p034 6:4 Духовные силы безошибочно ищут и достигают своих собственных изначальных уровней. Выйдя из Вечного, они туда же и вернутся, приведя с собой всех тех детей времени и пространства, которые воспринимали водительство и учение внутреннего Настройщика, которые воистину были «рожденными от Духа», сынами веры, сынами Бога.
\vs p034 6:5 \pc Божественный Дух --- это источник непрерывного служения и ободрения для детей человеческих. Ваши силы и достижения --- «согласно его милосердию, через обновление Духа». Духовная жизнь расточается, как и физическая энергия. Результатом духовных усилий является относительное духовное истощение. Весь опыт восходящего --- реальный, равно как и духовный; поэтому истинно написано: «Именно Дух оживляет». «Дух дает жизнь».
\vs p034 6:6 Мертвая теория даже высших религиозных доктрин бессильна преобразовать человеческий характер или регулировать человеческое поведение. Что нужно сегодняшнему миру, так это истина, которую в давние времена возвестил ваш учитель: «Не только в слове, но также в силе и в Святом Духе». Семя теоретической истины мертво, высочайшие нравственные идеи безрезультатны, если и пока божественный Дух не вдохнет жизнь в формы истины и не оживит формулы праведности.
\vs p034 6:7 Те, кто получили и осознали пребывание Бога внутри себя, родились от Духа. «Ты --- храм Бога, и дух Бога пребывает в тебе». Не достаточно того, что этот дух изольется на тебя; божественный Дух должен управлять и доминировать над каждой стадией человеческого опыта.
\vs p034 6:8 Именно присутствие божественного Духа, источника жизни, предотвращает всепоглощающую жажду человеческого недовольства и этот неописуемый голод неодухотворенного человеческого разума. Движимые Духом существа «никогда не жаждут, ибо эта духовная вода станет для них источником удовлетворения, текущего в жизнь вечную». Все такие божественно орошенные души почти независимы от материального окружения во всем, что касается радостей жизни и удовольствия от земного существования. Они духовно озарены и обновлены, нравственно укреплены и одарены.
\vs p034 6:9 \pc В каждом смертном существует двуединая природа: наследие животных тенденций и высокие побуждения духовного дарования. На протяжении короткой жизни, проживаемой на Урантии, эти два различных и противоположных побуждения редко могут быть полностью примирены; их едва ли можно гармонизировать и объединить; но на протяжении всего времени твоей жизни объединенный Дух всегда служит, чтобы помочь тебе все более и более подчинить плоть водительству Духа. Хотя ты и должен до конца прожить свою материальную жизнь и не можешь освободиться от тела и его потребностей, тем не менее, в намерениях и помыслах тебе дается возможность постепенно подчинить животную природу господству Духа. Внутри тебя действительно существует согласие духовных сил, единение божественных сил, исключительная цель которых --- добиться твоего окончательного избавления от материального рабства и конечных препятствий.
\vs p034 6:10 Цель всего этого служения в том, «Чтобы ты укрепился силой через Его дух, пребывающий в человеке». И все это представляет собой лишь предварительные шаги к конечному достижению совершенства веры и служения, того опыта, в котором ты будешь «наполнен всей полнотой Бога», «ибо все те, кого ведет дух Бога, являются сынами Бога».
\vs p034 6:11 \pc Дух никогда не \bibemph{заставляет,} он только ведет. Если ты с готовностью учишься, если хочешь достичь духовных уровней и добраться до божественных высот, если искренне желаешь достичь вечной цели, тогда божественный Дух нежно и с любовью поведет тебя по пути сыновства и духовного прогресса. Каждый шаг, который ты делаешь, должен быть шагом добровольного, сознательного и просветленного сотрудничества, господство Духа ни в малейшей степени не заражено насилием, и не скомпрометировано принуждением.
\vs p034 6:12 И когда такая жизнь под духовным водительством принимается свободно и осмысленно, в человеческом разуме постепенно развивается позитивное осознание контакта с Божеством и уверенность в приобщении к духу; раньше или позже «Дух засвидетельствует с твоим духом (Настройщиком), что ты дитя Бога». Твой собственный Настройщик Мысли уже сообщил тебе о твоем родстве с Богом, так что запись подтверждает, что Дух свидетельствует «\bibemph{с} твоим духом», а не \bibemph{твоему} духу.
\vs p034 6:13 Сознание господства духа в человеческой жизни вскоре сопровождается возрастающим проявлением свойств духа в жизненных реакциях ведомых духом смертных, «плод же духа: любовь, радость, мир, долготерпение, благость, милосердие, вера, кротость и умеренность». Такие ведомые духом и божественно озаренные смертные, хотя они еще идут смиренной стезей тяжкого труда и с человеческой верностью исполняют обязанности своих земных предназначений, уже начали различать огни вечной жизни, мерцающие на далеких берегах другого мира; они уже стали понимать реальность этой вдохновляющей и утешающей истины: «Царство Бога --- не пища и питье, но праведность, мир и радость в Святом Духе». И во всех испытаниях и при всех трудностях души рожденных от духа поддерживает надежда, которая позволяет пересилить всякий страх, потому что любовь Бога проливается во все сердца присутствием божественного Духа.
\usection{7. Дух и плоть}
\vs p034 7:1 Плоть, врожденная природа, унаследованная от рас животного происхождения, конечно, не приносит плоды божественного Духа. Когда человеческая природа стала более высокой благодаря влиянию природы Материальных Сынов Бога --- как расы Урантии в определенной мере продвинулись вперед благодаря пришествию Адама, --- тогда лучше подготовлен путь для сотрудничества Духа Истины с внутренним Настройщиком, чтобы дать прекрасный урожай плодов духа человеческого характера. Если вы не отвергнете этот дух, «он наставит вас на путь истинный», даже если для выполнения этой задачи может потребоваться вечность.
\vs p034 7:2 Эволюционные смертные, обитающие в нормальных мирах духовного развития, не испытывают чувства острого противоречия между духом и плотью, характерного для нынешних рас Урантии. Но даже на самых идеальных планетах доадамический человек должен прилагать большие усилия для восхождения от чисто животного существования к последовательно возрастающим интеллектуальным уровням и более высоким духовным ценностям.
\vs p034 7:3 Смертные нормальных миров не обуреваемы внутренней борьбой между своей физической и духовной природой. Они сталкиваются с необходимостью подниматься от стадий животного существования к более высоким уровням духовной жизни, но по сравнению с глубокими конфликтами смертных Урантии в области противоречий между материальной природой и духовной это восхождение больше похоже на образовательную подготовку.
\vs p034 7:4 \pc Народы Урантии страдают от последствий того, что дважды были лишены помощи в деле прогрессивного планетарного духовного достижения. Переворот Калигастии вызвал всемирное смятение и лишил все последующие поколения моральной поддержки, которую обеспечило бы хорошо упорядоченное общество. Но адамический срыв был еще более губительным, поскольку он лишил расы той физической природы высшего типа, которая была бы более сообразна с духовными стремлениями.
\vs p034 7:5 Смертные Урантии вынуждены находиться в состоянии такой явной борьбы между духом и плотью потому, что их далекие предки не были более полно адамизированы эдемским пришествием. По божественному плану смертные расы Урантии должны были иметь физическую природу, более созвучную духовным стремлениям.
\vs p034 7:6 \pc Несмотря на это двойное бедствие для природы человека и его окружения, теперешние смертные в меньшей степени были бы раздираемы столь явными противоречиями между плотью и духом, если бы они вошли в царство духа, в котором верующие сыны Бога относительно избавлены от рабской зависимости от плоти при просвещенной и раскрепощающей службе беззаветной преданности исполнению воли Отца небесного. Иисус показал человечеству новый образ жизни смертных, при котором люди могут в значительной степени избавиться от пагубных последствий бунта Калигастии и наиболее эффективно компенсировать то, чего они лишились в результате адамического срыва. «Дух жизни Христа Иисуса сделал нас свободными от законов животной жизни и искушений зла и греха». «Это победа, побеждающая плоть, это ваша вера».
\vs p034 7:7 Те знающие Бога мужчины и женщины, которые родились от Духа, вступают в конфликт со своей человеческой природой не в большей степени, чем обитатели самых нормальным миров --- планет, которые никогда не были заражены грехом и которых не коснулся бунт. Верующие сыны действуют на интеллектуальных и живут на духовных уровнях, гораздо выше конфликтов, вызываемых необузданными и неестественными физическими желаниями. Нормальные побуждения животных существ и естественные влечения и импульсы физической природы не вступают в конфликт даже с самыми высшими духовными достижениями, кроме как в умах невежественных, неправильно обученных или к прискорбию, чрезмерно совестливых людей.
\vs p034 7:8 \pc Отправившись по пути жизни вечной, приняв назначение и получив приказы продвигаться вперед, не страшитесь опасностей человеческой забывчивости и смертного непостоянства, не поддавайтесь сомнениям относительно неудачи или ставящему в тупик замешательству, не спотыкайтесь и не сомневайтесь в своем статусе и положении, ибо в каждый темный час, на каждом распутье в поступательной борьбе Дух Истины всегда обратится к вам и скажет: «Вот путь, иди по нему».
\vsetoff
\vs p034 7:9 [Представлено Могучим Вестником, временно назначенным служить на Урантию.]
