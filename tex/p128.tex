\upaper{128}{Иисус в период ранней зрелости}
\author{Комиссия срединников}
\vs p128 0:1 Когда Иисус из Назарета только вступил во взрослую жизнь, он, как и прежде, продолжал вести на земле обычную жизнь среднего человека. Иисус пришел в этот мир так же, как приходит любой другой ребенок; он не имел возможности выбирать себе родителей. Он сам выбрал себе именно этот мир в качестве планеты, на которой ему предстояло осуществить свое последнее, седьмое пришествие, воплотиться в смертной плоти, но в остальном он явился в мир естественным путем и рос, как и всякий ребенок в этом мире, борясь с окружавшими его трудностями точно так же, как другие смертные в этом и сходном мирах.
\vs p128 0:2 Следует всегда помнить о том, что цель пришествия Михаила на Урантию была двоякой:
\vs p128 0:3 \ublistelem{1.}\bibnobreakspace Наилучшим образом прожить целую человеческую жизнь во плоти смертного существа и достичь полноты своего владычества в Небадоне.
\vs p128 0:4 \pc \ublistelem{2.}\bibnobreakspace Донести откровение Отца Всего Сущего смертным обитателям миров во времени и пространстве и более действенно привести этих смертных к лучшему пониманию Отца Всего Сущего.
\vs p128 0:5 \pc Все остальные преимущества, которые получили создания его вселенной, и польза для вселенной, были случайными и вторичными по отношению к этим главным целям его пришествия в облике смертного.
\usection{1. Двадцать первый год (15~г.\,н.э.)}
\vs p128 1:1 С достижением возраста зрелости Иисус серьезно и абсолютно осознанно приступил к задаче, состоявшей в том, чтобы завершить опыт достижения совершенного знания о жизни низшей формы сотворенных им разумных существ и тем самым окончательно и полностью обрести право неограниченного управления сотворенной им вселенной. Он взялся за решение этой грандиозной задачи, полностью осознавая двойственность своей природы. Но он уже успешно слил обе эти природы воедино в одно существо --- Иисус из Назарета.
\vs p128 1:2 Иешуа бен Иосиф очень хорошо знал, что он --- человек, смертный человек, родившийся от женщины. Это явствует из выбора им своего первого звания --- \bibemph{Сын Человеческий.} Он и в самом деле жил этой жизнью во плоти и крови, и даже сейчас, когда он обрел полную власть вершителя судеб вселенной, он все еще носит наряду со своими многочисленными заслуженными титулами имя Сына Человеческого. И это верно буквально, что созидающее Слово --- Сын\hyp{}Творец --- Отца Всего Сущего «стало плотью и обитало с нами» на Урантии. Он трудился, уставал, отдыхал и спал. Он испытывал голод и удовлетворял эту потребность пищей; он жаждал и утолял жажду водой. Он испытал всю гамму человеческих чувств и эмоций; он, «подобно нам, искушен во всем», и он страдал и умер.
\vs p128 1:3 Он получил знания, приобрел опыт и они, слившись воедино, стали мудростью, так же, как это происходит с другими смертными этого мира. До своего крещения он не прибегал ни к какой сверхъестественной власти. Он не пользовался никакими силами, которые не были бы частью его человеческой натуры как сына Иосифа и Марии.
\vs p128 1:4 Что касается атрибутов его дочеловеческого существования, то он полностью освободил себя от них. До начала своего публичного служения он всецело ограничил себя собственным знанием людей и событий. Он был воистину человеком среди людей.
\vs p128 1:5 \pc И эта величественная правда пребудет вечно: «Ибо мы имеем не такого правителя, который не может сострадать нам в немощах наших, но который, подобно нам, искушен во всем, кроме греха». И так как он сам страдал, подвергаясь испытаниям и соблазнам, он способен в полной мере понять и помочь тем, кто пребывает в смятении и горе.
\vs p128 1:6 \pc Теперь плотник из Назарета до конца осознал предстоявшую ему работу, но предпочел прожить свою человеческую жизнь в русле ее естественного течения. И в некоторых отношениях он действительно является примером для своих смертных созданий, так, как и написано: «Пусть в вас будет разум тот же, что и в Христе Иисусе: Он, будучи природы Божьей, не почитал хищением быть равным Богу; но унижил Себя Самого, приняв образ творения, сделавшись подобным людям и по виду став как человек; смирил Себя, быв послушлин даже до смерти, и смерти на кресте».
\vs p128 1:7 Он прожил свою смертную жизнь так же, как могут жить свои жизни любые другие представители рода человеческого, «Он во дни плоти своей с сильным воплем и со слезами принес молитвы и моления могущему спасти его от смерти, и услышан был потому, что уверовал». И для того подобало ему \bibemph{во всех отношениях} уподобиться своим собратьям, чтобы стать милосердным и всепонимающим правителем над ними.
\vs p128 1:8 Он никогда не сомневался в том, что его природа была человеческой; это было самоочевидно и всегда присутствовало в его сознании. Но что касается божественности его природы, то всегда оставалось место для сомнений и предположений, по крайней мере это было так до его крещения. Самореализация божественности была медленным и, с человеческой точки зрения, естественным эволюционным раскрытием. Это раскрытие и самореализация божественности начались в Иерусалиме, когда Иисусу еще не было полных тринадцати лет, с первого сверхъестественного явления в его человеческом существовании; и этот опыт осуществления самореализации его божественной природы был завершен во время второго сверхъестественного события во плоти --- эпизода, связанного с принятием им крещения в Иордане от Иоанна и ознаменовавшего начало его публичной деятельности служения и учительства.
\vs p128 1:9 Между этими двумя небесными посещениями, одно из которых состоялось, когда ему шел тринадцатый год, а другое --- во время крещения, в жизни воплощенного Сына\hyp{}Творца не происходило ничего сверхъестественного или сверхчеловеческого. Тем не менее, младенец из Вифлеема, мальчик, юноша и мужчина из Назарета на самом деле был воплотившимся Творцом вселенной; но в течение всей своей жизни, вплоть до своего крещения Иоанном, он ни разу ни в малейшей степени не использовал своей силы и ни разу не прибег к руководству со стороны небесных личностей, не считая помощи своих ангелов\hyp{}хранительниц. И мы, свидетельствующие об этом, знаем, о чем мы говорим.
\vs p128 1:10 И все же на протяжении всех этих лет жизни во плоти он был поистине божественным. Он действительно был Сыном\hyp{}Творцом Райского Отца. И как только он начал свою публичную деятельность, после полного завершения его чисто человеческого опыта достижения владычества, он, не колеблясь, публично признал себя Сыном Бога. Он без колебаний объявил: «Я есть Альфа и Омега, начало и конец, первый и последний». Он не возражал и в дальнейшем, когда его называли Господь Славы, Правитель Вселенной, Господь Бог всего творения, Святой Израиля, Господь всего, Господь мой и Бог мой, с нами Бог, имеющий имя выше всякого имени и всех колен небесных, земных и преисподних, Всемогущий вселенной, Вселенский разум этого творения, Тот, в ком скрыты все сокровища премудрости и ведения, полнота Наполняющего все во всем, предвечное Слово предвечного Бога, Тот, кто есть прежде всего и кем все стоит, Творец, сотворивший небо и землю, Вседержитель вселенной, Судия всей земли, Податель вечной жизни, Верный Пастырь, Освободитель миров и Наставник нашего спасения.
\vs p128 1:11 \pc Он никогда не возражал ни против одного из этих титулов, когда с ними обращались к нему после того, как в более поздние годы из своей чисто человеческой жизни он перешел к осознанию своего божественного служения в человечестве, для человечества и за человечество в этом мире и для всех других миров. Иисус возразил только против одного титула, примененного к нему: когда его однажды назвали Иммануилом, он просто ответил: «Это не я, это мой старший брат».
\vs p128 1:12 Даже тогда, когда в земной жизни он стал облечен более широкими полномочиями, Иисус покорно подчинялся воле своего Небесного Отца.
\vs p128 1:13 После своего крещения он, не задумываясь, позволял своим благодарным и искренне верующим последователям поклоняться ему. Даже в то время, когда он боролся с бедностью и трудом рук своих обеспечивал жизненные потребности своей семьи, его уверенность в том, что он Сын Бога, росла; он знал, что он является творцом небес и той самой земли, на которой сейчас проходит его человеческое существование. И сонмы небесных существ во всей огромной и наблюдающей за ним вселенной равным образом знали, что этот человек из Назарета --- их возлюбленный Владыка и Творец\hyp{}отец. Глубочайшее чувство неизвестности пронизывало вселенную Небадона все эти годы; взоры всех небесных существ были постоянно прикованы к Урантии --- к Палестине.
\vs p128 1:14 \pc В этом году Иисус направился с Иосифом в Иерусалим, чтобы отпраздновать Пасху. Приведя Иакова в храм для посвящения, он полагал своим долгом привести и Иосифа. В отношениях с членами своей семьи Иисус всегда был совершенно беспристрастен. Он пошел с Иосифом в Иерусалим обычной дорогой через долину Иордана, но вернулся в Назарет по восточной Иорданской дороге, которая шла через Амафу. Спускаясь к Иордану, Иисус излагал Иосифу историю евреев, а на обратном пути рассказал ему о жизни знаменитых колен Рувимова, Гадова и Галаадова, которые традиционно населяли эти места к востоку от реки.
\vs p128 1:15 Иосиф задавал ему множество наводящих вопросов о его жизненном предназначении, но на большинство из них Иисус отвечал только одно: «Мой час еще не настал». Тем не менее, во время этих задушевных бесед было сказано много слов, которые Иосиф вспомнил во время волнующих событий последующих лет. Как обычно, когда он участвовал в этих праздничных торжествах в Иерусалиме, Иисус вместе с Иосифом провел эту Пасху со своими тремя друзьями из Вифании.
\usection{2. Двадцать второй год (16~г.\,н.э.)}
\vs p128 2:1 Это был один из тех годов, в которые братья и сестры Иисуса сталкивались с испытаниями и трудностями, связанными с проблемами, и изменениями, присущими юности. Теперь у Иисуса были братья и сестры в возрасте от семи до восемнадцати лет, и он был постоянно занят тем, что помогал им приспособиться к новым проявлениям их интеллектуальной и эмоциональной жизни. Тем самым, ему приходилось справляться с проблемами подросткового возраста, по мере того как они обнаруживались в жизни его младших братьев и сестер.
\vs p128 2:2 В этом году Симон окончил школу и начал работать с каменщиком Иаковом, старым другом детства и постоянным защитником Иисуса. В результате нескольких семейных обсуждений было решено, что было бы неразумно всем мальчикам избрать ремесло плотника. Они пришли к выводу, что, овладев различными ремеслами, они смогут в дальнейшем заключать договоры на строительство зданий полностью. И опять же, так как трое из них работали плотниками, не для всех всегда хватало работы.
\vs p128 2:3 В этом году Иисус продолжал отделывать дома и заниматься тонкой столярной работой, но большую часть времени он проводил в мастерской, рядом со стоянкой караванов. Иаков начал подменять его в мастерской. К концу года, когда плотницкой работы в Назарете оставалось мало, Иисус поручил Иакову ремонтную мастерскую, Иосифу --- домашний верстак, сам же отправился в Сефорис работать в кузнице. Шесть месяцев он работал с различными металлами и достиг значительного мастерства в кузнечном деле.
\vs p128 2:4 \pc Прежде чем приступить к своей новой деятельности в Сефорисе, Иисус устроил один из периодически собираемых семейных советов и поручил Иакову, которому только что исполнилось восемнадцать лет, обязанности главы семейства. Он обещал брату дружескую поддержку и полное сотрудничество и взял с каждого члена семьи слово повиноваться Иакову. С этого дня Иаков целиком принял на себя ответственность за материальное благополучие семьи, при этом и Иисус делал еженедельный взнос. Больше никогда Иисус не принимал бразды правления из рук Иакова. Работая в Сефорисе, он мог бы при необходимости каждую ночь возвращаться домой, однако намеренно оставался там, ссылаясь на погоду и другие причины, но его истинной целью было научить Иакова и Иосифа нести ответственность за семью. Он начал медленно отдаляться от семьи. Иисус возвращался в Назарет каждую субботу, а иногда и в течение недели, если того требовали обстоятельства, чтобы наблюдать за осуществлением нового плана, дать полезный совет или помочь чем\hyp{}нибудь.
\vs p128 2:5 \pc Живя в течение шести месяцев в основном в Сефорисе, Иисус получил новую возможность лучше познакомиться с взглядами на жизнь неевреев. Он работал с неевреями, жил с неевреями и всеми возможными способами пристально и усердно изучал их обычаи и образ мышления.
\vs p128 2:6 Моральные устои этого города, родного города Ирода Антипы, были настолько ниже даже того, что было принято в стоящем на перекрестке караванных путей городе Назарете, что после шестимесячного пребывания в Сефорисе Иисус охотно нашел предлог, чтобы вернуться в Назарет. Бригада, в которой он работал, должна была участвовать в общественных работах как в Сефорисе, так и в новом городе Тивериаде, а Иисус не был склонен заниматься хоть чем\hyp{}нибудь под руководством Ирода Антипы. Были и другие причины, которые, по мнению Иисуса, делали целесообразным его возвращение в Назарет. По возвращении в ремонтную мастерскую он не стал снова брать на себя управление семейными делами. Он работал в мастерской вместе с Иаковом и, насколько это было возможно, позволял ему и дальше руководить домом. Иаков продолжал распоряжаться семейными расходами и распределением бюджета семьи.
\vs p128 2:7 Таким мудрым и продуманным планированием Иисус подготовил путь, чтобы окончательно устраниться от деятельного участия в делах семьи. После того, как Иаков приобрел двухлетний опыт в качестве действующего главы семейства --- и за два года до того, как ему (Иакову) предстояло жениться, --- ответственность за домашний бюджет и общее ведение дома было возложено на Иосифа.
\usection{3. Двадцать третий год (17~г.\,н.э.)}
\vs p128 3:1 В этом году материальное положение несколько улучшилось, так как работали четверо. Мириам прилично зарабатывала продажей молока и масла; Марфа стала искусной ткачихой. Стоимость ремонтной мастерской была больше чем на одну треть выплачена. Положение было таково, что Иисус на три недели прервал работу, чтобы взять Симона в Иерусалим на Пасху, и это был самый длинный период отдыха от ежедневной тяжелой работы, которая выпала ему на долю с того времени, как умер отец.
\vs p128 3:2 Они пошли в Иерусалим дорогой через Десятиградие и через Пеллу, Герасу, Филадельфию, Есевон и Иерихон. Они вернулись в Назарет по дороге, идущей по побережью, заходя в Лидду, Иоппию, Кесарию, оттуда вокруг горы Кармил к Птолемаиде и Назарету. За это путешествие Иисус прекрасно познакомился со всей Палестиной к северу от Иерусалимской области.
\vs p128 3:3 В Филадельфии Иисус и Симон познакомились с купцом из Дамаска, которому так понравились двое из Назарета, что он настоял на том, чтобы они остановились в Иерусалиме в его владении. Пока Симон посещал храм, Иисус много времени проводил в беседах с этим хорошо образованным и много путешествовавшим человеком, сведущим в делах мира. Этот купец владел почти четырьмя тысячами караванных верблюдов; у него были дела по всему Римскому миру и теперь он собирался отправиться в Рим. Он предложил Иисусу приехать в Дамаск, чтобы войти в его дело по ввозу товаров с Востока, но Иисус объяснил, что он не чувствует себя вправе прямо сразу уехать так далеко от своей семьи. Но на пути домой он много думал об этих далеких городах и о еще более удаленных странах Дальнего Запада и Дальнего Востока, странах, о которых он так часто слышал от пассажиров и проводников караванов.
\vs p128 3:4 Симону очень понравилось пребывание в Иерусалиме. Он должным образом получил гражданство Израиля во время Пасхального посвящения новых «сынов заповеди». Пока Симон посещал Пасхальные службы, Иисус смешивался с толпой посетителей и вступал во множество интересных личных бесед с многочисленными новообращенными неевреями.
\vs p128 3:5 Возможно, одним из самых интересных было общение с молодым эллином по имени Стефан. Молодой человек был в Иерусалиме впервые, и случилось так, что он встретился с Иисусом вечером в четверг на Пасхальной неделе. Пока каждый из них прогуливался, разглядывая дворец Асмонеев, Иисус завел случайный разговор, в результате которого они заинтересовались друг другом и четыре часа беседовали о жизненном пути и об истинном Боге и о поклонении ему. Стефан был чрезвычайно потрясен тем, что сказал Иисус; он никогда не забывал его слов.
\vs p128 3:6 И это был тот самый Стефан, который впоследствии уверовал в учение Иисуса и чья смелая проповедь этого нового евангелия привела к тому, что он был до смерти побит камнями разгневанными евреями. Отчасти необычайная смелость проповеди Стефаном своего понимания нового евангелия была прямым следствием этого давнего разговора с Иисусом. Но Стефан даже в малейшей степени не подозревал, что тот галилеянин, с которым он разговаривал за пятнадцать лет до того, был тем же самым человеком, которого он позже провозгласил Спасителем мира и за которого ему предстояло так скоро умереть, став таким образом первым мучеником за зарождающуюся христианскую веру. Когда Стефан ценой своей жизни заплатил за свои нападки на Иудейский храм и его традиционное богослужение, рядом с ним стоял некто Савл, гражданин Тарса. И когда Савл увидел, как этот грек готов умереть за свою веру, в его сердце пробудились чувства, которые в конце концов привели к тому, что он стал поддерживать дело, за которое умер Стефан; позже он стал деятельным и неукротимым Павлом, философом, едва ли не единственным основателем христианской религии.
\vs p128 3:7 \pc В воскресенье после Пасхальной недели Иисус и Симон пустились обратно в Назарет. Симон никогда не забывал того, чему Иисус учил его во время этого путешествия. Он всегда любил Иисуса, но теперь он почувствовал, что начал понимать своего отца\hyp{}брата. У них было много задушевных бесед, пока они путешествовали по стране и готовили себе еду у дороги. Они прибыли домой в четверг в полдень, и Симон до поздней ночи не давал семье уснуть, рассказывая о своих впечатлениях.
\vs p128 3:8 Мария была очень огорчена рассказами Симона о том, что большую часть времени в Иерусалиме Иисус проводил, «общаясь с чужеземцами, особенно с теми, кто прибыл из дальних стран». Семья Иисуса никогда не могла понять его живого интереса к людям, его острой потребности общаться с ними, узнать их образ жизни и выяснить, о чем они думают.
\vs p128 3:9 \pc Насущные человеческие проблемы все больше и больше поглощали назаретское семейство; будущая миссия Иисуса упоминалась не часто, и сам он очень редко говорил о своей будущей деятельности. Его мать редко думала о том, что он --- обетованное дитя. Постепенно она переставала считать, что Иисусу предстоит исполнить какую\hyp{}либо божественную миссию на земле, и все же время от времени, когда она вспоминала посещение Гавриила перед рождением ребенка, ее вера возрождалась.
\usection{4. Дамасский эпизод}
\vs p128 4:1 Последние четыре месяца этого года Иисус провел в Дамаске в гостях у того купца, которого он в первый раз встретил в Филадельфии по дороге в Иерусалим. Представитель этого купца, проезжая через Назарет, разыскал Иисуса и проводил его до Дамаска. Этот купец, бывший наполовину евреем, предложил пожертвовать огромную сумму денег на основание в Дамаске школы религиозной философии. Он собирался создать учебный центр, который мог бы соперничать с Александрией. И он предложил Иисусу немедленно отправиться в длительную поездку по мировым образовательным центрам, чтобы подготовить себя к тому, чтобы возглавить этот новый проект. Это было одним из величайших искушений, с которым Иисус сталкивался в своей чисто человеческой жизни.
\vs p128 4:2 Вскоре этот купец представил Иисусу группу из двенадцати купцов и банкиров, согласившихся поддержать проект новой школы. Иисус проявил глубокий интерес к планируемой школе, помог им спланировать ее устройство, но он постоянно выражал опасение, что некоторые другие ранее принятые им на себя обязательства, о которых он не может сказать, не позволят ему взять на себя управление таким многообещающим предприятием. Его добровольный благодетель был настойчив; дома он предоставил Иисусу хорошо оплачиваемую работу переводчика, а тем временем он сам, его жена и их сыновья и дочери надеялись переубедить Иисуса и заставить его принять предложенную ему честь. Но он не соглашался. Он твердо знал, что его миссия на земле не должна пользоваться поддержкой образовательных учреждений; он знал, что ни в малейшей степени не должен связывать себя руководством со стороны «совета людей», пусть даже действующих из самых благих намерений.
\vs p128 4:3 Иисус, будучи отвергнут религиозными лидерами Иерусалима даже после того, как продемонстрировал им свое превосходство, был признан и с радостью принят как главный учитель дельцами и банкирами Дамаска, и все это при том, что он был простым и никому не известным плотником из Назарета.
\vs p128 4:4 Он никогда не говорил своей семье об этом предложении, и в конце года он снова был в Назарете и исполнял свои ежедневные обязанности так, словно никогда не подвергался искушению соблазнительных предложений дамасских друзей. И эти люди из Дамаска тоже никогда не связывали того, кто позже стал жителем Капернаума, приведшим в полное замешательство все еврейство, с плотником из Назарета, некогда осмелившимся отказаться от чести, которую их объединенное богатство могло бы ему предоставить.
\vs p128 4:5 \pc Иисус намеренно и в высшей степени умно и умело разделил разные эпизоды своей жизни, так что они, в глазах мира, никогда не связывались воедино как деяния одного человека. В последующие годы он много раз выслушивал историю об этом странном галилеянине, который отказался от возможности основать школу в Дамаске, способную состязаться с Александрией.
\vs p128 4:6 Одна из причин, которой руководствовался Иисус, стараясь отделять определенные периоды своего жизненного опыта, состояла в том, что он хотел воспрепятствовать созданию истории такой многогранной и эффектной деятельности, которая заставила бы последующие поколения почитать учителя, вместо того, чтобы следовать той истине, ради которой он жил и которой учил. Иисус не хотел оставлять подобных свидетельств человеческих достижений, которые отвлекли бы внимание от его учения. Он очень рано понял, что у его последователей будет искушение создать религию \bibemph{о нем,} которая может соперничать с тем евангелием царства, которое ему суждено было возвестить миру. Соответственно, он старался последовательно пресекать в своей богатой событиями жизни все, что, по его мнению, могло бы послужить этой естественной человеческой склонности превозносить учителя, вместо того, чтобы нести миру его учение.
\vs p128 4:7 Этим же объясняется, почему он позволил, чтобы в различные периоды его разнообразной жизни на земле его знали под разными именами. Вместе с тем, он не хотел неподобающим образом влиять на свою семью и других людей, влиять таким образом, чтобы они поверили в него вопреки своим истинным убеждениям. Он всегда отказывался пользоваться неположенными или несправедливыми преимуществами, которые дает человеческий разум. Он не хотел, чтобы люди верили в него, если их сердца не были открыты зову духовных реальностей, которые раскрывало его учение.
\vs p128 4:8 \pc К концу этого года дела в Назаретском доме пошли вполне гладко. Дети росли, и Мария начала привыкать к отсутствию в доме Иисуса. Он продолжал передавать Иакову для поддержания семьи почти весь свой заработок, оставляя себе лишь небольшую часть на необходимые личные расходы.
\vs p128 4:9 Когда миновали эти годы, стало еще труднее осознавать, что этот человек есть Сын Божий на земле. Он, казалось, окончательно стал личностью этого мира, просто еще одним человеком среди людей. И то, что пришествие происходит именно таким образом, было предопределено Отцом небесным.
\usection{5. Двадцать четвертый год (18~г.\,н.э.)}
\vs p128 5:1 Это был первый год относительной свободы Иисуса от ответственности за семью. С помощью советов и финансовой поддержки Иисуса Иаков очень успешно управлялся с домом.
\vs p128 5:2 \pc В этом году на следующей после Пасхи неделе в Назарет приехал молодой человек из Александрии, чтобы договориться о встрече Иисуса с группой александрийских евреев в течение этого года где\hyp{}нибудь на Палестинском побережье. Эта встреча была назначена на середину июня, и Иисус отправился в Кесарию, чтобы увидеться с пятью известными евреями из Александрии, которые просили его обосноваться в их городе в качестве религиозного учителя, предлагая ему для начала место помощника хазана в их главной синагоге.
\vs p128 5:3 Представитель этой группы объяснил Иисусу, что Александрии предназначено стать центром мировой еврейской культуры; что эллинистическое направление в еврейских делах фактически обогнало вавилонскую школу философии. Они напомнили Иисусу о зловещих признаках бунта в Иерусалиме, и по всей Палестине, и уверяли его, что любое восстание палестинских евреев будет равносильно самоубийству нации, что железная рука Рима подавит восстание за три месяца, что Иерусалим будет разрушен, а храм сравняют с землей, и камня не останется на камне.
\vs p128 5:4 Иисус выслушал все сказанное ими, поблагодарил их за доверие и, отказавшись идти в Александрию, по существу, ответил: «Мой час еще не настал». Очевидное безразличие Иисуса к той чести, которую, как они думали, оказали ему, привело их в замешательство. Прежде чем расстаться с Иисусом, они вручили ему некоторую сумму денег в знак признательности со стороны его александрийских друзей и чтобы возместить время и расходы на приезд в Кесарию для встречи с ними. Но Иисус отказался также и от денег, говоря: «Дом Иосифа никогда не получал милостыню, и мы не можем есть чужой хлеб, пока у меня есть сила в руках и пока мои братья могут трудиться».
\vs p128 5:5 Его друзья из Египта отплыли домой, и в последующие годы, когда до них доходили слухи о корабеле из Капернаума, который вызвал такое смятение в Палестине, немногие из них догадывались, что это и тот ребенок из Вифлеема, ставший взрослым, и тот самый странный галилеянин, который так бесцеремонно отклонил приглашение стать великим учителем в Александрии.
\vs p128 5:6 \pc Иисус вернулся в Назарет. За всю его жизнь у него не было другого периода, настолько бедного событиями, как шесть месяцев, остававшихся до конца этого года. Он наслаждался этой временной передышкой в обычной череде проблем, которые необходимо было решить, и трудностей, которые надо было преодолеть. Он много общался со своим Небесным Отцом и очень преуспел в управлении своим человеческим разумом.
\vs p128 5:7 Но в мирах пространства и времени человеческие дела не могут долгое время идти гладко. В декабре у Иакова был доверительный разговор с Иисусом, во время которого открыл ему, что он очень полюбил Эсту, молодую женщину из Назарета, и что они хотели бы со временем пожениться, если это будет возможно. Иаков отметил, что Иосифу скоро должно исполниться восемнадцать лет и что возможность стать настоящим главой семьи была бы для него полезным опытом. Иисус дал согласие на женитьбу Иакова через два года, при условии, что за оставшееся время он научит Иосифа, как следует управлять домашними делами.
\vs p128 5:8 И вот события начали следовать одно за другим --- в воздухе запахло свадьбами. Успех Иакова, получившего согласие Иисуса на брак, вдохновил Мириам, и она тоже посвятила брата\hyp{}отца в свои планы. Иаков, молодой каменщик, некогда добровольный телохранитель Иисуса, а теперь --- партнер Иакова и Иосифа, давно мечтал взять Мириам в жены. После того, как Мириам рассказала о своих планах Иисусу, он решил, что Иаков должен прийти к нему и официально попросить руки сестры, и обещал ей свое благословение на брак, как только она почувствует, что Марфа готова принять на себя обязанности старшей дочери.
\vs p128 5:9 \pc Когда Иисус бывал дома, он продолжал три раза в неделю вести занятия в вечерней школе, часто по субботам читал Писание в синагоге, проводил время со своей матерью, учил детей и вообще вел себя как достойный и уважаемый гражданин Назарета в государстве Израилевом.
\usection{6. Двадцать пятый год (19~г.\,н.э.)}
\vs p128 6:1 К началу этого года все назаретское семейство пребывало в добром здравии, и все дети закончили регулярное обучение, оставалась только небольшая работа, которую Марфа должна была сделать для Руфи.
\vs p128 6:2 \pc Иисус был одним из самых здоровых и совершенных представителей рода человеческого, появившихся на земле со времен Адама. Его физическое развитие было превосходным. Его ум был деятельным, острым и проницательным --- по сравнению со средним умственным развитием своих современников он обладал поразительными возможностями --- а дух его был поистине по\hyp{}человечески божественным.
\vs p128 6:3 \pc С того времени, как была утрачена собственность Иосифа, денежные дела семьи находились в наилучшем состоянии. За мастерскую, обслуживавшую караваны, были сделаны последние выплаты; они никому не были должны и впервые за много лет имели некоторую сумму денег про запас. Учитывая это обстоятельство, а также и то, что раньше он уже брал с собой в Иерусалим других братьев на их первую Пасхальную службу, Иисус решил сопровождать Иуду (который только что закончил синагогальную школу) в его первом посещении храма.
\vs p128 6:4 Они отправились в Иерусалим и вернулись обратно одной и той же дорогой, через долину Иордана, так как Иисус опасался возможных неприятностей, если он поведет своего младшего брата через Самарию. Еще в Назарете Иуда несколько раз попадал в весьма неприятные положения из\hyp{}за своего несдержанного характера, усугубленного его сильными патриотическими чувствами.
\vs p128 6:5 В положенное время они прибыли в Иерусалим и отправились впервые посетить храм, один вид которого взволновал Иуду до глубины души и привел его в трепет, и здесь им случайно повстречался Лазарь из Вифании. Пока Иисус разговаривал с Лазарем и пытался договориться о совместном праздновании Пасхи, Иуда вовлек их всех в серьезные неприятности. Рядом с ними стоял римский стражник, который отпустил какое\hyp{}то непристойное замечание, глядя на проходившую мимо еврейскую девушку. Иуда возгорелся яростным негодованием и не замедлил выразить свои чувства по поводу подобной непристойности прямо и так, чтобы солдат услышал. В то время римские легионеры были очень нетерпимы к любой непочтительности со стороны евреев; поэтому стражник немедленно взял Иуду под арест. Это было уже слишком для юного патриота, и прежде чем Иисус смог взглядом предостеречь его, тот стал многословно изливать все, что накопилось у него на душе против римлян, и это только еще больше ухудшило положение дел. Иуда, в сопровождении шедшего рядом с ним Иисуса, был тотчас же отведен в военную тюрьму.
\vs p128 6:6 Иисус попытался добиться или немедленного слушания дела Иуды, или его освобождения на время празднования Пасхи в тот вечер, но это ему не удалось. Так как следующий день был днем «священного собрания» в Иерусалиме, даже римляне не осмеливались рассматривать обвинение против еврея. Соответственно, Иуда остался в тюрьме до утра второго дня после ареста, и Иисус находился в тюрьме вместе с ним. Они не присутствовали в храме на церемонии посвящения сынов завета в полноправные граждане Израиля. Прошло еще несколько лет, прежде чем Иуде удалось пройти эту официальную церемонию, когда он снова побывал в Иерусалиме на празднике Пасхи в связи с пропагандистской деятельностью в пользу зилотов, патриотической организации, в которую он входил и где занимал очень активную позицию.
\vs p128 6:7 На следующее утро после второго дня пребывания в тюрьме Иисус предстал перед военным судьей, чтобы защищать интересы Иуды. Сославшись в качестве смягчающего обстоятельства на молодость брата и на провокационный характер эпизода, который повлек за собой арест, и указав другие убедительные, но благоразумные доводы, Иисус так представил дело, что судья пришел к мнению, что, возможно, яростная вспышка юного еврея могла иметь некоторые оправдания. Предупредив Иуду, чтобы тот впредь не давал воли подобной несдержанности, он, отпуская их, сказал Иисусу: «Тебе лучше не спускать глаз с брата. Похоже, он способен доставить вам всем множество неприятностей». И судья\hyp{}римлянин был прав. Иуда доставлял Иисусу немало неприятностей, и причина всегда была одна и та же --- столкновения с гражданскими властями из\hyp{}за его бездумных и неразумных взрывов патриотизма.
\vs p128 6:8 На ночь Иисус и Иуда отправились в Вифанию и объяснили, почему они не смогли, как было условлено, попасть на Пасхальную трапезу, и на следующий день отбыли в Назарет. Иисус не рассказал семье об аресте Иуды в Иерусалиме, но у него был долгий разговор с Иудой об этом эпизоде примерно через три недели после возвращения. После этого разговора с Иисусом Иуда сам все рассказал семье. Он никогда не забывал того терпения и воздержанности, которые его брат\hyp{}отец проявлял во время всего этого мучительного испытания.
\vs p128 6:9 Это была последняя Пасха, на которой Иисус присутствовал с кем\hyp{}либо из членов своей семьи. Сын Человеческий все больше отстраняться от тесной связи со своей собственной плотью и кровью.
\vs p128 6:10 \pc В этом году периоды его глубоких раздумий часто нарушались Руфью и ее товарищами по играм. И Иисус всегда готов был отложить размышления о своей будущей работе для мира и для вселенной, чтобы принять участие в ребяческом веселье и юношеской радости этих детей, которым никогда не надоедало слушать рассказы Иисуса о различных случаях, происходивших во время его поездок в Иерусалим. Они также очень любили его рассказы о природе и животных.
\vs p128 6:11 В ремонтной мастерской всегда были рады детям. Иисус заготавливал песок, обрубки бревна и камни рядом с мастерской, и стайки детей собирались там, чтобы поиграть. Когда они уставали от своих игр, самые храбрые из них заглядывали в мастерскую, и если ее хозяин не был занят, они отваживались войти и попросить: «Дядюшка Иешуа, выйдите к нам и расскажите какую\hyp{}нибудь историю подлиннее». Потом они тащили Иисуса наружу за руки, он усаживался на камень на углу мастерской, а дети рассаживались на земле вокруг него. И как же эти маленькие человечки любили своего дядюшку Иешуа! Он учил их смеяться, и смеяться от всего сердца. Обычно один или двое самых младших забирались к нему на колени и сидели там, восхищенно вглядываясь в его выразительное лицо, пока он рассказывал свои истории. Дети любили Иисуса, и Иисус любил детей.
\vs p128 6:12 Его друзьям было трудно понять его интеллектуальную многогранность, то, как он неожиданно и полностью мог переключиться с обсуждения глубоких политических, философских или религиозных проблем на беззаботную и радостную игривость этих пяти\hyp{}десятилетних малышей. По мере того, как подрастали его собственные братья и сестры, а внуки еще не появились, у него оказалось больше свободного времени, и он уделял его этим малышам. Но он не прожил на земле достаточно долго для того, чтобы порадоваться внукам.
\usection{7. Двадцать шестой год (20~г.\,н.э.)}
\vs p128 7:1 К началу этого года Иисус из Назарета уже явственно осознавал, что обладает огромной сферой потенциального могущества. Но вместе с тем он был совершенно уверен, что это могущество не должно употребляться им лично как Сыном Человеческим, по крайней мере до тех пор, пока его час не настал.
\vs p128 7:2 В это время он много думал, но мало говорил о своих отношениях с Небесным Отцом. Все эти размышления однажды выразились в его молитве на вершине холма, когда он сказал: «Независимо, от того кто я есть и какой властью я могу или не могу обладать, я всегда был и всегда буду следовать воле моего Райского Отца». Однако когда этот человек шел по Назарету с работы или на работу, было совершенно явно --- это касается всей обширной вселенной, --- что в нем «сокрыты все сокровища премудрости и ведения».
\vs p128 7:3 \pc Весь этот год дела всей семьи, за исключением Иуды, шли гладко. У Иакова много лет были неприятности с младшим братом, который не был расположен ни к тому, чтобы заняться какой\hyp{}нибудь работой, ни к тому, чтобы принять на себя часть семейных расходов. В то время, что он жил дома, он не слишком добросовестно зарабатывал средства для семейного бюджета.
\vs p128 7:4 Иисус был миролюбивым человеком, и его то и дело смущали воинственные подвиги и постоянные взрывы патриотизма Иуды. Иаков и Иосиф склонялись к тому, чтобы изгнать его, но Иисус не соглашался. Когда они выходили из себя, Иисус всегда советовал: «Будьте терпеливы. Пусть ваши советы будут мудры, а ваша жизнь служить красноречивым предметом, чтобы ваш младший брат сначала мог бы узнать лучший путь, а потом уже был бы вынужден следовать по нему за вами». Этот мудрый и исполненный любви совет Иисуса предотвратил раскол в семье; они остались вместе. Но Иуда так и не стал рассудительным до тех пор, пока не женился.
\vs p128 7:5 Мария редко заводила разговор о будущей миссии Иисуса. Когда бы ни затрагивался этот вопрос, Иисус отвечал лишь одно: «Мой час еще не настал». Иисус почти выполнил трудную задачу, состоявшую в том, чтобы отучить свою семью от зависимости от его постоянного личного присутствия. Он торопился подготовиться к тому дню, когда смог бы окончательно покинуть этот дом в Назарете, чтобы начать более деятельную подготовку к своему истинному служению людям.
\vs p128 7:6 Всегда следует помнить, что главной миссией седьмого пришествия Иисуса было обретение тварного опыта, достижение владычества над Небадоном. И в процессе получения этого самого опыта он дал Урантии и всей ее локальной вселенной верховное откровение Райского Отца. Помимо этого он взял на себя разрешение сложного положения дел на планете, возникшего в связи с восстанием Люцифера.
\vs p128 7:7 \pc В этом году у Иисуса было больше свободного времени, чем обычно, и он уделял много времени тому, чтобы обучить Иакова управлять ремонтной мастерской, а Иосифа --- руководить домашними делами. Мария чувствовала, что он готовится покинуть их. Но куда он собирался направиться? И что он намеревался делать? Она почти отказалась от мысли о том, что Иисус является Мессией. Она не могла понять его; она просто не могла постичь всей глубины своего сына\hyp{}первенца.
\vs p128 7:8 В этом году Иисус много времени проводил с каждым членом своей семьи по отдельности. Он брал их с собой в длинные и частые прогулки в горы и по окрестностям. Перед началом жатвы он взял Иуду к своему дяде, чье хозяйство находилось к югу от Назарета, но после окончания жатвы Иуда пробыл там недолго. Он убежал, и позже Симон нашел его с рыбаками на озере. Когда Симон привел его обратно домой, Иисус обсудил это происшествие со сбежавшим мальчиком и, так как тот захотел стать рыбаком, поехал с ним в Магдалу и поручил его заботам родственника, рыбака; и начиная с этого времени и до момента своей женитьбы Иуда работал достаточно хорошо и регулярно, он продолжал заниматься рыболовством и после женитьбы.
\vs p128 7:9 \pc Наконец настал день, когда все братья Иисуса выбрали себе жизненное поприще и утвердились в нем. Все было готово для отъезда Иисуса из дома.
\vs p128 7:10 \pc В ноябре состоялись сразу две свадьбы. Иаков и Эста и Мириам и Иаков поженились. Это было поистине радостное событие. Даже Мария была опять счастлива, но счастье ее то и дело омрачалось сознанием, что Иисус готовится к отъезду. Она мучилась полнейшей неизвестностью; если бы только Иисус мог сесть рядом с ней и, не таясь, обсудить все, как он делал раньше, когда был мальчиком. Но он упорно не хотел вступать в обсуждения и хранил глубокое молчание о своем будущем.
\vs p128 7:11 Иаков и его жена Эста переселились в аккуратный маленький домик в западной части города, подаренный ее отцом. Хотя Иаков продолжал оказывать поддержку дому своей матери, из\hyp{}за женитьбы его доля была сокращена наполовину, и Иисус официально назначил главой семьи Иосифа. Иуда теперь очень исправно каждый месяц присылал домой свою часть денег. Женитьбы Иакова и Мириам оказали на Иуду очень благоприятное воздействие, и когда на следующий день после двойной свадьбы он уезжал рыбачить, он уверил Иосифа, что тот может рассчитывать на него, что «он полностью исполнит свой долг и даже более того, если это потребуется». И он сдержал свое обещание.
\vs p128 7:12 Мириам жила в доме Иакова по соседству с Марией, а Иаков\hyp{}старший нашел упокоение рядом со своими предками. Марфа заняла место Мириам в доме, и до конца года это новое устройство семьи не давало сбоев.
\vs p128 7:13 \pc На следующий день после этой двойной свадьбы у Иисуса было важное совещание с Иаковом. Он конфиденциально сообщил ему, что собирается покинуть дом. Он передал ему полное право на владение ремонтной мастерской, официально и торжественно сложил с себя звание главы дома Иосифа и самым трогательным образом назначил своего брата Иакова «главой и защитником дома моего отца». Он начертал, и они оба подписали, тайный договор, в котором было оговорено, что в обмен на подаренное ему право владеть ремонтной мастерской Иаков впредь берет на себя полную финансовую ответственность за семью, тем самым освобождая Иисуса от всех дальнейших обязательств в этом отношении. После того, как соглашение было подписано и после того, как бюджет семьи был распределен таким образом, чтобы все насущные расходы семьи покрывались без вклада со стороны Иисуса, Иисус сказал Иосифу: «Однако, сын мой, я буду продолжать посылать тебе что\hyp{}нибудь каждый месяц, пока не настанет мой час, но то, что я буду посылать тебе, ты должен будешь использовать так, как того потребуют обстоятельства. Используй мои деньги на нужды или на развлечения семьи, как сочтешь нужным. Используй их в случае болезни или истрать на покрытие непредвиденных расходов, которые могут выпасть на долю каждого отдельного члена семьи».
\vs p128 7:14 И таким образом Иисус приготовился вступить во вторую и отдельную от его дома фазу своей взрослой жизни перед публичным вступлением в дело своего Отца.
