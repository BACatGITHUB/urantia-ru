\upaper{182}{В Гефсимании}
\author{Комиссия срединников}
\vs p182 0:1 В этот четверг примерно в десять часов вечера Иисус повел одиннадцать апостолов из дома Илии и Марии Марк обратно в лагерь в Гефсимании. Иоанн Марк с того дня в горах постоянно держал Иисуса в поле зрения. Пока Учитель был со своими апостолами в верхней комнате, Иоанн несколько часов отдыхал, однако, услышав, что они спускаются по лестнице, встал и, быстро накинув полотняный плащ, пошел за ними через город, за поток Кедрон в их уединенный лагерь, находившийся рядом с Гефсиманским садом. Всю эту ночь и следующий день Иоанн Марк оставался рядом с Иисусом, так что видел все, и слышал многое из того, что сказал Учитель с этого времени и до часа распятия.
\vs p182 0:2 Когда Иисус и одиннадцать апостолов возвращались в лагерь, апостолы начали интересоваться, почему столь долго отсутствует Иуда, и обсуждали друг с другом предсказание Учителя о том, что один из них его предаст, впервые заподозрив, что с Иудой Искариотом что\hyp{}то неладно. Однако они не говорили об Иуде открыто, пока не дошли до лагеря и не увидели, что он их там не ждет и не встречает. Когда же все апостолы стали осаждать Андрея вопросами, чтобы узнать, что случилось с Иудой, их глава только отметил: «Не знаю, где Иуда, но боюсь, что он нас бросил».
\usection{1. Последняя общая молитва}
\vs p182 1:1 Через несколько минут после прихода в лагерь, Иисус сказал апостолам: «Друзья и братья мои, уже совсем не долго осталось мне быть с вами, и я хочу, чтобы мы, уединившись, помолились нашему небесному Отцу ниспослать нам силу, которая бы нас укрепила в этот час и укрепляла бы впредь во всех делах, которые мы должны совершить во имя него».
\vs p182 1:2 Сказав это, Иисус, впереди всех, поднялся еще немного выше на Масличную гору и там, откуда был виден весь Иерусалим, повелел апостолам так же, как в день посвящения, встать вокруг него на колени на большом плоском камне, а затем, коленопреклоненный, в ореоле мягкого лунного света поднял глаза к небу и стал молиться:
\vs p182 1:3 «Отец, мой час пришел; прославь же теперь Сына Твоего, да и Сын Твой прославит Тебя. Я знаю, что ты дал мне полную власть над всякой плотью в царстве моем, и я дам жизнь вечную всем, кто станет верующим сыном Бога. И эта жизнь вечная послужит тому, чтобы мои создания познали тебя как единственного истинного Бога и Отца всех и верили в того, кого ты послал в мир. Отец, я прославил тебя на земле и совершил дело, которое ты поручил мне исполнить. Я почти завершил мое пришествие к детям, сотворенным нами; мне остается лишь отдать свою жизнь во плоти. И ныне, Отец мой, прославь меня славою, которую ты дал мне прежде основания мира сего, и снова прими меня одесную тебя.
\vs p182 1:4 Я явил тебя людям, которых ты избрал от мира и дал мне. Они твои --- как и всякая жизнь в руках твоих --- ты дал их мне, и я жил среди них, уча их пути жизни, и они уверовали. Эти люди уразумели, что я исшел от тебя и что жизнь, которой я живу во плоти, нужна, дабы явить мирам Отца моего. Истину, которую ты дал мне, я им открыл. Эти друзья и посланцы мои, искренне избрали принять слово твое. Я сказал им, что исшел от тебя, что ты послал меня в этот мир и что я готовлюсь вернуться к тебе. Отец, молю об этих избранных. Молю же о них не так, как молил бы о мире, но как об избранных мною от мира представлять меня миру после того, как вернусь к делу твоему, как и я представлял тебя в этом мире, пребывая во плоти. Эти люди мои; ты дал их мне; но все мое всегда твое, и все, что было твоим, ты ныне сделал моим. Ты прославился во мне, и ныне молю, чтоб и мне удостоиться прославиться в этих людях. Более не могу быть в этом мире и скоро вернусь к делу, которое ты поручил мне исполнить. Я должен оставить этих людей представлять нас и наше царство среди людей. Отец, храни этих людей верными, пока я готовлюсь отдать свою жизнь во плоти. Помоги этим друзьям моим быть едиными в духе, как едины мы. Покуда я мог быть с ними, я мог их охранять и вести, но теперь я собираюсь уйти. Будь рядом с ними, Отец, пока мы не сможем послать нового учителя утешать и укреплять их.
\vs p182 1:5 Ты дал мне двенадцать человек, и я сохранил их всех, кроме одного, сына погибели, который не желает более быть с нами. Эти люди слабы и немощны, но я знаю, что мы можем им доверять; я их проверил: они любят меня так же, как почитают тебя. И хотя им предстоит много пострадать за меня, но я хочу, чтобы они исполнились радостной уверенностью сыновства в царстве небесном. Людям этим передал я слово твое и научил их истине. Мир может возненавидеть их, как меня возненавидел, но я не прошу, чтобы ты взял их из мира, а прошу лишь о том, чтобы ты сохранил их от зла мира. Освяти их истиною; слово твое есть истина. И как ты послал меня в этот мир, так и я готовлюсь послать этих людей в мир. Ради них я жил среди людей и жизнь мою посвятил служению тебе, дабы вдохновить их очиститься истиной, которой я их научил, и любовью, которую я им открыл. Отец мой, я хорошо знаю, что мне нет нужды просить тебя хранить этих братьев после того, как уйду; я знаю, ты любишь их так же, как я, но делаю это, чтобы они могли лучше осознать, что Отец любит смертных людей, как любит их Сын.
\vs p182 1:6 Ныне, Отец мой, молю тебя не только об этих одиннадцати, но и обо всех остальных, которые веруют теперь или, возможно, уверуют в евангелие царства по слову их будущего служения. Я хочу, чтобы все они были едины, как едины мы с тобой. Ты во мне, и я в тебе, и я хочу, чтобы так и эти верующие были в нас; чтобы и твой, и мой дух пребывали в них. Если дети мои будут едины, как едины мы, если они будут любить друг друга, как любил их я, тогда все люди поверят, что я исшел от тебя, и захотят принять откровение истины и славы, которое я сотворил. Славу, которую ты дал мне, я явил этим верующим. Как ты жил со мной в духе, так и я жил с ними во плоти. Как ты был со мною едино, так и я был едино с ними, и так едино с ними и в них всегда будет новый учитель. И все это я совершил, чтобы братья мои во плоти узнали, что Отец любит их, как любит их Сын, и что ты любишь их так, как меня любишь. Отец, трудись со мною, дабы спасти этих верующих, чтоб и они, спустя некоторое время, могли прийти и быть со мною во славе, а затем, идя дальше, слились с тобой в объятиях Рая. Те, кто служит со мной в унижении, чтобы были со мной во славе и увидели все, что ты дал в руки мои как вечный урожай семени, посеянного во времени в подобии смертной плоти. Я страстно желаю показать земным братьям моим славу, которую ты дал мне прежде основания этого мира. Мир же этот, Отче праведный, знает о тебе очень мало, но я знаю тебя и открыл тебя этим верующим; они же откроют имя твое другим поколениям. И ныне открою им, что ты будешь с ними в мире, как был ты со мною, --- точно так же».
\vs p182 1:7 Одиннадцать апостолов еще несколько минут стояли на коленях вокруг Иисуса, и лишь затем встали и молча пошли обратно в лагерь, находившийся неподалеку.
\vs p182 1:8 \pc Иисус молился о \bibemph{единстве} своих последователей, но единообразия не желал. Грех порождает однообразную инерцию зла; праведность же пестует творческий дух индивидуального опыта в живых реалиях вечной истины и в нарастающем общении божественных духов Отца и Сына. В духовных отношениях верующего сына и божественного Отца не может быть догматической окончательности и сектантского сознания превосходства своей группы над другими.
\vs p182 1:9 Во время последней молитвы с апостолами Учитель упомянул о том, что он явил миру \bibemph{имя} Отца. И истинно это он свершил, явив Бога в своей совершенной жизни во плоти. Отец Небесный пытался явить себя Моисею, но сумел добиться лишь того, чтобы было сказано: «Я ЕСТЬ». Когда же он стал настаивать на дальнейшем откровении о себе, открылось лишь: «Я ЕСТЬ ТО, что Я ЕСТЬ». Однако когда Иисус завершил свою земную жизнь, имя Отца открылось настолько, что Учитель, который был воплощением Отца, истинно мог сказать:
\vs p182 1:10 Я есть хлеб жизни.
\vs p182 1:11 Я есть вода живая.
\vs p182 1:12 Я есть свет миру.
\vs p182 1:13 Я есть желание всех времен.
\vs p182 1:14 Я есть открытая дверь в вечное спасение.
\vs p182 1:15 Я есть реальность бесконечной жизни.
\vs p182 1:16 Я есть пастырь добрый.
\vs p182 1:17 Я есть путь бесконечного совершенствования.
\vs p182 1:18 Я есть воскресение и жизнь.
\vs p182 1:19 Я есть тайна жизни вечной.
\vs p182 1:20 Я есть путь, истина и жизнь.
\vs p182 1:21 Я есть бесконечный Отец конечных детей моих.
\vs p182 1:22 Я есть истинная лоза виноградная, а вы --- ветви.
\vs p182 1:23 Я есть надежда всех познавших живую истину.
\vs p182 1:24 Я есть живой мост из одного мира в другой.
\vs p182 1:25 Я есть живое звено между временем и вечностью.
\vs p182 1:26 \pc Таким образом, Иисус расширил живое откровение имени Бога для всех поколений. И как божественная любовь являет природу Бога, вечная истина все более и более раскрывает его имя.
\usection{2. Последний час перед предательством}
\vs p182 2:1 Апостолы были безмерно потрясены, когда, вернувшись в лагерь, не нашли там Иуды. Пока одиннадцать апостолов горячо спорили о своем собрате\hyp{}апостоле, совершившем предательство, Давид Зеведеев и Иоанн Марк отвели Иисуса в сторону и рассказали ему, что в течение нескольких дней следили за Иудой и знают о его намерении предать Иисуса в руки врагов. Иисус их выслушал, но сказал лишь: «Друзья мои, с Сыном Человеческим не может ничего случиться, если на то нет воли Отца Небесного. Да не тревожатся сердца ваши; все, что произойдет, послужит славе Бога и спасению людей».
\vs p182 2:2 Бодрое настроение Иисуса иссякало. В течение этого часа он становился все более и более серьезным, даже печальным. Апостолы же, чрезвычайно взволнованные, не желали возвращаться в свои шатры даже тогда, когда их об этом попросил сам Учитель. Закончив разговор с Давидом и Иоанном, он обратился с последними словами к одиннадцати апостолам и сказал: «Друзья мои, ложитесь спать. Приготовьтесь к завтрашним трудам. Помните: мы все должны подчиняться воле Отца Небесного. Мир мой оставляю вам». И, сказав это, жестом велел им идти в свои шатры, но когда те пошли, позвал Петра, Иакова и Иоанна и сказал: «Хочу, чтобы вы остались со мной ненадолго».
\vs p182 2:3 Апостолы уснули лишь потому, что были буквально измучены; с момента прихода в Иерусалим они постоянно недосыпали. Перед тем, как апостолы разошлись на ночлег, Симон Зилот всех их привел в свой шатер, где хранил мечи и другое оружие, и раздал каждому из них это боевое снаряжение. Все, кроме Нафанаила, приняли оружие и тотчас препоясались. Нафанаил же, отказавшись вооружаться, сказал: «Братья мои, Учитель неоднократно говорил нам, что царство его не от мира сего и что ученики его не должны сражаться мечом, чтобы совершить его установление. Я верю этому; и не думаю, что Учитель нуждается в том, чтобы мы мечом защищали его. Мы все видели его могучую силу и знаем, что он сам мог бы защитить себя от врагов своих, если бы того пожелал. Если же он не хочет сопротивляться врагам своим, значит, такое поведение отвечает его стремлению исполнить волю его Отца. Я буду молиться, но меч не возьму». Выслушав речь Нафанаила, Андрей вернул свой меч Симону Зилоту. Итак, когда они разошлись спать, девять из них были вооружены.
\vs p182 2:4 Негодование, вызванное тем, что Иуда оказался предателем, на мгновение затмило в умах апостолов все остальное. Слова Учителя о Иуде, сказанные во время последней молитвы, открыли им глаза на то, что Иуда их бросил.
\vs p182 2:5 \pc После того, как восемь апостолов наконец разошлись на ночлег, и в то время, когда Петр, Иаков и Иоанн вместе ждали повеления Учителя, Иисус обратился к Давиду Зеведееву: «Пришли мне своего самого быстрого и надежного вестника». Когда же Давид привел Учителю некого Иакова, в прошлом скорохода в службе вестников между Иерусалимом и Вифсаидой, доставлявшей послания за ночь, Иисус обратился к нему и сказал: «Как можно скорее отправляйся в Филадельфию к Авениру и скажи: „Учитель шлет тебе пожелания мира и говорит, что настал час, когда он будет предан в руки врагов, которые его казнят, но он воскреснет из мертвых и незадолго до того, как идти к Отцу, явится тебе и тогда даст наставления на все время до того, как новый учитель поселится в сердцах ваших“». Когда же Иаков повторил это послание, Учитель, убедившись, что тот его верно запомнил, отправил его в путь и сказал: «Не бойся, Иаков, никто ничего не сможет сделать тебе, ибо этой ночью невидимый вестник будет бежать рядом с тобой».
\vs p182 2:6 Затем Иисус обратился к старшему из посещавших его греков, которые расположились с ними в одном лагере, и сказал: «Брат мой, пусть то, что вскоре произойдет, не тревожит тебя, ибо я тебя уже предостерег. Сын Человеческий по наущению врагов своих, первосвященников и правителей евреев, будет казнен, но я воскресну, чтобы недолго побыть с тобой, прежде чем отправиться к Отцу. Увидев же, что все сие произошло, восславь Бога и укрепи братьев твоих».
\vs p182 2:7 \pc При обычных обстоятельствах апостолы лично пожелали бы Учителю спокойной ночи, однако в этот вечер они были так поглощены внезапным осознанием измены Иуды и настолько ошеломлены необычным строем прощальной молитвы Учителя, что, выслушав его прощальное напутствие, молча разошлись.
\vs p182 2:8 В ту ночь Иисус сказал Андрею, когда тот расставался с ним, следующее: «Андрей, делай все возможное, дабы удержать братьев твоих вместе, пока я не приду к вам опять после того, как выпью чашу сию. Укрепляй братьев твоих, ибо я уже все тебе рассказал. Да будет мир с тобою».
\vs p182 2:9 Никто из апостолов не ожидал, что в ту ночь случится нечто из ряда вон выходящее, ибо было уже очень поздно. Они пошли спать, чтобы, встав рано утром, приготовиться к худшему. Апостолы думали, что первосвященники попытаются арестовать их Учителя рано утром, поскольку в день приготовления к Пасхе после полудня никакая мирская работа никогда не делалась. Лишь Давид Зеведеев и Иоанн Марк понимали, что враги Иисуса придут с Иудой именно этой ночью.
\vs p182 2:10 \pc В ту ночь Давид приготовился встать на страже на верхней тропе, которая вела к дороге из Вифании в Иерусалим, Иоанн же Марк должен был следить за дорогой, идущей вдоль потока Кедрон к Гефсимании. Перед тем, как приступить к исполнению добровольно взятой на себя обязанности охранять, Давид попрощался с Иисусом и сказал: «Учитель, мое служение у тебя принесло мне великую радость. Мои братья --- твои апостолы, но я с наслаждением исполнял менее важные обязанности, как надлежит их исполнять, и когда ты уйдешь, буду тосковать по тебе всем сердцем своим». Тогда Иисус сказал Давиду: «Давид, сын мой, другие делали то, что им было приказано делать; ты же это служение исполнял по велению сердца, и я не оставил без внимания твою преданность. Ты тоже однажды будешь служить со мной в вечном царстве».
\vs p182 2:11 Тогда, готовясь охранять верхнюю тропу, Давид сказал Иисусу: «Знаешь, Учитель, я послал за твоей семьей и вестник принес известие, что сегодня ночью они уже в Иерихоне. Они будут здесь завтра до полудня, поскольку идти по этой страшной дороге ночью было бы для них опасно». И Иисус, посмотрев на Давида, сказал: «Да будет так, Давид».
\vs p182 2:12 \pc Когда Давид взошел на Масличную гору, Иоанн Марк встал на стражу у дороги, которая вдоль потока спускалась к Иерусалиму. И Иоанн оставался бы на этом посту, если бы не страстное желание быть рядом с Иисусом и знать, что происходит. Вскоре после того, как Давид оставил Иисуса, Иоанн Марк увидел, что Иисус с Петром, Иаковом и Иоанном ушли в близлежащую лощину и чувство преданности и любопытство настолько овладели им, что он бросил свой пост и последовал за ними, прячась в кустах, откуда видел и слышал все, что случилось в эти последние минуты в саду перед тем, как Иуда и вооруженные стражники явились арестовать Иисуса.
\vs p182 2:13 \pc Пока все это происходило в лагере Учителя, Иуда Искариот совещался с капитаном храмовых стражников, который, собрав своих людей, готовился отправиться под началом предателя арестовывать Иисуса.
\usection{3. Один в Гефсимании}
\vs p182 3:1 Когда в лагере все успокоилось и стихло, Иисус, взяв Петра, Иакова и Иоанна, проделал небольшой путь до близлежащей лощины, куда он прежде часто ходил молиться и общаться с Отцом. Три апостола не могли не заметить, что Иисус горестно угнетен; прежде они никогда не видели своего Учителя таким подавленным и печальным. Когда же они пришли к месту его поклонения, он велел трем апостолам сесть и бодрствовать вместе с ним, а сам отошел на небольшое расстояние молиться. И пал на землю, и молился: «Отец мой, я пришел в этот мир исполнить твою волю и сделал это. Я знаю, что настал час положить мою жизнь во плоти, и не уклоняюсь от этого, но хочу знать, что такова твоя воля, чтобы я пил чашу эту. Дай мне уверенность, что смерть моя угодна тебе, как угодна была моя жизнь».
\vs p182 3:2 Несколько минут Учитель оставался в молитвенной позе, а затем, подойдя к трем апостолам, нашел их спящими, ибо глаза их отяжелели, и они не могли бодрствовать. Разбудив их, Иисус сказал: «Что! Неужели и часа не могли вы бодрствовать со мной? Неужели не видите, что душа моя скорбит смертельно и что я нуждаюсь в вашей поддержке?» Когда трое очнулись ото сна, Учитель опять уединился и, пав на землю, снова молился: «Отец, я знаю, что возможно избежать сей чаши --- ибо с тобой все возможно, --- но я пришел исполнить волю твою, и хоть чаша сия горька, я ее буду пить, если такова твоя воля». Когда же он так молился, могущественный ангел опустился рядом с ним и, обратившись к нему, коснулся его и укрепил.
\vs p182 3:3 Когда Иисус вернулся, чтобы поговорить с тремя апостолами, то опять нашел их спящими. Он разбудил их и сказал: «В такой час мне нужно, чтобы вы бодрствовали и молились вместе со мной --- вам же тем более молиться нужно, чтобы не впасть в искушение, --- почему засыпаете, когда я оставляю вас?»
\vs p182 3:4 И затем Учитель в третий раз отошел и молился: «Отец, ты видишь моих спящих апостолов; помилуй их. Дух воистину бодр, плоть же немощна. Теперь же, Отец, если не может чаша сия миновать меня, я буду ее пить. Не моя воля, но твоя да будет». И, кончив молиться, минуту лежал, распростершись на земле. Когда же он встал и вернулся к своим апостолам, то снова нашел их спящими. Иисус посмотрел на них и с жестом сожаления ласково сказал: «Теперь продолжайте спать и почивать; время решений прошло. Ныне пришел час, в который Сын Человеческий будет предан в руки врагов своих». И, встряхнув их, чтобы разбудить, сказал: «Вставайте, пойдем обратно в лагерь, ибо вот, приблизился предающий меня, и настал час, когда стадо мое рассеется. Но об этом я уже вам рассказал».
\vs p182 3:5 \pc На протяжении тех лет, когда Иисус жил среди своих последователей, у них действительно было множество доказательств его божественной сущности, однако только теперь им предстояло увидеть новые свидетельства его человеческой природы. Накануне величайшего из всех откровений о его божественности --- его воскресения --- должны явиться величайшие доказательства его смертной природы --- его унижение и распятие.
\vs p182 3:6 Каждый раз, когда Иисус молился в саду, его человеческая природа еще прочнее овладевала его божественностью; его человеческая воля еще полнее соединялась с божественной волей Отца. Среди прочего, сказанного ему могущественным ангелом, были слова о том, что Отец желает, чтобы Сын его завершил свое земное пришествие, пройдя через тварный опыт смерти так же, как все смертные создания должны испытать материальный распад при переходе из бытия во времени в вечное совершенствование.
\vs p182 3:7 Ранее вечером пить чашу не казалось столь трудно; когда же человек Иисус простился со своими апостолами и отослал их спать, предстоящее испытание стало намного ужаснее. Иисус испытал естественные перепады настроения, свойственные всякому человеческому переживанию, и теперь был утомлен работой и изнурен долгими часами напряженного труда и болезненной тревоги, связанной с безопасностью его апостолов. Хотя ни один смертный и не может понять мыслей и чувств воплотившегося Сына Бога в такое время, как это, мы знаем, что он терпел великую муку и страдал от невыразимой печали, и пот огромными каплями стекал с лица его. Иисус наконец убедился, что Отец намерен позволить событиям развиваться своим естественным путем, и окончательно решился ради спасения себя самого не пользоваться своей властью владыки, которой он обладал как верховный глава вселенной.
\vs p182 3:8 И вот собравшиеся воинства необъятного творения под временным совместным водительством Гавриила и Персонализированного Настройщика Иисуса парят над сим местом. Начальники же подразделений небесных армий были многократно предупреждены не вмешиваться в ход событий на земле, пока Иисус сам не прикажет им вмешаться.
\vs p182 3:9 \pc Переживания из\hyp{}за расставания с апостолами отдались великой болью в человеческом сердце Иисуса; сия печаль любви мучила его и делала еще более трудной его встречу с такой смертью, которая, как он хорошо знал, ожидала его. Он сознавал, как слабы и как невежественны его апостолы, и боялся оставить их. Он хорошо знал, что пришло время его смерти, но его человеческое сердце хотело узнать, не возможен ли какой\hyp{}нибудь правомерный путь избежать этого ужасного состояния страдания и скорби. Когда же сердце Иисуса, пытаясь найти выход, сделать этого не сумело, оно приготовилось пить чашу. Божественный ум Михаила знал, что для двенадцати апостолов он сделал все возможное; человеческое же сердце Иисуса желало сделать для них еще больше прежде, чем они останутся в мире одни. Сердце Иисуса сокрушалось; он истинно любил своих братьев. Он был оторван от своей семьи во плоти; один из его избранных сподвижников предавал его. Народ его отца Иосифа его отверг и тем самым привел конец своей особой миссией на земле. Душу Иисуса терзали обманутая любовь и отвергнутое милосердие. Это было одно из тех ужасных для человека мгновений, когда кажется, что все рушится с сокрушительной жестокостью и повергает в ужасную муку.
\vs p182 3:10 Человеческой природе Иисуса была небезразлична эта ситуация личного одиночества, публичного стыда и кажущейся неудачи его дела. Все эти чувства давили на него неописуемой тяжестью. В этой великой печали его ум возвращался ко дням детства в Назарете и к началу его трудов в Галилее. Во время сего великого испытания в уме Иисуса мелькали воспоминания о многочисленных приятных эпизодах его земного служения. Этими\hyp{}то старыми воспоминаниями о Назарете, Капернауме, горе Ермон, восходах и закатах над мерцающим Галилейским морем он и утешал себя, укрепляя свое человеческое сердце и готовя его к встрече с изменником, который так скоро его предаст.
\vs p182 3:11 Перед приходом Иуды и отряда воинов Учитель полностью восстановил привычное для себя равновесие; дух возобладал над плотью; вера восторжествовала над всеми человеческими страхами или сомнениями. Верховное испытание всей сущности человеческой природы состоялось и было достойно выдержано. Сын Человеческий снова приготовился встретить своих врагов хладнокровно и в полной уверенности в своей неуязвимости как смертный человек, целиком посвятивший себя исполнению воли своего Отца.
