\upaper{75}{Срыв Адама и Евы}
\author{Солония}
\vs p075 0:1 Больше ста лет пытался Адам улучшить жизнь на Урантии, но смог достичь лишь незначительного прогресса за пределами Сада; мир, в целом, казалось, не претерпел серьезных изменений. Осуществление улучшения человеческих рас отодвигалось, по\hyp{}видимому, на неопределенное будущее, и ситуация казалась настолько безнадежной, что для облегчения положения требовалось сделать что\hyp{}то, что не предусматривалось в первоначальных планах. По крайней мере, такие мысли часто посещали Адама, и он много раз говорил об этом Еве. Адам и его супруга оставались лояльными, но были лишены общения с себе подобными и глубоко подавлены удручающим состоянием вверенного им мира.
\usection{1. Проблема Урантии}
\vs p075 1:1 Адамическая миссия на экспериментальной, испытавшей тяжелые последствия бунта и изолированной Урантии представляла собой трудно выполнимое предприятие. Поэтому и Материальный Сын, и Материальная Дочь уже на раннем этапе осознавали трудности и сложности своего планетарного задания. Тем не менее, они мужественно взялись за решение своих многочисленных проблем. Но когда приступили к важнейшей работе по устранению умственно отсталых и неполноценных людей из человеческих племен, то были совершенно обескуражены. Они не видели выхода из этого трудного положения и не могли посоветоваться со своими начальниками ни в Иерусеме, ни в Эдентии. Здесь они находились в изоляции, и ежедневно сталкивались с новой, сложной и запутанной проблемой, казавшейся неразрешимой.
\vs p075 1:2 В нормальных условиях первой задачей планетарных Адама и Евы должна была бы быть работа по координации и смешению рас. Но на Урантии подобный план казался почти неосуществимым, поскольку ее расы, хотя и соответствовали этому биологически, но никогда не были избавлены от своих отсталых и неполноценных элементов.
\vs p075 1:3 Адам и Ева оказались на сфере, абсолютно неподготовленной к провозглашению принципа братства людей, в мире, идущем ощупью в полной духовной темноте, мучимом смятением, которое еще больше усилилось вследствие ошибок миссии предыдущего правления. Разум и мораль находились на низком уровне, и вместо того, чтобы приступить к делу осуществления религиозного единства, они должны были заново начать работу по приобщению населения к самым примитивным формам религиозных верований. Вместо наличия языка, годного для всеобщего употребления, они столкнулись в мире с мешаниной сотен и сотен местных диалектов. Ни один Адам планетарной службы не отправлялся в мир, находящийся в более тяжелом положении; препятствия казались непреодолимыми, а проблемы --- за пределами того, что было по силам тварному существу.
\vs p075 1:4 Они очутились в изоляции, и испытываемое ими чувство страшного одиночества еще более усилилось после скорого отъезда Мелхиседеков\hyp{}исполнителей. С любыми существами за пределами планеты они могли связаться не напрямую, а лишь через посредство ангельских чинов. Их стойкость медленно таяла, настроение падало, и временами казалось, что и вера почти поколеблена.
\vs p075 1:5 Именно в таком состоянии подавленности находились эти две благородные души, когда размышляли о стоящих перед ними проблемах. Они очень хорошо представляли себе огромные трудности, связанные с выполнением своего планетарного задания.
\vs p075 1:6 Вероятно, ни один Материальный Сын Небадона не сталкивался с такой трудной и практически безнадежной задачей, какая стояла перед Адамом и Евой в связи с плачевным положением Урантии. Однако они смогли бы когда\hyp{}нибудь добиться успеха, будь они более дальновидны и \bibemph{терпеливы.} Но они оба, особенно, Ева, были все\hyp{}таки слишком нетерпеливы; они не были готовы к тому, чтобы пройти долгое, долгое испытание на выносливость. Они хотели видеть немедленный результат, и они его увидели, но этот результат оказался бедственным и для них самих, и для их мира.
\usection{2. Ловушка Калигастии}
\vs p075 2:1 Калигастия часто приходил в Сад и вел долгие разговоры с Адамом и Евой, но те были непреклонны и не соглашались ни на какие предложенные им компромиссы и скороспелые авантюры. Они хорошо знали, к чему приводит бунт, чтобы выработать действенный иммунитет ко всем подобным инсинуациям. Даже на младших детей Адама предложения Далигастии не производили впечатления. И, конечно, ни Калигастия, ни его единомышленник не обладали властью заставить кого\hyp{}нибудь сделать что\hyp{}то против его воли, еще менее они были способны склонить детей Адама к дурным поступкам.
\vs p075 2:2 Нужно помнить, что Калигастия оставался номинальным Планетарным Принцем Урантии, заблудшим, но тем не менее, высоким Сыном локальной вселенной. Он так и не был окончательно смещен до прихода на Урантию Христа Михаила.
\vs p075 2:3 Но падший Принц был настойчив и решителен. Скоро он отказался от мысли воздействовать на Адама и решил с помощью коварного обходного маневра попытаться подступиться к Еве. Враг пришел к заключению, что сможет надеяться на успех только в том случае, если ему удастся хитроумно использовать соответствующих людей, принадлежащих к верхушке общества Нодитов, потомков тех, кто когда\hyp{}то в облике человека были его помощниками. И поэтому его планы заключались в том, чтобы заманить в ловушку мать фиолетовой расы.
\vs p075 2:4 \pc В намерения Евы совершенно не входило делать что\hyp{}то, что могло бы помешать планам Адама или подвергло бы опасности вверенный их опеке мир. Зная склонность женщины рассчитывать скорее на достижение немедленных результатов, чем на предусмотрительное планирование отдаленных, перед отбытием Евы Мелхиседеки специально указали ей на особую опасность, их изолированного положения на планете, и, в частности, предупредили о необходимости всегда быть рядом со своим супругом, т.е. никогда даже не пытаться предпринимать какие\hyp{}либо собственные или тайные действия, чтобы добиться осуществления совместных с Адамом предприятий. В течение более ста лет Ева в высшей степени точно следовала этим инструкциям, ей и в голову не приходило, что какая\hyp{}либо опасность может быть связана с участившимися конфиденциальными визитами одного из вождей Нодитов по имени Серапататья, которого она с удовольствием принимала. Все происходило так постепенно и так естественно, что она была застигнута врасплох.
\vs p075 2:5 С первых дней Эдема жители Сада поддерживали контакты с Нодитами. Эти смешанные потомки мятежников из штата Калигастии сотрудничали с жителями Сада и оказывали им ценную помощь, но теперь, из\hyp{}за этих самых потомков, Эдемический порядок шел к своему полному краху и окончательному уничтожению.
\usection{3. Искушение Евы}
\vs p075 3:1 Как только закончились первые сто лет пребывания Адама на земле, Серапататья после смерти своего отца стал вождем западной, или Сирийской конфедерации племен Нодитов. Серапататья был человеком с кожей коричневого оттенка, человеком выдающихся способностей. Его предок занимал одно время пост главы комиссии Даламатии по здравоохранению и в те далекие дни был женат на одной из самых умных представительниц голубой расы. В течение многих веков их потомки обладали властью и огромным влиянием среди западных племен Нодитов.
\vs p075 3:2 Серапататья несколько раз побывал в Саду, и праведность дела Адама произвела на него глубокое впечатление. Вскоре после того, как он стал вождем сирийских Нодитов, он объявил о своем намерении присоединиться к работе, которую Адам и Ева проводили в Саду. Большинство его народа поддержали его в этом предприятии, и Адам был обрадован вестью, что самое могущественное и самое интеллектуально развитое племя из всех соседних племен почти полностью намерено поддержать его план усовершенствования мира. И вскоре после этого великого события Серапататья со своим новым штатом был принят Адамом и Евой в их собственном доме.
\vs p075 3:3 \pc Серапататья стал одним из самых способных и самых эффективных помощников Адама. Во всех своих делах он был абсолютно честным и искренним; он никогда и не предполагал, что в силу сложившихся обстоятельств служит орудием в руках коварного Калигастии.
\vs p075 3:4 \pc Вскоре Серапататья стал заместителем председателя Эдемической комиссии по связям между племенами, и предложил множество планов, предусматривающих более энергичное проведение работы по вовлечению отдаленных племен в дело Сада.
\vs p075 3:5 Он провел множество совещаний с Адамом и Евой, особенно с Евой, обсуждая различные планы улучшения методов совместной работы. Однажды, во время разговора с Евой, Серапататье пришло в голову, что, пока ожидается набор большого числа новобранцев фиолетовой расы, было бы не плохо между тем что\hyp{}то немедленно сделать для развития бедствующих племен. Серапататья утверждал, что, если бы Нодиты, как наиболее прогрессивная и сотрудничающая с Эдемом раса, имели бы вождя, в чьих жилах с рождения текла бы кровь и фиолетовой расы, то это создало бы прочную основу, более тесно соединяющую эти народы с Садом. И все это, как совершенно честно и искренно полагалось, пойдет миру на благо, поскольку такой человек, воспитанный и получивший образование в Саду, будет оказывать очень большое и благотворное влияние на свой родной народ.
\vs p075 3:6 Здесь снова необходимо подчеркнуть, что Серапататья был абсолютно честен и искренен во всем, что он предлагал. Он никогда не подозревал, что играет на руку Калигастии и Далигастии. Серапататья был всецело предан плану создания мощного отряда людей фиолетовой расы до начала повсеместного совершенствования сбитых с толку народов Урантии. Но на это ушли бы многие сотни лет, а он был нетерпелив, он хотел видеть хоть какие\hyp{}то результаты немедленно --- уже при его собственной жизни. Он обратил внимание Евы на то, что Адам часто бывает обескуражен тем, как мало сделано для реализации подъема мира.
\vs p075 3:7 \pc Более пяти лет втайне зрели эти планы. Наконец, дело дошло до того, что Ева согласилась на тайную встречу с Кано, самым блестящим умом в соседней колонии дружественных Нодитов, их настоящим вождем; действительно он был истинным духовным лидером тех соседних Нодитов, которые поддерживали дружеские связи с Садом.
\vs p075 3:8 Роковая встреча произошла в сумерках осеннего вечера, недалеко от дома Адама. Ева никогда прежде не встречалась с прекрасным и восторженным Кано --- это был великолепный эталон превосходного телосложения и выдающегося ума, доставшихся ему в наследство от далеких прародителей из штата Принца. И Кано также был абсолютно убежден в правильности плана Серапататьи. (За пределами Сада многоженство было обычным делом.)
\vs p075 3:9 Под влиянием лести, энтузиазма, большого личного обаяния Ева тогда согласилась приступить к осуществлению предприятия, обсуждавшегося столь долго, --- пополнить своим ограниченным планом спасения мира более обширный и более далеко идущий божественный план. Прежде чем она полностью осознала, что произошло, фатальный шаг был сделан. Свершилось.
\usection{4. Осознание срыва}
\vs p075 4:1 Небесная жизнь планеты пришла в замешательство. Адам понял, что что\hyp{}то не так, и попросил Еву прогуляться с ним по Саду. И тогда впервые Адам услышал полный рассказ о долго лелеемом проекте ускорения процесса совершенствования мира в двух направлениях: выполнения божественного плана с одновременной реализацией предложения Серапататьи.
\vs p075 4:2 И когда Материальные Сын и Дочь беседовали в залитом лунным светом Саду, «голос в Саду» осудил их за непослушание. И голос этот был ничем иным, как моим собственным оповещением Эдемической паре, что они нарушили завет Сада, не выполнили предписаний Мелхиседеков, оказались неспособными сдержать клятву верности владыке вселенной.
\vs p075 4:3 Ева согласилась участвовать в делах добра и зла. Добро --- это осуществление божественных планов; грех --- сознательное нарушение божественной воли. Зло --- это неправильная адаптация планов и неумение использовать методы, что ведет к нарушению гармонии во вселенной и планетарной смуте.
\vs p075 4:4 Каждый раз, когда Эдемическая пара вкушала плод дерева жизни, архангел\hyp{}хранитель предостерегал их не уступать соблазну сочетать добро со злом, исходящему от Калигастии. Им было дано такое предостережение: «В день, когда смешаете добро со злом, станете смертными мира сего; и вы умрете».
\vs p075 4:5 Ева рассказала Кано об этом часто повторяющемся предупреждении на их роковой тайной встрече, но Кано, не осведомленный ни о происхождении, ни о значении такого предостережения, уверил ее в том, что мужчина и женщина, исходя из добрых побуждений и с самыми честными намерениями, не могут сотворить никакого зла; что она наверняка не умрет, а, скорее всего, возродится для новой жизни в своем потомке, который вырастет чтобы принести благословение и стабильность этому миру.
\vs p075 4:6 И хотя этот проект изменения божественного плана был задуман и воплощен с абсолютной искренностью и исходя из самых высоких побуждений и мыслей о благоденствии мира, но это было зло, потому что он являл собой ошибочный путь достижения праведной цели, уводящий в сторону от правильного пути, т.е. от божественного плана.
\vs p075 4:7 Правда, Ева нашла Кано внешне привлекательным, и она ощутила все то, что ее соблазнитель обещал ей в процессе «нового расширенного познания человеческих отношений и стимуляции понимания человеческой природы, в дополнение к пониманию, присущему Адамической природе».
\vs p075 4:8 В ту ночь в Саду я разговаривал с отцом и матерью фиолетовой расы, что соответствовало моему долгу в этих печальных обстоятельствах. Я выслушал обстоятельный рассказ обо всем, что привело к срыву Матери Евы, и дал им обоим советы и наставления относительно сложившейся ситуации. Некоторым из этих советов они последовали, некоторые оставили без внимания. Эта встреча отражена в ваших летописях словами: «И воззвал Господь Бог к Адаму и Еве в Саду и спросил, где вы?». Позднейшие поколения обычно объясняли все необыкновенные и удивительные явления в жизни или в природе исключительно личным вмешательством Бога.
\usection{5. Последствия срыва}
\vs p075 5:1 Разочарование Евы было поистине отчаянным. Адам понимал всю сложность положения и, хотя был убит горем, испытывал к своей оступившейся супруге только сострадание и жалость.
\vs p075 5:2 И вот, в отчаянии от произошедшего, Адам на следующий день после проступка Евы отыскал Лаотту, прекрасную женщину из племени Нодитов, которая заведовала западными школами Сада, и преднамеренно совершил тот же безумный поступок, который совершила Ева. Не следует заблуждаться --- Адама никто не заставлял и не обманывал, он прекрасно знал, что делает. Он вполне сознательно решил разделить судьбу Евы. Он любил свою супругу неземной любовью, и мысль о том, что ему без нее придется влачить на Урантии одинокое существование, была для Адама невыносима.
\vs p075 5:3 Узнав о том, что случилось с Евой, разъяренные жители Сада стали неуправляемыми. Они объявили войну соседнему поселению Нодитов. Устремившись из ворот Эдема, они захватили врасплох этих людей, уничтожая все на своем пути --- не пощадили ни мужчин, ни женщин, ни детей. Погиб и Кано, отец еще не родившегося Каина.
\vs p075 5:4 Осознав, что произошло, Серапататья пришел в ужас, он был вне себя от страха и раскаянья. На следующий день он утопился в глубокой реке.
\vs p075 5:5 Дети Адама пытались утешить свою мать, обезумевшую от горя, в то время как их отец в течение тридцати дней скитался в одиночестве. К концу этого срока разум взял верх, и Адам возвратился домой и приступил к выработке плана дальнейших совместных действий.
\vs p075 5:6 Последствия безрассудных поступков, совершенных заблудшими родителями, так часто разделяются и их невинными детьми. Честные и благородные сыновья и дочери Адама и Евы невыразимо сожалели о немыслимой трагедии, которая так внезапно и жестоко обрушилась на них. И за пятьдесят лет старшие из детей не оправились от печали и горя тех трагических дней, особенно ужасны были тридцать дней, в течение которых отца не было дома, а обезумевшая от горя мать находилась в полном неведении относительно его местонахождения и судьбы.
\vs p075 5:7 Для Евы эти тридцать дней показались долгими годами горя и страданий. Эта благородная душа так никогда и не оправилась от последствий этого мучительного периода страданий ума и скорби духа. Ничто в последующих лишениях и тяжелых материальных испытаниях Евы невозможно даже сравнить с теми ужасными днями и страшными ночами одиночества и невыносимой неопределенности. Она знала о безумном поступке Сепарататьи и не знала, покончил ли с собой ее супруг от горя или же он удален с планеты в наказание за ее проступок. И когда Адам вернулся, Ева испытала чувство радости и благодарности, которое осталось в ней на всю жизнь, в течение всех лет долгого и нелегкого союза, связанного с выполнением тяжелейших обязанностей.
\vs p075 5:8 \pc Проходило время, но Адам не знал ничего определенного о том, как квалифицируется их проступок, до тех пор, пока через семьдесят дней после срыва Евы не вернулись на Урантию Мелхиседеки\hyp{}исполнители и не взяли в свои руки управление земными делами. И тогда он понял, что они потерпели неудачу.
\vs p075 5:9 \pc Но назревали дальнейшие несчастья: вести об уничтожении поселения Нодитов вблизи Эдема не замедлили дойти до родного племени Серапататьи на севере, и уже собралась огромная толпа, готовая двинуться на Сад. Это стало началом длительной и жестокой войны между Адамитами и Нодитами; их враждебность сохранялась в течении многих лет после того, как Адам и его сторонники переселились во второй сад в долине Евфрата. Существовала постоянная ожесточенная «вражда между тем мужчиной и этой женщиной, между его семенем и ее семенем».
\usection{6. Адам и Ева покидают Сад}
\vs p075 6:1 Когда Адам узнал, что приближаются Нодиты, он обратился за помощью к Мелхиседекам, но те отказались дать ему совет, говоря лишь, что он должен поступать так, как считает нужным, и обещая дружески сотрудничать с ним, насколько это будет возможно, в любом предприятии, на которое он решится. Мелхиседекам было запрещено вмешиваться в личные планы Адама и Евы.
\vs p075 6:2 Адам знал, что они с Евой потерпели неудачу. Об этом говорило присутствие Мелхиседеков\hyp{}исполнителей, хотя он ничего не знал ни о своем личном статусе, ни о будущей судьбе. Он провел совещание с тысячью двумястами верными сторонниками, которые дали торжественное обещание следовать за своим вождем. Совещание длилось всю ночь, а в полдень следующего дня эти пилигримы ушли из Эдема в поисках своего нового дома. Адам не хотел вести войну и поэтому решил отдать первый сад Нодитам без сопротивления.
\vs p075 6:3 На третий день своего исхода из Сада караван Эдемитов был остановлен прибытием серафимов перемещения из Иерусема. И тогда Адаму и Еве впервые сообщили о судьбе их детей. Серафимы перемещения ожидали, пока те из детей, кто достиг возраста выбора (двадцати лет), примет решение: остаться ли на Урантии вместе со своими родителями, или уйти под опеку Всевышних Норлатиадека. Две трети решили вернуться на Эдентию, около одной трети предпочли остаться с родителями. Все дети, не достигшие возраста выбора, были взяты на Эдентию. Каждый, кто наблюдал горестную картину разлуки Материальных Сына и Дочери со своими детьми, понимал, как тяжела участь правонарушителя. Эти потомки Адама и Евы находятся теперь на Эдентии; нам не известно, какие распоряжения будут сделаны относительно них.
\vs p075 6:4 Печаль и уныние охватили путников, когда караван готовился продолжить путешествие. Разве могло произойти что\hyp{}либо более трагическое! Прийти в мир полными светлых надежд, быть так хорошо принятыми, а затем бежать из Эдема обесчещенными и, еще не найдя нового пристанища, потерять три четверти своих детей!
\usection{7. Адам и Ева становятся смертными}
\vs p075 7:1 В то время, когда караван в Эдентию был остановлен, Адам и Ева были извещены о том, как квалифицируется их проступок, и о том, что их ждет в дальнейшем. Прибыл Гавриил, чтобы провозгласить решение суда. Приговор был таков: Планетарные Адам и Ева Урантии виновны в невыполнении обязательств; будучи правителями этого обитаемого мира, они нарушили обет о попечительстве.
\vs p075 7:2 Находясь в подавленном состоянии, вызванном чувством вины, Адам и Ева были, тем не менее, очень обрадованы сообщением, что их судьи в Спасограде сняли с них все обвинения в «неуважении к правительству вселенной». Они не считались виновными в бунте.
\vs p075 7:3 Эдемической паре было сообщено, что они низвели себя до положения смертных мира сего; что отныне они должны вести себя как урантийские мужчина и женщина, рассматривая будущее народов мира как свое собственное будущее.
\vs p075 7:4 Задолго до того, как Адам и Ева покинули Иерусем, их наставники подробно обрисовали им последствия любого существенного отклонения от божественных планов. Я лично многократно предупреждал их, до и после их прибытия на Урантию, что низведение до положения смертной плоти, несомненно, будет результатом, неотвратимой карой, которая неизбежно последует за срывом выполнения их планетарной миссии. Но для того, чтобы ясно понимать последствия срыва Адама и Евы, очень важно осознать, что такое бессмертие по отношению к материальному чину сыновства.
\vs p075 7:5 \pc \ublistelem{1.}\bibnobreakspace Адам и Ева, как и их товарищи в Иерусеме, сохраняли бессмертие благодаря подсоединению интеллекта к гравитационной цепи разума Духа. Если эта жизненно важная связь нарушена вследствие размыкания мыслительного контакта, то, независимо от духовного уровня существования живого существа, оно перестает быть бессмертным. Превращение в смертных и последующий физический распад были неизбежным результатом интеллектуального срыва Адама и Евы.
\vs p075 7:6 \pc \ublistelem{2.}\bibnobreakspace Материальные Сын и Дочь Урантии, получив облик по образу и подобию смертных мира сего, оказались в дальнейшем зависимы от двух систем кровообращения, одна из которых определялась их физической природой, а вторая функционировала благодаря сверхэнергии, заключенной в плодах дерева жизни. Архангел\hyp{}хранитель постоянно предупреждал Адама и Еву, что невыполнение обязательств опекунства приведет в результате к потере бессмертия, и доступ к источнику энергии был закрыт сразу же после их срыва.
\vs p075 7:7 \pc Калигастии удалось заманить Адама и Еву в ловушку, но он не достиг свой цели --- не смог склонить их к открытому бунту против правительства вселенной. Конечно, то, что они совершили, было злом, но их никогда нельзя было обвинить в презрении к истине, и они никогда сознательно не примыкали к бунту против справедливого правления Отца Всего Сущего и его Сына\hyp{}Творца.
\usection{8. Так называемое падение человека}
\vs p075 8:1 Адам и Ева действительно испытали падение с высот материального сыновства до скромного положения смертных людей. Но это было не падение человека. Несмотря на ближайшие последствия Адамического срыва, род человеческий испытал подъем. Хотя божественный план введения фиолетовой расы в семью народов Урантии не удался, смертные расы получили громадные преимущества даже в результате того ограниченного вклада, который Адам и его потомки внесли в расы Урантии.
\vs p075 8:2 «Падения человека» не было. История человеческого рода --- это история прогрессирующей эволюции, и Адамическое пришествие значительно улучшило первоначальное биологическое состояние народов мира. Роды Урантии, находящиеся на более высоком уровне развития, содержат теперь наследственные факторы, приобретенные, по меньшей мере, от четырех независимых источников: Андонитов, народов Сангика, Нодитов и Адамитов.
\vs p075 8:3 Не следует рассматривать Адама как причину проклятия рода человеческого. Хотя он и потерпел неудачу в реализации божественных планов, хотя он и нарушил договоренность с Божеством, хотя он и его супруга, несомненно, были низведены до уровня тварных созданий, несмотря на все это, их работа на благо человеческого рода чрезвычайно способствовала прогрессу цивилизации на Урантии.
\vs p075 8:4 \pc При оценке результатов Адамической миссии в вашем мире было бы справедливо принять во внимание условия, существовавшие на планете. Адам столкнулся с почти безнадежной задачей, когда вместе со своей прекрасной супругой был доставлен из Иерусема на эту темную, пребывающую в хаосе планету. Но если бы они следовали советам Мелхиседеков и их сподвижников, \bibemph{если бы они были более терпеливы,} они бы, в конце концов, достигли успеха. Но Ева уступила коварной пропаганде личной свободы и свободы действий на планете. Это привело ее к эксперименту с жизненной плазмой, принадлежащей материальному чину сыновства; эксперимент состоял в том, что она позволила этому жизненному агенту преждевременно смешаться с жизненным агентом смешанного порядка, который уже был соединен с плазмой размножающихся существ, когда\hyp{}то принадлежащих к штату Планетарного Принца.
\vs p075 8:5 В вашем восхождении к Раю вы никогда ничего не добьетесь, если будете пытаться второпях обойти предустановленный божественный план более коротким путем или используя какие\hyp{}либо собственные изобретения и поправочные устройства на пути совершенства, к совершенству и ради вечного совершенства.
\vs p075 8:6 \pc Вообще говоря, никогда мудрость не терпела более сокрушительного поражения ни на одной из планет в Небадоне. Но неудивительно, что такие сбои происходят в делах эволюционирующих вселенных. Мы --- часть гигантского мироздания, и не должно казаться странным, что не все идеально получается; наша вселенная не создана совершенной. Совершенство --- наша цель в вечности, а не наша исходная точка.
\vs p075 8:7 Если бы это была механистическая вселенная, если бы Великий Первоисточник и Центр был только силой, а не личностью и силой, если бы все сотворенное было огромным нагромождением физической материи, которая управлялась бы с помощью точных законов, характеризующихся неизменными действиями энергии, тогда можно было бы достичь совершенства, даже несмотря на незавершенность статуса вселенной. Не было бы разногласий, не было бы трений. Но в нашей эволюционирующей вселенной, обладающей относительными совершенством и несовершенством, мы рады тому, что возможны разногласия и размолвки, ибо в этом проявляется сам факт существования и действия личности во вселенной. И если наше мироздание есть бытие, в котором доминирует личность, тогда вам может быть гарантирована возможность выживания личности, возможность ее прогресса и развития; мы можем быть уверены, что личность будет расти, набираться опыта и переживать различные события. Как же великолепна эта вселенная, личностная и прогрессирующая, а не просто механистическая или даже пассивно совершенная!
\vsetoff
\vs p075 8:8 [Представлено Солонией, серафическим «голосом в Саду».]
