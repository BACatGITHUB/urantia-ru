\upaper{165}{Начало миссии в Перее}
\author{Комиссия срединников}
\vs p165 0:1 Во вторник 3 января 30 года н.э. Авенир, бывший руководитель двенадцати апостолов Иоанна Крестителя, назорей и некогда глава назарейской школы в Ен\hyp{}Геди, а ныне руководитель семидесяти вестников царства, собрал своих сподвижников и перед тем как отправить их с миссией во все города и селения Переи дал им заключительные наставления. Эта миссия в Перее продолжалась почти три месяца и была последним служением Учителя. После этих трудов Иисус пошел прямо в Иерусалим, чтобы, пройдя через последние испытания, завершить там свою жизнь во плоти. Семьдесят же вестников, периодически получая помощь от Иисуса и двенадцати апостолов, трудились в почти пятидесяти селениях и в следующих больших и малых городах: Зафон, Гадара, Макад, Арбела, Рамаф, Едрея, Босора, Каспин, Миспех, Гераса, Рагава, Суккот, Амафа, Адам, Пенуэл, Капитолий, Дион, Хатита, Гадда, Филадельфия, Иогбега, Гилеад, Беф\hyp{}Нимрах, Тир, Елеале, Ливия, Есевон, Каллирхоэ, Беф\hyp{}Пеор, Ситтим, Сибма, Медева, Беф\hyp{}Меон, Ареополь и Ароер.
\vs p165 0:2 Во время этого путешествия по Перее женский отряд, который насчитывал теперь 62 человека, принял на себя большую часть работы по оказанию помощи больным. То был завершающий период развития высших духовных аспектов евангелия царства, и потому не было места чудотворству. Ни в какой другой части Палестины не трудились столь основательно апостолы и ученики Иисуса, и ни в каком другом районе лучшие слои населения не принимали столь единодушно учение Учителя.
\vs p165 0:3 В Перее в это время было почти одинаковое число неевреев и евреев, поскольку последние в основном были изгнаны из этих областей во времена Иуды Маккавея. Перея была самой прекрасной и живописной провинцией всей Палестины. Как правило, евреи называли ее «землей за Иорданом».
\vs p165 0:4 Все это время Иисус или находился в лагере в Пелле, или путешествовал с двенадцатью апостолами, помогая семидесяти вестникам в различных городах, где те учили и проповедовали. По указанию Авенира семьдесят вестников крестили всех верующих, хотя Иисус и не поручал им этого.
\usection{1. В лагере в Пелле}
\vs p165 1:1 К середине января в Пелле собралось более тысячи двухсот человек, и когда Иисус находился в лагере, то учил их по крайней мере один раз в день, начиная, как правило, с девяти часов утра, если только не мешал дождь. Петр и другие апостолы тоже учили каждый день, но уже после полудня. Вечера же Иисус посвящал общению с двенадцатью апостолами и другими наиболее способными учениками. В среднем в вечерние часы собиралось до пятидесяти человек.
\vs p165 1:2 К середине марта, ко времени, когда Иисус начал свое путешествие к Иерусалиму, уже более четырех тысяч человек составляли огромную аудиторию, которая каждое утро слушала проповеди Иисуса или Петра. Учитель решил завершить свой труд на земле, когда интерес к его вести достиг апогея, наивысшей точки в этой второй, лишенной чудес, стадии распространения царства. Хотя три четверти этой массы были искателями истин, в ней также присутствовали многочисленные фарисеи из Иерусалима и других мест, а также всевозможные скептики и казуисты.
\vs p165 1:3 Иисус и двенадцать апостолов большую часть своего времени уделяли народу, собравшемуся в лагере в Пелле. Двенадцать апостолов почти не работали за пределами лагеря и лишь время от времени вместе с Иисусом ходили навестить соратников Авенира. Авенир очень хорошо знал Перейскую область, так как именно здесь в основном трудился его бывший учитель Иоанн Креститель. Начав миссию в Перее, Авенир и семьдесят вестников, однако, больше не возвращались в Перейский лагерь.
\usection{2. Проповедь о добром пастыре}
\vs p165 2:1 Более трехсот жителей Иерусалима, фарисеев и других последовали за Иисусом на север в Пеллу, когда тот после окончания праздника посвящения поспешил уйти из области, подчиненной еврейским правителям; и именно в присутствии этих еврейских учителей, руководителей и двенадцати апостолов Иисус произнес проповедь «О добром пастыре». После получасовой неформальной дискуссии, обращаясь к толпе, состоявшей примерно из ста человек, Иисус сказал:
\vs p165 2:2 \pc «Этой ночью я должен вам многое сказать, и поскольку многие из вас --- мои ученики, а некоторые из вас --- мои злейшие враги, я изложу мое учение в притче, так чтобы каждый из вас мог взять для себя то, что принимает его сердце.
\vs p165 2:3 Сегодня ночью здесь передо мной есть люди, которые готовы умереть за меня и сие евангелие царства, и в предстоящие годы многих из них постигнет эта участь; есть здесь и рабы традиций, которые последовали за мной из Иерусалима и которые заодно с вашими ослепленными и заблуждающимися вождями стремятся убить Сына Человеческого. Жизнь, которой я ныне живу во плоти, рассудит и тех и других --- истинных и ложных пастырей. Если бы ложный пастырь был слеп, то не имел бы греха, но как вы говорите, что видите, и называете себя учителями в Израиле, грех ваш остается на вас.
\vs p165 2:4 Истинный пастырь в опасное время на ночь сгоняет свое стадо на двор. И когда настает утро, входит во двор через дверь, и когда зовет, овцы знают голос его. Всякий пастырь, входящий на овечий двор не через дверь, а иначе --- вор и разбойник. Истинный пастырь входит на двор после того, как придверник откроет ему дверь, и овцы его, зная его голос, выходят на слово его; и когда выходят те, что принадлежат ему, истинный пастырь идет перед ними; он указывает путь, а овцы идут за ним. Овцы его идут за ним, потому что знают голос его, а за чужим не пойдут. От чужого разбегутся, потому что не знают голоса его. Сия толпа, собравшаяся здесь вокруг нас, все равно, что овцы без пастыря, но когда мы говорим им, они знают наш голос и идут за нами; по крайней мере те, кто алчет истины и жаждет праведности. Иные же из вас не из моего стада; вы моего голоса не знаете и за мной не пойдете. И потому что вы ложные пастыри, овцы не знают вашего голоса и за вами не пойдут».
\vs p165 2:5 \pc И когда Иисус поведал эту притчу, никто не задал ему ни одного вопроса. Спустя какое\hyp{}то время он продолжил свою речь и перешел к обсуждению притчи:
\vs p165 2:6 «Вы, желающие стать подпасками стад Отца моего, должны быть не только достойными руководителями, но также должны \bibemph{кормить} стадо доброй пищей; вы не есть истинные пастыри, если не ведете стада ваши на зеленые пастбища у тихих вод.
\vs p165 2:7 Теперь же, чтобы иные из вас не слишком буквально поняли сию притчу, объявляю, что я есть и дверь овечьего двора, и одновременно истинный пастырь стадам Отца моего. Всякий пастырь, пытающийся войти на двор без меня, не сможет, и овцы не услышат голоса его. Я и те, кто служит со мной, есть дверь. Всякая душа, вступающая на вечный путь посредством того, что я сотворил и предписал, спасена будет и сможет идти вперед вплоть до достижения вечных пастбищ Рая.
\vs p165 2:8 Но я также истинный пастырь, готовый и жизнь свою положить за овец. Вор врывается на двор только для того, чтобы украсть, убить и погубить; я же пришел для того, чтобы все вы имели жизнь и имели с избытком. Наемник, когда возникает опасность, бежит и допускает, чтобы овец разогнали и погубили; истинный же пастырь не бежит, когда приходит волк; он защищает стадо свое, а если нужно, то и жизнь свою отдает за овец. Истинно, истинно говорю вам, друзья и враги, я пастырь истинный; я знаю моих и мои знают меня. Перед лицом опасности я не убегу. Я завершу мое служение и до конца исполню волю Отца моего и не брошу стадо, которое Отец поручил мне охранять.
\vs p165 2:9 Но есть у меня много других овец не сего двора, и слова эти истинны не только для мира сего. Эти другие овцы тоже слышат и знают голос мой, и я обещал Отцу и их привести на один двор, в одно братство сыновей Бога. И тогда все вы будете знать голос одного пастыря, пастыря истинного, и все признаете Бога Отцом.
\vs p165 2:10 И так узнаете, почему Отец любит меня и отдал все стада свои в руки мои на хранение; это потому, что Отец знает, что я не дрогну, охраняя овечий двор, не брошу овец моих и, если потребуется, не побоюсь положить жизнь мою в служении многочисленным стадам его. Однако знайте, если я отдам жизнь мою, то приму ее снова. Ни человек, ни любое иное творение не может отнять жизнь мою. Я имею право и власть отдать жизнь мою и такую же власть и право опять принять ее. Вы не можете понять этого, но я получил такую власть от Отца моего еще прежде, чем был этот мир».
\vs p165 2:11 \pc Услышав эти слова, апостолы смутились, ученики изумились, фарисеи же из Иерусалима и прочих мест ушли в ночь, говоря: «Он либо безумен, либо одержим бесом». Но даже некоторые из иерусалимских учителей говорили: «Он говорит как власть имеющий; а кроме того, видел ли кто когда\hyp{}нибудь, чтобы одержимый бесом отверзал очи слепорожденного и творил чудеса, которые сотворил человек сей?»
\vs p165 2:12 На следующий день половина этих еврейских учителей открыто признались в своей вере в Иисуса, другая же половина в страхе вернулась в Иерусалим и в дома свои.
\usection{3. Субботняя проповедь в Пелле}
\vs p165 3:1 К концу января толпы, собиравшиеся в субботу после полудня, насчитывали до трех тысяч человек. В субботу 28 января Иисус произнес памятную проповедь «Доверие и духовная подготовленность». После вступительного слова Симона Петра Учитель сказал:
\vs p165 3:2 \pc «То, что я много раз говорил апостолам моим и ученикам моим, ныне возвещаю сему множеству: берегитесь закваски фарисейской, которая суть лицемерие, рожденное предрассудком и взращенное в рабстве традиций, хотя многие из фарисеев честны сердцем и иные из них пребывают здесь как мои ученики. Вскоре все вы поймете учение мое, ибо нет ничего тайного, что не сделалось бы явным. То, что ныне сокрыто от вас, все станет известным, когда Сын Человеческий завершит свою миссию на земле и во плоти.
\vs p165 3:3 Скоро, очень скоро то, что ныне замышляют враги наши в тайне и во тьме, явится на свет и провозглашено будет на кровлях. Говорю же вам, друзьям моим, не бойтесь их, когда они ищут погубить Сына Человеческого. Не бойтесь тех, кто хоть и может убить тело, после сего более не властен над вами. Призываю вас никого не бояться, ни на небе, ни на земле, но радоваться знанию Имеющего власть избавить вас от всей неправедности и поставить вас непорочными перед судом вселенной.
\vs p165 3:4 Не пять ли воробъев продаются за две копейки? И все же, когда птицы эти летают, ища пропитания, ни одна из них не живет вне знания Отца, источника всей жизни. Для серафимов\hyp{}хранительниц и волосы на головах ваших сочтены. Если же все это правда, то отчего живете в страхе перед многими пустяками, случающимися в вашей повседневной жизни? Говорю вам: не бойтесь; вы гораздо дороже многих малых птиц.
\vs p165 3:5 Всех из вас, имеющих смелость исповедовать веру в мое евангелие перед людьми, я вскоре признаю перед ангелами небесными; но кто сознательно отвергает истину моих учений перед людьми, тот отвержен будет его хранительницей предназначения даже перед ангелами небесными.
\vs p165 3:6 Говорите о Сыне Человеческом, что хотите, и будет прощено вам; но кто осмелится Бога хулить, тот едва ли найдет прощение. Когда люди заходят столь далеко, что сознательно приписывают деяния Бога силам зла, то такие злонамеренные бунтовщики едва ли будут искать прощения за свои грехи.
\vs p165 3:7 И когда враги наши приведут вас к начальникам синагог и другим высоким властям, не заботьтесь, что говорить, и не тревожьтесь, как отвечать на вопросы их, ибо дух, пребывающий в вас, обязательно научит вас в тот час, что говорить вам во славу евангелия царства.
\vs p165 3:8 Долго ли будете оставаться в долине решения? Почему колеблетесь между двух мнений? Почему еврей или нееврей не решается принять благую весть о том, что он --- сын вечного Бога? Сколько времени потребуется нам, чтобы убедить вас с радостью войти в ваше духовное наследие? Я пришел в этот мир, дабы открыть вам Отца и отвести вас к Отцу. Первое я уже сделал, последнего же без вашего на то согласия сделать не могу; Отец никогда и никого не принуждает входить в царство. Приглашение же всегда было и всегда будет таким: кто хочет, приходи и вольно вкушай воды жизни».
\vs p165 3:9 \pc Когда Иисус кончил говорить, многие пошли креститься у апостолов в Иордане, Иисус же внимал вопросам тех, кто остался.
\usection{4. Раздел наследства}
\vs p165 4:1 Пока апостолы крестили верующих, Учитель беседовал с оставшимися с ним. Один молодой человек сказал ему: «Учитель, отец мой умер, оставив большое имение мне и брату моему, но брат мой отказывается отдать мне мое. Не прикажешь ли брату моему разделить наследство со мной?» Иисус был в какой\hyp{}то мере возмущен тем, что этот думающий о материальном юноша выдвинул для обсуждения такой деловой вопрос; но решил воспользоваться случаем для дальнейшего наставления. Иисус сказал: «Человек, кто поставил меня делить вас? Почему ты решил, что я занимаюсь материальными делами мира сего?» И затем, обращаясь ко всем, кто был рядом с ним, сказал: «Смотрите, берегитесь любостяжания, ибо жизнь человека не зависит от изобилия его имения. Счастье приходит не от власти, которую дает богатство, и радость возникает не от изобилия. Богатство само по себе не беда, но любовь к богатству часто приводит к такой зависимости от вещей мира сего, что душа становится слепой к благодатной привлекательности духовных реалий царства Бога на земле и к радостям вечной жизни на небе.
\vs p165 4:2 \pc Позвольте мне рассказать вам историю об одном богатом человеке, чья земля давала обильный урожай; сильно разбогатев, он стал рассуждать сам с собою, говоря: „Что мне делать со всем моим богатством? Теперь я имею так много, что мне некуда собрать богатство мое“. Поразмыслив над этой проблемой, он сказал: „Вот что сделаю; сломаю житницы мои и построю большие; таким образом, у меня будет достаточно места, где хранить мои плоды и добро мое. Тогда смогу сказать душе моей: душа, много добра лежит у тебя на многие годы; покойся, ешь, пей и веселись, ибо ты богата и прибавилось добра у тебя“.
\vs p165 4:3 Но сей богатый человек был к тому же глуп. Беспокоясь о материальных потребностях своего ума и тела, он не заботился скопить сокровища на небе для удовлетворения духа и спасения души. И даже тогда ему не довелось насладиться удовольствием от обладания своим несметным состоянием, ибо в ту самую ночь потребовали у него душу его. В ту ночь разбойники ворвались в его дом, чтобы убить его, и, ограбив его житницы, оставшееся добро сожгли. За имущество же, уцелевшее после грабителей, стали бороться между собой наследники его. Сей человек копил для себя сокровища на земле, а не в Боге богател».
\vs p165 4:4 \pc Иисус так обошелся с молодым человеком и его наследством, поскольку знал, что корыстолюбие --- это его беда. Даже если бы это было не так, Учитель все равно бы не вмешался, ибо он никогда не занимался мирскими делами даже своих апостолов, и уж своих учеников тем более.
\vs p165 4:5 Когда Иисус закончил свой рассказ, другой человек встал и спросил его: «Учитель, я знаю, что твои апостолы, чтобы следовать за тобой, продали все свое земное имение и что у них, как у ессеев, все общее, однако хочешь ли ты, чтобы все мы, ученики твои, поступали так же? Грешно ли обладать честным богатством?» И Иисус ответил на этот вопрос: «Друг мой, иметь честное богатство не грешно; но грех, если ты превращаешь материальное имение \bibemph{в сокровища,} которые могут увлечь тебя и отвратить тебя от стремления посвятить себя духовным исканиям царства. В обладании честным имением на земле нет греха при условии, что \bibemph{сокровище} твое на небесах, ибо где сокровище твое, там и сердце твое будет. Существует огромная разница между богатством, ведущим к корыстолюбию и эгоизму, и богатством, которым владеют и которое раздается в духе попечительства теми, кто имеет изобилие благ мира сего и столь щедро жертвует на поддержку тех, кто все свои силы посвящает делу царства. Многие из вас, присутствующих здесь и не имеющих денег, питаются и живут в этом палаточном городке, потому что щедрые и состоятельные мужчины и женщины пожертвовали средства вашему хозяину Давиду Зеведееву на эти цели.
\vs p165 4:6 Но никогда не забывайте, что в конце концов богатство непостоянно. Слишком часто любовь к богатству замутняет и даже разрушает духовное зрение. Не просмотрите момент, когда богатство из вашего слуги превратится в вашего хозяина».
\vs p165 4:7 \pc Иисус не проповедовал и не одобрял расточительность, праздность, нежелание обеспечить физические потребности своей семьи необходимыми предметами или нищенство. Но он учил, что материальное и временное должно быть подчинено благополучию души и совершенствованию духовной природы в царстве небесном.
\vs p165 4:8 \pc Затем, когда народ пошел к реке, чтобы посмотреть на крещение, человек, который спрашивал Иисуса о своем наследии, снова подошел к нему, чтобы поговорить наедине, так как считал, что Иисус обошелся с ним сурово; выслушав его снова, Учитель ответил: «Сын мой, почему упускаешь возможность вкусить хлеба живого в такой день ради потворства своему корыстолюбивому нраву? Разве не знаешь, что еврейские законы о наследстве будут справедливо исполнены, если ты со своей жалобой придешь в суд синагоги? Неужели не видишь, что мое дело связано с тем, чтобы ты узнал о своем небесном наследии? Разве не читал ты в Писании: „Иной делается богатым от осмотрительности и бережливости своей, и это часть награды его; он говорит: Я нашел покой и теперь смогу постоянно вкушать от благ моих и не знает, что принесет время ему и что он должен будет оставить все другим, когда умрет“. Разве не читал ты заповедь: „Не желай“. И еще: „Они ели и насыщались и тучнели и затем обратились к иным богам“. Читал ли ты в Псалтыри, что „Господь ненавидит корыстолюбца“ и что „малое у праведника лучше богатства многих нечестивых“. „Когда богатство умножается, не прилагай к нему сердца твоего“. Читал ли книгу Иеремии, где сказано: „Да не уповай богатый на богатство свое“; и Иезекииль говорил правду, когда сказал: „Устами своими они показывали любовь, но в сердцах своих увлечены корыстью их“».
\vs p165 4:9 Отсылая молодого человека, Иисус сказал ему: «Сын мой, какая польза тебе, если ты весь мир приобретешь, а душу свою потеряешь?»
\vs p165 4:10 Другому же, стоявшему рядом, спросившему у него, что будет с богатыми в судный день, Иисус ответил: «Не богатых и не бедных пришел я судить, но вершить суд над всеми людьми будет жизнь, которой они живут. О чем же спросят в суде у богатых? Всем, кто приобрел великое богатство, придется ответить, по крайней мере, на три вопроса, и вопросы эти таковы:
\vs p165 4:11 \ublistelem{1.}\bibnobreakspace Какое богатство ты накопил?
\vs p165 4:12 \ublistelem{2.}\bibnobreakspace Как получил это богатство?
\vs p165 4:13 \ublistelem{3.}\bibnobreakspace Как распорядился своим богатством?»
\vs p165 4:14 \pc Затем Иисус вошел в палатку, чтобы немного отдохнуть перед вечерней трапезой. Закончив крещения, пришли и апостолы и хотели с ним поговорить о богатстве на земле и сокровище на небесах, но Иисус спал.
\usection{5. Беседа о богатстве с апостолами}
\vs p165 5:1 В тот же вечер после ужина, когда Иисус и двенадцать апостолов собрались на свою ежевечернюю беседу, Андрей спросил: «Учитель, пока мы крестили верующих, ты сказал оставшимся много слов, которых мы не слышали. Не хочешь ли повторить слова эти и нам?» И в ответ на просьбу Андрея Иисус сказал:
\vs p165 5:2 \pc «Да, Андрей, я поговорю с вами о богатстве и хлебе насущном, но слова мои, обращенные к вам, апостолам, должны несколько отличаться от сказанных ученикам и толпе, поскольку вы бросили все не только затем, чтобы следовать за мной, но и затем, чтобы быть посвященными в посланцы царства. Вы обладаете уже многолетним опытом и знаете, что Отец, чье царство вы возвещаете, не оставит вас. Вы посвятили свои жизни служению царству; поэтому не беспокойтесь и не тревожьтесь о вещах временной жизни, что вам есть, ни о теле своем, во что одеться. Благоденствие души больше пищи и пития; совершенствование в духе выше потребности в одежде. Когда сомнение в том, что вы получите хлеб ваш, искушает вас, смотрите на воронов; они не сеют, ни жнут, ни собирают в житницы, и все же Бог дает пищу каждому ищущему из них. Насколько же ценнее вы птиц многих! А кроме того, все ваше беспокойство и мучительные сомнения ничем не могут помочь удовлетворить ваши материальные потребности. Кто из вас, заботясь, может прибавить себе росту хотя бы на локоть или хоть на день продлить свою жизнь? Раз решение подобных вопросов не в ваших руках, что тревожитесь и думаете об этих проблемах?
\vs p165 5:3 Посмотрите на лилии, как они растут; они не трудятся, ни прядут, но говорю вам, что и Соломон во всей славе своей не одевался так, как всякая их них. Если же траву полевую, которая сегодня есть, а завтра будет скошена и брошена в печь, Бог так одевает, то насколько же лучше оденет он вас, посланцев царства небесного. О маловеры! Целиком посвящая себя провозглашению царства небесного, не беспокойтесь о том, как содержать себя или семьи, оставленные вами. Если истинно отдаете жизни ваши евангелию, то и живите по евангелию. Если же вы только верующие ученики, то зарабатывайте хлеб свой и давайте на поддержание всех тех, кто учит, проповедует и исцеляет. Если тревожитесь о хлебе вашем и о воде, то чем отличаетесь от народов мира, столь усердно заботящихся о подобных потребностях? Отдавайте себя делу вашему, веруя, что Отец и я знаем, что вы имеете нужду во всем этом. Позвольте мне раз и навсегда заверить вас: если вы посвятите жизни ваши делу царства, то все ваши мирские потребности будут удовлетворены. Ищите большего и в нем найдете меньшее; просите о небесном и земное приложится вам. Ведь тень обязательно следует за сутью.
\vs p165 5:4 Вас немного, но если будете иметь веру, если не споткнетесь в страхе, то, объявляю, Отец соблаговолит отдать вам царство сие. Вы сложили сокровища ваши туда, где мешки не ветшают, где воры не грабят и моль не съедает. И, как сказал я народу, где сокровище ваше, там будет и сердце ваше.
\vs p165 5:5 Однако в деле, которое ждет нас, и в деле, которое останется вам, когда я уйду к Отцу, вы подвергнетесь горестным испытаниям. Все вы должны быть бдительны, чтобы дать отпор страху и сомнениям. Каждый из вас да препояшет чресла ума своего и да хранит светильник свой горящим. Будьте подобны людям, ожидающим возвращения господина своего с брачного праздника, дабы, когда придет и постучит, тотчас отворить ему. Таких бдительных слуг благословит господин, нашедший их бодрствующими в столь великую минуту. И усадит господин слуг своих и сам будет служить им. Истинно, истинно говорю вам: впереди у вас перелом в жизнях ваших; будьте же и вы бдительны и готовы.
\vs p165 5:6 Вы понимаете, что никто не допустил бы, чтобы в дом его ворвались, если бы знал, в который час придет вор. Бодрствуйте и вы о себе, ибо в час, о котором меньше всего подозреваете, и так, как не думаете, уйдет Сын Человеческий».
\vs p165 5:7 \pc Несколько минут двенадцать апостолов сидели молча. Некоторые из этих предостережений они уже слышали прежде, но совсем не в таком контексте, в каком услышали на этот раз.
\usection{6. Ответ на вопрос Петра}
\vs p165 6:1 Пока они сидели и думали, Симон Петр спросил: «К нам ли, твоим апостолам, говоришь эту притчу, или и ко всем ученикам?» И Иисус ответил:
\vs p165 6:2 \pc «Во время испытания открывается душа человека; испытание показывает, что действительно у человека в сердце. Когда слуга испытан и проверен, тогда домоправитель может поставить такого слугу над домом своим и спокойно доверить сему верному слуге смотреть, чтобы детей его кормили и воспитывали. Так и я вскоре узнаю, кому можно доверить благополучие детей моих, когда должен буду вернуться к Отцу. Как домоправитель поставит верного и испытанного слугу над делами семьи своей, так и я возвышу в делах моего царства тех, кто перенесет испытания часа сего.
\vs p165 6:3 Но если слуга ленив и скажет в сердце своем: „Не скоро придет господин мой“ и начнет обижать своих товарищей\hyp{}слуг, есть и пить с пьяными, тогда придет господин того слуги во время, в которое тот не ожидает его, и, найдя его неверным себе, изгонит его с позором. Поэтому вы правильно делаете, что готовите себя ко дню, когда внезапно посетят вас и притом так, как вы не ожидаете. Помните, что много вам дано, а потому от вас много потребуется. Испытания огнем приближаются к вам. Крещением должен я креститься и я настороже, пока сие совершится. Вы проповедуете мир на земле, но моя миссия не принесет мира в мирские дела людей --- так будет, по крайней мере, какое\hyp{}то время. Лишь разлад может произойти, когда два члена семьи верят в меня, а три отвергают сие евангелие. Друзьям, родственникам и близким суждено восстать друг против друга от евангелия, которое вы проповедуете. Верно, каждый из верующих будет иметь великий и вечный мир в сердце своем, но мир на земле не настанет, пока все не будут готовы верить и войти в свое славное наследие сыновства Бога. Тем не менее, идите по всему миру, возвещая евангелие всем народам --- каждому мужчине, каждой женщине и каждому ребенку».
\vs p165 6:4 \pc И на этом закончилась насыщенная событиями и делами суббота. На следующий день Иисус и двенадцать апостолов пошли в города северной Переи, чтобы встретиться с семьюдесятью вестниками, трудившимися в этих областях под руководством Авенира.
