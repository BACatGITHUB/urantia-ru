\upaper{84}{Брак и семейная жизнь}
\author{Глава Серафимов}
\vs p084 0:1 Материальная необходимость основала брак, половая потребность его украсила, религия освятила и возвысила, государство в нем нуждалось и его регулировало, тогда как в последнее время усилившееся чувство любви начинает оправдывать и прославлять брак как прародителя и создателя самого полезного и возвышенного института цивилизации --- института семьи. Строительство семьи должно быть центром и сущностью всего образования.
\vs p084 0:2 Половые отношения --- это лишь акт самоувековечения, сочетаемый с различными степенями самоудовлетворения; брак, строительство семьи --- это во многом вопрос самоподдержания, подразумевающий эволюцию общества. Само же общество есть совокупная структура семейных ячеек. Индивидуумы как планетарные факторы весьма временны, и только семьи являются постоянными факторами в эволюции общества. Семья --- это русло, по которому река культуры и знаний течет от поколения к поколению.
\vs p084 0:3 Семья --- прежде всего институт социологический. Брак произошел от сотрудничества в самоподдержании и партнерства в самоувековечении, при этом элемент самоудовлетворения был во многом случайным. Тем не менее семья объемлет все три основные функции человеческого бытия, в то же время рождение новой жизни делает ее основным человеческим институтом, а половые отношения выделяют из всех остальных видов общественной деятельности.
\usection{1. Первобытные связи между двумя людьми}
\vs p084 1:1 Брак не был основан на половых отношениях, которые к тому же носили случайный характер. Брак не был нужен первобытному человеку, который свободно удовлетворял свое половое влечение, не обременяя себя ответственностью по отношению к жене, детям и семье.
\vs p084 1:2 Из\hyp{}за физической и эмоциональной привязанности к своему потомству женщина зависит от сотрудничества с мужчиной, и это вынуждает ее искать защиту и кров, которые дает брак. Однако нет прямого побуждения, которое бы вело к браку мужчину, а тем более удерживало бы его в нем. Не любовь сделала брак привлекательным для мужчин, а голод, притягивающий первобытного мужчину к женщине и примитивному крову, который разделяли с ней ее дети.
\vs p084 1:3 \pc Возникновение брака не было вызвано и сознательным пониманием обязательств, вытекающих из половых отношений. Первобытный человек не осознавал связи между удовлетворением полового влечения и последующим рождением ребенка. Одно время повсеместно верили, что девственница может забеременеть. У первобытных людей рано зародилось представление, будто детей делают в стране духов; беременность считалась следствием того, что в женщину вошел дух, развивающийся призрак. Считалось также, что беременность у девственницы или незамужней женщины способны вызывать как особая пища, так и дурной глаз, а более поздние верования связывали истоки жизни с дыханием и солнечным светом.
\vs p084 1:4 Многие древние народы ассоциировали призраков с морем; поэтому девственницам не разрешалось часто мыться; молодые женщины боялись купаться в море во время прилива гораздо больше, чем половых отношений. К уродам или недоношенным детям относились как к детенышам животных, проникшим в тело женщины вследствие легкомысленного купания или злых козней духа. Первобытные люди, естественно, считали, что удушить такое потомство при рождении --- обычное дело.
\vs p084 1:5 Просвещение началось с веры в то, что половые отношения открывают путь для оплодотворяющего духа и дают ему войти в женщину. С этого момента человек открыл для себя, что отец и мать поровну предоставляют живые наследственные факторы, которые зачинают потомство. Однако даже в двадцатом веке многие родители все еще пытаются удержать своих детей в большем или меньшем неведении относительно того, как зарождается человеческая жизнь.
\vs p084 1:6 \pc Создание семьи достаточно простого типа делалось возможным благодаря тому, что репродуктивная функция обусловливает отношения между матерью и ребенком. Материнская любовь инстинктивна; в отличие от брака она произошла не от нравов. Материнская любовь у всех млекопитающих является врожденным даром духов\hyp{}помощников разума локальной вселенной и своей силой и преданностью всегда прямо пропорциональна продолжительности беспомощного младенчества у вида.
\vs p084 1:7 Связь между матерью и ребенком естественна, сильна и инстинктивна, а потому вынуждала первобытных женщин подчиняться множеству странных условий и терпеть невыразимые трудности. Это чувство непреодолимой материнской любви мешает женщине, ибо всегда ставило ее в чрезвычайно невыгодное положение в любой борьбе с мужчиной. Но даже при этом материнский инстинкт у людей в принципе преодолим и может быть подавлен честолюбием, эгоизмом и религиозными убеждениями.
\vs p084 1:8 Хотя союз матери и ребенка не является ни браком, ни семьей, он был ядром, из которого возникли и брак, и семья. Великий прогресс в эволюции половых отношений был достигнут, когда эти временные союзы стали сохраняться достаточно долго, что позволяло вырастить появляющееся благодаря им потомство, ибо это и было фактом создания семьи.
\vs p084 1:9 Независимо от антагонизмов, существовавших в древних супружеских парах, несмотря на непрочность их связи, благодаря таким союзам мужчины и женщины вероятность выживания сильно возросла. Мужчина и женщина, сотрудничающие хотя бы и вне семьи и потомства, в большинстве отношений превосходят как двоих мужчин, так и двух женщин. Такое объединение в пары представителей противоположных полов повышало шансы на выживание и положило начало человеческому обществу. Разделение труда по половому признаку также сделало жизнь удобнее и счастливее.
\usection{2. Древняя семья во главе с матерью}
\vs p084 2:1 Периодическое кровотечение у женщин и дальнейшая потеря крови при родах довольно рано привели к предположению, что кровь является создателем ребенка (равно как и местом, где находится душа), и породили идею о кровных узах в человеческих отношениях. В древности любое происхождение велось по женской линии, поскольку она была единственным звеном родословной, в котором вообще можно было быть уверенным.
\vs p084 2:2 Первобытная семья, возникшая вследствие инстинктивных биологических кровных уз, связующих мать и ребенка, неизбежно была семьей, во главе которой стояла мать; и многие племена долго придерживались этого порядка. Семья, во главе которой стояла мать, была единственно возможной при переходе от группового брака в стаде к более поздней и развитой семейной жизни полигамных и моногамных семей во главе с отцом. Семья во главе с матерью была естественной и биологической; семья же, во главе которой стоит отец, является социальной, экономической и политической. У североамериканского красного человека именно сохранившийся порядок семьи во главе с матерью и является одной из главных причин, почему развитые во многих отношениях ирокезы так и не сумели создать настоящее государство.
\vs p084 2:3 В семьях, во главе которых стояла мать, мать жены пользовалась фактически верховной властью; и даже братья жены и их сыновья в управлении семьей играли более активную роль, нежели муж. Отцам часто давали новое имя в честь их же детей.
\vs p084 2:4 Древнейшие расы мало уважали отца, считая, что ребенок полностью происходит от матери. Они верили, что дети похожи на отца вследствие общения с ним или что они таким образом «помечены», потому что мать хотела, чтобы они походили на отца. Позднее, когда произошел переход от семьи, во главе которой стояла мать, к семье во главе с отцом, появление на свет ребенка целиком и полностью считалось заслугой отца, и ко многим табу на беременную женщину впоследствии добавились и другие, включившие в сферу своего действия и ее мужа. С приближением родов будущий отец прекращал работать и во время родов ложился в постель рядом с женой, отдыхая от трех до восьми дней. Жена могла встать и заняться тяжелым трудом уже на следующий день после родов, зато муж оставался в постели получать поздравления; все это было частью древних нравов, служивших утверждению права отца на ребенка.
\vs p084 2:5 Вначале было принято, чтобы мужчина приходил жить к родственникам жены, однако позднее, после того, как мужчина выплачивал или отрабатывал выкуп за невесту, он мог забрать свою жену и детей и вернуться к своим родственникам. Переход от семьи, во главе которой стояла мать, к семье во главе с отцом объясняет, почему в одних случаях существовали бессмысленные запреты на некоторые типы браков между двоюродными братом и сестрой при том, что в других практически такие же браки между людьми, состоящими в той же степени родства, одобрялись.
\vs p084 2:6 По мере того, как уходило время, когда главным занятием человека была охота, и наступала эпоха скотоводства, позволившего человеку овладеть главным источником пищи, семья, во главе которой стояла мать, быстро исчезала. Она перестала существовать просто потому, что не могла успешно соперничать с семьей нового типа --- во главе с отцом. Сила, которой располагали мужчины --- родственники матери не могла противостоять силе, сосредоточенной у мужа и отца. Женщина была не в состоянии справиться со сложными задачами --- рожать и воспитывать детей и в то же время неустанно руководить и управлять всем домом. Появление обычая похищения жен с последующим их выкупом ускорило процесс отмирания семьи во главе с матерью.
\vs p084 2:7 Имевший огромное значение переход от семьи, возглавляемой матерью, к семье во главе с отцом --- один из самых решительных и крутых поворотов, когда\hyp{}либо сделанных человечеством. Эта перемена сразу же привела к тому, что общество стало более четко выраженным, а семейная жизнь --- богаче.
\usection{3. Семья под властью отца}
\vs p084 3:1 Возможно, что женщину к браку привел инстинкт материнства, однако оставаться в нем ее фактически вынуждали мужское превосходство в силе и влияние нравов. Жизнь скотоводов имела тенденцию к созданию новой системы нравов --- семейной жизни патриархального типа, и основой единства семьи в условиях нравов времен скотоводства и начала земледелия являлась неоспоримая и деспотическая власть отца. Всякое общество, будь то национальное или семейное, прошло через этап автократической власти патриархального строя.
\vs p084 3:2 Неуважительное отношение к женщине в ветхозаветную эру является точным отражением нравов пастухов. Все еврейские патриархи были пастухами, о чем свидетельствует высказывание: «Господь --- Пастырь мой».
\vs p084 3:3 Однако в своем невысоком мнении о женщине мужчина был виноват не больше самой женщины. В первобытные времена женщина не смогла добиться общественного признания, потому что она не действовала при чрезвычайных обстоятельствах и не была героем в особых или критических ситуациях. В борьбе за существование материнство было явным препятствием; материнская любовь мешала женщине, когда приходилось защищать племя.
\vs p084 3:4 Первобытные женщины к тому же сами бессознательно создавали свою зависимость от мужчины, восхищаясь и восторгаясь его мужеством и умением драться. Такое возвеличивание воина возвышало собственное «я» мужчины и одновременно настолько же понижало собственное «я» женщины, делая ее более зависимой; военная форма до сих пор вызывает у женщин сильные эмоции.
\vs p084 3:5 У более развитых рас женщины более хрупкого телосложения и менее сильные, чем мужчины. Поэтому, будучи слабыми, женщины стали более тактичными и рано научились извлекать выгоду из прелестей своего пола. Женщина стала бдительнее и консервативнее мужчины, хотя и несколько более поверхностной. Мужчина превосходил женщину на поле брани и в охоте; однако в семье женщина, как правило, брала верх даже над самыми примитивными из мужчин.
\vs p084 3:6 \pc Пастух смотрел на свои стада как на средство к существованию, однако на всем протяжении эпохи скотоводства женщина все равно должна была добывать растительную пищу. Первобытный мужчина избегал работы на земле; это занятие было слишком мирным и нерискованным. Существовал также древний предрассудок, будто женщины могут выращивать растения лучше мужчин; ведь они были матерями. Сегодня во многих отсталых племенах мужчины готовят мясо, а женщины овощи, и когда первобытные племена Австралии кочуют, женщины никогда не охотятся на зверя, а мужчины не наклоняются, чтобы откопать корень.
\vs p084 3:7 Женщине всегда приходилось работать, и, по крайней мере до наступления новейших времен, она была настоящим тружеником. Мужчина же, как правило, выбирал занятие полегче, и это неравенство существовало на протяжении всей истории человечества. Женщина всегда носила тяжести и присматривала за детьми, освобождая, таким образом, руки мужчин для борьбы и охоты.
\vs p084 3:8 Раскрепощение женщины началось, когда мужчина согласился обрабатывать землю, согласился делать то, что до этого считалось женской работой. Огромный шаг вперед был сделан тогда, когда пленных мужчин перестали убивать, а, сделав их рабами, заставляли работать на земле. Это привело к освобождению женщины, так что она смогла уделять больше времени домашней работе и воспитанию детей.
\vs p084 3:9 Кормление детей молоком животных привело к тому, что младенцев стали раньше отнимать от груди и, следовательно, матери, у которых вследствие этого сокращался период временного бесплодия, стали чаще рожать, а употребление коровьего и козьего молока сильно уменьшило детскую смертность. До скотоводческого этапа развития общества, матери, как правило, кормили своих детей грудью до четырех или пяти лет.
\vs p084 3:10 Постепенное сокращение числа первобытных войн привело к значительному уменьшению неравенства в разделении труда по половому признаку. Однако женщине по\hyp{}прежнему приходилось заниматься реальной работой, тогда как мужчины занимались охраной. Ни один лагерь или поселение нельзя было оставлять без охраны ни днем, ни ночью, однако приручение собаки упростило даже эту задачу. В целом появление земледелия повысило престиж женщины и ее положение в обществе; так было по крайней мере до того времени, как мужчина сам стал земледельцем. Как только мужчина обратился к возделыванию земли, немедленно произошло огромное усовершенствование в методах земледелия, продолжавшееся усилиями следующих поколений. Охотясь и воюя, мужчина научился ценить организацию и внедрил ее принципы в производство, а позже, взяв на себя большую часть женской работы, значительно усовершенствовал и ее неупорядоченные методы труда.
\usection{4. Положение женщины в первобытном обществе}
\vs p084 4:1 Вообще говоря, в любую эпоху положение женщины служит верным критерием эволюции брака как общественного института, тогда как прогресс, достигнутый самим браком, является достаточно точным показателем, определяющим уровень развития человеческой цивилизации.
\vs p084 4:2 \pc Положение женщины в обществе всегда было парадоксально; она всегда проницательно руководила мужчиной и всегда пользовалась его более сильным половым влечением в своих собственных интересах и во благо самой себе. Умело пуская в ход свои прелести, она часто была способна властвовать над мужчиной даже тогда, когда он держал ее в состоянии презренного рабства.
\vs p084 4:3 Первобытная женщина была для мужчины не другом, возлюбленной, любовницей или партнером, но предметом собственности, слугой или рабыней, а позднее --- экономическим партнером, игрушкой и той, кто рожала ему детей. Все же правильные и удачные половые отношения всегда включали в себя элемент выбора и сотрудничества со стороны женщин, что неизменно позволяло умным женщинам оказывать значительное влияние на свое непосредственное и личное положение, вопреки своему месту в обществе как пола. Однако тот факт, что, пытаясь ослабить свою зависимость, женщины были вынуждены прибегать к хитростям, не уменьшал недоверия и подозрительности мужчин.
\vs p084 4:4 \pc Представители разных полов с огромным трудом понимали друг друга. Мужчине было трудно понять женщину; он относился к ней с невежественным недоверием, удивительно сочетающимся с боязливым благоговением, и в то же время с подозрением и презрением. Многие племенные и расовые традиции вину за все беды перекладывают на Еву, Пандору, либо какую\hyp{}нибудь другую представительницу женского пола. Эти легенды всегда искажались, чтобы представить все так, будто женщина навлекала зло на мужчину; причем все это свидетельствует о некогда повсеместном недоверии к женщине. Среди причин, приводившихся в поддержку безбрачия духовенства, главной была женская низость. И тот факт, что большинство подозреваемых в ведовстве были женщинами, отнюдь не улучшал былую репутацию этого пола.
\vs p084 4:5 Мужчины издавна считали женщин странными, даже ненормальными. Они верили даже, что у женщин нет души, и поэтому не давали им имен. В древности существовал великий страх перед первой половой близостью с женщиной; с тех пор и возник обычай, чтобы первое половое сношение с девственницей имел священник. Считалось, что опасна даже тень женщины.
\vs p084 4:6 Одно время было распространено мнение, будто роды делают женщину опасной и нечистой. И нравственные нормы у многих племен предписывали, что мать после рождения ребенка должна подвергаться длительным обрядам очищения. Как правило, в племенах, кроме тех, где было принято, чтобы муж участвовал в родах, женщин, готовящихся стать матерью, остерегались, оставляли их одних. Древние даже избегали рождения ребенка в доме. В конце концов ухаживать за матерью во время родов позволили пожилым женщинам; этот обычай и породил профессию акушерки. Чтобы помочь разрешению, во время родов говорилась и совершалась масса глупостей. Чтобы предотвратить вмешательство призраков, существовал обычай окроплять новорожденных святой водой.
\vs p084 4:7 У женщин, чьи племена не подверглись смешению, роды были сравнительно легкими и занимали лишь два\hyp{}три часа; у женщин же смешанных рас роды такими легкими бывают редко. Если женщина умирала от родов, особенно при разрешении от близнецов, считалось, что она виновна в измене с духом. Позднее более развитые племена стали считать смерть от родов волей небес; на таких матерей смотрели как на погибших во имя благородного дела.
\vs p084 4:8 Что же касается так называемой скромности женщин в одежде и демонстрации своего тела, то это следствие смертельного страха быть увиденной во время менструального периода. Быть замеченной в таком состоянии считалось тяжким грехом, нарушением табу. По обычаям древности каждая женщина с юности до конца детородного периода каждый месяц на одну полную неделю подвергалась полной изоляции от семьи и общества. Все, к чему она могла прикоснуться, все, на чем она могла сидеть или лежать, считалось «оскверненным». В течение долгого времени в обычае было жестоко избивать девушку после каждой менструации, дабы изгнать из ее тела злого духа. Однако когда женщина выходила из детородного возраста, с ней, как правило, обращались более деликатно, давали ей больше прав и привилегий. Ввиду всего этого не было ничего странного в том, что на женщин смотрели свысока. Даже греки, и те считали женщину во время менструации одной из трех причин осквернения; другими двумя были свинина и чеснок.
\vs p084 4:9 Однако какими бы глупыми ни были представления прошлого, они приносили и некоторую пользу, поскольку каждый месяц давали переутомленным работой женщинам, по крайней мере пока те были молоды, неделю для желанного отдыха и полезных размышлений. Таким образом, они могли оттачивать свой ум для общения со своими соплеменниками\hyp{}мужчинами в остальное время. Эта изоляция женщины также удерживала мужчин от злоупотреблений сексом и тем самым косвенно способствовала ограничению роста населения и усилению самообладания.
\vs p084 4:10 \pc Огромным достижением стал запрет, лишающий мужчин права убивать свою жену, когда ему заблагорассудится. Равный по значимости шаг вперед был сделан, когда женщина стала владеть своими свадебными подарками. Позднее она добилась законного права владеть, управлять собственностью и даже продавать ее, однако в течение длительного времени ей отказывалось в праве занимать должности как в церкви, так и в государстве. Вплоть до двадцатого века и даже в двадцатом веке после пришествия Христа с женщиной всегда в той или иной степени обращались как с собственностью. Еще далеко не везде в мире она добилась свободы от ограничений под властью мужчины. Даже у самых развитых народов стремление мужчины защитить женщину всегда было негласным притязанием на превосходство.
\vs p084 4:11 Однако первобытные женщины не жалели себя, как имеют обыкновение поступать их современные сестры. В конце концов, они были вполне счастливы и довольны и не решались представлять себе лучший или другой способ существования.
\usection{5. Женщина в условиях развивающихся нравов}
\vs p084 5:1 В самоувековечении женщина равна мужчине, однако в партнерстве во имя самоподдержания она трудится, занимая явно невыгодное положение, и та помеха, которую представляет собой вынужденное материнство, может быть скомпенсирована лишь просвещенными нравами развивающейся цивилизации и усиливающимся чувством справедливости, которое постепенно овладевает мужчинами.
\vs p084 5:2 С развитием общества требования к системе половых отношений у женщин поднимались на более высокий уровень, поскольку они больше страдали от последствий нарушения нравов в сфере половых отношений. Нормы же поведения мужчин в половых отношениях изменяются лишь медленно по прошествии какого\hyp{}то времени, под влиянием простого чувства справедливости, которого требует цивилизация. Природе не ведома справедливость, и она заставляет женщину терпеть муки рождения одну.
\vs p084 5:3 Современная идея равенства полов прекрасна и достойна расцветающей цивилизации, но в природе ее не существует. Когда прав сильный, мужчина господствует над женщиной; когда же более преобладают справедливость, мир и честность, женщина постепенно выходит из рабства и безвестности. В любой нации и в любую эпоху общественное положение женщины изменялось обратно пропорционально уровню воинственности.
\vs p084 5:4 Однако мужчина отнюдь не сознательно или намеренно захватил права женщины, а затем постепенно и неохотно возвращал их ей; все это было бессознательным и незапланированным эпизодом эволюции общества. Когда для женщины время пользоваться дополнительными правами действительно пришло, она их получила и притом совершенно независимо от сознательной позиции мужчины. Нравы медленно, но верно изменяются так, что обуславливают те социальные перестройки, которые являются частью непрекращающейся эволюции цивилизации. Совершенствовавшиеся нравственные нормы постепенно обеспечивали все лучшее и лучшее отношение к женщинам; те же племена, которые упорствовали в жестоком обращении с ними, не выжили.
\vs p084 5:5 \pc У адамитов и нодитов женщина получила широкое признание, и те народности, что подверглись влиянию мигрировавших андитов, были склонны подпадать под влияние эдемских учений о месте женщины в обществе.
\vs p084 5:6 Древние китайцы и греки обращались с женщинами лучше, чем большинство окружавших их народов, в то время как евреи относились к женщинам с крайним недоверием. На Западе, где доктрины Павла, ставшие частью христианского учения, пользовались большим влиянием, женщине было очень трудно добиться улучшения своего положения в обществе, хотя христианство и усовершенствовало нравы, наложив на мужчину в сфере половых отношений более строгие обязательства. Угнетение, которому подвергается женщина в магометанстве, делает ее положение почти безнадежным, а в странах, исповедующих некоторые другие восточные религии, ей живется и того хуже.
\vs p084 5:7 \pc По\hyp{}настоящему эмансипировала женщину наука, а не религия; современная фабрика во многом освободила ее от семейного заточения. В новой системе существования физические возможности мужчин перестали быть жизненно важной основой; наука изменила условия жизни настолько, что мужская сила хотя все еще и превосходит силу женщины, но уже не в такой степени.
\vs p084 5:8 Эти изменения вели к освобождению женщины от домашнего рабства и произвели в ее положении такую перемену, что теперь она практически обладает той же степенью личной свободы и такой же возможностью самостоятельно решать, как строить свою половую жизнь, что и мужчина. Когда\hyp{}то ценность женщины заключалась в ее способности производить пищу, технические достижения и изобилие позволили ей создать новый мир, в котором она может действовать, --- это сфера красоты и обаяния. Таким образом, промышленность выиграла свое неосознанное и неумышленное сражение за социальное и экономическое освобождение женщин. И опять\hyp{}таки эволюции удалось сделать то, чего не смогло совершить даже откровение.
\vs p084 5:9 \pc Реакция просвещенных народов на несправедливые нравы, определяющие положение женщины в обществе, в своих крайностях была действительно подобна поведению маятника. У промышленно развитых рас женщина получила почти все права и освобождена от многих обязанностей, например, таких как военная служба. Каждое достижение, облегчающее борьбу за существование, способствовало освобождению женщины, и она извлекала прямую выгоду из каждого продвижения к моногамии. Ведь слабейший всегда выигрывает намного больше от каждой перестройки нравов в ходе постепенной эволюции общества.
\vs p084 5:10 В идеалах брака между двумя людьми женщина окончательно добилась признания, уважения своего достоинства, независимости, равенства и права на образование; но окажется ли она достойной всех этих новых и беспрецедентных завоеваний? Ответит ли современная женщина на это великое достижение социального освобождения праздностью, безразличием, бесплодием и неверностью? В двадцатом веке долгое существование женщины в мире подвергается решающему испытанию!
\vs p084 5:11 В воспроизводстве расы женщина является равным партнером мужчины и, следовательно, в развитии расы играет такую же важную роль; поэтому эволюция во все большей степени и приводила к реализации женских прав. Однако права женщины --- отнюдь не права мужчины. Женщина не может преуспеть, пользуясь правами мужчин, так же, как и мужчина не может достичь процветания, пользуясь правами женщины.
\vs p084 5:12 Каждый пол имеет свои четкие границы существования, а также свои права внутри этих границ. Если женщина стремится пользоваться буквально всеми правами мужчины, то рано или поздно безжалостное и лишенное чувств соперничество обязательно заменит то рыцарское и особое уважение, которым сейчас пользуются многие женщины и которое они так недавно отвоевали у мужчин.
\vs p084 5:13 Цивилизация никогда не сможет уничтожить пропасть, которая лежит между поведением противоположных полов. Нравственные нормы меняются от века к веку, но инстинкт --- никогда. Врожденная материнская любовь никогда не допустит, чтобы эмансипированная женщина стала серьезным соперником мужчины на производстве. Каждый пол навсегда сохранит свое превосходство в своей области, причем области эти определяются и биологическим различием, и несходством мышления.
\vs p084 5:14 У каждого пола всегда будет свое поприще, хотя эти поприща то и дело пересекаются. И лишь в социальном плане мужчины и женщины будут состязаться на равных основаниях.
\usection{6. Партнерство мужчины и женщины}
\vs p084 6:1 Стремление к воспроизводству неизменно сближает мужчин и женщин для самоувековечения, но само по себе не гарантирует того, что они останутся вместе во взаимном сотрудничестве --- в создании семьи.
\vs p084 6:2 Каждый успешный человеческий институт заключает в себе антагонизмы личных интересов, антагонизмы, подвергшиеся согласованию до состояния практической рабочей гармонии, и создание семьи --- не исключение. Брак, основа строительства семьи, является высшим проявлением того антагонистического сотрудничества, которое столь характерно для взаимодействия природы и общества. Конфликт неизбежен. Половые отношения --- явление врожденное; они естественны. Брак же --- явление не биологическое, а социальное. Страсть способствует сближению мужчины и женщины, но удерживает их вместе более слабый родительский инстинкт и нравы общества.
\vs p084 6:3 \pc Мужчина и женщина, с практической точки зрения, являются двумя четко выраженными разновидностями одного и того же вида, живущими в тесном и интимном союзе. Их восприятие и вообще отношение к жизни, по существу, различны, и они не способны до конца и по\hyp{}настоящему понимать друг друга. Полное взаимопонимание между противоположными полами недостижимо.
\vs p084 6:4 Женщины, видимо, обладают большей интуицией, нежели мужчины, но зато кажутся в чем\hyp{}то менее логичными. Женщина, однако, всегда была носителем нравственных норм и духовным лидером человечества. Рука, качающая колыбель, по\hyp{}прежнему дружит с судьбой.
\vs p084 6:5 \pc Различия в природе, реакции, восприятии и мышлении, существующие между мужчинами и женщинами, не должны вызывать беспокойства, их следует рассматривать как крайне благоприятные для человечества, и в индивидуальном, и в коллективном плане. Многие чины творений вселенной созданы в двойных фазах проявления личности. У смертных, Материальных Сынов и мидсонитеров, эта разница описывается как разница между мужским и женским; у серафимов, херувимов и Моронтийных Компаньонов она обозначается как разница между положительным, или активным, и отрицательным, или пассивным. Такие двойные союзы умножают разносторонность и преодолевают врожденные недостатки, как это делают определенные триединые союзы в системе Рай\hyp{}Хавона.
\vs p084 6:6 Мужчины и женщины в своей моронтийной и духовной жизни нуждаются друг в друге так же, как и в жизни смертной. Различия между мужским и женским восприятием сохраняются даже после первой жизни и на всем протяжении восхождений в локальной и сверхвселенной. И даже в Хавоне пилигримы, которые когда\hyp{}то были мужчинами и женщинами, будут по\hyp{}прежнему помогать друг другу в своем восхождении к Раю. Никогда, даже в Отряде Финальности, творение не будет подвергаться метаморфозе настолько, что произойдет стирание личностных признаков, которые люди называют мужскими и женскими; эти две основные разновидности рода человеческого будут всегда продолжать интересовать, стимулировать, воодушевлять друг друга и друг другу помогать; они будут всегда взаимно зависеть от сотрудничества в решении сложных вселенских проблем и в преодолении множества космических трудностей.
\vs p084 6:7 \pc Несмотря на то, что представители противоположных полов не могут надеяться на полное взаимопонимание, они прекрасно дополняют друг друга, и хотя сотрудничество в личном плане часто бывает в каком\hyp{}то смысле антагонистическим, оно способно сохранять и воспроизводить общество. Брак --- это институт, предназначенный для того, чтобы гармонизировать половые различия и одновременно осуществлять продолжение цивилизации и обеспечивать воспроизводство расы.
\vs p084 6:8 Брак --- это мать всех человеческих институтов, ибо он прямо ведет к созданию и сохранению семьи, которая является структурной основой общества. Семья неразрывно связана с механизмом самоподдержания; в условиях нравов цивилизации она является единственной надеждой увековечения расы и одновременно наиболее эффективно обеспечивает определенные чрезвычайно удачные формы самоудовлетворения. Семья --- это величайшее чисто человеческое достижение, объединяющее эволюцию биологических отношений мужчины и женщины с социальными отношениями мужа и жены.
\usection{7. Идеалы семейной жизни}
\vs p084 7:1 Половые отношения инстинктивны, дети --- их естественное последствие; семья же, таким образом, возникает автоматически. Каковы семьи расы или нации, таково и ее общество. Если семьи хороши, то хорошо и общество. Великая культурная стабильность еврейского и китайского народов кроется в силе их семейных групп.
\vs p084 7:2 Женский инстинкт любви к детям и заботы о них сделал ее стороной, заинтересованной в установлении брака и первобытной семейной жизни. Мужчина же был вынужден лишь участвовать в создании семьи, побуждаемый к этому более поздними нравами и обычаями общества; к учреждению института брака и семьи он испытывал малый интерес, поскольку половой акт не имел для него биологических последствий.
\vs p084 7:3 Половая связь естественна, брак же --- социален и всегда регулировался нравами. Нравы (религиозные, моральные и этические) в сочетании с собственностью, гордостью и благородством укрепляют институты брака и семьи. Как только нравы становятся неустойчивыми, возникает и неустойчивость института семьи и брака. В настоящее время брак переходит из стадии имущественных отношений в эру личных. Раньше мужчина защищал женщину, потому что она была его собственностью; по той же причине подчинялась и она ему. Несмотря на эти ее особенности, эта система обеспечивала стабильность. Сейчас женщину собственностью больше не считают, и возникают нравы, предназначенные стабилизировать институт семьи и брака:
\vs p084 7:4 \ublistelem{1.}\bibnobreakspace Новая роль религии --- учение о том, что родительский опыт --- это главное, идея рождения граждан космоса, расширение понимания привилегии рождения --- привилегии давать Отцу сыновей.
\vs p084 7:5 \ublistelem{2.}\bibnobreakspace Новая роль науки --- рождение детей все больше и больше становится добровольным, подверженным контролю человека. В древности недостаток понимания приводил к появлению детей при отсутствии всякого на то желания.
\vs p084 7:6 \ublistelem{3.}\bibnobreakspace Новая функция искушения удовольствием --- вводит в выживание расы новый фактор; древний человек предавал нежелательных детей смерти; современные люди отказываются их рожать.
\vs p084 7:7 \ublistelem{4.}\bibnobreakspace Усиление родительского инстинкта. Теперь каждое поколение склонно устранить из репродуктивного потока расы тех индивидуумов, у которых родительский инстинкт недостаточно силен для того, чтобы гарантировать рождение детей, будущих родителей следующего поколения.
\vs p084 7:8 \pc Однако семья как институт, партнерские отношения между одним мужчиной и одной женщиной, возник, если говорить точнее, во времена Даламатии, около полумиллиона лет тому назад, причем от моногамных обычаев Андона и его непосредственных потомков отказались задолго до этого. Семейная жизнь, однако, до нодитов и более поздних адамитов не была чем\hyp{}то таким, чем бы можно было гордиться. Адам и Ева оказали несомненное влияние на все человечество; впервые в истории мира мужчину и женщину увидели работающими вместе, бок о бок в саду. Эдемский идеал, целая семья садовников, была на Урантии новой идеей.
\vs p084 7:9 Древняя семья представляла собой связанную родственными отношениями рабочую группу, включавшую в себя и рабов, причем все жили под одним кровом. Брак и семейная жизнь не всегда были одним и тем же, но вследствие необходимости были тесно связаны. Женщина всегда хотела иметь свою семью и в конце концов своего добилась.
\vs p084 7:10 \pc Любовь к потомству распространена почти повсеместно и несомненно представляет собой ценность, от которой зависит выживание. Древние всегда жертвовали интересами матери ради благополучия ребенка; мать\hyp{}эскимоска даже теперь лижет своего младенца вместо мытья. Однако первобытные матери кормили своих детей и заботились о них, пока те были совсем юны, и, подобно животным, бросали их, как только дети вырастали. Длительные и неразрывные человеческие связи никогда не основывались только на биологическом чувстве привязанности. Животные любят своих детенышей; человек же --- цивилизованный человек --- любит детей своих детей. Чем выше цивилизация, тем большую радость находят родители в развитии и успехах своих детей; отсюда и произошло новое и более высокое понимание \bibemph{фамильной} гордости.
\vs p084 7:11 Большие семьи у древних народов не всегда были построены на любви. Иметь много детей хотели, потому что:
\vs p084 7:12 \ublistelem{1.}\bibnobreakspace Они были ценны как работники.
\vs p084 7:13 \ublistelem{2.}\bibnobreakspace Они были поддержкой в старости.
\vs p084 7:14 \ublistelem{3.}\bibnobreakspace Дочерей можно было продать.
\vs p084 7:15 \ublistelem{4.}\bibnobreakspace Фамильная гордость требовала продолжения рода.
\vs p084 7:16 \ublistelem{5.}\bibnobreakspace Сыновья помогали защищаться и обороняться.
\vs p084 7:17 \ublistelem{6.}\bibnobreakspace Боязнь призраков порождала страх одиночества.
\vs p084 7:18 \ublistelem{7.}\bibnobreakspace Некоторые религии требовали потомства.
\vs p084 7:19 \pc Почитатели предков считают отсутствие сыновей высшим несчастьем на все времена и в вечности. Более всего они желают иметь сыновей, чтобы те совершали посмертные поминальные пиршества и приносили жертвы, необходимые для прохождения призрака по стране духов.
\vs p084 7:20 У древних первобытных людей дисциплинирование детей начиналось очень рано; и ребенок рано понимал, что непослушание так же, как для животных, приводит к неудаче или даже смерти. В непослушании современных детей во многом виновна защита ребенка от естественных последствий неразумного поведения, которую дает цивилизация.
\vs p084 7:21 Дети эскимосов успешно развиваются при столь незначительном дисциплинарном воздействии и воспитательных мерах просто потому, что они, по природе своей, покорные маленькие зверьки; дети у красных и желтых людей послушны почти в такой же степени. Однако у рас, обладающих наследственными качествами андитов, дети не такие спокойные; эти отличающиеся большим воображением и озорные молодые люди требуют большего воспитания и дисциплины. Современные проблемы детского воспитания становятся намного сложнее из\hyp{}за:
\vs p084 7:22 \ublistelem{1.}\bibnobreakspace Большой степени смешения рас.
\vs p084 7:23 \ublistelem{2.}\bibnobreakspace Искусственного и поверхностного образования.
\vs p084 7:24 \ublistelem{3.}\bibnobreakspace Неспособности ребенка овладеть культурой путем подражания родителям, поскольку те значительную часть времени в семье отсутствуют.
\vs p084 7:25 \pc Идеи прошлого о семейной дисциплине были биологическими и обусловливались осознанием того, что родители были создателями младенца. Совершенствующиеся идеалы семейной жизни ведут к представлению, согласно которому рождение ребенка в мир вместо предоставления определенных родительских прав, влечет за собой верховную в человеческой жизни ответственность.
\vs p084 7:26 Цивилизация считает родителей принимающими на себя все обязанности, а детей имеющими все права. Уважение ребенка к своим родителям возникает не от знания обязательств, вытекающих из самого факта рождения детей родителями, а естественно вырастает как результат заботы, воспитания и любви, с нежностью проявляемых, чтобы помочь ребенку одержать победу в сражении, которое представляет собой жизнь. Истинный родитель занят непрерывным служением, которое умный ребенок признает и ценит.
\vs p084 7:27 \pc В современную индустриальную и урбанистическую эру институт брака развивается в соответствии с новыми экономическими тенденциями. Семейная жизнь становится все более и более дорогой, а дети, которые обычно были активом, теперь стали экономическим пассивом. Однако безопасность самой цивилизации по\hyp{}прежнему зиждется на возрастающей готовности одного поколения делать вложения в благополучие следующего и будущих поколений. И любая попытка переложить родительскую ответственность на государство или церковь для благополучия и развития цивилизации окажется самоубийственной.
\vs p084 7:28 \pc Брак наряду с детьми и последующей семейной жизнью стимулирует высочайшие потенциалы человеческой природы и одновременно предоставляет идеальный путь для выражения этих стимулированных атрибутов смертной личности. Семья обеспечивает биологическое увековечение человеческого вида. Семья --- вот естественная социальная арена, на которой подрастающие дети могут усвоить этику кровного братства. Семья есть фундаментальная единица братства, в котором родители и дети постигают уроки терпения, альтруизма, терпимости и выдержки, столь необходимые для достижения братских отношений между всеми людьми.
\vs p084 7:29 Человеческое общество станет намного лучше, если цивилизованные расы будут более широко вводить у себя обычаи семейных советов, которые были приняты у андитов. У андитов не было патриархальной или автократической формы семейного управления. Они жили по\hyp{}братски и по\hyp{}товарищески, свободно и откровенно обсуждая каждое предложение и правило семейной жизни. В управлении семьей они были идеальными собратьями. В идеальной семье сыновняя и родительская любовь усиливаются братской привязанностью.
\vs p084 7:30 Семейная жизнь --- вот прародитель истинной морали и предшественник сознания верности долгу. Предопределенные связи семейной жизни стабилизируют личность и через принуждение необходимостью приспосабливаться к другим, самым разным личностям, стимулируют ее рост. Но более того, истинная семья --- хорошая семья --- в еще большей степени открывает родителям, рождающим детей, отношение Творца к своим детям, а такие истинные родители являют своим детям первое из долгого ряда восходящих открытий любви Райского родителя ко всем детям во вселенной.
\usection{8. Опасности самоудовлетворения}
\vs p084 8:1 Великой опасностью для семейной жизни является угрожающе растущая волна самоудовлетворения, современная мания удовольствия. Первичное побуждение к браку, как правило, было экономическим, а половое влечение --- вторичным. Брак, основанный на самоподдержании, вел к самоувековечению и попутно давал одну из самых полезных форм самоудовлетворения. Это единственный институт человеческого общества, который объемлет собой все три великих побуждения к жизни.
\vs p084 8:2 Первоначально основным институтом самоподдержания была собственность, тогда как брак функционировал как единственный в своем роде институт самоувековечения. Хотя удовольствие от пищи, игра, юмор и периодические занятия сексом являлись средствами самоудовлетворения, фактом остается и то, что развивавшиеся нравы не смогли создать отдельного института самоудовлетворения. И в связи с тем, что не удалось выработать особые методы приятного удовольствия все человеческие институты пронизаны погоней за наслаждением. Накопление собственности становится инструментом, усиливающим все формы самоудовлетворения, а брак часто считают лишь средством получения наслаждения. И это чрезмерное увлечение, эта широко распространившаяся мания наслаждения, теперь представляет собой величайшую угрозу, когда\hyp{}либо нависавшую над общественно\hyp{}эволюционным институтом семейной жизни, над семьей.
\vs p084 8:3 Фиолетовая раса внесла в опыт человечества новое и лишь недостаточно хорошо реализованное свойство --- игровой инстинкт, сочетаемый с чувством юмора. У сангиков и андонитов он был развит до некоторой степени, однако адамические народы, возвысив, превратили эту примитивную наклонность в \bibemph{потенциал удовольствия,} в новую и благородную форму самоудовлетворения. Основным типом самоудовлетворения, помимо утоления голода, является удовлетворение полового чувства, причем в результате смешения сангиков и андитов эта форма чувственного удовольствия чрезвычайно усилилась.
\vs p084 8:4 В сочетании беспокойности, любопытства, стремления к приключениям и несдержанности в удовольствиях, характерных для постадамических рас, кроется настоящая опасность. Жажду души нельзя утолить физическими удовольствиями; от неразумной погони за удовольствием любовь к семье и детям не усиливается. Хоть вы и истощаете запасы искусства, цвета, звука, ритма, музыки и нарядов, вы не можете надеяться на то, что, благодаря этому, возвысите душу или взлелеете дух. Тщеславие и мода не могут служить созданию семьи и воспитанию детей; гордость и соперничество не способны усилить качества будущего поколения, обеспечивающие их спасение.
\vs p084 8:5 Все совершенствующиеся небесные существа наслаждаются покоем и служением руководителей восстановления. Хороши все попытки получить полезное развлечение и заняться бодрящей игрой; освежающий сон, отдых, развлечения и все игры, предотвращающие монотонную скуку, --- полезны. Соревнования, сочинение рассказов и даже вкус хорошей пищи могут служить формами самоудовлетворения. (Пользуясь солью для придания пище вкуса, остановитесь и подумайте о том, что в течение почти миллиона лет человек мог получить соль, лишь окуная свою пищу в пепел.)
\vs p084 8:6 \pc Пусть человек наслаждается собой; пусть человечество получает удовольствие тысячью и одним способом; пусть эволюционирующее человечество испробует все виды законного самоудовлетворения, плоды долгой ведущей вверх биологической борьбы. Человек вполне заслужил некоторые из своих современных радостей и удовольствий. Но не спускайте глаз с цели судьбы! Удовольствия поистине самоубийственны, если они успешно разрушают собственность, ставшую институтом самоподдержания; и самоудовлетворение будет действительно иметь роковые последствия, если оно приведет к падению брака, упадку семейной жизни и разрушению семьи --- высшего эволюционного достижения человека и единственной надежды цивилизации на выживание.
\vsetoff
\vs p084 8:7 [Представлено Главой Серафимов, находящимся на Урантии.]
