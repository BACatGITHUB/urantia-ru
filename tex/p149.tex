\upaper{149}{Второе путешествие с проповедями}
\vs p149 0:1 Второе путешествие с проповедями по Галилее началось в воскресенье 3 октября 28 года н.э., продолжалось почти три месяца и закончилось 30 декабря. В этом предприятии участвовали Иисус и его двенадцать апостолов, которым помогали недавно набранный отряд из 117 евангелистов и многие другие заинтересовавшиеся люди. В этом путешествии они посетили Гадару, Птолемииду, Яффий, Дабаритту, Мегиддо, Изреель, Скифополь, Тарихею, Гиппос, Гамалу, Вифсаиду\hyp{}Юлию и многие другие города и селения.
\vs p149 0:2 В это воскресное утро перед тем, как отправиться, Андрей и Петр попросили Иисуса дать новым евангелистам последнее напутствие, но Учитель отказался, сказав, что не будет делать то, с чем вполне могут справиться и другие. После подобающего обсуждения решено было, что напутствие даст Иаков Заведеев. Когда Иаков кончил говорить, Иисус сказал евангелистам: <<Теперь идите и делайте порученное вам дело, и позднее, когда вы проявите свои умение и верность, я дам вам посвящение проповедовать евангелие царства>>.
\vs p149 0:3 В этом путешествии Иисуса сопровождали только Иаков и Иоанн. Петр же и каждый из остальных апостолов взяли с собой почти по дюжине евангелистов и поддерживали с ними тесный контакт, пока те занимались своей работой, проповедуя и уча. Как только верующие были готовы войти в царство, апостолы совершали крещение. Иисус и его два попутчика на протяжении этих трех месяцев много путешествовали и, чтобы наблюдать за работой евангелистов и поддержать их усилия, направленные на установление царства, часто посещали за один день два города. Все второе путешествие с проповедями, главным образом, было попыткой предоставить этому отряду из 117 евангелистов возможность обрести практический опыт.
\vs p149 0:4 \P\ На протяжении всего этого периода, а также впоследствии вплоть до времени последнего отправления Иисуса и двенадцати апостолов в Иерусалим в доме отца Давида Заведеева в Вифсаиде находился постоянный центр, координирующий работу в царстве. Это был информационный центр для дела Иисуса на земле и отправительная станция вестников (которыми руководил Давид), поддерживавших связи между тружениками царства в различных частях Палестины и в соседних краях. Все это он делал по своей собственной инициативе, но с одобрения Андрея. В этом информационном отряде быстро расширявшегося и разраставшегося дела царства Давид использовал от сорока до пятидесяти вестников. Посвятив себя этому делу, Давид отчасти содержал себя, отдавая какую\hyp{}то часть своего времени своему старому занятию --- рыболовству.
\usection{1.\bibnobreakspace Широко распространившаяся слава Иисуса}
\vs p149 1:1 Ко времени расформирования лагеря в Вифсаиде слава Иисуса, в особенности как целителя, распространилась повсюду в Палестине, по всей Сирии и в окружающих странах. В течение нескольких недель после того, как они оставили Вифсаиду, сюда продолжали прибывать больные, которые, не найдя Учителя и узнав от Давида, где он находится, шли его искать. В этом путешествии Иисус умышленно не совершал так называемых чудес исцеления. Тем не менее, ко множеству больных вернулись здоровье и счастье благодаря восстановительной мощи сильной веры, которая побуждала их искать исцеления.
\vs p149 1:2 Приблизительно во время этой миссии начали --- и на протяжении всей оставшейся жизни Иисуса на земле продолжали --- происходить необычные и необъяснимые явления исцеления. В ходе этого трехмесячного путешествия более ста мужчин, женщин и детей из Иудеи, Идумеи, Галилеи, Сирии, Тира и Сидона, а также из земель за Иорданом получили эти ненамеренных исцеления отИисуса и, вернувшись к своим домам, способствовали распространению его славы. Поступали же они так, несмотря на то, что всякий раз, замечая один из таких случаев спонтанного исцеления, Иисус прямо приказывал исцелившемуся <<никому не говорить об этом>>.
\vs p149 1:3 \P\ Нам так и не было открыто, что именно происходило в этих случаях спонтанного или бессознательного исцеления. Учитель никогда не объяснял своим апостолам, как совершались эти исцеления, и лишь в нескольких случаях просто сказал: <<Я чувствую, что из меня вышла сила>>. Один же раз, когда его коснулся больной ребенок, он заметил: <<Я чувствую, что из меня вышла жизнь>>.
\vs p149 1:4 При отсутствии прямого указания Учителя относительно природы этих случаев спонтанного исцеления взять на себя объяснение того, как они осуществлялись, с нашей стороны было бы самонадеянно, однако записать наше мнение о всех подобных явлениях исцеления вполне возможно. Мы считаем, что многие из этих кажущихся чудес исцеления, происходивших в ходе земного служения Иисуса, были результатом сосуществования следующих трех мощных, могущественных и взаимосвязанных влияний:
\vs p149 1:5 \ublistelem{1.}\bibnobreakspace Присутствие сильной, преобладающей над всем и живой веры в сердце человеческого существа, настойчиво искавшего исцеления, в сочетании с тем, что подобное исцеление было желаемым из\hyp{}за духовной пользы, которую оно приносило, а не ради просто физического выздоровления.
\vs p149 1:6 \P\ \ublistelem{2.}\bibnobreakspace Наличие, наряду с такой человеческой верой, огромного сочувствия и сострадания со стороны воплотившегося и преисполненного милосердия Божиего Сына\hyp{}Творца, который действительно внутренне обладал почти неограниченными и вневременными созидательными целительными силами и возможностями.
\vs p149 1:7 \P\ \ublistelem{3.}\bibnobreakspace Наряду с верой создания и жизнью Творца следует также отметить, что сей Богочеловек был олицетворенным выражением воли Отца. Если соприкосновение человеческой нужды и божественной силы, направленной на ее удовлетворение, не противоречило воле Отца, то возникало их слияние и исцеление происходило бессознательно для человека Иисуса, но немедленно распознавалось его божественной природой. Следовательно, объяснение многих из этих случаев исцеления необходимо искать в великом и давно известном нам законе, а именно: чего желает Сын\hyp{}Творец и вечный Отец велит --- СОВЕРШАЕТСЯ.
\vs p149 1:8 \P\ Поэтому наше мнение таково: в личном присутствии Иисуса определенные формы сильной человеческой веры были буквально и истинно \bibemph{непреодолимы} в проявлениях исцеления определенными творческими силами и личностями вселенной, которые в то время были столь тесно связаны с Сыном Человеческим. Поэтому является установленным фактом, что Иисус действительно часто в своем присутствии позволял людям исцелять самих себя их сильной личной верой.
\vs p149 1:9 Многие другие искали исцеления исключительно в эгоистических целях. Так, богатая вдова из Тира пришла со своей свитой, ища исцеления от своих многочисленных немощей; следуя за Иисусом по Галилее, она продолжала предлагать все больше и больше денег, как будто сила Бога была чем\hyp{}то таким, что ее приобрести мог тот, кто даст наибольшую цену. Она вовсе не интересовалась евангелием царства; она искала лишь исцеления от своих телесных недугов.
\usection{2.\bibnobreakspace Отношение народа}
\vs p149 2:1 Иисус понимал мышление чеовека. Он знал, что происходит в сердце человека, и если бы его учения остались в том виде, в каком он представил их сам, когда единственным комментарием к ним было вдохновенное толкование, данное его жизнью, тогда бы все народы и все религии мира быстро приняли евангелие царства. Попытки первых последователей Иисуса, действующих из лучших побуждений, сформулировать его учения заново, так, чтобы сделать их более приемлемыми для определенных наций, рас и религий, привели лишь к тому, что сделали подобные учения менее приемлемыми для всех остальных наций, рас и религий.
\vs p149 2:2 Пытаясь привлечь к учению Иисуса внимание определенных групп того времени, Апостол Павел написал множество писем с наставлениями и увещаниями. Другие учителя евангелия Иисуса делали то же самое, однако ни один из них не сознавал, что некоторые из этих писаний впоследствии будут собраны воедино теми, кто будет выдавать их за воплощение учений Иисуса. Таким образом, хотя в так называемом христианстве содержится гораздо больше евангелия Учителя, нежели в любой другой религии, в нем есть также много того, чему Иисус не учил. Помимо включения в раннее христианство многих учений из персидских мистерий и многого из философии греков, были сделаны две великие ошибки:
\vs p149 2:3 \ublistelem{1.}\bibnobreakspace Попытка связать учение евангелия непосредственно с еврейской теологией, нашедшая отражение в христианских доктринах искупления --- в учении о том, будто Иисус был принесенным в жертву Сыном, который ублаготворит неумолимое правосудие Отца и умиротворит божественный гнев. Эти учения произошли от достойной похвалы попытки сделать евангелие царства более приемлемым для неверующих евреев. Хотя что касается обращения евреев, эти попытки оказались безуспешными, они смутили и отвратили множество искренных душ во всех последующих поколениях.
\vs p149 2:4 \P\ \ublistelem{2.}\bibnobreakspace Вторая грубая ошибка первых последователей Учителя, ошибка, которую упорно повторяли все последующие поколения, заключалась в построении христианского учения исключительно вокруг \bibemph{личности} Иисуса. Это чрезмерное подчеркивание личности Иисуса в теологии христианства затуманило его учения, и все это еще больше затруднило принятие учений Иисуса евреями, магометанами, индусами и приверженцами других восточных религий. Мы не станем умалять значения Иисуса как личности в религии, которая носит его имя, но мы не позволим подобному соображению затмить его вдохновенную жизнь или вытеснить его спасительное послание об отцовстве Бога и братстве людей.
\vs p149 2:5 \P\ Учителя религии Иисуса должны подходить к другим религиям, признавая те истины, которые являются для них общими (многие из них прямо или косвенно заимствованы из послания Иисуса), и воздерживаясь от сильного подчеркивания различий.
\vs p149 2:6 \P\ Хотя конкретно в то время слава Иисуса главным образом происходила из его известности как целителя, это еще не означает, что так было и впоследствии. С течением времени к нему все больше и больше обращались за духовной помощью. Однако прямо и непосредственно простые люди молили именно об исцелении тела. Иисуса все больше и больше искали жертвы нравственного порабощения и душевного расстройства, и он неизменно указывал им пути избавления. Отцы искали его совета относительно того, как управлять своими сыновьями, а матери приходили за помощью в руководстве своими дочерьми. Те, кто пребывал во тьме, приходили к нему, и он открывал им свет жизни. Его ухо было всегда чутко к скорбям рода человеческого, и он всегда помогал тем, кто искал его служения.
\vs p149 2:7 Когда на земле был сам Творец, воплощенный в подобии смертной плоти, неизбежно происходили некоторые необычные явления. Однако вы никогда не должны подходить к Иисусу с позиции этих так называемых чудесных явлений. Научитесь подходить к чуду через Иисуса, но не делайте ошибку, подходя к Иисусу через чудо. Эта наказ необходим, несмотря на то, что Иисус из Назарета был единственным основателем религии, который совершал внематериальные деяния на земле.
\vs p149 2:8 \P\ Самой поразительной и самой революционной особенностью земной миссии Михаила было его отношение к женщине. В те времена и в том поколении, когда человеку, находящемуся в общественном месте, не дозволялось приветствовать даже свою собственную жену, Иисус решился взять в третье путешествие по Галилее женщин в качестве учителей евангелия. И притом имел непревзойденную смелость делать это вопреки учения раввинов, объявлявшего, что <<лучше словам закона быть сожженным в огне, чем быть произнесенным женщиной>>.
\vs p149 2:9 На протяжении жизни одного поколения Иисус возвысил женщину, избавив ее от неуважительного забвения и веками продолжавшегося рабского изнурительного труда. И постыдной чертой религии, осмелившейся взять имя Иисуса, является то, что ей не хватило нравственной смелости в дальнейшем следовать этому благородному примеру в ее отношении к женщине.
\vs p149 2:10 \P\ Когда Иисус общался с людьми, они видели, что он совершенно свободен от суеверий того времени. Он был свободен от религиозных предрассудков и никогда не был нетерпим. В его сердце не было ничего похожего на социальный антагонизм. Соглашаясь с хорошим в религии своих отцов, он без колебания отказывался следовать созданным человеком традициям суеверия и рабства. Он осмеливался учить, что природные катастрофы, несчастные случаи, происходящие во времени, и другие бедственные происшествия не являются карой божественных осуждений либо таинственными диспенсациями Провидения. Он осуждал рабскую приверженность бессмысленным обрядам и разоблачал ложность материалистического поклонения. Он смело провозглашал духовную свободу человека и отваживался учить, что смертные во плоти --- действительно и воистину сыновья живого Бога.
\vs p149 2:11 Иисус превзошел все учения своих предшественников, когда смело заменил чистые руки на чистое сердце как цель истинной религии. На место традиции он поставил реальность и отверг все тщеславные и лицемерные притязания. И все же этот бесстрашный Божий человек не позволял себе ни разрушитльной критики, ни проявления пренебрежения религиозными, социальными, экономическими и политическими обычаями своего времени. Он не был воинственным революционером, а был прогрессивным эволюционистом. Он занимался разрушением того, что \bibemph{было,} лишь одновременно предложив своим собратьям то высшее, что \bibemph{быть должно.}
\vs p149 2:12 \P\ Иисус получал послушание своих последователей, не требуя его. Только три человека, к которым он лично обращался, отказались принять приглашение стать его учениками. Для людей он обладал великой притягательной силой, но диктатором не был. Он внушал доверие, и никто и никогда не возмущался его приказаниями. Он обладал полной властью над своими учениками, но никто и никогда не возражал против этого. Он позволял своим последователям называть себя Учителем.
\vs p149 2:13 Учителем восхищались все, кто знал его, кроме тех, кто находился во власти глубоко укоренившихся религиозных предрассудков, либо тех, кто видел в его учениях угрозу своей политике. Людей поражали оригинальность и авторитетность его учения. Они удивлялись его терпению при общении с отсталыми и раздражающими людьми, интересовавшимися его учением. Он вселял надежду и уверенность в сердца всех, на кого распространялось его служение. Боялись же его лишь те, кто с ним не встречался, а ненавидели только те, кто считал его поборником той истины, которой суждено было уничтожить зло и заблуждения, которые они решили удержать в своих сердцах любой ценой.
\vs p149 2:14 И на друзей, и на врагов он оказывал сильное и необычайно пленительное влияние. Толпы неделями ходили за ним лишь затем, чтобы услышать его благодатные слова и увидеть его простую жизнь. Преданные Иисусу мужчины и женщины любили его почти нечеловеческой любовью. И чем больше они узнавали его, тем больше любили. Все это истинно и поныне; даже сегодня чем больше человек узнает этого Богочеловека, тем больше его любит и тем больше следует его примеру; и так будет во веки веков.
\usection{3.\bibnobreakspace Враждебность религиозных лидеров}
\vs p149 3:1 Несмотря на то, что простой народ одобрительнопринимал Иисуса и его учение, религиозные лидеры Иерусалима становились все более обеспокоенными и враждебными. Фарисеи сформулировали систематическую и догматическую теологию. Иисус же был учителем, который учил, когда предоставлялся случай, и методичным учителем не был. Иисус учил, опираясь не столько на законы, сколько исходя из жизни, с помощью притч. (Прибегая же для иллюстрации своего послания к притче, он предполагал с этой целью использовать только \bibemph{одну} особенность истории. Множество ложных идей об учениях Иисуса могут происходить от попыток аллегорически истолковать эти притчи).
\vs p149 3:2 Религиозные лидеры Иерусалима почти обезумели из\hyp{}за недавнего обращения молодого Авраама и дезертирства трех шпионов, которых окрестил Петр и которые теперь были с евангелистами в этом втором путешествии с проповедями по Галилее. Еврейских лидеров все больше ослепляли страх и предрассудки; сердца же их ожесточались от того, что они продолжали отвергать притягательные истины евангелия царства. Когда люди отказываются от воззвания к духу, пребывающему внутри них, немногое можно сделать, чтобы изменить их отношение.
\vs p149 3:3 Когда Иисус впервые встретился с евангелистами в вифсаидском лагере, то, завершая свое обращение к ним, сказал: <<Вы должны помнить, что телом и разумом--- эмоционально --- люди на все реагируют индивидуально. Единственное, что во всех людях \bibemph{одинаково,} --- это дух, пребывающий в них. Хотя божественные духи несколько отличаются по природе и масштабам своего опыта, они, тем не менее, одинаково реагируют на все духовные мольбы. Лишь через этот дух и обращаясь к нему человечество сможет достичь единства и братства>>. Однако многие из еврейских лидеров закрыли свои сердца для духовного призыва евангелия. С того самого момента они не прекращали строить планы и замыслы, направленные на уничтожение Учителя. Они были убеждены, что Иисус должен быть схвачен, осужден и казнен как религиозный преступник, нарушитель главных учений еврейского священного закона.
\usection{4.\bibnobreakspace Ход путешествия с проповедями}
\vs p149 4:1 В этом путешествии с проповедями Иисус очень мало занимался публичной работой, однако он провел множество вечерних занятий с верующими в большинстве городов и селений, где ему довелось побывать с Иаковом и Иоанном. Во время одного из этих вечерних занятий один из младших по возрасту евангелистов задал Иисусу вопрос о гневе, и Учитель в ответ на него среди прочего сказал:
\vs p149 4:2 \P\ <<Гнев --- это материальное проявление, которое в общем представляет собой меру неспособности духовного начала превалировать над совокупностью интеллектуального и физического начал. Гнев свидетельствует о том, что вам недостает терпимой братской любви и самоуважения и самоконтроля. Гнев истощает здоровье, унижает ум и мешает духовному учителю души человека. Разве не читали вы в Писании, что >>глупца убивает гневливость<< и что >>человек раздирает себя в гневе своем<<? Что >>у терпеливого человека много разума, а раздражительный выказывает глупость<<? Вы все знаете, что >>кроткий отвращает гнев<< и как >>оскорбительное слово возбуждает ярость<<. >>Благоразумие отвращает гнев<<, а >>человек, не владеющий собой, подобен беззащитному городу без стен<<. >>Жесток гнев и неукротима ярость<<. >>Люди гневливые заводят ссору, а вспыльчивые умножают грехи свои<<. >>Да не будет дух твой поспешен на гнев, ибо гнев гнездится в сердце глупых<<. Перед тем, как закончить свою речь, Иисус сказал: >>Пусть в сердцах ваших господствует любовь, чтобы духу, ведущему вас, было легко избавить вас от склонности давать выход тем порывам животного гнева, что несовместимы со статусом божественного сыновства<<.
\vs p149 4:3 \P\ В тот же вечер Учитель беседовал с группой о желательности обладать уравновешенным характером. Он признал, что для большинства необходимо посвятить себя овладению той или иной профессией, но порицал всякое стремление к узкой специализации, к тому, чтобы стать ограниченным и стесненным в жизненных делах. Он обратил внимание на то, что любая добродетель, если ее довести до крайности, может стать пороком. Иисус всегда проповедовал умеренность и учил постоянству --- соразмерному урегулированию жизненных проблем. Он отметил, что чрезмерное сочувствие и жалость могут выродиться в серьезную эмоциональную неустойчивость; что энтузиазм может перерасти в фанатизм. Он обсудил в качестве примера одного из их бывших сподвижников, которого воображение вовлекло в фантастические и непрактичные предприятия. В то же время он предостерег их против опасностей, свойственных тупости сверхконсервативной посредственности.
\vs p149 4:4 Затем Иисус рассуждал об опасностях, сопутствующих смелости и вере, о том, как они порой приводят легкомысленные души к безрассудству и самонадеянности. Он также показал, как благоразумие и осторожность, зашедшие слишком далеко, ведут к трусости и неудаче. Он убеждал своих слушателей стремиться к оригинальности, избегая всякого стремления к эксцентричности. Он призывал к сочувствию без сентиментальности и к благочестию, лишенному ханжества. Он учил благоговению, свободному от страха и суеверия.
\vs p149 4:5 На сподвижников Иисуса произвело впечатление не столько то, чему он учил об уравновешенном характере, сколько тот факт, что его собственная жизнь являла собой пример его учения. Он жил посреди бурь и смятений, но никогда не колебался. Его враги постоянно устраивали ему западни, но так и не сумели его поймать. Мудрые и ученые пытались подставить ему ножку, но он не спотыкался. Они старались втянуть его в спор, но его ответы всегда были поучительны, достойны и окончательны. Когда в ходе рассуждений Иисуса перебивали многочисленными вопросами, его ответы были всегда весомы и убедительны. В ответ на постоянное давление со стороны своих врагов, которые без колебания использовали любые лживые, бесчестные и неправедные нападки на него, он никогда на прибегал к недостойным приемам.
\vs p149 4:6 Хотя и верно, что многие мужчины и женщины должны заниматься каким\hyp{}либо определенным делом, чтобы иметь средства к существованию, тем не менее, в высшей степени желательно, чтобы люди обладали широкими знаниями о жизни тех цивилизаций, которые существуют на земле. Истинно образованные люди не могут мириться с незнанием того, как живут и трудятся их собратья.
\usection{5.\bibnobreakspace Урок о удовлетворенности}
\vs p149 5:1 Когда Иисус навещал группу евангелистов, трудившихся под руководством Симона Зилота, во время их вечерней беседы Симон спросил Учителя: <<Почему одни люди намного счастливее и удовлетвореннее других? Зависит ли удовлетворенность от религиозного опыта?>> В ответ на вопрос Симона Иисус помимо прочего сказал:
\vs p149 5:2 \P\ <<Симон, некоторые люди по природе своей счастливее других. Многое, очень многое зависит от готовности человека быть ведомым и руководимым духом Отца, который живет внутри него. Разве не читали вы в Писании слова мудреца: “Светильник Господень --- дух человека, испытывающий все глубины”? А также, что подобные ведомые духом смертные говорят: “Межи мои прошли по прекрасным местам; да, наследие мое приятно мне”. “Малое у праведника лучше богатства многих нечестивых”, ибо “добрый человек от себя насытится”. “Веселое сердце делает лицо веселым, и у него всегда пир. Лучше немногое при почитании Господа, нежели большое сокровище, и при нем тревога. Лучше блюдо зелени, и при нем любовь, нежели откормленный бык, и при нем ненависть. Лучше немногое с праведностью, нежели множество прибытков без честности”. “Веселое сердце благотворно, как врачевание”. “Лучше горсть со спокойствием, нежели сверхизобилие с печалью и томлением духа”.
\vs p149 5:3 Печали человека главным образом есть следствие крушения его притязаний и обиды, нанесенной его гордости. Хотя люди сами обязаны сделать все, чтобы как можно лучше прожить свою жизнь, они, таким образом, искренне стараясь сделать для этого все возможное, должны бодро принимать свой удел и проявлять изобретательность, стремясь получить максимум от того, что им было дано. Слишком много человеческих бед произрастает из\hyp{}за страха живущего в их собственном сердце. “Нечестивый бежит, когда никто не гонится за ним”. “Нечестивые --- как море взволнованное, ибо оно не может успокоиться; воды же его выбрасывают ил и грязь; нет мира нечестивым, --- говорит Бог”.
\vs p149 5:4 Поэтому не ищите ложного мира и преходящей радости, а ищите уверенности веры и ручательства божественного сыновства, что приносят спокойствие, удовлетворение и высшую радость в духе>>.
\vs p149 5:5 \P\ Едва ли Иисус воспринимал этот мир как <<юдоль слез>>. Скорее, он смотрел на него как на сферу рождения вечных и бессмертных духов Райского восхождения, как на <<юдоль сотворения души>>.
\usection{6.\bibnobreakspace <<Страх Господень>>}
\vs p149 6:1 В Гамале во время вечерней беседы Филипп сказал Иисусу: <<Учитель, почему Писание учит нас >>страшиться Господа<<, а ты хочешь, чтобы мы смотрели на Отца Небесного без страха? Как нам согласовать эти учения?>> Иисус ответил Филиппу, сказав:
\vs p149 6:2 \P\ <<Дети мои, я не удивляюсь тому, что вы задаете подобные вопросы. Вначале научиться благоговению человек мог только через страх, однако я пришел открыть любовь Отца, так чтобы вас влекла к почитанию Вечного притягательная сила сыновней любви в ответ на великую и совершенную любовь Отца. Я хочу избавить вас от бремени принуждать себя из\hyp{}за рабского страха к безотрадному служению ревнивому и гневливому Богу\hyp{}Царю. Я хочу научить вас отношениям между Богом и человеком, подобным тем, что существуют между Отцом и сыном, так чтобы вы с радостью перешли к величественному и высокому свободному почитанию любящего, справедливого и милосердного Бога\hyp{}Отца.
\vs p149 6:3 Слова <<страх Господень>> имели различные значения в сменявшие друг друга эпохи --- от страха, внушенного болью и ужасом, до страха, внушаемого благоговением и почитанием. Теперь же от почитания я проведу вас через признание, осознание и понимание к \bibemph{любви.} Когда человек сознает лишь деяния Бога, он вынужден страшиться Верховного, когда же человек начинает понимать и ощущать личность и природу Бога живого, он вынужден все больше любить такого благого и совершенного, вселенского и вечного Отца. Именно эта перемена отношения человека к Богу и составляет миссию Сына Человеческого на земле.
\vs p149 6:4 Разумные дети не повинуються своему Отца чтобы они могли получить хорошие подарки из его руки; но, получив в изобилии блага, дарованные повелениями отеческой любви к своим сыновьям и дочерям, эти дети, в ответ на столь сильную любовь, признавая и понимая щедрость благодеяния, вынуждены любить своего отца. Доброта Бога ведет к покаянию; благодеяние Бога ведет к служению, тогда как любовь Бога ведет к разумному и идущему от всего сердца почитанию.
\vs p149 6:5 Ваши предки боялись Бога, потому что он могуч и таинственен. Вы же должны его обожать потому, что он велик в любви, изобилен в милосердии и славен в истине. Могущество Бога порождает страх в сердце человека, однако благородство и праведность его личности рождают благоговение\hyp{}почитание, любовь и добровольное поклонение. Исполненный сознания долга и любящий сын не боится и не страшится даже могущественного и благородного отца. Я пришел в этот мир, чтобы страх уступил место любови, печаль --- радости, сомнения --- уверенности, рабские узы и бессмысленные обряды --- полному любви служению и возвышенному почитанию. Но для пребывающих во тьме по\hyp{}прежнему верно то, что <<начало мудрости --- страх Господень>>. Однако, когда свет явился в большей полноте, сыновья Бога должны славить Бесконечного за то, что он \bibemph{есть,} а не за то, что он \bibemph{делает.}
\vs p149 6:6 Когда дети молоды и легкомысленны, их, несомненно, нужно увещать чтить своих родителей; когда же они становятся старше и начинают больше ценить родительское служение и защиту, тогда благодаря осознанному уважению и все возрастающей привязанности устремляются к тому уровню опыта, где они действительно любят своих родителей больше за то, что они есть, чем за то, что они сделали. Отец естественно любит своего ребенка; ребенок же должен развивать в себе любовь к отцу --- от страха перед тем, что может сделать отец, через благоговение, боязнь, зависимость и почтение к сознательному и почтительному уважению, полному любви.
\vs p149 6:7 Вас учили, что вы должны <<бояться Бога и соблюдать его заповеди, потому что в этом состоит долг человека >>. Я же пришел дать вам новую и более высокую заповедь. Я буду учить вас <<любить Бога и учиться исполнять волю Отца, ибо это --- величайшая привилегия освобожденных сыновей Бога>>. Ваших отцов учили <<бояться Бога, Всемогущего Царя>>. Я же учу вас: <<Любите Бога, Отца всемилостивого>>.
\vs p149 6:8 В царстве небесном, которое я пришел возвещать, нет высшего и могущественного царя; сие царство есть божественная семья. Всеми признанным и безоговорочно почитаемым центром и главой этого обширного братства разумных существ является мой Отец и ваш Отец. Я его Сын, и вы тоже его сыновья. Стало быть, вечная истина заключается в том, что мы с вами --- небесное братство, тем более что мы уже стали братьями во плоти земной жизни. Поэтому перестаньте бояться Бога как царя или служить ему как господину; научитесь поклонятся ему как Творцу; чтите его как Отца вашей духовной юности; любите его как милосердного защитника; и в конце концов почитайте его как любящего и премудрого Отца вашего более зрелого духовного осознания и понимания.
\vs p149 6:9 Ваши неверные представления об Отце Небесном порождают ваши ложные понятия о смирении и являются основным источником вашего лицемерия. Человек по своей природе и происхождению может быть червем из праха; но когда в него вселяется дух Отца, человек становится божественным в своем предназначении. Дарованный дух Отца моего обязательно возвратится к божественному источнику и на вселенский уровень, откуда он происходит, и человеческая душа смертного человека, которая должна стать заново рожденным ребенком этого пребывающего духа, несомненно, вместе с божественным духом вознесется в само присутствие вечного Отца.
\vs p149 6:10 Смирение, действительно, присуще смертному человеку, получающему все эти дары от Отца Небесного, хотя всем подобным людям, ставшим благодаря своей вере кандидатами на вечное восхождение в царствие небесное, присуще божественное чувство собственного достоинства. Бессмысленность и раболепство показного и ложного смирения несовместимы с пониманием источника вашего спасения и осознанием предназначения ваших душ, рожденных от духа. Смирение перед Богом совершенно уместно, если оно коренится в глубинах сердец ваших; кротость перед людьми достойна похвалы; лицемерие же притворного и показного смирения ребяческо и недостойно просвещенных сыновей царства.
\vs p149 6:11 Вы поступаете правильно, когда ведете себя смиренно перед Богом и сдерживаете себя перед людьми, но да будет смирение ваше духовным, а не ведущим к самообольщению проявлением сознательного и самодовольного чувства собственного превосходства. Неслучайно пророком сказано: <<Ходи смиренно перед Богом>>, ибо хотя Отец Небесный есть Бесконечный и Вечный, он также пребывает с <<сокрушенными умом и смиренными духом>>. Отец мой презирает гордню, не выносит лицемерия и ненавидит порочность. Чтобы подчеркнуть ценность искренности и совершенного доверия любящей поддержке и вечному водительству Отца Небесного, я и говорил столь часто о малом ребенке, поясняя таким образом, какие склад ума и реакциядуха необходимы смертному человеку, чтобы войти в духовные реальности царства небесного.
\vs p149 6:12 Пророк Иеремия правильно описал многих смертных, сказав: <<В устах ты близок к Богу, но далек от него в сердце>>. Разве не читали вы также грозные предостережения пророка, который сказал: <<Священники учат за плату и пророки предвещают за деньги. Между тем притворяются благочестивыми и возвещают, что с ними Господь>>? Разве не предостерегали вас против тех, кто <<с ближними своими говорят о мире, когда в сердце у них зло>>, против тех, <<у кого на устах лесть, а сердце отдано двурушничеству>>? Из всех печалей для доверчивого человека нет ничего ужаснее, как быть обиженным в доме верного друга.
\usection{7.\bibnobreakspace Возвращение в Вифсаиду}
\vs p149 7:1 Посоветовавшись с Симоном Петром и с одобрения Иисуса, Андрей дал наказ Давиду в Вифсаиде отправить посланников к разным проповедующим группам с указанием закончить путешествие и возвратиться в Вифсаиду к определенному времени в четверг 30 декабря. К ужину в тот дождливый день все из отряда апостолов и учителей\hyp{}евангелистов прибыли в дом Заведея.
\vs p149 7:2 Группа оставалась вместе до конца субботы, разместившись в домах Вифсаиды и близлежащего Капернаума, после чего всему отряду был предоставлен двухнедельный отпуск, чтобы сходить домой к своим семьям, навестить своих друзей или заняться рыболовством. Два или три дня, проведенные ими вместе в Вифсаиде, были поистине радостными и вдохновляющими; даже старшие из учителей смогли поучиться у младших по возрасту проповедников, когда те говорили о своем опыте.
\vs p149 7:3 Из 117 евангелистов, которые участвовали во втором путешествии с проповедями, лишь примерно семьдесят пять выдержали испытание реальным опытом и после двухнедельного отпуска пришли, чтобы получить назначение ко служению. Иисус с Андреем, Петром, Иаковом и Иоанном оставались в доме Заведея и провели много времени, беседуя о благоденствии и расширении царства.
