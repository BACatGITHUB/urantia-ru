\upaper{62}{Начальные расы раннего человека}
\vs p062 0:1 Около одного миллиона лет назад в результате трех последовательных и неожиданных мутаций от ранней ветви лемурного типа плацентарных млекопитающих отделились непосредственные предки человечества. Доминирующие качества этих ранних лемуров произошли от западной, или поздней американской группы эволюционирующей плазмы. Но прежде чем установилась прямая линия предков человека, доминантные признаки этого племени были усилены воздействием центральной имплантации жизни, эволюционирующей в Африке. Восточная группа жизни мало, практически ничего не добавила к действительному созданию на планете предков человека.
\usection{1. Ранние типы лемуров}
\vs p062 1:1 Ранние лемуры, имевшие отношение к предковым типам человека, не состояли в прямом родстве с уже существовавшими племенами гиббонов и человекообразных обезьян, обитавших в то время в Евразии и северной Африке, чье потомство живет и в настоящее время. Не были они также и потомками современного типа лемуров, хотя и происходят от общего для них, давно уже вымершего, предка.
\vs p062 1:2 Пока в западном полушарии эволюционировали эти ранние лемуры, в юго\hyp{}западной Азии, в исходной области центральной имплантации жизни, но на границах восточных регионов, произошло установление прямых млекопитающих предков человечества. Несколько миллионов лет назад лемуры североамериканского типа мигрировали на запад через Берингов сухопутный мост и медленно продвигались к юго\hyp{}западу вдоль Азиатского побережья. Эти мигрирующие племена в конце концов достигли благодатных мест, лежащих тогда между расширившимся Средиземным морем и поднимающимися гористыми регионами Индостана. В этих землях к западу от Индии они объединились с другими сходными племенами, дав таким образом начало будущим предкам человеческой расы.
\vs p062 1:3 С течением времени морское побережье Индии к юго\hyp{}западу от гор постепенно ушло под воду, полностью изолировав жизнь этого региона. Проникнуть или покинуть этот Месопотамский, или Персидский, полуостров было невозможно: он был связан с остальным миром только через северный путь; но и тот периодически перекрывался дрейфующими на юг ледниками. И в этой, тогда почти райской области от высших потомков лемурного типа млекопитающих выделились две важнейшие группы --- прапредки современных обезьян и людей.
\usection{2. Ранние млекопитающие}
\vs p062 2:1 Немногим более миллиона лет назад \bibemph{неожиданно} возникли Месопотамские ранние млекопитающие --- прямые потомки плацентарных млекопитающих североамериканского лемурного типа. Это были небольшие активные создания, почти трех футов в высоту; и они, хотя обычно не ходили на задних ногах, легко могли стоять выпрямившись. Они были волосатыми и подвижными и болтали, как обезьяны, но, в отличие от обезьян, они были плотоядными. У них был примитивный противопоставленный большой палец руки и очень полезный крупный хватательный большой палец ноги. Начиная с этого момента у дочеловеческих видов успешно развивался противопоставленный большой палец руки и в то же время постепенно утрачивалась хватательная способность большого пальца ноги. Поздние племена человекообразных обезьян сохранили крупный хватательный большой палец ноги, но у них не развился человеческий тип большого пальца руки.
\vs p062 2:2 Эти ранние млекопитающие достигали полного роста в возрасте трех или четырех лет, имея потенциальную продолжительность жизни в среднем около двадцати лет. Как правило, они приносили одного детеныша, хотя время от времени и близнецов.
\vs p062 2:3 Из всех животных, которые до того времени существовали на земле, у представителей этого нового вида был наиболее крупный по сравнению с их размерами мозг. У них было много эмоций и много инстинктов, которые позднее стали характерны для примитивного человека, они были очень любознательными, и у них значительно поднималось настроение, когда им удавалось что\hyp{}либо сделать. Чувство голода и половое влечение были сильно развиты, и определенный сексуальный отбор проявлялся в грубых формах ухаживания и избрания партнеров. Они свирепо дрались, защищая своих родных, и были по\hyp{}своему нежны в семейном союзе, обладая чувством самоуничижения, граничащего со стыдом и угрызениями совести. Они были очень любящими и трогательно верными своим партнерам, но если обстоятельства разлучали их, они выбирали новых партнеров.
\vs p062 2:4 Они не были могучего телосложения, но обладали острым умом, чтобы понимать опасности своего лесного местообитания, они находились в состоянии постоянного страха, которое вело к тем мудрым мерам предосторожности, сильно помогавшим в выживании: строительству примитивных убежищ на вершинах деревьев, что позволяло избежать многих опасностей наземной жизни. Склонность человечества испытывать чувство страха зародилась именно в те времена.
\vs p062 2:5 Этим ранним млекопитающим присуще стадное чувство в большей степени, чем это проявлялось у кого\hyp{}либо ранее. Они были по\hyp{}настоящему общительными существами, но тем не менее чрезвычайно драчливыми, если их каким\hyp{}либо образом отвлекали от повседневных занятий, и демонстрировали вспыльчивый темперамент, когда были разозлены. Их агрессивная природа, однако, служила доброй цели; развитые группы без колебаний уничтожали более слабых соседей, и, таким образом, вид прогрессивно улучшался за счет естественного отбора. Очень скоро они начали доминировать в среде более слабых созданий этого региона, и из более древних обезьяноподобных неплотоядных племен выжило всего лишь несколько.
\vs p062 2:6 Эти агрессивные маленькие животные размножались и расселялись по Месопотамскому полуострову в течение более тысячи лет, постоянно развиваясь физически и интеллектуально. И спустя всего лишь семьдесят поколений, после того как от высшего типа лемурного предка появилось это новое племя, произошло следующее эпохальное событие --- \bibemph{внезапная} дифференциация предков следующей жизненно важной ступени в эволюции человеческих существ на Урантии.
\usection{3. Срединные млекопитающие}
\vs p062 3:1 В начале существования ранних млекопитающих, в жилище на вершине дерева у самой развитой пары этих проворных созданий родились близнецы, самец и самка. По сравнению со своими родичами они были по\hyp{}настоящему красивыми маленькими созданиями. У них на теле было мало волос, но это не создавало неудобств, потому что они жили в теплом и ровном климате.
\vs p062 3:2 Эти детеныши выросли немногим более четырех футов в высоту. Они были во всех отношениях крупнее своих родителей, у них были более длинные ноги и более короткие руки. У них был почти идеально противопоставленный большой палец руки, почти так же приспособленный для различных работ, как и большой палец руки современного человека. Они ходили выпрямившись, ступня у них была почти так же приспособлена для ходьбы, как и у более поздних человеческих рас.
\vs p062 3:3 Их мозг был хуже развит и меньше, чем у человеческих существ, но намного мощнее и сравнительно более крупный, чем у их предков. Близнецы рано проявили высокоразвитый интеллект, и вскоре были признаны главами целого племени ранних млекопитающих, таким образом, были основаны примитивная форма общества и незрелое экономическое разделение труда. Эти брат и сестра сошлись и вскоре обзавелись двадцатью одним ребенком, очень похожими на них --- все выше четырех футов, во всех отношениях более развитые, чем предковый вид. Эта новая группа сформировала ядро срединных млекопитающих
\vs p062 3:4 Когда число членов этой новой и выделяющейся группы сильно увеличилось, разразилась война, безжалостная война; и когда завершилась ужасная битва, в живых не осталось никого из существовавшей прежде предковой расы ранних млекопитающих. Менее многочисленное, но более сильное и интеллектуальное потомство этого вида выжило за счет своих предков.
\vs p062 3:5 И теперь, в течение почти пятнадцати тысяч лет (шестисот поколений), эти создания стали ужасом этой части мира. Все огромные и злобные животные прежних времен вымерли. Крупные твари, которые изначально обитали в этом регионе, не были плотоядными, а более крупные виды семейства кошачьих --- львы и тигры еще не расселились в этом необычайно защищенном уголке земли. Поэтому срединные млекопитающие расхрабрились и подчинили себе весь этот регион планеты.
\vs p062 3:6 \P\ Срединные млекопитающие превзошли предковые виды во всех отношениях. Даже жить они могли дольше --- около двадцати пяти лет. У этого нового вида появились отдельные рудиментарные человеческие черты. В дополнение к врожденным склонностям, свойственным их предкам, срединные млекопитающие были способны негодовать в определенных отталкивающих ситуациях. Более того, они имели хорошо выраженный накопительный рефлекс --- прятали пищу для последующего использования и собирали гладкую округлую гальку и определенного типа окатанные камни, пригодные для защиты и нападения.
\vs p062 3:7 Эти срединные млекопитающие были первыми, продемонстрировавшими выраженные строительные способности, что доказано их результатами в строительстве как домов на вершинах деревьев, так и много туннельных подземных убежищ; они были самыми первыми видами млекопитающих, которые обеспечивали свою безопасность в убежищах на деревьях и под землей. Они, как правило, покидали деревья, --- свое жилище, чтобы провести день на земле, и снова возвращались ночевать на вершины деревьев.
\vs p062 3:8 С течением времени естественный прирост численности в конечном итоге привел к серьезной борьбе за пищу и к сексуальному соперничеству, и все это вылилось в череду междоусобных столкновений, которые почти уничтожили весь вид. Эти стычки продолжались до тех пор, пока в живых не осталось меньше ста особей из одной группы. Но мир восторжествовал в очередной раз, и эта единственная сохранившаяся стая заново построила свои древесные спальни и опять продолжила нормальное и полумирное существование.
\vs p062 3:9 \P\ Вы вряд ли сможете представить, какая узкая грань отделяла порой ваших дочеловеческих предков от вымирания. Если бы прапредок всего человечества --- лягушка прыгнула бы на два дюйма ближе в определенном случае, весь ход эволюции заметно бы изменился. Лемуроподобная особь --- мать из вида ранних млекопитающих была на волосок от гибели не менее пяти раз, прежде чем она родила родоначальника нового, более высоко развитого отряда млекопитающих. В особенности концом всего мог стать момент, когда молния ударила в дерево, на котором спала будущая мать близнецов\hyp{}Приматов. Оба родителя, относящиеся к срединным млекопитающим, были потрясены, изранены и сильно обожжены; трое из семи детенышей были убиты этим ударом молнии с небес. Эти эволюционирующие животные были почти суеверны. Эта пара, чей древесный дом был уничтожен, действительно являлась лидером самой развитой группы срединных млекопитающих; и, следуя их примеру, не меньше половины племени, включая более интеллектуальные семейства, переместились почти на две мили от этого места и начали постройку новых домов на деревьях и новых подземных убежищ --- своих временных укрытий от всякой нежданной опасности.
\vs p062 3:10 Вскоре после завершения строительства их дома, эта пара, ветераны многих схваток, стала гордыми родителями близнецов, наиболее интересных и важных животных, когда\hyp{}либо до того рожденных в мире, так как они были первыми представителями нового вида \bibemph{Приматов,} образующих следующую жизненно важную ступень дочеловеческой эволюции.
\vs p062 3:11 \P\ Одновременно с рождением этих близнецов\hyp{}Приматов, другая пара --- особенно отставшие в развитии самец и самка племени срединных млекопитающих, пара, которая была худшей как ментально, так и физически, --- также родила близнецов. Эти близнецы, самец и самка, были безразличны к завоеваниям, их интересовало только добывание пищи, а поскольку они не были плотоядными, то вскоре потеряли всякий интерес к поиску жертвы. Эти отсталые близнецы стали прародителями современных обезьян. Их потомки находили теплые южные края с мягким климатом и изобилием тропических фруктов, где они продолжали жить почти так же, как и всегда, те же ветви которые спарились с ранними типами гиббонов и человекообразных обезьян, впоследствии сильно деградировали.
\vs p062 3:12 И, таким образом, можно легко понять, что человек и человекообразные обезьяны состоят в родстве только потому, что выделились из срединных млекопитающих --- племени, в котором произошло одновременное рождение и последующее разделение двух пар близнецов: низшей пары, образовавшей современные типы обезьяны --- бабуина, шимпанзе и гориллу; и высшей пары, продолжившей линию восхождения, увенчавшейся появлением человека.
\vs p062 3:13 Современный человек и обезьяны произошли от одного и того же племени и вида, но от разных родителей. Предки человека происходят из высших ветвей племени срединных млекопитающих, тогда как современные обезьяны (исключая определенные, существовавшие ранее типы лемуров, гиббонов, человекообразных обезьян и других обезьяноподобных созданий) являются потомками самой низшей пары этой группы срединных млекопитающих; пары, которая выжила только потому, что во время последней самой жестокой стычки своего племени в течение двух недель пряталась в подземном хранилище пищи и вышла наружу только после того, как опасность осталась позади.
\usection{4. Приматы}
\vs p062 4:1 Возвратимся к рождению высших близнецов, самца и самки, двум лидирующим представителям племени срединных млекопитающих: эти животные детеныши были необычными, у них было еще меньше волос, чем у их родителей, и с очень раннего возраста они упорно ходили выпрямившись. Все их предки только учились ходить на задних лапах, но эти близнецы\hyp{}Приматы выпрямились с самого начала. Они достигли роста свыше пяти футов, и их головы были крупнее, чем у других их соплеменников. Рано научившись общаться друг с другом жестами и звуками, они так никогда и не добились понимания этой новой символики со стороны их сородичей.
\vs p062 4:2 Когда им было почти четырнадцать лет, они сбежали из племени, пошли на запад, образовали свою семью и основали новый вид Приматов. Эти новые создания очень точно называны \bibemph{Приматами,} поскольку были прямыми и непосредственными животными предками человеческого рода как такового.
\vs p062 4:3 Таким образом, Приматы заняли западное побережье Месопотамского полуострова, который тогда вытянулся в южное море, тогда как менее интеллектуальные и близкородственные племена жили вокруг оконечности полуострова и на восточном берегу.
\vs p062 4:4 \P\ Приматы были скорее людьми, чем животными, в отличие от их срединных млекопитающих\hyp{}предшественников. Пропорции скелета у этого нового вида и у примитивных человеческих рас были очень сходны. Полностью развился человеческий тип руки и ноги, эти создания могли ходить и даже бегать, так же как и любой их более поздний человеческий потомок. Они практически отказались от обитания на деревьях, хотя и продолжали, в качестве меры безопасности, убегать на вершины деревьев по ночам, поскольку, как и их ранние предки, были сильно подвержены страху. Они стали чаще прибегать к помощи рук, что значительно увеличивало врожденные умственные способности, но они все еще не обладали разумом, который действительно можно назвать человеческим.
\vs p062 4:5 Хотя по эмоциональной природе Приматы мало отличалась от их предков, они проявляли больше человеческих черт во всех своих склонностях. Они были по\hyp{}настоящему великолепными и высшими животными; зрелости достигали примерно к десяти годам, и естественная продолжительность жизни составляла около сорока лет. То есть они должны были столько прожить перед тем, как умереть естественной смертью, но в те древние времена очень немногие животные умирали естественной смертью; борьба за существование вообще была слишком жестокой.
\vs p062 4:6 И наконец, в результате эволюции почти девятисот поколений, занявшей примерно двадцать одну тысячу лет после появления ранних млекопитающих, Приматы \bibemph{неожиданно} дали жизнь двум замечательным созданиям --- первым по\hyp{}настоящему человеческим существам.
\vs p062 4:7 \P\ Таким образом, ранние млекопитающие, произошедшие от североамериканского типа лемуров, дали начало срединным млекопитающим, а срединные млекопитающие произвели высших приматов, которые стали непосредственными предками примитивной человеческой расы. Племена приматов были последним жизненно важным звеном в эволюции человека, но менее чем через пять тысяч лет от этих удивительных племен не осталось ни одной особи.
\usection{5. Первые человеческие существа}
\vs p062 5:1 С рождения первых двух человеческих существ до 1934 года после Рождества Христа прошло ровно 993 419 лет.
\vs p062 5:2 Оба эти замечательные создания были уже настоящими человеческими существами. У них были превосходно развитые большие пальцы рук, как и у многих их предков, но их ступня была такой же совершенной, как у существующих в настоящее время человеческих рас. Они были ходоки и бегуны, а не лазатели; хватательная функция большого пальца ноги отсутствовала, совершенно отсутствовала. Когда опасность загоняла их на вершины деревьев, они влезали на них точно так же, как люди сегодня. Они взбирались по стволу дерева, как медведь, а не раскачиваясь на ветвях, как шимпанзе или горилла.
\vs p062 5:3 Эти первые человеческие существа (и их потомки) достигали полной зрелости к двенадцати годам и потенциально продолжительность их жизни была примерно семьдесят пять лет.
\vs p062 5:4 У этих человеческих близнецов рано появилось много новых эмоций. Они восхищались и предметами, и другими существами, и были достаточно тщеславны. Но наиболее замечательным результатом эмоционального развития было внезапное появление новой группы подлинно человеческих чувств, связанных с поклонением, включающим благоговение, почтение, смирение и даже примитивную форму благодарности. Страх, в сочетании с незнанием природных явлений, ведет к зарождению примитивной религии.
\vs p062 5:5 Не только эти человеческие чувства проявлялись у примитивных людей, в упрощенной форме им были свойственны и многие другие более высокие чуства. Они в какой\hyp{}то степени испытывали сожаление, стыд и укор, остро чувствовали любовь, ненависть и месть, им было также не чуждо и сильное чувство ревности.
\vs p062 5:6 Эти два первых человека --- близнецы --- были огромным испытанием для своих Приматов\hyp{}родителей. Они были столь любопытны и отважны, что уже к восьми годам не раз были близки к гибели. И поэтому к двадцати годам они были покрыты шрамами.
\vs p062 5:7 Очень рано они научились словесному общению; к десяти годам они выработали улучшенный жестовый и словесный язык, включающий больше полусотни понятий, и очень сильно улучшили и расширили примитивные способы общения своих предков. Но несмотря на все усилия, они смогли научить своих родителей лишь нескольким новым знакам и символам.
\vs p062 5:8 В возрасте около девяти лет в один прекрасный день они отправились путешествовать вниз по реке и многое обсудили. Я и каждый небесный интеллект, присутствующий в тот момент на Урантии, наблюдали за этой полуденной встречей. В этот насыщенный событиями день они договорились жить друг с другом и друг для друга, и это было первым из цепи таких соглашений, которые в конечном итоге привели к решению сбежать от своих низших животных товарищей и отправиться на север, так, сами того не понимая, они стали родоначальниками человеческой расы.
\vs p062 5:9 Мы, сильно заинтригованные тем, что задумали эти два маленьких дикаря, были в то же время бессильны контролировать работу их ума; мы никак не повлияли --- не могли повлиять --- на их решение. Но, в пределах, дозволенных нашей планетарной функцией, мы, Носители Жизни, вместе с нашими сподвижниками, все же задумали увести человеческих близнецов к северу, подальше от их волосатых и по\hyp{}прежнему обитающих на деревьях сородичей. Итак, руководствуясь собственным сознательным выбором, близнецы \bibemph{переселились,} и благодаря нашему руководству они переселились \bibemph{к северу,} в уединенное место, где и избежали возможной биологической деградации в результате смешения с низшими сородичами из племени Приматов.
\vs p062 5:10 Незадолго перед уходом из родных лесов они потеряли мать при набеге гиббонов. Хотя она и не обладала их интеллектом, она испытывала присущее млекопитающим сильное чувство привязанности к своим детенышам, и бесстрашно отдала свою жизнь, спасая замечательную пару. Ее жертва была не напрасной, так как она сдерживала врагов, пока не подоспел отец с подкреплением и не рассеял захватчиков.
\vs p062 5:11 Вскоре после этого молодая пара покинула своих товарищей для того, чтобы в конечном итоге основать человеческую расу, их отец\hyp{}Примат был неутешен --- сердце его было разбито. Он отказывался есть, даже когда другие его дети приносили ему пищу, его блестящие отпрыски были потеряны, жизнь среди обычных собратьев казалась ему бессмысленной; поэтому он отправился в лес, был схвачен врагами --- гиббонами и забит до смерти.
\usection{6. Эволюция человеческого разума}
\vs p062 6:1 С того дня, когда мы, носители Жизни на Урантии, впервые привнесли в планетарные воды жизненную плазму, прошел необыкновенно долгий период тщательных наблюдений и ожиданий: естественно, что появление первых действительно разумных и волевых существ доставило нам огромную радость и верховное удовлетворение.
\vs p062 6:2 Мы следили за тем, как близнецы развивались ментально, наблюдая за действиями семи духов\hyp{}помощников разума, назначенных на Урантию к моменту нашего прибытия на планету. Во время долгого эволюционного развития планетарной жизни эти неутомимые служители разума всегда регистрировали свою возрастающую способность контактировать с успешно расширяющимися способностями мозга прогрессивно развивающихся животных созданий.
\vs p062 6:3 Вначале только \bibemph{дух интуиции} мог функционировать в инстинктивном и рефлексивном поведении изначальной животной жизни. С дифференциацией высших типов \bibemph{дух понимания} был способен наделить такие создания даром спонтанной ассоциации идей. Позже мы наблюдали \bibemph{дух отваги} в действии; эволюционирующие животные по\hyp{}настоящему развили простую форму защитного самосознания. После появления групп млекопитающих мы заметили \bibemph{дух знания,} проявляющий себя во все возрастающей степени. И эволюция высших млекопитающих привела в действие \bibemph{дух обсуждения,} что вызвало рост стадного чувства и начало примитивного социального развития.
\vs p062 6:4 От ранних млекопитающих, через срединных млекопитающих и Приматов, мы наблюдали постепенно усиливающееся влияние первых пяти духов\hyp{}помощников. Но никогда до этого оставшиеся два высших служителя разума не могли еще действовать в типе эволюционирующего разума Урантии.
\vs p062 6:5 Представьте себе нашу радость в тот день --- близнецам было около десяти лет от роду, --- когда \bibemph{дух почитания} осуществил свой первый контакт с разумом женщины\hyp{}близнеца, а вскоре и с мужчиной. Мы знали, что развитие чего\hyp{}то близкородственного человеческому разуму приближалось к кульминации; и когда, около года спустя, в результате обдуманных размышлений и сознательного решения, они наконец дерзнули уйти из дома и отправиться на север, именно тогда \bibemph{дух мудрости} начал действовать на Урантии, а в этих двух теперь уже распознавался человеческий разум.
\vs p062 6:6 Это был непосредственный и новый порядок мобилизации семи духов\hyp{}помощников разума. Мы жили ожиданием; мы понимали, что долгожданный час приближается; мы знали, что были на пороге осуществления наших затянувшихся усилий по развитию волевых созданий на Урантии.
\usection{7. Признание в качестве населенного мира}
\vs p062 7:1 Мы ждали недолго. В полдень, на следующий день после побега близнецов, произошла начальная пробная вспышка сигнала вселенского контура на планетарном приемном фокусе Урантии. Мы, конечно, были на ногах, прекрасно осознавая, что надвигается великое событие; но поскольку этот мир был лишь местом экспериментов над жизнью, у нас не было ни малейшего понятия о том, как мы будем извещены о факте признания интеллектуальной жизни на планете. Но мы ожидали недолго. На третий день после тайного бегства близнецов и перед отбытием отряда Носителей Жизни, прибыл Небадонский архангел из штата установления начального планетарного контура.
\vs p062 7:2 На Урантии в этот полный событиями день наша маленькая группа собралась вокруг планетарного полюса космической связи и получила первое сообщение из Спасограда по вновь установленному контуру разума планеты. И это первое сообщение, продиктованное главой отряда архангелов, гласило:
\vs p062 7:3 «Носителям Жизни на Урантии --- приветствия! Мы в Спасограде, Эдентии и Иерусеме выражаем огромную радость по поводу регистрации в центрах Небадона сигнала о существовании на Урантии разума наделенного волей. Сознательное решение близнецов убежать на север и отделить своих потомков от низших предков было отмечено. Это первое решение разума --- человеческого типа разума --- на Урантии автоматически устанавливает контур коммуникации, через который и передается это начальное сообщение\hyp{}подтверждение».
\vs p062 7:4 Следующим по этому контуру пришли поздравления от Всевышних Эдентии, содержащие инструкции для пребывающих на земле Носителей Жизни, запрещающие вмешиваться в установленные нами формы жизни. Нам предписывалось не вмешиваться в дела человеческого прогресса. Это не означает, что Носители Жизни когда\hyp{}либо произвольно и механически вмешивались в естественный ход планетарных эволюционных процессов, вовсе нет. Но до того момента нам разрешалось манипулировать окружающей средой и защищать жизненную плазму специальным образом, и именно этот необычный, но вполне естественный надзор должен был быть прекращен.
\vs p062 7:5 И как только Всевышние прекратили свою речь, сразу же стало поступать на планету прекрасное сообщение от Люцифера, тогдашнего владыки системы Сатании. Теперь Носители Жизни услышали приветственные слова от их собственного главы и получили его разрешение вернуться в Иерусем. Это сообщение Люцифера содержало официальное признание работы Носителей Жизни на Урантии и освобождало нас в будущем от любой критики наших усилий по улучшению форм жизни Небадона, основанной в системе Сатания.
\vs p062 7:6 Эти послания из Спасограда, Эдентии и Иерусема формально означали, что окончился продолжительный надзор Носителей Жизни на планете. Веками мы выполняли наши обязанности, нам помогали только семь духов\hyp{}помощников разума и Мастера Физические Контролеры. И теперь воля, способность выбирать почитание и восхождение, появилась в эволюционирующих созданиях на планете, и мы осознали, что наша работа закончена, и наша группа приготовилась к отбытию. Поскольку Урантия является планетой с видоизмененной жизнью, нам было дано разрешение оставить двух старших Носителей Жизни с двенадцатью помощниками, и я был избран одним из этой группы и с тех пор пребываю на Урантии.
\vs p062 7:7 993 408 лет назад (считая с 1934 года от Рождества Христа) во вселенной Небадон Урантия была формально признана планетой, населенной людьми. Биологическая эволюция еще раз достигла уровня людей обладающих волей; человек появился на планете 606 Сатании.
\vs p062 7:8 [Представлено Носителем Жизни Небадона, постоянно пребывающим на Урантии.]
