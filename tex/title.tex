\vspace*{\stretch{0.1}}
\begin{center}
\bibcovertitlefont\urantiabook\\[3ex]
\LARGE ПЯТОЕ ЭПОХАЛЬНОЕ ОТКРОВЕНИЕ\\
\vspace*{\stretch{0.1}}
\ifmultivol
\Large
\ifvoli ТОМ I: ПРЕДИСЛОВИЕ, ТЕКСТЫ 1--94\\\fi
\ifvolii ТОМ II: ТЕКСТЫ 95--196\\\fi
\fi
\vspace*{\stretch{0.6}}
\titlesepbig\\
\vspace*{\stretch{0.1}}
\end{center}

\titleframe

\newpage

\begin{center}
\vspace*{\stretch{0.3}}
\begin{center}\shadowbox{\strut\parbox{8cm}{\large\tunemarkup{pgcrownq}{\Large}\tunemarkup{pgafour}{\LARGE}\bfseries\itshape ``Из всех человеческих знаний наиценнейшее — знание религиозной жизни Иисуса и как он её прожил.'' \bibref[(196:1.3)]{p196 1:3}}}\end{center}
\vspace*{\stretch{0.7}}
\itshape
\tunemarkup{pgafour}{\fontsize{18}{22}\selectfont}
\tunemarkup{pgletter}{\fontsize{18}{22}\selectfont}
\tunemarkup{pgluluhb}{\fontsize{12}{15}\selectfont}
\tunemarkup{pghanlin}{\fontsize{9}{12}\selectfont}
\tunemarkup{pgcrownq}{\fontsize{10}{15}\selectfont}
\tunemarkup{pgveligor}{\fontsize{10}{15}\selectfont}
Copyright \textcopyright\ Urantia Society of Greater New York, {\upshape\bfseries www.urantia.nyc}.\\
\tux\ Вёрстка Bibles.org.uk, используя \XeLaTeX\ в системе Linux.\\
Символом {\upshape \P} отмечен первый параграф группы. В 1-ом издании 1955~г. группы параграфов отделяла пустая строка.\\
Текст набран шрифтом \textbf{Adobe \urantiamainfont} \urantiamainfontsize pt.\\[4pt]
\upshape\normalsize\bfseries\mytoday{}\\
\end{center}

\titleframe
