\upaper{147}{Краткое посещение Иерусалима}
\author{Комиссия срединников}
\vs p147 0:1 Иисус и апостолы прибыли в Капернаум в среду 17 марта и перед тем, как отправиться в Иерусалим, провели две недели в Вифсаиде, в доме, где они обычно останавливались. Эти две недели апостолы учили народ у берега моря, тогда как Иисус большую часть времени провел в горах один, посвятив себя делу Отца. В течение этого времени Иисус, сопровождаемый Иаковом и Иоанном Зеведеевыми, совершил два тайных путешествия в Тивериаду, где они встречались с верующими и наставляли их о евангелии царства.
\vs p147 0:2 Многие из домочадцев Ирода верили в Иисуса и посещали эти собрания. Влияние этих верующих в среде официальной семьи Ирода и помогло ослабить неприязнь этого правителя к Иисусу. Эти верующие из Тивериады полностью разъяснили Ироду, что «царство», которое возвещал Иисус, по природе своей духовно и не является политическим начинанием. Ирод вполне доверял этим членам своей собственной семьи и потому не позволял себе чрезмерно тревожиться о распространении слухов об учении и исцелениях Иисуса. У него не было возражений против деятельности Иисуса как целителя или религиозного учителя. Однако, несмотря на благосклонное расположение многих из советников Ирода и даже самого Ирода, существовала группа его подчиненных, которые были настолько подвержены влиянию религиозных лидеров Иерусалима, что оставались злейшими и грозными врагами Иисуса и апостолов и позднее многое сделали, чтобы помешать их публичной деятельности. Величайшая опасность для Иисуса исходила от религиозных лидеров Иерусалима, а не от Ирода. Именно по этой причине Иисус и апостолы и проводили так много времени и выступали с публичными проповедями большей частью в Галилее, а не в Иерусалиме и в Иудее.
\usection{1. Слуга центуриона}
\vs p147 1:1 За день до того, как Иисус и апостолы приготовились идти в Иерусалим на праздник Пасхи, некий Мангус, центурион, или капитан, римской стражи, расположившейся в Капернауме, пришел к управителям синагоги и сказал: «Мой верный слуга болен и уже при смерти. Не сходите ли вы поэтому к Иисусу и не попросите ли его от моего имени исцелить моего слугу?» Поступил же так римский капитан потому, что думал, что еврейские лидеры смогут больше повлиять на Иисуса. Поэтому старейшины пошли к Иисусу, и их представитель сказал: «Учитель, мы убедительно просим тебя пойти в Капернаум и спасти любимого слугу римского центуриона, который достоин твоего внимания, потому что любит наш народ и даже построил нам ту самую синагогу, в которой ты столько раз говорил».
\vs p147 1:2 Выслушав их, Иисус сказал: «Я пойду с вами». Когда же он вместе с ними пришел к дому центуриона и перед тем, как они вошли на его двор, римский солдат выслал своих друзей приветствовать Иисуса и велел им сказать: «Господи, не тревожь себя и не входи в мой дом, ибо я не достоин, чтобы ты вошел под кров мой. Я также не считал себя достойным прийти к тебе; и по этой причине послал старейшин народа твоего. Но я знаю, что можешь ты сказать слово там, где ты стоишь, и слуга мой будет исцелен. Ибо я сам подчиняюсь приказам других и мне подчиняются солдаты и я говорю одному: «„пойди“, и идет; и другому „приди“, и приходит, и слугам моим „сделайте это“, и делают».
\vs p147 1:3 Услышав эти слова, Иисус повернулся и сказал своим апостолам и тем, кто был с ними: «Я удивлен верой этого нееврея. Истинно, истинно говорю вам: не нашел я такой великой веры в Израиле». Отвернувшись от дома, Иисус сказал: «Посему пойдем». И друзья центуриона вошли в дом и сообщили Мангусу, что сказал Иисус. И с того часа слуга начал поправляться и в конце концов выздоровел и снова смог работать.
\vs p147 1:4 Но мы так и не узнали, что же случилось тогда. Это всего лишь запись, и о том, участвовали или не участвовали невидимые существа в исцелении слуги сотника, не было открыто тем, кто сопровождал Иисуса. Нам известен лишь факт полного исцеления слуги.
\usection{2. Путешествие в Иерусалим}
\vs p147 2:1 Рано утром во вторник 30 марта Иисус и группа апостолов отправились через Иорданскую долину в Иерусалим на праздник Пасхи. В Иерусалим они прибыли 2 апреля в пятницу после полудня и, как обычно, обосновались в Вифании. Проходя через Иерихон, они остановились передохнуть, пока Иуда делал вклад некоторой части их общей казны в банк, принадлежавший другу его семьи. Это был первый раз, когда у Иуды появился излишек денег, и этот вклад оставался нетронутым до тех пор, пока они снова не проходили через Иерихон во время того последнего и богатого событиями путешествия в Иерусалим, которое состоялось перед судом над Иисусом и его смертью.
\vs p147 2:2 В этом их путешествии в Иерусалим не было ничего примечательного, но едва они успели обосноваться в Вифании, как отовсюду начали собираться ищущие исцеления для своих тел, утешения для встревоженных умов и спасения для своих душ, и людей этих было так много, что у Иисуса почти не оставалось времени для отдыха. Поэтому они разбили палатки в Гефсимании, и Учитель переходил из Вифании в Гефсиманию и из Гефсимании в Вифанию, чтобы скрыться от толп, которые столь постоянно его осаждали. Апостолы провели в Иерусалиме почти три недели, но Иисус повелел им не выступать с публичными проповедями, а учить лишь наедине и заниматься только личной работой.
\vs p147 2:3 В Вифании они тихо отпраздновали Пасху. И это был первый раз, когда Иисус и все двенадцать вкусили от бескровного пасхального яства. Апостолы Иоанна не ели в эту Пасху с Иисусом и его апостолами, а отмечали праздник с Авениром и многими из первых поверивших проповеди Иоанна. Это была вторая Пасха, которую Иисус со своими апостолами праздновал в Иерусалиме.
\vs p147 2:4 Когда Иисус и двенадцать апостолов отправились в Капернаум, апостолы Иоанна не вернулись вместе с ними. Под руководством Авенира они оставались в Иерусалиме и его предместьях и потихоньку работали на благо расширения царства, тогда как Иисус и двенадцать вернулись трудиться в Галилею. Никогда больше, кроме короткого промежутка времени перед наделением полномочиями и отправкой семидесяти евангелистов, двадцать четыре апостола не были вместе. Однако эти две группы сотрудничали, и, несмотря на различия во мнениях, поддерживали между собой самые лучшие отношения.
\usection{3. В купальне Вифезда}
\vs p147 3:1 После полудня во вторую субботу пребывания в Иерусалиме, когда Учитель и апостолы готовились к участию в службе во храме, Иоанн сказал Иисусу: «Пойдем со мной, я тебе кое\hyp{}что покажу». Иоанн вывел Иисуса через одни из ворот Иерусалима к купальне, называемой Вифезда. Вокруг этой купальни было сооружение с пятью крытыми галереями, где большая группа страдальцев проводила время в поисках исцеления. Это был горячий источник с красноватой водой, которая периодически начинала пузыриться из\hyp{}за скопления газов в каменных пустотах под купальней. Многие верили, что бурление теплой воды было вызвано действием сверхъестественных сил, и в народе было распространено поверье, что первый, вошедший в воду после такого клокотания, исцелится, какой бы немощью он ни страдал.
\vs p147 3:2 После введенных Иисусом ограничений апостолы чувствовали себя несколько беспокойно, и особенно Иоанн, самый молодой из двенадцати. Он привел Иисуса к купальне, думая, что вид собравшихся страдальцев взовет к состраданию Учителя с такой силой, что он будет тронут и совершит чудо исцеления, отчего весь Иерусалим будет потрясен и сразу же обратится к вере в евангелие царства. Иоанн сказал Иисусу: «Учитель, посмотри на всех этих страждущих; неужели мы ничем не можем им помочь?» И Иисус ответил: «Иоанн, зачем искушаешь меня свернуть с избранного мною пути? Почему по\hyp{}прежнему желаешь заменить возвещение евангелия вечной истины совершением чудес и исцелением больных? Сын мой, я не могу сделать то, чего ты желаешь, но собери вместе этих больных и страждущих, чтобы я мог сказать им добрые ободряющие слова вечного утешения».
\vs p147 3:3 Обращаясь к собравшимся, Иисус сказал: «Многие из вас здесь больны и поражены недугами, ибо много лет вели неправильную жизнь. Некоторые страдают из\hyp{}за несчастных случаев, другие --- вследствие ошибок своих предков, тогда как некоторые из вас томятся под гнетом несовершенных условий вашего временного бытия. Однако Отец мой трудится, и буду трудиться я ради того, чтобы улучшить ваше земное состояние, особенно же во имя того, чтобы обеспечить ваше вечное бытие. Никто из нас не может многого сделать, чтобы устранить трудности жизни, если только не увидим мы, что того желает Отец Небесный. В конце концов все мы обязаны исполнять волю Отца. Если бы все вы могли исцелиться от телесных недугов ваших, вы бы действительно удивились, однако куда важнее, чтобы вы очистились от всех духовных болезней и почувствовали себя исцеленными от всех нравственных немощей. Все вы --- дети Бога; вы --- сыновья Отца Небесного. Казалось бы, узы времени причиняют вам боль и страдание, но Бог вечности любит вас. И когда придет судный день, не бойтесь, все вы найдете не только справедливость, но и безграничное милосердие. Истинно, истинно говорю вам: кто слышит евангелие царства и верит в сие учение о сыновстве по отношению к Богу, тот имеет жизнь вечную; такие верующие уже переходят от суда и смерти во свет и жизнь. Грядет час, когда даже те, кто лежат в могилах, услышат глас воскрешения».
\vs p147 3:4 И многие из слушавших поверили в евангелие царства. Некоторые же из страждущих были так вдохновлены и так воспряли духом, что пошли, возвещая о том, что они были также исцелены от своих телесных недугов.
\vs p147 3:5 Один человек, который долгие годы находился в удрученном состоянии и мучительно страдал от немощей своего расстроенного рассудка, возрадовался словам Иисуса и, взяв свою постель, пошел домой, хотя и был день субботы. Все эти годы этот страдалец ждал \bibemph{кого\hyp{}нибудь,} кто бы ему помог; он был жертвой чувства собственной беспомощности до такой степени, что его ни разу не посетила мысль помочь самому себе, что, как выяснилось, он и должен был сделать для своего выздоровления --- взять свою постель и пойти.
\vs p147 3:6 Затем Иисус сказал Иоанну: «Пойдем отсюда, пока первосвященники и книжники не увидели нас и не обвинили нас в том, что мы сказали слова жизни страждущим сим». И они вернулись в храм, чтобы присоединиться к своим товарищам, и вскоре все они отправились на ночлег в Вифанию. Но Иоанн так и не рассказал другим апостолам о своем с Иисусом посещении купальни Вифезды в этот субботний день.
\usection{4. Правило жизни}
\vs p147 4:1 Вечером того же субботнего дня в Вифании, когда Иисус, двенадцать апостолов и группа верующих собрались у костра в саду Лазаря, Нафанаил задал Иисусу этот вопрос: «Учитель, хотя ты учил нас позитивной версии старого правила жизни и велел нам поступать с другими так же, как сами хотим, чтобы поступали с нами, я не понимаю, каким образом мы могли бы всегда твердо придерживаться подобного предписания. Позволь мне проиллюстрировать свое утверждение, сославшись на пример похотливого развратника, который именно безнравственно смотрит на предполагаемого соучастника в грехе. Как можем мы учить, что этот злонамеренный человек должен поступать с другими так же, как хотел бы он сам, чтобы поступали с ним?»
\vs p147 4:2 Услышав вопрос Нафанаила, Иисус тотчас встал на ноги и, указав пальцем на апостола, сказал: «Нафанаил, Нафанаил! Какое направление принимают мысли в сердце твоем! Неужели ты не воспринимаешь мое учение как рожденный от духа? Неужели не внимаешь истине, как люди, наделенные мудростью и духовным пониманием? Призывая вас поступать с другими так же, как хотели бы вы, чтобы поступали с вами, я говорил с людьми высоких идеалов, а не с теми, кто будет искушаем извратить мое учение и представить дело так, будто оно безнравственно поощряет зло».
\vs p147 4:3 Когда Учитель кончил говорить, Нафанаил встал и сказал: «Однако, Учитель, ты не должен думать, будто я одобряю такого рода толкование твоего учения. Я задал этот вопрос, поскольку предположил, что многие подобные люди могут составить себе такое неверное суждение о твоем увещании, и надеялся, что ты дашь нам дальнейшие наставления по этим вопросам». Когда Нафанаил сел, Иисус продолжил свою речь: «Я хорошо знаю, Нафанаил, что ум твой не мог одобрить подобное представление о зле, но я огорчен тем, что всем вам столь часто не удается придать подлинно духовное толкование моим простейшим учениям, наставлению, которое должно быть дано вам на человеческом языке, и в доступных выражениях. Позволь же мне теперь преподать вам урок о различных уровнях значения, придаваемых толкованию этого правила жизни, сей заповеди „поступать с другими так же, как хочешь сам, чтобы поступали с тобой“.
\vs p147 4:4 . \bibemph{Уровень плоти.} Такому чисто эгоистическому и похотливому толкованию примером служит предположение, содержащееся в твоем вопросе.
\vs p147 4:5 \pc . \bibemph{Уровень чувств.} Этот уровень на ступень выше уровня плоти и предполагает, что сочувствие и жалость усугубят толкования человеком этого правила жизни.
\vs p147 4:6 \pc . \bibemph{Уровень разума.} Здесь начинают действовать доводы разума и знание опыта. Здравое суждение диктует, что подобное правило жизни должно толковаться созвучно высочайшему идеализму, воплощенному в возвышенном глубоком чувстве собственного достоинства.
\vs p147 4:7 \pc . \bibemph{Уровень братской любви.} Еще выше находится уровень бескорыстного служения благополучию собратьев. На этой более высокой ступени искреннего общественного служения, происходящего из осознания отцовства Бога и вытекающего из него признания братства людей, обнаруживается новое и еще более прекрасное толкование этого основного правила жизни.
\vs p147 4:8 \pc . \bibemph{Нравственный уровень.} И затем, когда вы достигнете истинных философских уровней толкования, когда обретете подлинную способность проникновения в \bibemph{истинность} и \bibemph{ложность} вещей, когда вы станете ощущать вечную гармоничность человеческих отношений, тогда вы начнете смотреть на подобную проблему толкования так, как, по вашему представлению, воспринимало бы и истолковывало бы благородное, идеалистически настроенное, мудрое и беспристрастное лицо ваши личные проблемы, касающиеся урегулирования ваших жизненных ситуаций.
\vs p147 4:9 \pc . \bibemph{Духовный уровень.} И затем мы восходим на последний и величайший из всех уровень\hyp{}уровень внутреннего, духовного понимания и духовного толкования, который побуждает нас видеть в этом правиле жизни божественную заповедь обращаться со всеми людьми так же, как, по вашему представлению, обращался бы с ними Бог. Таков вселенский идеал человеческих отношений. И таков ваш подход ко всем подобным проблемам, когда постоянное исполнение воли Отца становится вашим верховным желанием. Поэтому я хотел бы, чтобы вы поступали со всеми людьми так же, как, вы знаете, я бы поступал с ними в сходных обстоятельствах».
\vs p147 4:10 \pc Ничто из всего до сих пор сказанного Иисусом апостолам не изумляло их больше. И они продолжали обсуждать слова Учителя еще долго после того, как он ушел спать. Хотя Нафанаил долго не мог прийти в себя от сделанного им предположения, будто Иисус неверно понял суть его вопроса, другие были более чем благодарны тому, что их собрат апостол\hyp{}философ имел смелость задать такой заставляющий задуматься вопрос.
\usection{5. В гостях у фарисея Симона}
\vs p147 5:1 Хотя Симон не являлся членом еврейского синедриона, он был влиятельным фарисеем Иерусалима. Его сердце наполовину уверовало, и он несмотря на то, что мог за это подвергнуться суровому осуждению, решился пригласить Иисуса и его ближайших сподвижников Петра, Иакова и Иоанна в свой дом на званый пир. Симон давно наблюдал за Учителем и его весьма привлекали его учения, а еще больше его личность.
\vs p147 5:2 Богатые фарисеи любили раздавать милостыню и не стеснялись афишировать свою филантропию. Иногда, собираясь облагодетельствовать какого\hyp{}нибудь нищего, они даже трубили в трубу. У этих фарисеев был обычай --- давая пир для знатных гостей, оставлять двери в свой дом открытыми, так чтобы в него могли войти нищие, и, стоя вдоль стен позади ложа для пирующих, имели возможность получать куски пищи, которые бросали им участники пира.
\vs p147 5:3 В данном конкретном случае в доме Симона среди тех, кто пришел с улицы, была женщина сомнительной репутации, которая недавно уверовала в благую весть евангелия царства. Эта женщина была известна всему Иерусалиму как бывшая содержательница одного из так называемых первоклассных публичных домов, расположенных рядом со двором язычников. Приняв учение Иисуса, она закрыла свое неблаговидное заведение и убедила большинство связанных с ней женщин принять евангелие и изменить свой образ жизни; но, несмотря на это, фарисеи продолжали относиться к ней с великим презрением, и она была вынуждена носить распущенные волосы --- признак распутства. Эта неизвестная женщина принесла с собой большую склянку с благовонным помазанием и, стоя позади Иисуса, когда тот возлежал за трапезой, начала умащать его ноги, увлажняя их своими благодарными слезами и вытирая их волосами головы своей. Закончив же сие помазание, она продолжала плакать и целовать его ноги.
\vs p147 5:4 Увидев все это, Симон подумал про себя: «Этот человек, если бы он был пророк, то понимал бы, кто и какая женщина прикасается к нему; что она отъявленная грешница». И Иисус, зная, что происходит в уме у Симона, обратился к нему и сказал: «Симон, я хочу кое\hyp{}что сказать тебе». Симон ответил: «Скажи, Учитель». Тогда Иисус сказал: «У одного богатого заимодавца было два должника. Один должен был пятьсот динариев, а другой пятьдесят. Но так как ни один из них не имел чем заплатить, он простил им обоим. Как ты думаешь, Симон, который из них возлюбит его больше?» Симон отвечал: «Думаю, тот, которому более простил». И Иисус сказал: «Правильно ты рассудил» и, указав на женщину, продолжал: «Хорошенько посмотри на эту женщину, Симон. Я пришел в дом твой как приглашенный гость, и ты воды не дал для ног моих. А эта благодарная женщина слезами умыла мои ноги и волосами головы своей отерла. Ты целования в знак дружеского приветствия мне не дал, а она с тех пор, как я пришел, не перестает целовать у меня ноги. Ты голову маслом мне пренебрег помазать, а она драгоценным миром помазала мне ноги. Каков же смысл всего этого? А тот, что многие грехи ее прощены, и за то она возлюбила много. А кому мало прощается, тот мало любит». И повернувшись к женщине, взял ее руку и, поднимая ее с колен, сказал: «Ты действительно раскаялась в грехах своих, и они прощены. Не падай духом от бездумного и недоброго отношения ближних твоих; оставайся в радости и свободе царства небесного».
\vs p147 5:5 \pc Когда Симон и друзья его, сидящие за столом с ним, услышали эти слова, то изумились еще больше и стали шептаться между собой: «Кто этот человек, что осмеливается даже прощать грехи?» И, услышав их шепот, Иисус повернулся, чтобы отпустить женщину, и сказал: «Иди с миром, женщина; вера твоя спасла тебя».
\vs p147 5:6 Поднявшись вместе со своими друзьями, чтобы уходить, Иисус повернулся к Симону и сказал: «Я знаю сердце твое, Симон; знаю, как ты разрываешься между верой и сомнениями, как ты растерян от страха и обуреваем гордыней; но я молюсь за тебя, чтобы ты смог подчиниться свету и в своем положении в жизни смог испытать такое же мощное преображение ума и духа, какое бы можно было сравнить с огромными переменами, которые евангелие царства уже произвело в сердце твоей незванной и непрошенной гостьи. Объявляю вам всем, что Отец открыл врата царства небесного для всех имеющих веру, чтобы войти, и ни человек, ни группа людей не смогут закрыть эти врата даже для смиреннейшей души или для того, кого считают самым страшным грешником на земле, если таковые искренне ищут входа». И Иисус с Петром, Иаковом и Иоанном оставили принимавшего их хозяина и присоединились к остальным апостолам в лагере в Гефсиманском саду.
\vs p147 5:7 \pc В тот же вечер Иисус обратился к апостолам со словами надолго запечатлевшимися в их памяти об относительности ценности положения по отношению к Богу и о совершенствовании на вечном пути восхождения к Раю. Иисус сказал: «Дети мои, если между чадом и Отцом существует истинная и живая связь, то дитя обязательно будет непрерывно развиваться по направлению к идеалам Отца. Да, поначалу дитя может развиваться медленно, но его развитие, тем не менее, несомненно. Важна не быстрота вашего развития, а его несомненность. Ваши действительные достижения не так важны, как то, что \bibemph{направление} вашего развития устремлено к Богу. То, чем вы становитесь день за днем, бесконечно важнее того, чем вы являетесь сегодня.
\vs p147 5:8 Эта обратившаяся женщина, которую некоторые из вас сегодня видели в доме Симона, в настоящий момент живет на уровне, гораздо более низком, нежели Симон и его благонамеренные товарищи; однако, в то время как эти фарисеи заняты ложным совершенствованием иллюзии вращения по обманчивым кругам бессмысленных ритуальных служб, эта женщина с твердой решимостью приступила к долгим и богатым событиями поискам Бога, и путь ее к небу не преграждают ни духовная гордыня, ни нравственное самодовольство. С человеческой точки зрения, эта женщина намного дальше от Бога, чем Симон, но душа ее находится в поступательном движении; она на пути к вечной цели. У этой женщины впереди огромные духовные возможности. Некоторые из вас могут и не находиться на высоких уровнях души и духа, но вы ежедневно движетесь вперед по пути жизни, открытому через веру в Бога. В каждом из вас существуют огромные возможности для духовного роста. Гораздо лучше иметь хоть небольшую, но живую и возрастающую веру, нежели обладать великим интеллектом с его мертвыми запасами мирской мудрости и духовного неверия».
\vs p147 5:9 Однако Иисус со всей серьезностью остерег своих апостолов от глупости дитя Бога, чересчур полагающегося на любовь Отца. Он заявил, что Отец Небесный --- отнюдь не безвольный, нетребовательный или глупо балующий своих детей родитель, всегда готовый мирится с грехом или простить безрассудство. Он предостерег своих слушателей от ошибки использовать его примеры об отце и сыне таким образом, что могло создаться впечатление, будто Бог подобен некоторым чрезмерно балующим детей неразумным родителям, которые заодно с глупцами земли потворствуют нравственной гибели своих бездумных детей и которые вследствие этого несомненно и непосредственно способствуют падению и раннему нравственному разложению своего собственного потомства. Иисус сказал: «Отец мой не одобряет снисходительно саморазрушительные и губительные для всякого нравственного роста и духовного развития поступки и действия своих детей. Подобные грешные дела --- мерзость в глазах Бога».
\vs p147 5:10 \pc Перед тем, как отправиться в Капернаум, Иисус посетил еще много полузакрытых встреч и пиршеств, где были люди и благородного, и низкого происхождения, богатые и бедные Иерусалима. И весьма многие уверовали в евангелие царства и впоследствии приняли крещение от Авенира и его соратников, которые остались, дабы блюсти интересы царства в Иерусалиме и его предместьях.
\usection{6. Возвращение в Капернаум}
\vs p147 6:1 В последнюю неделю апреля Иисус и двенадцать апостолов отбыли из своего вифанийского пристанища близ Иерусалима и через Иерихон и вдоль Иордана отправились назад в Капернаум.
\vs p147 6:2 Первосвященники и религиозные лидеры евреев много и тайно совещались, пытаясь решить, как поступить с Иисусом. Все они соглашались: необходимо что\hyp{}нибудь сделать, чтобы положить конец его учению, но не могли договориться о способе. Они надеялись, что с ним разделаются гражданские власти, как устранил Иоанна Ирод, но обнаружили, что Иисус вел свое дело так, что римских чиновников не особенно тревожили его проповеди. Поэтому на встрече, которая состоялась за день до отбытия Иисуса в Капернаум, было решено, что он будет арестован по обвинению в религиозных преступлениях и судим синедрионом. Для этого назначили комиссию из шести тайных шпионов, которым поручили следовать за Иисусом, следить за его словами и поступками, а, накопив достаточно улик, свидетельствующих о нарушении закона и богохульстве, вернуться в Иерусалим со своим донесением. Эти шесть евреев догнали отряд апостолов, насчитывавший около тридцати человек, в Иерихоне и, притворившись, что они желают стать учениками, присоединились к семейству последователей Иисуса и оставались с группой до времени начала второго путешествия с проповедями по Галилее; после чего трое из них вернулись в Иерусалим, чтобы представить свое донесение первосвященникам и синедриону.
\vs p147 6:3 \pc Петр произнес проповедь перед множеством людей, собравшихся у переправы через Иордан, и на следующее утро они пошли вверх по течению реки к Амафе. Они хотели проследовать прямо до Капернаума, но здесь собралась такая толпа, что им пришлось задержаться на три дня, проповедуя, уча и крестя. К дому они не трогались до раннего субботнего утра, первого дня мая. Иерусалимские шпионы были уверены: теперь\hyp{}то они обеспечат первое обвинение против Иисуса --- нарушение субботы --- поскольку он предполагал начать свое путешествие в этот день. Но им пришлось разочароваться, поскольку перед самым их отбытием Иисус призвал к себе Андрея и перед всеми велел ему пройти расстояние всего в тысячу шагов, расстояние, которое разрешал закон проходить еврею в день субботы.
\vs p147 6:4 Однако шпионам не пришлось долго ждать возможности обвинить Иисуса и его соратников в нарушении субботы. Когда отряд апостолов проходил по узкой дороге, с каждой стороны которой на расстоянии вытянутой руки колыхались уже созревшие хлеба, некоторые из апостолов, будучи голодны, срывали колосья и ели. Есть зерна, идя по дороге, для путешественников обычное дело, а потому никто не думал, что в подобном поступке содержится нечто преступное. Однако шпионы ухватились за него как за предлог для обвинения Иисуса. Увидев, что Андрей растирает зерна в руке, они подошли к нему и сказали: «Разве ты не знаешь, что срывать и растирать колосья в день субботы противно закону?» Андрей ответил: «Но мы голодны и растирали лишь столько, сколько нужно для утоления голода; и с каких это пор есть зерно в день субботы стало грехом?» Но фарисеи отвечали: «В том, чтобы есть, нет ничего дурного, но вы нарушаете закон, срывая колосья и растирая их между ваших рук; наверняка твой Учитель не одобрит подобные действия». Тогда Андрей сказал: «Но если в том, чтобы есть зерно, нет ничего дурного, то, наверное, и в растирании руками едва ли больше труда, чем в пережевывании зерен, которое вы допускаете; почему вы придираетесь к таким пустякам?» Когда Андрей дал понять, что считает их слова придирками, они возмутились и бросились к Иисусу, который шел позади, беседуя с Матфеем, и с негодованием сказали: «Вот, Учитель, твои апостолы делают то, что закон запрещает делать в субботу; они срывают колосья, растирают их и едят зерно. Мы уверены, что ты прикажешь им прекратить». Тогда Иисус сказал обвинителям: «Вы действительно ревностные защитники закона и хорошо делаете, что помните о дне субботнем и свято храните его; но разве не читали вы в Писании, что однажды, когда Давид был голоден, он и бывшие с ним вошли в дом Бога и ели хлебы предложения, которые не должно было есть никому, кроме священников? А Давид дал этот хлеб и бывшим с ним. И разве вы не читали в законе нашем, что он позволяет делать много насущных дел в день субботы? И не увижу ли я еще до окончания дня, как вы едите то, что принесли для этого дня? Дорогие мои, вы правильно делаете, что ревностно соблюдаете субботу, но будет еще лучше, если вы станете оберегать здоровье и благополучие ваших собратьев. Объявляю вам: суббота для человека, а не человек для субботы. Если же вы здесь с нами для того, чтобы следить за моими словами, то открыто говорю вам: Сын Человеческий есть господин и субботы».
\vs p147 6:5 Фарисеи были изумлены и смущены его мудрыми и проницательными словами. До конца дня они держались отдельно и не решались более задавать вопросы.
\vs p147 6:6 \pc Сопротивление Иисуса еврейским традициям и рабским обрядам было всегда \bibemph{позитивным.} Оно выражалось в том, что он делал, и в том, что он утверждал. Учитель не тратил много времени на негативные обвинения. Он учил, что знающие Бога могут наслаждаться свободой жизни, не обманывая себя разрешениями на право грешить. Иисус сказал апостолам: «Блаженны вы, люди, если просвещены истиной и действительно знаете, что делаете; если же не ведаете божественной стези, то несчастны вы и уже нарушители закона».
\usection{7. Снова в Капернауме}
\vs p147 7:1 В понедельник 3 мая около полудня Иисус и двенадцать апостолов прибыли на лодке в Вифсаиду из Тарихеи. На лодке же они плыли затем, чтобы отделаться от тех, кто путешествовал вместе с ними. Но на следующий день остальные путешественники, включая и официальных шпионов из Иерусалима, снова нашли Иисуса.
\vs p147 7:2 Во вторник вечером, когда Иисус проводил одно из своих обычных занятий, состоявших из вопросов и ответов, главный из шести шпионов сказал ему: «Сегодня я говорил с одним из учеников Иоанна, который присутствует здесь при твоем учении, и мы с ним затрудняемся понять, почему ты никогда не приказываешь своим ученикам поститься и молиться, как постимся мы, фарисеи, и как велел Иоанн своим последователям?» И Иисус, сославшись на утверждение Иоанна, ответил тому, кто задал вопрос: «Постятся ли сыны чертога брачного, пока с ними жених? Пока жених с ними, они едва ли могут поститься. Но приближается время, когда удалится от них жених, и тогда дети чертога брачного, несомненно, будут поститься и молиться. Молиться --- естественно для детей света, однако пост не является частью евангелия царства небесного. Позвольте напомнить вам, что мудрый портной не пришивает заплату из новой и нестиранной ткани к старой одежде, чтобы она, намокнув, не села и не сделала дыры еще хуже. И не наливают вина молодого в мехи ветхие, чтобы не прорвало мехи молодое вино и не пропало и вино, и мехи. Мудрый человек наливает молодое вино в мехи новые. Поэтому ученики мои проявляют мудрость, ибо не привносят много от старого порядка в новое учение евангелия царства. У вас, утративших учителя вашего, возможно, существуют основания поститься какое\hyp{}то время. Пост может являться уместной частью закона Моисеева, но в грядущем царстве сыновья Бога будут испытывать свободу от страха и радость в божественном духе». Услышав эти слова, ученики Иоанна утешились, тогда как сами фарисеи смутились еще больше.
\vs p147 7:3 Тогда Учитель стал предостерегать своих слушателей против мнения, будто все старое учение должно быть полностью заменено новыми доктринами. Иисус сказал: «Что старо и к тому же \bibemph{верно,} должно остаться. Подобно тому, что ново, но ложно, должно быть отвергнуто. Однако имейте веру и смелость принять новое и к тому же истинное. Помните, что написано: „Старого друга не бросай, ибо нового с ним не сравнишь. Как вино молодое, так и новый друг; если станет оно старым, с радостью будешь его пить“».
\usection{8. Праздник духовной добродетели}
\vs p147 8:1 В ту ночь, спустя несколько часов после того, как обычные слушатели ушли спать, Иисус продолжал учить своих апостолов. Это особое наставление он начал, процитировав из Пророка Исайи:
\vs p147 8:2 \pc «„Почему вы постились? Ради чего смиряете души ваши, продолжая находить удовольствие в угнетении и получать наслаждение от несправедливости? Вот, вы поститесь ради ссор и распрей и для того, чтобы дерзкой рукой бить других. Однако вы не должны поститься так, чтобы голос ваш был услышан на высоте.
\vs p147 8:3 Таков ли пост, который я избрал, --- день, в который томит человек душу свою; когда гнет голову свою, как тростник, и пресмыкается в рубище и прахе? Это ли осмелишься называть постом и днем, угодным Господу? Не этот ли пост я должен избрать: освободиться от оков неправды, развязать узы бремени, и угнетенных отпустить на свободу, и расторгнуть всякое ярмо, разделить с голодным хлеб мой и бездомных и бедных ввести в дом мой? И, видя нагих, одеть их.
\vs p147 8:4 Тогда откроется, как заря, свет твой, и исцеление твое скоро приумножится. Правда твоя пойдет перед тобою, а слава Господня будет сопровождать тебя. Тогда воззовешь к Господу, и он ответит; возопиешь, и он скажет: „вот я“. И все это совершит он, если воздержишься от угнетения, осуждения и тщеславия. Отцу угоднее, чтобы ты отдал сердце твое голодному и служил душам страдальцев; тогда свет твой взойдет во тьме и даже мрак твой будет как полдень. Тогда Господь, будет направлять тебя, насыщая душу твою и обновляя силы твои. Ты будешь, как напоенный водою сад и как источник, которого воды никогда не иссякают. И те, кто поступает так, вернут утраченную славу; и восстановят они основания многих поколений; и будут называть их восстановителями разрушенных стен, восстановителями надежных путей, на которых обретаться“».
\vs p147 8:5 \pc И потом Иисус до глубокой ночи внушал своим апостолам истину о том, что только вера их, а не томление души и не усмирение тела обеспечила им пребывание в царстве настоящего и будущего. Он призвал апостолов, по крайней мере жить, согласно идеям древнего пророка и выразил надежду, что они намного превзойдут даже идеалы Исайи и более древних пророков. В ту ночь его последние слова были таковы: «Возрастайте в благодати через веру живую, которой ведаете о том, что вы --- сыны Бога, и в то же время считаете каждого человека своим братом».
\vs p147 8:6 Когда Иисус перестал говорить и каждый ушел к себе спать, было уже позже двух часов ночи.
