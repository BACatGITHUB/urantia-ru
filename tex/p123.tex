\upaper{123}{Pаннее детство Иисуса}
\vs p123 0:1 Из\hyp{}за того, что положение семьи в Вифлееме было неопределенным и полным тревог, Мария не отняла ребенка от груди, пока они благополучно не прибыли в Александрию, где получили возможность вести нормальную жизнь. Они поселились вместе с родственниками, и Иосиф мог содержать семью, так как вскоре после приезда он получил работу. Несколько месяцев он работал плотником, а затем достиг положения десятника большой группы рабочих, занятых строительством одного из общественных зданий. Этот новый опыт подал ему мысль после возвращения в Назарет стать подрядчиком и строителем.
\vs p123 0:2 \P\ На протяжении этих ранних лет младенчества Иисуса Мария неустанно и бдительно следила за тем, чтобы с ее ребенком не случилось ничего такого, что могло бы подвергнуть опасности его благополучие или каким\hyp{}либо образом воспрепятствовать исполнению его миссии на земле; не было матери, более преданной своему ребенку. В доме, где жил Иисус, было двое других детей примерно его возраста, и у близких соседей было еще шестеро, по возрасту достаточно близких Иисусу, чтобы быть подходящими товарищами по играм. Сначала Мария склонялась к тому, чтобы не отпускать Иисуса от себя. Она боялась, что если ему будет разрешено играть в саду с другими детьми, с ним может что\hyp{}нибудь случиться, но Иосифу с помощью родных удалось убедить ее, что нельзя лишать Иисуса опыта, необходимого, чтобы научиться общаться с детьми своего возраста. Мария, осознав, что чрезмерная и ненужная опека может привести к тому, что Иисус станет застенчивым и несколько эгоцентричным, наконец согласилась, что обетованное дитя должно расти так же, как любой другой ребенок; но и приняв это решение, она всегда была начеку, когда малыши играли возле дома или в саду. Только любящая мать может понять беспокойство о безопасности сына, которое Мария носила в своем сердце все годы младенчества и раннего детства Иисуса.
\vs p123 0:3 На протяжении двух лет, проведенных в Александрии, Иисус был вполне здоров и продолжал нормально расти. Кроме нескольких друзей и родственников, никто не знал, что Иисус --- «обетованное дитя». Один из родственников Иосифа открыл это нескольким друзьям в Мемфисе, потомкам древнего Эхнатона, и они незадолго до возвращения назаретского семейства в Палестину собрались вместе с небольшой группой верующих из Александрии в роскошном доме родственника\hyp{}покровителя Иосифа, чтобы пожелать добра семье из Назарета и выразить уважение к ребенку. По этому случаю собравшиеся друзья подарили Иисусу рукопись полного греческого перевода иудейского Писания. Но эта рукопись иудейских священных текстов не была передана Иосифу до тех пор, пока и он, и Мария окончательно не решили отказаться от приглашения друзей из Мемфиса и Александрии остаться в Египте. Эти верующие люди настаивали, что судьбоносный ребенок сможет оказать гораздо большее влияние на мир, будучи жителем Александрии, чем в любом другом месте Палестины. Из\hyp{}за этих уговоров отъезд в Палестину был отложен на некоторое время после того, как они получили известие о смерти Ирода.
\vs p123 0:4 \P\ В конце концов Иосиф и Мария покинули Александрию в лодке, принадлежавшей их другу Езраиону, направлявшемуся в Иоппию, и прибыли в этот порт к концу августа 4 года до н.э\ldots Они отправились прямо в Вифлеем, где провели весь сентябрь, обсуждая с друзьями и родственниками, оставаться ли им там или вернуться в Назарет.
\vs p123 0:5 Мария никогда до конца не отказывалась от мысли, что Иисус должен вырасти в Вифлееме, городе Давида. Иосиф по\hyp{}настоящему не верил, что их сын должен стать царственным избавителем Израиля. Кроме того, он знал, что сам в действительности не являлся потомком Давида; Иосиф считался потомком Давидовым из\hyp{}за того, что один из его предков был усыновлен коленом Давидовым. Мария, конечно, считала Город Давида самым подходящим местом для воспитания нового претендента на трон Давидов, но Иосиф предпочел сделать ставку на Ирода Антипу, а не на его брата Архелая. Он серьезно опасался за безопасность ребенка в Вифлееме или любом другом городе Иудеи и предполагал, что скорее Архелай будет продолжать опасную линию поведения своего отца, чем Антипа в Галилее. Помимо прочих причин, Иосиф откровенно заявлял, что Галилея будет лучшим местом для воспитания и образования ребенка, но понадобилось три недели, чтобы убедить в этом Марию.
\vs p123 0:6 К первому октября Иосиф убедил Марию и всех друзей, что им лучше вернуться в Назарет. Соответственно, в начале октября 4 года до н.э. они отбыли из Вифлеема в Назарет через Лидду и Скифополь. Они выехали ранним воскресным утром, Мария и ребенок ехали верхом на недавно приобретенном вьючном животном, а Иосиф и пятеро сопровождавших их родных шли пешком; родственники Иосифа не отпустили их в путешествие до Назарета одних. Они боялись идти в Галилею через Иерусалим и Иорданскую долину, а западные пути не были вполне безопасны для двух одиноких путешественников с младенцем.
\usection{1. Возвращение в Назарет}
\vs p123 1:1 На четвертый день путники достигли цели своего путешествия целыми и невредимыми. Они, не предупредив заранее, отправились в Назарете в дом, который более трех лет занимал один из женатых братьев Иосифа, и тот очень удивился, увидев их. Все произошло так быстро и незаметно, что ни семья Иосифа, ни семья Марии даже не знали, что они покинули Александрию. На следующий день брат Иосифа уехал со своим семейством, и Мария впервые с момента рождения Иисуса с радостью обосновалась со своей небольшой семьей в собственном доме. Менее чем через неделю Иосиф нашел работу плотника, и они были очень счастливы.
\vs p123 1:2 Ко времени возвращения в Назарет Иисусу было почти три года и два месяца. Он очень хорошо перенес путешествие, был совершенно здоров и по\hyp{}детски ликовал и радовался тому, что у него появился собственный дом, в котором он мог бегать и веселиться. Однако Иисус сильно скучал по обществу своих александрийских товарищей по играм.
\vs p123 1:3 По дороге в Назарет Иосиф убедил Марию, что неразумно распространять среди галилейских друзей и родственников слухи о том, что Иисус есть обетованное дитя. Они решили воздерживаться от любых упоминаний об этом в присутствии кого бы то ни было. И оба очень строго соблюдали этот обет.
\vs p123 1:4 Весь четвертый год жизни Иисуса был периодом нормального физического развития и необычайной умственной активности. Между тем он сильно привязался к соседскому мальчику примерно того же возраста, звали его Иаков. Иисус и Иаков всегда были счастливы, играя вместе, а когда выросли, стали большими друзьями и верными соратниками.
\vs p123 1:5 Следующим важным событием в жизни назаретской семьи было рождение второго ребенка, Иакова, происшедшее ранним утром 2 апреля 3 г. до н.э. Иисус трепетал при одной мысли о том, что у него появился маленький братишка, и часами простаивал рядом с ним, просто наблюдая за его первыми движениями.
\vs p123 1:6 В середине лета того же года Иосиф построил небольшую мастерскую рядом с деревенским источником и недалеко от места временной стоянки караванов. С этих пор он выполнял совсем немного плотницкой работы. Вместе с ним работали два его брата и несколько других ремесленников, которые выполняли заказы, пока Иосиф оставался в мастерской, изготавливая ярма, плуги и другие изделия из дерева. Он делал также вещи из кожи, вервия и холста. И Иисус, по мере того как подрастал, свободное от учебы время делил примерно поровну, помогая матери с домашними делами и наблюдая за работой отца в мастерской, при этом он становился свидетелем разговоров и сплетен проводников караванов и их пассажиров, прибывавших со всех четырех концов света.
\vs p123 1:7 В июле этого года, за месяц до того, как Иисусу исполнилось 4 года, произошла вспышка инфекционного кишечного заболевания, распространившегося от контакта с путниками караванов. Мария была так обеспокоена опасностью, которая грозила Иисусу из\hyp{}за эпидемии, что взяв обоих детей, бежала в деревенский дом своего брата рядом с Саридом, в нескольких милях южнее Назарета по Меггидосской дороге. Они не возвращались в Назарет больше двух месяцев; Иисусу очень понравился этот его первый опыт пребывания в сельской местности.
\usection{2. Пятый год жизни Иисуса (2 г. до н.э.)}
\vs p123 2:1 Через год с небольшим после возвращения в Назарет мальчик Иисус достиг возраста, соответствующего принятию первого личного и настоящего нравственного решения; и тогда был послан, чтобы пребывать с ним, Настройщик Мысли, божественный дар Pайского Отца, который некогда служил Махивенте Мелхиседеку и тем самым обрел опыт, связанный с воплощением бессмертного существа, живущего во плоти смертного человека. Случай этот произошел 11 февраля 2 г. до н.э. Иисус осознал появление у себя божественного Настройщика не более, чем миллионы и миллионы других детей, которым до и после этого дня посылались Настройщики Мысли, чтобы постоянно пребывать в их умах и работать для эвентуального окончательного одухотворения их и вечного существования их развивающихся бессмертных душ.
\vs p123 2:2 С этого февральского дня прямое и личное наблюдение Вселенских Правителей за целостностью младенческого воплощения Михаила было завершено. С этого времени и на протяжении всей жизни в образе человеческом Иисуса должен был опекать постоянно пребывающий с ним Настройщик Мысли и присоединяющиеся к нему ангелы\hyp{}хранительницы, а также время от времени помогали срединники, которым было поручено выполнение определенных обязанностей согласно указаниям их планетарных руководителей.
\vs p123 2:3 \P\ В августе того же года Иисусу исполнилось пять лет, и мы будем считать этот год пятым (календарным) годом его жизни. В этом, 2 году до н.э., когда оставалось немногим больше месяца до того дня, когда Иисусу исполнялось пять лет, он был совершенно счастлив, потому что на свет появилась его сестра Мириам, которая родилась ночью 11 июля. Вечером следующего дня Иисус долго беседовал с отцом о том, каким образом разнообразные группы живых существ рождаются на свет как отдельные индивидуумы. Самое ценное в начальном образовании Иисуса было получено от родителей в виде ответов на вдумчивые и проницательные вопросы. Иосиф всегда полностью исполнял свой долг и уделял много сил и времени, отвечая на многочисленные вопросы мальчика. С пятилетнего возраста и до того, как ему исполнилось десять, Иисус состоял из одних непрерывных вопросительных знаков. Хотя Иосиф и Мария не всегда могли ответить на его вопросы, они никогда не отказывались подробно обсудить их и любыми другими способами помочь ему найти удовлетворительные решения проблем, которые занимали его живой ум.
\vs p123 2:4 После возвращения в Назарет домашнее хозяйство семьи требовало больших хлопот, и Иосиф был очень занят строительством новой мастерской и налаживанием своего дела заново. Он был настолько поглощен делами, что не нашел даже времени сделать колыбель для Иакова, но исправил упущенное задолго до появления Мириам, так что у ней была детская очень удобная кроватка, в которой она покоилась, в то время как вся семья любовалась малюткой. И маленький Иисус охотно участвовал во всех этих естественных и обычных семейных делах. Он очень любил своего маленького брата и малютку\hyp{}сестру и помогал Марии заботиться о них.
\vs p123 2:5 В нееврейском мире тех дней нашлось бы немного домов, где ребенку могли бы дать лучшее интеллектуальное, нравственное и религиозное обучение, чем в еврейском доме в Галилее. У этих евреев была систематизированная программа воспитания и образования детей. Они делили жизнь ребенка на семь периодов:
\vs p123 2:6 \ublistelem{1.}\bibnobreakspace Новорожденный ребенок, с первого по восьмой день.
\vs p123 2:7 \ublistelem{2.}\bibnobreakspace Грудной младенец.
\vs p123 2:8 \ublistelem{3.}\bibnobreakspace Pебенок, отнятый от груди.
\vs p123 2:9 \ublistelem{4.}\bibnobreakspace Период зависимости от матери, продолжающийся до конца пятого года.
\vs p123 2:10 \ublistelem{5.}\bibnobreakspace Начало независимости ребенка и, для сыновей, периода, когда ответственность за образование ребенка берет на себя отец.
\vs p123 2:11 \ublistelem{6.}\bibnobreakspace Мальчики и девочки\hyp{}подростки.
\vs p123 2:12 \ublistelem{7.}\bibnobreakspace Молодые мужчины и женщины.
\vs p123 2:13 \P\ По обычаям евреев Галилеи, мать несла ответственность за обучение ребенка, пока ему не исполнялось пять лет, а затем, если ребенок был мальчиком, ответственность переходила к отцу. Таким образом, в этом году Иисус вступил в пятый период развития ребенка по обычаям галилейских евреев, и, соответственно, 21 августа 2 года до н.э. Мария официально передала его для дальнейшего руководства Иосифу.
\vs p123 2:14 Хотя теперь Иосиф был непосредственно ответственен за интеллектуальное и религиозное образование Иисуса, мать продолжала интересоваться его домашним обучением. Она учила его разбираться и ухаживать за виноградом и цветами, которые росли у стен сада, со всех сторон окружавшего приусадебный участок. Кроме того, на крыше дома (в летней спальне) она приготовила несколько невысоких ящиков с песком, на которых Иисус чертил карты и много тренировался в арамейском и греческом письме и, позже, в письме на древнееврейском, и со временем научиться бегло читать, писать и говорить на всех трех языках.
\vs p123 2:15 Иисус оказался почти совершенным ребенком в физическом отношении и продолжал нормально развиваться умственно и эмоционально. К концу этого пятого (календарного) года он перенес легкое пищеварительное расстройство, первое небольшое заболевание.
\vs p123 2:16 Хотя Иосиф и Мария часто говорили о будущем старшего ребенка, будь вы там, вы бы всего лишь увидели, как растет нормальный для того времени и места здоровый, беззаботный, но чрезвычайно любознательный ребенок.
\usection{3. События шестого года жизни (1 г. до н.э.)}
\vs p123 3:1 С помощью матери Иисус уже овладел галилейским диалектом арамейского языка; а теперь отец начал обучать его греческому. Мария немного говорила по\hyp{}гречески, а Иосиф говорил и по арамейски, и по\hyp{}гречески свободно. Учебником для изучения греческого языка служил перевод иудейского Писания --- полный текст Закона и Пророков, включая Псалмы, --- который был подарен им, когда они покидали Египет. Во всем Назарете было всего две копии перевода на греческий всего Писания, и то, что одним из них владела семья плотника, делало дом Иосифа местом, где собиралось много народу, и это позволило Иисусу, пока он рос, столкнуться с множеством людей, серьезно изучавших и искренне стремившихся познать истину. Еще до конца года эта бесценная рукопись была передана на хранение Иисусу, которому в шестой день рождения было сказано, что священная книга была подарена ему друзьями и родственниками из Александрии. И вскоре он уже мог без труда читать ее.
\vs p123 3:2 \P\ Первое серьезное потрясение, выпавшее на долю Иисуса в детстве, постигло его, когда ему вот\hyp{}вот должно было исполниться шесть лет. До этого Иисусу казалось, что отец --- или, по крайней мере, его отец и мать вместе --- знают все. Представьте же себе удивление этого любознательного мальчика, когда он спросил отца о причине только что произошедшего небольшого землетрясения, и Иосиф ответил: «Сын мой, я действительно не знаю». Так началось это длительное и приводившее Иисуса в замешательство разрушение иллюзий, в процессе которого он обнаружил, что земные родители вовсе не были самыми мудрыми и всезнающими.
\vs p123 3:3 Первой мыслью Иосифа было ответить Иисусу, что землетрясение вызвал Бог, но минутное размышление убедило его, что такой ответ немедленно повлечет за собой последующие и еще более трудные вопросы. Даже когда Иисус был совсем ребенком, было очень трудно отвечать на его вопросы о физических или социальных явлениях, бездумно говоря ему, что за них отвечают Бог или дьявол. В соответствии с господствовавшей среди евреев верой, Иисус был готов охотно принять учение о добрых и злых духах как возможное объяснение психических и духовных явлений мира, но он рано начал сомневаться в том, что такие невидимые влияния отвечают за все физические события в естественном мире.
\vs p123 3:4 \P\ Перед тем как Иисусу исполнилось шесть лет, в начале 1 года до н.э., Захария, Елизавета и их сын Иоанн приехали навестить семью из Назарета. Иисус и Иоанн приятно провели время во время этой первой на их памяти встречи. Хотя гости смогли пробыть в Назарете всего несколько дней, родители успели поговорить о многом, в том числе и о планах на будущее сыновей. Пока они проводили время таким образом, мальчики играли в кубики в песке на крыше дома и развлекались другим принятым среди мальчиков образом.
\vs p123 3:5 \P\ Встретившись с Иоанном, который прибыл из окрестностей Иерусалима, Иисус начал проявлять необычайный интерес к истории Израиля и подробно расспрашивать о значении обрядов Субботы, о проповедях в синагоге и круге ежегодных праздников. Отец объяснял ему значение каждого из этих времен года. Первый --- праздник освещения огней, длился восемь дней; в первую ночь зажигалась одна свеча, и в каждую последующую ночь к ней добавлялось еще по одной; это был праздник в честь освящения храма после восстановления Моисеевых служб Иудой Маккавеем. Следующий праздник --- Пурим, наступал ранней весной, праздник, посвященный Эсфири и освобождению благодаря ей Израиля. Затем следовала торжественная Пасха, которую все взрослые, если была хоть какая\hyp{}то возможность, праздновали в Иерусалиме, в то время как дети, остававшиеся дома, должны были помнить, что целую неделю нельзя есть никакого дрожжевого хлеба. Потом приходил праздник первых фруктов, праздник урожая; и наконец, самый торжественный из всех, праздник нового года, день искупления. Хотя юному разуму Иисуса было нелегко понять некоторые из этих праздников и обрядов, он серьезно размышлял о них и затем всецело насладился радостью праздника кущей, времени ежегодного отдыха всего иудейского народа, когда евреи устраивали ночевки под открытым небом в крытых листвой шалашах и предавались радостям и удовольствиям.
\vs p123 3:6 \P\ В этом году Иосиф и Мария испытали сильное беспокойство по поводу молитв Иисуса. Он настаивал на том, что разговаривает со своим небесным Отцом почти так же, как с Иосифом, своим земным отцом. Это отклонение от более торжественных и благоговейных форм общения с Божеством несколько расстраивало родителей, особенно мать, но его невозможно было убедить измениться; он произносил молитвы точно так, как его учили, после чего настаивал на том, чтобы «немножко поговорить со своим Отцом на небе».
\vs p123 3:7 В июне того года Иосиф отдал свою мастерскую братьям и официально приступил к работе в качестве строителя. Еще до конца года доход семьи увеличился более чем втрое. С тех пор и до самой смерти Иосифа семья из Назарета больше никогда не терпела нужды. Семья становилась все больше и больше, и они тратили много денег на дополнительное образование детей и поездки, но растущие доходы Иосифа всегда опережали растущие расходы.
\vs p123 3:8 В течение нескольких следующих лет Иосиф выполнил значительные работы в Кане, Вифлееме (Галилейском), Магдале, Наине, Сефорисе, Капернауме и Ендоре; он также много строил в самом Назарете и в его окрестностях. По мере того как Иаков подрастал и мог уже помогать матери в домашней работе и уходе за младшими детьми, Иисус начал часто совершать поездки с отцом по этим близлежащим городам и деревням. Иисус был внимательным наблюдателем и извлекал много практических знаний из этих поездок. Он неутомимо накапливал знания о человеке и о том, как он живет на земле.
\vs p123 3:9 \P\ В этом году Иисус научился чрезвычайно успешно соизмерять свои сильные чувства и мощные порывы с требованиями семейного общежития и домашней дисциплины. Мария была любящей матерью, но требовала строгой дисциплины. Тем не менее, во многих отношениях Иосиф обладал большим влиянием на Иисуса, так как имел обыкновение садиться рядом с мальчиком и подробно объяснять ему истинные и главные причины необходимости усмирять свои личные желания ради благополучия и спокойствия всей семьи. После того, как ситуация становилась ясна Иисусу, он всегда разумно и охотно согласовывал свои действия с желаниями родителей и требованиями семьи.
\vs p123 3:10 \P\ Большую часть своего свободного времени --- когда матери не требовалась его помощь по дому --- Иисус проводил днем за изучением цветов и растений, а ночью --- звезд. Проявилась беспокоившая родных привычка лежать на спине с любопытством разглядывая звездное небо гораздо позже того часа, когда, по распорядку Назаретского семейства, ему давно уже полагалось идти спать.
\usection{4. Седьмой год (1 г. н.э.)}
\vs p123 4:1 Этот год в жизни Иисуса был поистине полон событиями. В начале января в Галилее разразилась великая снежная буря. Толщина снега достигала двух футов, это был сильнейший снегопад, который Иисусу довелось увидеть когда\hyp{}либо в своей жизни, и один из сильнейших в Назарете за сто лет.
\vs p123 4:2 Игры еврейских детей времен Иисуса были довольно ограниченными; чаще всего в играх они подражали тем серьезным вещам, которые видели в жизни взрослых. Они много играли в свадьбы и похороны, в церемонии, которые так часто видели и которые были такими захватывающими. Они танцевали и пели, но у них было мало организованных игр, которые так нравились детям в более поздние времена.
\vs p123 4:3 Иисус, вместе с соседским мальчиком и, позже, со своим братом Иаковом, обожал играть в дальнем углу домашней плотницкой мастерской, где они с удовольствием возились с деревянными стружками и обрезками. Иисусу всегда было трудно понять, чем плохи игры, запрещенные в Субботу, но он никогда не перечил воле своих родителей. У него было хорошее чувство юмора и способность к играм, которые не имели возможности проявиться в условиях его времени и поколения, но --- до четырнадцати лет он в основном бывал весел и беззаботен.
\vs p123 4:4 На крыше загона для скота, примыкавшего к дому, у Марии была устроена голубятня, и доходы от продажи голубей они использовали как особый благотворительный фонд, которым после того, как из него вычиталась десятая часть, отдаваемая чиновнику из синагоги, распоряжался Иисус.
\vs p123 4:5 \P\ Единственным серьезным происшествием, которое случилось с Иисусом за это время, было падение с каменной лестницы на заднем дворе, которая вела к крытой холстом спальне. Это случилось в июле во время неожиданно налетевшей с востока пыльной бури. Горячие ветры, которые несли тучи сухого песка, обычно дули в сезон дождей, особенно в марте и апреле. Для июля такой ураган был необычен. Когда разразилась буря, Иисус, как это часто бывало, играл на крыше дома --- обычном месте его игр в сухое время года. Он был ослеплен песком, когда спускался с лестницы, и упал. После этого несчастного случая Иосиф по обеим сторонам лестницы приделал перила.
\vs p123 4:6 Этот несчастный случай невозможно было предотвратить. Нельзя было винить за пренебрежение своими обязанностями его земных хранителей\hyp{}срединников, одного первого рода и одного второго рода, которые должны были оберегать мальчика; невозможно также винить в этом и ангела\hyp{}хранительницу. Этого просто невозможно было избежать. Но это небольшое происшествие, случившееся в то время, когда Иосиф был в Ендоре, породило в сердце Марии столь сильное беспокойство, что она несколько месяцев неразумно пыталась не отпускать от себя Иисуса ни на шаг.
\vs p123 4:7 Небесные существа не могут по своей воле вмешиваться в происшествия материального плана, в обычные явления физической природы. При обычных обстоятельствах только срединники могут вмешиваться в материальные условия, чтобы защитить личность мужчин и женщин с определенной миссией, и даже в особых обстоятельствах эти существа могут действовать таким образом, только исполняя специальные поручения своих руководителей.
\vs p123 4:8 Это был всего один эпизод из ряда ему подобных небольших происшествий, которые впоследствии случались с этим любознательным и смелым юношей. Если вы вообразите нормальные годы детства и юности любого энергичного мальчика, вы составите себе представление о юношеском развитии Иисуса и поймете, сколько беспокойства он причинял родителям, особенно матери.
\vs p123 4:9 \P\ Четвертый ребенок в назаретском семействе --- мальчик Иосиф родился в среду утром, 16 марта 1 г. н.э.
\usection{5. Школьные годы в Назарете}
\vs p123 5:1 Теперь Иисусу было уже семь лет --- возраст, когда еврейским детям полагалось начать свое официальное образование в школе при синагоге. Соответственно, в августе этого года он вступил в свою наполненную событиями школьную жизнь в Назарете. Мальчик уже бегло читал, писал и говорил на двух языках --- арамейском и греческом. Теперь пред ним стояла новая задача --- научиться читать, писать и говорить на древнееврейском. И он очень стремился к этой новой школьной жизни, которая ему предстояла.
\vs p123 5:2 Три года --- до того, как ему исполнилось десять лет --- Иисус посещал начальную школу при Назаретской синагоге. В течение этих трех лет он изучал основы Книги Закона, в том виде, как она была записана на древнееврейском языке. Следующие три года он учился в средней школе и, повторяя вслух, заучивал наизусть более труднопостижимые учения священного закона. Он окончил синагогальную школу когда ему было около тринадцати лет, и руководители синагоги передали его родителям как образованного «сына заповеди» --- с этого момента ставшего полноправным гражданином государства Израилева, из чего следовало, что он может присутствовать на празднике Пасхи в Иерусалиме; соответственно, в том году он впервые провел Пасху в Иерусалиме вместе с отцом и матерью.
\vs p123 5:3 \P\ В Назарете ученики сидели полукругом на полу, а их учитель, хазан, служитель синагоги, сидел лицом к ним. Начав с Книги Левит, они переходили к изучению других книг закона, за которыми следовало изучение книг Пророков и Псалмов. В Назаретской синагоге был полный текст Священного писания на древнееврейском языке. До двенадцати лет мальчики не изучали ничего, кроме Священного писания. В летние месяцы в школе они проводили намного меньше времени.
\vs p123 5:4 Иисус рано овладел в совершенстве древнееврейским, и если в Назарете не оказывалось никаких важных гостей, его, как молодого человека, часто просили читать Священное писание верующим, которые собирались в синагоге на субботние службы.
\vs p123 5:5 Конечно же в таких школах при синагогах не было учебников. Во время урока хазан произносил стих, а ученики хором повторяли вслух за ним. Когда ученики получали доступ к текстам книг закона, они учили уроки, читая вслух и постоянно повторяя прочитанное.
\vs p123 5:6 \P\ Кроме того в придачу к формальному образованию, Иисус начал беседовать с людьми со всех концов света, так как жители из многих стран проходили через мастерскую его отца. Пока Иисус рос, он свободно общался с людьми из караванов, которые останавливались у источника для отдыха и еды. Он бегло говорил по\hyp{}гречески, и ему было совсем не трудно беседовать с большинством проводников караванов и путешественников.
\vs p123 5:7 Назарет был местом остановки и перекрестком караванных путей, и население города было в основном нееврейским; в то же время он был широко известен как центр вольной трактовки традиционного иудейского закона. В Галилее евреи общались с неевреями более свободно, чем принято было в Иудее. И среди всех городов Галилеи евреи из Назарета наиболее либерально трактовали социальные ограничения, вызванные страхом оскверниться вследствие контакта с неевреями. Поэтому в Иудее возникла поговорка: «Из Назарета может ли быть что доброе?».
\vs p123 5:8 Уроки морали и духовную культуру Иисус получил преимущественно в собственном доме. Интеллектуальное и теологическое образование он почерпнул в основном от хазана. Но свое истинное образование --- то, которое вооружает ум и сердце для настоящих испытаний, коим является борьба с трудными жизненными проблемами, --- он приобрел, общаясь с людьми. Именно тесное общение с людьми, молодыми и старыми, евреями и неевреями, дало ему возможность узнать род человеческий. Иисус достиг высот образованности в своем знании людей и преданной любви к ним.
\vs p123 5:9 \P\ За время учебы в синагоге Иисус стал блестящим учеником, обладавшим большими преимуществами благодаря хорошему знанию трех языков. Назаретский учитель, по случаю окончания Иисусом школы, признался Иосифу, что, кажется, сам «узнал из пытливых вопросов Иисуса больше», чем «был в состоянии научить мальчика».
\vs p123 5:10 За годы учебы Иисус многое узнал и почерпнул большое вдохновение из регулярных субботних служб в синагоге. Было принято приглашать выдающихся гостей, останавливавшихся в Назарете, в субботу выступать в синагоге. По мере того как Иисус рос, ему доводилось слышать многих великих мыслителей со всего еврейского мира, излагавших свои взгляды, и также многих людей, которых едва ли можно было назвать ортодоксальными евреями, так как синагога Назарета была передовым и либеральным центром иудейской мысли и культуры.
\vs p123 5:11 Поступая в семь лет в школу (в то время евреи только что ввели закон об обязательном образовании), ученики по обычаю выбирали себе «текст на день рождения», нечто вроде золотого правила, которым руководствовались в период занятий и о котором часто говорили во время окончания школы в тринадцать лет. Текст, выбранный Иисусом, был из пророка Исаии: «Дух Господа Бога на мне, ибо он помазал меня; он послал меня принести благую весть кротким, уврачевать сокрушенных сердцем, провозгласить свободу плененным и освободить духовных пленников».
\vs p123 5:12 \P\ Назарет был одним из двадцати четырех священнических центров еврейского народа. Галилейское священство было более либеральным в толковании традиционных законов, чем книжники и раввины из Иудеи. И в Назарете также более либерально относились к соблюдению субботы. Поэтому Иосиф имел обыкновение в субботу во второй половине дня брать Иисуса с собой на прогулку, во время которой они любили взбираться неподалеку от дома на высокий холм, с которого открывался во все стороны вид на всю Галилею. В ясные дни на северо\hyp{}западе был виден длинный, сбегающий к морю склон горы Кармил, и много раз Иисус слышал, как его отец рассказывает историю Илии, который был одним из первых в длинном ряду иудейских пророков, который порицал Ахава и разоблачил жрецов Ваала. На севере в чудном великолепии возвышалась снежная вершина горы Ермон; ее верхние склоны, достигавшие высоты почти 3000 футов, сверкали вечными снегами. Далеко на западе можно было различить долину реки Иордан, а еще дальше за ней лежали скалистые холмы Моава. А на юге и востоке, когда солнце сверкало на мраморных стенах, они могли увидеть греко\hyp{}римские города Десятиградия, с их амфитеатрами и роскошными храмами. А когда они задерживались до заката, на западе можно было различить корабли, плывущие по далекому Средиземному морю.
\vs p123 5:13 Со всех четырех сторон Иисус видел вереницы караванов, направлявшихся в Назарет и из него, и на юге мог окинуть взглядом широкую и плодородную Ездрилонскую равнину, простиравшуюся до горы Гелвуй и Самарии.
\vs p123 5:14 Когда они не взбирались на вершины, чтобы полюбоваться открывающейся панорамой, то бродили по окрестностям Назарета и изучали природу в ее различных проявлениях в череде времен года. Самое раннее образование Иисуса, кроме того, которое дал ему родной дом, он получил благодаря благоговейному и проникновенному общению с природой.
\vs p123 5:15 \P\ Ему еще не было восьми лет, а его уже знали все матери и молодые женщины Назарета, которые встречали его и беседовали с ним у протекавшего недалеко от их дома источника, где общался и обсуждал новости весь город. В этом году Иисус научился доить корову, которую они держали, и ухаживать за другими животными. За этот и следующий год он научился также делать сыр и ткать. Когда ему было десять лет, Иисус прекрасно управлял ткацким станом. Примерно в это время он и соседский мальчик Иаков очень подружились с гончаром, работавшим рядом с источником; когда они смотрели, как ловкие руки Натана мнут глину на гончарном круге, то всякий раз оба решали, что станут гончарами, когда вырастут. Натану очень нравились мальчики, и он часто давал им глину, чтобы поиграть, и пытался пробудить в них творческое воображение, предлагая им посоревноваться в изготовлении различных предметов и животных.
\usection{6. Восьмой год его жизни (2 г. н.э.)}
\vs p123 6:1 Этот год в школе был очень интересным. Хотя Иисус не был необычным учеником, он прилежно учился и входил в треть лучших учеников класса; он так хорошо учился, что каждый месяц в течение недели освобождался от посещения школы. Эту неделю он проводил обычно или со своим дядей\hyp{}рыбаком на берегу Галилейского моря недалеко от Магдалы, либо в хозяйстве своего другого дяди (брата матери) в пяти милях к югу от Назарета.
\vs p123 6:2 Хотя мать стала чрезмерно волноваться о его здоровье и безопасности, постепенно она смирилась с этими отлучками из дому. Все дяди и тети очень любили Иисуса, и между ними возникало веселое соревнование за право радоваться его обществу во время этих ежемесячных посещений в течение этого года и следующих за ним лет. Его первое (после младенчества) недельное пребывание у дяди произошло в январе того года; первая неделя рыбной ловли на Галилейском море пришлась на май.
\vs p123 6:3 Примерно в это время Иисус познакомился с учителем математики из Дамаска и, узнав от него некоторые новые математические приемы, в течение нескольких лет много времени уделял математике. Он развил в себе тонкое понимание чисел, расстояний и пропорций.
\vs p123 6:4 Иисус стал получать большое удовольствие от общения со своим братом Иаковом, и в конце этого года начал обучать его алфавиту.
\vs p123 6:5 В этом году Иисус договорился о том, что будет платить молочными продуктами за уроки игры на арфе. Он необычайно любил все, относящееся к музыке. Позже он многое сделал для того, чтобы пробудить в своих друзьях интерес к вокальной музыке. К одиннадцати годам он стал искусным арфистом и очень любил развлекать своих родных и друзей необычным исполнением и талантливыми импровизациями.
\vs p123 6:6 Хотя Иисус продолжал делать в школе завидные успехи, все шло не слишком гладко как для его родителей, так и для учителей. Он упорно продолжал задавать смущавшие всех вопросы как о науке, так и о религии, особенно о географии и астрономии. Он особенно настойчиво пытался выяснить, почему в Палестине два времени года --- сезон дождей и сухой сезон. Он снова и снова искал объяснения тому, почему столь большая разница между температурой в Назарете и в долине Иордана. Он просто никогда не переставал задавать такие умные, но приводяшие в замешательство вопросы.
\vs p123 6:7 \P\ Его третий брат, Симон, родился в пятницу вечером, 14 апреля этого, второго, года н.э.
\vs p123 6:8 \P\ В феврале один из преподавателей Иерусалимской академии раввинов, Нахор, приехал посмотреть на Иисуса, перед этим побывав с аналогичным заданием в доме Захарии возле Иерусалима. Мысль поехать в Назарет подал ему отец Иоанна. Хотя поначалу он был несколько шокирован откровенностью Иисуса и его необычным подходом к вопросам религии, он отнес это за счет удаленности Галилеи от центров иудейской образованности и культуры и посоветовал Иосифу и Марии позволить ему взять Иисуса с собой в Иерусалим, где он мог бы пользоваться преимуществами образования и обучения в центре еврейской культуры. Он почти убедил Марию согласиться; она была уверена, что ее старший сын должен стать Мессией, избавителем евреев; Иосиф колебался; он тоже был уверен, что Иисус, когда вырастет, станет человеком, судьба которого предопределена, но не был убежден, что понимает, в чем именно будет состоять это предопределение. Но он никогда по\hyp{}настоящему не ставил под сомнение то, что сыну предстоит исполнить некую великую миссию на земле. Чем больше он размышлял над предложением Нахора, тем сильнее сомневался в разумности поездки в Иерусалим.
\vs p123 6:9 Так как мнения Иосифа и Марии разошлись, Нахор попросил разрешения изложить суть дела Иисусу. Иисус внимательно выслушал его, поговорил с Иосифом, с Марией, с соседом, каменщиком Иаковом, сын которого был его любимым товарищем по играм, а потом, два дня спустя, ответил, что поскольку мнения родителей и советчиков так сильно различаются, а он не чувствует себя достаточно уверенным, чтобы взять на себя ответственность за такое решение, не склоняясь определенно ни в ту, ни в другую сторону, то ввиду всей этой ситуации он в конце концов решил «поговорить с моим Отцом, который на небесах»; и хотя он не был полностью уверен в ответе, он посчитал, что ему лучше оставаться дома «с моим отцом и моей матерью», и добавил, что «они, которые так сильно любят меня, смогут сделать для меня больше и руководить мной более успешно, чем чужие люди, которые могут только видеть мое тело и наблюдать мой ум, но едва ли могут по\hyp{}настоящему знать меня». Они все изумились, и Нахор вернулся обратно в Иерусалим. С тех пор прошло много лет, прежде чем вопрос об отъезде Иисуса из дома возник снова.
