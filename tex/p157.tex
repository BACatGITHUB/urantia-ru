\upaper{157}{В Кесарии Филипповой}
\vs p157 0:1 Перед тем, как ненадолго отправиться с двенадцатью апостолами в окрестности Кесарии Филипповой, Иисус известил через вестников Давида о своем намерении побывать в воскресенье 7 августа в Капернауме, чтобы встретиться со своей семьей. Как ранее было договорено, эта встреча должна была произойти в лодочной мастерской Зеведеев. Давид Зеведей условился с Иудой, братом Иисуса, что будет присутствовать вся Назаретская семья --- Мария и все братья и сестры Иисуса, --- и Иисус отправился с Андреем и Петром на назначенную встречу. Мария и дети, безусловно, намеревались прийти на эту встречу, но получилось так, что несколько фарисеев, зная, что Иисус находится на противоположном берегу озера во владениях Филиппа, решили зайти к Марии, чтобы выведать все возможное о его местонахождении. Приход этих иерусалимских эмиссаров чрезвычайно обеспокоил Марию, и, заметив напряженность и нервозность всей семьи, они заключили, что, скорее всего, семья ожидает Иисуса. Поэтому они расположились в доме у Марии, и послали за подмогой и стали терпеливо ожидать прибытия Иисуса. И это, конечно, исключало всякую возможность для любого члена семьи отправиться на назначенную встречу с Иисусом. Несколько раз в течение дня и Иуда, и Руфь пытались ускользнуть от бдительности фарисеев и передать весть Иисусу, но тщетно.
\vs p157 0:2 Вскоре после полудня вестники Давида сообщили Иисусу, что фарисеи расположились у порога дома его матери, и поэтому он не стал далее пытаться навестить семью. Итак, снова Иисусу и его семье не по их вине не удалось встретиться.
\usection{1.\bibnobreakspace Сборщик храмовой подати}
\vs p157 1:1 Когда Иисус с Андреем и Петром находились у озера возле лодочной мастерской, их случайно встретил сборщик храмовой подати и, узнав Иисуса, отозвал Петра в сторону и сказал: <<Разве твой Учитель не платит храмовый налог?>> Петр хотел возмутиться от одной мысли о том, что Иисус должен материально поддерживать религиозную деятельность своих заклятых врагов, но, заметив странное выражение лица у сборщика подати, правильно заподозрил, что тот намеревался заманить их в ловушку из\hyp{}за отказа платить обычные полшекеля на оплату служб в Иерусалимском храме. Естественно, что Петр ответил: <<Ну, конечно же Учитель платит храмовую подать. Подожди у ворот, и я сейчас же вернусь с податью>>.
\vs p157 1:2 Но Петр сказал это, не подумав. Их деньги хранились у Иуды, а он был на другой стороне озера. Ни он, ни его брат, ни Иисус не взяли с собой никаких денег. А зная, что их ищут фарисеи, они не могли пойти в Вифсаиду за деньгами. Когда Петр рассказал Иисусу про сборщика податей и про то, что пообещал ему деньги, Иисус сказал: <<Если ты пообещал, то тебе следует заплатить. Но каким образом ты выполнишь свое обещание? Сможешь ли ты снова стать рыбаком, чтобы сдержать свое слово? В любом случае, Петр, в данных обстоятельствах нам хорошо бы заплатить налог. Не будем давать этим людям никаких поводов для нападок на нас. Мы подождем здесь, а ты отправляйся на лодке и поймай в сети рыбу, а продав ее вон на том рынке, заплати сборщику подать за всех нас троих>>.
\vs p157 1:3 Все это было услышано тайным вестником Давида, стоявшим поблизости, он дал знак одному из друзей, ловившему рыбу неподалеку от берега, чтобы тот быстро подплыл. Когда Петр уже собрался отплывать ловить рыбу, этот посланник и его друг рыбак подарили ему несколько больших корзин рыбы и помогли отнести их к находившемуся поблизости торговцу, который купил улов, заплатив довольно хорошо, немного добавил и посланник Давида, и этого хватило, чтобы уплатить храмовую подать за всех троих. Сборщик принял подать, не стал штрафовать их за несвоевременную уплату, поскольку какое\hyp{}то время их не было в Галилеи.
\vs p157 1:4 Нет ничего странного в предании о том, как Петр поймал рыбу с шекелем во рту. В те дни ходило много историй о сокровищах, найденных во рту у рыб; такие легенды о почти невероятных событиях встречались повсеместно. Так что, когда Петр расстался с ними и направился к лодке, Иисус, полушутя, заметил: <<Странно, что сыновья царя должны платить подать; обычно, наоборот, прочий люд облагается налогами на содержание двора; но нам не следует давать повод властям придраться к нам. Иди же! может быть, ты поймаешь рыбу с шекелем во рту>>. Не удивительно, что после таких слов Иисуса и столь быстрого возвращения Петра с храмовым налогом этот случай позже стал восприниматься как чудо, что и описано в Евангелие от Матфея.
\vs p157 1:5 Иисус с Андреем и Петром ждали у берега почти до захода солнца. Вестники сообщили, что дом Марии все еще находится под наблюдением; поэтому, когда стемнело, трое ожидавших мужчин сели в лодку и медленно поплыли на веслах в сторону восточного берега Галилейского озера.
\usection{2.\bibnobreakspace В Вифсаиде\hyp{}Юлия}
\vs p157 2:1 В понедельник 8 августа, когда Иисус и двенадцать апостолов расположились лагерем в Магаданском лесу возле Вифсаиды\hyp{}Юлия, более сотни верующих, евангелисты, женский корпус и прочие заинтересованные в установлении царства, пришли из Капернаума на совет. И многие фарисеи, узнав, что Иисус находится здесь, тоже пришли. К этому времени некоторые из саддукеев объединились с фарисеями, чтобы заманить Иисуса в ловушку. Прежде, чем пойти на закрытое совещание с верующими, Иисус провел открытую встречу с народом, на которой присутствовали фарисеи; последние перебивали его и всячески пытались помешать. Глава подстрекателей сказал: <<Учитель, мы хотели бы, чтобы ты подтвердил нам знамением свое право учить, и тогда, если это произойдет, все люди будут знать, что ты послан Богом>>. И Иисус ответил им: <<Вечером, говорите вы, будет хорошая погода, потому что небо багряное; утром будет скверная погода, потому что небо багряное и хмурое. Когда вы видите поднимающуюся на западе тучу, вы говорите, что пойдет дождь; когда ветер дует с юга, вы говорите, что наступит палящий зной. Как же получается, что вы так хорошо умеете понимать приметы неба, но полностью неспособны распознавать знамения времени? Тем, кто хочет знать истину, знамение уже дано; но злонамеренному и лицемерному поколению не будет дано никакого знамения>>.
\vs p157 2:2 \P\ Сказав так, Иисус удалился и приготовился к вечерней встрече со своими последователями. На ней было решено совместно отправиться с проповедями по всем городам и селениям Десятиградия, после того, как Иисус и двенадцать апостолов вернутся из своей задуманной поездки в Кесарию Филипову. Учитель принимал участие в планировании путешествия в Десятиградие и перед тем, как всех отпустить, сказал: <<Говорю вам, остерегайтесь фарисейской и саддукейской закваски>>. Да не обманет вас их большая ученость и тщательное соблюдение религиозных обрядов. Пусть вас заботит только дух живой истины и сила истинной религии. Спасет вас не страх перед мертвой религии, а ваша вера в живой опыт постижения духовных реальностей царства. Не давайте предрассудкам ослеплять вас и страху парализовать вас. И не допускайте, чтобы уважение к традициям так затмило ваш разум, что ваши глаза перестанут видеть, а уши перестанут слышать. Цель истинной религии не просто принести мир, но обеспечить прогресс. И не может быть мира в сердце и прогресса в сознании, если вы не полюбите всем сердцем истину, идеалы вечных реалий. Перед вами встают вопросы жизни и смерти --- преходящие греховные удовольствия в противоположность вечным праведным реалиям. Уже сейчас по мере того, как вы входите в новую жизнь с верой и надеждой, вам следует начинать искать избавления от оков страха и сомнения. И когда в вашей душе появляется желание служить своим собратьям, не подавляйте его; когда в вашем сердце возникает чувство любви к ближнему, дайте выход этому чувству, мудро помогая своим собратьям, когда они в этом по\hyp{}настоящему нуждаются.
\usection{3.\bibnobreakspace Признание Петра}
\vs p157 3:1 Во вторник рано утром Иисус и двенадцать апостолов отправились из Магаданского леса в Кесарию Филиппову, столицу владений тетрарха Филиппа. Кесария Филиппова была расположена в необычайной красивой местности. Это была чудесная долина между живописными холмами, где Иордан вытекал из подземной пещеры. К северу отчетливо были видны очертания горы Ермон, а к югу с холмов открывался великолепный вид на верховья Иордана и Галилейское озеро.
\vs p157 3:2 Иисус был на горе Ермон когда еще только начинал заниматься делами царства, и теперь, когда он приближался к конечному периоду своей деятельности, он хотел вернуться на эту гору испытаний и торжества, где, как он надеялся, апостолы могли обрести новое видение своих обязанностей и новую силу в преддверии тяжелых времен. По дороге, примерно тогда, когда они шли южнее вод Мером, апостолы стали обсуждать свою недавнюю деятельность в Финикии и прочих местах и вспоминать, как принималась их весть и кем разные народы считали их Учителя.
\vs p157 3:3 Когда они остановились пообедать, Иисус вдруг впервые за все время поставил перед двенадцатью апостолами вопрос относительно его самого. Он задал такой неожиданный вопрос: <<Кто я такой, по словам людей?>>
\vs p157 3:4 \P\ Иисус многие месяцы учил этих апостолов тому, что касалось природы и характера царства небесного, и он хорошо знал, что пришло время начать учить их тому, что касается его собственной природы и его личных отношений с царством. И теперь, когда они расположились под тутовыми деревьями, Учитель собрался провести одно из самых важных занятий за весь долгий период общения с избранными апостолами.
\vs p157 3:5 \P\ Больше половины апостолов попытались ответить на вопрос Иисуса. Они сказали ему, что все, кто его знали, считали его пророком или необыкновенным человеком; что даже его враги чрезвычайно боялись его, объясняя его силу связью с князем тьмы. Они сказали ему, что некоторые люди в Иудее и Самарии, которые не встречались с ним лично, верили, что он был Иоанном Крестителем, восставшим из мертвых. Петр объяснил, что разные люди в разное время сравнивали его с Моисеем, Илией, Исаией и Иеремией. Выслушав все это, Иисус поднялся на ноги и, глядя на двенадцать апостолов, сидящих вокруг него полукругом, решительно указал на них широким движением руки и спросил: <<Но что говорите вы: кто я?>> Наступила напряженная тишина. Двенадцать апостолов не отрывали глаз от Учителя, и затем Симон Петр, вскочив на ноги, воскликнул: <<Ты --- Спаситель, Сын живого Бога>>. И все одиннадцать сидевших апостолов в едином порыве поднялись на ноги, показывая тем самым, что Петр выразил мнение всех.
\vs p157 3:6 Призвав их жестом садиться и по\hyp{}прежнему стоя перед ними, Иисус сказал: <<Это открыто вам моим Отцом. Настал час узнать вам правду обо мне. Но пока я повелеваю вам не рассказывать об этом ни одному человеку. Продолжим же наш путь>>.
\vs p157 3:7 Итак, они продолжили свое путешествие в Кесарию Филиппову, добрались туда поздно вечером и остановились в доме Кельса, который их ожидал. В ту ночь апостолы спали мало; казалось, они чувствовали, что в их жизни и деятельности, связанной с царством, произошло великое событие.
\usection{4.\bibnobreakspace Беседа о царстве}
\vs p157 4:1 С момента крещения Иисуса Иоанном и превращения воды в вино в Кане апостолы временами, в сущности, воспринимали его как Мессию. В отдельные моменты некоторые из них действительно верили, что он был ожидаемым Спасителем. Но едва только такие надежды пробуждались в их сердцах, Учитель разбивал их вдребезги каким\hyp{}нибудь сокрушительным словом или разочаровывающим поступком. Они давно пребывали в состоянии смятения из\hyp{}за противоречия между существующим в их умах представлении об ожидаемом Мессии и хранимым в их сердцах опытом удивительного общения с этим необыкновенным человеком.
\vs p157 4:2 В эту среду незадолго до полудня апостолы собрались в саду у Кельса на полуденную трапезу. Большую часть ночи и утром после того, как все встали, Симон Петр и Симон Зилот настоятельно пытались убедить их всем сердцем принять Учителя не просто как Мессию, но как божественного Сына живого Бога. Оба Симона практически былив полном согласии в оценке Иисуса и старались, чтобы их собратья полностью разделили их взгляды. Андрей по\hyp{}прежнему возглавлял апостолов, а его брат, Симон Петр, постепенно со всеобщего взаимного согласия становился представителем всех их.
\vs p157 4:3 Около полудня все уже сидели в саду, когда появился Учитель. На них лежала печать величавой торжественности, и все встали, когда он подошел к ним. Иисус снял напряжение той характерной для него приветливой братской улыбкой, которая всякий раз появлялась, когда его последователи начинали слишком серьезно воспринимать себя или что\hyp{}то происходящее, связанное с ними. Повелительным жестом он предложил всем сесть. Никогда больше двенадцать апостолов не приветствовали своего Учителя вставанием, когда он входил. Они поняли, что он не одобряет, когда они таким образом проявляют свое уважение.
\vs p157 4:4 После трапезы они стали обсуждать планы предстоящего путешествия в Десятиградие и вдруг Иисус посмотрел на них и сказал: <<Теперь, когда прошел целый день с тех пор, как вы согласились с утверждением Симона Петра относительно личности Сына Человеческого, я хотел бы спросить, по\hyp{}прежнему согласны с таким решением?>> Услышав это, двенадцать апостолов встали, а Симон Петр, выйдя на несколько шагов вперед, приблизился к Иисусу и сказал: >>Да, Учитель, это так. Мы верим, что ты --- Сын живого Бога<<. И Петр вместе с собратьями сел.
\vs p157 4:5 Тогда Иисус, по\hyp{}прежнему стоя, сказал двенадцати апостолам: <<Вы мои избранные посланцы, но я знаю, что в данных обстоятельствах это ваше убеждение не могло вытекать просто из человеческого знания. Это откровение духа моего Отца, в глубине ваших душ. И поэтому, благодаря осознанию живущего внутри вас духа моего Отца, вы делаете это признание, я должен возвестить, что на этом фундаменте построю братство царствия небесного. На этой скале духовной реальности я построю живой храм духовного братства, в вечных реалиях царства моего Отца. Все силы зла и сонмы греха не одолеют это человеческое братство божественного духа. И хотя дух моего Отца вовеки будет божественным проводником и наставником всех, кто связывает себя узами этого духовного братства, но сейчас вам и вашим преемникам я вручаю ключи от зримого царства --- власть над мирскими вещами --- социальными и экономическими аспектами этого сообщества мужчин и женщин, сограждан царства>>. И снова он велел им не рассказывать пока ни одному человеку, что он Сын Бога.
\vs p157 4:6 \P\ Иисус начинал твердо верить в преданность и искренность своих апостолов. Учитель понимал, что вера, выдержавшая то, через что в последнее время пришлось пройти его избранным представителям, несомненно выдержит предстоящие жестокие испытания и после явного крушения всех надежд восстанет к новому свету новой диспенсации и, тем самым, сможет двигаться вперед, освещая мир, пребывающий во тьме. В этот день Учитель обрел уверенность в вере всех своих апостолов, за исключением одного.
\vs p157 4:7 И впредь с того дня Иисус строил живой храм на вечном фундаменте его божественного сыновства, и те, кто осознанно становятся сыновьями Бога, служат кирпичиками, из которых складывается этот живой храм сыновства, возносящийся к славе и триумфу мудрости и любви вечного Отца духа.
\vs p157 4:8 \P\ И сказав это, Иисус послал до времени вечерней трапезы двенадцать апостолов поодиночке в горы искать мудрости, силы и духовного водительства. И они сделали так, как велел им Учитель.
\usection{5.\bibnobreakspace Новое понятие.}
\vs p157 5:1 Новым и чрезвычайно важным в этих словах Петра было то, что Иисус открыто и явно признавался Сыном Бога, признавалась его несомненная божественность. Со времени его крещения и свадьбы в Кане апостолы уже так или иначе считали его Мессией, но еврейское представление о национальном спасителе не подразумевало, что он должен быть \bibemph{божественным.} Евреи не учили, что Мессия произойдет от божества; он должен быть <<помазанником>>, но они в принципе не рассматривали его как <<Сына Бога>>. Во втором признании больший упор делался на \bibemph{двуединой природе,} на божественном факте, что он был Сыном Человеческим \bibemph{и} Сыном Бога, и Иисус провозгласил, что построит царство именно на основе этой великой истины о единстве человеческой природы и природы божественной.
\vs p157 5:2 Иисус стремился прожить свою жизнь на земле и выполнить миссию своего пришествия как Сын Человеческий. Его последователи были склонны видеть в нем ожидаемого Мессию. Зная, что он никак не может полностью удовлетворить этим мессианским ожиданиям, Иисус пытался так изменить их представления о Мессии, чтобы в какой\hyp{}то мере соответствовать их ожиданиям. Но теперь он осознал, что такой план вряд ли мог быть осуществим. Поэтому он смело избрал третий путь --- открыто возвестить о своей божественности, подтвердить истинность признания Петра и прямо объявить двенадцати апостолам, что он --- Сын Бога.
\vs p157 5:3 Три года Иисус возвещал, что он --- <<Сын Человеческий>>, в то время как в течение этих же трех лет все апостолы утверждали, что он --- ожидаемый еврейский Мессия. Теперь он открыл, что он --- Сын Бога и решил строить царство небесное на основе понятия о \bibemph{двуединой природе} Сына Человеческого и Сына Бога. Он решил воздержаться от дальнейших попыток убедить их, что он --- не Мессия. Теперь он решил открыть им, кто он \bibemph{есть,} и далее уже не обращать внимания на их упорное стремление видеть в нем Мессию.
\usection{6.\bibnobreakspace На следующий день}
\vs p157 6:1 Иисус и апостолы провели еще один день в доме у Кельса, ожидая, когда прибудут вестники от Давида Зеведея с деньгами. Падение популярности Иисуса у народа привели к резкому сокращению доходов. Когда они добрались до Кесарии Филипповой, казна была пуста. Матфей не хотел покидать Иисуса и его собратьев в такое время, а у него уже не было собственных денег, которые он мог бы отдать Иуде, как он это неоднократно делал раньше. Однако Давид Зеведей предвидел, что доходы, вероятнее всего, сократятся, и поэтому велел своим посланникам на пути через Иудею, Самарию и Галилею заняться сбором средств, чтобы передать изгнанным апостолам и Учителю. И поэтому посланники, прибывшие из Вифсаиды к вечеру этого дня, принесли достаточно денег, чтобы поддержать апостолов до их возвращения и чтобы они могли совершать путешествие по Десятиградию. Матфей рассчитывал к тому времени получить деньги от продажи своей последней собственности в Капернауме, договорившись, чтобы эти средства были анонимно переданы Иуде.
\vs p157 6:2 \P\ Ни у Петра, ни у остальных апостолов не было достаточно четкого представления о божественности Иисуса. Они слабо понимали, что начинается новый период в жизни их Учителя на земле --- время, когда учитель\hyp{}целитель становиться по\hyp{}новому постигаемым Мессией --- Сыном Бога. С тех пор впредь в учении Иисуса появился новый аспект. С этого момента его единственным жизненным идеалом было раскрытие Отца, а единственной его идей --- показать его вселенной воплощение той высшей мудрости, которая может быть постигнута только если ею жить. Он пришел, чтобы у нас была жизнь и жизнь эта приумножалась.
\vs p157 6:3 Иисус вступил теперь в четвертый и последний период своей человеческой жизни во плоти. Первым периодом было детство --- годы, когда он лишь смутно осознавал свое происхождение, природу и свою человеческую судьбу. Вторым --- сопровождающиеся ростом самосознания годы юности и последующего возмужания, когда он стал яснее понимать свою божественную природу и человеческую миссию. Этот период закончился его крещением и связанными с этим событиемпереживаниями и откровениями. Третий период земного опыта Учителя длился с момента крещения все годы его служения учителем и целителем вплоть до важнейшего момента признания Петра в Кесарии Филипповой. Этот третий период его земной жизни охватывал то время, когда апостолы и ближайшие последователи знали его как Сына Человеческого и рассматривали как Мессию. Четвертый и последний период его земной жизни начался здесь, в Кесарии Филипповой, и продолжался до распятия. Эта стадия его служения отмечена признанием им своей божественности и включала труды его последнего года жизни во плоти. В четвертый период, хотя большинство его последователей по\hyp{}прежнему воспринимали его как Мессию, апостолы знали его как Сын Бога. Признание Петра ознаменовало начало нового периода более полного осознания истины о его высшем служении в качестве Сына, пришедшего на Урантию ради всей вселенной, и признания этого факта, хотя бы смутно, его избранными посланцами.
\vs p157 6:4 Таким образом, Иисус в своей жизнью показал пример того, чему он учил в своей религии: духовное совершенствование путем деятельногоразвития. Он не придавал особого значения, как это делали более поздние его последователи, непрестанной борьбе между душой и телом. Вместо этого он учил, что дух легко одерживает победу и над тем, и над другим и играет действенную роль в благодатном примирении интеллектуального и инстинктивного.
\vs p157 6:5 \P\ С этого момента все учения Иисуса дополнительно приобретают новый смысл. До Кесарии Филипповой он представал как главный учитель евангелия царства. После Кесарии Филипповой он явил себя не просто как учитель, но как божественный представитель вечного Отца, который есть и центр, и обрамление этого духовного царства, и требовалось, чтобы он свершал все это как человек, как Сын Человеческий.
\vs p157 6:6 Иисус искренне пытался ввести своих последователей в духовное царство как учитель, затем как учитель\hyp{}целитель, но это не получалось. Он хорошо знал, что его земная миссия не могла удовлетворить мессианским ожиданиям еврейского народа; былые пророки изобразили Мессию таким, каким он никогда не мог быть. Он пытался создать царство Отца как Сын Человеческий, но его последователи не пошли бы дальше в этом деле. Понимая это, Иисус решил пойти навстречу верующим в него и приготовился открыто принять на себя роль пришедшего Сына Бога.
\vs p157 6:7 Соответственно, апостолы услышали много нового, когда Иисус говорил с ними в этот день в саду. И некоторые из этих заявлений показались странными даже им. Из того, что они услышали, их больше всего поразило следующее:
\vs p157 6:8 \P\ <<Впредь с этого момента, если кто\hyp{}то войдет в наше братство, пусть он примет на себя обязательства сыновства и следует за мной. А когда меня уже не будет с вами, не думайте, что мир будет относиться к вам лучше, чем он относился ко мне, вашему Учителю. Если вы любите меня, приготовьтесь доказать вашу любовь своей готовностью принести величайшую жертву>>.
\vs p157 6:9 \P\ И хорошо запомните мои слова: я пришел воззвать не к праведникам, но к грешникам. Сын Человеческий пришел не для того, чтобы ему служили, но чтобы служить и принести свою жизнь в дар всем. Объявляю вам, что я пришел искать и спасать заблудших.
\vs p157 6:10 \P\ Ни один человек в мире сейчас не видит Отца, кроме Сына, который пришел от Отца. Но если Сын вознесется, он приблизит к себе всех людей, и всякий, кто верит в эту истину двуединой природы Сына, получит жизнь вечную.
\vs p157 6:11 \P\ Мы еще не можем публично провозгласить, что Сын Человеческий есть Сын Бога, но вам это открыто; поэтому я смело говорю вам об этих тайнах. Хотя я стою перед вами в этом физическом облике, я пришел от Бога Отца. Прежде нежели был Авраам, я есмь. Я пришел от Отца в этот мир таким, каким вы знаете меня, и я объявляю вам, что должен вскоре покинуть этот мир и вернуться к делам моего Отца.
\vs p157 6:12 \P\ И может ли теперь ваша вера осознать истинность этих откровений при том, что я предупреждаю вас, что Сын Человеческий не будет соответствовать ожиданиям ваших отцов, их представлениям о Мессии? Мое царство не от мира сего. Можете ли вы поверить в истину обо мне, несмотря на то, что я не имею, где приклонить голову, хотя лисы имеют норы и птицы небесные имеют гнезда?
\vs p157 6:13 \P\ Тем не менее, говорю вам, что Отец и я --- одно. Тот, кто видел меня, видел моего Отца. Мой Отец трудится вместе со мной над всем этим, и он никогда не покинет меня в моей миссии, так же, как и я никогда не оставлю вас, когда вы вскоре отправитесь возвещать это евангелие по всему миру.
\vs p157 6:14 А теперь я взял вас, чтобы вы, пребывая какое\hyp{}то время в уединении со мной и друг с другом, смогли осознать великолепие и понять величие той жизни, к которой я призвал вас: движимого верой искания, состоящего в установлении царства моего Отца в сердцах людей, создания моего братства истинного единения с душами всех, кто верует в это евангелие>>.
\vs p157 6:15 \P\ Апостолы молча выслушали эти смелые и поразительные слова; они были потрясены. Разделившись на небольшие группы, они стали обсуждать и обдумывать слова Учителя. Они признали, что он был Сыном Бога, но не могли осознать всю значимость того, что он призвал их свершить.
\usection{7.\bibnobreakspace Совещания с Андреем}
\vs p157 7:1 В тот вечер Андрей по собственной инициативе лично обменялся с каждым из своих собратьев, кроме Иуды Искариота, мнениями в плодотворных и обнадеживающих беседах. У Андрея никогда не было таких близких отношений с Иудой, какие были с остальными апостолами, и поэтому он не придавал серьезного значения тому, что Иуда никогда не вел себя с главой апостольской группы свободно и доверительно. Но теперь Андрей был настолько обеспокоен отношением Иуды, что позже в ту же ночь, когда все апостолы крепко уснули, нашел Иисуса и изложил Учителю причины своего беспокойства. Иисус сказал: <<Нет ничего плохого, Андрей, в том, что ты пришел с этим ко мне, но лучшее, что мы сможем сделать, --- это продолжать оказывать апостолу Иуде максимальное доверие. И ничего не говори товарищам об этом разговоре со мной>>.
\vs p157 7:2 И это все, чего Андрей смог добиться от Иисуса. В отношениях между уроженцем Иудеи и его галилейскими товарищами всегда была какая\hyp{}то натянутость. Иуда был потрясен смертью Иоанна Крестителя, несколько раз он сильно обижался на упреки Учителя, был разочарован отказом Иисуса стать царем, воспринял как нечто унизительное бегство от фарисеев, был раздосадован отказом принять вызов фарисеев и дать знамение, обескуражен отказом своего Учителя явить свое могущество и к тому же в последнее время он был подавлен и удручен отсутствием денег. И Иуде очень не хватало толпы.
\vs p157 7:3 Такие же испытания и несчастья в той или иной мере оказывали подобное же влияние и на всех остальных апостолов, но они любили Иисуса. По крайней мере, они, должно быть, любили его больше, чем Иуда, поскольку дошли вместе с ним до самого горького конца.
\vs p157 7:4 Иуда, как выходец из Иудеи, воспринял как личное оскорбление недавнее предупреждение Иисуса апостолам <<опасаться влияния фарисейской закваски>>; он был склонен считать, что эти слова в завуалированном виде относятся к нему самому. Но величайшая ошибка Иуды заключалась в следующем: время от времени, когда Иисус отсылал своих апостолов молиться в одиночестве, Иуда вместо искреннего общения с духовными силами вселенной предавался мыслям, порождаемым человеческим страхом, при этом он постоянно испытывал легкое сомнение относительно миссии Иисуса и к тому же, к сожалению, был склонен таить чувство мести.
\vs p157 7:5 \P\ И теперь Иисус намеревался взять с собой апостолов на гору Ермон, посещением которой он решил ознаменовать начало четвертого периода своего земного служения в качестве Сына Бога. Некоторые из них присутствовали при его крещении в Иордане и были свидетелями начала его жизненного пути в качестве Сына Человеческого, и он хотел, чтобы некоторые из них присутствовали и услышали о его праве публично принять на себя новую роль --- Сына Бога. Поэтому утром в пятницу 12 августа Иисус сказал двенадцати апостолам: <<Сложите продукты и приготовьтесь держать путь вон к той горе, куда велит мне отправиться дух, чтобы одарить меня для окончания моей миссии на земле. И я хочу взять с собой вас, мои собратья, чтобы вы тоже могли укрепиться в преддверии тяжелых времен, когда вы будете проходить вместе со мной эти испытания>>.
