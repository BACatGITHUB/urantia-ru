\upaper{122}{Рождение и детство Иисуса}
\author{Комиссия срединников}
\vs p122 0:1 Нелегко в полной мере разъяснить множество причин, которые привели к выбору Палестины в качестве страны для пришествия Михаила, и особенно то, почему именно семья Иосифа и Марии была избрана непосредственно для воплощения Сына Бога на Урантии.
\vs p122 0:2 Изучив специально подготовленный Мелхиседеком доклад относительно статуса изолированных миров, Михаил на совете с Гавриилом, наконец, избрал Урантию планетой, на которой должно было совершиться его последнее пришествие. После того, как это решение было принято, сам Гавриил лично посетил Урантию. Изучив различные группы людей и исследовав духовные, интеллектуальные, расовые и географические особенности этого мира и его обитателей, он пришел к выводу, что именно евреи обладают относительными преимуществами, определившими выбор их в качестве народа, среди которого должно осуществиться воплощение Михаила. После того, как это решение было одобрено Михаилом, Гавриил образовал и направил на Урантию Семейную Комиссию из двенадцати избранных личностей, принадлежащих высшим чинам Вселенной, которая должна была исследовать семейный уклад евреев. После того как комиссия завершила свои труды, Гавриил явился на Урантию и получил отчет, в котором были указаны три возможные пары, в равной мере подходящие, по мнению комиссии, чтобы стать семьей для будущего воплощения Михаила.
\vs p122 0:3 Из трех предложенных пар сам Гавриил избрал Иосифа и Марию, явившись впоследствии лично Марии, чтобы принести радостную весть о том, что ей предназначено стать земной матерью обетованного младенца.
\usection{1. Иосиф и Мария}
\vs p122 1:1 Иосиф, земной отец Иисуса (Иешуа бен Иосифа), был иудеем из иудеев, хотя у него было немало и нееврейских расовых линий, которые привносили время от времени в его родословную предки с материнской стороны. Родословная отца Иисуса уходила вглубь ко временам Авраама и через этого могучего патриарха к более ранним линиям, идущим от шумеров и нодитов и, через южные племена древней голубой расы, к Андону и Фонте. Ни царь Давид, ни Соломон не являлись предками Иосифа по прямой линии, род Иосифа не происходил также по прямой линии и от Адама. Наиболее близкие по времени предки Иосифа были ремесленниками --- строителями, плотниками, каменотесами, кузнецами. Сам Иосиф был плотником, а затем --- подрядчиком. Его семья принадлежала к длинной и вполне типичной ветви высшего слоя простых людей, которая вновь и вновь являла себя, давая миру необыкновенных индивидуумов, отличившихся в связи с эволюцией религии на Урантии.
\vs p122 1:2 Мария --- земная мать Иисуса, происходила из древнего и славного рода, давшего немало выдающихся женщин в истории развития рас на Урантии. Хотя Мария была обыкновенной женщиной своего времени и своего поколения, обладавшей нормальным темпераментом, в числе ее предков были такие известные женщины, как Аннон, Тамар, Руфь, Батшеба, Анси, Хлоя, Ева, Энта и Ратта. Ни одна еврейская женщина того времени не обладала более замечательной родословной или линией предков, которая имела бы более благоприятное происхождение. Среди предков Марии так же, как и среди предков Иосифа, преобладали сильные, но вполне обыкновенные люди, среди которых время от времени появлялись личности, проявившие себя в развитии цивилизации и эволюции религии. С точки зрения расовой принадлежности, вряд ли правильно считать Марию еврейкой. По своей культуре и верованиям она была еврейкой, но в жилах ее текла смесь сирийской, хеттской, финикийской, греческой и египетской крови, ее расовое наследие было гораздо шире, чем у Иосифа.
\vs p122 1:3 Из всех супружеских пар, живших в Палестине во времена предполагаемого воплощения Михаила, Иосиф и Мария обладали самой идеальной комбинацией широких расовых связей и высших личных качеств. Таков был план Михаила --- появиться на земле как \bibemph{обычный человек,} чтобы обычные люди могли понять и принять его. Именно поэтому Гавриил избрал таких людей, как Иосиф и Мария, чтобы они стали родителями обетованного ребенка.
\usection{2. Гавриил является Елизавете}
\vs p122 2:1 Миссия Иисуса на Урантии в действительности была начата Иоанном Крестителем. Захария, отец Иоанна, принадлежал к иудейскому священству, мать же его, Елизавета, происходила из преуспевающей ветви того же большого рода, к которому принадлежала и Мария, мать Иисуса. Хотя Захария и Елизавета были женаты уже много лет, детей у них не было.
\vs p122 2:2 \pc В конце июня 8 года до н.э. спустя три месяца после замужества Иосифа и Марии Гавриил явился однажды в полдень Елизавете, как позднее явился Марии. Гавриил сказал ей:
\vs p122 2:3 «Пока твой муж, Захария, предстоит перед алтарем в Иерусалиме и собравшийся там народ молится о пришествии спасителя, я, Гавриил, пришел, чтобы возвестить тебе, что скоро ты понесешь сына, который станет предшественником этого божественного учителя, и ты назовешь сына Иоанном. Он вырастет в преданности Господу нашему Богу и, когда достигнет зрелости, будет радовать твое сердце тем, что обратит к Богу многие души, и он возвестит пришествие целителя душ твоего народа и освободителя духа всего человечества. Твоя родственница, Мария, станет матерью этого обетованного ребенка, и я явлюсь также и ей».
\vs p122 2:4 Это видение сильно испугало Елизавету. После исчезновения Гавриила она снова и снова возвращалась мыслями к тому, что произошло, подолгу размышляя о словах чудесного вестника, но никому, кроме мужа, не рассказала о своем откровении до той поры, пока не встретилась с Марией в начале февраля следующего года.
\vs p122 2:5 \pc В течение пяти месяцев, однако, Елизавета таила свой секрет даже от мужа. Услышав от жены о посещении Гавриила, Захария отнесся к рассказу с недоверием и на протяжении нескольких недель пребывал в сомнениях. Он лишь в известной степени уверовал в этот визит, когда более не оставалось сомнений, что Елизавета ожидает ребенка. Захарию сильно взволновала весть о том, что Елизавета должна стать матерью, но он не сомневался в добропорядочности своей жены, несмотря на свой весьма преклонный возраст. Лишь за шесть недель до рождения Иоанна Захарию приснился сон, который произвел на него большое впечатление и окончательно убедил в том, что Елизавета должна стать матерью судьбоносного сына, чье предназначение --- подготовить путь грядущему Мессии.
\vs p122 2:6 Гавриил явился Марии примерно в середине ноября 8 года до н.э., когда она работала в своем доме в Назарете. Позднее, когда Мария уже не сомневалась, что станет матерью, она упросила Иосифа разрешить ей отправиться в Город Иуды, который находился на холмах в четырех милях к западу от Иерусалима, чтобы навестить Елизавету. Гавриил сообщил каждой из будущих матерей о своем явлении другой. Разумеется, обе они с нетерпением ждали встречи друг с другом, хотели поделиться тем, что произошло, поговорить о будущем, которое, возможно, ожидает их детей. Мария оставалась у своей родственницы три недели. Елизавета многое сделала, чтобы укрепить веру Марии в предвещание Гавриила, и та возвратилась в свой дом, еще более посвятив себя призванию стать матерью обетованного ребенка, которого она скоро представит миру как беспомощное дитя, обычного и нормального ребенка земли.
\vs p122 2:7 \pc Иоанн родился в Городе Иуды 25 марта 7 года до н.э. Захария и Елизавета очень радовались, осознавая, что у них появился сын, как и обещал им Гавриил. И когда на восьмой день они принесли ребенка, чтобы совершить обряд обрезания, то официально дали ему имя Иоанн, как и было им сказано задолго до того. Племянник Захарии был послан в Назарет с вестью от Елизаветы, которая извещала Марию о рождении сына, который назван Иоанном.
\vs p122 2:8 С самого раннего детства на Иоанна глубокое впечатление произвела идея родителей о том, что когда он вырастет, то должен будет стать духовным лидером и религиозным учителем. И семена этих идей пали на благодатную почву в сердце Иоанна. Даже когда он был ребенком, Иоанна часто находили в храме в то время, когда там служил отец, и на мальчика производила огромное впечатление значительность всего, что он видел.
\usection{3. Предвещание Гавриила Марии}
\vs p122 3:1 Однажды вечером, когда Иосиф еще не возвратился домой, Гавриил явился Марии у низкого каменного стола и, когда она оправилась от испуга, сказал: «Я пришел от имени того, кто является моим Господом и кого ты будешь любить и лелеять. Тебе, Мария, я принес радостную весть; возвещаю тебе, что зачатое тобою дитя послано небом; и в должное время у тебя родится сын, ты назовешь его Иешуа и он установит царство небесное на земле и среди людей. Не говори об этом никому, кроме Иосифа и Елизаветы, твоей родственницы, к ней я тоже являлся и она также ждет теперь сына, который будет назван Иоанном. Он будет подготавливать путь для той вести освобождения, которую твой сын весьма решительно и проникновенно возвестит людям. Не подвергай сомнению мои слова, Мария, потому что этот дом избран, чтобы стать земной обителью дитя предназначения. Мое благословение будет с тобою, пусть сила Всевышних укрепит тебя и Господь всей земли сохранит тебя».
\vs p122 3:2 \pc В течение нескольких недель Мария в тайне обращалась в сердце своем к этому явлению до тех пор, пока не обрела уверенность, что действительно ждет ребенка, только тогда осмелилась она рассказать об этом необыкновенном событии своему мужу. Иосиф, услышав об этом, сильно встревожился, хотя и доверял полностью своей жене, и многие ночи провел без сна. Сначала Иосиф сомневался в посещении Гавриила. Потом, когда он пришел к убеждению, что Мария действительно слышала голос и созерцала божественного вестника, он мучился, задаваясь вопросом, как могло случиться такое. Каким образом рожденный человеком может быть ребенком божественной судьбы? Иосиф никак не мог разрешить эти спорные вопросы до тех пор, пока после нескольких недель раздумий они оба, он и Мария, не пришли к выводу, что они избраны стать родителями Мессии, хотя по представлению иудеев ожидаемый спаситель вовсе не должен был иметь божественную природу. Придя к этому знаменательному решению, Мария заторопилась навестить Елизавету.
\vs p122 3:3 После своего возвращения Мария посетила своих родителей Иокима и Анну. Двое ее братьев и две сестры, так же как и родители, всегда весьма скептически относились к божественной миссии Иисуса, хотя, конечно, в то время они ничего не знали о визите Гавриила. Однако Мария открыла сестре, Саломее, свои мысли о том, что ее сыну суждено стать великим учителем.
\vs p122 3:4 \pc Сообщение Гавриила Марии произошло на следующий день после зачатия Иисуса и было единственным сверхъестественным событием, связанным со всем периодом, пока она вынашивала и давала жизнь ребенку обетования.
\usection{4. Сон Иосифа}
\vs p122 4:1 В этом сне ему явился сияющий небесный вестник и, помимо всего прочего, сказал: «Иосиф, я являюсь тебе по повелению Того, кто правит на небесах, и должен наставить тебя в том, что касается сына, который будет зачат Марией и станет величайшим светочем мира. В нем будет жизнь, и его жизнь станет светом для человечества. Прежде всего он явится своему народу, но они вряд ли примут его; тем же, кто примет его, он откроет истину, что они дети Бога». После этого сна Иосиф никогда более не подвергал сомнению историю Марии о явлении Гавриила и обещании, что еще не рожденное дитя должно стать божественным вестником миру.
\vs p122 4:2 \pc Во время этих явлений ничего не было сказано о доме Давидовом. В них не было даже и намека на то, что Иисус станет «избавителем евреев», ни даже на то, что он должен стать долгожданным Мессией. Иисус не был тем Мессией, которого ожидали евреи, он был \bibemph{избавителем мира.} Его миссия предназначалась всем расам и народам, а не какой\hyp{}либо одной группе.
\vs p122 4:3 Иосиф не происходил из рода Давидова. Род Марии был больше связан с потомками Давида, чем предки Иосифа. Иосиф действительно отправился в город Давида, Вифлеем, чтобы пройти римскую перепись, но это было связано с тем, что за шесть поколений до того предок Иосифа по отцовской линии, будучи сиротой, был усыновлен неким Задоком, который был прямым потомком Давида, поэтому и считалось, что Иосиф происходит из «дома Давидова».
\vs p122 4:4 Большинство так называемых «мессианских» пророчеств Ветхого завета были отнесены к Иисусу много времени спустя после его земной жизни. На протяжении столетий иудейские пророки возвещали пришествие избавителя, и эти обещания последующими поколениями были отнесены к грядущему иудейскому правителю, который займет трон Давидов и тем же чудесным образом, что и некогда Моисей, утвердит в Палестине положение евреев как могущественной нации, свободной от иноземного господства. И впоследствии многие образные фрагменты из Ветхого завета были ошибочно отнесены к жизненной миссии Иисуса. Многие тексты Ветхого Завета были искажены, чтобы приспособить их к тем или иным эпизодам его жизни на земле. Сам Иисус однажды публично отрицал какую\hyp{}либо свою связь с домом Давида. Даже строки «Дева родит сына» были изменены на «девственница родит сына». То же относится к генеалогическим описаниям как Иосифа, так и Марии, составленным уже после земной жизни Михаила, некоторые из этих описаний содержат много верной информации о предках Иисуса, но в целом они не достоверны и на них нельзя опираться как на фактический материал. Ранние последователи Иисуса слишком часто поддавались искушению показать, что все возможные ветхозаветные пророчества осуществились в жизни их Господа и Учителя.
\usection{5. Земные родители Иисуса}
\vs p122 5:1 Иосиф был человеком мягким, необычайно честным во всем верно следовавшим религиозным обычаям и обрядам своего народа. Он мало говорил, но много думал. Тяжелое положение евреев доставляло Иосифу много огорчений. В юности в кругу своих восьми братьев и сестер, он бывал более весел, но в начальные годы своей семейной жизни (в годы детства Иисуса) Иосиф порой испытывал определенные духовные разочарования. Проявления этих черт характера Иосифа практически исчезли незадолго до его безвременной кончины, когда экономическое положение семьи улучшилось благодаря тому, что он сменил работу плотника на должность преуспевающего подрядчика.
\vs p122 5:2 Характер Марии был совсем иным, чем у мужа. Она всегда была весела, очень редко впадала в уныние и обычно пребывала в лучезарном расположении духа. Мария часто давала волю свободному проявлению своих эмоций, и ее никогда не видели печальной вплоть до самой скоропостижной кончины Иосифа. Едва она оправилась от этого потрясения, как на нее обрушились тревоги и проблемы, связанные с необычайной судьбой старшего сына, которая столь стремительно разворачивалась перед ее изумленным взором. Но в течение всех этих необычайных событий Мария оставалась спокойной, отважной и достаточно мудрой в отношениях со своим старшим, странным и малопонятным для нее сыном и его подрастающими братьями и сестрами.
\vs p122 5:3 Иисус во многом унаследовал свою необыкновенную мягкость и замечательный талант сочувствовать и понимать человеческую природу от отца; он получил дар великого учителя и свою потрясающую способность к праведному гневу от матери. В эмоциональных реакциях на окружение во взрослой жизни Иисус бывал временами похож на отца, задумчив и молитвенно настроен, иногда явно опечален, но чаще он общался с окружающими на манер своей оптимистичной и целеустремленной матери. В конце концов, темперамент Марии начинал все более преобладать в характере божественного сына по мере того, как Иисус взрослел и делал важнейшие шаги в своей взрослой жизни. Кое в чем Иисус соединял в себе свойства характера обоих родителей, в других ситуациях черты одного из родителей преобладали над чертами другого.
\vs p122 5:4 От Иосифа Иисус получил строгое воспитание в вопросах еврейских обрядов и необычайное знание Иудейского писания; от Марии он унаследовал широту взглядов на религиозную жизнь и более либеральные представления о личной духовной свободе.
\vs p122 5:5 Обе семьи, и Иосифа, и Марии, были хорошо образованными для своего времени. По образованию Иосиф и Мария значительно превосходили средний для своего времени и положения уровень. Иосиф был мыслителем, Мария --- замечательно умела планировать, приспосабливаться к обстоятельствам и быстро предпринимать практические действия. Иосиф был черноглазым брюнетом; у Марии были карие глаза и светлые волосы.
\vs p122 5:6 Останься Иосиф в живых, он несомненно твердо уверовал бы в божественную миссию своего старшего сына. В своем отношении к сыну Мария часто колебалась между верой и сомнением, ибо была подвержена влиянию мнений остальных детей, а также друзей и родственников, однако в конечном счете ее отношение оставалось неизменным и определялось воспоминаниями о явлении Гавриила, произошедшем сразу после того, как ребенок был зачат.
\vs p122 5:7 Мария была искусной ткачихой и больше, чем многие ее современницы, преуспела во многих домашних ремеслах; она была хорошей хозяйкой и превосходной матерью семейства. Оба, и Иосиф, и Мария, являлись хорошими учителями и следили за тем, чтобы дети получили образование, принятое в то время.
\vs p122 5:8 \pc В юности Иосиф работал у отца Марии, выполняя строительные работы для пристройки к дому, в то время Мария принесла ему к полуденной трапезе сосуд с водой, с чего и начались отношения пары, которой суждено было стать родителями Иисуса.
\vs p122 5:9 Свадьба Иосифа и Марии состоялась по еврейским обычаям в доме Марии в окрестностях Назарета, когда Иосифу исполнился 21 год. Этим браком закончился период обычных ухаживаний, который длился почти два года. Вскоре после этого молодые переехали в новый дом в Назарете, который Иосиф построил с помощью двух своих братьев. Дом был расположен вблизи подножия холма, откуда открывался прелестный вид на окружавшую его сельскую местностью. В этом доме, специально приготовленном, юные и полные надежд родители рассчитывали встретить явление на свет обетованного ребенка, не подозревая, что этот момент жизни вселенной должен произойти, когда они отлучатся из дома, чтобы отправиться в Вифлеем в Иудее.
\vs p122 5:10 \pc Большинство членов семьи Иосифа стали последователями учения Иисуса, тогда как очень немногие из родственников Марии поверили в него до того, как он покинул этот мир. Иосиф придерживался духовной концепции ожидаемого Мессии, тогда как Мария и ее семья, особенно отец, рассматривали Мессии как временного освободителя и политического правителя. Предки Марии в те не такие уж далекие времена были достаточно тесно связаны с деятельностью Маккавеев.
\vs p122 5:11 Иосиф придерживался более строгого, восточного или вавилонского подхода к еврейской религии; Мария скорее склонялась к более либеральному и широкому западному, или эллинистическому, толкованию закона и пророков.
\usection{6. Дом в Назарете}
\vs p122 6:1 Дом Иисуса был расположен неподалеку от высокого холма в северной части Назарета, на некотором расстоянии от источника, находившегося в восточной части города. Семья Иисуса жила на окраине города, и впоследствии ему было легко отправляться на частые прогулки по окрестностям и совершать путешествия к вершине, этой ближайшей к дому возвышенности, высочайшей из всех в южной Галилее за исключением горы Табор к востоку и холма Наин, имевшего примерно ту же высоту. Дом семьи Иисуса был расположен на небольшом расстоянии к юго\hyp{}востоку от южного отрога этого холма на полпути между основанием возвышенности и дорогой, ведущей из Назарета в Кану. Иисус любил взбираться на холмы, и еще его любимым занятием были прогулки по узкой тропинке, которая вилась вокруг подножия холма в северо\hyp{}восточном направлении вплоть до места соединения с дорогой на Сефорис.
\vs p122 6:2 Дом Иосифа и Марии был однокомнатным каменным строением с плоской крышей, к которому примыкала постройка для скота. Обстановка дома состояла из низкого каменного стола, глиняной посуды, каменных блюд и горшков, ткацкого станка, лампы, нескольких маленьких табуреток и лежаков на каменном полу. На заднем дворе около пристройки для скота под навесом помещался очаг и мельница для перемалывания зерна. На такой мельнице нужно было работать вдвоем: один должен был молоть, а другой подсыпать зерно. Маленьким мальчиком Иисус часто подсыпал зерно в мельницу, которую вращала его мать.
\vs p122 6:3 Позднее, когда семья выросла, все они часто собирались за большим каменным столом, семья ела с одного большого блюда или горшка. Зимой во время вечерней трапезы стол приходилось освещать при помощи маленькой, плоской глиняной лампы, в которую наливали оливковое масло. После рождения Марфы Иосиф пристроил к дому большую комнату, которая днем использовалась как мастерская плотника, а ночью как спальня.
\usection{7. Путешествие в Вифлеем}
\vs p122 7:1 В марте 8 года до н.э. (месяц, когда состоялась свадьба Иосифа и Марии) цезарь Август издал указ, согласно которому все население Римской империи должно было пройти перепись, что было необходимо, чтобы более эффективно собирать налоги. У евреев всегда было сильное предубеждение против любых попыток «пересчитывать народ», и это в сочетании с серьезными проблемами в семье иудейского царя Ирода задержало перепись в Иудейском царстве на год. По всей Римской империи перепись была проведена в 8 году до н.э., кроме Палестинского царства Ирода, где она проходила год спустя --- в 7 году до н.э.
\vs p122 7:2 Марии не обязательно было ехать в Вифлеем --- Иосиф имел право зарегистрировать всю семью, --- но Мария, будучи энергичной и любящей путешествовать женщиной, настояла на том, чтобы сопровождать мужа. Она, пока ребенок не родится, боялась во время отсутствия Иосифа оставаться одна, и кроме того, поскольку от Вифлеема было недалеко до города Иуды, Мария предвкушала радость возможной встречи со своей родственницей Елизаветой.
\vs p122 7:3 Иосиф искренне пытался запретить Марии ехать с собой, но безуспешно; она приготовила в дорогу двойной запас еды на три\hyp{}четыре дня и собралась в путь. И еще до начала пути Иосиф примирился с тем, что Мария поедет с ним, и они весело отправились на рассвете из Назарета.
\vs p122 7:4 Иосиф и Мария были бедны, и поскольку у них был только один вьючный ослик, Мария, у которой был уже большой срок беременности, ехала верхом вместе с провизией, в то время как Иосиф вел животное. Иосиф израсходовал почти все свои средства на строительство и обстановку дома, кроме того он должен был еще помогать родителям, поскольку его отец незадолго до этого стал инвалидом. Итак эта еврейская чета отправилась из своего дома ранним утром 18 августа 7 года до н.э. в путешествие в Вифлеем.
\vs p122 7:5 В первый день путешествия они добрались до подножия горы Гелвуй и, остановившись на ночь на берегу реки Иордан, долго обсуждали, каким будет их сын, который скоро появится на свет, при этом Иосиф склонялся к тому, что он будет духовным учителем, Марии же была ближе мысль о том, что он будет Мессией, освободителем еврейского народа.
\vs p122 7:6 Ранним и солнечным утром 19 августа Иосиф и Мария вновь двинулись в путь. Они остановились у подножия горы Сартаба, откуда открывался вид на долину реки Иордан, для полуденной трапезы, затем вновь отправились в путь и к ночи достигли Иерихона, где остановились в придорожной гостинице на окраине города. После вечерней трапезы и долгого обсуждения занимавших их вопросов об угнетательской для евреев власти римлян, об Ироде, проведении переписи и значении влияния Иерусалима и Александрии как центров еврейской учености и культуры, назаретские путешественники отправились на покой. Ранним утром 20 августа они вновь тронулись в путь и до полудня прибыли в Иерусалим, где посетили храм, а затем продолжили свое путешествие и в два часа пополудни приехали в Вифлеем.
\vs p122 7:7 Постоялый двор был переполнен, и Иосиф попытался найти место для ночлега у дальних родственников, однако все комнаты в Вифлееме были забиты до отказа людьми. Возвратившись на постоялый двор, он услыхал, что стойла для караванов, выбитые прямо в скале и располагавшиеся под гостиницей, освободили от животных и вычистили, чтобы принять постояльцев. Оставив своего ослика во дворе, Иосиф, взвалив на плечи мешки с вещами и провизией, вместе с Марией спустился по каменным ступеням вниз, чтобы разместиться на ночлег. Их разместили в помещении для хранения зерна, которое находилось перед стойлами и яслями для скота. Помещение было отгорожено занавеской, и путешественники были счастливы, что нашли такой удобный ночлег.
\vs p122 7:8 Иосиф хотел немедленно отправиться, чтобы пройти перепись, но Мария чувствовала себя усталой; она тревожилась и уговаривала Иосифа остаться подле нее, что он и сделал.
\usection{8. Рождение Иисуса}
\vs p122 8:1 Всю ночь Мария не находила себе места, так что оба не спали. К утру стало ясно, что начались родовые схватки, и к полудню 21 августа 7 года до н.э. при помощи и добром наставлении женщин, оказавшихся их попутчицами, Мария родила мальчика. Иисус из Назарета родился на свет, Мария спеленала его пеленками, которые прихватила на этот случай, и положила в ясли рядом собой.
\vs p122 8:2 Обетованный ребенок родился точно так же, как являлись на свет все другие дети до и после него. И на восьмой день, согласно иудейским законам, он был обрезан и официально назван Иешуа (Иисус).
\vs p122 8:3 На следующий день после рождения Иисуса Иосиф прошел перепись. Он встретил человека, с которым беседовал за две ночи до этого в Иерихоне, и тот отвел его к зажиточному другу, у которого была комната на постоялом дворе, и тот сказал, что охотно уступит место паре из Назарета. Днем они перебрались наверх в комнату и оставались там почти три недели, пока не нашли жилье в доме дальнего родственника Иосифа.
\vs p122 8:4 Через два дня после рождения Иисуса Мария послала Елизавете весточку о том, что ее ребенок явился на свет, и в ответ получила письмо, в котором Иосифа приглашали приехать в Иерусалим, чтобы обсудить все дела с Захарией. На следующей неделе Иосиф отправился в Иерусалим, чтобы посоветоваться с Захарией. И Захария, и Елизавета искренне верили в то, что Иисус воистину должен стать освободителем еврейского народа, Мессией, и что их сын, Иоанн, должен стать главой его последователей, его правой рукой. И поскольку Мария придерживалась того же мнения, ей было несложно убедить Иосифа остаться в Вифлееме, городе Давида, дабы Иисус мог вырасти здесь, чтобы стать преемником Давида на троне всего Израиля. Поэтому они оставались в Вифлееме больше года. И Иосиф все это время работал плотником.
\vs p122 8:5 \pc В полдень, когда Иисус родился, серафимы Урантии собрались под началом своих руководителей и пели гимны славы над вифлеемскими яслями, но эти славословия были не доступны человеческому слуху. Ни пастухи, ни какие\hyp{}либо другие смертные люди не приходили, чтобы поклониться ребенку из Вифлеема, до того дня, когда прибыли некие священники из Ура, которые были присланы из Иерусалима Захарией.
\vs p122 8:6 Этим священникам из Месопотамии некий необыкновенный религиозный учитель их страны рассказал, что видел сон, в котором ему было сказано, что «свет жизни» скоро появится на земле как ребенок, и среди евреев. И поэтому эти три учителя отправились искать этот «свет жизни». После многих недель бесплодных поисков в Иерусалиме они уже хотели возвращаться в Ур, но тут встретили Захарию, открывшего им, что именно Иисус --- тот, кого они ищут, и пославшего их в Вифлеем, где они нашли ребенка и принесли дары Марии, его земной матери. Во время их посещения ребенку было почти три недели от роду.
\vs p122 8:7 Эти мудрецы не видели звезды, которая привела бы их в Вифлеем. Прекрасная легенда о Вифлеемской звезде возникла так: Иисус родился 21 августа в полдень 7 года до н.э. а 29 мая 7 года до н.э., произошло необычайное сближение Юпитера и Сатурна в созвездии Рыб. Необычным это астрономическое явление было потому, что аналогичное соединение происходило 29 сентября и 5 декабря того же года. Основываясь на этом необычном, но вполне естественном явлении, исполненные самых лучших намерений приверженцы христианства следующих поколений создали чудесную легенду о звезде Вифлеема и волхвах, которых она привела к яслям, где они нашли новорожденного младенца и поклонились ему. Восточные и ближневосточные народы находят удовольствие в сказках, и постоянно изобретают красивые мифы о жизни религиозных лидеров и политических героев. При отсутствии печатных книг большая часть человеческих знаний передавалась из уст в уста, из поколения в поколение, мифы при этом очень легко становились традициями и как традиции очевидно начинали восприниматься как факты.
\usection{9. Представление в храме}
\vs p122 9:1 Моисей учил евреев, что каждый перворожденный сын принадлежит Богу, и вместо того, чтобы приносить его в жертву по обычаям язычников, родители могли изменить его участь, принеся в жертву специально избранному для этой цели священнику пять шекелей. Один из Моисеевых законов гласил также, что мать через определенный период времени должна была сама (или кто\hyp{}то, кто сможет принести соответствующую жертву за нее) пройти в храме обряд очищения. Обычно оба эти обряда происходили одновременно. Таким образом, Иосиф и Мария отправились в Иерусалимский храм, чтобы лично представить Иисуса священнику и принести соответствующую жертву для обряда очищения Марии от нечистоты, связанной с родами.
\vs p122 9:2 \pc При храме постоянно находились два замечательных человека: Симеон, певчий и Анна, поэтесса. Симеон был родом из Иудеи, Анна же происходила из Галилеи. Они много времени проводили вместе и оба были близки к Захарии, поведавшему им секрет Иоанна и Иисуса. И Симеон, и Анна жаждали пришествия Мессии и доверие к Захарии заставляло их уверовать, что Иисус --- долгожданный спаситель еврейского народа.
\vs p122 9:3 Захарии был известен день, когда Иосиф и Мария должны были появиться в храме с Иисусом, и он заранее договорился с Симеоном и Анной, что укажет им, подняв в приветствии руку, какой из перворожденных младенцев и есть Иисус.
\vs p122 9:4 По этому случаю Анна написала поэму, которую Симеон пропел к глубокому изумлению Иосифа, Марии и всех, кто собрались во дворе храма. Это был гимн искуплению перворожденного сына.
\vs p122 9:5 Благословен Господь Бог Израиля,
\vs p122 9:6 Что посетил народ Свой и сотворил избавление ему,
\vs p122 9:7 И воздвиг рог спасения нам
\vs p122 9:8 В дому Давида, слуги своего;
\vs p122 9:9 Как возвестил устами бывших от века святых пророков Своих ---
\vs p122 9:10 Что спасет нас от врагов наших и от руки всех ненавидящих нас;
\vs p122 9:11 Сотворит милость отцам нашим и помнит святой завет свой ---
\vs p122 9:12 Клятву, которой клялся он Аврааму, отцу нашему,
\vs p122 9:13 Дать нам избавлении от рук врагов наших;
\vs p122 9:14 Дабы мы служили ему безбоязненно
\vs p122 9:15 Служить Ему в святости и правде перед Ним во все дни жизни нашей;
\vs p122 9:16 И ты, дитя заветное, наречешься пророком Всевышнего,
\vs p122 9:17 Ибо, пойдешь пред лицом Господа --- установить царство Его;
\vs p122 9:18 Дать уразуметь народу Его спасение,
\vs p122 9:19 И прощение грехов их.
\vs p122 9:20 И возрадуйся благоутробному милосердию Бога нашего, ибо заря новая посетила нас,
\vs p122 9:21 Просветить сидящих в тьме и тени смертной,
\vs p122 9:22 Направить ноги наши на путь мира.
\vs p122 9:23 Ныне отпускаешь раба твоего, Владыка, по слову твоему с миром;
\vs p122 9:24 Ибо видели очи мои спасение твое,
\vs p122 9:25 Которое ты уготовил пред лицом всех;
\vs p122 9:26 Свет даже к просвещению язычников
\vs p122 9:27 И славу народа твоего Израиля.
\vs p122 9:28 \pc На обратном пути в Вифлеем Иосиф и Мария молчали --- они были смущены и испытывали благоговейный страх. Марию встревожило прощальное приветствие Анны, престарелой поэтессы, а Иосиф чувствовал себя неуютно из\hyp{}за преждевременной попытки сделать из Иисуса долгожданного Мессию еврейского народа.
\usection{10. Действия Ирода}
\vs p122 10:1 Но и шпионы Ирода не сидели без дела. Когда они донесли ему о посещении Вифлеема священниками из Ура, Ирод приказал, чтобы этих халдейских гостей доставили к нему. Он тщательно допросил мудрецов о новорожденном «царе иудейском», и они сказали, что младенец рожден женщиной, которая прибыла в Вифлеем с мужем, чтобы пройти перепись, но это вовсе не успокоило его. Ирод, не удовлетворившись таким ответом, дал мудрецам деньги и велел найти ребенка, чтобы и он тоже мог пойти и поклониться ему, поскольку они провозглашали, что его царство должно быть духовным, а не временным. Однако когда мудрецы не возвратились, подозрительность Ирода усилилась. И пока он обдумывал все это, возвратились его соглядатаи и подробно рассказали о произошедшем в храме, и принесли копию части гимна, который пел Симеон на церемонии выкупа Иисуса. Однако они не последовали за Иосифом и Марией, и Ирод разгневался на них потому, что они не могли указать, куда родители унесли ребенка. Он отправил своих соглядатаев разыскивать Иосифа и Марию. Узнав, что Ирод преследует назаретское семейство, Захария и Елизавета держались в стороне от Вифлеема. Мальчика спрятали родные Иосифа.
\vs p122 10:2 Иосиф боялся искать работу, и скромные сбережения семьи быстро таяли. Даже во время церемонии очищения в храме Иосиф считал что его бедность вполне оправдывает то, что в жертву за Марию были принесены лишь два юных голубя, как заповедал Моисей для очищения бедных матерей.
\vs p122 10:3 После того как шпионы Ирода более чем за год не смогли найти Иисуса, и поскольку существовало подозрение, что он все еще прячется в Вифлееме, Ирод отдал приказ обыскать все дома в Вифлееме и убить всех младенцев мужского пола младше двух лет. Таким образом Ирод рассчитывал, что этот ребенок, который должен был стать «царем Иудейским», будет уничтожен. Так погибли за один день шестнадцать мальчиков в иудейском городе Вифлееме. Но интриги и убийства, даже в своей собственной семье, были обычными при дворе Ирода.
\vs p122 10:4 Это избиение младенцев произошло в середине октября 6 года до н.э., и Иисусу было тогда чуть больше года. Но даже среди окружения Ирода были люди, веровавшие в пришествие Мессии, и один из них, узнав о приказе уничтожить всех младенцев\hyp{}мальчиков в Вифлееме, сообщил об этом Захарии, а тот, в свою очередь, послал гонца к Иосифу; и ночью в канун избиения Иосиф и Мария покинули Вифлеем и бежали с ребенком в Александрию, в Египет. Чтобы не привлекать к себе внимания, они отправились в Египет одни, с Иисусом. Они прибыли в Александрию, имея деньги, полученные от Захарии, и Иосиф начал работать, в то время как Мария с Иисусом поселились у хорошо обеспеченных родных семьи Иосифа. Они оставались в Александрии полных два года и не возвращались в Вифлеем до самой смерти Ирода.
