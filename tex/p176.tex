\upaper{176}{Во вторник вечером на Масличной горе}
\author{Комиссия срединников}
\vs p176 0:1 В этот вторник днем, когда Иисус и апостолы вышли из храма, направляясь к Гефсиманскому лагерю, Матфей, указывая на строения храма, сказал: «Учитель, посмотри, какие здания. Взгляни на массивные камни и прекрасную отделку; возможно ли, что этим зданиям суждено быть разрушенными?» Они продолжили путь к Масличной горе, и Иисус сказал: «Вы видите эти камни и этот массивный храм; истинно, истинно говорю я вам: в дни, которые скоро настанут, не останется и камня на камне. Все будет низвергнуто». Эти слова, живописующие разрушение святого храма, вызвали любопытство у шедших за Учителем апостолов; они не могли себе представить ничего, кроме, разве, конца света, что привело бы к разрушению храма.
\vs p176 0:2 Чтобы не встречаться с толпами, следующими по долине Кедрон в сторону Гефсимании, Иисус и его сподвижники предпочли немного подняться по западному склону Масличной горы, а затем идти тропой, ведущей к их уединенному лагерю возле Гефсимании, который находился несколько выше того места, где располагался общий лагерь для народа. Когда они свернули с дороги, ведущей к Вифании, их взглядам открылся храм, прекрасный в лучах заходящего солнца; и пока они задержались на горе и любовались красотой освещенного храма, они видели, как в городе загораются огни и там, при мягком свете полной луны, Иисус и двенадцать апостолов устроили привал. Учитель стал беседовать с ними, и вскоре Нафанаил задал такой вопрос: «Скажи нам, Учитель, как мы узнаем, когда эти события должны совершиться?»
\usection{1. Разрушение Иерусалима}
\vs p176 1:1 В ответ на вопрос Нафанаила Иисус сказал: «Да, я поведаю вам о временах, когда сей народ переполнит чашу своей неправедностью; когда справедливость быстро обрушится на этот город наших отцов. Я скоро покину вас; я отправляюсь к Отцу. После того, как я покину вас, остерегайтесь, чтобы кто не прельстил вас, ибо многие будут приходить под именем спасителей и многих прельстят. Когда вы услышите о войнах и о военных слухах, не ужасайтесь, ибо, хотя все это будет происходить, это еще не конец Иерусалима. Вас не должны волновать голод и землетрясения; не следует вам тревожиться и тогда, когда вас доставят к светским властям и будут преследовать за евангелие. Вы будете изгнаны из синагоги и брошены в темницу из\hyp{}за меня, и некоторые из вас будут убиты. Когда вы предстанете перед властителями и правителями, то это будет для того, чтобы подтвердить вашу веру и доказать вашу непоколебимость в евангелии царства. А когда вы предстанете перед судьями, не заботьтесь наперед, что вам говорить, ибо в тот час дух научит вас, что отвечать своим противникам. В эти дни тяжелых испытаний даже ваши собственные родственники, руководимые теми, кто отверг Сына Человеческого, будут предавать вас в темницу и на смерть. Какое\hyp{}то время все люди могут возненавидеть вас из\hyp{}за меня, но даже во дни этих гонений я не оставлю вас; дух мой вас не покинет. Будьте терпеливы! Не сомневайтесь в том, что это евангелие царства восторжествует над всеми врагами и, в конце концов, будет возвещено всем народам».
\vs p176 1:2 Иисус сделал паузу и посмотрел вниз на город. Учитель понимал, что отвержение духовного представления о Мессии, решимость упорно и слепо настаивать на материальной стороне миссии ожидаемого спасителя вскоре приведет евреев к прямому конфликту с мощными римскими армиями и что следствием такого столкновения станет лишь окончательное и полное низвержение еврейской нации. Отвергнув его духовный дар и отказавшись принять небесный свет, так милостиво ниспосланный на них, его народ, тем самым, подписал приговор своему статусу суверенного народа с особой духовной миссией на земле. Даже еврейские правители впоследствии осознали, что именно эта мирская идея о Мессии прямо привела к смутам, которые, в конце концов, стали причиной их гибели.
\vs p176 1:3 Поскольку Иерусалим должен был стать колыбелью раннего евангелического движения, Иисус не хотел, чтобы его учителя и проповедники погибли во время ужасного низвержения еврейского народа, связанного с разрушением Иерусалима; поэтому он и дал эти наставления своим последователям. Иисус очень беспокоился, чтобы ни один из его учеников не оказался втянутым в приближающиеся волнения и не погиб бы в результате при падении Иерусалима.
\vs p176 1:4 Затем Андрей спросил: «Но, Учитель, если Священному Городу и храму предстоит быть разрушенными и если тебя здесь не будет, чтобы направлять нас, когда же нам следует покинуть Иерусалим?» Иисус сказал: «Вы можете оставаться в городе после того, как я уйду, даже во время тяжких испытаний и жестоких гонений, но когда вы увидите, наконец, как после восстания лжепророков римские армии окружат Иерусалим, тогда знайте, что приблизилось опустение его; тогда вы должны бежать в горы. Пусть никто из находящихся в городе и поблизости от него не остаются, чтобы что\hyp{}либо спасти, и пусть находящиеся в его окрестностях не пытаются в него войти. Будет великое горе, ибо это будут дни мести язычников. И после того, как вы покинете город, этот непокорный народ падет от острия меча и будет уведен в плен во все народы; и так Иерусалим будет попираем язычниками. Между тем я предупреждаю вас, не давайте себя обмануть. Если какой\hyp{}то человек придет к вам со словами: „Смотрите, вот Спаситель“, или „Смотрите, вон он“ --- не верьте этому, ибо появится много лжеучителей и прельстят многих; но вы не должны обманываться, ибо я наперед сказал вам все».
\vs p176 1:5 Долго сидели апостолы молча при свете луны, и в их озадаченных умах запечатлевались эти поразительные предсказания Учителя. И именно в соответствии с этим предупреждением при первом же появлении римских войск практически все верующие и ученики бежали из Иерусалима, найдя безопасное прибежище в Пелле, на севере.
\vs p176 1:6 Даже после такого ясного предупреждения многие из последователей Иисуса считали, что в этих предсказаниях речь идет о переменах, которые, очевидно, произойдут в Иерусалиме, когда новое явление Мессии приведет к основанию Нового Иерусалима и к расширению города, который станет столицей мира. В сознании этих евреев разрушение храма было неразрывно связано с «концом света». Они верили, что Новый Иерусалим заполнит всю Палестину; что за концом света сразу же последует появление «новых небес и новой земли». И поэтому не было ничего странного в том, что Петр сказал: «Учитель, мы знаем, что все исчезнет, когда появятся новые небеса и новая земля, но как нам узнать, когда ты вернешься, чтобы осуществить все это?»
\vs p176 1:7 Услышав это, Иисус некоторое время пребывал в задумчивости, а затем сказал: «Вы постоянно заблуждаетесь, потому что всегда пытаетесь соединить новое учение со старым; вы упорно не желаете уяснить для себя все мое учение; вы настойчиво толкуете евангелие в соответствии с вашими укоренившимися представлениями. Тем не менее, я попытаюсь просветить вас».
\usection{2. Второе пришествие Учителя}
\vs p176 2:1 Несколько раз Иисус делал заявления, которые приводили его слушателей к выводу, что хотя он и намеревается вскоре покинуть этот мир, но непременно вернется, чтобы довести до конца дело царства небесного. Поскольку у его последователей крепло убеждение, что он собирается покинуть их, то, после того как он ушел из этого мира, было вполне естественно, что все верующие прочно ухватились за эти его обещания вернуться. Таким образом, с ранних времен идея о втором пришествии Христа стала частью учений христиан, и почти каждое последующее поколение адептов свято верило в эту истину и с уверенностью ожидало, что однажды он придет.
\vs p176 2:2 Коль скоро первым ученикам и апостолам предстояло расстаться со своим Учителем и Наставником, они особенно сильно уповали на это обещание вернуться, и тут же стали связывать предсказанное разрушение Иерусалима с обещанным вторым пришествием. Именно так они продолжали толковать его слова, несмотря на то, что весь этот вечер, прошедший в наставлениях на Масличной горе, Учитель прилагал особые усилия к тому, чтобы предотвратить именно такую ошибку.
\vs p176 2:3 \pc Продолжая отвечать на вопрос Петра, Иисус сказал: «Почему вы по\hyp{}прежнему ждете, что Сын Человеческий воссядет на трон Давида, и ожидаете, что исполнятся мирские мечты евреев? Не говорил ли я вам все эти годы, что мое царство не от мира сего? Тому, на что вы сейчас взираете, настанет конец, но это будет новым началом, из которого евангелие царства пойдет по всему миру и спасение распространится на все народы. А когда во всей полноте будет установлено царство, будьте уверены, что Отец Небесный непременно посетит вас, шире открывая истину и все более являя праведность, равно как он послал уже в этот мир того, кто стал принцем тьмы, а затем Адама, за которым последовал Мелхиседек, а ныне Сына Человеческого. И так мой Отец продолжит проявлять свою милость и являть свою любовь даже к такому темному и порочному миру. И я тоже, после того, как мой Отец облечет меня всей властью и полномочиями, продолжу следить за вашими судьбами и направлять вас в делах царства посредством моего духа, который скоро изольется на всякую плоть. И хотя я, таким образом, буду духовно присутствовать с вами, я, кроме того, обещаю, что в свое время вернусь в этот мир, где прожил эту жизнь во плоти и приобрел опыт открытия Бога человеку и, одновременно, приведения человека к Богу. Очень скоро я должен буду покинуть вас и приняться за дело, порученное мне Отцом, но не падайте духом, ибо в свое время я вернусь. А тем временем вас будет утешать и направлять мой Дух Истины вселенной.
\vs p176 2:4 Сейчас вы видите меня в слабости и во плоти, но вернусь я в силе и в духе. Плотский глаз видит Сына Человеческого во плоти, но только духовный глаз увидит, как Сын Человеческий, восславленный Отцом, появится на земле во имя себя самого.
\vs p176 2:5 Но время следующего появления Сына Человеческого известно лишь в Райских советах; даже ангелы небесные не знают, когда это произойдет. Однако вам следует понимать, что когда для спасения всех народов это евангелие царства будет возвещено всему миру и когда минует эта эпоха, Отец пошлет вам еще одно избавительное пришествие или же Сын Человеческий вернется, чтобы вынести приговор эпохе.
\vs p176 2:6 А что касается тяжких испытаний для Иерусалима, о которых я говорил вам, то мои слова исполнятся прежде, чем уйдет это поколение; но никто ни на небе, ни на земле не может взять на себя смелость что\hyp{}либо сказать о времени нового пришествия Сына Человеческого. Но вы должны быть прозорливы чтобы видеть конец эпохи; вы должны быстро распознавать приметы времени. Вы знаете, что когда ветви смоковницы дают новые побеги и пускают листья, значит близится лето. Точно так же, когда мир пройдет долгую зиму материалистического сознания и вы заметите наступление духовной весны новой диспенсации, знайте, что близится лето нового пришествия.
\vs p176 2:7 Но в чем значение этого учения, касающегося пришествия Сына Бога? Разве вы не осознаете, что когда каждый из вас будет призван закончить свой жизненный путь и, пройдя через врата смерти, сразу же предстанет перед судом и вы окажетесь в новой диспенсации вечного служения бесконечному Отцу? То, что непременно ожидает мир в конце эпохи, то же предстоит в качестве личного опыта и каждому из вас, конкретных людей, когда вы достигнете конца своей естественной жизни и, таким образом, после смерти столкнетесь с теми условиями и требованиями, которые неразрывно связаны со следующим откровением вечного развития царства Отца».
\vs p176 2:8 Ни одно из рассуждений, когда\hyp{}либо высказанных Учителем перед его апостолами, не привело к такой путанице в их умах, как это, произнесенное вечером во вторник на Масличной горе по поводу как разрушения Иерусалима, так и его собственного второго пришествия. Именно поэтому так плохо согласуются между собой последующие письменные свидетельства, основанные на воспоминаниях о том, что сказал Учитель в этот необыкновенный вечер. Таким образом, поскольку в записях отсутствовало многое из сказанного в тот вторник вечером, возникло множество преданий; и в самом начале второго века еврейское пророчество о Мессии, написанное неким Селтой, служившим при дворе императора Калигулы, было целиком переписано в евангелие от Матфея, а впоследствии было включено (частично) в евангелия от Марка и от Луки. Именно в этом тексте Селты появилась притча о десяти девах. Ни одна часть евангелия не содержит таких вводящих в заблуждение искажений, как та, которая касается наставлений этого вечера. И только апостол Иоанн никогда не впадал в подобное заблуждение.
\vs p176 2:9 Когда тринадцать человек продолжили свой путь к лагерю, они пребывали в безмолвии и в огромном эмоциональном напряжении. Иуда окончательно утвердился в своем решении покинуть своих сотоварищей. Был уже поздний час, когда Давид Зеведеев, Иоанн Марк и некоторые из ближайших учеников радушно встретили Иисуса и двенадцать апостолов в новом лагере, но апостолы не хотели спать; они хотели больше узнать о разрушении Иерусалима, уходе Учителя и конце света.
\usection{3. Продолжение беседы в лагере}
\vs p176 3:1 Когда они, человек двенадцать, собрались вокруг костра, Фома спросил: «Поскольку тебе предстоит вернуться, чтобы завершить дело царства, каким должно быть наше поведение, пока ты будешь отсутствовать, находясь в том, что принадлежит Отцу?» Иисус окинул взглядом всех, сидящих у огня, и ответил:
\vs p176 3:2 \pc «И даже ты, Фома, не смог понять того, что я говорил. Не учил ли я вас все это время, что ваша связь с царством --- духовная и у каждого своя, и полностью зависит от личного духовного опыта при осознании через веру того, что ты --- сын Бога? Что же еще мне сказать? Падение наций, крушение империй, гибель неверующих евреев, конец эпохи, даже конец света --- что все это значит для того, кто верит в это евангелие и уберег свою жизнь гарантией вечного царства? Вы, знающие Бога и верующие в евангелие, уже получили гарантии вечной жизни. Поскольку ваши жизни прожиты в духе и для Отца, ничто не может быть для вас предметом серьезного беспокойства. Строителей царства, законных граждан небесных миров не должны беспокоить преходящие перевороты и волновать земные катаклизмы. Что значат для вас, верующих в это евангелие царства, низвержение наций, конец эпохи или крушение всего зримого, если вы знаете, что ваша жизнь есть дар Сына и что она вечно хранима Отцом? Прожив конечную жизнь с верой и принося плоды духа в виде исполненного любви служения своим ближним, вы можете с уверенностью ожидать следующей стадии вечного пути с той же самой спасительной верой, которая провела вас через полный событий первый, земной этап сыновства у Бога.
\vs p176 3:3 Каждое поколение верующих должно вершить свой труд в предвидении возможного возвращения Сына Человеческого, точно так же, как каждый верующий свершает труд своей жизни в предвидении неизбежной и неминуемой естественной смерти. Если однажды через веру вы утвердились как сын Бога, ничто больше не сможет поколебать уверенность в спасении. Но не заблуждайтесь! Эта спасительная вера --- живая вера, и она все более являет плоды того божественного духа, который сначала вдохнул ее в сердце человека. То, что вы некогда приняли сыновство небесного царства, не спасет вас в случае, если вы знаете и при этом упорно отвергаете те истины, которые касаются роста духовной плодотворности плотских сынов Бога. Вы, кто были вместе со мной в том, что принадлежит Отцу на земле, даже сейчас можете покинуть царство, если сочтете, что вам не люб указанный Отцом путь служения человечеству.
\vs p176 3:4 Вы как индивидуумы и как поколение верующих, послушайте, я расскажу притчу: был некий важный человек, который прежде, чем отправиться в долгий путь в другую страну, позвал к себе всех своих доверенных слуг и передал в их руки все свое добро. Одному он дал пять талантов, другому два, еще одному --- один. Итак --- всем своим уважаемым слугам, каждому он вверил свое добро в соответствии с его личными способностями; а затем отправился в путь. Когда их господин ушел, слуги принялись за работу, чтобы извлечь прибыль из вверенного им богатства. Получивший пять талантов сразу же употребил их в дело и очень скоро получил доход еще в пять талантов. Аналогично, получивший два таланта вскоре заработал еще два. Итак, прибыль своему хозяину принесли все слуги, кроме того, кто получил всего лишь один талант. Он пошел, выкопал яму в земле и спрятал туда деньги своего господина. Вскоре господин неожиданно вернулся и позвал тех слуг для отчета. И когда они предстали перед своим хозяином, то получивший пять талантов подошел с вверенными ему деньгами и принес еще пять талантов, сказав: „Господин, ты дал мне пять талантов, чтобы пустить их в дело, и я с радостью отдаю еще пять талантов, которые я заработал“. И тогда его господин сказал ему: „Молодец, добрый и верный слуга, ты был верен в малом; теперь я поставлю тебя над многими; войди в радость господина твоего“. И затем подошел получивший два таланта и сказал: „Господин, ты передал в мои руки два таланта; смотри, я заработал еще два таланта“. И тогда его господин сказал ему: „Молодец, добрый и верный слуга; ты тоже был верен в малом, и теперь я поставлю тебя над многими; войди в радость господина твоего“. А затем дошло до отчета того, кто получил один талант. Этот слуга подошел и сказал: „Господин, я знал тебя и понимал, что ты хитрый человек и ожидаешь получить доход там, где лично не работал; поэтому я побоялся рисковать хоть чем\hyp{}то из того, что мне было вверено. Я надежно скрыл твой талант в землю; вот он; теперь то, что тебе принадлежит, --- у тебя“. Но его господин ответил: „Ты лукавый и ленивый слуга. По твоим же собственным словам, ты знал, что я потребую от тебя отчета о разумной прибыли, такого же, какой представили сегодня твои прилежные товарищи. Зная это, ты должен был, хотя бы, отдать мои деньги в рост, чтобы по возвращении я смог получить свои деньги с процентом“. И затем этот господин сказал управляющему слуге: „Забери этот один талант у не приносящего прибыли слуги и отдай его тому, у которого десять талантов“.
\vs p176 3:5 Всякому имеющему дастся и приумножится еще больше, а у неимеющего отнимется даже то, что имеет. Нельзя стоять на месте в делах вечного царства. Мой Отец требует от всех своих детей возрастать в благодати и знании истины. Вы, знающие эти истины, должны приумножать плоды духа и все больше посвящать себя самоотверженному служению своим ближним. И помните, так, как вы сделали это одному из сих братьев моих меньших, то сделали мне.
\vs p176 3:6 Итак, вы должны заниматься делом Отца сейчас, и впредь, и вовеки. Продолжайте, пока я не приду. Исправно исполняйте то, что на вас возложено, и, в результате, вы будете готовы, когда смерть призовет вас к отчету. И прожив так во славу Отца и к удовлетворению Сына, вы с радостью и великим удовольствием войдете в вечное царство для вечного служения».
\vs p176 3:7 \pc Истина --- жива; Дух Истины вечно ведет детей света в новое царство духовных реалий и божественного служения. Истина дается вам не для того, чтобы она обретала устоявшиеся, надежные и почитаемые формы. Ваше открытие истины должно быть настолько усилено вашим личным опытом, чтобы новая красота и подлинный духовный рост открылись всем, кто взирает на ваши духовные плоды, и, вследствие этого, побудили бы их восславить Отца Небесного. Только те верные слуги, кто таким образом возрастают в знании истины и тем самым совершенствуют способность к божественному восприятию духовных реалий, могут надеяться «полностью войти в радость своего Господа». Какое печальное зрелище будут являть собой следующие поколения признавших себя последователей Иисуса, если скажут по поводу своего служения божественной истине: «Учитель, вот истина, которую ты передал нам сто или тысячу лет назад. Мы ничего не потеряли; мы честно сохранили все, что ты нам дал; мы не допускали никаких изменений в том, чему ты научил нас; вот та истина, которую ты нам дал». Но такое объяснение духовной лени не оправдает бесплодного слугу истины перед Учителем. Учитель потребует отчета о прибыли, соответствующей той истине, которая была передана в ваши руки.
\vs p176 3:8 В последующем мире вас попросят дать отчет о дарованиях и служении этому миру. Малы ли природные таланты или велики, предстоит справедливый и милосердный суд. Если дарования используются лишь в эгоистических целях и не уделяется внимание высшему долгу, заключающемуся в приумножении плодов духа, проявляющемся в постоянно расширяющемся служении людям и почитании Бога, то таким эгоистичным слугам придется испытать последствия своего сознательного выбора.
\vs p176 3:9 И насколько же этот нерадивый слуга с одним талантом похож на всех эгоистов --- смертных в том, что он переложил вину за свою леность непосредственно на своего господина. Как же склонен человек, сталкиваясь с неудачами, в которых сам же повинен, перекладывать вину на других, нередко и на тех, кто меньше всего этого заслуживает!
\vs p176 3:10 Когда они отправлялись отдыхать в ту ночь, Иисус сказал: «Даром вы получили; поэтому даром следует вам передавать небесную истину, и по мере вашего служения ей при передаче эта истина будет приумножаться и излучать усиливающийся свет спасительной благодати».
\usection{4. Возвращение Михаила}
\vs p176 4:1 Из всех учений Учителя ни одно положение не было понято так превратно, как его обещание вернуться в свое время в этот мир. Нет ничего странного в том, что Михаилу интересно было бы когда\hyp{}нибудь вернуться на планету, на которой он пережил свое седьмое и последнее пришествие в качестве смертного этого мира. Вполне естественно полагать, что Иисусу из Назарета, являющемуся теперь суверенным правителем обширной вселенной, было бы интересно не только единожды, но даже много раз вернуться в тот мир, где он прожил такую необыкновенную жизнь и, в конце концов, заслужил дар Отца --- безграничную вселенскую силу и власть. Урантия вечно будет одной из семи родных сфер Михаила, на которых он заслужил владычество над вселенной.
\vs p176 4:2 Иисус в многочисленных ситуациях и многим людям заявлял о своем намерении вернуться в этот мир. Когда его последователи осознали тот факт, что их Учитель не будет выступать в роли смертного спасителя, и когда они слушали его предсказания о падении Иерусалима и низвержении еврейской нации, они, вполне естественно, стали связывать его обещанное возвращение с этими катастрофическими событиями. Но когда римские армии сравняли стены Иерусалима с землей, разрушили храм и рассеяли иудейских евреев, а Учитель по\hyp{}прежнему не являлся в силе и славе, у его последователей начало складываться представление, которое, со временем, стало связывать второе пришествие Христа с концом эпохи, даже с концом света.
\vs p176 4:3 Иисус обещал сделать две вещи после того, как вознесется к Отцу и вся власть на небе и на земле будет передана в его руки. Он обещал, во\hyp{}первых, послать в этот мир вместо себя другого учителя --- Дух Истины; и он сделал это в день Пятидесятницы. Во\hyp{}вторых, он совершенно определенно обещал своим последователям, что когда\hyp{}нибудь лично вернется в этот мир. Но он не говорил, как, где и когда вновь посетит эту планету, где он пережил опыт пришествия во плоти. Один раз он поведал, что если глаза плоти видели его тогда, когда он жил здесь во плоти, то при его возвращении (по крайней мере, в одно из его возможных пришествий) он будет видим лишь глазами духовной веры.
\vs p176 4:4 Многие из нас склонны верить, что в течение предстоящих эпох Иисус будет возвращаться на Урантию много раз. Он не давал конкретного обещания совершать эти неоднократные визиты, но вполне вероятно, что он, носящий в числе своих вселенских титулов и титул Планетарного Принца Урантии, много раз посетит тот мир, завоевание которого принесло ему такой уникальный титул.
\vs p176 4:5 Мы, безусловно, верим, что Михаил снова лично явится на Урантию, но у нас нет ни малейшего представления, когда и каким образом он решит явиться. Будет ли его второе пришествие на землю приурочено к последнему суду этой нынешней эпохи, который будет или же не будет связан с появлением Сына\hyp{}Повелителя? Придет ли он в связи с окончанием какой\hyp{}либо последующей эпохи Урантии? Явится ли он без всякого возвещения, и это будет единичным событием? Мы не знаем. Лишь в одном мы уверены, а именно: когда он вернется, весь мир, по\hyp{}видимому, узнает об этом, ибо он должен прийти как верховный правитель вселенной, а не как безвестный младенец из Вифлеема. Но если каждый должен будет узреть его и если только духовные глаза смогут различить его присутствие, тогда его пришествие должно быть отложено еще надолго.
\vs p176 4:6 Поэтому правильнее было бы перестать ассоциировать личное возвращение Учителя на землю с какими бы то ни было конкретными событиями или определенными эпохами. Мы уверены лишь в одном: Он обещал вернуться. Мы не имеем никакого представления, когда он исполнит свое обещание и в связи с чем. Насколько мы понимаем, он может появиться на земле в любой день, а может и не появиться в течении многих эпох, суд над которыми будут вершить его сподвижники из отряда Райских Сынов.
\vs p176 4:7 Второе пришествие Михаила на землю --- это событие огромной эмоциональной важности как для срединников, так и для людей; но в остальном оно не имеет непосредственного значения для срединников, а для людей оно с практической точки зрения ничуть не более важно, чем ординарное явление --- естественная смерть, которая так внезапно ввергает смертного человека во власть той цепи вселенских событий, которая прямо ведет его непосредственно к тому же самому Иисусу, верховному правителю нашей вселенной. Всем детям света суждено его увидеть, и не имеет существенного значения, мы ли отправимся к нему сами или же случится так, что сначала он придет к нам. Будьте же поэтому всегда готовы встречать его на земле, как и он постоянно готов встретить вас на небе. Мы с уверенностью ждем его знаменательного появления, даже многократных появлений, но нам совершенно неведомо, как, когда и в связи с чем суждено ему появиться.
