\upaper{1}{Отец Всего Сущего}
\author{Божественный Советник}
\vs p001 0:1 Отец Всего Сущего есть Бог всего сотворенного, Первоисточник и Центр всех вещей и существ. Сначала подумайте о Боге как о творце, затем как о контролере и, в последнюю очередь, как о бесконечном вседержителе. Истина об Отце Всего Сущего начала зарождаться у людей, когда пророк сказал: «Ты, Бог, есть один, никого нет равного тебе. Ты создал небеса, и небеса небес, со всеми их силами; Ты охраняешь их и управляешь ими. Вселенные были созданы Сынами Божьими. Творец покрывает себя светом как одеждой и простирает небеса как завесу». Только концепция Отца Всего Сущего --- одного Бога вместо многих богов --- позволила смертному человеку воспринять Отца как божественного творца и бесконечного контролера.
\vs p001 0:2 Мириады планетных систем были созданы, чтобы в результате стать населенными многообразными разумными созданиями, существами, способными познать Бога, получить божественную любовь и в ответ полюбить его. Вселенная вселенных есть дело рук Божьих и место пребывания всевозможных его созданий. «Бог создал небеса и сотворил Землю: он образовал вселенную и создал этот мир не напрасно; он сотворил его для жизни.»
\vs p001 0:3 Все просветленные миры признают и почитают Отца Всего Сущего, вечного творца и бесконечного вседержителя всего сущего. Обладающие волей создания многочисленных вселенных вступают в долгое\hyp{}долгое путешествие к Раю, на восхитительную стезю вечного странствия, чтобы постичь Бога Отца. Трансцендентная цель детей времени --- найти вечного Бога, понять божественную природу, распознать Отца Всего Сущего. Осознающие Бога создания имеют одно только верховное стремление, единственное всепоглощающее желание. Оно заключается в том, чтобы самим стать похожими на него в его Райском совершенстве личности и в его вселенской сфере праведного верховенства. От Отца Всего Сущего, который обитает в вечности, пришел верховный завет: «Будь совершенен, как и я совершенен». В любви и милости Райские посланники несли это божественное послание сквозь века и вселенные, даже к таким созданиям низкого уровня, происходящим от животных, как человеческие расы Урантии.
\vs p001 0:4 Великолепное и вселенское предписание стремиться к достижению совершенства божественности есть первый долг, и это должно стать высочайшим устремлением всех борющихся созданий, являющихся творением Бога совершенства. Эта возможность достижения божественного совершенства есть конечная и определенная судьба вечного духовного прогресса для всех людей.
\vs p001 0:5 Смертные Урантии едва ли могут надеяться на то, чтобы стать совершенными в бесконечном смысле, но для человеческих существ, начинающих на этой планете, вполне возможно достижение безграничной и божественной цели, которую бесконечный Бог поставил перед смертным человеком; и когда они устремятся к этой судьбе, они добьются во всем, что касается самореализации и достижений разума, такой полноты в своей сфере божественного совершенства, какой обладает сам Бог в его сфере бесконечного и вечного. Такое совершенство может не быть универсальным в материальном смысле, беспредельным в интеллектуальном понимании или конечным в духовном опыте, но оно является конечным и завершенным во всех конечных аспектах божественности воли, совершенства личностной мотивации и Богосознания.
\vs p001 0:6 Это истинное значение божественного указания «Будь совершенен, как и я совершенен», которое всегда устремляет смертного человека вперед и манит его внутрь той долгой и восхитительной борьбы за достижение все более и более высоких уровней духовных ценностей и истинных вселенских значений. Этот возвышенный поиск Бога вселенных есть верховное искание обитателей всех миров пространства и времени.
\usection{1. Имя Отца}
\vs p001 1:1 Из всех имен, под которыми Бог Отец известен во всех вселенных, наиболее часто встречаются Первоисточник и Центр Вселенной. Первичный Отец известен под разными именами в различных вселенных и различных частях одной и той же вселенной. Имена, которые создания дают Творцу, в значительной степени зависят от той концепции Творца, которой обладают эти создания. Первоисточник и Центр Вселенной никогда не раскрывает себя по имени, только по природе. Если мы верим, что все мы дети этого творца, вполне естественно, что мы будем называть его Отцом. Но это имя мы выбираем сами, и оно возникает путем познания наших личностных взаимоотношений с Первоисточником и Центром.
\vs p001 1:2 Отец Всего Сущего никогда не навязывает созданиям, обладающим разумом и волей, никаких форм произвольного познания, официального поклонения или рабского служения. Эволюционирующие обитатели миров времени и пространства должны сами --- по велению своих сердец --- узнавать, любить и добровольно почитать его. Творец отказывается от того, чтобы заставлять или принуждать к покорности свободные духовные воли своих материальных созданий. Исполненное любви посвящение человеческой воли выполнению воли Отца --- это самый ценный дар Богу; действительно, такое посвящение воли создания составляет единственно возможный дар, обладающий истинной ценностью для Райского Отца. В Боге человек живет, и движется, и существует; нет ничего, что человек может дать Богу, кроме этого выбора предать себя воле Отца, и такие решения, принимаемые разумными созданиями вселенной, обладающими волей, образуют то истинное богопочитание, которое приносит такое удовлетворение природе Отца\hyp{}Творца, исполненной любви.
\vs p001 1:3 Когда однажды ты осознаешь Бога, после того, как ты действительно откроешь великолепие Творца и начнешь испытывать на опыте присутствие божественного контролера, тогда, в соответствии с твоим просветлением и согласно характеру и методу, с помощью которых божественные Сыны открывают Бога, ты найдешь имя для Отца Всего Сущего, которое будет адекватным выражением твоей концепции Великого Первоисточника и Центра. И таким образом в различных мирах и в различных вселенных Творец становится известен под множеством имен; по сути все они имеют одно и то же значение, но в виде слов и символов каждое имя соответствует уровню, глубине его воцарения в сердцах его созданий любого данного мира.
\vs p001 1:4 \pc Вблизи центра вселенной вселенных Отец Всего Сущего обычно известен по имени, которое можно рассматривать как Первоисточник. Далее, в пространственных вселенных, Отца Всего Сущего называют терминами, которые чаще всего означают Вселенский Центр. Еще далее в широте небесного мира он известен, как Первичный и Созидательный Источник и Божественный Центр. На соседнем к вам созвездии Бога величают Отцом Вселенных. В другом --- Бесконечный Вседержитель и к востоку --- Божественный Контролер. Кроме этого, он именуется как Отец Светов, Дар Жизни, Всемогущий.
\vs p001 1:5 В тех мирах, где Райский Сын прожил жизнь в пришествии, бог обычно известен под именем, указывающим на личностные отношения, нежную любовь и преданность отца. В центре вашего созвездия к Богу обращаются, как к Отцу Всего Сущего, и на разных планетах в вашей локальной системе обитаемых миров он известен по\hyp{}разному --- как Отец Отцов, Райский Отец, Отец Хавоны или Отец Дух. Те, кто знают Бога через откровения пришествия Райских Сынов, обычно отдаются чувству трогательных взаимоотношений между созданием и Создателем и обращаются к Богу, как к «нашему Отцу».
\vs p001 1:6 На планете с созданиями, имеющими пол, в мире, где сердцам разумных существ присущи импульсы родительских эмоций, имя Отец очень выразительно и подходяще для вечного Бога. Он лучше всего известен, повсеместно признается на вашей планете Урантия под именем \bibemph{Бог.} Имя, которое ему дано, не имеет большого значения; существенно, что вы должны знать его и стремиться быть похожими на него. Ваши древние пророки правильно называли его «вечным Богом» и относились к нему, как к тому, кто «пребывает в Вечности».
\usection{2. Реальность Бога}
\vs p001 2:1 Бог есть первичная реальность в духовном мире; Бог есть источник истины в сферах разума; Бог простирается на все в материальных сферах. Для всех созданных разумных существ Бог есть личность, а для вселенной вселенных он есть Первоисточник и Центр вечной реальности. Бог не похож ни на человека, ни на машину. Первичный Отец есть вселенский дух, вечная истина, бесконечная реальность и личность Отца.
\vs p001 2:2 \pc Вечный Бог есть бесконечно большее, чем идеализированная реальность или персонализированная вселенная. Бог не есть просто верховное желание человека, объективированный поиск смертного. Бог не есть также только концепция, мощь --- потенциал праведности. Отец Всего Сущего не является ни синонимом природы, ни олицетворением закона природы. Бог есть трансцендентная реальность, а не просто традиционная для человека концепция верховных человеческих ценностей. Бог не есть ни психологическое средоточие духовных значений, ни «благороднейшая работа человека». Бог может быть любой или всеми этими концепциями в умах людей, но он есть большее. Он есть спаситель и любящий Отец для всех, кто обладает духовным миром на земле и кто жаждет продолжить существование в посмертии.
\vs p001 2:3 \pc Реальность существования Бога демонстрируется в человеческом опыте через пребывание божественного присутствия, духовного Наблюдателя, посланного из Рая, чтобы обитать в разуме смертного, чтобы помочь в развитии бессмертной души для вечного продолжения существования в посмертии. Присутствие божественного Настройщика Мысли в человеческом разуме раскрывается тремя эмпирическими феноменами:
\vs p001 2:4 \ublistelem{1.}\bibnobreakspace Интеллектуальная способность разума познавать Бога --- осознания Бога.
\vs p001 2:5 \ublistelem{2.}\bibnobreakspace Духовное стремление найти Бога --- поиск Бога.
\vs p001 2:6 \ublistelem{3.}\bibnobreakspace Стремление личности быть подобной Богу --- чистосердечное желание исполнять волю Отца.
\vs p001 2:7 \pc Существование Бога никогда не может быть доказано с помощью научного эксперимента или путем только дедуктивных рассуждений. Бог может быть осознан только на основе человеческого опыта, тем не менее истинная концепция существования Бога приемлема для логики, правдоподобна для философии, существенна для религии, совершенно необходима для любой надежды на продолжение существования в посмертии.
\vs p001 2:8 Те, кто знают Бога, испытали факт его присутствия; такие знающие Бога смертные несут в своем личном опыте единственное позитивное доказательство существования живого бога, которое одно человеческое существо может предложить другому. Существование Бога не может быть явлено никоим иным образом, кроме контактов между сознающим Бога человеческим разумом и Божественным присутствием Настройщика Мысли, который пребывает в разуме смертного и дается человеку как свободный дар Отца Всего Сущего.
\vs p001 2:9 \pc Теоретически вы можете думать о Боге как о Творце, и он есть личностный Творец Рая и центральной вселенной совершенства, но вселенные со временем и пространством созданы и организованы Райскими отрядами Сынов\hyp{}Творцов. Отец Всего Сущего не является личностным Творцом локальной вселенной Небадон; вселенная, в которой живете вы, --- творение его Сына Михаила. Хотя Отец лично не создает развивающиеся вселенные, он контролирует многие их вселенские взаимосвязи и некоторые проявления их физических, ментальных и духовных энергий. Бог Отец есть личностный Творец Райской вселенной и, вместе с Вечным Сыном, он --- создатель всех других личностных Творцов вселенных.
\vs p001 2:10 \pc Как физический контролер в материальной вселенной вселенных, Первоисточник и Центр действует в паттернах вечного острова Рая и через этот абсолютный гравитационный центр вечный Бог осуществляет космический сверхконтроль физического уровня равно как в центральной вселенной, так и повсюду во вселенной вселенных. В качестве разума Бог функционирует в Божестве Бесконечного Духа; как дух Бог проявляется в лице Вечного Сына и в лицах божественных детей Вечного Сына. Эти взаимоотношения Первоисточника и Центра с равными ему по статусу Личностями и Абсолютами Рая ни в малейшей степени не устраняют \bibemph{прямого} личностного действия Отца Всего Сущего во всех творениях всех его уровней. Через присутствие фрагментов духа Творец Отец поддерживает непосредственный контакт со своими детьми\hyp{}созданиями и своими вселенными.
\usection{3. Бог Есть Вселенский Дух}
\vs p001 3:1 «Бог есть дух». Он есть вселенское духовное присутствие. Отец Всего Сущего есть бесконечная духовная реальность; он есть «владыка, вечный, бессмертный, невидимый и единственный истинный Бог». Хотя вы являетесь «отпрысками Бога», вы не должны думать, что Отец похож на вас по форме и физическим свойствам, потому что о вас говорят как о созданных «по его образу» --- в которых пребывают Таинственные Наблюдатели, посланные из центральной обители его вечного присутствия. Духовные существа реальны, несмотря на то, что они невидимы человеческому глазу; и не созданы из плоти и крови.
\vs p001 3:2 Сказано пророками древности: «Смотрите, он идет мимо меня и я не вижу его, он проходит и я не постигаю его». Мы можем постоянно наблюдать труды Бога, мы можем в высшей степени осознавать материальные свидетельства его великолепного действия, но редко перед нашими глазами оказывается видимое проявление его божественности, мы даже не можем созерцать дух его.
\vs p001 3:3 Отец Всего Сущего не является невидимым потому, что он прячет себя от существ низкого материалистического уровня и ограниченного духовного развития. Ситуация представляется так: «Вы не можете увидеть мое лицо, ибо ни один смертный не может увидеть меня и жить». Ни один материальный человек не мог бы узреть Бога Духа и сохранить смертное существование. Более низкие группы духовных существ или любых материальных личностей не могут приблизиться к великолепию и духовному сиянию божественной личности. Духовное сияние личного присутствия Отца --- это «свет, к которому не может подступить смертный человек; которого ни одно материальное существо не видело и не может видеть». Но нет необходимости видеть Бога глазами плоти, чтобы увидеть его взором веры одухотворенного разума.
\vs p001 3:4 \pc Духовная природа Отца Всего Сущего полностью разделяется им с сосуществующим Вечным Сыном Рая. Как Отец, так и Сын одинаково и полностью разделяют вселенский и вечный дух с равноправной с ними объединенной личностью --- Бесконечным Духом. Дух Бога сам по себе абсолютен; в Сыне он неограничен, в Духе всемирен, и в них и через них --- бесконечен.
\vs p001 3:5 \pc Бог есть вселенский дух; Бог есть вселенская личность. Верховная личностная реальность конечного создания есть Дух; предельная действительность личностного космоса есть абсонитный дух. Только уровни бесконечности являются абсолютными, и только на таких уровнях существует законченность единства между материей, разумом и духом.
\vs p001 3:6 \pc Во вселенных Бог Отец есть в потенциале сверхконтролер материи, разума и духа. Только с помощью своего широко разветвленного контура личности Бог имеет дело непосредственно с личностями своего огромного творения, населенного созданиями, обладающими волей, но эти создания с ним могут контактировать (вне Рая) только через его фрагменты, представляющие волю Бога, пребывающую во вселенных. Этот Райский дух, который пребывает в умах смертных во времени и способствует эволюции бессмертной души созданий, продолжающих существование в посмертии, обладает природой и божественностью Отца Всего Сущего. Но умы таких эволюционирующих созданий возникают в локальных вселенных и должны обрести божественное совершенство благодаря достижению таких переживаемых опытным путем духовных свершений, которые являются неизбежным результатом сделанного личностью выбора выполнять волю Отца на небесах.
\vs p001 3:7 \pc Во внутреннем опыте человека сознание соединено с материей. Такое соединенное с материей сознание не может продолжать существование после смерти. Способ существования в посмертии осуществляется в тех настройках человеческой воли и тех трансформациях смертного разума, посредством которых осознающий Бога интеллект постепенно становится обученным духом и ведомым духом. Эта эволюция человеческого сознания от материальных связей к духовному союзу приводит к превращению потенциально духовных фаз смертного разума в моронтийную реальность бессмертной души. Смертный разум, служащий материи, предназначен стать все более материальным и, следовательно, пережить неизбежное угасание личности; разум, подчиняющийся духу, предназначен стать все более духовным и окончательно достигнуть единства с ведущим его божественным духом и этим путем достичь продолжения существования в посмертии и вечного существования личности.
\vs p001 3:8 Я прихожу из Вечного, и я много раз возвращался вновь к присутствию Отца Всего Сущего. Я знаю о действительности и личности Первоисточника и Центра, Вечного Отца Всего Сущего. Я знаю, что в то время как великий Бог абсолютен, вечен и бесконечен, он так же добр, божественен и милосерден. Я знаю истину великих высказываний: «Бог есть дух» и «Бог есть любовь», и эти два атрибута наиболее полно раскрываются для вселенной в Вечном Сыне.
\usection{4. Таинство Бога}
\vs p001 4:1 Бесконечность совершенства Бога такова, что она вечно делает его таинственным. И самая великая из всех непознаваемых тайн Бога --- божественное пребывание в умах смертных. Способ, с помощью которого Отец Всего Сущего пребывает с созданиями времени, является наиболее глубокой из всех тайн вселенной; божественное присутствие в уме человека есть таинство таинств.
\vs p001 4:2 Физические тела смертных --- это «храмы Бога». Несмотря на то, что Владыки Сыны\hyp{}Творцы приближаются к созданиям своих обитаемых миров и «привлекают к себе всех людей», что они «стоят при дверях» сознания «и стучат» и рады прийти ко всем, кто «откроет двери своих сердец»; что существует личностное общение между Сынами\hyp{}Творцами и их смертными творениями, смертные люди тем не менее, имеют что\hyp{}то от самого Бога, которое действительно пребывает в них; таким образом, их тела есть храмы.
\vs p001 4:3 Когда вы окончите свою жизнь на этой земле, когда будет завершен ваш путь во временной форме на земле, когда будет закончен путь ваших испытаний во плоти, когда прах, который составляет смертного как храм души, «возвращается в землю, откуда он вышел, тогда пребывающий Дух возвратится к Богу, который дал его». В каждом нравственном существе этой планеты пребывает фрагмент Бога, неотъемлемая часть божественности. Вы еще не обладаете им, но ему предназначено слиться с вами, если вы продолжите существование в посмертии.
\vs p001 4:4 \pc Мы постоянно сталкиваемся с тайной Бога; нас ставит в тупик непрестанно расширяющаяся бесконечная панорама истины его безграничной добродетели, бескрайней милости, несравненной мудрости и величественного характера.
\vs p001 4:5 \pc Божественная тайна --- неотъемлемое свойство различия, которое существует между конечным и бесконечным, временным и вечным, существом живущим в пространстве и времени и Творцом Всего Сущего, материальным и духовным, несовершенством человека и совершенством Райского Божества. Бог любви неистощимо открывает себя для каждого из своих созданий в той степени полноты, с которую его создания способны духовно воспринять свойства божественной истины, красоты и добродетели.
\vs p001 4:6 Для каждого духовного существа и каждого смертного создания в каждой сфере и в каждом мире вселенной вселенных Отец Всего Сущего открывает свое милостивое и божественное Я в той степени, в которой может различаться или осознаваться этими духовными существами и смертными созданиями. Бог не взирает на лица, как на материальные, так и на духовные. Божественное присутствие, которым может обладать каждый из детей вселенной в каждый данный момент, ограничено только способностью подобного создания получать и различать духовную действительность сверхматериального мира.
\vs p001 4:7 Бог, как реальность в человеческом духовном опыте, не есть тайна. Но когда физический разум материального порядка пытается прояснить реальности духовного мира, появляется тайна; тайна настолько тонкая и глубокая, что только благодаря вере знающий Бога смертный способен достигнуть чуда познания Бесконечного конечным, постижения вечного Бога развивающимися смертными материальных миров пространства и времени.
\usection{5. Личность Отца Всего Сущего}
\vs p001 5:1 Не допускайте, чтобы величие Бога и его бесконечность скрывали или затемняли его личность. «Он, кто замыслил ухо, сможет ли не услышать? Он, кто создал глаз, сможет ли не увидеть?» Отец Всего Сущего есть кульминация божественной личности; он есть исток и судьба личности во всем творении. Бог есть и бесконечное и личностное; он есть бесконечная личность. Отец есть истинно личность, несмотря на то, что бесконечность его личности ставит его навсегда за пределы полного понимания материальных и конечных созданий.
\vs p001 5:2 Бог есть много большее, чем личность в том смысле, как понимает личность человеческий разум; он даже гораздо более, чем возможная концепция сверхличности. Но совершенно тщетно обсуждать такие непознаваемые концепции божественной личности, необъяснимые для умов материальных созданий, чья максимальная концепция реальности существования состоит в идее и идеале личности. Наивысшая возможная концепция Творца Всего Сущего для материального существа заключается в пределах духовных идеалов возвышенной идеи божественной личности. Таким образом, хотя вам может быть известно, что Бог должен быть намного больше, чем человеческая концепция личности, вы также хорошо знаете, что Отец Всего Сущего не может быть чем\hyp{}либо меньшим, чем вечная, бесконечная, истинная, добродетельная и прекрасная личность.
\vs p001 5:3 Бог не прячется ни от кого из своих творений. Он недостижим для такого количества чинов существ только потому, что он «пребывает в свете, к которому не может приблизиться ни одно материальное создание». Необъятность и величие божественной личности находится вне пределов овладения несовершенного разума эволюционных смертных. Он «измеряет воды ладонью, измеряет вселенную, простирая руку. Это он, кто сидит на кругах Земли, простирает небеса как завесу и распространяет их по вселенной». «Подымите глаза ввысь и узрите того, кто создал все вещи, сотворил их бесчисленные миры и дал им всем имена»; и поэтому правда, что «невидимые творения Бога частично понимаются как творения видимые». Сегодня вы, такие как вы есть, должны различить невидимого Создателя через его многочисленные и разнообразные творения, так же как и через откровение и служение его Сынов и их многочисленных подчиненных.
\vs p001 5:4 Несмотря на то, что материальные смертные не могут увидеть личность Бога, они должны радоваться в уверенности, что он есть; верой принять истину, что Отец Всего Сущего настолько любит мир, что обеспечил духовный прогресс его стоящих на низком уровне обитателей; что он «радуется в своих детях». У Бога нет недостатка в тех сверхчеловеческих и божественных атрибутах, которые составляют совершенную, вечную, любящую и бесконечную личность Творца.
\vs p001 5:5 \pc В локальных творениях (за исключением персонала сверхвселенных) Бог личностно не проявляется, кроме Райских Сынов\hyp{}Творцов, которые есть отцы обитаемых миров и владыки локальных вселенных. Если бы вера сотворенного была совершенной, он бы с уверенностью мог осознать, что когда он видел Сына\hyp{}Творца, он видел Отца Всего Сущего; в поиске Отца не должно ни просить, ни ожидать увидеть кого\hyp{}либо, кроме Сына. Смертный человек просто не может увидеть Бога, пока он не совершит завершенного духовного превращения и действительно не достигнет Рая.
\vs p001 5:6 Природа Райских Сынов\hyp{}Творцов не охватывает всех неограниченных потенциалов вселенской абсолютности бесконечной природы Великого Первоисточника и Центра, но Отец Всего Сущего действительно \bibemph{божественно} присутствует в Сынах\hyp{}Творцах. Отец и его Сыны есть единое. Эти Райские Сыны чина Михаила есть совершенные личности, даже паттерны, для всех личностей локальной вселенной --- от таких, как Яркая и Утренняя Звезда до самого низкого человеческого создания в прогрессирующей эволюции животных.
\vs p001 5:7 \pc Без Бога и при отсутствии его великой и центральной личности не было бы личности во всей огромной вселенной вселенных. \bibemph{Бог есть личность.}
\vs p001 5:8 \pc Несмотря на то, что Бог есть вечное могущество, величественное присутствие, трансцендентный идеал и исполненный славы дух, хотя он вмещает не только это, но и бесконечно больше; он, тем не менее, есть истинно и постоянно пребывающая личность совершенного Творца, лицо, которое может «знать и быть познанным», которое может «любить и быть любимым», которое может быть нашим другом; в то время как ты можешь быть известен, как были известны другие люди, как друг Бога. Он есть реальный дух и духовная реальность.
\vs p001 5:9 Поскольку мы видим Отца Всего Сущего раскрывающимся во всей его вселенной; поскольку различаем его пребывающим в мириадах созданий; поскольку видим его в лице его Владык Сынов; поскольку мы продолжаем ощущать его божественное присутствие и здесь и там, близко и далеко, нам не следует сомневаться в первичности его личности. Несмотря на все эти обширные сферы распространения, он остается истинной личностью и всегда поддерживает персональную связь с бессчетным сонмом своих созданий, рассеянных повсюду во вселенной вселенных.
\vs p001 5:10 \pc Понятие личности Отца Всего Сущего является расширенной и более правдивой концепцией Бога, которая пришла к человечеству, главным образом, через откровение. Разум, мудрость и религиозный опыт --- все они подразумевают и указывают на личность Бога, но они не обосновывают ее в целом. Даже постоянно пребывающий в нас Настройщик Мысли является предличностным. Истина и зрелость любой религии прямо пропорциональны ее концепции бесконечной личности Бога и ее осознанию абсолютного единства Божества. После того, как религия впервые сформировала концепцию единства Бога, осознание идеи личностного Божества становится мерой религиозной зрелости.
\vs p001 5:11 Примитивная религия имела много личностных богов, и они представлялись в образах человека. Откровение утверждает подлинность концепции Бога как личности, которая является только возможной в научном постулате о Первопричине и предложена только условно в философской идее Вселенского Единства. Любой человек может начать понимать единство Бога только через личность. Отрицание личности Первоисточника и Центра оставляет выбор только из двух философских дилемм: материализм или пантеизм.
\vs p001 5:12 В созерцании Божества концепция личности должна быть лишена идеи вещественности. Материальное тело не обязательно как для личности человека, так и для личности Бога. Ошибка вещественности проявлена в обоих крайних течениях человеческой философии. В материализме --- поскольку человек теряет свое тело при смерти, он прекращает существовать как личность; в пантеизме --- поскольку Бог бестелесен, он, таким образом, не является личностью. Сверхчеловеческий тип прогрессирующей личности функционирует в союзе разума и духа.
\vs p001 5:13 \pc Личность не есть просто атрибут Бога; она, скорее, выступает как полнота координированной бесконечной природы и объединенной божественной воли, которая проявляется в вечности и всеобщности совершенного выражения. Личность, в верховном смысле, есть откровение Бога для вселенной вселенных.
\vs p001 5:14 \pc Бог, будучи вечным, вселенским, абсолютным и бесконечным, не возрастает ни в знании, ни в мудрости. Бог не обретает опыта, как конечный человек мог бы понимать или предполагать, но он, в сфере собственной личности, обладает тем продолжающимся расширением самореализации, которое в определенном смысле аналогично и сравнимо с обретением нового опыта, получаемого конечными созданиями эволюционирующих миров.
\vs p001 5:15 Абсолютное совершенство бесконечного Бога могло бы стать причиной величественных ограничений неограниченной законченности совершенства, если бы Отец Всего Сущего непосредственно не участвовал бы в борьбе личности каждой несовершенной души в обширной вселенной, которая, с божественной помощью, стремится взойти к высотам духовно совершенных миров. Этот прогрессивный опыт каждого духовного существа и каждого смертного создания повсюду во вселенной вселенных есть часть все более расширяющегося Божественного сознания Отца в бесконечном божественном круге непрекращающейся самореализации.
\vs p001 5:16 Буквально истинно: «Во всех ваших несчастьях он несчастен». «Во всех ваших ликованиях он ликует в вас и с вами». Его предличностный божественный дух есть реальная часть вас. Остров Рая откликается на все физические метаморфозы во вселенной вселенных; Вечный Сын включает все духовные импульсы всех созданий; Носитель Объединенных Действий заключает в себе все выражение разума расширяющегося космоса. Отец Всего Сущего осознает в полноте божественного сознания весь опыт прогрессивной борьбы расширяющихся разумов и восходящих духов каждой сущности и личности всего эволюционирующего творения времени и пространства. И все это буквально истинно, так как «в Нем мы все живем, и движемся, и существуем».
\usection{6. Личность Во Вселенной}
\vs p001 6:1 Человеческая личность есть тень образа божественной личности Создателя, отраженной во времени и в пространстве. И никакая действительность никогда не может быть адекватно воспринята с помощью изучения ее тени. Тени должны интерпретироваться в терминах истинной субстанции.
\vs p001 6:2 \pc Бог есть для науки --- причина, для философии --- идея, для религии --- личность, и даже любящий небесный Отец. Бог есть для ученого первичная сила, для философа --- гипотеза единства, для верующего --- живой духовный опыт. Неадекватная концепция личности Отца Всего Сущего, выдвигаемая человеком, может совершенствоваться только путем духовного прогресса человека во вселенной и станет истинно адекватной, только когда пилигримы времени и пространства в конце концов достигнут божественного объятия живого Бога в Раю.
\vs p001 6:3 Никогда не упускайте из вида противоположных точек зрения на личность, как она представляется Богом и человеком. Человек рассматривает и воспринимает личность, глядя из конечного на бесконечное; Бог смотрит из бесконечного на конечное. Человек обладает самым низшим типом личности; Бог --- наивысшим, даже верховным, предельным и абсолютным. Таким образом, лучшие концепции божественной личности должны терпеливо ожидать более совершенных идей о человеческой личности, особенно, более полного откровения как человеческой, так и божественной личности в урантийской посвященческой жизни Михаила, Сына\hyp{}Творца.
\vs p001 6:4 \pc Предличностный божественный дух, который пребывает в разуме смертного, самим своим присутствием есть веское доказательство своего действительного существования, но концепция божественной личности может быть воспринята только духовно, посредством истинного личного религиозного опыта. Любая личность, божественная или человеческая, может быть познана и понята совершенно независимо от внешних реакций или материального присутствия этой личности.
\vs p001 6:5 Для дружбы между двумя людьми существенной является определенная степень морального сходства и духовной гармонии; любящая личность едва ли сможет открыться нелюбящему. Равно, как и для того, чтобы приблизиться к познанию божественной личности, необходимо полностью сконцентрировать все силы человеческой личности; частичное посвящение, идущее лишь наполовину от сердца, будет безуспешным.
\vs p001 6:6 Чем более полно человек понимает себя и признает личностные ценности своих товарищей, чем больше он будет жаждать познания Первичной Личности, тем более искренне подобный познающий Бога человек будет стремиться стать похожим на Первичную Личность. Вы можете спорить и обмениваться мнениями о Боге, но опыт пребывания с ним и в нем существует свыше и вне всех человеческих противоречий и чистой логики разума. Познающий Бога человек описывает свои духовные опыты не для того, чтобы убедить неверующих, а для наставления и общего удовлетворения верующих.
\vs p001 6:7 \pc Предположить, что вселенная познаваема, что она доступна интеллекту, то же, что предположить, что вселенная создана разумом и управляется личностью. Человеческий разум может понять только феномен разума других разумов, будь то человеческие или сверхчеловеческие. Если человеческая личность может воспринять вселенную, то существует божественный разум и реальная личность, таящаяся где\hyp{}то в этой вселенной.
\vs p001 6:8 \pc Бог есть дух --- духовная личность; человек есть также дух --- потенциальная духовная личность. Иисус из Назарета достиг полной реализации такого потенциала духовной личности в человеческом опыте; таким образом, его жизнь достижения воли Отца становится для человека наиболее реальным и идеальным откровением личности Бога. Несмотря на то, что личность Отца Всего Сущего может быть открыта только в действительном религиозном опыте, в земной жизни Иисуса мы вдохновляемся совершенной демонстрацией такой реализации и откровения личности Бога в истинном человеческом опыте.
\usection{7. Духовная Ценность Концепции Личности}
\vs p001 7:1 Когда Иисус говорил о «живом Боге», он имел в виду личностное Божество --- Отца на небесах. Концепция личности Божества подвигает нас к общности; она покровительствует разумному богопочитанию, она способствует укреплению освежающей доверчивости. Между безличностными сущностями может возникать взаимодействие, но не общность. Общение между отцом и сыном, так же как между Богом и человеком, не может существовать, если оба они не являются личностями. Только личности могут общаться друг с другом, хотя это общение личностей может быть значительно подкреплено присутствием как раз такой неличностной сущности, как Настройщик Мыслей.
\vs p001 7:2 Человек не достигает такого единства с Богом, как капля воды с океаном. Человек достигает божественного объединения с помощью развивающегося взаимного духовного общения посредством личностного сношения с личностным Богом, путем возрастающего достижения божественной природы благодаря идущему от всего сердца и разумному подчинению божественной воле. Такие возвышенные отношения могут существовать только между личностями.
\vs p001 7:3 \pc Концепция истины, по\hyp{}видимому могла бы иметь место без личности, концепция красоты может существовать без личности, но концепция божественной добродетели может быть понята, только в отношении к личности. Только \bibemph{личность} может любить и быть любимой. Даже красота и истина должны расстаться с надеждой на продолжение существования, если они не являются атрибутами личностного Бога, любящего Отца.
\vs p001 7:4 \pc Нам не дано полностью понять, как Бог может быть первичным, неизменным, всемогущим и совершенным и в то же самое время быть окруженным постоянно изменяющейся и развивающейся и явно подчиняющейся законам вселенной, вселенной относительно несовершенной. Но мы в состоянии \bibemph{узнать} такую истину из нашего собственного опыта, поскольку все мы сохраняем идентичность личности и единство воли, несмотря на постоянное изменение как нас самих, так и нашего окружения.
\vs p001 7:5 Предельная реальность вселенной не может быть познана математикой, логикой или философией, но только личным опытом в развивающемся подчинении себя Божественной воле личностного Отца. Ни наука, ни философия, ни теология не могут сделать действительной личность Бога. Только личный опыт верующих Сынов небесного Отца может привести к действительной духовной реализации личности Бога.
\vs p001 7:6 \pc Более высокие концепции вселенской личности подразумевают: идентичность, самосознание, собственную волю и возможность для раскрытия себя. И эти черты в дальнейшем подразумевают братство с другими и равными личностями; такое как существующее в личностных связях Райских Божеств. И абсолютное единство этих связей настолько совершенно, что божественность становится познанной через неделимость. «Господь Бог есть \bibemph{один}». Неделимость личности не является помехой для дарования Богом духа для жизни в сердцах людей. Неделимость личности смертного отца не препятствует воспроизведению смертных сыновей и дочерей.
\vs p001 7:7 Концепция неделимости вместе с концепцией единства подразумевает трансцендентность Предельности Божества как над временем, так и над пространством; таким образом, ни пространство, ни время не могут быть абсолютны или бесконечны. Первоисточник и Центр есть та бесконечность, которая безгранично превосходит весь разум, всю материю, весь дух.
\vs p001 7:8 Факт Райской Троицы никоим образом не противоречит истине божественного единства. Три личности Райского Божества во всех действиях вселенской реальности и во всех связях с созданиями едины. Существование этих трех вечных личностей не нарушает истинности неделимости Божества. Я в полной мере сознаю, что не владею языком, позволяющим выразить достаточно ясно для разума смертного, как эти проблемы вселенной представляются нам. Но это не должно обескуражить вас; не все эти вещи до конца ясны даже высокостоящим личностям, принадлежащим к группе Райских существ. Учтите, что эти глубокие истины, относящиеся к Божеству, будут все более проясняться по мере того, как ваши умы будут становиться все более духовными в последующие эпохи длительного восхождения смертного к Раю.
\vsetoff
\vs p001 7:9 [Представлено Божественным Советником, членом группы небесных личностей, назначенным Древними Дней на Уверсе, центре седьмой сверхвселенной, для наблюдения за теми частями этого откровения, которые описывают вопросы, выходящие за пределы локальной вселенной Небадон. Я уполномочен поддерживать труды, отражающие природу и атрибуты Бога, потому, что представляю наивысший источник информации, доступный для такой цели во всех обитаемых мирах. Я служил Божественным Советником во всех семи сверхвселенных и долго пребывал в Райском центре всех вещей. Много раз наслаждался я высшей радостью от пребывания непосредственно в личном присутствии Отца Всего Сущего. Я отражаю реальность и истинность природы Отца и его атрибутов с неоспоримым авторитетом. Я знаю то, о чем говорю.]
