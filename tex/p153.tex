\upaper{153}{Кризис в Капернауме}
\author{Комиссия срединников}
\vs p153 0:1 В пятницу вечером, в день прибытия в Вифсаиду, и в субботу утром апостолы видели, что Иисус поглощен какой\hyp{}то серьезной проблемой; они понимали, что Учитель глубоко задумался над каким\hyp{}то важным вопросом. Завтракать он не стал и лишь немного поел в полдень. Все субботнее утро и предыдущий вечер двенадцать апостолов и их сподвижники провели вместе, собираясь небольшими группами у дома в саду и на берегу моря. Все они находились в состоянии крайней неопределенности и постоянной тревоги. После того, как они покинули Иерусалим, Иисус практически не разговаривал с ними.
\vs p153 0:2 В течение последних нескольких месяцев им не приходилось видеть, чтобы Учитель был настолько поглощен своими мыслями и так необщителен, как теперь. Даже Симон Петр был удручен, если не сказать подавлен. Андрей не знал, чем помочь своим огорченным товарищам. Нафанаил сказал, что их состояние сейчас словно «затишье перед бурей». Фома высказал мнение, что «должно произойти нечто из ряда вон выходящее». Филипп посоветовал Давиду Зеведееву «забыть о планах насыщения и размещения толпы до тех пор, пока мы не узнаем, о чем думает Учитель». Матфей предпринимал новые усилия для пополнения казны. Иаков и Иоанн говорили о предстоящей проповеди в синагоге и много размышляли о том, в чем, возможно, будет заключаться ее суть и какой круг вопросов будет в ней затронут. Симон Зилот выразил веру, а скорее надежду, на то, что «Отец Небесный, возможно, намерен каким\hyp{}то неожиданным образом вмешаться, чтобы защитить и поддержать своего Сына», тогда как Иуда Искариот осмелился предположить, что Иисус, вероятно, удручен, поскольку сожалеет о том, что «у него не хватило смелости и решимости позволить пяти тысячам провозгласить себя царем евреев».
\vs p153 0:3 И вот от таких подавленных и удрученных последователей Иисус пошел в этот прекрасный субботний день, чтобы произнести свою эпохальную проповедь в капернаумской синагоге. Единственный из его ближайших последователей, кто ободрил его и пожелал добра, был один из ничего не подозревавших близнецов Алфеевых, который, когда Иисус вышел из дома и направился к синагоге, радостно поздоровался с ним и сказал: «Мы молимся, чтобы Отец помог тебе и чтобы у нас было еще больше последователей чем прежде».
\usection{1. Подготовка}
\vs p153 1:1 В этот чудесный субботний день в три часа пополудни в новой капернаумской синагоге Иисуса приветствовало собрание выдающихся представителей религиозного сообщества. Председательствовал Иаир, который и передал Иисусу Писание, чтобы тот его читал. За день до этого из Иерусалима прибыло пятьдесят три фарисея и саддукея; присутствовало также более тридцати лидеров и глав соседних синагог. Эти еврейские религиозные деятели действовали по прямому приказу иерусалимского синедриона и составляли головной отряд ортодоксов, пришедших начать открытую войну с Иисусом и его учениками. Рядом с этими еврейскими лидерами на почетных местах синагоги сидели официальные представители Ирода Антипы, которым было предписано установить истину относительно тревожных донесений о том, что во владениях его брата Филиппа населением была предпринята попытка провозгласить Иисуса царем евреев.
\vs p153 1:2 Иисус понимал, что это прямое объявление его врагами (число которых продолжало увеличиваться) нескрываемой и открытой войны, и решил смело перейти в наступление. Во время насыщения пяти тысяч он бросил вызов их представлениям о материальном Мессии; теперь же он решил снова открыто выступить против их концепции о еврейском избавителе. Эти события, которые начались с насыщения пяти тысяч и закончились этой послеполуденной проповедью, стали тем поворотным моментом, после которого поток народной славы и признания повернул вспять. Впредь дело царства все больше и больше должно было быть связанным с более важной задачей вовлечения в истинно религиозное братство людей тех, кто был воистину духовно обращен. Эта проповедь отражает переход от периода споров, дискуссий и решений к периоду открытой войны и либо окончательного принятия, либо окончательного отвержения.
\vs p153 1:3 Учитель хорошо знал: многие из его последователей медленно, но неуклонно внутренне готовились к тому, чтобы в конце концов отречься от него. Точно так же он знал, что многие из его учеников медленно, но упорно совершенствовали свой ум и приуготовляли душу, что позволит им впоследствии победить сомнения и смело отстаивать свою твердую веру в евангелие царства. Иисус ясно понимал, что люди, находясь в повторяющихся ситуациях, когда нужно выбирать между добром и злом, постепенно готовятся принимать кардинальные решения и без колебаний и смело делать свой выбор. Он непрестанно ставил своих верных вестников в такое положение, когда они начинали испытывать разочарование, и часто предоставлял им возможность проверить себя, выбирая между правильным и ложным путем преодоления духовных испытаний. Он знал, что может положиться на своих последователей, ибо в последнем испытании они примут свои важнейшие решения согласно выработанным прежде и твердым убеждениям разума и духовным реакциям.
\vs p153 1:4 \pc Перелом в земной жизни Иисуса начался с насыщения пяти тысяч и закончился этой проповедью в синагоге; перелом же в жизни апостолов начался с этой проповеди и, продолжаясь целый год, закончился, когда свершились суд над Учителем и его распятие.
\vs p153 1:5 \pc В тот день, когда они сидели в синагоге, перед тем, как Иисус начал говорить, их всех беспокоила только одна загадка, лишь один верховный вопрос. И друзья, и враги Иисуса были поглощены единственной мыслью, и мысль эта была такова: «Почему он сам столь преднамеренно и искусно повернул вспять волну народного энтузиазма?» И перед самым началом проповеди и сразу же после нее сомнения и разочарования его неудовлетворенных приверженцев начали перерастать в неосознанное неприятие и в конце концов вылились в настоящую ненависть. Именно после этой проповеди в синагоге Иуду Искариота впервые посетила осознанная мысль покинуть Иисуса. Однако он какое\hyp{}то время умело подавлял в себе подобные желания.
\vs p153 1:6 Все пребывали в состоянии недоумения. Иисус их ошеломил и озадачил. Еще недавно он представил величайшее доказательство сверхъестественной силы всего его земного пути. Насыщение пяти тысяч было тем деянием его земной жизни, которое более всего импонировало еврейскому представлению об ожидаемом Мессии. Однако это чрезвычайно благоприятное положение было тотчас же и необъяснимо сведено на нет его немедленным и недвусмысленным отказом стать царем.
\vs p153 1:7 В пятницу вечером и в субботу утром иерусалимские религиозные деятели долго и серьезно уговаривали Иаира отменить выступление Иисуса в синагоге, но бесполезно. Единственный ответ Иаира на все эти уговоры был таков: «Я ответил согласием на эту просьбу, и слова своего не нарушу».
\usection{2. Эпохальная проповедь}
\vs p153 2:1 Эту проповедь Иисус начал, прочтя из закона, в книге Второзаконие: «Если народ сей не будет слушать гласа Божьего, то проклятия преступления постигнут его. Предаст вас Господь на поражение врагам вашим; вы будете рассеяны по всем царствам земли. И отведет Господь вас и царя вашего, которого вы поставите над собою, в руки чужого народа. И будете удивлением, притчею и поговоркой у всех народов. Сыновья и дочери ваши пойдут в плен. Пришельцы среди вас возвысятся во власти, а вы опуститесь весьма низко. И все сие постигнет вас и семя ваше навечно, потому что вы не слушали слова Господнего. Посему будете служить врагам вашим, которые придут против вас. Будете терпеть голод и жажду и носить чужое ярмо из железа. Пошлет на вас Господь народ издалека, от края земли, народ, которого языка вы не будете разуметь, народ с лицом свирепым, народ, который будет мало с вами считаться и будет осаждать вас во всех городах ваших, доколе не падут высокие крепкие стены, которым вы доверяли; и вся земля перейдет в руки его, и вынуждены будете есть плод чрева вашего, плоть сыновей и дочерей ваших во время этой осады из\hyp{}за стеснения, в которое ввергнут вас враги ваши».
\vs p153 2:2 Закончив же это чтение, Иисус перешел к Пророкам и прочел из Книги Иеремии: «„Если вы не будете слушать слов слуг моих, пророков, которых я послал вам, то я сделаю с домом сим то же, что с Силомом, и город сей сделаю проклятием для всех народов земли“. И священники и учителя слушали, как Иеремия говорил сии слова в доме Господнем. И когда Иеремия сказал все, что Господь повелел ему сказать всему народу, священники и учителя схватили его и сказали: „Ты должен умереть“. И столпился народ вокруг Иеремии в доме Господнем. И когда услышали сие князья иудейские, стали судить Иеремию. Тогда сказали священники и учителя князьям и всему народу, говоря: „Смертный приговор --- этому человеку, потому что пророчествовал против города сего, и вы своими ушами слышали его“. Тогда сказал Иеремия всем князьям и всему народу: „Господь послал меня пророчествовать против дома сего и против города сего все те слова, которые вы слышали. Итак, исправьте пути ваши и обновите деяния ваши и послушайтесь гласа Господа, Бога вашего, чтобы избежать вам зла, изреченного против вас. А что до меня, вот, я --- в ваших руках. Делайте со мной, что в глазах ваших хорошо и справедливо. Только твердо знайте: если вы умертвите меня, то невинную кровь возложите на себя и на народ сей, ибо истинно Господь послал меня сказать все эти слова в уши ваши“.
\vs p153 2:3 Священники и учителя того времени пытались убить Иеремию, однако судьи не согласились с этим, хотя за слова предостережения спустили его на веревках в грязную яму, пока не оказался он в грязи до подмышек. Вот что народ сей сделал с Пророком Иеремией, когда он послушался приказа Господнего предупредить братьев своих об их предстоящей политической гибели. Я же сегодня желаю спросить вас: что сделают первосвященники и религиозные лидеры народа сего с человеком, который осмелится предостеречь их о дне их духовной гибели? Постараетесь ли и вы послать на смерть учителя, который осмеливается провозгласить слово Господа и не боится указать, где вы отказываетесь идти по пути света, ведущему ко входу в царство небесное?
\vs p153 2:4 Какого свидетельства о моей миссии на земле вы ищите? Мы не нарушали ваш покой в местах вашего влияния и власти, а проповедовали благую весть бедным и отверженным. Мы не делали враждебных выпадов против того, что вы почитаете, а возвещали новую свободу для охваченной страхом души человека. Я пришел в этот мир открывать Отца моего и установить на земле духовное братство сыновей Бога, царство небесное. И несмотря на то, что я столько раз напоминал вам, что царство мое не от мира сего, Отец мой все же даровал вам множество материальных чудес в дополнение к еще более очевидным проявлениям духовного преобразования и возрождения.
\vs p153 2:5 Какого нового знамения ищите вы в руках моих? Заявляю вам: вы уже имеете достаточно свидетельств, позволяющих вам принять ваше решение. Истинно, истинно говорю многим, сидящим предо мной в этот день: вы стоите перед необходимостью выбора, по какому пути идти; и говорю вам, как говорил Иисус Навин предкам вашим: „Изберите себе ныне, кому служить“. Сегодня многие из вас стоят на распутье.
\vs p153 2:6 Некоторые из вас, не сумев найти меня после пиршества толпы на другой стороне озера, наняли все рыбачьи лодки Тивериады, которые неделей раньше неподалеку отсюда нашли убежище во время бури, и зачем же? Не ради истины и праведности или не затем, чтобы лучше узнать, как служить и помогать своим собратьям! Но затем, чтобы иметь больше хлеба, ради которого вы не трудились. Не для того, чтобы насытить души ваши словом жизни, а затем лишь, чтобы наполнить живот полученным даром хлебом. Вас долго учили: когда придет Мессия, он совершит те чудеса, которые сделают для всего избранного народа жизнь легкой и приятной. Неудивительно поэтому, что вы, кого так учили, должны желать хлебов и рыбы. Но я объявляю вам: не в этом миссия Сына Человеческого. Я пришел возвещать духовную свободу, учить вечной истине и воспитывать живую веру.
\vs p153 2:7 Братья мои, не желайте пищи, которая тленна, но ищите духовной пищи, которая даст вам войти в жизнь вечную; это хлеб жизни, который Сын дает всем желающим его взять и есть, ибо Отец дал Сыну этой жизни без меры. И когда вы спрашивали меня: „Что нам делать, чтобы творить дела Бога?“, я прямо говорил вам: „Вот дело Бога --- чтобы вы верили в того, кого он послал“».
\vs p153 2:8 И затем, указав на декорированный гроздьями винограда сосуд с манной, украшавший притолоку двери в этой новой синагоге, Иисус сказал: «Вы думали, что предки ваши ели в пустыне манну --- хлеб небесный, --- а я говорю вам: это был хлеб земной. Хотя Моисей не дал отцам вашим хлеба небесного, ныне Отец мой готов дать вам истинного хлеба жизни. Хлеб же небесный есть тот, который сходит от Бога и дает жизнь вечную людям мира. И когда вы будете говорить мне: Дай нам сего хлеба живого, я буду отвечать: я есть этот хлеб жизни. Приходящий ко мне не будет алкать, и верующий в меня не будет жаждать никогда. Вы видели меня, жили со мной, наблюдали мои труды и все же не верите, что я пришел от Отца. Однако те, кто верует, --- не бойтесь. Все, ведомые от Отца, придут ко мне, и приходящий ко мне ни в коем случае не будет изгнан вон.
\vs p153 2:9 А теперь позвольте мне раз и навсегда объявить вам, что я сошел на землю не для того, чтобы творить волю мою, но волю Пославшего меня. Окончательная же воля Пославшего меня есть та, чтобы из тех, кого он дал мне, я не потерял никого. И воля Отца есть та, чтобы всякий, видящий Сына и верующий в него, имел жизнь вечную. Только вчера я насытил вас хлебом для тел ваших; сегодня же я предлагаю вам хлеб жизни для ваших жаждущих душ. Возьмете ли ныне хлеб духа с такой же готовностью, с какой тогда ели хлеб мира сего?»
\vs p153 2:10 \pc Когда Иисус на мгновение прервал свою речь, чтобы окинуть взором собравшихся, один из учителей из Иерусалима (член синедриона) поднялся и спросил: «Верно ли я понимаю: ты сказал, что ты есть хлеб, сходящий с неба, и что манна, которую Моисей дал отцам нашим в пустыне, им не была?» И Иисус ответил фарисею: «Ты все правильно понял». Тогда фарисей сказал: «Не ты ли Иисус из Назарета, сын плотника Иосифа? Разве не знакомы многим из нас отец твой и мать, а также братья твои и сестры? Как же тогда являешься ты сюда, в дом Божий, и объявляешь, что ты сошел с неба?»
\vs p153 2:11 К этому времени в синагоге поднялся сильный ропот, и он угрожал перерасти в такой сильный шум, что Иисус встал и сказал: «Наберемся терпения; честное изучение не пойдет во вред истине. Я есть все, что ты сказал, но не только это. Отец и я --- одно; Сын делает лишь то, чему учит его Отец; всех же, кто дан Сыну Отцом, Сын примет к себе. Вы читали у Пророков: „И будут все научены Богом“ и что „Те, кого учит Отец, услышат и Сына его“. Всякий, следующий учению духа Отца, пребывающего в нем, в конце концов придет и ко мне. Это не значит, что человек видел Отца, но дух Отца живет в человеке. Сын, сошедший с небес, --- он, несомненно, видел Отца. Истинно же верующие в Сына сего уже имеют жизнь вечную.
\vs p153 2:12 Я --- хлеб жизни. Отцы ваши ели манну в пустыне и умерли. Сей же хлеб, сходящий от Бога, таков, что тот, кто его ест, никогда не умрет в духе. Повторяю: я --- этот хлеб живой, и всякая душа, достигшая понимания этой единой природы Бога и человека, будет жить вечно. И этот хлеб жизни, который я даю всем желающим принять его, есть моя собственная живая и совокупная сущность Богочеловека. Отец в Сыне и Сын, единый с Отцом, --- вот мое животворящее откровение миру и мой спасительный дар всем народам».
\vs p153 2:13 Когда Иисус кончил говорить, управитель синагоги распустил собрание, но люди не хотели расходиться. Одни толпились вокруг Иисуса, чтобы задать ему новые вопросы, тогда как другие роптали и спорили между собой. И так продолжалось более трех часов. Когда же слушатели наконец разошлись, было уже больше семи часов.
\usection{3. После проповеди}
\vs p153 3:1 После проповеди Иисусу было задано много вопросов. Одни из них задавали его сбитые с толку ученики, но еще больше вопросов задавали придирчивые неверующие, которые пытались лишь смутить его или поймать в ловушку.
\vs p153 3:2 Так, один из пришедших фарисеев залез на подставку светильника и выкрикнул такой вопрос: «Ты говоришь нам, что ты --- хлеб жизни. Как можешь ты дать нам есть твою плоть и пить твою кровь? Какой прок в твоем учении, если оно неосуществимо?» Иисус же ответил на этот вопрос, говоря: «Я не учил вас, что моя плоть --- хлеб, а кровь --- вода жизни. Но я, действительно, сказал, что моя жизнь во плоти есть пришествие хлеба небесного. Сам факт Слова Бога, ниспосланного во плоти, и явление Сына Человеческого, подвластного воле Бога, составляют сущность опыта, равносильную божественной пище. Вы не можете ни есть мою плоть, ни пить мою кровь, но вы можете в духе стать со мной одним целым, так же как я един в духе с Отцом. Вы можете питаться вечным словом Бога, которое, действительно, есть хлеб жизни и которое явлено в подобии смертной плоти; и можете утолять жажду души божественным духом, который истинно есть вода жизни. Отец послал меня в мир, дабы показать, как желает он пребывать во всех людях и направлять их; и я так жил жизнью во плоти, чтобы также вдохновлять всех людей постоянно пытаться узнать и исполнить волю пребывающего в них Отца Небесного».
\vs p153 3:3 Тогда один из иерусалимских шпионов, которые следили за Иисусом и его апостолами, сказал: «Мы замечаем, что ни ты, ни твои апостолы не умываете рук ваших, как подобает, перед тем, как есть хлеб. Вам должно быть известно, что такие действия, как есть хлеб грязными и немытыми руками, суть нарушение закона старейшин. Вы также как следует не моете кружки, из которых пьете, и посуду, из которой едите. Почему вы выказываете такое неуважение к традициям отцов и законам наших старейшин?» Услышав, что он говорит, Иисус сказал: «Почему вы законами традиций ваших нарушаете заповеди Божьи? Заповедь гласит: „почитай отца и мать“ и повелевает, чтобы вы при необходимости делили с ними ваше имущество; вы же прибегаете к закону традиции, который позволяет лишенным чувства долга детям говорить, что деньги, которыми можно было помочь родителям, „отданы Богу“. Закон старейшин, таким образом, избавляет подобных хитроумных детей от их обязанности, несмотря на то, что эти дети в дальнейшем пользуются этими деньгами для своего собственного удовольствия. Почему вы таким образом нарушаете заповедь вашими же традициями? Хорошо пророчествовал о вас, лицемерах, Исайя, говоря: „Люди сии чтут меня устами своими, сердца же их далеко отстоят от меня. Тщетно поклоняются мне, уча учениям своим, заповедям человеческим“
\vs p153 3:4 Вы сами видите, как, забыв заповедь, вы твердо придерживаетесь традиций человеческих. И вы вполне готовы отвергнуть слово Бога, а свои собственные традиции соблюдаете. И многими иными путями осмеливаетесь ставить свои собственные учения выше закона и пророков».
\vs p153 3:5 Затем Иисус обратился ко всем присутствовавшим. Он сказал: «Послушайте меня все. Не то, что входит в уста, духовно оскверняет человека, но то, что выходит из уст и из сердца». Однако даже апостолы не сумели до конца понять смысл его слов, ибо Симон Петр также спросил его: «Чтобы не оскорбить без надобности некоторых из твоих слушателей, не объяснишь ли нам смысл этих слов?» И тогда Иисус сказал Петру: «Неужели так трудно понять? Разве не знаете вы, что всякое растение, которое не Отец мой Небесный посадил, искоренится? Обратите теперь внимание ваше на тех, кто хотел бы узнать истину. Людей нельзя заставить любить истину принуждением. Многие из учителей этих --- слепые вожди. А вы знаете: если слепой поведет слепого, оба упадут в яму. Внимайте, когда я говорю вам истину о том, что нравственно оскверняет и духовно разлагает людей. Я объявляю: не то, что входит в тело через уста или проникает в разум через глаза и уши, оскверняет человека. Человека оскверняет лишь зло, которое может возникать в сердце и которое находит выражение в словах и поступках подобных нечестивых людей. Разве не знаете вы, что из сердца исходят злые мысли, порочные замыслы убийства, кражи и прелюбодеяния, а также ревность, гордость, злоба, месть, брань и лжесвидетельства? Именно это и оскверняет людей, а не то, что они едят хлеб нечистыми с формальной точки зрения руками».
\vs p153 3:6 Фарисеи\hyp{}уполномоченные иерусалимского синедриона теперь были почти убеждены, что Иисус должен быть схвачен по обвинению в богохульстве или в глумлении над священным законом евреев; этим и объясняются их попытки втянуть его в обсуждение некоторых традиций старейшин или так называемых неписаных законов народа и, возможно, выступить против них. Независимо от того, насколько скудны могли быть запасы воды, эти порабощенные традициями евреи никогда не забывали совершить необходимый обряд омовения рук перед каждой едой. Их вера была такова, что «лучше умереть, чем нарушить заповеди старейшин». Шпионы задали этот вопрос потому, что, согласно доносу, Иисус говорил: «Спасение зависит от чистоты сердец, а не от чистоты рук». Однако от подобных верований трудно избавиться, если они стали частью религиозных верований человека. Даже спустя много лет после этого дня Апостол Петр боялся нарушить многие из этих традиций, касавшихся чистого и нечистого, и окончательно избавился от этого страха лишь после необыкновенного и яркого сновидения. Все это можно лучше понять, вспомнив о том, что эти евреи относились к еде немытыми руками так же, как к общению с блудницей, причем и то, и другое одинаково наказывалось отлучением.
\vs p153 3:7 Итак, Учитель решил обсудить и разоблачить нелепость всей раввинской системы правил и предписаний, представленной неписаным законом --- традициями старейшин, которые все почитались еще более священными и обязательными для евреев, нежели даже учения Писания. И Иисус высказывался менее сдержанно, поскольку знал, что пришел час, когда он уже не может ничего сделать, чтобы предотвратить открытый разрыв с религиозными правителями.
\usection{4. Последние слова в синагоге}
\vs p153 4:1 Во время дискуссий, происходивших после проповеди, один из фарисеев из Иерусалима привел к Иисусу обезумевшего юношу, одержимого непокорным и мятежным духом. Подведя этого сумасшедшего мальчика к Иисусу, он сказал: «Что можешь ты сделать с такой болезнью, как эта? Можешь ли ты изгнать бесов?» Посмотрев на юношу, Учитель исполнился сострадания и, подозвав его к себе, взял его за руку и сказал: «Ты знаешь, кто я; выйди из него; я также приказываю одному из твоих лояльных собратьев проследить за тем, чтобы ты не вернулся». И немедленно мальчик стал нормальным и в здравом уме. Это был первый случай, когда Иисус действительно изгнал «нечистого духа» из человеческого существа. Все предыдущие были лишь случаями мнимой одержимости бесом; это же был случай подлинной одержимости, такой, какие иногда происходили в то время и вплоть до дня Пятидесятницы, когда дух Учителя излился на всякую плоть и навсегда лишил этих нескольких небесных мятежников возможности так обманывать определенные типы неуравновешенных людей.
\vs p153 4:2 Когда народ изумился случившемуся, один из фарисеев встал и обвинил Иисуса в том, что он смог сделать подобное, поскольку находился в сговоре с бесами; и что языком, которым Иисус пользовался, изгоняя этого беса, он подтвердил, что они с бесом знают друг друга; затем фарисей заявил, что религиозные учителя и лидеры в Иерусалиме пришли к заключению, что Иисус совершил все свои так называемые чудеса силой Вельзевула, князя бесовского. Фарисей сказал: «Не имейте никаких дел с этим человеком; он заодно с Сатаной».
\vs p153 4:3 Тогда Иисус сказал: «Как может Сатана изгонять Сатану? Царство, разделившееся в самом себе, не может устоять; если дом разделится в самом себе, то вскоре опустеет. Может ли город выдержать осаду, если в нем нет единства? Если Сатана изгоняет Сатану, значит, он разделился в самом себе; как же тогда устоит его царство? Но вы должны знать, что никто не может войти в дом сильного человека и расхитить вещи его, если прежде не пересилит и не свяжет того сильного человека. Итак, если я силою Вельзевула изгоняю бесов, то сыновья ваши чьей силой изгоняют их? Посему они будут вам судьями. Если же я духом Бога изгоняю бесов, значит, истинно пришло к вам царство Бога. Если бы вас не ослепляли предрассудки и не сбивали с пути страх и гордость, вы бы без труда поняли, что больший бесов стоит посреди вас. Вы вынуждаете меня заявить: кто не со мною, тот против меня, и кто не собирает со мною, тот расточает. Позвольте мне серьезно предостеречь вас, осмелившихся преднамеренно со злым умыслом и сознательно приписывать деяния Бога проискам бесов! Истинно, истинно говорю вам: все грехи ваши будут прощены, даже все хуления ваши, но кто будет хулить Бога обдуманно и злонамеренно, тот не получит прощения вовек. Поскольку те, кто так усердно вершит беззакония, никогда не будут искать прощения и никогда не получат его, они повинны в грехе вечного отвержения божественного прощения.
\vs p153 4:4 Многие из вас сегодня стоят на распутье; вы подошли к тому моменту, когда должны сделать неизбежный выбор между волей Отца и путями тьмы, избранными вами самими. И какой выбор вы сделаете сейчас, такими в конце концов и станете. Вы должны либо сделать дерево хорошим и плоды его хорошими, либо дерево станет худым и плоды его худыми. Объявляю вам: в вечном царстве Отца моего дерево познается по плодам. Но некоторые из вас подобны змеям, как можете вы, уже избрав зло, приносить хорошие плоды? От бездны зла в сердцах ваших говорят ваши уста».
\vs p153 4:5 Тогда встал другой фарисей и сказал: «Хотелось бы нам, чтобы ты дал нам предопределенное знамение, с которым мы согласились бы как с доказательством твоей власти и права учить. Готов ли ты пойти на такое соглашение?» Услышав это, Иисус сказал: «Это неверующее и ищущее знамения поколение ищет примету, но не будет дано вам иного знамения, кроме того, которое вы уже имеете и которое увидите, когда Сын Человеческий покинет вас».
\vs p153 4:6 И когда Иисус закончил говорить, апостолы окружили его и вывели из синагоги. Молча они пошли вместе с ним домой в Вифсаиду. Все они были поражены и до какой\hyp{}то степени охвачены ужасом от резкой перемены в поведении Учителя. Они совершенно не привыкли видеть, чтобы он поступал подобным воинственным образом.
\usection{5. В субботу вечером}
\vs p153 5:1 Иисус неоднократно развенчивал надежды своих апостолов и постоянно сокрушал их самые лелеемые ожидания, однако ни одно разочарование и никакая печаль не могли сравниться с тем, что теперь овладело ими. К тому же теперь к их состоянию угнетенности примешивался страх за свою безопасность. Все они были крайне напуганы тем, как внезапно народ отвернулся от них. Они также были отчасти напуганы и смущены неожиданной дерзостью и самоуверенной решимостью фарисеев, пришедших из Иерусалима. Но больше всего их сбивала с толку резкая перемена в поведении Иисуса. При обычных обстоятельствах они бы одобрили эту более воинственную манеру, однако то, что ей сопутствовали многие другие неожиданные события, испугало их.
\vs p153 5:2 И теперь, когда они пришли домой, ко всем этим треволнениям Иисус еще и отказался есть. Несколько часов он провел в одиночестве в комнате наверху. Была уже полночь, когда вернулся Иоав, старший из евангелистов, и сказал, что почти треть его сподвижников бросила дело. Весь вечер приходили и уходили верные ученики, сообщая, что внезапная перемена чувств к Учителю была общей в Капернауме. Религиозные лидеры из Иерусалима не упустили возможности усилить это чувство недовольства и всеми мыслимыми способами старались поддержать отступничество от Иисуса и его учений. На протяжении этих трудных часов в доме Петра находились двенадцать женщин. Они были страшно расстроены, но ни одна из них не ушла.
\vs p153 5:3 Было чуть больше полуночи, когда Иисус спустился из верхней комнаты и встал среди двенадцати и их сподвижников, которых всего было около тридцати. Он сказал: «Я понимаю, что это отступничество царства причиняет вам страдание, но оно неизбежно. И все же после всей полученной вами подготовки, могли ли мои слова стать достаточной причиной того, чтобы усомниться? Почему вы преисполнились страха и ужаса, увидев, что царство лишилось этих равнодушных толп и слабых духом учеников? Почему горюете, когда наступает новый день для воссияния новой славы духовных учений царства небесного? Если вы находите, что это слишком трудное испытание, то что же вы будете делать, когда Сын Человеческий должен будет вернуться к Отцу? Где и как вы будете приуготовлять себя ко времени, когда я вознесусь туда, откуда пришел в этот мир?
\vs p153 5:4 Возлюбленные мои, вы должны помнить, что животворит дух; от плоти же и всего, имеющего отношение к ней, мало пользы. Слова, которые я сказал вам, суть дух и жизнь. Ободритесь! Я не оставил вас. Многих обидят откровенные речи этих дней. Уже слышали вы, что многие ученики мои отвернулись от меня и больше не ходят со мной. От самого начала я знал, что эти слабые духом верующие отпадут по пути. Разве не избрал я вас, двенадцать человек, и не выделил вас как посланников царства? И ныне, в такое время, как это, неужели и вы оставите меня? Пусть каждый из вас оценит свою собственную веру, ибо одному из вас грозит смертельная опасность». И когда Иисус кончил говорить, Симон Петр сказал: «Да, Господи, мы опечалены и смятены, но мы никогда не покинем тебя. Ты учил нас словам вечной жизни. Мы уверовали в тебя и следовали за тобой все это время. Мы не повернем назад, потому что знаем: ты послан Богом». И когда Петр перестал говорить, все единодушно в знак одобрения его обета верности кивнули головой.
\vs p153 5:5 Тогда Иисус сказал: «Ступайте отдыхать, ибо на нас надвигаются беспокойные времена впереди у нас много дел».
