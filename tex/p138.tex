\upaper{138}{Подготовка посланцев царства}
\author{Комиссия срединников}
\vs p138 0:1 После проповеди «О Царстве» Иисус созвал всех шестерых апостолов после полудня и стал открывать им свои планы посещения городов, расположенных в окрестности Галилейского моря. Его братья Иаков и Иуда были очень задеты тем, что их не пригласили на эту встречу. До сих пор они считали, что принадлежат к приближенным сподвижникам Иисуса. Однако Иисус не планировал иметь близких родственников среди членов этого отряда апостольских наставников царства. Отказ ввести Иакова с Иудой в число немногих избранных, а также кажущаяся отчужденность от матери после происшедшего в Кане стали причиной все увеличивавшегося отчуждения между Иисусом и его семьей. Эта ситуация продолжалась на протяжении всего его публичного служения --- семья почти отвергла его, --- и эти расхождения не были до конца устранены вплоть до его смерти и воскресения. Его мать все время колебалась между постоянно изменяющейся верой и надеждой на него и все более возраставшими чувствами разочарования, унижения и отчаяния. Только Руфь, младшая, неизменно оставалась верна своему отцу\hyp{}брату.
\vs p138 0:2 До воскресения Иисуса вся его семья была очень мало связана с его служением. Если пророк всюду находит почитание, кроме собственного отечества, то он и всюду находит понимание, кроме собственной семьи.
\usection{1. Последние наставления}
\vs p138 1:1 На следующий день, в воскресенье 23 июня, 26 года н.э., Иисус дал последние наставления шести апостолам. Он повелел им отправиться по двое, чтобы учить благой вести о царстве. Он запретил им крестить и рекомендовал воздерживаться от публичных проповедей. Он продолжал объяснять, что позже позволит им публично проповедовать, но пока на какое\hyp{}то время по многим причинам желал бы, чтобы они приобрели практический опыт личного общения со своими собратьями. Иисус предложил им, чтобы первое их путешествие было посвящено целиком \bibemph{индивидуальной работе.} Хотя это сообщение несколько разочаровало апостолов, но все же они понимали, хотя бы отчасти, почему Иисус решил таким образом начать возвещение царства, и они отправились с легким сердцем и твердым энтузиазмом. Он отправил их парами, Иаков с Иоанном направились в Хересу, Андрей и Петр --- в Капернаум, Филипп и Нафанаил --- в Тарихею.
\vs p138 1:2 Прежде чем они начали эти первые две недели служения, Иисус объявил им, что хочет посвятить двенадцать апостолов, которые продолжали бы работу царства после его ухода, и дал право каждому из них выбрать одного человека среди новообращенных для того, чтобы включить их в предполагаемый отряд апостолов. И Иоанн откровенно спросил: «Учитель, неужели эти шесть человек войдут в наш круг и все станут равными с нами, которые были с тобой еще с Иордана и слышали все твое учение, подготовившее нас к нашему первому труду для царства?» Иисус ответил: «Да, Иоанн, эти люди, которых вы выберете, станут едины с нами, и вы научите их всему, что касается царства, так же, как я учил вас». Сказав так, Иисус оставил их.
\vs p138 1:3 Но шестеро не разошлись, чтобы приступить к своим трудам, пока долго и многословно не обсудили задание учителя избрать каждому нового апостола. Наконец победило мнение Андрея, и они отправились каждый по своим делам. Андрей, по сути, сказал: «Учитель прав: нас слишком мало для такой работы. Нужны еще учителя, и Учитель проявил великое доверие к нам, поручив нам самим выбрать этих шестерых новых апостолов». Этим утром, когда они расставались, чтобы отправиться к своим трудам, у каждого была затаенная печаль на сердце. Они знали, что им будет не хватать Иисуса и, кроме того, испытывали страх и робость, и совсем не так представляли себе возвещение царства.
\vs p138 1:4 Было условлено, что шестеро должны работать в течение двух недель, а затем вернуться в дом Зеведея для совещания. Иисус в это время отправился в Назарет навестить Иосифа с Симоном и других членов своей семьи, живших в той округе. Иисус делал все, что было в человеческих силах и совместимо с его предназначением следовать воле Отца, чтобы сохранить доверие и любовь семьи. В этом отношении он выполнял свой долг полностью и более того.
\vs p138 1:5 В то время как апостолы отсутствовали, исполняя свою миссию, Иисус много думал об Иоанне, теперь находящемся в тюрьме. Он боролся с искушением использовать свои потенциальные силы, чтобы освободить его, --- но вновь он смирился с тем, чтобы «ожидать воли Отца».
\usection{2. Избрание шести}
\vs p138 2:1 Это первое миссионерское путешествие шести апостолов было чрезвычайно успешным. Каждый из них открыл для себя великую ценность прямого личного контакта с людьми. Они возвратились к Иисусу, понимая более ясно, чем прежде, что в конечном счете религия есть целиком и полностью \bibemph{личный опыт.} Они впервые ощутили, как недостает простым людям слов религиозного утешения и духовной поддержки. Когда они собрались вокруг Иисуса, все они хотели сразу высказаться, но Андрей взял на себя руководство и вызывал всех по очереди, чтобы они официально отчитались перед Учителем и представили своих кандидатов в число шести новых апостолов.
\vs p138 2:2 После того, как каждый кандидат на избрание в качестве нового апостола был представлен, Иисус просил всех остальных голосовать отдельно за каждую кандидатуру; таким образом шестеро новых апостолов были формально приняты шестью прежними. Затем Иисус сказал, что они посетят всех кандидатов и призовут их на служение.
\vs p138 2:3 Новоизбранными апостолами были:
\vs p138 2:4 \ublistelem{1.}\bibnobreakspace \bibemph{Матфей Леви,} сборщик налогов из Капернаума, исполнявший свою должность к востоку от города, на границе с Батанией. Он был избран Андреем.
\vs p138 2:5 \ublistelem{2.}\bibnobreakspace \bibemph{Фома Дидымус,} рыбак из Тарихеи, некогда плотник и каменщик из Гадары. Он был избран Филиппом.
\vs p138 2:6 \ublistelem{3.}\bibnobreakspace \bibemph{Иаков Алфеев,} рыбак и крестьянин из Хоразина, избран Иаковом Зеведеем.
\vs p138 2:7 \ublistelem{4.}\bibnobreakspace \bibemph{Иуда Алфеев,} брат\hyp{}близнец Иакова Алфеева, также рыбак, был избран Иоанном Зеведеем.
\vs p138 2:8 \ublistelem{5.}\bibnobreakspace \bibemph{Симон Зилот ---} офицер высокого ранга в патриотической организации зилотов, отказавшийся от этого положения, чтобы стать апостолом Иисуса. Прежде чем присоединиться к зилотам, Симон был купцом. Он был избран Петром.
\vs p138 2:9 \ublistelem{6.}\bibnobreakspace \bibemph{Иуда Искариот ---} единственный сын богатой еврейской семьи, живущей в Иерихоне. Он стал последователем Иоанна Крестителя, и родители --- саддукеи --- отреклись от него. Он искал работу в этих местах, когда его встретили апостолы Иисуса; Нафанаил пригласил его присоединиться к ним, главным образом потому, что он был опытен в финансовых делах. Иуда Искариот был единственным иудеем среди двенадцати апостолов.
\vs p138 2:10 \pc Иисус провел весь день со своими первыми шестью апостолами, отвечая на их вопросы и слушая рассказы об их путешествиях, потому что они обрели интересный и полезный опыт, которым могли поделиться. Они убедились теперь в мудрости плана Учителя, пославшего их на скромную индивидуальную работу с людьми, прежде чем допустить к более активной публичной деятельности.
\usection{3. Призвание Матфея и Симона}
\vs p138 3:1 На следующий день Иисус и шестеро апостолов отправились к Матфею, сборщику налогов. Матфей ожидал их, подведя все балансы в своих книгах и приготовившись передать дела своему брату. Когда они приблизились к помещению, где взимались пошлины, Андрей вышел вперед вместе с Иисусом, который, глядя в глаза Матфею, сказал: «Следуй за мной». И тот встал и пошел к своему дому с Иисусом и апостолами.
\vs p138 3:2 Матфей сказал Иисусу о празднестве, которое он собирается устроить этим вечером, по крайней мере, он хотел бы дать такой пир для семьи и друзей, если Иисус одобрит это и согласится быть почетным гостем. Иисус кивком выразил согласие. Тогда Петр отвел Матфея в сторону и, сказав, что он пригласил некоего Симона присоединиться к апостолам, получил разрешение и его позвать на празднество.
\vs p138 3:3 \pc После полуденной трапезы в доме Матфея все пошли с Петром к Симону Зилоту, которого нашли у его прежней купеческой лавки, где его сменил теперь племянник. Когда Петр подвел Иисуса к Симону, Учитель приветствовал пылкого патриота и сказал лишь: «Следуй за мной».
\vs p138 3:4 \pc Все они возвратились в дом Матфея и проговорили там о политике и религии до часа вечерней трапезы. Семья Леви долгое время занималась торговлей и сбором пошлин, поэтому многих из гостей, приглашенных Матфеем на этот пир, фарисеи отнесли бы к «мытарям и грешникам».
\vs p138 3:5 В те дни было обычаем, что когда видный человек устраивал прием такого рода, все желающие могли прийти в помещение, где он происходил, --- чтобы посмотреть на гостей за едой, послушать беседы и речи значительных людей. Соответственно, большинство фарисеев Капернаума собрались по этому случаю, чтобы увидеть, как будет вести себя Иисус на этом необычном публичном сборище.
\vs p138 3:6 Во время пира радостное настроение его участников постепенно переросло в прекрасное расположение духа, и видно было, что они искренне наслаждаются застольем, так что наблюдавшие за этим фарисеи в душе стали критиковать Иисуса за участие в таком беспечном и веселом времяпрепровождении. Позднее в этот вечер, когда перешли к речам, один из наиболее злобствующих фарисеев зашел так далеко, что стал укорять Иисуса и, обращаясь к Петру, сказал: «Как смеете вы учить тому, что этот человек праведен, если он ест с мытарями и грешниками и вместе с ними проводит время в досужем веселье?» Петр успел шепотом передать это обвинение Иисусу прежде, чем тот произнес свое прощальное благословение участникам празднества. Когда Иисус начал говорить, он сказал: «Придя сюда сегодня, чтобы приветствовать Матфея и Симона, которые стали членами нашего братства, я рад видеть веселье и прекрасное настроение в вашем обществе, но возвеселитесь еще больше, ибо многие из вас войдут в грядущее царство духа, где вы испытаете еще большую радость от богатств царства небесного. А вам, которые стоят вокруг, осуждая меня в сердце своем за то, что я пришел сюда, чтобы повеселиться с этими друзьями, я хочу сказать, что пришел на землю возвестить радость угнетенным обществом и духовную свободу --- порабощенным моралью. Нужно ли напоминать вам, что не здоровым требуется врач, но больным? Я пришел звать не праведных, но грешников».
\vs p138 3:7 Воистину, это было действительно необычно для еврейства --- видеть человека, праведного по натуре и с благородными чувствами, который свободно и радостно общается с простыми людьми, даже с далекими от религии и любящими удовольствия мытарями и отъявленными грешниками. Симон Зилот пожелал было произнести речь на этом сборище в доме Матфея, но Андрей, помня, что Иисус не хочет смешивать движение зилотов с делами грядущего царства, убедил его воздержаться от публичных выступлений.
\vs p138 3:8 Иисус и апостолы остались на эту ночь в доме Матфея, и люди, расходясь по домам, только и говорили, о доброте и дружелюбии Иисуса.
\usection{4. Призвание близнецов}
\vs p138 4:1 Утром они вдевятером на лодке отправились в Хоразин, чтобы совершить формальное призвание следующих двух апостолов, близнецов Иакова и Иуду, сыновей Алфея, являвшихся кандидатами Иакова и Иоанна, сыновей Зеведея. Братья\hyp{}рыбаки ждали Иисуса с его апостолами и поэтому поджидали их на берегу. Иаков Зеведей представил Учителя хоразинским рыбакам; Иисус, пристально взглянув на них, кивнул и сказал: «Следуйте за мной».
\vs p138 4:2 \pc Во время после полудня, которое они провели вместе, Иисус дал им исчерпывающее наставление об участии в празднествах, завершив его словами: «Все люди --- мои братья. Мой небесный Отец не презирает ни одно создание, сотворенное нами. Царство небесное открыто всем мужчинам и женщинам. Ни один человек не вправе захлопнуть дверь милосердия перед страждущей душой, которая может искать входа. Мы будем сидеть за трапезой со всеми, кто пожелает слушать о царстве. Когда наш Отец на небесах смотрит вниз на людей, они все одинаковы. Потому не отказывайтесь преломить хлеб с фарисеем или грешником, саддукеем или мытарем, римлянином или евреем, богатым или бедным, свободным или рабом. Двери царства широко открыты для всех, кто хочет узнать истину и найти Бога».
\vs p138 4:3 \pc Этим вечером за простым ужином в доме Алфея близнецы были приняты в апостольское братство. Позже в тот вечер Иисус дал своим апостолам первый урок на тему происхождения, природы и предназначения нечистых духов, но они не смогли уразуметь важность того, о чем он говорил. Для них было очень легко любить Иисуса и восхищаться им, но очень трудно --- понимать многое из его учения.
\vs p138 4:4 После ночного отдыха вся группа, состоящая теперь из одиннадцати человек, на лодке отправилась в Тарихею.
\usection{5. Призвание Фомы и Иуды}
\vs p138 5:1 Рыбак Фома и странник Иуда встретили Иисуса с апостолами на стоянке рыбацких лодок в Тарихее, и Фома повел всех в свой дом, находившийся поблизости. Там Филипп представил Фому как своего кандидата в апостолы, а Нафанаил --- Иуду Искариота, иудея, как своего кандидата. Иисус посмотрел на Фому и сказал: «Фома, у тебя недостает веры; однако я принимаю тебя. Следуй за мной.» Иуде Искариоту Учитель сказал: «Иуда, мы все одной плоти, и принимая тебя к нам как своего, я молюсь, чтобы ты всегда оставался верен галилейским братьям. Следуй за мной».
\vs p138 5:2 \pc После того, как они подкрепились, Иисус на некоторое время уединился с ними для молитвы и наставления о природе и деятельности Святого Духа. Но снова они во многом не смогли понять значение этих удивительных истин, которым он пытался научить их. Кто\hyp{}то усвоил что\hyp{}то одно в его уроке, кто\hyp{}то понял другое, но ни один из них не смог воспринять его учение в целом. Всякий раз они делали одну и ту же ошибку, пытаясь втиснуть новое евангелие Иисуса в старые формы своих религиозных верований. Они не в силах были осознать идею о том, что Иисус пришел возвестить новое евангелие спасения и утвердить новый путь к Богу; они не разобрались, что он \bibemph{был} новым откровением Отца на небесах.
\vs p138 5:3 На следующий день Иисус предоставил двенадцать апостолов самим себе; он хотел, чтобы они получше узнали друг друга, а также обсудили между собой то, чему он их учил. Он вернулся к вечерней трапезе и в часы после трапезы говорил с ними о служении серафимов, и некоторые из апостолов поняли его учение. Ночью они отдыхали, и на следующий день отбыли на лодке в Капернаум.
\vs p138 5:4 Зеведей и Саломея переселились к сыну Давиду, оставив свой большой дом Иисусу и его двенадцати апостолам. Здесь Иисус спокойно провел субботу со своими избранными вестниками. Он тщательно разъяснил им свои планы возвещения царства и досконально объяснил им, как важно избегать любых столкновений со светскими властями, говоря: «Если светские правители заслуживают осуждения, оставьте это мне. Смотрите, не осуждайте кесаря или его слуг». Именно в тот вечер Иуда Искариот отвел Иисуса в сторону и спросил, почему ничего не было сделано для вызволения Иоанна из тюрьмы. И остался не вполне удовлетворен позицией Иисуса.
\usection{6. Неделя интенсивной подготовки}
\vs p138 6:1 Следующая неделя была посвящена программе интенсивной подготовки. Ежедневно каждый из шести новых апостолов поступал на попечение того, кто избрал его, чтобы получить полное представление о том, что тот узнал и чему научился в процессе подготовки к работе для царства. Старшие апостолы старательно излагали новым шестерым учения, услышанные от Иисуса до последнего времени. По вечерам все собирались в саду Зеведея, чтобы получить наставления Иисуса.
\vs p138 6:2 Именно тогда Иисус выделил день в середине недели для отдыха и развлечения. И они, следуя этому плану до конца его земной жизни, один день в неделю делали перерыв для отдыха. Как правило, они никогда не занимались своей повседневной работой по средам. В этот еженедельный выходной Иисус обычно уходил от них, говоря: «Дети мои, дайте себе отдых. Отвлекайтесь от напряженных трудов царства и освежите силы, обратившись к вашей прежней работе или найдя новые формы восстанавливающих силы занятий». Самому Иисусу в этот период его земной жизни практически не требовался отдых, но он следовал этому порядку, зная, что так лучше для его человеческих сподвижников. Иисус был Учитель, его сподвижники --- его учениками --- последователями.
\vs p138 6:3 \pc Иисус стремился сделать так, чтобы апостолы осознали различие между его учениями и \bibemph{его жизнью среди них} и учениями, которые впоследствии могут возникнуть \bibemph{о} нем. Он сказал: «Главное в вашем послании --- мое царство и его евангелие. Не отклоняйтесь в сторону, проповедуя \bibemph{обо} мне и \bibemph{о} моих учениях. Возвещайте евангелие царства, являйте мое откровение небесного Отца, но не обманывайтесь, творя легенды и создавая культы, которые суть верования и учения \bibemph{о} моих верованиях и учениях». И опять они не поняли, почему он так говорит, и ни один не осмелился спросить, почему он так их учит.
\vs p138 6:4 В этих ранних наставлениях Иисус стремился избегать насколько возможно споров с апостолами, если только не встречал ложное представление о небесном Отце. Во всех таких случаях он никогда не колебался исправить ошибочные убеждения. После крещения в своей жизни Иисус на Урантии был движим только одним: дать лучшее и более истинное откровение своего райского Отца; он был первопроходцем нового и лучшего пути к Богу, пути веры и любви. Он постоянно призывал апостолов: «Ищите грешников; находите упавших духом и утешайте пребывающих в тревоге».
\vs p138 6:5 Иисус прекрасно понимал ситуацию; обладая неограниченной властью, он мог бы использовать ее для продвижения своей миссии, но он целиком довольствовался средствами и личностями, которых большинство людей сочли бы неадекватными и смотрели бы на них как на незначительных. Он выполнял миссию, имеющую возможность производить грандиозный эффект, но он настойчиво продолжал работать в том, что принадлежало Отцу, самым спокойным и недраматичным образом, тщательно избегая любых проявлений силы. И теперь он предполагал спокойно работать по крайней мере еще несколько месяцев со своими двенадцатью апостолами в районе Галилейского озера.
\usection{7. Еще одно разочарование}
\vs p138 7:1 Иисус предполагал, что индивидуальная миссионерская деятельность продлится пять месяцев. Он не сказал апостолам, как долго это продлится; они работали от недели к неделе. В начале первого дня недели, когда он собирался сообщить это двенадцати апостолам, Симон Петр, Иаков Зеведей и Иуда Искариот подошли к нему, чтобы с ним поговорить. Отведя Иисуса в сторону, Петр решительно сказал: «Учитель, по поручению наших товарищей, мы хотим спросить тебя: не настало ли время войти в царство? И провозгласишь ты царство в Капернауме или же мы двинемся в Иерусалим? И когда мы узнаем, каждый из нас, положение, которое мы должны занять вместе с тобой в руководстве царства?\ldots » Петр задавал бы еще вопросы, но Иисус предостерегающе поднял руку. Знаком подозвав остальных апостолов, стоявших неподалеку, Иисус сказал: «Дети мои малые, сколько еще нянчиться с вами! Разве не объяснял я, что мое царство --- не от мира сего? Много раз я вам говорил, что пришел не для того, чтобы сесть на трон Давида; и вот теперь вы спрашиваете, какое положение каждый из вас займет в царстве Отца. Неужели не можете понять, что я призвал вас как посланцев духовного царства? И разве вы не понимаете, что скоро, очень скоро вы будете представлять меня в этом мире и в возвещении царства, так же, как я сейчас представляю моего Отца, который на небесах? Возможно ли, что я избрал вас и наставлял вас как вестников царства, и все же вы не поняли природу и значение этого грядущего царства божественного торжества в сердцах человеческих? Друзья, выслушайте меня еще раз. Изгоните из своих умов мысль о том, что мое царство --- это правление силы или царство славы. Хотя воистину вся власть на земле и на небе скоро будет в моих руках, но не в том состоит воля Отца, чтобы мы использовали этот божественный дар для возвеличения самих себя в этом веке. Наступит время, когда вы будете воистину сидеть со мной во власти и славе, но ныне надлежит нам подчиниться воле Отца и с кротким послушанием выполнять его повеления на земле».
\vs p138 7:2 Вновь его сподвижники были потрясены и ошеломлены. Иисус отослал их по двое молиться, попросив возвратиться к полудню. В это критическое предполуденное время каждый из них пытался обрести Бога, каждый стремился ободрить и укрепить другого, и они вернулись к Иисусу в указанное время.
\vs p138 7:3 Теперь Иисус напомнил им о приходе Иоанна, крещении на Иордане, бракосочетании в Кане, недавнем избрании шестерых, отчуждении собственных его братьев по крови и предупредил, что враг царства постарается отдалить и их. Когда он окончил эту краткую, но очень убедительную речь, апостолы встали и, во главе с Петром, заявили о неуклонной преданности своему Учителю и об обете своей неизменной верности царству --- говоря словами Фомы, «этому наступающему царству, чем бы оно ни было, хотя я не вполне понимаю его». Все они подлинно \bibemph{верили в Иисуса,} хотя и не вполне понимали его учение.
\vs p138 7:4 Затем Иисус спросил, сколько у кого при себе денег и как они обеспечили свои семьи. Когда выяснилось, что у них едва ли хватит, чтобы прожить две недели, он сказал: «Не по воле моего Отца в таких обстоятельствах начинать нашу работу. Останемся здесь на побережье еще две недели, будем заниматься ловлей рыбы или иным трудом, какой найдем себе; за это время вы под началом Андрея, первозванного апостола, должны организовать все так, чтобы суметь обеспечить себя всем необходимым для вашей дальнейшей работы, --- как при теперешнем личном служении, так и позднее, когда я впоследствии посвящу вас на проповедь евангелия и наставления верующих». Апостолы были очень ободрены этими словами: наконец\hyp{}то Учитель ясно и позитивно высказался, что планирует более активную и открытую публичную деятельность.
\vs p138 7:5 Остаток дня апостолы провели, совершенствуя свое организационное устройство, а также готовя лодки и сети на завтрашнее утро, ибо все они решили заняться рыбной ловлей: большинство из них были рыбаки, да и сам Иисус был опытный лодочный мастер и рыбак. Многие из лодок, которые они использовали в последующие несколько лет, Иисус сделал собственными руками. Это были хорошие и надежные лодки.
\vs p138 7:6 Иисус поручил им заниматься рыбной ловлей ближайшие две недели, добавив: «И потом станете ловцами людей». Они рыбачили, разделившись на три группы; Иисус каждый вечер отправлялся рыбачить с разными группами. И все они таким образом наслаждались обществом Иисуса. Он был хорошим рыбаком, неунывающим спутником, воодушевляющим другом; чем больше они трудились вместе с ним, тем больше его любили. Однажды Матфей сказал: «Некоторых людей чем больше понимаешь, тем меньше они тебе нравятся; этого же человека я все меньше понимаю, но все больше люблю».
\vs p138 7:7 Этот порядок жизни: две недели рыбной ловли, затем две недели личной миссионерской работы для царства --- сохранялся ими более пяти месяцев, до самого конца 26 года и до прекращения гонений на учеников Иоанна, которые велись со времени его ареста.
\usection{8. Первая работа двенадцати}
\vs p138 8:1 После продажи двухнедельного улова Иуда Искариот, выбранный казначеем для всех двенадцати, отложил деньги для семей, остальное разделил на шесть равных частей. Примерно в середине августа 26 года н.э. они по двое отправились в местности, указанные им Андреем. Первые две недели Иисус ходил вместе с Андреем и Петром, следующие --- с Иаковом и Иоанном, и далее со всеми остальными парами в порядке их апостольского избрания. Так он смог хотя бы по одному разу поработать с каждой парой, прежде чем созвал всех вместе для начала публичного служения.
\vs p138 8:2 Иисус учил апостолов проповедовать прощение греха через \bibemph{веру в Бога,} без покаяния или жертвы, и то, что Отец на небесах любит всех своих детей одинаковой вечной любовью. Он приказал им воздерживаться от обсуждения следующих тем.
\vs p138 8:3 \ublistelem{1.}\bibnobreakspace Деятельность и заточение Иоанна Крестителя.
\vs p138 8:4 \ublistelem{2.}\bibnobreakspace Голос во время его крещения. Иисус сказал: «Лишь те, кто слышали этот голос, вправе ссылаться на него. Говорите только то, что слышали от меня; не говорите с чужих слов».
\vs p138 8:5 \ublistelem{3.}\bibnobreakspace Превращение воды в вино в Кане. Об этом Иисус твердо запретил упоминать: «Не говорите ни одному человеку о воде и вине».
\vs p138 8:6 \pc Эти пять или шесть месяцев были замечательным временем, когда они по две недели трудились как рыбаки, зарабатывая при этом достаточно денег, чтобы содержать себя в течение следующих двух недель миссионерской работы для царства.
\vs p138 8:7 Простые люди изумлялись учению и служению Иисуса и его апостолов. Раввины всегда учили евреев, что невежественный человек не может быть благочестив или праведен. Но апостолы Иисуса были благочестивы и праведны, находясь в блаженном неведении многих знаний раввинов и мудрости мира.
\vs p138 8:8 \pc Иисус объяснял апостолам различие между покаянием через так называемые добрые поступки, которому учат у евреев, и изменением сознания через веру --- новым рождением, --- которого требовал он как плату за вход в царство. Он учил своих апостолов, что \bibemph{вера ---} единственное условие входа в царство Отца. Иоанн учил: «Покайтесь, чтобы избежать грядущего гнева». Иисус учил о том, что «вера есть открытая дверь, чтобы обрести сущую, совершенную и вечную любовь Бога». Иисус говорил не как пророк --- не как тот, кто приходит объявить слово Божье. Казалось, он говорит от себя как имеющий на это право. Он стремился отвратить их умы от поиска чудес, обратив их к обретению истинного и личностного переживания, которое состоит в удовлетворенности и уверенности в том, что в них пребывает божественный дух любви и спасительной благодати.
\vs p138 8:9 Ученики рано усвоили, что Учитель с глубоким уважением и доброй заботой относится к \bibemph{каждому} встречаемому им человеку; на них производила огромное впечатление его неизменная, всегда одинаковая внимательность абсолютно ко всем мужчинам, женщинам и детям. Он мог остановится в самом разгаре проникновенной речи для того, чтобы выйти на дорогу сказать слово ободрения проходящей мимо женщине, отягощенной своей ношей души и тела. Он мог прервать серьезную беседу со своими апостолами, чтобы дружески поговорить с прибежавшим ребенком. Казалось, ничто не было для него важнее \bibemph{индивидуального человеческого существа,} которое случайно оказалось рядом с ним. Он был мастером и учителем, но он был и большим --- он был другом и близким человеком, понимающим товарищем.
\vs p138 8:10 \pc Публичные выступления Иисуса состояли в основном из притч и кратких речей, но апостолов он неизменно учил посредством вопросов и ответов. Во время его последующих публичных выступлений он всегда был готов прервать речь, чтобы ответить на искренне заданный вопрос.
\vs p138 8:11 Апостолы были сначала шокированы обращением Иисуса с женщинами, но быстро привыкли к нему. Он очень ясно дал им понять, что в царстве женщины имеют равные права с мужчинами.
\usection{9. Пять месяцев испытания}
\vs p138 9:1 Несколько однообразная жизнь этого периода, в которой чередовались рыбная ловля и индивидуальная работа, оказалась изнурительным опытом для двенадцати апостолов, но они выдержали испытание. При всех своих роптаниях, сомнениях и преходящих недовольствах они остались верны обету преданности и верности Учителю. Именно личный контакт с Иисусом в эти месяцы испытания так сблизил их с ним, что все они (кроме Иуды Искариота) остались преданы и верны ему даже в тяжкие часы суда и распятия. Будучи настоящими мужчинами они просто не могли оставить почитаемого учителя, который жил так близко с ними и был так предан им, как Иисус. В черные часы смерти учителя сердцами апостолов владела одна необычайно сильная человеческая эмоция, отодвинувшая все суждения, оценки и логику, --- высшее чувство дружеской верности. Пять месяцев совместной работы с Иисусом привели этих апостолов, каждого из них, к тому, чтобы относиться к нему как к самому близкому \bibemph{другу,} который у него когда\hyp{}либо был. Именно эти человеческие чувства, а отнюдь не возвышенное учение, не сверхъестественные деяния Иисуса удержали их вместе до воскресения и возобновления проповеди евангелия царства.
\vs p138 9:2 Эти месяцы скромной индивидуальной работы были великим испытанием не только для апостолов, которое они выдержали; этот период публичной бездеятельности был большим испытанием для семьи Иисуса. К тому моменту, когда он был готов начать публичное служение, вся его семья (кроме Руфи) практически отступилась от него. Лишь несколько раз впоследствии они искали контакта с ним, и единственно для того, чтобы убедить вернуться с ними домой, потому что они почти поверили в то, что он сошел с ума. Они просто не могли ни оценить его философию, ни понять его учение; это было чересчур для тех, кто был его собственной плоти и крови.
\vs p138 9:3 \pc Апостолы вели свою индивидуальную работу в Капернауме, Вифсаиде\hyp{}Юлии, Хоразине, Герасе, Гиппосе, Магдале, Кане, Вифлееме Галилейском, Иотапате, Раме, Сафеде, Гишале, Гадаре и Абиле. Кроме этих городов, они действовали во многих деревнях и поселениях. К концу этого периода двенадцать апостолов выработали планы достаточно удовлетворительного обеспечения своих семей. Большинство апостолов были женаты, у некоторых было по несколько детей, но им удалось сделать такие распоряжения, что, при небольшой поддержке из общего апостольского фонда, их семьи были полностью обеспечены и они могли посвятить всю свою энергию делу Учителя, не беспокоясь о финансовом благополучии своих семей.
\usection{10. Порядок организации двенадцати апостолов}
\vs p138 10:1 Апостолы рано организовались следующим образом.
\vs p138 10:2 \ublistelem{1.}\bibnobreakspace Андрей, первозванный апостол, был назначен предводителем и главой среди двенадцати.
\vs p138 10:3 \ublistelem{2.}\bibnobreakspace Петр, Иаков и Иоанн стали личными помощниками Иисуса. Они должны были находиться рядом с ним день и ночь, служить физическим и другим его нуждам и сопутствовать ему в ночных бдениях, посвященных молитве и таинственному общению с Отцом Небесным.
\vs p138 10:4 \ublistelem{3.}\bibnobreakspace Филипп стал экономом группы. Его обязанностью было обеспечивать еду и следить за тем, чтобы посетителей, даже временами огромную толпу слушателей, было чем накормить.
\vs p138 10:5 \ublistelem{4.}\bibnobreakspace Нафанаил следил за обеспечением семей апостолов. Он регулярно получал сообщения о нуждах каждой из апостольских семей и делал соответствующие заявки казначею Иуде о том, чтобы тот еженедельно посылал деньги нуждающимся.
\vs p138 10:6 \ublistelem{5.}\bibnobreakspace Матфей следил за состоянием финансов группы апостолов. Он отвечал за сбалансированность бюджета, за восполнение трат. Если общая касса не пополнялась, если пожертвования были недостаточны для содержания группы, он имел право призвать всех к рыболовным сетям на некоторое время. Но с тех пор, как они начали публичную работу, в этом ни разу не возникла необходимость: у казначея всегда в наличии было достаточно средств для финансирования деятельности.
\vs p138 10:7 \ublistelem{6.}\bibnobreakspace Фома был проводником. Он занимался организацией ночлега и, вообще, выбирал места для учения и проповеди, при этом обеспечивая удобный и быстрый путь.
\vs p138 10:8 \ublistelem{7.}\bibnobreakspace На близнецов Иакова с Иудой, сыновей Алфея, была возложена задача регулировать движение больших масс народа. Их задачей было с достаточным числом помощников поддерживать порядок во время проповедей, собиравших огромные толпы людей.
\vs p138 10:9 \ublistelem{8.}\bibnobreakspace Симон Зилот заботился об отдыхе и досуге. Он составлял программы на среды, а также по возможности на несколько часов отдыха и развлечений каждого дня.
\vs p138 10:10 \ublistelem{9.}\bibnobreakspace Иуда Искариот был назначен казначеем. Он «нес сумку». Иуда оплачивал все расходы и вел книги. Каждую неделю он рассчитывал бюджет для Матфея и еженедельно отчитывался перед Андреем. По распоряжениям Андрея он делал выплаты.
\vs p138 10:11 \pc Таково было разделение обязанностей среди двенадцати почти с самого начала их организации и вплоть до того момента, когда после отступничества Иуды потребовалась реорганизация. Спокойная жизнь Иисуса с его учениками\hyp{}апостолами продолжалась до воскресенья 12 января 27 года н.э., когда он созвал их вместе, чтобы формально посвятить их посланцами царства и проповедниками благих вестей о нем. Вскоре они были готовы отправиться в Иерусалим и Иудею в свое первое публичное проповедническое путешествие.
