\upaper{97}{Эволюция представления о Боге у евреев}
\author{Мелхиседек}
\vs p097 0:1 Духовные вожди иудеев сделали то, чего до них никогда не удавалось сделать другим, --- они деантропоморфизировали свое представление о Боге, при этом не превратив его в абстракцию Божества, понятную только философам. Даже простые люди, и те были способны рассматривать зрелое представление о Яхве как об Отце, если не отдельно взятого человека, то, по крайней мере, всей расы.
\vs p097 0:2 Представление о личности Бога, хоть ему ясно учили в Салиме во дни Мелхиседека, во время бегства из Египта было размытым и туманным и лишь постепенно из поколения в поколение развивалось в сознании иудеев в ответ на учение духовных вождей. В своей постепенной эволюции гораздо более непрерывным было осознание личности Яхве, чем любого другого атрибута Божества. От Моисея до Малахии в сознании иудеев происходил почти непрерывный рост понятия личности Бога; именно это понимание в конечном итоге и было возвышено и прославлено учениями Иисуса об Отце небесном.
\usection{1. Самуил --- первый иудейский пророк}
\vs p097 1:1 Враждебное поведение окружающих народов в Палестине научило иудейских шейхов тому, что они не могут надеяться на выживание, пока не объединят свои племенные организации под централизованным управлением. Именно эта централизация административной власти и предоставила Самуилу более полную возможность действовать в качестве учителя и реформатора.
\vs p097 1:2 Самуил происходил из огромной плеяды салимских учителей, которые упорно отстаивали истины Мелхиседека как часть своих форм поклонения. Этот учитель был мужественным и решительным человеком. Устоять перед лицом почти всеобщей оппозиции, с которой он столкнулся, когда начал возвращать весь Израиль к почитанию верховного Яхве времен Моисея, ему позволила лишь его великая набожность, сочетавшаяся с необычайной решимостью. Однако даже при всем этом ему удалось добиться только частичного успеха; он обратил к служению высшему представлению о Яхве лишь наиболее разумную половину иудеев; другая же половина продолжала поклоняться племенным богам сельской местности и более низменному пониманию Яхве.
\vs p097 1:3 Самуил относился к тому типу людей, которые действуют жестко, но энергично, и был пророком\hyp{}практиком, который мог в один день выйти со своими сподвижниками и разрушить два десятка мест поклонения Ваалу. Прогресс, которого он добивался, достигался просто принуждением; он мало проповедовал, еще меньше учил, но зато действовал. В один день он высмеивал священника Ваала, а на другой разрубал на куски пленного царя. Он самозабвенно верил в единого Бога и имел ясное представление об этом едином Боге как о создателе неба и земли: «У Господа основания земли, и он утвердил на них мир».
\vs p097 1:4 Однако великим вкладом, сделанным Самуилом в развитие представления о Божестве, явилось его замечательное утверждение о том, что Яхве \bibemph{неизменен,} что он --- вечное олицетворение непогрешимого совершенства и божественности. В эти времена Яхве считали непостоянным и капризным Богом ревнителем, всегда сожалеющим о том, что он поступил так\hyp{}то и так\hyp{}то; теперь же впервые после исхода из Египта, иудеи услышали эти поразительные слова: «Не солжет и не раскается Сила Израиля, ибо не человек он, чтобы раскаяться ему». Устойчивость отношений с Божеством была провозглашена. Самуил повторил завет Мелхиседека Аврааму и объявил, что Господь Бог Израиля --- источник всякой истины, стабильности и постоянства. Раньше иудеи всегда смотрели на своего Бога как на человека, сверхчеловека, как на возвышенный дух неизвестного происхождения; теперь же они услышали, как бывший дух Хорива превозносится как Бог неизменный, Бог, обладающий совершенством творца. Самуил способствовал восхождению совершенствовавшегося представления о Боге до высот, превосходящих изменчивое состояние людских умов и превратности смертного бытия. Благодаря его учению, Бог иудеев начинал восхождение от идеи, соответствующей уровню племенных богов, до идеала всесильного и неизменного Творца и Р\bibemph{уководителя} всего творения.
\vs p097 1:5 И Самуил заново проповедовал историю о искренности Бога, его надежности в соблюдении завета. Самуил сказал: «Господь не оставит народа своего». «Он заключил с нами вечный договор, твердый и непреложный». И так по всей Палестине звучал призыв вернуться к почитанию верховного Яхве. Сей энергичный учитель всегда провозглашал: «Велик ты, Господь Бог, ибо нет подобного тебе и нет Бога, кроме тебя».
\vs p097 1:6 \pc До сих пор иудеи рассматривали благоволение Яхве главным образом с точки зрения материального процветания. Когда же Самуил решился провозгласить: «Господь делает нищим и обогащает; унижает и возвышает. Из праха поднимает бедных и возвышает нищих, дабы посадить их с вельможами и престол славы дать им в наследие», это было потрясением для Израиля и чуть было не стоило Самуилу жизни. Такие утешительные обещания простым и менее удачливым людям не провозглашались со времен Моисея, и тысячи отчаявшихся среди бедных стали надеяться на то, что они смогут улучшить свое духовное положение.
\vs p097 1:7 Но Самуил ушел от понятия племенного бога не очень далеко. Он провозгласил Яхве, который сотворил всех людей, но занят в основном иудеями, своим избранным народом. Тем не менее, как и во дни Моисея, представление о Боге вновь изображало Божество, святое и праведное. «Нет никого святее Господа. Кто сравнится со святым Господом Богом?»
\vs p097 1:8 По прошествии лет убеленный сединами старый вождь пошел дальше в понимании Бога, ибо провозгласил: «Господь есть Бог знания, и дела у него взвешены. Господь будет судить концы земли, с милостивыми поступая милостиво, и со справедливым будет так же справедлив». Заря милосердия занялась уже здесь, хотя милосердие это и ограничивалось теми, кто был сам милосерден. Позднее Самуил сделал еще один шаг вперед, когда в годину несчастий призвал свой народ: «Пусть впадем мы в руки Господа, ибо велико милосердие его». «Для господа нетрудно спасти многих или немногих».
\vs p097 1:9 \pc И это постепенное развитие представления о сущности Яхве продолжалось при служении преемников Самуила. Они пытались представить Яхве Богом, соблюдающим завет, но с трудом поддерживали темп, заданный Самуилом, и не сумели развить идею о милосердии Бога, как позднее понимал его сам Самуил. Неуклонный возврат к признанию других богов все же происходил, несмотря на утверждение о том, что Яхве превыше всех богов. «Твое, Господи, царство, и ты вознесен превыше всех как владычествующий».
\vs p097 1:10 Лейтмотивом этой эры была божественная сила; пророки той эпохи проповедовали религию, предназначенную утвердить царя на иудейском престоле. «Твое, Господи, величие, и могущество, и слава, и победа, и великолепие. В руке твоей сила и могущество, и во власти твоей возвеличить и укрепить все». Так обстояло дело с представлением о Боге во времена Самуила и его непосредственных преемников.
\usection{2. Илия и Елисей}
\vs p097 2:1 В десятом столетии до Рождества Христа иудейская нация разделилась на два царства. В обеих этих политических половинах многие учителя истины пытались противостоять реакционной волне духовного упадка, которая возникла и продолжала опасно нарастать после войны, приведшей к разделению. Однако эти усилия усовершенствовать иудейскую религию не приводили к успеху, пока решительный и бесстрашный воитель за праведность Илия не начал свое учение. Илия вернул северному царству понимание Бога, сравнимое с представлением, которого придерживались во дни Самуила. Но у Илии было мало возможностей представить более совершенное понятие о Боге; так же, как Самуил до него, он был занят низвержением алтарей Ваала и разрушением идолов ложных богов. Причем свои реформы он осуществлял перед лицом сопротивления монарха\hyp{}идолопоклонника, и его задача была еще более гигантской и сложной, чем та, что стояла перед Самуилом.
\vs p097 2:2 Когда Илия был отозван, начатое им дело продолжил его верный соратник Елисей, который, пользуясь неоценимой помощью малоизвестного Михея, не давал угаснуть свету истины в Палестине.
\vs p097 2:3 Однако это не были времена совершенствования представления о Божестве. Иудеи тогда еще не сумели подняться даже до Моисеева идеала. И эпоха Илии и Елисея закончилась возвратом лучших классов общества к почитанию верховного Яхве и стала свидетелем восстановления идеи о Творце всего сущего приблизительно до того состояния, в каком ее оставил Самуил.
\usection{3. Яхве и Ваал}
\vs p097 3:1 Затянувшийся конфликт между верующими в Яхве и последователями Ваала был столкновением социально\hyp{}экономических воззрений, а не различием в религиозных верованиях.
\vs p097 3:2 \pc Жители Палестины по\hyp{}разному относились к частному владению землей. Южные или кочующие аравийские племена (поклонники Яхве) считали землю неотчуждаемой --- даром Божества клану. Они утверждали, что земля не может продаваться или закладываться. Ведь «Яхве сказал, говоря: „Землю не должно продавать, ибо земля моя“».
\vs p097 3:3 Северные и более оседлые хананеи (служители Ваала) свободно покупали, продавали и закладывали свои земли. Само слово «Ваал» означает «владелец». Культ Ваала был основан на двух основных доктринах: во\hyp{}первых, узаконивании обмена собственностью, контрактов и соглашений --- узаконивании права покупать и продавать землю; во\hyp{}вторых, считалось, что Ваал посылает дождь --- он был богом плодородия почвы. Хорошие урожаи зависели от благоволения Ваала. Культ Ваала был во многом связан с \bibemph{землей,} владением ею и плодородием.
\vs p097 3:4 Служители Ваала, как правило, владели домами, землями и рабами. Они были аристократами\hyp{}домовладельцами и жили в городах. Каждый Ваал имел святилище, священников и «святых женщин», ритуальных проституток.
\vs p097 3:5 Из этого существенного различия в отношении к земле и развились серьезные антагонизмы в общественных, экономических, нравственных и религиозных позициях, занимаемых хананеями и иудеями. Однако до наступления времен Илии этот социально\hyp{}экономический спор не приобретал характера явно религиозного разногласия. Со дней же сего воинственного пророка противостояние перешло, в основном, на стезю религиозной вражды --- к борьбе между Яхве \bibemph{против} Ваала --- и закончилось победой Яхве и последующим переходом к монотеизму.
\vs p097 3:6 Илия перевел спор между поклонниками Яхве и Ваала с вопроса о земле на религиозный аспект иудейской и ханаанской идеологий. Когда Ахав в интриге, разыгранной ради того, чтобы завладеть землей Навуфеев, убил их, Илия сделал существовавшие ранее обычаи владения землей предметом нравственного спора и начал свою энергичную кампанию против служителей Ваала. То была также борьба деревенских жителей против господства городов. Главным образом при Илие Яхве стал Элохим. Пророк начинал как сторонник земельной реформы, а закончил возвышением Божества. Ваалов было много, а Яхве --- \bibemph{один ---} монотеизм победил политеизм.
\usection{4. Амос и Осия}
\vs p097 4:1 Великий шаг в переходе от племенного бога --- бога, которому так долго служили жертвами и обрядами, Яхве древних иудеев, --- к Богу, который карал бы преступления и безнравственность даже в среде своего собственного народа, был сделан Амосом, явившимся из среды обитателей южных гор, дабы осудить преступность, пьянство, угнетение и безнравственность северных племен. Со времен Моисея столь великие истины в Палестине не провозглашал никто.
\vs p097 4:2 Амос не был просто сторонником реставрации или реформ; он был открывателем новых представлений о Божестве. Он провозглашал многое о Боге, что уже объявляли его предшественники, и смело критиковал веру в Божественное Существо, которое поощряло грех в среде своего так называемого избранного народа. Впервые со дней Мелхиседека человеческие уши слышали осуждение двойных стандартов в национальном правосудии и морали. Впервые в своей истории иудеи услышали своими ушами, что их Бог Яхве не будет терпеть преступление и грех в их жизнях больше, чем в среде любого другого народа. Амос обладал видением сурового и справедливого Бога Самуила и Илии, который, когда дело касалось наказания за грех, не думал о иудеях иначе, чем о любой другой нации. Это было прямой нападкой на эгоистическую доктрину «избранного народа», и многих иудеев того времени она сильно возмущала.
\vs p097 4:3 Амос сказал: «Вот, он, который образовал горы и сотворил ветер; ищи того, кто сотворил семизвездие и Орион, превращает смертную тень в ясное утро, а день делает темным, как ночь». Причем, осуждая своих полурелигиозных, приспосабливающихся и порой безнравственных собратьев, он пытался показать неумолимое правосудие неизменного Яхве, когда говорил о грешниках: «Хотя бы они зарылись в преисподнюю, и оттуда достану их; хотя бы взошли на небо, и оттуда свергну их». «И даже если пойдут в плен впереди врагов своих, то повелю мечу правосудия, и он убьет их». Амос еще больше пугал своих слушателей, когда, показывая на них осуждающим и обвиняющим перстом, объявлял от имени Яхве: «Поистине во веки не забуду ни одного из дел ваших». «И просею дом Израилев по всем народам, как просеивают зерна в решете».
\vs p097 4:4 Амос провозгласил Яхве «Богом всех народов» и предупреждал израильтян, что ритуал не должен занимать место праведности. И пока этого смелого учителя не забили камнями на дрожжах насаждаемой им истины поднялось учение о верховном Яхве, и он обеспечил дальнейшую эволюцию откровения Мелхиседека.
\vs p097 4:5 \pc Воскресив Моисеево понятие о Боге любви, Осия стал преемником Амоса и его учения о всеобщем Боге справедливости. Осия проповедовал прощение через покаяние, а не жертву. Он провозгласил евангелие благости и божественного милосердия, говоря: «Обручу тебя мне навек; обручу тебя мне в праведности и суде, в благости и милосердии. Обручу тебя мне в верности». «Возлюблю их по благоволению, ибо гнев мой отвратился».
\vs p097 4:6 Осия верно продолжал нравственные предостережения Амоса, говоря о Боге: «По желанию моему накажу их». Однако когда Осия сказал: «Скажу тем, кто не был моим народом: „Ты --- мой народ“, и они скажут: „Ты --- наш Бог“», израильтяне сочли это жестокосердием, граничащим с предательством. Осия же продолжал проповедовать покаяние и прощение, говоря: «Уврачую отпадение их; возлюблю их по благоволению, ибо гнев мой отвратился». Осия всегда провозглашал надежду и прощение. И суть его послания всегда была такова: «Помилую народ мой. И не будет он знать иного Бога кроме меня, ибо, кроме меня, нет спасителя».
\vs p097 4:7 \pc Амос пробудил национальное сознание иудеев к признанию того, что Яхве не простит преступление и грех среди них, потому что они якобы являются избранным народом; Осия же сыграл начальные ноты в более поздних милосердных аккордах божественного сострадания и доброты, основанной на любви, которые столь изысканно пропел Исайя и его сподвижники.
\usection{5. Исайя Первый}
\vs p097 5:1 То были времена, когда одни провозглашали угрозы наказания за личные грехи и преступление нации в среде северных кланов, а другие предсказывали бедствие в возмездии за грехи южного царства. Именно на этой волне пробуждения совести и сознания у иудейских наций и появился Исайя Первый.
\vs p097 5:2 Исайя продолжил проповедь вечной природы Бога, его бесконечной мудрости, его неизменно совершенной надежности. И представил Бога Израиля высказыванием: «И поставлю суд мерилом и праведность весами». «Господь даст тебе покой от печали твоей и от страха твоего и тяжкого рабства, в котором человек был создан служить». «И уши твои услышат слово, говорящее позади тебя: „от путь, иди по нему“». «Вот, Бог --- спасение мое; уповать буду и не буду бояться, ибо Господь --- сила моя и песнь моя». «„Придите ныне и рассудим вместе“, --- говорит Господь. --- „Если будут грехи ваши, как багрянец, --- как снег убелю; если будут красны, как пурпур, --- станут как руно“».
\vs p097 5:3 Обращаясь к охваченным страхом и душой алчущим иудеям, сей пророк сказал: «Восстань и светись, ибо пришел свет твой, и слава Господня взошла над тобою». «Дух Господень на мне, ибо он помазал меня благовествовать кротким, послал меня исцелять сокрушенных сердцем, возвещать пленным освобождение и узникам --- открытые темницы». «Великой радостью возрадуюсь о Господе, возвеселится душа моя о Боге моем, ибо он облек меня в ризы спасения и одеждою праведности одел меня». «Во всех скорбях их он скорбел, и ангел лица его спасал их. По любви своей и жалости своей искупил их».
\vs p097 5:4 \pc За этим Исайей последовали Михей и Авдий, которые утвердили и разработали ублаготворяющее душу евангелие. Причем эти два смелых вестника открыто осуждали ритуалы иудеев, в которых господствовали священники, и бесстрашно критиковали всю систему жертвоприношений.
\vs p097 5:5 Михей осуждал «правителей, которые судят за подношения, и священников, которые учат за плату, и пророков, которые пророчествуют за деньги». Он учил о дне свободы от предрассудков и козней духовенства и говорил: «Но каждый будет сидеть под своей виноградной лозою, и никто не будет устрашать его, ибо все будут жить, каждый согласно своему пониманию Бога».
\vs p097 5:6 Суть послания Михея всегда была такова: «Предстать ли мне перед Господом со всесожжениями? Можно ли угодить Господу тысячами овнов или десятью тысячами потоков елейных? Дать ли мне первенца моего за прегрешение мое, плод тела моего --- за грех души моей? О человек! Он показал мне, что --- добро, и чего требует от тебя Господь: действовать справедливо, любить милосердие и смиренно ходить с Богом твоим». То была великая эпоха, поистине многознаменательные времена, когда более двух с половиной тысячелетий тому назад смертный человек услышал подобные спасительные послания, и некоторые люди даже поверили в них. И если бы не упрямое сопротивление священников, эти учителя низвергли бы весь кровавый церемониал иудейского ритуала поклонения.
\usection{6. Иеремия бесстрашный}
\vs p097 6:1 В то время, как отдельные учителя продолжали толковать евангелие Исайи, Иеремии выпало лишь сделать следующий смелый шаг и интернационализировать Яхве, Бога иудеев.
\vs p097 6:2 Иеремия бесстрашно заявлял, что Яхве отнюдь не стоит на стороне иудеев в их военных сражениях с другими нациями. Он утверждал, что Яхве --- Бог всей земли, всех наций и всех народов. Учение Иеремии явилось крещендо нараставшей волны интернационализации Бога Израиля; этот неустрашимый пророк окончательно и навсегда провозгласил, что Яхве --- Бог всех наций и что нет ни Осириса для египтян, ни Бела для вавилонян, ни Ассура для ассирийцев, ни Дагона для филистимлян. Таким образом, религия иудеев стала частью всемирного ренессанса монотеизма, происходившего приблизительно в это и в последующее время; наконец представление о Яхве достигло уровня планетарного и даже космического величия. Однако многие из сподвижников Иеремии с трудом представляли себе Яхве в отрыве от иудейской нации.
\vs p097 6:3 Иеремия также проповедовал справедливого и любящего Бога, описанного Исайей, и говорил: «Любовью вечной возлюбил я тебя и потому с благоволением приблизил тебя». «Ибо он не по изволению сердца своего наказывает сынов человеческих».
\vs p097 6:4 Сей бесстрашный пророк говорил: «Праведен Господь наш, великий в советах и сильный в делах. Глаза его открыты на все пути сынов человеческих, чтобы воздавать каждому по путям его и по плодам дел его». Однако, когда во время осады Иерусалима Иеремия сказал: «И ныне я отдаю все земли сии в руку Навуходоносора, царя Вавилонского, раба моего», это сочли богохульной изменой. И когда Иеремия посоветовал сдать город, священники и мирские правители бросили его в грязную яму в мрачной подземной тюрьме.
\usection{7. Исайя Второй}
\vs p097 7:1 Уничтожение еврейского государства и пленение иудеев в Месопотамии могло бы принести большую пользу их развивающейся теологии, если бы не решительные действия еврейских священников. Государство евреев пало перед армиями Вавилона, и их националистический Яхве пострадал от интернационалистских проповедей духовных вождей. Чувство обиды, вызванное утратой своего национального бога, и заставило еврейских священников зайти так далеко в выдумке небылиц и умножении числа кажущихся чудесами событий в иудейской истории, чтобы снова сделать евреев избранным народом даже в новом и расширенном представлении об интернационализированном Боге всех наций.
\vs p097 7:2 Во время пленения евреи подверглись сильному влиянию вавилонских традиций и легенд, хотя следует отметить, что они неизменно совершенствовали нравственный настрой и духовный смысл халдейских историй, которые заимствовали, несмотря на то, что эти легенды ими постоянно искажались, дабы покрыть честью и славой прародителей Израиля и его историю.
\vs p097 7:3 Эти иудейские священники и книжники думали лишь об одном --- о восстановлении еврейской нации, прославлении иудейских традиций и возвеличивании истории своей расы. И если тот факт, что эти священники навязали свои ошибочные идеи столь большой части Западного мира, и вызывает возмущение, то следует вспомнить, что делали они это ненамеренно; они не утверждали, что пишут по вдохновению, и не заявляли, что пишут священную книгу. А просто готовили руководство, предназначенное поддержать иссякающее мужество у своих соотечественников. У них была конкретная цель --- улучшить национальный дух и настрой своих соотечественников. Жившим после них людям оставалось лишь собрать эти и другие писания в наставление, состоящее из якобы непогрешимых учений.
\vs p097 7:4 После пленения еврейские священники весьма вольно использовали эти писания, однако влиять на своих собратьев\hyp{}пленников им мешало присутствие среди них молодого и неукротимого пророка Исайи Второго, который был всецело обращен в учение Исайи\hyp{}старшего о Боге правосудия, любви, праведности и милосердия. Вместе с Иеремией он также верил, что Яхве стал Богом всех народов. И проповедовал эти теории о природе Бога с таким красноречием, что одинаково обращал и евреев, и тех, кто взял их в плен. Причем этот молодой проповедник оставил свои учения в письменном виде, которые враждебные и нетерпимые священники пытались лишить какой бы то ни было связи с ним самим, хотя истинное уважение к их красоте и величию привели к их объединению с писаниями Исайи Первого. Таким образом, писания Исайи Второго можно найти в книге, носящей это имя, в главах с сороковой по пятьдесят пятую включительно.
\vs p097 7:5 \pc Ни один пророк или религиозный учитель от Махивенты до времен Иисуса не достиг того высокого понимания Бога, которое провозглашал Исайя Второй во время пленения. Этот духовный вождь провозглашал отнюдь не мелкого, антропоморфного, созданного человеком Бога. «Вот, острова как порошинку поднимает он». «И, как небо выше земли, так пути мои выше путей ваших, и мысли мои выше мыслей ваших».
\vs p097 7:6 Наконец Махивента Мелхиседек увидел, как учителя человеческие провозглашают смертному человеку настоящего Бога. Подобно Исайе Первому, этот вождь проповедовал Бога Творца и Вседержителя вселенной. «Я создал землю и поставил человека на ней. Не напрасно сотворил ее: я образовал ее для жительства». «Я первый и последний; и нет другого Бога, кроме меня». Говоря от имени Господа Бога Израиля, этот новый пророк сказал: «Небеса могут исчезнуть и земля обветшать, а праведность моя пребудет вечно и спасение мое --- из рода в род». «Не бойся, ибо я --- с тобою; не смущайся, ибо я --- Бог твой». «Нет иного Бога, кроме меня, --- Бога праведного и Спасителя».
\vs p097 7:7 И еврейских пленников так же, как тысячи тысяч после них, утешали такие слова, как эти: «Так говорит Господь: „Я сотворил тебя, я искупил тебя, я назвал тебя по имени твоему; ты --- мой“». «Когда будешь проходить через воды, я буду с тобой, так как ты дорог в глазах моих». «Может ли женщина забыть младенца своего, чтобы не пожалеть сына своего? Да, может забыть, и все же я не забуду детей моих, ибо вот, я начертал их на дланях моих; и укрыл их сенью рук моих». «Да оставит порочный пути свои и неправедный мысли свои, и да вернется к Господу, и он помилует его, и к Богу нашему, ибо он многомилостив».
\vs p097 7:8 Еще раз вслушайтесь в евангелие сего нового откровения о Боге Салима: «Как пастырь он будет пасти стадо свое; агнцев будет брать на руки свои и носить на груди своей. Он дает утомленному силу и изнемогшему дарует крепость. Надеющиеся на Господа обновятся в силе, поднимут крылья, как орлы; побегут, и не устанут, пойдут, и не утомятся».
\vs p097 7:9 Этот Исайя широко проповедовал евангелие расширенного представления о верховном Яхве. И соперничал в красноречии с Моисеем, и подобно ему изображал Господа Бога Израиля Творцом Всего Сущего. Описывая бесконечные атрибуты Отца Всего Сущего, он был настоящим поэтом. Никто и никогда не произносил таких прекрасных изречений об Отце небесном. Как и Псалтырь, писания Исайи относятся к самым возвышенным и истинным представлениям духовного понимания Бога, которые когда\hyp{}либо услаждали ухо смертного человека до прихода на Урантию Михаила. Послушайте, как он изображает Божество: «Я --- высокий и превознесенный, живущий в вечности». «Я первый и последний, и кроме меня нет другого Бога». «И рука Господа не сократилась на то, чтобы спасать, и ухо его не отяжелело для того, чтобы слышать». И когда сей кроткий, но непреклонный пророк упорно продолжал проповедовать божественную неизменность, верность Бога, это явилось новым учением в среде еврейства. Он заявлял, что «Бог не забудет и не оставит».
\vs p097 7:10 Этот отважный учитель провозглашал, что человек очень тесно связан с Богом, и говорил: «Каждого, кто называется именем моим, Я сотворил для славы моей, и они будут возвещать славу мою. Я, я сам изглаживаю преступления их ради себя самого, и грехов их не помяну».
\vs p097 7:11 Послушайте, как этот великий иудей разрушает представления о национальном Боге и одновременно во славе возвещает божественность Отца Всего Сущего, о котором говорит: «Небо --- престол мой, а земля --- подножие ног моих»\ldots И тем не менее, Бог Исайи свят, величествен, справедлив и неисследим. Представление о злом, мстительном и ревнивом Яхве бедуинов пустыни почти сошло на нет. В сознание смертного человека вселилось новое представление о верховном и всеобщем Яхве, которое не исчезало из поля зрения человека уже никогда. Осознание божественной справедливости начало разрушать примитивный магический и биологический страх. Наконец\hyp{}то, человек познакомился со вселенной, основанной на законе и порядке, а также с всемирным Богом, надежным и окончательно определенным.
\vs p097 7:12 Причем этот проповедник небесного Бога никогда не прекращал провозглашать сего \bibemph{Бога любви.} «Я живу на небе и во святилище, а также с сокрушенным и смиренным духом». Говорил сей великий учитель своим современникам и слова еще большего утешения: «И будет Господь вождем твоим всегда и будет насыщать душу твою. Ты будешь, как напоенный водою сад и как источник, воды которого не иссякают. И если враг придет, как поток, дух Господень воздвигнет преграду пред ним». И снова разрушающее страх евангелие Мелхиседека и рождающая упование религия воссияли, дабы благословить человечество.
\vs p097 7:13 Своим возвышенным описанием величия и вселенского всемогущества высшего Яхве, Бога любви, правителя вселенной и любящего Отца всего человечества дальновидный и смелый Исайя сумел затмить Яхве националистического. И с тех полных событиями дней высшее представление Запада о Боге всегда включало в себя вселенскую справедливость, божественное милосердие и вечную праведность. На прекрасном языке и с несравненным изяществом сей великий учитель изобразил всесильного Творца и вселюбящего Отца.
\vs p097 7:14 Сей пророк пленения проповедовал своему народу и представителям других наций, а те слушали его у реки в Вавилоне. И сей Исайя Второй многое сделал, дабы противодействовать многим неверным и в расовом отношении эгоистическим представлениям о миссии обещанного Мессии. Однако в этом он полного успеха не достиг. Если бы священники не посвятили себя делу воспитания неверно понимаемого национализма, то учения Исайи Первого и Исайи Второго подготовили бы путь для признания и принятия обещанного Мессии.
\usection{8. Священная и мирская история}
\vs p097 8:1 Обычай смотреть на записи о пережитом иудеями как на священную историю, а на писания остального мира как на историю мирскую --- вот что объясняет большую часть путаницы в человеческих умах относительно толкования истории. Причем трудность эта происходит от того, что светской истории евреев не существует. После того, как священники вавилонского изгнания подготовили свои новые записи о якобы чудесных отношениях Бога с иудеями, священную историю Израиля, какой она показана в Ветхом Завете, они тщательно и полностью уничтожили все существовавшие записи о делах иудеев --- такие книги, как «Деяния царей Израиля» и «Деяния царей Иудеи», а также несколько других более или менее точных записей о еврейской истории.
\vs p097 8:2 Чтобы понять, как пагубное давление и неизбежное влияние мирской истории смогли вселить в пленных и управляемых чужеземцами евреев такой страх, что они предприняли попытку полностью переписать и переделать свою историю, нам необходимо коротко рассмотреть записи, свидетельствующие о трудностях, пережитых евреями как нацией. Следует помнить, что евреи не сумели выработать адекватную нетеологическую философию жизни. Они боролись со своими изначальными и египетскими представлениями о божественном воздаянии за праведность, сочетаемом с суровым наказанием за грех. И драма Иова отчасти заключалась в протесте против этой ошибочной философии. А откровенный пессимизм Экклезиаста явился мирской и мудрой реакцией на эти сверхоптимистические верования в Провидение.
\vs p097 8:3 Однако пятьсот лет господства иноземных правителей не смогли вынести даже терпеливые и многострадальные евреи. Пророки и священники начали восклицать: «Надолго ли, Господи, надолго ли?» Когда честный еврей изучал Писание, он оказывался в еще большем смятении. Древний провидец обещал, что Бог защитит и спасет свой «избранный народ». Амос угрожал, что Бог оставит Израиль, если он не восстановит свои нормы национальной праведности. Написавший Второзаконие описал Великий Выбор как выбор между добром и злом, благословением и проклятием. Исайя Первый проповедовал милосердного царя\hyp{}избавителя. Иеремия провозглашал эру внутренней праведности --- эру завета, написанного на скрижалях сердца. Исайя Второй говорил о спасении путем жертвы и искупления. Иезекииль провозгласил избавление через преданное служение, а Ездра обещал процветание благодаря соблюдению закона. Однако, несмотря на все это, евреи оставались в рабстве, и избавление не приходило. Тогда Даниил драматически представил надвигающийся «кризис» --- разрушение великого идола и немедленное установление вечного царствования праведности, мессианского царства.
\vs p097 8:4 Причем все составляющие этой ложной надежды повергли народ в такое разочарование и отчаяние, что вожди евреев запутались и не сумели признать и принять миссию и служение божественного Райского Сына, когда тот в конце концов пришел к ним в подобии смертной плоти --- в воплощении Сына Человеческого.
\vs p097 8:5 \pc Пытаясь дать чудесное толкование определенным эпохам человеческой истории, все современные религии совершали серьезные ошибки. И хоть верно то, что Бог многократно погружал отеческую руку чудесного вмешательства в поток человеческих дел, ошибкой было бы рассматривать теологические догмы и религиозные предрассудки как сверхъестественные отложения, благодаря чудесному деянию появившиеся в этом потоке человеческой истории. То, что «Всевышние правят в царствах людей», отнюдь не превращает мирскую историю в так называемую историю священную.
\vs p097 8:6 Авторы Нового Завета и более поздние христианские писатели своими благонамеренными попытками трансцендентализировать иудейских пророков исказили иудейскую историю еще больше. Иудейская история, таким образом, страшно эксплуатировалась и еврейскими, и христианскими писателями. Мирская история иудеев была полностью догматизирована. Ее превратили в вымышленную священную историю, и она стала неразрывно связанной с нравственными понятиями и религиозными учениями так называемых христианских наций.
\vs p097 8:7 \pc Краткий рассказ о высших моментах еврейской истории покажет, как иудейские священники в Вавилоне изменили факты в записях, чтобы превратить повседневную мирскую историю своего народа в историю, вымышленную и священную.
\usection{9. История Иудеев}
\vs p097 9:1 Двенадцати колен Израиля никогда не существовало, а существовало лишь три или четыре племени, поселившихся в Палестине. Иудейская нация возникла как результат союза так называемых израильтян и хананеев. «И жили дети Израилевы среди хананеев. И брали дочерей их себе в жены и своих дочерей отдавали за сыновей их». Иудеи никогда не изгоняли хананеев из Палестины, несмотря на то, что записи священников, повествующие об этом, без колебаний утверждают, что они это сделали.
\vs p097 9:2 Израильтяне, как народ, зародились в горной стране Ефрем; еврейство, как народ, возникло в более поздние времена в южном клане Иуды. Евреи (иудеи) всегда старались обесславить и очернить записи северных израильтян (ефремлян).
\vs p097 9:3 \pc Претенциозная иудейская история начинается с объединения Саулом северных кланов, дабы отразить нападение аммонитян на своих соплеменников --- галаадетян --- к востоку от Иордана. С армией, которая насчитывала немногим более трех тысяч человек, он нанес врагу поражение, и именно этот подвиг заставил горные племена сделать его царем. Когда же оказавшиеся в изгнании священники переписывали эту историю, то увеличили численность армии Саула до 330000 человек, а в список племен, участвовавших в битве, добавили колено «Иудино».
\vs p097 9:4 Сразу после поражения аммонитян армия дружно проголосовала за Саула и сделала его царем. Ни пророк, ни священник не участвовали в этом. Однако позднее священники внесли в записи, будто Саул был коронован на царство, согласно божественным указаниям, пророком Самуилом. Сделано же это было для того, чтобы установить «божественное происхождение» царствования Давида в Иудее.
\vs p097 9:5 Величайшее из всех искажений еврейской истории было связано с Давидом. После победы Саула над аммонитянами (которую он приписывал Яхве) филистимляне встревожились и стали нападать на северные кланы. Давид и Саул никак не могли достичь договоренности. Давид и с ним шестьсот человек вступил в союз с филистимлянами и вдоль побережья пошел на Ездрилон. В Гафе филистимляне приказали Давиду покинуть поле сражения; потому что боялись, что он может перейти на сторону Саула. Давид удалился; филистимляне напали и нанесли Саулу поражение. Чего бы они сделать не смогли, если бы Давид был верен Израилю. Армия Давида представляла собой разноязыкое сборище мятежников и большей частью состояла из изгоев общества и беглецов, скрывавшихся от правосудия.
\vs p097 9:6 Трагическое поражение, нанесенное Саулу филистимлянами у горы Гелвуй, умалило значение Яхве перед другими богами в глазах окружающих хананеев. При обычных обстоятельствах поражение Саула было бы приписано отступничеству от Яхве, но на сей раз редакторы\hyp{}иудеи объяснили его ритуальными ошибками. Ведь для оправдания царствования Давида им требовалась традиция Саула и Самуила.
\vs p097 9:7 Давид со своей небольшой армией обосновался в неиудейском городе Хевроне. Вскоре соотечественники Давида провозгласили его царем нового царства Иудеи. Иудея большей частью состояла из неиудейских элементов --- кенеев, халевеев, иевусеев и других хананеев. Они были кочевниками\hyp{}пастухами --- и поэтому были верны еврейской идее землевладения. И придерживались идеологии пустынных кланов.
\vs p097 9:8 \pc Разница между священной и мирской историей хорошо иллюстрируется двумя различными историями о том, как Давид был сделан царем, какими мы находим их в Ветхом Завете. Часть мирской истории о том, как ближайшие последователи Давида (его армия) сделали его царем, была по невнимательности оставлена в записях священниками, которые впоследствии подготовили пространное и прозаичное повествование о священной истории, где и показано, как пророк Самуил, по божественному указанию, избрал Давида из собратьев его, после чего официально и с соблюдением сложных и торжественных церемоний помазал его на царство иудеев, а затем провозгласил преемником Саула.
\vs p097 9:9 Весьма часто, готовя свои вымышленные повествования о чудесных отношениях Бога с Израилем, священникам не удавалось полностью удалить явные и невыдуманные факты, которые уже были в записях.
\vs p097 9:10 \pc Давид пытался укрепиться политически, сначала женившись на дочери Саула, затем на вдове богатого идумея Навала, а затем на дочери Фалмая, царя Гессурского. Он взял также шесть жен из женщин иевусейских, не говоря уже о Вирсавии, жене хеттеянина.
\vs p097 9:11 Именно такими методами и из такого народа Давид и создал вымысел о божественном царстве Иудее как преемнике наследия и традиций исчезающего северного царства Израиля ефремлян. Давидово космополитичное колено Иудино было более нееврейским, чем еврейским; тем не менее, угнетенные старейшины Ефрема пришли и «помазали его на царство Израиля». После возникновения военной угрозы Давид соединился с иевусеями и основал свою столицу объединенного царства в Иевусе (Иерусалиме), это был обнесенный крепкими стенами город, находившийся на полпути между Иудеей и Израилем. Филистимляне возмутились и вскоре напали на Давида. После отчаянного сражения они потерпели поражение, и Яхве был вновь утвержден как «Господь Бог Саваоф».
\vs p097 9:12 Однако волей\hyp{}неволей Яхве пришлось поделиться некоторой частью этой славы с богами хананеев, ибо армия Давида в основном состояла из неиудеев. Именно поэтому в ваших записях и появляется это (пропущенное редакторами\hyp{}иудеями) говорящее о многом утверждение: «Яхве разнес врагов моих предо мной. Посему и место это он назвал Ваал\hyp{}Перацим». Сделано же это было потому, что восемьдесят процентов солдат Давида были служителями Ваала.
\vs p097 9:13 Поражение Саула у горы Гелвуй Давид объяснил, указав на то, что Саул напал на хананейский город Гаваон, народ которого имел мирный договор с ефремлянами. Из\hyp{}за этого Яхве и оставил их. Но даже во времена Саула Давид оборонял хананейский город Кеиль от филистимлян, а затем расположил свою столицу в хананейском городе. Следуя политике соглашательства с хананеями, Давид выдал на повешение гаванитянам семерых потомков Саула.
\vs p097 9:14 После поражения филистимлян Давид завладел «ковчегом Яхве», внес его в Иерусалим и сделал поклонение Яхве официальной религией своего царства. Затем Давид обложил тяжкой данью соседние племена --- идумеев, моавитян, аммонитян и сирийцев.
\vs p097 9:15 Нарушая нравы иудеев, продажная политическая машина Давида начала переводить в личное владение северные земли и вскоре взяла под свой контроль сбор караванных пошлин, которые прежде взимали филистимляне. Вслед за этим был совершен целый ряд зверств, достигших своей высшей точки в убийстве Урии. Все судебные тяжбы решались в Иерусалиме, и «старейшины» более не могли выносить судебное решение. Неудивительно, что разразилось восстание. В наше время Авессалома можно было бы назвать демагогом; его мать была хананеянкой. Помимо сына Вирсавии Соломона престола добивалось еще полдюжины претендентов.
\vs p097 9:16 \pc После смерти Давида Соломон очистил политическую машину от всех влияний севера, но полностью сохранил тиранию и налоговый гнет режима своего отца. Соломон разорил народ своим расточительным двором и своим обширным планом строительства, куда входили: постройка дома Ливанского, дворец для дочери фараона, храм Яхве, царский дворец и восстановление стен многих городов. Соломон создал большой иудейский флот, на котором работали матросы\hyp{}сирийцы и который торговал со всем миром. Его гарем насчитывал почти тысячу женщин.
\vs p097 9:17 \pc К этому времени храм Яхве в Силохе утратил свое значение, и все поклонение нации сосредоточилось в Иевусе в великолепной царской часовне. Северное царство же в значительной степени вернулось к поклонению Элохим. Его жители пользовались расположением фараонов, которые позднее поработили Иудею и обложили южное царство данью.
\vs p097 9:18 Были и взлеты и падения --- войны между Израилем и Иудеей. После четырех лет гражданской войны и трех династий Израиль оказался во власти городских деспотов, которые начали торговать землей. Даже царь Амврий, и тот пытался купить имение Семира. Однако конец стал быстро приближаться, когда Салманассар Третий решил овладеть берегом Средиземного моря. Ефремский царь Ахав собрал десять других племен и оказал сопротивление в Каркаре; в сражении не было победителя. Ассирийцы были остановлены, но союзники погибли. В Ветхом Завете это великое сражение даже не упоминается.
\vs p097 9:19 Новая беда пришла, когда царь Ахав попытался купить землю у Навуфея. Его жена\hyp{}филистимлянка подделала имя Ахава на бумагах, где было указано, что земля Навуфея должна быть конфискована за то, что тот хулил имена «Элохим и царя». Навуфея и его сыновей немедленно казнили. Но на арене событий появился неутомимый Илия, осудивший Ахава за убийство Навуфея. Таким образом, Илия, один из величайших пророков, начал свое учение как поборник, который отстаивал старые порядки владения землей и выступал против служителей Ваала, желавших землей торговать, против попытки городов добиться господства над сельской местностью. Однако эта реформа не увенчалась успехом, пока землевладелец Ииуй не объединил усилия с цыганским вождем Ионадавом, дабы уничтожить пророков (агентов по торговле недвижимостью) Ваала в Самарии.
\vs p097 9:20 \pc Новая жизнь началась, когда Иоас и его сын Иеровоам освободили Израиль от врагов. Однако к этому времени в Самарии уже правили знатные разбойники, чьи опустошительные набеги не уступали былым набегам династии Давида. Государство и церковь действовали заодно. Попытка подавить свободу слова вынудила Илию, Амоса и Осию приступить к своим тайным писаниям; это\hyp{}то и стало реальным началом еврейской и христианской Библий.
\vs p097 9:21 \pc 97.9.21Однако северное царство не исчезло из истории до тех пор, пока царь Израиля не сговорился с царем Египта и не отказался платить дань Ассирии. Тогда\hyp{}то и началась трехлетняя осада, за которой последовало полное рассеяние северного царства. Ефрем (Израиль), таким образом, исчез. Иудея же --- евреи, «остаток Израиля» --- начала сосредоточивать землю в руках немногих, по словам Исайи, «прибавлявших дом к дому и поле к полю». Вскоре в Иерусалиме наряду с храмом Яхве появился храм Ваала. Этому царству террора положило конец восстание монотеистов во главе с мальчиком\hyp{}царем Иоасом, который сражался за Яхве в течение тридцати пяти лет.
\vs p097 9:22 Следующий царь Амасия имел неприятности с восставшими данниками\hyp{}идумеями и их соседями. После знаменательной победы он напал на своих северных соседей и потерпел столь же знаменательное поражение. Затем восстали сельские жители; они убили царя и посадили на престол его шестнадцатилетнего сына. Это был Азария, которого Исайя назвал Озией. После Озии положение ухудшилось еще больше, и Иудея просуществовала сто лет, платя дань ассирийским царям. Исайя Первый говорил евреям, что Иерусалим, будучи городом Яхве, никогда не падет. Однако Иеремия без колебаний провозгласил его падение.
\vs p097 9:23 \pc Однако настоящая гибель Иудеи произошла из\hyp{}за клики продажных и богатых политиков, которые действовали при правлении малолетнего царя Манассии. Изменявшаяся экономика благоприятствовала возврату к поклонению Ваалу, частные земельные сделки которого противоречили идеологии Яхве. Падение Ассирии и возвышение Египта принесло Иудее временное освобождение, и сельские жители взяли верх. И при Иосии смели иерусалимскую клику продажных политиков.
\vs p097 9:24 Однако и эта эпоха завершилась трагически, когда Иосия решился преградить путь могучей армии Нехао, когда та двигалась из Египта по берегу моря на помощь Ассирии, воевавшей с Вавилоном. Иосия был раздавлен, и Иудея стала платить дань Египту. В Иерусалиме к власти вернулась политическая партия Ваала, и в результате началось \bibemph{настоящее} египетское рабство. За этим последовал период, во время которого политики\hyp{}служители Ваала управляли и правосудием, и священниками. Поклонение Ваалу стало экономической и социальной системой, регулирующей имущественные отношения, а также связанной с плодородием почвы.
\vs p097 9:25 Со свержением Нехао Навуходоносором Иудея перешла под управление Вавилона и в течение десяти лет пользовалась его благоволением, но вскоре восстала. Когда же Навуходоносор пошел на иудеев войной, те, дабы повлиять на Яхве, начали такие социальные реформы, как освобождение рабов. И когда вавилонская армия временно отступила, иудеи возрадовались тому, что волшебство их реформ спасло их. Именно в этот период Иеремия и сказал о приближающейся гибели, и Навуходоносор вскоре возвратился.
\vs p097 9:26 Итак, конец Иудеи наступил внезапно. Город был разрушен, а народ увели в Вавилон. Борьба между Яхве и Ваалом закончилась пленом. Это пленение и толкнуло остаток Израиля к монотеизму.
\vs p097 9:27 \pc В Вавилоне евреи пришли к заключению, что в Палестине они не могут существовать как небольшая группа, имеющая свои собственные особые социальные и экономические обычаи, и что, если они хотят, чтобы их идеология восторжествовала, им следует обращать неевреев. Так и возникло их новое понимание судьбы --- идея, согласно которой евреи должны стать избранными слугами Яхве. Еврейская религия Ветхого Завета, действительно, развилась в Вавилоне во время пленения.
\vs p097 9:28 Учение о бессмертии также сформировалось в Вавилоне. Евреи считали, что идея о будущей жизни умаляет значение евангелия социальной справедливости. И вот теология впервые вытеснила социологию и экономику. Религия воплотилась в систему мышления и поведения, все больше и больше отстранявшуюся от политики, социологии и экономики.
\vs p097 9:29 \pc Итак, истина о еврейском народе открывает, что многое из того, что считалось священной историей, на самом деле лишь немногим отличается от хроники обычной истории мирской. Иудаизм был почвой, на которой возникло христианство, но евреи чудотворным народом не были.
\usection{10. Иудейская религия}
\vs p097 10:1 Вожди учили израильтян, что они --- народ, избранный отнюдь не для особого снисхождения и монополии на божественное благоволение, но для особого служения доносить истину о всеобщем и едином Боге до каждой нации. При этом они обещали евреям, что если те будут достойны этой судьбы, то станут духовными вождями всех народов, и что грядущий Мессия воцарится над ними и над всем светом как Принц Мира.
\vs p097 10:2 Когда евреев освободили персы, они вернулись в Палестину, но оказались в рабстве у своих же собственных кодексов законов, жертв и ритуалов, где безраздельно господствовали священники. И как отвергли иудейские кланы чудесную историю о Боге, представленную в прощальной речи Моисея, ради ритуалов жертв и покаяния, так и эти остатки иудейской нации отвергли величественное представление Исайи Второго ради правил, уставов и ритуалов своего множившегося духовенства.
\vs p097 10:3 Национальный эгоизм, ложная вера в неверно понимаемого обещанного Мессию и усиливающиеся рабство и тирания священников заставили навсегда замолчать голоса духовных вождей (кроме Даниила, Иезекииля, Аггея и Малахии); и с того дня до времени Иоанна Крестителя весь Израиль переживал все более глубокий духовный упадок. Однако евреи никогда не утрачивали представления об Отце Всего Сущего и даже в двадцатом веке после Рождества Христа продолжают придерживаться такого понимания Божества.
\vs p097 10:4 От Моисея до Иоанна Крестителя протянулась непрерывная череда верных учителей, которые из поколения в поколение передавали монотеистический факел света и все сильнее осуждали бессовестных правителей, порицали торгашество священников и постоянно призывали народ быть верным поклонению верховному Яхве, Господу Богу Израиля.
\vs p097 10:5 \pc Как нация евреи в конце концов утратили свое политическое лицо, но иудейская религия, основанная на искренней вере в единого и всеобщего Бога, продолжает жить в сердцах рассеянных изгнанников. Причем религия эта не умирает, потому что она эффективно действовала и сохраняла высшие ценности своих последователей. Еврейская религия сохранила идеалы народа, но не сумела ускорить прогресс и поощрить философские творческие открытия в сфере истины. У еврейской религии было много недостатков --- она была недостаточно философична и почти лишена эстетических качеств --- но сохранила нравственные ценности и поэтому продолжает существовать. По сравнению с другими пониманиями Божества верховный Яхве был ясно очерченным, живым, личным и нравственным.
\vs p097 10:6 Евреи любили справедливость, мудрость, истину и праведность, как любили немногие нации, но они меньше всех народов способствовали интеллектуальному восприятию и духовному пониманию этих божественных качеств. Хотя иудейская теология сопротивлялась расширению, она сыграла важную роль в развитии двух других мировых религий, а именно: христианства и ислама.
\vs p097 10:7 Еврейская религия сохранилась также благодаря своим институтам. Религии трудно выжить в качестве личных обычаев отдельных индивидуумов. Извечная ошибка религиозных лидеров заключается в том, что, видя пороки узаконенной религии, они пытаются разрушить метод коллективного функционирования. Между тем вместо того, чтобы уничтожать все ритуалы, было бы лучше, если бы они их реформировали. В этом отношении Иезекииль был мудрее своих современников; хотя он и соглашался с ними и настаивал на личной моральной ответственности, он в то же время занимался тем, что утверждал правильный порядок высшего и очищенного ритуала.
\vs p097 10:8 \pc Таким образом, сменявшие друг друга учителя Израиля свершили величайшее дело в эволюции религии, когда\hyp{}либо осуществлявшееся на Урантии: постепенную, но непрерывную трансформацию варварского представления о свирепом демоне Яхве, ревнивом и жестоком духе бога грохочущего Синайского вулкана, в более позднее и божественное представление о верховном Яхве, создателе всех вещей, любящем и милосердном Отце всего человечества. Причем это иудейское представление о Боге являлось высшим человеческим видением Отца Всего Сущего вплоть до времени, когда оно было еще более расширено и так изыскано усилено личными учениями и примером жизни его Сына, Михаила из Небадона.
\vsetoff
\vs p097 10:9 [Представлено Мелхиседеком Небадона.]
