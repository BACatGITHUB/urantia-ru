\upaper{171}{В пути в Иерусалим}
\vs p171 0:1 Через день после памятной проповеди «Царство Небесное» Иисус объявил, что на следующий день он и апостолы отправятся на Пасху в Иерусалим и по пути посетят многие города южной Переи.
\vs p171 0:2 Проповедь о царстве и объявление о том, что Иисус собирается на Пасху, привели всех последователей к мысли, что он идет в Иерусалим основать мирское царство верховенства евреев. Что бы ни говорил Иисус о нематериальном характере царства, он так и не смог полностью разубедить слушателей\hyp{}евреев в том, что Мессия должен основать некое националистическое государство с центром в Иерусалиме.
\vs p171 0:3 То, что Иисус сказал в своей субботней проповеди, только еще больше смутило большинство его слушателей; слова Учителя просветили очень немногих. Ближайщие сторонники Иисуса что\hyp{}то понимали в его учениях о внутреннем царстве («царство внутри вас»), однако они также знали, что он говорил о другом и грядущем царстве, и полагали, что именно это царство он и шел основывать в Иерусалим. Когда же они разочаровались в этом ожидании, когда отвергли евреи Иисуса, и позже, когда Иерусалим был в материальном смысле разрушен, они все равно цеплялись за эту надежду, искренне веря, что Учитель вскоре вернется в мир в великой силе и величественной славе и установит обещанное царство.
\vs p171 0:4 \P\ Именно в это воскресенье после полудня Саломея, мать Иакова и Иоанна Заведеевых, пришла к Иисусу со своими сыновьями\hyp{}апостолами и, держа себя так, будто она приближается к восточному властелину, пыталась добиться, чтобы Иисус заранее пообещал выполнить любую просьбу, с какой бы она ни обратилась. Но Учитель не хотел обещать; вместо этого он ее спросил: «Чего хочешь, чтобы я сделал для тебя?» Тогда Саломея ответила: «Учитель, теперь, когда ты идешь в Иерусалим, дабы установить царство, я хочу попросить, чтобы ты наперед пообещал мне, что сии сыновья мои удостоятся славы вместе с тобой, и один сядет от тебя по правую руку, а другой --- по левую в царстве твоем».
\vs p171 0:5 Услышав просьбу Саломеи, Иисус сказал: «Женщина, ты не знаешь, о чем просишь». И затем, обращаясь непосредственно к ищущим славы апостолам, сказал: «Потому ли что я давно знаю и люблю вас; потому ли что я жил в доме матери вашей; потому ли что Андрей велел вам быть со мной во все времена, позволяете матери вашей тайно приходить ко мне с сей неподобающей просьбой? Однако же позвольте и мне спросить вас: можете ли пить чашу, которую я буду пить?» И, ни минуты не подумав, Иаков и Иоанн ответили: «Да, Учитель, можем». Иисус сказал: «Я огорчен тем, что вы не знаете, почему мы идем в Иерусалим; я опечален тем, что вы не понимаете природы моего царства; я разочарован тем, что вы привели вашу мать, чтобы она обратилась ко мне с этой просьбой; но я знаю, вы любите меня в сердцах ваших; и потому объявляю: вы действительно будете пить мою горькую чашу и со мной разделите унижение мое, но дать сидеть от меня по правую руку и по левую --- не от меня зависит. Такие почести уготованы тем, кто был отмечен Отцом моим».
\vs p171 0:6 Тем временем кто\hyp{}то сообщил Петру и другим апостолам об этой беседе, и те сильно вознегодовали, что Иаков и Иоанн пытались добиться, чтобы им было отдано предпочтение перед ними, для чего со своей матерью тайно ходили и обращались с подобной просьбой. Когда же они стали спорить между собой, Иисус собрал их всех и сказал: «Вы хорошо понимаете, как правители неевреев господствуют над своими подданными и как властвуют вельможи. Но в царстве небесном не будет так. Кто хочет между вами быть большим, да будет вначале вам слугой. И кто хочет быть первым в царстве, да будет вам рабом. Объявляю вам, что Сын Человеческий не для того пришел, чтобы ему служили, но чтобы служить; и ныне же иду в Иерусалим положить жизнь мою во исполнение воли Отца и во служение братьям моим». Услышав эти слова, апостолы удалились молиться. В тот вечер усилиями Петра Иаков и Иоанн принесли десяти апостолам подобающие извинения и вернули доброе расположение своих братьев.
\vs p171 0:7 Прося себе места по правую и левую руку Иисуса в Иерусалиме, сыновья Зеведеевы мало понимали, что меньше чем через месяц их возлюбленный Учитель будет висеть на римском кресте, и умирающий вор будет с одной стороны и еще один преступник --- с другой. И мать их, бывшая при распятии, ясно вспомнила глупую просьбу, с которой обратилась к Иисусу, просьбу о почестях, которых она столь неразумно искала для своих сыновей\hyp{}апостолов.
\usection{1. Отправление из Пеллы}
\vs p171 1:1 В понедельник 13 марта еще до наступления полудня Иисус и его двенадцать апостолов окончательно покинули лагерь в Пелле и отправились на юг, по городам южной Переи, где трудились сподвижники Авенира. Они провели более двух недель с семидесятью вестниками, а затем пошли прямо в Иерусалим на Пасху.
\vs p171 1:2 Когда Учитель покинул Пеллу, около тысячи учеников, живших в лагере с апостолами, последовали за ними. Около половины из них оставили Иисуса на дороге в Иерихон у переправы через Иордан, когда, после того, как тот произнес проповедь об «Исчислении издержек», стало известно, что он идет в Есевон. Эти люди пошли в Иерусалим, а другая половина в течение двух недель следовала за Иисусом по городам южной Переи.
\vs p171 1:3 В целом большинство ближайших последователей Иисуса понимало, что лагерь в Пелле покинут навсегда, но они действительно думали: это означает, что их Учитель наконец вознамерился идти в Иерусалим и завладеть престолом Давида. Подавляющее большинство его последователей так и не сумело принять какое\hyp{}либо иное понимание царства небесного; чему бы ни учил их Иисус, они не желали расставаться с еврейской идеей царства.
\vs p171 1:4 Согласно наставлениям апостола Андрея Давид Зеведеев в среду 15 марта закрыл лагерь в Пелле. В это время в нем проживало почти четыре тысячи человек, и это не считая более тысячи живших с апостолами в так называемом лагере учителей и ушедших на юг с Иисусом и двенадцатью апостолами. Как ни хотелось Давиду не делать этого, он распродал все имущество многочисленным покупателям и с вырученными деньгами пошел в Иерусалим, где впоследствии передал их Иуде Искариоту.
\vs p171 1:5 \P\ В течение последней трагической недели Давид был в Иерусалиме, и после распятия забрал свою мать в Вифсаиду. В ожидании Иисуса и апостолов Давид остановился у Лазаря в Вифании и был просто потрясен тем, как фарисеи начали преследовать и травить того после его воскрешения. Андрей приказал Давиду упразднить службу вестников, что всеми было истолковано как признак близкого установления царства в Иерусалиме. Давид остался не у дел и уже почти решил стать добровольным защитником Лазаря, когда предмет его горячей заботы поспешно бежал в Филадельфию. Поэтому спустя какое\hyp{}то время после воскресения, но уже после смерти своей матери, Давид отправился в Филадельфию, но сначала помог Марфе и Марии продать их имение. Там вместе с Авениром и Лазарем он провел остаток своей жизни и ведал финансовыми делами всех больших дел царства, центр которых при жизни Авенира находился в Филадельфии.
\vs p171 1:6 Вскоре после разрушения Иерусалима Антиохия стала центром \bibemph{Павлова христианства;} Филадельфия же оставалась центром \bibemph{Авенирова царства небесного.} Из Антиохии Павлова версия учений Иисуса и об Иисусе распространялась по всему Западному миру; из Филадельфии же миссионеры Авенировой версии царства небесного распространяли ее в Мессопотамии и Аравии вплоть до времени, когда эти бескомпромиссные эмиссары учений Иисуса не были остановлены стремительным ростом влияния Ислама.
\usection{2. Об исчислении издержек}
\vs p171 2:1 Когда Иисус и почти тысяча его последователей подошли к Вифанийской переправе через Иордан, иногда называемой Вифаварой, его ученики начали понимать, что он не идет прямо в Иерусалим. Пока же они медлили и спорили между собой, Иисус взошел на большой камень и произнес проповедь, впоследствии ставшую известной под названием «Исчисление издержек». Учитель сказал:
\vs p171 2:2 \P\ «Вы, желающие отныне следовать за мной, должны быть готовы заплатить цену искреннего посвящения себя исполнению воли моего Отца. Если хотите быть моими учениками, вы должны быть готовы оставить отца, мать, жену, детей, братьев и сестер. Если кто из вас хочет ныне быть моим учеником, тот должен быть готов отдать свою жизнь так же, как Сын Человеческий готов пожертвовать своей жизнью во имя завершения миссии исполнения воли Отца на земле и во плоти.
\vs p171 2:3 Если же вы не готовы пойти на такие жертвы, то едва ли можете быть моими учениками. Перед тем как идти дальше, каждый из вас должен сесть и исчислить издержки, которые ему придется понести, дабы стать моим учеником. Кто из вас примется строить сторожевую башню на землях своих и не сядет прежде и не вычислит издержки, дабы определить, достаточно ли у него денег для совершения ее? Если же не сумеете оценить издержки, то, положив основание, можете обнаружить, что вы не способны закончить начатое, и все соседи ваши будут смеяться над вами, говоря: „Вот этот человек начал строить, но не смог закончить работу свою“. Опять же какой царь, готовясь к войне с другим царем, не сядет и не посоветуется прежде, сможет ли он с десятью тысячами человек встретить идущего против него с двадцатью тысячами? Если царь не может встретить врага своего, потому что к сему не готов, то посылает посольство к этому царю, когда тот еще далеко, спрашивая об условиях мира.
\vs p171 2:4 Теперь, стало быть, каждый из вас должен сесть и исчислить издержки, которые ему придется понести, дабы стать моим учеником. Впредь вы не сможете следовать за нами, слушая учение и наблюдая дела; вам придется столкнуться с жестокими преследованиями и свидетельствовать о сем евангелии, невзирая на сокрушительное разочарование. Если вы не готовы отрешиться от всего, что вы есть, и отдать все, что имеете, значит, вы недостойны быть моими учениками. Если же вы уже покорили себя в сердце своем, то вам нечего бояться той внешней победы, которую вам вскоре надлежит одержать, когда Сын Человеческий будет отвержен первосвященниками и саддукеями и отдан в руки насмехающимся неверующим.
\vs p171 2:5 Теперь вы должны разобраться в себе и определить, почему вам хочется быть моими учениками. Если вы ищете почестей и славы, если думаете о мирском, значит, вы подобны соли, потерявшей силу. Когда же то, что ценится за свою соленость, потеряло силу, чем исправишь его? Такая приправа бесполезна; она годится только, чтобы вы бросили ее вместе с мусором. Ныне я предостерег вас: возвращайтесь к домам вашим с миром, если не можете пить со мной приготовляемую чашу. Снова и снова я говорил вам, что царство мое не от мира сего, но вы не хотели мне верить. Имеющий уши да слышит, что я говорю».
\vs p171 2:6 \P\ Произнеся эти слова, Иисус сразу отправился в путь к Есевону, ведя за собой двенадцать апостолов, и вслед за ними пошло около пятисот человек. Немного помедлив, другая половина толпы пошла в Иерусалим. Апостолы Иисуса вместе с самыми выдающимися учениками много думали над этими словами, однако по\hyp{}прежнему продолжали верить, что после этого недолгого периода бед и испытаний, царство обязательно установится практически в полном соответствии с их давно лелеемыми надеждами.
\usection{3. Путешествие по Перее}
\vs p171 3:1 Больше двух недель Иисус и двенадцать апостолов, сопровождаемые толпой из нескольких сот учеников, ходили по южной Перее по всем городам, где трудились семьдесят вестников. В этой области жило много неевреев, а поскольку на праздник Пасхи в Иерусалим собирались идти очень немногие, вестники царства продолжали свою работу, уча и проповедуя.
\vs p171 3:2 В Есевоне Иисус встретился с Авениром, и Андрей дал указание, чтобы семьдесят вестников не прерывали труды свои в праздник Пасхи; Иисус же посоветовал им продолжать свое дело, совершенно не обращая внимания на то, что должно было произойти в Иерусалиме. Кроме того, он дал Авениру совет разрешить женскому отряду, или по крайней мере желающим, идти в Иерусалим на Пасху. И после этого Авенир больше уже не видел Иисуса во плоти. Прощаясь Иисус сказал Авениру: «Сын мой, я знаю, ты будешь верен царству, и молю Отца даровать тебе мудрость, чтобы ты мог любить и понимать своих братьев».
\vs p171 3:3 Пока они переходили из города в город, многие из следовавших за ними оставили Иисуса и пошли в Иерусалим, так что к моменту, когда Иисус отправился на Пасху, тех, кто день за днем следовал за ним, оставалось не более двухсот человек.
\vs p171 3:4 Апостолы понимали, что Иисус идет в Иерусалим на Пасху. Они знали: синедрион передал всему Израилю сообщение о том, что он приговорен к смерти, и приказал всем, кому известно его местонахождение, уведомлять об этом синедрион; и все же, несмотря на все это, они были гораздо меньше встревожены, чем тогда, когда в Филадельфии Иисус объявил им, что идет в Вифанию повидаться с Лазарем. Эта перемена настроения, переход от сильного страха к состоянию молчаливого ожидания, была вызвана главным образом воскрешением Лазаря. Они пришли к заключению, что в случае крайней необходимости Иисус проявит свою божественную силу и посрамит врагов. Эта надежда вкупе с их еще более твердой и зрелой верой в духовное верховенство их Учителя и объясняла заметную смелость, проявляемую его ближайшими последователями, которые теперь приготовились следовать за ним в Иерусалим, невзирая на открытое объявление синедриона о том, что он должен умереть.
\vs p171 3:5 Большинство апостолов и многие из ближайших учеников Иисуса не верили в возможность смерти Иисуса и, полагая, что он есть «воскресение и жизнь», считали его бессмертным и уже победившим смерть.
\usection{4. Учение в Ливии}
\vs p171 4:1 9 марта, в среду вечером, Иисус и его последователи, завершив свое путешествие по городам южной Переи, по пути в Иерусалим расположились лагерем в Ливии. Именно здесь, в Ливии, в эту ночь Симон Зилот и Симон Петр, получили на руки более ста мечей, о чем еще ранее имелась договоренность, и было раздано это оружие всем желающим, чтобы они носили его спрятанным под верхней одеждой. У Симона Петра был его меч и в ту ночь, когда он предал Учителя в саду.
\vs p171 4:2 В четверг рано утром, пока не проснулись другие, Иисус призвал Андрея и сказал: «Разбуди своих братьев! Я должен им что\hyp{}то сказать». Иисус знал о мечах и кто из апостолов получил и носил это оружие, но так и не открыл им, что ему известно об этом. Когда Андрей разбудил своих товарищей и те в стороне собрались, Иисус сказал: «Дети мои, давно вы со мной, и я научил вас многому, что необходимо в се время, но теперь хочу предостеречь вас: не полагайтесь ни на бренную плоть, ни на ненадежную человеческую защиту от испытаний и невзгод, которые ожидают нас впереди. Я призвал вас сюда одних, дабы еще раз прямо сказать вам: мы идем в Иерусалим, где, как вы знаете, Сына Человеческого уже приговорили к смерти. Я снова говорю вам, что Сын Человеческий будет предан в руки первосвященников и религиозных правителей; что они осудят его и предадут его язычникам. И надругаются над Сыном Человеческим, оплюют и будут бичевать его; и предадут его смерти. Когда же убьют Сына Человеческого, не впадайте в уныние, ибо объявляю, что на третий день он воскреснет. Будьте осторожны и помните, о чем я предостерег вас!»
\vs p171 4:3 И снова апостолы были изумлены и ошеломлены, но не могли решиться понять его слова в прямом смысле; они не могли осознать, что Учитель имеет в виду именно то, что говорит. Они были настолько ослеплены своей упорной верой во временное царство на земле с центром в Иерусалиме, что просто не могли --- не хотели --- позволить себе воспринять слова Иисуса буквально. Весь день они размышляли о том, что хотел сказать Иисус столь странными заявлениями. Но ни один из них не решился задать ему вопрос относительно этих высказываний. И лишь после смерти Иисуса сих сбитых с толку апостолов осенило, что, предвидя свое распятие, Учитель говорил с ними прямо и открыто.
\vs p171 4:4 \P\ Здесь же в Ливии сразу после завтрака некий дружественно настроенный фарисей подошел к Иисусу и сказал: «Беги быстрее из этих мест, ибо Ирод так же, как когда\hyp{}то искал убить Иоанна, ныне ищет убить тебя. Он боится народного восстания и решил тебя убить. Мы тебя предостерегаем, чтобы ты мог спастись».
\vs p171 4:5 И отчасти это было верно. Воскрешение Лазаря испугало и насторожило Ирода и, зная, что синедрион посмел осудить Иисуса еще до суда, Ирод решил либо убить Иисуса, либо изгнать его из своих владений. В действительности он хотел сделать только последнее, ибо настолько боялся Иисуса, что надеялся, что ему не придется его казнить.
\vs p171 4:6 Выслушав фарисея, Иисус возразил: «Об Ироде и его страхе перед сим евангелием царства мне известно все. Однако не заблуждайтесь, он скорее предпочтет, чтобы Сын Человеческий взошел в Иерусалим, дабы пострадать и умереть от рук первосвященников; запятнав руки кровью Иоанна, он не хочет быть ответственным и за смерть Сына Человеческого. Пойди и скажи этой лисице, что Сын Человеческий сегодня проповедует в Перее, завтра пойдет в Иудею, а через несколько дней завершит свою миссию на земле и будет готов вознестись к Отцу».
\vs p171 4:7 Затем, повернувшись к своим апостолам, Иисус сказал: «С древнейших времен гибли пророки в Иерусалиме, и Сыну Человеческому подобает взойти во град дома Отца и быть принесенным в жертву человеческому фанатизму, религиозным предубеждениям и духовной слепоте. О Иерусалим, Иерусалим, убивающий пророков и камнями побивающий учителей истины! Сколько раз хотел я собрать детей твоих, как птица собирает птенцов своих под крылья, и вы не захотели позволить мне сделать это! Се, оставляется вам дом ваш пуст! Много раз возжелаете увидеть меня, и не увидите. Будете искать меня, и не найдете». И, сказав это, обратился к окружавшим его, говоря: «Тем не менее, пойдем в Иерусалим на Пасху и совершим подобающее нам во исполнение воли Отца Небесного».
\vs p171 4:8 \P\ В этот день за Иисусом в Иерихон вошла группа смущенных и сбитых с толку верующих. В заявлениях Иисуса о царстве апостолы могли распознать только уверенную ноту окончательной победы; просто не могли даже представить себе что\hyp{}нибудь, что заставило бы их понять предостережение о предстоящей неудаче. Когда Иисус сказал о «воскресении на третий день», они ухватились за это высказывание, как предвещающее уверенную победу царства, которая немедленно последует за неприятной предварительной схваткой с еврейскими религиозными лидерами. «На третий день» было распространенным еврейским выражением, означавшим «вскоре» или «спустя какое\hyp{}то время». Когда Иисус говорил о «воскресении», они полагали, что он говорит о «воскресении царства».
\vs p171 4:9 Эти верующие воспринимали Иисуса как Мессию, и евреи знали очень мало или не знали ничего о страдающем Мессии. Они не понимали, что своей смертью Иисус должен совершить многое, чего нельзя было достигнуть его жизнью. В то время, как воскрешение Лазаря придавало апостолам мужества и помогало им войти в Иерусалим, Учителя в этот тяжелый период его пришествия укрепляло воспоминание о преображении.
\usection{5. Слепой в Иерихоне}
\vs p171 5:1 В четверг 30 марта спустя несколько часов по полудню Иисус и его апостолы во главе отряда, состоявшего почти из двухсот последователей, приблизились к стенам Иерихона. Подойдя к городским воротам, они неожиданно встретили толпу нищих, среди которых был некий Вартимей, старец, слепой с юности. Этот слепой нищий много слышал об Иисусе и знал все об исцелении им слепого Иосии в Иерусалиме. Он не знал о последнем посещении Иисусом Иерихона до тех пор, пока тот не ушел в Вифанию. Вартимей решил, что больше не допустит, чтобы, в то время как Иисус будет в Иерихоне, он не обратился бы к нему с просьбой вернуть ему зрение.
\vs p171 5:2 Весть о приближении Иисуса разнеслась по всему Иерихону, и сотни жителей вышли его встречать. Когда же эта огромная толпа возвращалась, сопровождая Учителя в город, Вартимей, слыша гул толпы, понял, что случилось нечто необычное, и потому спросил стоявших рядом с ним, что происходит. И один из нищих ответил: «Иисус Назорей идет». Услышав, что Иисус рядом, Вартимей стал во весь голос громко кричать: «Иисус, Иисус, помилуй меня!» И поскольку он продолжал кричать все громче и громче, некоторые шедшие с Иисусом подошли и упрекнули Вартимея, попросив его вести себя спокойно; но безуспешно; он кричал только сильнее и громче.
\vs p171 5:3 Услышав крики слепого, Иисус остановился. И, увидев его, сказал своим друзьям: «Приведите этого человека ко мне». Тогда они подошли к Вартимею и сказали: «Не бойся, пойдем с нами, ибо Учитель зовет тебя». Услышав эти слова, Вартимей сбросил с себя верхнюю одежду, выскочил на середину дороги, и бывшие рядом с ним подвели его к Иисусу. Обращаясь к Вартимею, Иисус сказал: «Чего ты хочешь от меня?» Тогда слепой ответил: «Чтобы мне прозреть». Иисус же, услышав эту просьбу и увидев веру его, сказал: «Ты прозреешь; ступай своей дорогой; вера твоя спасла тебя». И Вартимей сразу прозрел и оставался рядом с Иисусом, славя Бога, пока на следующий день Учитель не отправился в Иерусалим; тогда Вартимей пошел перед толпой, объявляя всем, как он прозрел в Иерихоне.
\usection{6. Посещение Закхея}
\vs p171 6:1 Когда процессия с Учителем вошла в Иерихон, время приближалось к закату, и он решил здесь заночевать. Когда Иисус проходил мимо таможни, рядом оказался начальник мытарей, или сборщик налогов, Закхей, который очень хотел увидеть Иисуса. Этот человек был очень богат и многое слышал о пророке из Галилеи. Закхей решил, что в следующий раз, когда Иисусу случится быть в Иерихоне, он увидит, что он за человек; поэтому Закхей старался протиснуться через толпу, но толпа была слишком велика, и он, будучи невысокого роста, ничего не видел за головами людей. Поэтому начальник мытарей шел за Иисусом вместе с толпой, пока они не приблизились к центру города и не оказались рядом с местом, где жил Закхей. Поняв, что протиснуться сквозь толпу не удастся, и думая, что Иисус может свернуть направо и пройти через город не останавливаясь, он забежал вперед и забрался на смоковницу, чьи раскидистые ветви свисали над дорогой. Он знал, что так сможет хорошо рассмотреть Учителя, когда тот будет проходить мимо. И Закхей не был разочарован, ибо, проходя мимо, Иисус остановился и, взглянув на Закхея, сказал: «Закхей, сойди скорее, ибо сегодня ночью мне надобно быть у тебя в доме». Услышав эти поразительные слова, Закхей, поспешно спускаясь, чуть не упал с дерева и, подойдя к Иисусу, выразил великую радость, что Учитель желает остановиться у него в доме.
\vs p171 6:2 Они сразу пошли к дому Закхея, и те, кто жил в Иерихоне, были сильно удивлены тем, что Иисус согласился гостить у начальника мытарей. Когда же Учитель и его апостолы стояли с Закхеем перед дверью его дома, один из иерихонских фарисеев, находившихся рядом, сказал: «Вы видите, как этот человек пришел остановиться у грешника, отступника из сыновей Авраамовых, который есть вымогатель и грабитель своего собственного народа». Услышав это, Иисус посмотрел на Закхея и улыбнулся. Тогда Закхей встал на стул и сказал: «Жители Иерихона, слушайте меня! Может быть я и мытарь, и грешник, но великий Учитель пришел остановиться в доме моем; и прежде, чем он вошел, говорю вам: я собираюсь половину всего моего имения отдать бедным и, начиная с завтрашнего дня, если я с кого неправильно что\hyp{}то взыскал, буду воздавать вчетверо. Я собираюсь искать спасения всем сердцем своим и учиться поступать праведно в глазах Бога».
\vs p171 6:3 Когда Закхей кончил говорить, Иисус сказал: «Ныне пришло спасение дому сему, и ты действительно стал сыном Авраама». И, повернувшись к толпе, собравшейся около них, сказал: «Не удивляйтесь тому, что я сказал, и не обижайтесь на то, что мы делаем, ибо я всегда заявлял, что Сын Человеческий пришел найти и спасти погибшее».
\vs p171 6:4 Ночь они провели у Закхея. А на следующее утро встали и, направляясь на Пасху в Иерусалим, пошли по «дороге грабителей» в Вифанию.
\usection{7. «Когда Иисус проходил мимо»}
\vs p171 7:1 Куда бы ни шел Иисус, он всюду приносил радость. Он был полон благоволения и истины. Его сподвижники не переставали удивляться благодатным словам, исходившим из его уст. Можно привить хорошие манеры, но благостность --- это аромат дружелюбия, исходящий от напоенной любовью души.
\vs p171 7:2 Добродетель всегда вызывает уважение, однако добродетель, лишенная благодати, часто любовь отталкивает. Добродетель притягательна для всех лишь тогда, когда она благостна. Добродетель действенна лишь тогда, когда она притягательна.
\vs p171 7:3 Иисус действительно понимал людей; вот почему он мог проявлять подлинное сочувствие и выказывать искреннее сострадание. Однако жалости он предавался редко. Хотя его сострадание не знало границ, его сочувствие было практичным, личным и созидательным. Никакое страдание не оставляло его равнодушным, и он умел служить попавшим в беду душам, не возбуждая в них жалость к самим себе.
\vs p171 7:4 Иисус мог так много помогать людям, потому что так искренне их любил. Он действительно любил каждого мужчину, каждую женщину и каждого ребенка. Таким верным другом он мог быть благодаря своей замечательной проницательности он прекрасно знал, что таится в сердце и в мыслях человека. Он был заинтересованным и тонким наблюдателем. Он прекрасно понимал нужды человеческие и умело разбирался в человеческих устремлениях.
\vs p171 7:5 Иисус никогда не спешил. У него всегда находилось время, чтобы утешить своих собратьев\hyp{}людей, «когда он проходил мимо». И он делал все, чтобы его друзья чувствовали себя непринужденно в его обществе. Он был чудесным слушателем. И никогда надоедливо не копался в душах своих сподвижников. Когда он утешал алчущие умы и служил жаждущим душам, те, кто искал его милосердия, чувствовали, что они не столько \bibemph{ему} исповедуются, сколько советуются \bibemph{с ним.} Они безгранично доверяли ему, потому что видели, сколько у него веры в них.
\vs p171 7:6 Он никогда не любопытствовал и никогда не проявлял желания управлять, руководить ими или проверять их. Во всех, кто наслаждался общением с ним, он вдохновлял глубокую уверенность в себе и большую смелость. Когда он кому\hyp{}нибудь улыбался, то такой смертный чувствовал, что у него появились силы решать свои многочисленные проблемы.
\vs p171 7:7 Иисус любил людей так сильно и так мудро, что всегда не колеблясь становился суровым, если положение требовало такого взыскания. Он часто помогал человеку тем, что просил помощи сам. Таким образом он возбуждал интерес, взывал к лучшему в человеческой природе.
\vs p171 7:8 Учитель сумел разглядеть спасительную веру в обычном суеверии женщины, пытавшейся исцелиться прикосновением к его одежде. Он был всегда готов прервать проповедь или заставить толпу ждать, пока помогал одному\hyp{}единственному нуждающемуся человеку, пусть даже ребенку, и делал это охотно. Великое совершалось не только потому, что люди верили в Иисуса, но и потому, что у Иисуса было так много веры в них.
\vs p171 7:9 Большинство поистине важных вещей, которые Иисус сказал или сделал, казалось, происходили случайно, «когда он проходил мимо». В земном служении Учителя было так мало профессионального, тщательно спланированного или преднамеренного. Идя по жизни, он естественно и с удовольствием раздавал здоровье и расточал счастье. «Он ходил, благотворя», и это было в буквальном смысле так.
\vs p171 7:10 Последователям Учителя, в какие бы времена они ни жили, следует научиться служить, «когда они проходят мимо», --- исполняя свои повседневные обязанности, бескорыстно делать добро.
\usection{8. Притча о минах}
\vs p171 8:1 Из Иерихона они вышли лишь около полудня, поскольку предыдущей ночью долго не ложились спать, слушая, как Иисус учил Закхея и его семью евангелию царства. Приблизительно на середине поднимавшегося в гору пути в Вифанию они остановились обедать; толпа же пошла дальше в Иерусалим, не зная, что Иисус и апостолы собирались провести ту ночь на Масличной горе.
\vs p171 8:2 Притча о минах в отличие от притчи о талантах, адресованной всем ученикам, была прежде всего рассказана исключительно апостолам и в основном основана на опыте Архелая и его тщетной попытке получить власть над Иудейским царством. Это --- одна из немногих притч Учителя, основанная на жизни подлинных исторических лиц. Неудивительно, что они должны были подумать об Архелае, тем более что и дом Закхея в Иерихоне был очень близко от богато украшенного дворца Архелая, и его акведук проходил вдоль дороги, по которой они шли, покинув Иерихон.
\vs p171 8:3 \P\ Иисус сказал: «Вы думаете, что Сын Человеческий идет в Иерусалим принять царство, но я объявляю вам: вас постигнет разочарование. Разве не помните вы о некоем принце, который пошел в дальнюю страну, чтобы получить себе царство; однако не успел он возвратиться, как граждане его провинции, в сердцах своих уже отвергшие его, выслали вслед за ним посольство, сказав: „Не хотим, чтобы сей человек царствовал над нами“? Как сей царь был отвергнут во временном правлении, так и Сын Человеческий будет отвергнут в правлении духовном. Я снова объявляю вам: царство мое не от мира сего; однако если бы Сыну Человеческому дано было духовное правление над своим народом, он принял бы такое царство человеческих душ и царствовал бы над таким владением сердец человеческих. Несмотря на то, что они отвергли мое духовное правление над ними, я снова вернусь и получу такое царство духа, в котором сейчас мне отказано, от других. Ныне вы увидите, как отвергают Сына Человеческого, но пройдет время и то, что дети Авраамовы теперь отвергают, будет принято и вознесено.
\vs p171 8:4 Теперь же, подобно отвергнутому человеку высокого рода из этой притчи, я хочу призвать к себе двенадцать моих слуг, моих особых управляющих, и, вручив каждому из вас одну мину, посоветовать каждому исполнять мои наставления и, пока меня нет, должным образом отдать вверенные вам средства в оборот, дабы у вас было, чем оправдать свое управление, когда я вернусь и от вас будет потребован расчет.
\vs p171 8:5 Если даже сей отверженный Сын не вернется, принимать царство будет другой Сын, и тогда Сын этот пошлет за всеми вами, дабы получить ваш отчет об управлении и порадоваться вашим приобретениям.
\vs p171 8:6 Когда же эти управляющие впоследствии были созваны для расчета, первый вышел вперед и сказал: „Господин, мина твоя принесла десять мин“. И его господин сказал ему: „Хорошо, ты --- добрый слуга; за то, что ты был верен в этом деле, я дам тебе в управление десять городов“. И второй пришел, говоря: „Мина, которую ты оставил мне, господин, принесла пять мин“. И сказал господин: „Посему сделаю тебя управителем над пятью городами“. И так же поступал с остальными, пока последний из слуг, будучи призван дать отчет, не ответил: „Господин, вот твоя мина, которую я надежно хранил, завернув в этот платок. Поступил же так я потому, что боялся тебя; я считал, что ты человек несправедливый, ибо видел, что ты берешь там, где не клал, и жнешь там, где не сеял“. Тогда его господин сказал: „Нерадивый и неверный слуга, твоими устами буду судить тебя. Ты знал, что я жну там, где, казалось бы, не сеял; следовательно, знал, что сей отчет будет потребован от тебя. Зная это, ты, по крайней мере, должен был отдать мои деньги в рост, чтобы я, пришедши, получил их с надлежащей прибылью“.
\vs p171 8:7 И затем сей правитель сказал стоявшим рядом: „Возьмите деньги у сего ленивого слуги и отдайте тому, кто приобрел десять мин“. Когда же они напомнили господину, что у того уже есть десять мин, господин сказал: „Всякому имеющему будет дано еще, а у неимеющего отнимется и то, что имеет“».
\vs p171 8:8 \P\ Тогда апостолы попытались узнать разницу между смыслом этой притчи и смыслом рассказанной им ранее притчи о талантах, но Иисус в ответ на их многочисленные вопросы только сказал: «Как следует обдумывайте слова эти в сердцах ваших, пока каждый из вас ищет их истинный смысл».
\vs p171 8:9 О значении этих двух притч особенно хорошо в последующие годы учил Нафанаил, который обобщил свои учения в следующих заключениях:
\vs p171 8:10 \ublistelem{1.}\bibnobreakspace Способности --- вот практическая мера жизненных возможностей. Никто не будет отвечать за совершение того, что превышает его способности.
\vs p171 8:11 \P\ \ublistelem{2.}\bibnobreakspace Верность --- вот безошибочная мера человеческой надежности. Верный в малом, вероятно, проявит верность и во всем, соответствующем его дарованиям.
\vs p171 8:12 \P\ \ublistelem{3.}\bibnobreakspace Учитель при равных возможностях за меньшую верность дает и меньшую награду.
\vs p171 8:13 \P\ \ublistelem{4.}\bibnobreakspace При меньших возможностях он дает равную награду за равную верность.
\vs p171 8:14 \P\ Когда они закончили трапезу и после того, как толпа последователей ушла в Иерусалим, Иисус, став перед апостолами в тени нависшей над дорогой скалы, с достоинством и величественно указал пальцем на запад и сказал: «Братья мои, пойдем в Иерусалим и примем там ожидающее нас; так мы во всем исполним волю Отца Небесного».
\vs p171 8:15 Итак, Иисус и его апостолы продолжили этот последний путь Учителя в Иерусалим в подобии плоти смертного человека.
