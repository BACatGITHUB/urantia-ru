\upaper{3}{Атрибуты Бога}
\vs p003 0:1 Бог вездесущ; Отец Всего Сущего правит кругом вечности. Однако в локальных вселенных он правит в лице своих Райских Сынов\hyp{}Творцов, равно как и дарует жизнь через этих Сынов. <<Бог дал нам жизнь вечную, и сия жизнь --- в Сыновьях его>>. Эти Сыны\hyp{}Творцы Бога являются личным выражением его самого в секторах времени и для детей, которые обитают на вращающихся планетах развивающихся вселенных пространства.
\vs p003 0:2 В высокой степени персонализированные Сыны Бога ясно видимы для низших чинов сотворенных разумных существ, а потому возмещают невидимость бесконечного и, следовательно, менее различимого Отца. Райские Сыны\hyp{}Творцы Отца Всего Сущего являются откровением об иначе невидимом существе --- невидимом из\hyp{}за абсолютности и бесконечности, присущих кругу вечности и личностям Райских Божеств.
\vs p003 0:3 \P\ Едва ли творчество является атрибутом Бога; скорее оно --- совокупность его действующей природы. Причем эта универсальная функция творчества всегда проявляется в том виде, в каком она обусловлена и управляется всеми согласованными атрибутами бесконечной и божественной реальности Первоисточника и Центра. Мы искренне сомневаемся, может ли какое\hyp{}либо одно свойство божественной природы рассматриваться в качестве предшествующего другим, но если бы это и было так, то творческая природа Божества предшествовала бы всем другим природам, деятельностям и атрибутам. Причем творчество Божества достигает кульминации во вселенской истине Отцовства Бога.
\usection{1.\bibnobreakspace Вездесущность Бога}
\vs p003 1:1 Способность Отца Всего Сущего присутствовать всюду и в одно и то же время образует его вездесущность. Один только Бог может быть в двух местах, в бесконечном количестве мест в одно и то же время. Бог одновременно присутствует <<на небе вверху и на земле внизу>>; как восклицал псалмопевец: <<Куда пойду от духа твоего? Или от присутствия твоего куда убегу?>>
\vs p003 1:2 <<Я --- Бог вблизи так же, как и вдали, --- говорит Господь. --- Не я ли наполняю небо и землю?>> Отец Всего Сущего все время присутствует во всех частях и во всех сердцах своего обширного творения. Он --- <<полнота наполняющего все во всем>> и <<совершающего все во всем>>; более того, представление о его личности таково, что <<небо (вселенная) и небо небес (вселенная вселенных) не могут вместить его>>. То, что Бог есть все и во всем, истинная правда. Но даже это --- не \bibemph{все} что есть Бог. Бесконечный может окончательно раскрыться лишь в бесконечности; причину нельзя полностью понять, анализируя последствия; живой Бог неизмеримо больше общей суммы всего творения, которое стало существовать в результате творческих актов его нестесненной свободной воли. Бог открывается через космос, но космосу никогда не вместить или заключить в себе полноту бесконечности Бога.
\vs p003 1:3 Присутствие Отца беспрестанно охраняет главную вселенную. <<От края небес исход его; и круг его до края небес; и ничто не скрыть от света его>>.
\vs p003 1:4 \P\ Не только творение существует в Боге, но и Бог живет в творении. <<Мы знаем, что пребываем в нем, потому что он живет в нас; он даровал нам дух свой. Этот дар от Райского Отца --- неразлучный спутник человека>>. <<Он --- вездесущий и всюду проникающий Бог>>. <<Дух вечного Отца сокрыт в разуме каждого смертного чада>>. <<Человек отправляется на поиски друга, в то время как друг этот живет в его собственном сердце>>. <<Истинный Бог не вдали; он --- часть нас; дух его говорит изнутри нас>>. <<Отец живет в детях. Бог всегда с нами. Он --- руководящий дух вечного предназначения>>.
\vs p003 1:5 О роде человеческом верно сказано: <<Вы от Бога>>, потому что <<пребывающий в любви пребывает в Боге, и Бог в нем>>. Греша, вы причиняете страдания пребывающему в вас дару Бога, ибо Настройщик Мысли должен пройти через последствия порочного мышления вместе с человеческим разумом, в котором Настройщик и заключен.
\vs p003 1:6 \P\ Вездесущность Бога в действительности является частью его бесконечной природы; пространство для Божества --- не препятствие. Бог в совершенстве и без ограничений присутствует явно только в Раю и в центральной вселенной. И не присутствует зримо в творениях, окружающих Хавону, ибо Бог ограничил свое непосредственное и действительное присутствие, признавая владычество и божественные прерогативы равных ему творцов и правителей вселенных времени и пространства. Следовательно, понятие божественного присутствия должно допускать широкий диапазон как вида, так и способа проявления, заключающего в себе контуры присутствия Вечного Сына, Бесконечного Духа и Острова Рая. Также не всегда возможно отличить присутствие Отца Всего Сущего от действий вечных и равных ему божеств и посланников, настолько совершенно исполняют они бесконечные требования его неизменного замысла. Однако в отношении контура личности Настройщиков все обстоит иначе; здесь Бог действует неповторимо, непосредственно и исключительно.
\vs p003 1:7 \P\ Вселенский Контролер потенциально присутствует в контурах тяготения Острова Рая во всех частях вселенной, во все времена и в одинаковой степени, в соответствии с массой, в ответ на физические потребности в этом присутствии и природной сущности всего творения, которая вынуждает все вещи быть приверженными ему и в нем заключаться. Точно также Первоисточник и Центр потенциально присутствуют в Неограниченном Абсолюте, хранилище несотворенных вселенных вечного будущего. Таким образом, Бог потенциально наполняет собой физические вселенные прошедшего, настоящего и будущего. Он --- первооснова гармоничности так называемого материального мира. Этот недуховный потенциал Божества становится действительностью там и сям, на всем уровне физических существований, когда какая\hyp{}либо одна из его исключительных сил необъяснимо вторгается на сцену действия во вселенной.
\vs p003 1:8 Присутствие разума Бога соотнесено с абсолютным разумом Носителя Объединенных действий, Бесконечным Духом, однако в конечных творениях оно лучше различимо в повсеместном действии космического разума Духов\hyp{}Мастеров Рая. Как Первоисточник и Центр потенциально присутствует в контурах разума Носителя Объединенных Действий, так и он потенциально присутствует в напряженностях Вселенского Абсолюта. Разум же человеческого чина является даром Дочерей Носителя Объединенных Действий, Божественных Служительниц развивающихся вселенных.
\vs p003 1:9 Вездесущий дух Отца Всего Сущего согласован с действием всемирного присутствия духа Вечного Сына и вечным божественным потенциалом Божественного Абсолюта. Однако ни духовная деятельность Вечного Сына и его Райских Сыновей, ни дарования разума Бесконечным Духом не исключают прямого действия Настройщиков Мысли, частиц Бога, пребывающих в сердцах сотворенных им детей.
\vs p003 1:10 Что же касается присутствия Бога на планете, в системе, в созвездии или во вселенной, то степень такого присутствия в любой единице творения является мерой степени развивающегося присутствия Верховного Существа: оно определяется общим признанием Бога и верностью ему со стороны огромной вселенской организации, вплоть до самих систем и планет. Поэтому иногда, надеясь сохранить и защитить эти фазы драгоценного присутствия Бога, когда некоторые планеты (или даже системы) глубоко погружаются в духовную тьму, тогда они в определенном смысле подвергаются карантину или частично изолируются от общения с более крупными единицами творения. Причем все это в том виде, в каком оно действует на Урантии, является духовно оборонительной реакцией большинства миров, направленной на то, чтобы по мере возможности избавить себя от страдания от изолирующих последствий отчуждающих действий своевольного, порочного и мятежного меньшинства.
\vs p003 1:11 \P\ Хотя Отец по\hyp{}родительски окружает всех своих сыновей --- все личности --- его влияние на них ограничено отдаленностью их происхождения от Второго и Третьего Лиц Божества и усиливается по мере достижения ими своего предназначения, приближающего к этим уровням. \bibemph{Факт} присутствия Бога в умах творения определяется тем, пребывают ли в них такие частицы Отца, как Таинственные Помощники, однако его \bibemph{действительное} присутствие определяется степенью сотрудничества, предоставленного этим Настройщикам, с умами тех, в ком они пребывают.
\vs p003 1:12 Флуктуации присутствия Отца отнюдь не вызваны изменчивостью Бога. Отец не удаляется в уединение, потому что им пренебрегают; и его любовь не охладевает из\hyp{}за греха творения. Скорее, наоборот, его дети, будучи наделенными возможностью выбора (в отношении его самого), при осуществлении этого выбора сами определяют степень и ограничения божественного влияния Отца в своих сердцах и душах. Отец бескорыстно даровал нам себя без ограничения и без предпочтения. Он не взирает на лица, планеты, системы или вселенные. В секторах времени он оказывает особые почести только Райским лицам Бога Семеричного, равным друг другу творцам конечных вселенных.
\usection{2.\bibnobreakspace Бесконечное могущество Бога}
\vs p003 2:1 Все вселенные знают, что <<всемогущий Господь Бог царствует>>. Делами этого мира и делами других миров божественно управляет. <<По воле своей он действует в небесном воинстве и среди живущих на земле>>. Слова: <<Нет могущества не от Бога>> вечно истинны.
\vs p003 2:2 В пределах совместимого с божественной природой буквально истинно то, что с <<Богом все возможно>>. Продолжительные растянувшиеся процессы эволюции народов, планет и вселенных находятся под совершенным контролем вселенских творцов и управителей и разворачиваются согласно вечному замыслу Отца Всего Сущего, происходя в гармонии, согласии и соответствии с премудрым планом Бога. Существует только один законодатель. Он поддерживает миры в пространстве и вращает вселенные по бесконечному кругу контура вечности.
\vs p003 2:3 Из всех божественных атрибутов всемогущество, особенно как оно распространено в материальной вселенной, понятно лучше всего. Если на Бога смотреть как на явление недуховное, то Бог есть энергия. Эта декларация физического факта основана на непостижимой истине, которая заключается в том, что Первоисточник и Центр является первопричиной вселенских физических явлений всего пространства. От этой божественной деятельности происходят вся физическая энергия и другие материальные проявления. Свет, а точнее, свет без тепла, представляет собой еще одно из недуховных проявлений Божеств. Причем существует еще одна форма недуховной энергии, которая на Урантии фактически неизвестна и пока еще не открыта.
\vs p003 2:4 Бог управляет всей мощью; он создал <<путь для молнии>> и назначил контуры для всякой энергии. Он определил время и способ проявления всех форм энергии\hyp{}материи. Причем все это навечно удерживается в его вечной власти --- в гравитационном контроле, сосредоточенном в нижнем Рае. Свет и энергия вечного Бога, таким образом, вечно вращаются по величественному контуру, а бесконечное, но упорядоченное шествие звездных воинств образует вселенную вселенных. Все творение вечно вращается вокруг Рая\hyp{}Личности, центра всех вещей и существ.
\vs p003 2:5 Всемогущество Отца относится к повсеместному господству абсолютного уровня, на котором в непосредственной близости к нему, Источнику всех вещей, три энергии --- материальная, умственная и духовная --- неразличимы. Разум творения, не будучи ни Райской монотой, ни Райским духом, непосредственно на Отца Всего Сущего не реагирует. Бог \bibemph{настраивает себя} на несовершенный разум --- смертных Урантии --- через Настройщиков Мысли.
\vs p003 2:6 \P\ Отец Всего Сущего --- не преходящая сила, переменчивая мощь или колеблющаяся энергия. Силы и мудрости Отца вполне достаточно, чтобы справиться с любой и всеми потребностями вселенной. По мере возникновения непредвиденностей человеческого опыта он их все предвидит и поэтому реагирует на дела вселенной не отстраненно, но согласно велениям вечной мудрости и в соответствии с указами бесконечного суждения. Несмотря на видимость, могущество Бога отнюдь не действует во вселенной как слепая сила.
\vs p003 2:7 Возникают ситуации, в которых кажется, что постановления крайней необходимости созданы, что естественные законы приостановлены, что неверные приспособления признаны и что сделано усилие для исправления возникшего положения; но это не так. Подобные представления о Боге происходят от ограниченной широты ваших взглядов, от конечности вашего понимания и ограниченного диапазона вашего видения: такое непонимание Бога вызвано вашим глубоким неведением относительно существования высших законов сферы, величия характера Отца, бесконечности его атрибутов и факта его свободной воли.
\vs p003 2:8 Планетарные создания, в которых пребывает дух Бога, разбросанные в различных направлениях по всем вселенным, существующим в пространстве, по числу и чину почти настолько бесконечны, их интеллекты настолько различны, их умы настолько ограничены, а порой и столь грубы, их видение настолько сужено и локализовано, что почти невозможно сформулировать обобщения закона, в достаточной степени отображающие бесконечные атрибуты Отца и в то же время в какой\hyp{}либо степени этим разумным сотворенным существам понятные. Поэтому для тебя, создания, многие из деяний всесильного Творца кажутся произвольными, отвлеченными, а нередко и бессердечными, и жестокими. Однако я вновь уверяю тебя в том, что это неверно. Все дела Бога целенаправленны, разумны, мудры, добры и вечно устремлены к высшему благу не всегда отдельно взятого существа, отдельно взятой расы, отдельно взятой планеты или даже отдельно взятой вселенной, но всегда к благополучию и высшему благу всех, кого это касается, от низших до высших. Во временных эпохах благополучие части иногда может казаться отличным от благополучия целого; в круге же вечности подобных кажущихся отличий не существует.
\vs p003 2:9 Все мы --- часть семьи Бога, а потому иногда должны разделять порядок всей семьи. Многие из деяний Бога, которые нас так беспокоят и смущают, являются следствием решений и окончательных постановлений премудрости, уполномочивающей Носителя Объединенных Действий исполнять решения непогрешимой воли бесконечного разума, проводить в жизнь решения совершенной личности, чья точка зрения, видение и внимание объемлют собой высшее и вечное благополучие всего его необъятного и обширного творения.
\vs p003 2:10 Таким образом, ваша отстраненная, местная, конечная, грубая и в значительной степени материалистическая точка зрения, а также ограничения, свойственные природе вашего существа, создают такое препятствие, что вы не способны видеть, понимать или познавать мудрость и доброту многих из божественных деяний, которые вам кажутся исполненными такой сокрушительной жестокости и которые кажутся отмеченными таким крайним безразличием к покою и благополучию, к планетарному счастью и личному процветанию ваших собратьев\hyp{}созданий. Это из\hyp{}за пределов человеческого видения, из\hyp{}за вашего ограниченного понимания и конечного разумения вы неверно понимаете побуждения и извращаете цели Бога. Однако в эволюционных мирах происходит много такого, что личными делами Отца Всего Сущего не является.
\vs p003 2:11 \P\ Божественное всемогущество в совершенстве согласованно с другими атрибутами личности Бога. Могущество Бога в своем вселенском духовном проявлении, как правило, ограничено только тремя условиями или ситуациями:
\vs p003 2:12 \ublistelem{1.}\bibnobreakspace Природой Бога, особенно его бесконечной любовью, истиной, красотой и добродетелью.
\vs p003 2:13 \ublistelem{2.}\bibnobreakspace Волей Бога, его милосердным служением и отеческими отношениями с личностями вселенной.
\vs p003 2:14 \ublistelem{3.}\bibnobreakspace Законом Бога, праведностью и правосудием вечной Райской Троицы.
\vs p003 2:15 \P\ 2Бог неограничен в могуществе, божественен в природе, окончателен в воле, бесконечен в атрибутах, вечен в мудрости и абсолютен в реальности. Однако все эти особенности Отца Всего Сущего объединены в Божестве и всесторонне выражены в Райской Троице и в божественных Сыновьях Троицы. В иных же отношениях вне Рая и центральной вселенной Хавоны все, относящееся к Богу, ограничено эволюционирующим присутствием Верховного, обусловлено выявляющимся присутствием Предельного и согласовано тремя экзистенциальными Абсолютами --- Божественным, Вселенским и Неограниченным. Причем присутствие Бога ограничено таким образом потому, что такова воля Бога.
\usection{3.\bibnobreakspace Всезнание Бога}
\vs p003 3:1 <<Бог знает все>>. Божественный разум сознает и понимает мысль всего творения. Его знание событий всеобъемлюще и совершенно. Божественные сущности, от него исходящие, являются частью его самого; <<уравновешивающий облака>> <<совершен и в знании>>. <<На всяком месте очи Господни>>. О ничего не значащем воробье ваш великий учитель сказал: <<Ни один из них не упадет на землю без знания Отца моего>>, а также: <<У вас же и волосы на голове все сочтены>>. <<Он исчисляет количество звезд; всех их называет именами их>>.
\vs p003 3:2 Отец Всего Сущего --- единственная во всей вселенной личность, которая действительно знает число звезд и планет, существующих в пространстве. Все миры каждой вселенной --- постоянно в сознании Бога. Он также говорит: <<Я увидел страдание народа моего, услышал вопль его и знаю скорби его>>. Ибо <<с небес смотрит Господь; видит всех сынов человеческих; с места, где обитает, смотрит на всех живущих на земле>>. Каждое сотворенное дитя истинно может сказать: <<Он знает путь мой, и когда испытает меня, выйду, как золото>>. <<Бог знает, когда мы садимся и когда встаем; разумеет помышления наши издали, и все пути наши известны ему>>. <<Все обнажено и открыто пред глазами того, кому мы должны давать отчет>>. И для каждого человека настоящим утешением должно стать понимание того, что <<он знает состав твой и помнит, что ты прах>>. Говоря о Боге живом, Иисус сказал: <<Отец ваш знает, в чем вы имеете нужду, прежде вашего прошения у него>>.
\vs p003 3:3 Бог обладает неограниченной способностью к познанию всех вещей; его сознание всеобъемлюще. Его личный контур заключает в себе все личности, и его знание даже низших творений пополняется --- косвенно благодаря нисходящей череде божественных Сыновей и непосредственно --- благодаря пребывающим в творениях Настройщикам Мысли. Более того, Бесконечный Дух вездесущ всегда.
\vs p003 3:4 Мы не до конца уверены, таково ли решение Бога, чтобы заранее знать, греховные поступки или нет. Однако даже если Бог желает предвидеть добровольные поступки (акты?) своих детей, то такое предвидение ничуть не лишает их свободы. В одном нет сомнения: Бог никогда не бывает подвержен удивлению.
\vs p003 3:5 \P\ Всемогущество не предполагает способность делать то, чего делать невозможно, совершать неподобающие Богу действия. Так же и всеведение не предполагает познание непознаваемого. Однако подобные утверждения едва ли можно сделать понятными конечному разуму. Творение едва ли может понять диапазон и ограничения воли Творца.
\usection{4.\bibnobreakspace Безграничность Бога}
\vs p003 4:1 Последовательные дарования самого себя вселенным по мере обретения ими бытия никоим образом не умаляют потенциал мощи или запас мудрости, поскольку они продолжают пребывать и покоиться в центральной личности Божества. В потенциале силы, мудрости и любви Отец никогда ничего не умаляет в своем владении и не лишается какого бы то ни было атрибута своей славной личности вследствие безграничного дарования себя Райским Сыновьям, подчиненным ему творениям и многочисленным сотворенным существам, в этих творениях живущим.
\vs p003 4:2 Сотворение каждой новой вселенной требует новой настройки тяготения; однако даже если бы сотворение продолжалось неопределенно долго, вечно и даже до бесконечности, так что в итоге материальное творение существовало бы безгранично, то и тогда способность управлять и согласовывать, покоящаяся на Райском Острове, была бы равной и достаточной для господства над такой бесконечной вселенной, для управления ею и ее согласования. Причем после этого дарования неограниченной силы и мощи безграничной вселенной Бесконечный все равно был бы в той же степени полон сил и энергии; Неограниченный Абсолют все равно не уменьшился бы; Бог все равно обладал бы тем же самым бесконечным потенциалом, как будто сила, энергия и мощь никогда не изливались для дарования их вселенной за вселенной.
\vs p003 4:3 То же самое касается и мудрости: тот факт, что разум столь щедро раздается для мышления миров, никоим образом не обедняет центральный источник божественной мудрости. По мере того, как вселенные умножаются, а жители миров возрастают числом до пределов постижимого, если бы разум и продолжал без конца дароваться этим существам, как высокого, так и низкого положения, то центральная личность Бога все равно продолжала бы заключать в себе тот же самый вечный, бесконечный и премудрый разум.
\vs p003 4:4 Тот факт, что он от себя посылает духовных вестников, чтобы те пребывали в мужчинах и женщинах вашего и иных миров, никоим образом не умаляет его способности действовать в качестве божественной и всемогущей духовной личности; причем абсолютно нет предела ни степени, ни числу таких духовных Наблюдателей, которых он способен и может послать от себя. Это дарование себя своим творениям создает для этих божественно одаренных смертных безграничную, почти непостижимую возможность все более совершенных и сменяющих друг друга существований. Причем это чудесное наделение самим собой в виде этих служебных духовных сущностей никоим образом не умаляет мудрость и совершенство истины и знания, которые покоятся в личности премудрого, всезнающего и всемогущего Отца.
\vs p003 4:5 \P\ Для смертных, живущих во времени, существует будущее; Бог же живет в вечности. Хоть я и происхожу почти что из места пребывания Божества, я, тем не менее, не смею говорить с совершенством понимания о бесконечности многих из божественных атрибутов. Один только бесконечный разум может полностью осознать бесконечность бытия и вечность действия.
\vs p003 4:6 \P\ Смертный человек не может познать бесконечность небесного Отца. Конечный разум не может осмыслить такую абсолютную истину или такой факт. Однако этот же самый конечный человек действительно может \bibemph{чувствовать ---} буквально переживать --- полное и неуменьшенное воздействие ЛЮБВИ такого бесконечного Отца. Такую любовь можно истинно испытать, несмотря на то, что, хотя качество опыта неограничено, количество такого опыта строго ограничено человеческой способностью к духовному восприятию и связанной с ним способностью любить Бога в ответ.
\vs p003 4:7 Конечная оценка бесконечных качеств намного превосходит логически ограниченные способности творения, потому что смертный человек создан по образу Бога --- в нем живет частица бесконечности. Поэтому кратчайший и наилучший подход человека к Богу --- это подход через любовь и посредством любви, ибо Бог есть любовь. Причем вся совокупность подобных уникальных отношений является действительным опытом в области космической социологии, отношениями между Творцом и творением --- любовью между Отцом и ребенком.
\usection{5.\bibnobreakspace Верховное правление Отца}
\vs p003 5:1 В своем контакте с пост\hyp{}хавонными творениями Отец Всего Сущего проявляет свое бесконечное могущество и окончательную власть не путем прямой передачи, а через своих Сыновей и подчиненные им личности. Причем все это Бог делает по своей собственной воле. Любое и всякое могущество при возникновении определенной ситуации, если таково будет решение божественного разума, может проявляться и непосредственно; однако, как правило, подобное действие имеет место только в результате неспособности уполномоченной на то личности оправдать божественное доверие. В такие времена, перед лицом такого срыва и в пределах области божественного могущества и потенциала Отец действует независимо и согласно собственному выбору; причем такой выбор --- всегда выбор непогрешимо совершенный и бесконечно мудрый.
\vs p003 5:2 Отец правит через своих Сыновей; от него через всю организацию вселенной тянется неразрывная череда правителей, заканчивающаяся Планетарными Принцами, которые руководят судьбами эволюционных сфер необъятных владений Отца. Слова: <<Господни --- земля и что наполняет ее>>; <<он низлагает царей и возвышает царей>>; <<Всевышние правят царствами людей>> --- не просто поэтические выражения.
\vs p003 5:3 В делах сердец человеческих воля Отца Всего Сущего не всегда превалирует; однако в поведении и в судьбе планеты преобладает божественный план; побеждает великий замысел, исполненный мудрости и любви.
\vs p003 5:4 Иисус сказал: <<Отец мой, который дал мне их, величественнее всех, и никто не может похитить их из руки Отца моего>>. Окидывая взором многочисленные произведения и глядя на ошеломительную огромность почти беспредельного творения Божиего, вы можете поколебаться в вашем представлении о его первенстве, однако вы не должны ставить под сомнение признание его надежно и вечно воцарившимся в Райском средоточии всех вещей, а также милосердным Отцом всех разумных существ. Существует только <<один Бог и Отец всех, который выше всего и во всем>>, <<ибо он есть прежде всего, и в нем заключается все>>.
\vs p003 5:5 \P\ Неопределенности жизни и превратности бытия никоим образом не противоречат понятию о вселенском владычестве Бога. Вся жизнь эволюционирующего творения окружена определенными \bibemph{неизбежностями.} Рассмотрим следующие:
\vs p003 5:6 \ublistelem{1.}\bibnobreakspace Желательна ли \bibemph{смелость ---} сила характера? Значит, человек должен выживать в окружении, которое обуславливает необходимость борьбы с трудностями и ответной реакции на разочарования.
\vs p003 5:7 \ublistelem{2.}\bibnobreakspace Желателен ли \bibemph{альтруизм ---} служение своим собратьям? Значит, жизненный опыт должен предусматривать столкновения с ситуациями общественного неравенства.
\vs p003 5:8 \ublistelem{3.}\bibnobreakspace Желательна ли \bibemph{надежда ---} величие доверия? Значит, человеческий опыт должен постоянно противостоять ненадежности и повторяющимся неопределенностям.
\vs p003 5:9 \ublistelem{4.}\bibnobreakspace Желательна ли \bibemph{вера ---} верховное утверждение человеческой мысли? Значит, разум человека должен попадать в то затруднительное положение, в котором он всегда знает меньше, чем может верить.
\vs p003 5:10 \ublistelem{5.}\bibnobreakspace Желательны ли \bibemph{любовь к истине} и готовность идти, куда бы та ни вела? Значит, человек должен вырастать в мире, где присутствует ошибка и всегда возможна ложь.
\vs p003 5:11 \ublistelem{6.}\bibnobreakspace Желателен ли \bibemph{идеализм ---} наиболее приближающее к истине представление о божественном? Значит, человек должен бороться в среде относительной добродетели и красоты, в окружении, стимулирующем неукротимое стремление к лучшему.
\vs p003 5:12 \ublistelem{7.}\bibnobreakspace Желательна ли \bibemph{верность ---} преданность высшему долгу? Значит, человек должен продолжать действовать в среде, где возможны предательство и измена. Мужество, исполненное верности долгу, заключается в предполагаемой опасности срыва.
\vs p003 5:13 \ublistelem{8.}\bibnobreakspace Желательно ли \bibemph{бескорыстие ---} дух самозабвения? Значит, смертный человек должен жить лицом к лицу с непрекращающимися шумными требованиями признания и почестей для собственного <<я>>, от которого не уйти. Человек не мог бы деятельно выбирать божественную жизнь, если бы не было жизни для самого себя, от которой следует отказаться. Человек никогда не смог бы достичь спасительного обладания праведностью, если бы не было потенциального зла, чтобы по контрасту с ним возвышать и отличать добро.
\vs p003 5:14 \ublistelem{9.}\bibnobreakspace Желательно ли \bibemph{удовольствие ---} удовлетворение, которое дает счастье? Значит, человек должен жить в мире, где альтернатива боли и вероятность страдания --- постоянно присутствующие возможности опыта.
\vs p003 5:15 \P\ Во всей вселенной каждая единица считается частью целого. Продолжение существования части зависит от сотрудничества с планом и замыслом целого, от искреннего желания и совершенной готовности исполнять божественную волю Отца. Способный эволюционировать мир без ошибок (возможности неразумного суждения) мог бы быть только миром без \bibemph{свободного} разума. Во вселенной Хавона существует миллиард совершенных миров с совершенными обитателями, однако развивающемуся человеку должно быть свойственно ошибаться, если ему суждено быть свободным. Свободный и неопытный разум не может быть однородно мудрым сразу. Возможность ошибочного суждения (зла) становится грехом лишь тогда, когда человеческая воля намеренно подтверждает и сознательно принимает умышленно аморальное суждение.
\vs p003 5:16 \P\ Полное признание истины, красоты и добродетели --- вот характерные особенности совершенства божественной вселенной. Жители миров Хавоны не нуждаются в потенциале уровней относительной ценности в качестве стимула для принятия решения; такие совершенные существа способны определять и выбирать добро и в отсутствии каких бы то ни было противоречивых и заставляющих мыслить моральных ситуаций. Однако все подобные совершенные существа по своей моральной природе и духовному статусу являются тем, чем они являются, в силу факта своего существования. Своим опытом они заслужили развитие только в пределах присущего им статуса. Смертный же человек даже свой статус кандидата на восхождение и тот добывает своими собственными верой и надеждой. Все божественное, понимаемое человеческим разумом и приобретаемое человеческой душой, является достижением опыта; это --- \bibemph{реальность} личного опыта и поэтому, в отличие от врожденной доброты и праведности непогрешимых личностей Хавоны, --- владение уникальное.
\vs p003 5:17 \P\ Создания Хавоны по своей природе мужественны, но по\hyp{}человечески не храбры. По своей природе они добры и внимательны, но по\hyp{}человечески едва ли альтруистичны. Они ожидают приятного будущего, но не надеются так, как возвышенно надеются доверяющие смертные изменчивых эволюционирующих миров. Они верят в стабильность вселенной, однако им в высшей степени чужда спасительная вера, благодаря которой смертный человек восходит от статуса животных до врат Рая. Они любят истину, но ничего не знают о ее душеспасительных качествах. Они --- идеалисты, но идеалистами они родились; они пребывают в полном неведении относительно экстаза, который наступает, когда таким становишься путем радостного выбора. Они верны, но никогда не испытывали трепета искренней и разумной преданности долгу перед лицом искушения свой долг не исполнять. Они бескорыстны, но так и не достигли подобных уровней опыта путем величественного покорения воинствующей самости. Они наслаждаются удовольствием, но не знают сладости исполненного удовольствием спасения от потенциала боли.
\usection{6.\bibnobreakspace Первенство Отца}
\vs p003 6:1 Отец Всего Сущего отдает власть и делегирует силу с божественным бескорыстием и совершенной щедростью, но все равно остается первым; его рука на мощном рычаге обстоятельств вселенских сфер; он сохраняет за собой все окончательные решения и непогрешимо с неоспоримой властью над благополучием и судьбой протяженного, вращающегося и постоянно обращающегося творения держит всемогущий скипетр вето своего вечного замысла.
\vs p003 6:2 Владычество Бога неограничено; оно --- основополагающий факт всего творения. Вселенная отнюдь не была неизбежной. Она --- не случайность и не существует сама по себе. Вселенная --- это плод творчества, а потому полностью подчинена воле Творца. Воля Бога --- это божественная истина, живая любовь; поэтому для совершенствующихся творений эволюционирующих вселенных характерна добродетель --- близость к божественности и потенциальное зло --- от божественности отдаленность.
\vs p003 6:3 \P\ Всякая религиозная философия рано или поздно приходит к идее единого правления во вселенной, единого Бога. Вселенские причины не могут быть ниже вселенских последствий. Источник потоков вселенской жизни и космического разума должен быть выше уровней их проявления. Человеческий разум в терминах более низких чинов бытия последовательно объяснить нельзя. Разум человека может быть истинно понят лишь благодаря признанию реальности более высоких порядков мысли и целенаправленной воли. Человек как существо моральное необъясним до тех пор, пока не признана реальность Отца Всего Сущего.
\vs p003 6:4 Философ\hyp{}механист утверждает, что отвергает идею о всеобъемлющей и суверенной воле, той самой независимой воле, чью деятельность в выработке вселенских законов он столь глубоко почитает. С каким же неосознанным почтением относится механист к Творцу закона, полагая, что такие законы действуют сами по себе и сами себя объясняют!
\vs p003 6:5 Очеловечивать Бога --- значит совершать великую ошибку (за исключением понятия о пребывающем в человеке Настройщике Мысли), но даже это не так глупо, как попытка полностью \bibemph{механизировать} идею о Первоисточнике и Центре.
\vs p003 6:6 \P\ Страдает ли Райский Отец? Я не знаю. Сыновья\hyp{}Творцы, несомненно, могут страдать и иногда страдают так же, как смертные. Вечный же Сын и Бесконечный Дух страдают в модифицированном смысле. Я думаю, что Отец Всего Сущего страдает, но не могу понять, \bibemph{как;} быть может через посредство контура личности или индивидуальность Настройщиков Мысли и другие дары его вечной природы? Он сказал о смертных: <<Во всех скорбях ваших я скорблю>>. Он несомненно преисполнен отеческого и сочувственного понимания; он истинно может страдать, но природа его страдания мне не понятна.
\vs p003 6:7 \P\ Бесконечный и вечный Правитель вселенной вселенных --- это мощь, форма, энергия, процесс, паттерн, принцип, присутствие и идеализированная реальность. Однако он больше этого; он --- личностный; он проявляет независимую волю, преисполнен сознания собственной божественности, исполняет указы творческого разума, стремится к удовлетворению, которое приносит осуществление вечного замысла, и являет любовь и привязанность Отца к своим детям во вселенной. Причем все эти наиболее личные черты Отца можно лучше понять, наблюдая их такими, какими открылись они в жизни пришествия Михаила, вашего Сына\hyp{}Творца, когда он жил во плоти на Урантии.
\vs p003 6:8 \P\ Бог Отец любит людей; Бог Сын людям служит; Бог Дух вдохновляет детей вселенной на странствие бесконечного восхождения в отыскании Бога Отца путями, предопределенными Богом Сыновьями через служение благодати Бога Духа.
\vs p003 6:9 [Будучи Божественным Советником, получившим поручение представлять откровение Отца Всего Сущего, я продолжил свое дело этим заявлением об атрибутах Божества.]
