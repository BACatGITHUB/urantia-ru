\upaper{162}{На празднике кущей}
\author{Комиссия срединников}
\vs p162 0:1 Когда Иисус с десятью апостолами отправился в Иерусалим, он решил идти через Самарию, поскольку это был наиболее короткий путь. Поэтому они прошли мимо восточного берега озера и через город Скифополь вошли в пределы Самарии. Когда наступили сумерки, Иисус послал Филиппа и Матфея в селение, расположенное на восточном склоне горы Гелвуй, чтобы обеспечить для всех ночлег. Оказалось, что жители этого селения крайне отрицательно относились к евреям, даже больше, чем средние самаритяне, а в тот момент эти настроения особенно усилились, поскольку очень многие евреи направлялись на праздник кущей. Эти люди практически ничего не знали об Иисусе, и они отказали ему в ночлеге, поскольку он и его сподвижники были евреями. Когда Матфей и Филипп выразили негодование и сказали самаритянам, что они отказываются принять израильского святого, разъяренные жители этого маленького городка прогнали их палками и камнями.
\vs p162 0:2 После того, как Филипп и Матфей вернулись к своим товарищам и сообщили, как их выгнали из селения, Иаков и Иоанн подошли к Иисусу и сказали: «Учитель, мы просим тебя позволить нам призвать огонь с неба, чтобы уничтожить этих дерзких и нераскаявшихся самаритян.» Но когда Иисус услышал эти слова о мщении, он обрушился на сыновей Зеведеевых и сурово отчитал их: «Вы даже не знаете, что за способ отношений вы демонстрируете. Месть несовместима с принципами царства небесного. Вместо того, чтобы спорить, давайте отправимся в маленькое селение у Иорданского брода». Так, из\hyp{}за сектанских предрассудков эти самаритяне лишили себя чести оказать гостеприимство Сыну\hyp{}Творцу вселенной.
\vs p162 0:3 Иисус с десятью спутниками остановились в деревне возле брода через Иордан. На следующий день спозаранку они переправились через реку и, продолжив путь в Иерусалим по восточно\hyp{}иорданской дороге, поздним вечером в среду пришли в Вифанию. Фома и Нафанаил подошли туда в пятницу, поскольку их задержали беседы с Роданом.
\vs p162 0:4 \P\ Иисус и двенадцать апостолов пробыли в окрестностях Иерусалима до конца следующего месяца (октября), примерно четыре с половиной недели. Иисус ходил в город только несколько раз, и эти короткие посещения произошли в дни праздника кущей. Большую часть октября он провел с Авениром и его сподвижниками в Вифлееме.
\usection{1. Опасность посещения Иерусалима}
\vs p162 1:1 Еще задолго до бегства из Галилеи последователи Иисуса убеждали его пойти в Иерусалим возвещать евангелие царства, чтобы его весть могла стать влиятельной после того, как будет возвещена в центре еврейской культуры и учености; но теперь, когда он наконец пришел в Иерусалим учить, они стали бояться за его жизнь. Зная, что Синедрион пытался доставить Иисуса в Иерусалим на суд, и, помня повторяемые Учителем в последнее время слова о том, что его постигнет смерть, апостолы были буквально потрясены его внезапным решением посетить праздник кущей. На все их предыдущие просьбы отправиться в Иерусалим, он отвечал: «Час еще не настал». Теперь же на все их вызванные страхом протесты он лишь отвечал: «Но час настал».
\vs p162 1:2 В продолжение праздника кущей Иисус несколько раз смело посещал Иерусалим и публично учил в храме. Он делал это, несмотря на все попытки апостолов отговорить его. Хотя они давно убеждали его провозгласить свою весть в Иерусалиме, но теперь, когда он входил в город, они уже боялись за него, прекрасно зная, что книжники и фарисеи намеревались предать его смерти.
\vs p162 1:3 Смелые посещения Иисусом Иерусалима более чем когда\hyp{}либо приводили в замешательство его последователей. Многие его ученики, даже апостол Иуда Искариот, осмеливались думать, что Иисус поспешно бежал в Финикию потому, что боялся еврейских правителей и Ирода Антипу. Они не могли понять значения поступков Учителя. Уже одного того, что он вопреки советам своих последователей, присутствовал в Иерусалиме на празднике кущей, оказалось достаточным, чтобы навсегда прекратить все пересуды относительно его страха и трусости.
\vs p162 1:4 В дни праздника кущей тысячи верующих со всех концов Римской империи видели Иисуса, слышали его учения, и многие даже побывали в Вифании только затем, чтобы посоветоваться с ним о распространении царства в тех местах, где они жили.
\vs p162 1:5 Было много причин, благодаря которым Иисус смог публично проповедовать на территории храма в дни праздника, главной же из них была неуверенность у членов Синедриона, возникшая в результате внутренних разногласий в их собственных рядах. Фактически многие из членов Синедриона или тайно верили в Иисуса, или же явно были не склонны арестовывать его во время праздника, когда в Иерусалиме присутствовало столько людей, многие из которых или верили в него, или, по крайней мере, были дружелюбно настроены к возглавляемому им духовному движению.
\vs p162 1:6 Усилия Авенира и его сподвижников на территории Иудеи тоже значительно укрепили благоприятное отношение к царству, причем настолько, что враги Иисуса даже не осмеливались слишком откровенно выражать свою неприязнь. Это было одной из причин, почему Иисус смог открыто посетить Иерусалим и уйти целым и невредимым. Случись это на месяц или два раньше, он наверняка был бы казнен.
\vs p162 1:7 Но беззаветная смелость Иисуса, открыто появившегося в Иерусалиме, внушила его врагам благоговейный страх; они не были готовы к такому смелому вызову. Несколько раз в течение этого месяца Синедрион предпринимал слабые попытки взять Учителя под арест, но они были безрезультатными. Враги Иисуса были настолько ошеломлены его неожиданным открытым появлением в Иерусалиме, что стали гадать, не обещали ли ему римские власти свою защиту. Зная, что Филипп (брат Ирода Антипы) был едва ли не последователем Иисуса, члены Синедриона предположили, что Филипп предоставил Иисусу гарантии защиты от его врагов. Иисус покинул подвластную им территорию прежде, чем они осознали, что ошибались, полагая, будто его внезапное и смелое появление в Иерусалиме было обусловлено секретным соглашением с римскими властями.
\vs p162 1:8 Когда они покидали Магадан, только двенадцать апостолов знали, что Иисус намеревался посетить праздник кущей. Другие последователи Учителя были чрезвычайно изумлены, когда он появился на территории храма и стал открыто учить, и еще больше были удивлены еврейские власти, когда стало известно, что он учит в храме.
\vs p162 1:9 Хотя ученики и не ожидали, что Иисус будет присутствовать на празднике, подавляющее большинство паломников, пришедших издалека и слышавших о нем, питали надежду, что смогут увидеть его в Иерусалиме. И они не были разочарованы, поскольку Иисус несколько раз учил на крыльце Соломона и в других местах на территории храма. Эти учения фактически были официальным, или формальным возвещением еврейскому народу и всему миру божественности Иисуса.
\vs p162 1:10 Толпы, слушавшие учения Учителя, разделились во мнениях. Одни говорили, что он --- хороший человек; другие --- что он пророк; третьи --- что он воистину Мессия; говорили также, что он вредный, сеющий смуту человек, что своими странными учениями он сбивает людей с толку. Его враги не решались открыто осуждать его из\hyp{}за страха перед расположенными к нему и верующими в него людьми, а его друзья боялись открыто признать его из\hyp{}за страха перед еврейскими властями, зная, что Синедрион был полон решимости предать его смерти. Но даже его враги удивлялись его наставлениям, зная, что он никогда не учился в школах раввинов.
\vs p162 1:11 Каждый раз, когда Иисус шел в Иерусалим, апостолов охватывал ужас. Но еще больше они боялись оттого, что слышали с каждым днем все более смелые заявления о характере его миссии на земле. Они не привыкли слышать от Иисуса таких конкретных заявлений и таких поразительных утверждений, даже когда он проповедовал среди друзей.
\usection{2. Первая беседа в храме}
\vs p162 2:1 В первый день, когда Иисус учил в храме, а множество людей сидели и слушали его слова о свободе нового евангелия и радости тех, кто верит в благую весть, один любознательный слушатель неожиданно прервал его и спросил: «Учитель, как ты можешь цитировать Писание и так складно учить людей, если мне говорили, что ты не обучался учености раввинов?» Иисус ответил: «Ни один человек не учил меня истинам, которые я возвещаю вам. И это учение не мое, но Того, кто послал меня. Любой человек, если он действительно желает исполнять волю моего Отца, безусловно поймет мое учение --- от Бога оно, или же я говорю от себя самого. Тот, кто говорит от себя самого, добивается собственной славы, но когда я возвещаю слова Отца, я, тем самым, добиваюсь славы для того, кто послал меня. Но прежде, чем попытаться вступить на новый путь, не надлежит ли вам следовать тем путем, которым уже идете? Моисей дал вам закон, но кто из вас честно стремится выполнять его? В этом законе Моисей предписывает вам: «Не убий»; несмотря на эту заповедь, некоторые из вас стремятся убить Сына Человеческого».
\vs p162 2:2 \P\ Когда толпа услышала эти слова, люди стали спорить между собой. Одни говорили, что он сумасшедший; другие --- что в нем сидит дьявол. Говорили также, что это воистину пророк из Галилеи, которого книжники и фарисеи давно намеревались убить. Некоторые говорили, что религиозные власти боялись доставлять ему неприятности; другие --- что они не поднимали на него руку потому, что уверовали в него. После продолжительного спора один человек выступил из толпы и спросил Иисуса: «Почему правители стремятся убить тебя?» И он ответил: «Правители стремятся убить меня потому, что они возмущены моим учением о благой вести царства, евангелии, которое освобождает людей от обременительных традиций формальной обрядовой религии, которую эти учителя намерены защищать любой ценой. В субботу они совершают обрезание в соответствии с законом, но они убили бы меня потому, что однажды в субботу я исцелил человека, томящегося под гнетом недугов. В субботу они следуют за мной, шпионят за мной, но убили бы меня потому, что как\hyp{}то раз в субботу я решил сделать тяжело больного человека совершенно здоровым. Они пытаются убить меня, потому что хорошо знают, что, если вы искренне поверите и решитесь принять мое учение, их система традиционной религии будет низвергнута, навсегда уничтожена. Таким образом, они будут лишены власти над тем, чему они посвятили свою жизнь, поскольку они упорно отказываются принять это новое и более прекрасное евангелие царства Божьего. И сейчас я взываю к каждому из вас: не судите по внешним впечатлениям, но судите по истинному духу этих учений; судите справедливо».
\vs p162 2:3 Затем другой человек спросил: «Да, Учитель, мы ищем Мессию, но когда он придет, мы знаем, что его явление будет окутано тайной. Мы знаем, откуда ты. Ты был среди своих братьев с самого начала. Спаситель явится в могуществе, чтобы восстановить трон царства Давида. Действительно ли ты утверждаешь, что ты Мессия?» И Иисус ответил: «Вы утверждаете, что знаете меня и знаете, откуда я. Я желал бы, чтобы эти ваши утверждения соответствовали действительности, ибо воистину тогда вы нашли бы в этом знании жизнь изобильную. Но я заявляю, что пришел к вам не сам; я послан Отцом, и тот, кто послал меня, истинен и верен. Отказываясь слушать меня, вы отказываетесь принять Того, кто послал меня. Если вы примете это евангелие, то узнаете пославшего меня. Я знаю Отца, ибо я пришел от Отца, чтобы возвестить вам о нем и открыть его вам».
\vs p162 2:4 Агенты книжников хотели схватить его, но побоялись толпы, поскольку многие верили в него. Со времени крещения Иисуса его деятельность стала хорошо известна всему еврейству, и когда многие из этих людей рассказывали о ней, они говорили между собой: «Хотя этот учитель --- из Галилеи и хотя он не соответствует всем нашим представлениям о Мессии, мы сомневаемся, что спаситель, когда он явится, свершит что\hyp{}то более удивительное, чем то, что уже свершил этот Иисус из Назарета».
\vs p162 2:5 Когда Фарисеи и их агенты услышали, что люди так говорят, они посоветовались со своими руководителями и решили, что немедленно следует что\hyp{}то предпринять, чтобы положить конец выступлениям Иисуса перед народом на территории храма. Предводители евреев, в общем, были настроены избегать столкновения с Иисусом, полагая, что римские власти обещали ему неприкосновенность. Они не могли иначе объяснить ту смелость, с которой он на этот раз пришел в Иерусалим; но члены Синедриона не вполне верили этой молве. Они рассуждали, что римские правители не сделали бы этого тайно, не известив --- высшее руководство еврейской нации.
\vs p162 2:6 Поэтому Эвер, один из членов Синедриона, с двумя помощниками был послан арестовать Иисуса. Когда Эвер пробирался к Иисусу, Учитель сказал: «Не бойся приблизиться ко мне. Подходи ближе и слушай мое учение. Я знаю, что ты послан схватить меня, но тебе следует понять, что ничто не случится с Сыном Человеческим, пока не придет его час. Ты выступаешь против меня не по своей воле, ты идешь лишь выполнять приказ своих хозяев, и даже эти правители евреев на самом деле считают, что служат Богу, когда тайно стремятся уничтожить меня.
\vs p162 2:7 Я не питаю ни к кому из вас неприязни. Отец любит вас, и поэтому я желаю, чтобы вы освободились от гнета предрассудков и мрака традиции. Я предлагаю вам свободу жизни и радость спасения. Я провозглашаю новый и живой путь, освобождение от зла и низвержение ига греха. Я пришел, чтобы вы могли жить, и жить вечно. Вы стремитесь избавиться от меня и моего нарушающего покой учения. Если бы вы только могли понять, что мне предстоит пробыть с вами совсем недолго! Очень скоро я отправлюсь к Тому, кто послал меня в этот мир. И тогда многие из вас будут упорно искать меня, но вы не найдете меня, ибо вы не можете прийти туда, куда я вот\hyp{}вот отправлюсь. Но все, кто истинно стремятся найти меня, когда\hyp{}нибудь обретут жизнь, ведущую к Отцу моему».
\vs p162 2:8 Некоторые насмешники говорили друг другу: «Куда же этот человек отправится, что мы не сможем найти его? Он отправится жить среди греков? Он уничтожит себя? Что он имеет в виду, когда заявляет, что скоро покинет нас и мы не сможем отправиться туда, куда отправится он?»
\vs p162 2:9 Эвер и его помощники отказались арестовать Иисуса; они вернулись на назначенное место встречи без него. Когда главные священники и фарисеи стали укорять Эвера и его помощников за то, что те не привели с собой Иисуса, Эвер ответил только: «Мы побоялись арестовать его в толпе, потому что многие верят в него. Кроме того, мы никогда не слышали, чтобы кто\hyp{}либо говорил так, как этот человек. В этом учителе есть что\hyp{}то необыкновенное. Всем вам стоило бы пойти послушать его». И когда верховные правители услышали эти слова, они были изумлены и насмешливо сказали Эверу: «Тебя тоже сбили с пути истинного? Ты готов поверить в этого обманщика? Слышал ли ты, чтобы кто\hyp{}нибудь из наших ученых мужей или кто\hyp{}либо из правителей поверил в него? Был ли кто\hyp{}либо из книжников или фарисеев обманут его хитроумными учениями? Как же смогло повлиять на тебя поведение этой невежественной толпы, которая не разбирается в законе и в пророках? Разве ты не понимаешь, что такие непросвещенные люди прокляты». И тогда Эвер ответил: «Пусть так, господа мои, но этот человек говорит народу слова сострадания и надежды. Он ободряет павших духом, и его слова благотворно повлияли даже на наши души. Что может быть плохого в этих учениях, даже если он, скорее всего, и не Мессия, о котором говорится в Писании? И кроме того --- разве наш закон не требует справедливости? Разве мы осуждаем человека прежде, чем выслушать его?» И разгневанный глава Синедриона, обрушился на Эвера со словами: «Не сошел ли ты с ума? Ты случайно не из Галилеи тоже? Посмотри хорошенько в Писании, и ты обнаружишь, что ни один пророк не происходит из Галилеи, а уж тем более Мессия».
\vs p162 2:10 Члены Синедриона разошлись в замешательстве, а Иисус удалился на ночь в Вифанию.
\usection{3. Женщина, изобличенная в прелюбодеянии}
\vs p162 3:1 В это посещение Иерусалима Иисус повстречался с некоей женщиной с дурной репутацией, которую привели к нему ее обвинители и его враги. Имеющееся искаженное изложение этого эпизода создает представление, что эту женщину привели к Иисусу книжники и фарисеи и что Иисус так повел себя с ними, чтобы показать, что этих религиозных лидеров евреев самих можно было обвинить в аморальности. Иисус хорошо знал, что, хотя эти книжники и фарисеи из\hyp{}за своей приверженности традиции были духовно слепы и склонны к догматизму они все\hyp{}таки были одними из самых высоконравственных людей того времени и того поколения.
\vs p162 3:2 На самом деле произошло следующее: рано утром на третий день праздника, когда Иисус приблизился к храму, он был встречен группой нанятых Синедрионом людей, которые тащили за собой женщину. Когда они приблизились, один из них сказал: «Учитель, эта женщина была изобличена в прелюбодеянии --- прямо в момент его совершения. Теперь закон Моисея предписывает нам побить такую женщину камнями. Что ты скажешь, как следует поступить с ней?»
\vs p162 3:3 План врагов Иисуса заключался в том, чтобы, если он поддержит закон Моисея, требующий побить камнями признавшегося нарушителя закона, навлечь на него гнев римских правителей, которые отрицали право евреев выносить смертный приговор без одобрения римского суда. Если бы он запретил побить женщину камнями, они обвинили бы его перед Синедрионом в том, что он ставит себя выше Моисея и закона евреев. Если бы он хранил молчание, они обвинили бы его в трусости. Но Учитель повернул дело так, что вся интрига рухнула под тяжестью своей собственной гнусности.
\vs p162 3:4 Эта женщина, некогда миловидная, была женой человека, принадлежавшего к низшим слоям общества Назарета и всю свою юность сеявшего смуту против Иисуса. Женившись на этой женщине, этот человек позорнейшим образом заставил ее зарабатывать на жизнь, торгуя своим телом. Он пришел на праздник в Иерусалим, чтобы его жена могла, как проститутка, извлечь выгоду из своих физических дарований. Он вступил в сделку с наймитами еврейских правителей, согласившись изобличить свою собственную жену в пороке, который давал им средства существования. Итак, они пришли с этой женщиной и ее партнером по прелюбодеянию, чтобы заманить Иисуса в ловушку, заставив высказать любое суждение, которое могло бы быть использовано против него в случае его ареста.
\vs p162 3:5 Иисус, оглядев толпу, увидел ее мужа, стоящего позади всех. Он знал, что это был за человек, и понял, что он был соучастником этого низкого дела. Сначала Иисус обошел толпу, приблизившись к месту, где стоял опустившийся муж, и написал на песке несколько слов, которые побудили того поспешно удалиться. Затем он вернулся к женщине и снова написал на земле нечто для ее мнимых обвинителей; и, прочитав его слова, они тоже один за другим ушли. А когда Учитель в третий раз написал что\hyp{}то на песке, ушел и партнер женщины по прелюбодеянию, так что когда Учитель кончил писать и выпрямился, то увидел, что женщина стоит перед ним в одиночестве. Иисус сказал: «Женщина, где твои обвинители? Ни один человек не остался, чтобы побить тебя камнями?» И женщина, подняв глаза, ответила: «Ни один человек, Господи». И тогда Иисус сказал: «Я знаю о тебе; и я не осуждаю тебя. Иди с миром». И эта женщина, Хилдана, оставила своего подлого мужа и присоединилась к ученикам царства.
\usection{4. Праздник кущей}
\vs p162 4:1 Присутствие людей со всего мира, от Испании до Индии, на празднике кущей представляло Иисусу идеальную возможность полностью и открыто возвестить свое евангелие в Иерусалиме. Во время этого праздника люди жили, в основном, на открытом воздухе, в шалашах, покрытых листьями. Этот праздник сбора урожая приходился на прохладные осенние месяцы, и на него собиралось больше евреев, чем на Пасху в конце зимы или на Пятидесятницу в начале лета. Апостолы, наконец, увидели, как их Учитель перед всем миром смело заявляет о своей миссии на земле.
\vs p162 4:2 Это был праздник праздников, поскольку на нем могли быть принесены любые жертвы, не принесенные в другие праздники. В это время делались пожертвования храму; здесь удовольствия отдыха сочетались с торжественными обрядами религиозного служения. Это был национальный праздник, сопровождавшийся жертвоприношениями, песнопениями левитов и торжественными серебристыми звуками труб священников. Ночью впечатляющая панорама храма и множества его паломников была ярко освещена огромными канделябрами, ярко горевшими на женском дворе, и десятками факелов, водруженных во дворах храма. Весь город был живописно украшен, только римский замок Антонии составлял резкий контраст этому праздничному и благочестивому зрелищу. И как же ненавидели евреи это постоянно присутствующее напоминание о римском иге!
\vs p162 4:3 Во время праздника были принесены в жертву семьдесят волов как символ семидесяти народов языческого мира. Церемония излияния воды символизировала излияние божественного духа. Эта церемония состоялась во время восхода солнца после шествия священников и левитов. Верующие спускались по ступеням, ведущим от двора Израиля к женскому двору, в то время как вновь и вновь раздавался серебристый звук труб. И затем верующие следовали дальше, в сторону красных ворот, ведущих во двор неевреев. Здесь они обращались лицом к западу, повторяли свои песнопения и после продолжали шествие за символической водой.
\vs p162 4:4 \P\ В последний день праздника четыреста пятьдесят священников вместе с соответствующим числом левитов совершали богослужение. На рассвете со всех концов города собрались паломники, и каждый держал в правой руке пучок миртовых, ивовых и пальмовых веток, а в левой руке ветку райской яблони --- цитрона, или «запретного плода». Для проведения ранней утренней церемонии паломники разделились на три группы. Одна группа осталась в храме, чтобы присутствовать при утренней службе; другая группа отправилась из Иерусалима в окрестности Мазы, чтобы нарезать ивовых ветвей для украшения жертвенного алтаря, а третья группа составила процессию, следующую под серебристые звуки труб из храма за священником с золотым кувшином, в который должна быть налита символическая вода. Эта третья группа шла через Офел к Силоаму, где находились врата источника. После того, как кувшин был наполнен водой из Силоамского водоема, процессия проследовала обратно к храму, вошла в него через водяные ворота и направилась прямо во двор священников, где к священнику, несущему кувшин с водой, присоединился священник, несущий вино для жертвоприношения. Затем оба священника направились к серебряным воронкам, ведущим к подножию алтаря, и вылили в них содержимое кувшинов. Обряд излияния вина и воды послужил сигналом для всех собравшихся паломников начать пение псалмов с 113 до 118 включительно, они чередовали свое пение с пением левитов. И повторяя строки псалмов, они махали пучками веток в сторону алтаря. Затем были совершены жертвоприношения этого дня, сопровождающиеся повторением псалма этого дня, последнего дня праздника, это был восемьдесят второй псалом, начиная с пятого стиха.
\usection{5. Проповедь о свете мира}
\vs p162 5:1 Вечером в предпоследний день праздника в ярком свете канделябров и факелов Иисус встал посреди собравшейся толпы и сказал:
\vs p162 5:2 \P\ «Я --- свет мира. Тот, кто следует за мной, не будет идти во тьме, но будет иметь свет жизни. Осмеливаясь предать меня суду и беря на себя роль моих судей, вы заявляете, что если я свидетельствую сам о себе, мое свидетельство не может быть истинным. Но никогда не может создание вершить суд над Творцом. Даже если я свидетельствую сам о себе, мое свидетельство вовеки истинно, ибо я знаю, откуда я пришел, кто я есть и куда я иду. Вы, которые хотели бы убить Сына Человеческого, не знаете, откуда я пришел, кто я есть и куда я иду. Вы судите только по плотским проявлениям; вы не воспринимаете духовных сущностей. Я не сужу ни одного человека, даже своего злейшего врага. Но если бы я решил судить, мой суд был бы справедливым и праведным, ибо я судил бы не один, а вместе с моим Отцом, который послал меня в этот мир и который есть источник всякого праведного суда. Вы признаете, что свидетельство двух заслуживающих доверия лиц может быть принято --- что ж, тогда я свидетельствую в пользу этих истин; и то же самое делает мой Отец Небесный. И когда я сказал вам это вчера, в своей темноте вы спросили меня: „Где твой Отец?“ Воистину, вы не знаете ни меня, ни моего Отца, ибо если бы вы знали меня, то знали бы также и моего Отца.
\vs p162 5:3 Я уже говорил вам, что ухожу и что вы будете искать меня и не найдете, ибо туда, куда я ухожу, вы не можете прийти. Вы, которые хотели бы отвергнуть этот свет, родились внизу; я же происхожу свыше. Вы, предпочитающие сидеть в темноте, --- из этого мира; я --- не из этого мира и живу в вечном свете Отца всякого света. Все вы многократно имели возможность узнать, кто я, но у вас будут еще и другие свидетельства, подтверждающие личность Сына Человеческого. Я --- свет жизни, и каждый, кто обдуманно и сознательно отвергает этот спасительный свет, умрет во грехах. Я многое могу сказать вам, но вы неспособны воспринять мои слова. Однако тот, кто послал меня, истен и верен, мой Отец любит даже своих заблудших детей. И все, что говорил мой Отец, я тоже возвещаю миру.
\vs p162 5:4 Когда Сын Человеческий вознесется, тогда все вы узнаете, что я --- это он и что я ничего не делал от себя самого, но делал лишь так, как учил меня Отец. Я говорю эти слова вам и вашим детям. И тот, кто послал меня, даже и сейчас со мной; он не оставил меня одного, ибо я всегда делаю то, что угодно ему».
\vs p162 5:5 \P\ Когда Иисус учил таким образом паломников, во дворах храма многие уверовали. И ни один человек не осмелился поднять на него руку.
\usection{6. Беседа о воде жизни}
\vs p162 6:1 В последний день, великий день праздника, когда процессия, идущая от Силоамского водоема, проследовала через дворы храма, и сразу после того, как вода и вино были возлиты священниками на алтарь, Иисус, стоявший среди паломников, сказал: «Если какой\hyp{}либо человек испытывает жажду, пусть он придет ко мне и напьется. От Отца на небесах несу я в этот мир воду жизни. Тот, кто верит мне, наполнится духом, который символизирует эта вода, ибо даже в Писании сказано: „Из него потекут реки живой воды“. Когда Сын Человеческий закончит свою миссию на земле, на всякую плоть прольется живой Дух Истины. Те, кто примут этот дух, никогда не испытают духовной жажды».
\vs p162 6:2 Иисус не прерывал службу, произнося эти слова. Он обратился к верующим сразу после пения Халлела --- ответного чтения псалмов, сопровождаемого помахиванием ветвями перед алтарем. И пока готовилось жертвоприношение, как раз была пауза, и именно в это время паломники услышали, как чарующий голос Учителя провозгласил, что он --- источник живой воды для каждой души, испытывающей духовную жажду.
\vs p162 6:3 Во время заключительной части этой ранней утренней службы Иисус продолжил учить народ, сказав: «Не читали ли вы в Писании: „Смотрите, как вода льется на сухую землю и растекается по пересохшей почве, так же и я дам дух святости, чтобы лился он на ваших детей для благословения даже детей ваших детей“? Почему вы испытываете жажду по духовному пастырству, но при этом стремитесь оросить свои души людскими традициями, выливаемыми из разбитых кувшинов церемониальной службы? То, что вы видите, что происходит в этом храме --- это обряд, которым ваши отцы стремились символизировать сошествие божественного духа на детей веры, и вы хорошо сделали, что сохранили его по сей день. Но теперь к сегодняшнему поколению пришло откровение духовного Отца через пришествие его Сына, и за всем этим, несомненно, последует нисшествие духа Отца и Сына на детей человеческих. Для каждого, имеющего веру, это пришествие духа станет истинным проводником на пути, ведущем к вечной жизни, к истинным водам жизни царства небесного на земле и там, в Раю у Отца».
\vs p162 6:4 И Иисус продолжил отвечать на вопросы толпы и фарисеев. Некоторые считали, что он пророк; некоторые считали его Мессией; другие же говорили, что он не может быть Христом, поскольку пришел из Галилеи, а Мессия должен восстановить трон Давида. И все же они не осмеливались арестовать его.
\usection{7. Беседа о духовной свободе}
\vs p162 7:1 В последний день праздника после полудня апостолы пытались убедить его скрыться из Иерусалима, но это им не удалось, Иисус снова пошел в храм учить. Увидев большую толпу верующих, собравшихся на Соломоновом крыльце, он обратился к ним и сказал:
\vs p162 7:2 \P\ «Если мои слова живут в вас и вы намерены исполнять волю моего Отца, тогда вы действительно мои ученики. Вы узнаете истину, и истина сделает вас свободными. Я знаю, как вы ответите мне: мы дети Авраама, и мы никому не подвластны; как же тогда мы сделаемся свободными? Пусть так, я не говорю о внешнем подчинении чужой власти; я говорю о свободе души. Истинно, истинно говорю я вам, каждый, кто совершает грех, --- подневольный слуга греха. А вы знаете, что подневольный слуга вряд ли всегда будет жить в доме господина. И вы знаете также, что сын остается в доме отца. Поэтому если Сын сделает вас свободными, сделает вас сыновьями, то воистину вы будете свободны.
\vs p162 7:3 Я знаю, что вы --- семя Авраама, однако ваши правители стремятся убить меня, потому что моему слову не дано преобразить их сердца. Их души скованы предрассудками и ослеплены гордыней мести. Я возвещаю вам истину, которую указывает мне вечный Отец, тогда как эти заблуждающиеся учителя стремятся делать то, чему они научились лишь у своих смертных отцов. И когда вы отвечаете, что Авраам --- ваш отец, тогда я говорю вам, что если бы вы были детьми Авраама, то вы служили бы делу Авраама. Одни из вас верят в мое учение, но другие пытаются уничтожить меня, потому что я принес вам истину, которую принял от Бога. Но Авраам не так относился к истине Бога. Я чувствую, что некоторые среди вас намерены служить делу нечистого. Если бы Бог был вашим Отцом, вы узнали бы меня и возлюбили бы истину, которую я открываю. Неужели вы не видите, что я пришел от Отца, что я послан Богом, что я все это делаю не от себя лично? Почему вы не понимаете моих слов? Потому ли это, что вы предпочли стать детьми зла? Если вы дети тьмы, едва ли вы ступите на путь света истины, которую я открываю. Дети зла следуют лишь по пути своего отца, который был обманщиком и не стоял за истину, потому что в нем не стало истины. Но теперь пришел Сын Человеческий, говорящий истину и живущий по истине, а многие из вас отказываются верить.
\vs p162 7:4 Кто из вас изобличит меня в грехе? И если я возвещаю истину, данную мне Отцом, и живу в соответствии с ней, почему вы не верите? Тот, кто от Бога, с радостью слышит слова Бога; многие из вас не слышат моих слов по той причине, что вы не от Бога. Ваши учителя даже осмелились говорить, что мне в делах моих помогает принц тьмы. Один из стоящих неподалеку только что сказал, что во мне дьявол, что я --- сын дьявола. Но все те из вас, кто честно следуют велению своих душ, прекрасно знают, что я не дьявол. Вы знаете, что я почитаю Отца, хотя вы и не почитаете меня. Я не стремлюсь к своей славе, но лишь к славе моего Райского Отца. И я не сужу вас, ибо есть тот, кто судит за меня.
\vs p162 7:5 Истинно, истинно говорю я вам, верующим в евангелие, что, если это слово истины будет жить в сердце человека, он никогда не испытает смерти. А теперь книжник, что рядом со мной, говорит, что так как Авраам мертв, равно как и пророки, то это как раз и доказывает, что во мне дьявол. И он спрашивает: «Неужели ты настолько более велик, чем Авраам и пророки, что осмеливаешься стоять здесь и говорить, будто всякий, в чьем сердце будет жить твое слово, не испытает смерти? Кем же ты себя считаешь, если осмеливаешься произносить такие богохульства?» И я говорю всем таким, как он, что если я возвеличу себя сам, мое величие --- ничто. Но возвеличит меня мой Отец, тот Отец, которого вы называете Богом. Но вы не узнали вашего Бога и моего Отца, и я пришел, чтобы свести вас вместе; чтобы показать вам, как стать воистину сыновьями Бога. Хотя вы не знаете Отца, я воистину знаю его. Даже и Авраам возрадовался, провидя мой день, и благодаря вере узнал о нем и ликовал».
\vs p162 7:6 \P\ Когда неверующие евреи и люди Синедриона, собравшиеся к этому времени вокруг него, услышали эти слова, они подняли шум, крича: «Тебе нет и пятидесяти лет, и при этом ты говоришь, будто видел Авраама; ты --- сын дьявола!» Иисус был не в состоянии продолжать свою речь. Уходя, он лишь сказал: «Истинно, истинно говорю я вам: я есть прежде, чем был Авраам». Многие из неверящих схватили камни, чтобы побить его, люди Синедриона попытались арестовать его, но Учитель быстро проследовал по коридорам храма и скрылся в укромном месте возле Вифании, где его ожидали Марфа, Мария и Лазарь.
\usection{8. Посещение Марфы и Марии}
\vs p162 8:1 Было решено, что Иисус будет жить с Лазарем и его сестрами в доме у друга, а апостолы разместились в разных местах небольшими группами. Такие меры предосторожности были приняты потому, что еврейские власти вновь вознамерились исполнить свой план и арестовать его.
\vs p162 8:2 Уже много лет эти трое, как только появлялся Иисус, все бросали и внимали его учению. После утраты родителей Марфа взяла на себя заботы по дому, так что на этот раз, пока Лазарь и Мария сидели у ног Иисуса, впитывая его живительное учение, Марфа готовила вечернюю трапезу. Следует сказать, что Марфа без особой необходимости часто отвлекалась на многочисленные ненужные хлопоты и тем самым загружала себя множеством мелких забот; таков уж был у нее характер.
\vs p162 8:3 Занимаясь всеми этими делами, казавшимися ей важными, Марфа была возмущена тем, что Мария ничем ей не помогала. Поэтому она подошла к Иисусу и сказала: «Учитель, разве тебя не беспокоит, что моя сестра оставила меня одну и не помогает готовить еду? Не велишь ли ты ей пойти и помочь мне?» Иисус ответил: «Марфа, Марфа, почему ты всегда озабочена столькими вещами и беспокоишься по стольким пустякам? Только на одно действительно стоит тратить время, а Мария как раз и выбрала это доброе и нужное занятие, и я не буду отрывать ее от него. Но когда же вы обе научитесь жить так, как я учил вас: чтобы обе совместно занимались делами и обе в согласии питали свои души? Неужели вы не можете усвоить, что для всего есть свое время --- что менее важные жизненные дела должны уступать дорогу более важным делам небесного царства?»
\usection{9. В Вифлееме с Авениром}
\vs p162 9:1 На протяжении всей недели, последовавшей за праздником кущей, десятки верующих стекались в Вифанию, чтобы получить наставления от двенадцати апостолов. Синедрион не предпринимал попыток преследования этих людей, поскольку там не присутствовал Иисус; он все это время работал с Авениром и его сподвижниками в Вифлееме. На следующий день после праздника Иисус ушел в Вифанию, и в это посещение Иерусалима он больше не учил в храме.
\vs p162 9:2 \P\ В это время Авенир сделал своим главным пристанищем Вифлеем, и из этого центра многочисленные миссионеры отправлялись в города Иудеи и южной Самарии и даже в Александрию. В течение нескольких дней после прибытия Иисуса они с Авениром согласовали все вопросы, существенные для объединения деятельности двух групп апостолов.
\vs p162 9:3 В праздник кущей Иисус уделял примерно равное внимание Вифании и Вифлеему. В Вифании значительную часть времени он проводил с апостолами; в Вифлееме он давал много наставлений Авениру и другим бывшим апостолам Иоанна. И именно это личное общение в конечном счете привело к тому, что все они поверили в него. На бывших апостолов Иоанна Крестителя произвело впечатление мужество, которое он явил, когда учил народ в Иерусалиме, и отзывчивость и понимание, которые они почувствовали, когда он лично наставлял их в Вифлееме. Сила этого влияния полностью и окончательно привела каждого из сподвижников Авенира к тому, что они всем сердцем приняли царство и все с ним связанное.
\vs p162 9:4 \P\ Прежде, чем окончательно покинуть Вифлеем, Учитель договорился со всеми ними предпринять совместные действия, которые должны были предшествовать завершению его земной жизни во плоти. Договорились, что Авенир и его сподвижники в ближайшем будущем присоединятся к Иисусу и его двенадцати апостолам в Магаданском парке.
\vs p162 9:5 Согласно этой договоренности в начале ноября Авенир и его одиннадцать товарищей соединили свою судьбу с Иисусом и его двенадцатью апостолами и трудились вместе с ними, составляя единое целое, вплоть до распятия.
\vs p162 9:6 Во второй половине октября Иисус и двенадцать апостолов ушли из близлежащих окрестностей Иерусалима. В воскресенье 30 октября Иисус и его сподвижники покинули город Ефраим (там он несколько дней отдыхал в уединении) и, следуя западно\hyp{}иорданской дорогой, в среду 2 ноября после полудня добрались до Магаданского леса.
\vs p162 9:7 Апостолы испытали огромное облегчение, когда Учитель снова оказался в дружественных землях; больше они никогда не уговаривали его отправиться в Иерусалим возвещать евангелие царства.
