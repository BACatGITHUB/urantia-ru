\upaper{173}{Понедельник в Иерусалиме}
\author{Комиссия срединников}
\vs p173 0:1 В этот понедельник рано утром Иисус и апостолы, как договорились накануне, собрались в доме у Симона в Вифании и после короткого совещания отправились в Иерусалим. По дороге к храму двенадцать апостолов были непривычно молчаливы; они еще не пришли в себя от потрясений предыдущего дня. Они пребывали в напряженном ожидании, испытывали страх и чувство какой\hyp{}то глубокой отрешенности, вызванное внезапной переменой в поведении Учителя и его указанием не заниматься никаким публичным учением всю пасхальную неделю.
\vs p173 0:2 Когда они спускались с Масличной горы, Иисус шел впереди, а апостолы вплотную следовали за ним в задумчивом молчании. У всех, кроме Иуды Искариота, на уме была лишь одна мысль: как Учитель поступит сегодня? Иуда же был поглощен мыслью: как мне поступить? Идти ли мне дальше вместе с Иисусом и своими друзьями или же бросить их? И если я собираюсь уйти, как мне порвать с ними?
\vs p173 0:3 В это прекрасное утро примерно в девять часов эти люди дошли до храма. Они сразу направились в большой двор, где Иисус так часто учил, и, поприветствовав ожидавших его верующих, Иисус поднялся на одну из кафедр и обратился к собирающейся толпе. Апостолы отошли на некоторое расстояние и ждали развития событий.
\usection{1. Очищение храма}
\vs p173 1:1 Шла бойкая торговля, связанная с храмовыми службами и церемониями. Это была коммерческая деятельность по поставке соответствующих животных для разных жертвоприношений. Хотя и допускалось, чтобы верующий сам приносил животное для совершаемого им жертвоприношения, но требовалось, чтобы это животное было лишено всех «недостатков» с точки зрения закона левитов и его толкования официальными инспекторами храма. Многие прихожане испытывали чувство унижения, когда их, как они считали, безупречные животные отвергались храмовыми ревизорами. Поэтому со временем повелось покупать животных для жертвоприношений в храме, и хотя на расположенной неподалеку Масличной горе было несколько мест, где их можно было купить, вошло в обыкновение покупать этих животных прямо из храмовых загонов. Постепенно появился обычай продавать всевозможных животных для жертвоприношений и на территории храма. Таким образом было положено начало широкой коммерческой деятельности, приносящей огромные доходы. Часть этих доходов предназначалась для храмовой казны, но большая часть окольными путями попадала в руки семей первосвященников, находящихся у власти.
\vs p173 1:2 Продажа животных в храме процветала еще и потому, что, когда прихожанин приобретал такое животное, хотя цена могла быть довольно высокой, больше уже не надо было ничего платить, и он мог быть уверен, что жертва не будет отвергнута по причине того, что обладает недостатками с реальной или формальной точки зрения. Время от времени по отношению к простым людям вводилась система непомерного завышения цен, особенно в великие национальные праздники. Однажды жадность священников зашла настолько далеко, что за пару голубей, которые следовало бы продавать бедным за несколько грошей, они стали требовать сумму, равную недельному заработку. «Сыновья Анны» начали уже организовывать на территории храма свои базары, те самые торговые рынки, которые продолжали существовать, пока их окончательно не смела толпа за три года до разрушения самого храма.
\vs p173 1:3 \pc Но дворы храма осквернялись не только торговлей животными для жертвоприношений и прочими товарами. В это время процветала широкая система банковского и коммерческого обмена денег, который производился прямо на территории храма. А возникло все это следующим образом: в период Хасмонейской династии евреи чеканили свою собственную серебряную монету, и стало обычным требование, чтобы храмовый сбор в размере полушекеля и все прочие храмовые платежи платились этой еврейской монетой. Это правило привело к необходимости выдавать менялам разрешение обменивать многочисленные монеты, находившиеся в обращении в Палестине и других провинциях Римской империи, на шекель еврейской чеканки. Главный храмовый налог, который должны были платить все, кроме женщин, рабов и несовершеннолетних, равнялся полушекелю, монете размером примерно с десятицентовую американскую монету, но в два раза толще. Во времена Иисуса священники тоже были освобождены от уплаты храмовых сборов. Соответственно, с 15 по 25 число месяца, предшествующего Пасхе, менялы, имеющие официальное разрешение, устанавливали свои лавки в основных городах Палестины, чтобы обеспечить еврейский народ по прибытии в Иерусалим соответствующими деньгами, пригодными для уплаты храмовых сборов. После этого десятидневного срока менялы перебирались в Иерусалим и начинали устанавливать свои меняльные столы во дворах храма. В качестве комиссионных за обмен монеты достоинством примерно в десять центов им позволялось брать сумму, эквивалентную трем\hyp{}четырем центам, а в случае, если для обмена предлагалась монета большего достоинства, им разрешалось брать за обмен в два раза больше. Таким образом, храмовые менялы получали прибыль от обмена всех денег, предназначенных и для покупки животных для жертвоприношений, и для оплаты обрядов, и для пожертвований.
\vs p173 1:4 Храмовые менялы постоянно занимались банковским делом не только для извлечения дохода от обмена более двадцати видов денег, которые периодически приносили приходящие в Иерусалим паломники, но осуществляли также и все прочие виды операций, относящиеся к банковскому делу. И казна храма, и управители храма получали громадный доход от этой коммерческой деятельности. В храмовой казне нередко бывало свыше десяти миллионов долларов, тогда как простые люди прозябали в нищете и продолжали платить эти несправедливые подати.
\vs p173 1:5 \pc В этот понедельник утром среди шумного скопища менял, торговцев и продавцов скота Иисус пытался учить евангелию царствия небесного. Он был не одинок в своем негодовании по поводу осквернения храма; простые люди, особенно евреи, приходившие из других провинций, тоже искренне возмущались этим осквернением места их национального богослужения. В то же время и синедрион проводил свои регулярные советы в палате, среди этого гомона и сумятицы от торговли и мены.
\vs p173 1:6 Когда Иисус уже собрался начать свою речь, произошли два события, приковавшие его внимание. У меняльного стола расположившегося рядом менялы начался горячий и возбужденный спор из\hyp{}за якобы завышенных комиссионных, взятых с еврея из Александрии, и в тот же самый момент воздух задрожал от мычания стада примерно из ста волов, перегоняемых из одного загона для скота в другой. Когда Иисус сделал паузу, молчаливо, но внимательно разглядывая эту сцену торговли и гвалта, он увидел, как высокомерные и претендующие на превосходство иудеи осмеивают и отталкивают стоящего поблизости простоватого галилеянина, человека, с которым он однажды беседовал в Ироне; все это вместе взятое вызвало один из тех странных и периодически случающихся всплесков негодования в душе Иисуса.
\vs p173 1:7 К удивлению его апостолов, стоявших тут же рядом, которые воздержались от участия в том, что вскоре последовало, Иисус спустился с ораторского возвышения и, подойдя к парню, гнавшему скот через двор, взял у него веревочный бич и быстро выгнал животных из храма. Но это было не все; он величественно прошагал под удивленными взглядами тысяч людей, собравшихся во дворах храма, к самому дальнему загону для скота и принялся открывать ворота каждого стойла и выгонять стоявших там животных. К этому моменту собравшиеся паломники были столь возбуждены, что, с шумным криком двинувшись к базару, начали переворачивать столы менял. Менее чем за пять минут вся торговля была выметена из храма. К тому времени, когда на место событий прибежали находившиеся неподалеку римские стражники, все было тихо, и толпа уже успокоилась; Иисус вернулся на ораторское место и обратился к толпе: «В этот день вы стали свидетелями того, что сказано в Писании: „Дом мой назовется домом молитвы для всех народов, а вы сделали его вертепом разбойников“».
\vs p173 1:8 Но прежде, чем он смог произнести еще какие\hyp{}либо слова, огромная толпа разразилась восторженными возгласами, и вскоре множество юношей выступили из толпы, чтобы спеть хвалебные гимны в благодарность за изгнание бессовестных и алчных торговцев из святого храма. К этому времени на месте событий появилось несколько священников, и один из них сказал Иисусу: «Разве ты не слышишь, что говорят дети левитов?» И Учитель ответил: «Разве ты никогда не читал: „Из уст младенцев и грудных детей ты устроил хвалу?“» И весь остаток этого дня, пока Иисус учил, у каждого прохода стояли выставленные народом стражники, которые никому не позволяли проносить через дворы храма даже пустой сосуд.
\vs p173 1:9 \pc Когда первосвященники и книжники услышали об этих событиях, они были ошеломлены. Они все больше боялись Учителя и все больше крепла их решимость уничтожить его. Но они были в замешательстве. Они не знали, как погубить его, поскольку чрезвычайно боялись толп, которые теперь открыто выражали свое одобрение того, что он разогнал нечестивых торговцев. И весь этот день, день тишины и покоя во дворах храма, народ слушал учение Иисуса и буквально упивался его словами.
\vs p173 1:10 Удивительный поступок Иисуса был выше понимания его апостолов. Они были так поражены внезапным и неожиданным поведением своего Учителя, что на протяжении всего эпизода стояли все вместе тесной кучкой возле ораторского места; они ни разу даже не пошевелили пальцем, чтобы помочь очистить храм. Если бы это впечатляющее событие случилось накануне, во время триумфального входа Иисуса в храм, после того, как он под возгласы восторга прошел через ворота города и толпа бурно приветствовала его, они были бы готовы к нему, но теперь, когда оно произошло таким вот образом, они были не в состоянии участвовать в нем.
\vs p173 1:11 Это очищение храма раскрывает отношение Учителя к коммерциализации религиозных обрядов, его отвращение ко всем формам несправедливости и к получению наживы за счет бедных и простых людей. Этот эпизод также показывает, что Иисус был не против использования силы для защиты большинства какой\hyp{}либо группы людей от несправедливостей и порабощения их несправедливым меньшинством, которое оказалось в состоянии захватить прочные позиции в политической, финансовой или церковной власти. Расчетливым, злым и хитрым людям не должно быть позволено объединяться для эксплуатации и порабощения тех идеалистов, кто был не склонен прибегать к силе для самозащиты или для осуществления своих достойных похвалы жизненных планов.
\usection{2. Оспаривание права Учителя}
\vs p173 2:1 Триумфальный вход в Иерусалим в воскресенье внушил еврейским правителям такой страх, что они не решились взять Иисуса под стражу. Точно так же и на следующий день это впечатляющее очищение храма заставило отложить арест Учителя. День ото дня росла решимость правителей евреев уничтожить его, но им внушали страх два обстоятельства, что способствовало отсрочке рокового часа. Первосвященники и книжники не склонны были публично арестовывать Иисуса из\hyp{}за боязни, во\hyp{}первых, что ярость толпы может обратиться против них и, во\hyp{}вторых, что для подавления народного восстания потребуется вызвать римскую стражу.
\vs p173 2:2 В полдень на совете синедриона, поскольку на этом совете не было ни одного друга Учителя, было единодушно решено, что Иисус должен быть безотлагательно уничтожен. Но они не могли договориться о том, когда и как его следует взять под стражу. В конце концов они решили поручить пяти группам пойти и находиться среди людей, чтобы попытаться запутать его в его учении или каким\hyp{}то иным образом дискредитировать его в глазах тех, кто слушает его наставления. Поэтому когда около двух часов Иисус только начал свою проповедь о «Свободе сыновства», одна из групп этих старейшин Израиля протиснулась поближе к Иисусу и, в обычной манере прервав его, задала такой вопрос: «По какому праву ты это делаешь? Кто дал тебе это право?».
\vs p173 2:3 Руководителям храма и членам еврейского синедриона было вполне уместно задать этот вопрос любому, кто позволил бы себе учить или вести себя необычным образом, что было свойственно Иисусу и особенно проявилось в его недавнем поведении при очищении храма от всякой торговли. Все торговцы и менялы действовали по прямому разрешению высших правителей, и процент от их доходов должен был идти непосредственно в храмовую казну. Не забывайте, что \bibemph{право} было ключевым словом для всего еврейства. Пророки всегда вызывали волнения потому, что так смело решались учить без соответствующего права, без должного обучения в раввинских академиях и получения затем официального посвящения от синедриона. Отсутствие такого права в случае претензии на обучение народа рассматривалось или как невежественная самонадеянность, или как открытое бунтарство. В это время только синедрион мог посвящать в старейшины или учителя, и такая церемония должна была проходить в присутствии не менее трех человек, получивших уже до этого такое посвящение. Такое посвящение давало учителю титул «раввин» и право быть судьей, «утверждающим и разрешающим те дела, которые могут быть представлены ему для вынесения судебного решения».
\vs p173 2:4 Руководители храма предстали пред Иисусом в этот послеполуденный час, бросив вызов не только его учению, но и его поступкам. Иисус хорошо знал, что эти же люди долгое время открыто заявляли, что его право учить было от Сатаны и что все его могущественные деяния были содеяны силой принца дьяволов. Поэтому Учитель начал свой ответ на их вопрос с того, что задал им встречный вопрос. Иисус сказал: «Я тоже хотел бы задать вам один вопрос, и если вы мне ответите на него, я также отвечу вам, какой властью я это делаю. Крещение Иоанна, откуда оно? Получил ли Иоанн право с небес или от людей?»
\vs p173 2:5 И когда задавшие ему вопрос услышали это, они отошли в сторону посовещаться между собой, какой можно дать ответ. Они хотели смутить Иисуса перед толпой, но теперь сами оказались в сильном замешательстве перед всеми собравшимися в этот час во дворе храма. И их замешательство стало еще более очевидным, когда они вновь обратились к Иисусу, сказав: «Относительно крещения Иоанна мы не можем ответить; мы не знаем». А ответили они так Учителю потому, что рассудили промеж себя: если мы скажем, что с небес, тогда он скажет: «Почему же вы не поверили ему?» и добавит, возможно, что он получил свое право от Иоанна; а если мы скажем, что от людей, тогда народ может обратиться против нас, поскольку большинство из них считают Иоанна пророком; таким образом, они были вынуждены публично признаться перед Иисусом в том, что они, религиозные учителя и руководители Израиля, не могут (или не хотят) выразить какое\hyp{}либо мнение относительно миссии Иоанна. И когда они это произнесли, Иисус, глядя на них сверху, сказал: «И я вам тоже не скажу, какою властью это делаю».
\vs p173 2:6 \pc Иисус никогда не имел намерения ссылаться на Иоанна, чтобы обосновать свое право; Иоанн никогда не получал посвящения от синедриона. Право Иисуса было в нем самом и в вечном верховенстве его отца.
\vs p173 2:7 Поступая так со своими противниками, Иисус не имел в виду уклониться от вопроса. Поначалу может показаться, что его можно упрекнуть в намерении мастерски уклоняться, но это было не так. Иисус никогда не был склонен нечестно одерживать верх даже над своими врагами. Уклонившись, казалось бы, от ответа он, в действительности, дал всем своим слушателям ответ на вопрос фарисеев о праве на его миссию. Те утверждали, что он действовал по праву, данному принцем дьяволов. Иисус неоднократно утверждал, что все его учение и вся деятельность осуществлялись властью и правом Отца Небесного. Еврейские руководители отказывались согласиться с этим и стремились вынудить его признать, что он учит незаконно, поскольку никогда не был утвержден синедрионом. Ответив им так и, при этом, не утверждая, что получил право от Иоанна, он настолько удовлетворил народ, что попытка его врагов поймать его в ловушку на самом деле обратилась против них самих и сильно дискредитировала их в глазах всех присутствующих.
\vs p173 2:8 И именно это гениальное умение Учителя обращаться со своими противниками заставляло тех так сильно бояться его. В тот день они больше не пытались задавать ему вопросы; они удалились, чтобы еще посовещаться между собой. Но люди быстро распознали непорядочность и лицемерность вопросов, задаваемых еврейскими правителями. Даже простой народ не мог не увидеть разницу между моральным величием Учителя и коварным лицемерием его врагов. Но очищение храма привело саддукеев на сторону фарисеев в вопросе осуществления плана уничтожить Иисуса. А саддукеи теперь составляли большинство в синедрионе.
\usection{3. Притча о двух сыновьях}
\vs p173 3:1 Когда придирчивые фарисеи молча стояли перед Иисусом, он посмотрел на них и сказал: «Поскольку вы пребываете в сомнении относительно миссии Иоанна и враждебно выступаете против учения и деятельности Сына Человеческого, послушайте, я расскажу вам притчу: У некоего крупного и уважаемого землевладельца было два сына и, желая помощи от своих сыновей в управлении своими большими имениями, он пришел к одному из них и сказал: „Сын, пойди поработай сегодня в винограднике моем“. И этот легкомысленный сын в ответ своему отцу сказал: „Я не пойду“; но позже, раскаявшись, пошел. Когда он нашел своего старшего сына, то точно так же сказал ему: „Сын, иди поработай в винограднике моем“. И этот лицемерный и нечестный сын ответил: „Да, отец, я пойду“. Но когда его отец ушел, он не пошел. Позвольте мне спросить вас: который из сыновей действительно исполнил волю своего отца?»
\vs p173 3:2 И люди сказали в один голос: «Первый сын». И тогда Иисус сказал: «Именно так; и теперь я заявляю, что мытари и блудницы, даже если кажется, что они отвергают призыв к покаянию, поймут ошибочность своего пути и войдут в царство Божие прежде вас, старательно делающих вид, что служите Отцу Небесному, отказываясь в то же время выполнять работу для Отца. Не вы, фарисеи и книжники, поверили Иоанну, а мытари и грешники; не верите вы и моему учению, но простые люди радостно слушают мои слова».
\vs p173 3:3 Иисус не презирал фарисеев и саддукеев как таковых. Он только стремился дискредитировать их учения и дело. Он не испытывал враждебности ни к одному человеку, но здесь происходило неизбежное столкновение между новой и живой религией духа и старой религией церемоний, традиций и прав.
\vs p173 3:4 Все это время двенадцать апостолов стояли возле Учителя, но никоим образом не участвовали в спорах. Каждый из двенадцати по\hyp{}своему специфически реагировал на события заключительных дней служения Иисуса во плоти, и каждый в равной мере оставался послушным повелению Учителя воздерживаться от всякого публичного учения и проповедования в течение этой пасхальной недели.
\usection{4. Притча об отсутствовавшем хозяине}
\vs p173 4:1 Когда главные фарисеи и книжники, пытавшиеся запутать Иисуса своими вопросами, выслушали историю о двух сыновьях, они отошли, чтобы еще посоветоваться, а Учитель, обратился к внимающей ему толпе и рассказал еще одну притчу:
\vs p173 4:2 \pc «Был некий хороший человек, хозяин дома, и посадил он виноградник. Он установил вокруг него ограду, выкопал яму, чтобы давить вино, выстроил сторожевую башню для сторожей. Потом он сдал этот виноградник в аренду на время, пока ездил в долгое путешествие в другую страну. А когда приблизилось время сбора плодов, он послал к арендаторам слуг, чтобы получить плату за аренду. Но те посовещались между собой и отказались давать слугам плоды, причитающиеся их хозяину; вместо этого, они напали на его слуг, избив одного, побив камнями другого, и отослали остальных прочь ни с чем. И когда хозяин услышал обо всем этом, он послал для переговоров с этими бесчестными арендаторами других доверенных слуг, но и этих избили и тоже обошлись с ними бесчестно. И тогда хозяин послал своего любимого слугу, своего управляющего, и того они убили. И все же терпеливо и снисходительно он продолжал отправлять многих других слуг, но никого они не приняли. Некоторых избили, других убили, и когда с хозяином так обошлись, он решил послать своего сына для переговоров с этими неблагодарными арендаторами, сказав себе: „Они могут дурно обходиться с моими слугами, но они наверняка проявят уважение к моему возлюбленному сыну“. Но когда эти нераскаявшиеся бесчестные арендаторы увидели сына, они рассудили между собой: „Это --- наследник; давайте же убьем его и тогда наследство его будет наше“. Так что они схватили его, выгнали вон из виноградника и убили. Когда хозяин этого виноградника услышит, как они отвергли и убили его сына, что он сделает с этими неблагодарными и бесчестными арендаторами?»
\vs p173 4:3 \pc И когда люди услышали эту притчу и заданный Иисусом вопрос, они ответили: «Он уничтожит этих низких людей и сдаст виноградник в аренду другим, честным земледельцам, которые будут отдавать ему плоды после сбора урожая». И когда некоторые из слушавших поняли, что эта притча касается еврейской нации и ее обращения с пророками и предстоящего отвержения Иисуса и евангелия царства, они сказали с печалью: «Боже, упаси нас продолжать делать такие вещи».
\vs p173 4:4 Иисус увидел группу саддукеев и фарисеев, пробирающихся сквозь толпу, молча подождал, пока они приблизились к нему, и сказал: «Вы знаете, как ваши отцы отвергали пророков, и хорошо знаете, что ваши сердца настроены отвергнуть Сына Человеческого». И затем, пристально и испытующе глядя на священников и старейшин, стоявших поблизости от него, Иисус сказал: «Неужели вы никогда не читали в Писании о камне, который отвергли строители и который, когда люди обнаружили его, был положен во главу угла? Итак, снова предупреждаю вас, что, если вы будете продолжать отвергать это евангелие, царство Бога вскоре отнимется у вас и дано будет народу, желающему принять благую весть и приносить плоды духа. И в этом камне заключена тайна, потому что кто бы ни упал на него, даже если от этого он разобьется, будет спасен; но всякий, на кого упадет этот камень, будет стерт в порошок, и прах его будет развеян по четырем ветрам».
\vs p173 4:5 Когда фарисеи услышали эти слова, они поняли, что Иисус говорит о них самих и других еврейских руководителях. Они очень хотели схватить его сейчас же и тут же, но боялись народа. Однако они были так рассержены словами Учителя, что удалились и стали дальше совещаться между собой о том, как можно добиться его смерти. И в ту ночь фарисеи и саддукеи объединили усилия в намерении поймать его на следующий день в ловушку.
\usection{5. Притча о брачном пире}
\vs p173 5:1 После того, как книжники и правители удалились, Иисус снова обратился к собравшейся толпе и рассказал притчу о брачном пире. Он сказал:
\vs p173 5:2 \pc «Царство небесное может быть уподоблено некоему царю, который устроил брачный пир для своего сына и послал вестников звать тех, кто были заранее приглашены, говоря: „Все готово для свадебного ужина в царском дворце“. Теперь многие из тех, кто когда\hyp{}то пообещал прийти, на этот раз прийти отказались. Когда царь услышал, что они отказались от его приглашения, он послал других слуг и вестников, сказав им: „Скажите званным, чтобы они пришли, ибо, смотрите, моя трапеза готова. Мои тельцы и мои откормленные животные заколоты, и все готово для празднования приближающейся женитьбы моего сына“. Но беспечные люди снова не придали значения зову своего царя и отправились по своим делам: один в поле, другой в гончарную мастерскую, а кто на торговлю свою. Некоторые же не удовольствовались тем, что пренебрегли приглашением царя, но, проявив открытое неповиновение, подняли руку на посланников царя, оскорбили их и даже убили некоторых из них. И когда царь понял, что его избранные гости, даже те, кто предварительно принял приглашение и обещал прийти на брачный пир, в конце концов отвергли его призыв и, продемонстрировав неповиновение, напали и убили посланных им гонцов, он страшно разгневался. И тогда оскорбленный царь послал свои армии и армии своих союзников и велел им истребить мятежных убийц и сжечь город их.
\vs p173 5:3 И наказав тех, кто пренебрежительно отверг его приглашение, он назначил другой день для брачного пира и сказал своим вестникам: „Те, кто вначале были приглашены на свадьбу, не были достойны; так что идите сейчас на распутья и большие дороги и даже за пределы города и пригласите на брачный пир всех, кого найдете, даже чужестранцев“. И тогда слуги отправились на большие дороги и в отдаленные места и собрали всех, кого только нашли, хороших и плохих, богатых и бедных, и, в конце концов, свадебная палата наполнилась охотно пришедшими гостями. Когда все было готово, царь вошел посмотреть на гостей и, к своему большому удивлению, увидел там человека, на котором не было брачной одежды. Царь, поскольку он бесплатно предоставил брачную одежду всем своим гостям, обратился к этому человеку и спросил: „Друг, как так получилось, что ты по такому торжественному случаю вошел в мою палату для гостей не в брачной одежде?“ Но этот неготовый к ответу человек молчал. Тогда царь сказал своим слугам: „Вышвырните неразумного из моего дома, и пусть он разделит судьбу всех остальных, пренебрегших моим гостеприимством и отвергнувших мой призыв. Мне никто здесь не нужен, кроме тех, кто с удовольствием принимает мое приглашение и кто оказывает мне честь, надев ту одежду для гостей, которая бесплатно дана всем“».
\vs p173 5:4 \pc Рассказав эту притчу, Иисус собирался уже распустить народ, когда один доброжелательный верующий, пробравшись к нему сквозь толпу, спросил: «Но, Учитель, как нам узнать об этом? как нам быть готовыми к приглашению царя? какой знак ты нам подашь, по которому мы узнаем, что ты --- Сын Бога?» И услышав это, Учитель сказал: «Лишь один знак будет дан вам». И затем, указав на свое тело, он продолжил: «Разрушьте этот храм, и через три дня я возведу его». Но они не поняли его и, расходясь, говорили между собой: «Почти пятьдесят лет строился этот храм, однако он говорит, что уничтожит его и возведет за три дня». Даже его собственные апостолы не поняли смысла этого высказывания, но впоследствии, после его воскрешения, они вспомнили то, что он сказал.
\vs p173 5:5 Около четырех часов дня Иисус подал своим апостолам знак, показывая, что желает покинуть храм и отправиться в Вифанию для вечерней трапезы и ночного отдыха. По дороге на Масличную гору Иисус сказал Андрею, Филиппу и Фоме, что назавтра им следует разбить лагерь, в котором они могли бы провести оставшиеся дни пасхальной недели поближе к городу. Согласно этому указанию, на следующее утро они разбили свои палатки на участке земли, принадлежавшем Симону из Вифании и находившемся в лощине на склоне горы, откуда открывался вид на общественный Гефсиманский парк.
\vs p173 5:6 В понедельник вечером это снова была молчаливая группа евреев, держащих путь к западному склону Масличной горы. Эти двенадцать человек, как никогда раньше, начинали чувствовать, что должно произойти нечто трагическое. Если впечатляющее очищение храма рано утром пробудило в них надежды, что их Учитель постоит за себя и проявит свое огромное могущество, то события, произошедшие после полудня, вызвали лишь упадок настроения, поскольку все они явно указывали на то, что еврейские власти отвергают учения Иисуса. Апостолы были охвачены тревогой ожидания и всецело находились во власти ужасных сомнений. Они понимали, что лишь несколько коротких дней могут отделять события только что прошедшего дня от надвигающегося удара судьбы. Все они чувствовали, что вот\hyp{}вот случится что\hyp{}то потрясающее, но не знали, чего ожидать. Они разошлись по своим местам на ночлег, но почти не спали. Даже у близнецов Алфеевых пробудилось, наконец, осознание того, что череда событий в жизни Учителя быстро движется к своей заключительной кульминационной точке.
