\upaper{111}{Настройщик и душа}
\vs p111 0:1 Присутствие божественного Настройщика в человеческом разуме навсегда исключает и для науки и для философии возможность должным образом понять развивающуюся душу человеческой личности. Моронтийная душа --- дитя вселенной, и она может быть действительно познана только посредством космической проницательности и духовного открытия.
\vs p111 0:2 \P\ Представление о душе и о пребывающем в человеке духе не ново для Урантии, оно часто появлялось в различных системах планетарных верований. Многие восточные, а также западные верования осознавали, что человек божественен по своему происхождению и человечен по своей наследственности. Ощущение внутреннего присутствия Божества наряду с ощущением его внешней вездесущности давно стало частью многих Урантийских религий. Люди давно верили, что есть что\hyp{}то развивающееся внутри человеческой природы, что\hyp{}то жизненно важное, что предназначено продолжить существование после истечения короткого срока временной жизни.
\vs p111 0:3 До того, как человек осознал, что его развивающаяся душа порождена божественным духом, он думал, что она помещается в различных органах тела --- в глазах, в почках, в печение, в сердце и позднее --- в мозге. Дикарь связывал душу с кровью, дыханьем, тенями и со своим отражением в воде.
\vs p111 0:4 Концепция \bibemph{атмана} индуистских учителей действительно приближалась к осознаванию природы и присутствия Настройщика, но им не удалось разглядеть при этом и присутствие развивающейся и потенциально бессмертной души. Китайцы, однако, распознавали два аспекта человеческого существа --- \bibemph{янь} и \bibemph{инь,} душу и дух. Египтяне и многие африканские племена также верили в существование двух факторов --- \bibemph{ка} и \bibemph{ба,} причем не душа обычно мыслилась предсуществующей, а только дух.
\vs p111 0:5 Обитатели нильской долины верили, что при рождении или вскоре после него каждый избранный индивидуум одаряется духом\hyp{}защитником, который называется ка. Они учили, что этот дух\hyp{}хранитель остается со своим смертным подопечным в течение всей жизни и прежде него переходит в следующее состояние. На стенах храма в Луксоре, где изображено рождение Аменхотепа III, маленький принц запечатлен на руке бога Нила, а рядом с ним --- другой ребенок, внешностью сходный с принцем, что символизирует ту самую сущность, которую египтяне называли ка. Это изваяние было закончено в пятнадцатом столетии до Рождества Христова.
\vs p111 0:6 Думали, что ка был высшим духом\hyp{}гением, который стремился вести связанную с ним смертную душу по лучшим путям временной жизни, но особенно хотел влиять на судьбу своего человеческого подопечного после этой жизни. Когда египтянин того времени умирал, считали, что ка ждет его на другом берегу Великой Реки. Сначала полагали, что только цари имеют ка, но вскоре стали верить, что ими обладают все праведные люди. Одни египетский правитель, говоря о ка в своем сердце, сказал: <<Я не оставался безучастным к его словам. Я боялся нарушить его указания. Поэтому я в высшей степени преуспевал. Таким образом, я добивался успеха потому, что он понуждал меня делать. Я отличился благодаря его руководству>>. Многие верили, что ка есть <<глас Божий в каждом из людей>>. Многие верили, что они должны были бы >>вечно пребывать в довольстве сердца, находясь под покровительством Бога, который в вас<<.
\vs p111 0:7 У каждой расы развивающихся смертных Урантии есть слово, эквивалентное понятию душа. Многие первобытные народы верили, что душа выходит в мир через глаза человека; поэтому они страшно боялись недоброжелательства злого глаза. Они долго верили, что <<дух человека --- это светильник Господа>>. Ригведа говорит: <<Мой разум говорит с моим сердцем>>.
\usection{1.\bibnobreakspace Разум --- арена выбора}
\vs p111 1:1 Хотя работа Настройщиков духовна по природе, они волей\hyp{}неволей должны строить всю свою работу на интеллектуальной основе. Разум --- это та человеческая почва, из которой дух\hyp{}Помощник в сотрудничестве с личностью, в которой он пребывает, должен вырастить моронтийную душу.
\vs p111 1:2 На некоторых уровнях разума вселенной вселенных существует космическое единство. Интеллектуальные <<я>> берут свое начало в космическом разуме, как туманности берут свое начало в космических энергиях вселенского пространства. На человеческом (и поэтому личностном) уровне интеллектуальных <<я>> потенциал эволюции духа становится --- с санкции разума --- господствующим благодаря духовным способностям человеческой личности, а также творческому присутствию стержневой сущности абсолютной ценности в таких человеческих <<я>>. Но такое преобладание духовности в материальном разуме обусловлено двояким опытом: этот разум должен был развиваться посредством служения семи духов\hyp{}помощников разума и материальное (личностное) <<я>> должно решиться на сотрудничество с пребывающим в разуме Настройщиком в создании и воспитании моронтийного <<я>> --- эволюционирующей и потенциально бессмертной души.
\vs p111 1:3 \P\ Материальный разум есть арена, на которой человеческие личности живут, осознают себя как личность, принимают решения, выбирают Бога или отвергают его, делают себя вечными или уничтожают себя.
\vs p111 1:4 \P\ Материальная эволюция снабдила тебя механизмом жизни, твоим телом; сам Отец наделил тебя чистейшей духовной реальностью, существующей во вселенной, твоим Настройщиком Мысли. Но разум тебе дан в твои собственные руки, он подчинен твоим собственным решениям, и именно благодаря твоему разуму ты живешь или умираешь. Именно внутри этого разума и с помощью этого разума ты принимаешь те нравственные решения, которые делают тебя способными достичь подобия Настройщика, а это есть подобие Бога.
\vs p111 1:5 Смертный разум есть временная интеллектуальная система, одолженная человеческим существам для использования во время материальной жизни, и в зависимости от того, как они используют этот разум, они или принимают, или отвергают потенциал вечного существования. Во вселенской реальности разум --- единственное из того, чем ты обладаешь, что подчиняется твоей воле, и душа --- моронтийное <<я>> --- правдиво отобразят все временные решения, которые принимает смертное <<я>>. Внизу --- человеческое сознание зыбко покоится на электрохимическом механизме, а вверху --- нежно соприкасается с духовно\hyp{}моронтийной энергетической системой. В своей смертной жизни человеческое существо никогда полностью не осознает ни одну из этих систем; следовательно, оно должно действовать в разуме, который осознает. И не столько то, что разум понимает, сколько то, что разум стремится понять, обеспечивает продолжение существования в посмертии; не столько то, на что похож разум, как то, на что он жаждет быть похожим, составляет его духовную идентификацию. Не столько то, что человек осознает Бога, сколько то, что человек стремиться к Богу, проявляется во вселенском восхождении. То, чем вы сегодня являетесь, не столь важно, как то, чем вы становитесь день ото дня и в вечности.
\vs p111 1:6 Разум --- это космический инструмент, на котором человеческая воля может сыграть какофонию уничтожения или же на котором та же самая человеческая воля может воспроизвести прелестные мелодии тождественности с Богом и последующего вечного существования в посмертии. Дарованный человеку Настройщик в конечном счете, недоступен для зла, и неспособен к греху, но смертный разум действительно может быть искривлен, искажен, обращен во зло и обезображен греховными кознями извращенной и своекорыстной человеческой воли. Точно так же, этот разум может сделаться благородным, прекрасным, правильным и добрым --- действительно великолепным --- в соответствии с озаренной духом волей человеческого существа, знающего Бога.
\vs p111 1:7 \P\ Эволюционирующий разум является только тогда полностью стабильным и уравновешенным, когда он выражается в двух крайностях космической интелектуальности --- полностью механизированной и всецело одухотворенной. Между этими интеллектуальными крайностями чисто механического контроля и истинно духовной природы существует чрезвычайно большая группа развивающихся и восходящих разумов, чья стабильность и спокойствие зависят от выбора личности и тождественности с духом.
\vs p111 1:8 Но человек не пассивно, не рабски подчиняет свою волю Настройщику. Скорее, он, действуя активно, позитивно и в духе сотрудничества, решает следовать водительству Настройщика в том случае, когда такое водительство ощутимо отличается от желаний и стремлений природного смертного разума. Настройщики влияют на разум человека, но никогда не властвуют над ним против его воли; для Настройщиков человеческая воля верховна. Именно так они и относятся к ней и уважают ее, в то время как стремятся достичь духовных целей настройки мысли и преобразования характера на почти безграничной арене развивающегося человеческого интеллекта.
\vs p111 1:9 \P\ Разум --- это ваш корабль, Настройщик --- ваш лоцман, человеческая воля --- капитан. Владелец смертного судна должен обладать мудростью, чтобы доверить божественному лоцману направлять восходящую душу в моронтийные гавани вечного существования. Только из эгоизма, лености и греховности воля человека может отказатся от водительство такого любящего лоцмана и, в конце концов, погубить продвижение смертного на мелководье зла отвергнутого милосердия и на скалах избранного греха. Заручившись твоим согласием, этот верный лоцман безопасно проведет тебя через барьеры времени и препятствия пространства к самому источнику божественного разума и даже за его пределы --- к Райскому Отцу Настройщиков.
\usection{2.\bibnobreakspace Природа души}
\vs p111 2:1 Среди разумных функций космического интеллекта тотальность разума доминирует над частными интеллектуальными функциями. Разум, по существу, есть функциональное единство; поэтому разум никогда не устает выражать это существенное единство, даже в том случае, когда ему препятствуют и его стесняют неблагоразумные действия и выбор заблуждающегося <<я>>. И это единство разума неизменно стремится к духовному согласованию на всех уровнях его общения с другими <<я>>, которые обладают свободной волей и прерогативами восхождения.
\vs p111 2:2 Материальный разум смертного человека --- это космический ткацкий станок, несущий моронтийную основу, на которую пребывающий Настройщик Мысли нанизывает духовные паттерны вселенского свойства, вечных ценностей и божественных значений, --- создавая существующую в посмертии душу предельного предназначения и нескончаемого продвижения, потенциального финалита.
\vs p111 2:3 Человеческая личность отождествляется с разумом и духом, которые находятся в материальном теле в функциональной взаимосвязи, обуславливаемой жизнью. Эта функционирующая связь такого разума и такого духа не приводит к появлению некоей комбинации качеств или атрибутов, присущих разуму и духу, а, скорее, к появлению совершенно новой, оригинальной и уникальной вселенской ценности потенциально вечного характера, т.е. \bibemph{душе.}
\vs p111 2:4 \P\ В эволюционном процессе создания такой бессмертной души существуют три, а не два фактора. Этими тремя предшественниками моронтийной человеческой души являются:
\vs p111 2:5 \ublistelem{1.}\bibnobreakspace \bibemph{Человеческий разум} и все космические влияния, предшествующие ему и соприкасающиеся с ним.
\vs p111 2:6 \P\ \ublistelem{2.}\bibnobreakspace \bibemph{Божественный дух,} пребывающий в этом человеке, и все потенциалы, присущие такому фрагменту абсолютной духовности, вместе со всеми связанными с ним духовными влияниями и факторами, встречающимися в человеческой жизни.
\vs p111 2:7 \P\ \ublistelem{3.}\bibnobreakspace \bibemph{Связь между материальным разумом и божественным духом,} означающая ценность и содержащая значение, которые нельзя найти ни в одной из составляющих такого союза. Реальность этой уникальной связи не является ни материальной, ни духовной --- она моронтийна. Это и есть душа.
\vs p111 2:8 \P\ Срединные создания давно нарекли эту развивающуюся душу человека срединным разумом в отличие от низшего материального разума и высшего или космического разума. Этот срединный разум есть в действительности моронтийный феномен, поскольку он существует в области между материальным и духовным. Потенциал такой моронтийной эволюции свойственен двум вселенским побуждениям разума: стремлению конечного разума создания познать Бога и достичь божественности Творца и стремлению бесконечного разума Творца познать человека и достичь \bibemph{опыта} создания.
\vs p111 2:9 Это возвышенный процесс развития бессмертной души становится возможным, потому что смертный разум является, во\hyp{}первых, личностным, а во\hyp{}вторых, он находится в контакте со сверхчувственными реальностями; он обладает сверхматериальным даром космического служения, который обеспечивает эволюцию смертной природы, способной принимать нравственные решения, осуществляя тем самым истинный творческий контакт с сподвижниками духовного служения и с пребывающим в нем Настройщике Мысли.
\vs p111 2:10 Неизбежным результатом такого контактного одухотворения человеческого разума является постепенное рождение души, общего плода смертного разума, находящегося под влиянием человеческой воли, которая жаждет познать Бога, работающего в связи с духовными силами вселенной, которые находятся под сверхконтролем фактического фрагмента самого Бога всего мироздания --- Таинственного Помощника. И таким образом, материальная и смертная реальность <<я>> выходит за пределы временных ограничений механизма физической жизни и достигает нового выражения и новой идентификации в развивающемся носителе целостности индивидуальности --- в моронтийной и бессмертной душе.
\usection{3.\bibnobreakspace Развивающаяся душа}
\vs p111 3:1 Заблуждения смертного разума и ошибки человеческого поведения могут значительно отсрочить эволюцию души, хотя не могут подавить такой моронтийный феномен, если однажды он был инициирован пребывающим Настройщиком с согласия воли создания. Но в любой момент, предшествующий физической смерти, эта самая человеческая воля способна отменить такой выбор и отвергнуть продолжение существования в посмертии. Даже продолжая существовать в посмертии, восходящий смертный все же сохраняет это право отвергнуть вечную жизнь; в любой момент, предшествующий слиянию с Настройщиком, развивающееся восходящее создание может сделать выбор и отказаться следовать воле Райского Отца. Слияние с Настройщиком знаменует тот факт, что восходящий смертный навечно и безвозвратно решил исполнять волю Отца.
\vs p111 3:2 Во время жизни во плоти развивающаяся душа способна укрепить сверхматериальные решения смертного разума. Душа, будучи сверхматериальной, сама не функционирует на материальном уровне человеческого опыта. Эта субдуховная душа не может без сотрудничества с неким духом Божества, таким как Настройщик, функционировать и на уровне, находящемся выше моронтийного. Душа не принимает и окончательных решений, пока смерть или перемещение не освободит ее от материальной связи со смертным разумом, за исключением того случая, когда этот материальный разум свободно и добровольно отдает право принятия таких решений такой моронтийной душе, связанной с этим разумом Во время жизни смертная воля, личностная способность к решению\hyp{}выбору сосредоточена в контурах материального разума; по мере земного смертного развития это <<я>> вместе с его бесценными способностями выбора все больше и больше отождествляется с нарождающейся сущностью моронтийной души; после смерти и вслед за воскресением в мире\hyp{}обители человеческая личность полностью отождествляется с моронтийным <<я>>. Таким образом, душа есть эмбрион будущего моронтийного носителя личностной идентичности.
\vs p111 3:3 Эта бессмертная душа является в первую очередь всецело моронтийной по природе, но обладает такой способностью к развитию, что неизменно поднимается до истинных духовных уровней ценности слияния с духами Божества, обычно с самим духом Отца Всего Сущего, который инициировал в разуме создания такой творческий феномен.
\vs p111 3:4 И человеческий разум, и божественный Настройщик осознают присутствие и особую природу развивающейся души --- Настройщик полностью, а разум частично. Душа становится все более и более --- пропорционально своему эволюционному росту --- осознающей и разум, и Настройщика как связанные индивидуальности. Душа разделяет качества и человеческого разума, и божественного духа, но она неизменно развивается в направлении усиления духовного контроля и божественного господства посредством воспитания функции разума, значения которого стремятся к согласованию с истинными духовными ценностями.
\vs p111 3:5 Смертное продвижение, эволюция души является не столько испытанием, сколько обучением. Вера в вечное существование верховных ценностей есть суть религии; истинный религиозный опыт состоит в объединении верховных ценностей и космических значений как реализации вселенской реальности.
\vs p111 3:6 Разум познает количество, реальность, значения. Но качества --- ценности --- \bibemph{ощущаются.} То, что ощущает,, есть общее создание разума, который познает, и связанного с ним духа, который изображает реалии.
\vs p111 3:7 Поскольку развивающаяся моронтийная душа человека проникается истиной, красотой и добродетелью по мере воплощения ценностей осознания Бога, такое получающееся в результате существо становится неуничтожимым. Если вечные ценности не сохраняются в развивающейся душе человека, тогда смертное существование бессмысленно, а сама жизнь --- трагическая иллюзия. Но всегда истинно изречение: то, что ты начинаешь во времени, ты несомненно завершишь в вечности --- если оно того стоит.
\usection{4.\bibnobreakspace Внутренняя жизнь}
\vs p111 4:1 Узнавание есть интеллектуальный процесс соответствия чувственных впечатлений, полученных от внешнего мира, паттернам памяти индивидуума. Понимание означает, что эти узнанные чувственные впечатления и связанные с ними паттерны памяти интегрируются или организуются в динамическую сеть принципов.
\vs p111 4:2 Значения выводятся из сочетания узнавания и понимания. Значения не существуют в мире, который является всецело чувственным или материальным. Значения и ценности постигаются только во внутренних или сверхматериальных сферах человеческого опыта.
\vs p111 4:3 \P\ Всякий прогресс истинной цивилизации берет свое начало в этом внутреннем мире человечества. Но только внутренняя жизнь является истинно творческой. Цивилизация вряд ли может развиваться, если большинство молодежи любого поколения направляют свои интересы и энергию на материалистические цели чувственного или внешнего мира.
\vs p111 4:4 Внутренний и внешний миры имеют различные системы ценностей. Любая цивилизация находится в опасности, если три четверти ее молодежи избирают материалистические профессии и посвящают себя поискам чувственной деятельности внешнего мира. Цивилизация находится в опасности, если молодежь не обращает внимания на этику, социологию, евгенику, философию, изящные искусства, религию и космологию.
\vs p111 4:5 Только в высших уровнях надсознательного разума, когда он вторгается в духовную область человеческого опыта, вы можете обрести те высшие понятия, связанные с действенными главными паттернами, которые будут способствовать построению лучшей и более долговечной цивилизации. Личность является врожденно творческой, но так она функционирует лишь во внутренней жизни индивидуума.
\vs p111 4:6 \P\ Кристаллы снега всегда имеют форму шестиугольника, но нет ни одной похожей друг на друга снежинки. Дети соответствуют определенным типам, но не существует двух одинаковых детей, даже если это близнецы. Личность следует определенным типам, но она всегда уникальна.
\vs p111 4:7 \P\ Счастье и радость имеют своим источником внутреннюю жизнь. Сами по себе вы не можете испытать настоящую радость. Жизнь в одиночестве губительна для счастья. Даже семьи и народы получат большую радость, если разделят ее с другими.
\vs p111 4:8 \P\ Ты не можешь полностью контролировать внешний мир --- окружающую среду. Но лучше всего подчиняется твоему контролю творческая способность внутреннего мира, потому что именно твоя личность в столь значительной степени освобождена от законов априорной причинности. Существует ограниченное владычество воли, связанное с личностью.
\vs p111 4:9 Поскольку эта внутренняя жизнь человека является истинно творческой, на каждом индивидууме лежит ответственность выбора --- быть ли этой творческой способности спонтанной и совершенно случайной или же контролируемой, направляемой и конструктивной. Как может творческое воображение приносить достойные плоды, если поле его деятельности уже занято предубеждением, ненавистью, страхами, обидами, местью и фанатизмом?
\vs p111 4:10 Идеи могут возникать под воздействием стимулов внешнего мира, но идеалы рождаются только в творческих областях внутреннего мира. Сегодня народами мира руководят люди, у которых слишком много идей и чрезвычайно мало идеалов. Этим и объясняется бедность, разводы, войны и расовая ненависть.
\vs p111 4:11 Проблема состоит в том, что если человек, обладающий свободой воли, внутренне наделен творческими способностями, то тогда мы должны признать, что созидательная способность свободы воли заключает в себе потенциал разрушительной способности свободы воли. И если созидательная способность оборачивается способностью разрушать, вы оказываетесь лицом к лицу с губительным злом и грехом --- угнетением, войной и гибелью. Зло есть частица творческой способности, оно ведет к распаду и, в конце концов, к гибели. Всякий конфликт есть зло, потому что он подавляет творческую функцию внутренней жизни --- для личности он является своего рода гражданской войной.
\vs p111 4:12 \P\ Внутренняя творческая способность содействует облагораживанию характера благодаря интеграции личности и унификации индивидуальности. Всегда истинно изречение: прошлое нельзя изменить; только будущее можно изменить в результате служения существующей творческой способности внутреннего <<я>>.
\usection{5.\bibnobreakspace Посвящение выбора}
\vs p111 5:1 Следование воле Бога есть не больше и не меньше как проявление со стороны создания готовности разделить внутреннюю жизнь с Богом --- тем самым Богом, который сделал возможным для создания такую жизнь внутренних значений\hyp{}ценностей. Разделение Богоподобно, оно --- божественно. Бог разделяет все с Вечным Сыном и Бесконечным Духом, а они, в свою очередь, разделяют все с божественными Сынами и духовными Дочерьми вселенных.
\vs p111 5:2 Подражание Богу --- ключ к совершенству; следование его воле --- секрет продолжения существования в посмертии и совершенства в посмертии.
\vs p111 5:3 Смертные живут в Боге, и Бог так пожелал --- жить в смертных. Как люди поручают себя ему, так и он --- первым --- поручил части себя быть с людьми; он согласился жить в людях и пребывать в них по человеческой воле.
\vs p111 5:4 Мир в этой жизни, продолжение существования в посмертии, совершенство в следующей жизни и служение в вечности --- все это достигается (в духовной сфере) \bibemph{сейчас,} когда личность создания соглашается --- делает выбор --- подчинить волю создания воле Бога. А Отец уже сделал выбор, состоящий в том, чтобы подчинить фрагмент самого себя воле личности создания.
\vs p111 5:5 Такой выбор создания не есть отказ от воли. Он является посвящением воли, расширением воли, прославлением воли, совершенствованием воли; и такой выбор поднимает волю создания с уровня временного значимости к таком высшему состоянию, в котором личность тварного сына общается с личностью духовного Отца.
\vs p111 5:6 Этот выбор следования воле Отца есть духовное открытие смертным человеком Отца, который есть дух, даже если и эпоха должна пройти, прежде чем тварный сын сможет в действительности предстать в фактическом присутствии Бога в Раю. Этот выбор состоит не столько в отрицании воли создания: <<Не моя воля, но твоя да будет>>, --- сколько в положительном утверждении создания: <<\bibemph{Моя} воля в том, чтобы свершилась \bibemph{твоя} воля>>. И если этот выбор сделан, раньше или позже, сын, выбравший Бога, обретет внутреннее объединение (слияние) с пребывающим в нем фрагментом Бога, в то же время этот самый сын, идущий дорогой совершенства, обретет верховное личностное удовлетворение в богопочитании --- общении личности человека и личности его Творца, двух личностей, чьи творческие атрибуты навечно объединены в добровольно взаимном выражении --- в рождении вечного партнерства воли человека и воли Бога.
\usection{6.\bibnobreakspace Парадокс человека}
\vs p111 6:1 Многие временные несчастья смертного человека возникают из\hyp{}за его двойственной связи с космосом. Человек --- часть природы, он существует в природе, и все же он способен превзойти природу. Человек конечен, но в нем пребывает искра бесконечности. Такая двойственная ситуация не только обеспечивает возможность зла, но и порождает множество социальных и моральных ситуаций, чреватых большой неопределенностью и немалой тревогой.
\vs p111 6:2 Требуется мужество, чтобы одержать победу над природой и подняться над своим <<я>> --- это мужество, которое может поддаться искушениям гордыни. Тот смертный, который в состоянии подняться над своим <<я>>, может поддаться искушению обожествить свое собственное самосознание. Дилемма смертного заключается в двойственном факте: человек находится в зависимости от природы и в то же самое время он обладает уникальной свободой --- свободой духовного выбора и действия. На материальных уровнях человек оказывается подчиненным по отношению к природе, в то время как на духовных уровнях он торжествует над природой и над всеми временными и конечными вещами. Такой парадокс неотделим от искушения, потенциального зла, от ошибок в принятии решений, и когда <<я>> впадает в гордыню и самонадеянность, путь для греха открыт.
\vs p111 6:3 \P\ Проблема греха в конечном мире не существует сама по себе. Факт конечности не является злым или греховным. Конечный мир создан бесконечным Творцом --- он есть дело рук его божественных Сынов --- и, следовательно, должен быть \bibemph{благим.} Это злоупотребление, искажение и извращенность конечного порождают зло и грех.
\vs p111 6:4 \P\ Дух может властвовать над разумом; так и разум может контролировать энергию. Но разум может контролировать энергию только посредством своего собственного сознательного управления метаморфными потенциалами, присущими математическому уровню причин и следствий физических областей. Разум создания не обладает врожденной способностью контролировать энергию; это --- прерогатива Божества. Но разум создания может управлять энергией и делает это в той степени, в какой он овладевает секретами физической вселенной.
\vs p111 6:5 Когда человек хочет изменить физическую реальность, будь то он сам или окружающая его среда, он достигает успеха в той степени, в какой он открывает пути и средства контроля над материей и управления энергией. Без посторонней помощи разум не способен влиять на что\hyp{}либо материальное, если бы не его собственный физический механизм, с которым он неизбежно связан. Но посредством разумного использования телесного механизма разум может создавать другие механизмы, даже энергетические связи и живые связи, используя которые, этот разум может все больше и больше контролировать свой физический уровень во вселенной и даже господствовать над ним.
\vs p111 6:6 Наука есть источник фактов, и разум не может оперировать без фактов. Они являются строительными кирпичами при постройке здания мудрости, которые цементируются жизненным опытом. Человек может обрести любовь Бога без фактов, и человек может открыть законы Бога без любви, но человек никогда не оценит бесконечную симметрию, возвышенную гармонию и пронзительную наполненность включающей все в себя природы Первоисточника и Центра до тех пор, пока он не обрел божественную любовь и божественный закон и пока опытным путем не объединил их в своей собственной развивающейся космической философии.
\vs p111 6:7 Расширение материального знания допускает более высокую интеллектуальную оценку значений идей и ценностей идеалов. Человек может обрести истину в своем внутреннем опыте, но он нуждается в точном знании фактов, чтобы применить свое личное открытие истины к безжалостно практическим требованиям повседневной жизни.
\vs p111 6:8 \P\ Вполне естественно, что смертный человек должен испытывать тревогу и чувство незащищенности, когда он видит, что он нерасторжимо связан с природой и в то же время обладает духовной способностью, которая целиком превосходит все временные и конечные вещи. Только религиозная уверенность --- живая вера --- может поддержать человека в таких сложных и запутанных ситуациях.
\vs p111 6:9 \P\ Изо всех опасностей, которые окружают смертную природу человека и угрожают его духовной целостности, самой большой является гордыня. Мужество --- знак доблести, но эготизм --- знак тщеславия и самоубийства. Не должно порицать разумную уверенность в себе. Способность человека подняться над самим собой --- единственное, что выделяет его из животного царства.
\vs p111 6:10 \P\ Гордыня --- в ком бы она не обнаруживалась --- в отдельной личности, в группе, расе или нации, --- обманчива, опьяняюща и порождает грех. Воистину так: <<Гордыня предшествует грехопадению>>.
\usection{7.\bibnobreakspace Проблема Настройщика}
\vs p111 7:1 Неопределенность вместе с уверенностью --- суть пути к Раю --- неопределенность во времени и в разуме, неопределенность по отношению к событиям развертывающегося Райского восхождения; уверенность в духе и в вечности, уверенность в неограниченной вере тварного сына в божественное сострадание и бесконечную любовь Отца Всего Сущего; неопределенность неопытного гражданина вселенной; уверенность сына, восходящего во вселенских обителях всемогущего, всемудрого и вселюбящего Отца.
\vs p111 7:2 \P\ Можно ли мне просить тебя обратить внимание на отдаленное эхо призыва Настройщика к твоей душе? Пребывающий в тебе Настройщик не может остановить или даже существенно изменить твое продвижение, твою борьбу во времени; Настройщик не может уменьшить твои лишения на пути прохождения через этот мир тяжелого труда. Пребывающий в тебе Божественный Настройщик может лишь терпеливо оставаться в стороне, в то время как ты ведешь борьбу за жизнь, как это заведено на твоей планете; но ты мог бы, если ты только решишься на это, --- в то время, как ты работаешь и беспокоишься, борешься и тяжко трудишься --- позволить доблестному Настройщику бороться вместе с тобой и за тебя. Ты мог бы стать таким успокоенным и воодушевленным, таким зачарованным и заинтригованным, если бы ты только позволил Настройщику постоянно живописать тебе истинное побуждение, конечную цель и вечное предназначение всей этой трудной, тяжелой борьбы с обыденными проблемами твоего нынешнего материального мира.
\vs p111 7:3 Почему ты не помогаешь Настройщику в его задаче показать тебе духовную сторону всех этих напряженных материальных усилий? Почему ты не позволяешь Настройщику укрепить тебя духовными истинами космической мощи, в то время как ты борешься с временными трудностями тварного существования? Почему ты не поощряешь небесного помощника ободрить тебя ясным видением перспективы вселенской жизни, в то время как ты пристально вглядываешься в замешательстве в сиюминутные проблемы? Почему ты отказываешься от того, чтобы просветиться и воодушевиться, став на точку зрения вселенной, пока ты тяжко трудишься среди препятствий, создаваемых временем, и пытаешься выбраться из лабиринта сомнений, которые сопровождают твое продвижение по смертной жизни. Почему не позволить Настройщику одухотворить твои мысли, даже если твои ноги должны ступать по материальным тропам земных устремлений?
\vs p111 7:4 Высшие человеческие расы Урантии являются полностью смешанными; они --- смесь многих рас и племен различного происхождения. Из\hyp{}за того, что эта природа состоит из многих компонентов, Помощникам оказывается чрезвычайно трудно эффективно работать в течение жизни их подопечного, и это добавляет проблем и Настройщику, и серафиму\hyp{}хранительнице после его смерти. Не так давно я был в Спасограде и слышал, как хранительница предназначения представила официальное заявление в оправдание трудностей служения ее смертному подопечному. Этот серафим\hyp{}хранительница сказал:
\vs p111 7:5 \P\ <<Множество моих трудностей происходили вследствие нескончаемого конфликта между двумя типами природы моего подопечного: честолюбивым побуждениям противостояла животная инертность; идеалы высшего народа сталкивались с инстинктами низшей расы; высокие цели великого разума вступали в борьбу с побуждениями первобытной наследственности; широким взглядом дальновидного Помощника противодействовала близорукость создания времени; прогрессивные планы восходящего существа изменялись под воздействием желаний и страстей материальной природы; озарения вселенского ума подавлялись химико\hyp{}энергетическими установлениями развивающейся расы; порывам ангелов противопоставлялись эмоции животного; обучение интеллекта сводилось к нулю под воздействием инстинкта; опыту индивидуума противопоставлялись пристрастия, накопленные расой; направленность на самое лучшее заслонялась стремлением к самому худшему; взлет гения подавлялся тяжестью посредственности; прогресс добра сдерживался инерцией дурного; прекрасное искусство осквернялось присутствием злого начала; жизнерадостность здоровья подавлялась болезненной слабостью; колодец веры был заражен ядом страха; воды скорби делали горьким источник радости; надежду предвкушения разочаровывала горечь осуществления; радостям жизни всегда угрожали горести смерти. Такая жизнь на этой планете! И все же, благодаря всегда присутствующей помощи Настройщика Мысли и его понуждению, эта душа достигла\hyp{}таки в определенной степени счастья и успеха и даже поднялась теперь к судным чертогам обители>>.
\vs p111 7:6 [Представлено Одиночным Вестником Орвонтона.]
