\upaper{158}{Гора преображения}
\author{Комиссия срединников}
\vs p158 0:1 В пятницу 12 августа 29 года н.э. перед заходом солнца Иисус и его соратники пришли к подножью горы Ермон, примерно к тому месту, где однажды отрок Тиглат ожидал Учителя, пока тот в одиночестве совершал восхождение на гору, чтобы решить духовную судьбу Урантии, и особенно, чтобы положить конец восстанию Люцифера. Здесь они провели два дня, духовно готовя себя к событиям, которым предстояло произойти столь скоро.
\vs p158 0:2 В общем, Иисус заранее знал, что должно было произойти на горе, и очень желал, чтобы все его апостолы могли разделить этот опыт. Чтобы подготовить их к этому откровению о себе, он и задержался с ними у подножья горы. Но они не смогли достигнуть того духовного уровня, который бы позволил в полной мере изведать и пережить явление небесных существ, которым предстояло посетить землю в скором времени. А поскольку Иисус не мог взять с собой всех своих сподвижников, он решил взять только трех из них, тех, кто привык сопровождать его во всех подобных бдениях. Поэтому только Петр, Иаков и Иоанн пережили вместе с Учителем сей уникальный опыт.
\usection{1. Преображение}
\vs p158 1:1 Рано утром в понедельник 15 августа Иисус и трое апостолов начали восхождение на гору Ермон; это случилось через шесть дней после памятного признания Петра, которое он сделал в полдень возле дороги под тутовыми деревьями.
\vs p158 1:2 Иисус был призван взойти на гору в одиночку, чтобы решить важные вопросы, касавшиеся дальнейшего осуществления его пришествия во плоти, поскольку этот опыт касался вселенной, которую он сам сотворил. Особенно значимым было то, что необычайное событие было приурочено ко времени, когда Иисус и апостолы находились в землях, где жили неевреи, и что оно действительно совершилось на горе, которая была во владениях неевреев.
\vs p158 1:3 Незадолго до полудня они находились приблизительно на середине пути к вершине горы, и во время обеденной трапезы Иисус поведал трем апостолам кое\hyp{}что о том, что он пережил в горах к востоку от Иордана вскоре после своего крещения, и немного о том, что пережил на горе Ермон во время своего посещения этого уединенного места.
\vs p158 1:4 Мальчиком Иисус любил забраться на холм неподалеку от своего дома и воображать себе сражения между армиями империй на Ездрилонской равнине; теперь же он взошел на гору Ермон, чтобы получить те дарования, которые должны были подготовить его к тому, чтобы сойти в долины Иордана, где должен быть сыгран завершающий акт драмы его пришествия на Урантию. В этот день на горе Ермон Учитель еще мог отказаться от борьбы и вернуться к управлению своими вселенскими владениями, однако он не только решил следовать условиям, предъявляемым к его чину божественного сыновства и отмеченным в указе Вечного Сына в Раю, но и избрал в высшей и полной мере исполнить волю своего Райского Отца. В этот августовский день трое его апостолов увидели, как он отказался принять на себя всю полноту власти над вселенной. Они с удивлением смотрели, как удалились небесные вестники, оставив Иисуса одного завершать свою земную жизнь в качестве Сына Человеческого и Сына Божьего.
\vs p158 1:5 Во время насыщения пяти тысяч они верили особенно сильно, а потом их вера быстро сошла почти на нет. Теперь, вследствие признания Учителем своей божественности, вера двенадцати апостолов за несколько следующих недель усилилась и достигла высшей точки, но затем снова постепенно пошла на спад. Третье возрождение их веры произошло лишь после воскресения Учителя.
\vs p158 1:6 В этот прекрасный день в три часа после полудня Иисус, покидая троих апостолов, сказал: «Оставляю вас на время и иду общаться с Отцом и его вестниками; прошу вас оставаться здесь и, ожидая моего возвращения, молиться, чтобы воля Отца могла осуществиться во всех ваших деяниях, связанных с дальнейшей миссией пришествия Сына Человеческого». Сказав это апостолам, Иисус удалился на длительное совещание с Гавриилом и Отцом\hyp{}Мелхиседеком и не возвращался часов до шести. Увидев беспокойство апостолов, вызванное его долгим отсутствием, Иисус сказал: «Почему вы боитесь? Вы же хорошо знаете, что я должен быть в том, что принадлежит Отцу; почему же вы сомневаетесь, когда я не с вами? Ныне объявляю вам, что Сын Человеческий избрал пройти весь свой жизненный путь среди вас и как один из вас. Ободритесь; я не покину вас, пока не закончу свой труд».
\vs p158 1:7 Когда они вкушали от скромной вечерний трапезы, Петр спросил Учителя: «Как долго мы будем оставаться на этой горе вдали от братьев наших?» И Иисус ответил: «Пока не увидите славы Сына Человеческого и не узнаете, что все, провозглашенное мною вам, истинно». Сидя вокруг последних красных угольков костра, они говорили о бунте Люцифера, пока не сгустилась тьма и глаза апостолов не отяжелели, поскольку свое путешествие они начали ранним утром.
\vs p158 1:8 Проспав глубоким сном около получаса, трое апостолов вдруг проснулись от раздававшегося поблизости потрескивания и, осмотревшись, к своему огромному изумлению и испугу увидели Иисуса, ведущего доверительную беседу с двумя сверкающими существами, облаченными в одежды из света небесного мира. И лицо, и внешняя форма Иисуса сияли, небесным светом. Сии трое беседовали на неизвестном языке, но из определенных высказываний Петр ошибочно заключил, что с Иисусом были Моисей и Илия; в действительности же с ним были Гавриил и Отец\hyp{}Мелхиседек. По просьбе Иисуса физические контролеры устроили так, что апостолы стали свидетелями этой сцены.
\vs p158 1:9 Три апостола были так страшно испуганы, что не смогли понять, что происходит, однако когда ослепительное видение перед ними исчезло и они увидели Иисуса, стоявшего в одиночестве, Петр, первым пришедший в себя, сказал: «Иисус, Учитель, как хорошо, что мы тут были. Мы радуемся, видя славу сию. Мы не хотим возвращаться в бесславный мир. Если хочешь, останемся здесь и соорудим три шатра: один тебе, Моисею один, и один Илии». Сказал же это Петр потому, что был в смятении и именно в этот момент ничего другого не пришло ему на ум.
\vs p158 1:10 Когда Петр еще говорил, к ним приблизилось серебристое облако и осенило всех четырех. Апостолы исполнились великого страха и, пав в благоговении ниц, услышали, как глас, тот же самый глас, что глаголил во время крещения Иисуса, произнес: «Сей есть Сын мой возлюбленный; его слушайте». Когда же облако рассеялось, Иисус снова был один с тремя апостолами и, приступив, коснулся их и сказал: «Встаньте и не бойтесь; вы увидите еще более великое, чем это». Но апостолы были действительно испуганы; все трое молчали, погрузившись в задумчивость, пока готовились спускаться с горы до наступления полуночи.
\usection{2. Спуск с горы}
\vs p158 2:1 Приблизительно до середины спуска не было сказано ни единого слова. Разговор начал Иисус, заметив: «Смотрите, не рассказывайте никому, даже вашим собратьям, что видели и слышали на сей горе, пока Сын Человеческий не воскреснет из мертвых». Трое апостолов были потрясены и смущены словами Учителя: «пока Сын Человеческий не воскреснет из мертвых». Совсем недавно они вновь подтвердили свою веру в него как Избавителя, Сына Бога, только что своими глазами видели, как он преобразился во славе, и вот он начал говорить о «воскресении из мертвых»!
\vs p158 2:2 При мысли о смерти Учителя Петр содрогнулся --- мысль эта была невыносима --- и, опасаясь, как бы Иаков или Иоанн не задали какой\hyp{}нибудь неуместный вопрос, решил, что лучше всего направить беседу в иное русло и, не зная, о чем бы еще поговорить, высказал первую же мысль, пришедшую ему в голову: «Учитель, почему книжники говорят, что перед явлением Мессии сначала должен прийти Илия?» Иисус же, понимая, что Петр пытается избежать разговоров о его смерти и воскресении, ответил: «Это правда, сначала должен прийти Илия и приготовить путь для Сына Человеческого, которому надлежит много пострадать и в конце концов быть отвергнутым. Но говорю вам, что Илия уже пришел, и не приняли его, а поступили с ним, как хотели». Тогда три апостола поняли, что он говорил им об Иоанне Крестителе как Илии. Иисус знал: если они упорно продолжают считать его Мессией, то и Иоанн должен восприниматься ими как пророчимый Илия.
\vs p158 2:3 Иисус велел им молчать о виденном ими знамении славы, которую он обретет после воскресения, так как не хотел поддерживать их веру в то, что если они сейчас примут его как Мессию, то в нем в известной мере воплотится их ошибочное представление о пришествии избавителя, творящего чудеса. Хотя Петр, Иаков и Иоанн про себя размышляли об этом до воскресения Учителя, они никому ничего не рассказывали.
\vs p158 2:4 Пока они продолжали спускаться с горы, Иисус сказал им: «Вы не хотите принять меня как Сына Человеческого; поэтому я согласился, чтобы вы приняли меня сообразно вашим устоявшимся взглядам, но не заблуждайтесь: воля Отца моего должна восторжествовать. Если же вы решили действовать по своему усмотрению, вы должны приготовиться претерпеть много разочарований и пережить множество испытаний, однако уроков, которые я вам дал, должно хватить, чтобы с успехом преодолеть даже эти горести, которые вы сами и избрали».
\vs p158 2:5 Иисус взял с собой Петра, Иакова и Иоанна на гору преображения отнюдь не потому, что они во всех отношениях лучше других апостолов были подготовлены к тому, чтобы стать свидетелями случившегося, или же потому, что духовно они в большей степени были готовы к тому, чтобы быть наделенными такой редкой привилегией. Вовсе нет. Он хорошо знал, что к этому опыту из двенадцати апостолов духовно не был готов никто; поэтому и взял с собой только тех трех апостолов, в чьи обязанности входило сопровождать его, когда он желал уединиться, дабы наедине насладиться общением с Богом.
\usection{3. Смысл преображения}
\vs p158 3:1 То, чему стали свидетелями Петр, Иаков и Иоанн на горе преображения, было мимолетным видением небесного великолепия, которое явилось в тот богатый событиями день на горе Ермон. Преображение было событием, во время которого открылось:
\vs p158 3:2 \ublistelem{1.}\bibnobreakspace Одобрение Вечным Райским Матерью\hyp{}Сыном полноты пришествия воплощенной жизни Михаила на Урантии. Что касалось требований Вечного Сына, Иисус теперь получил уверение в их исполнении. Уверение в этом принес Иисусу Гавриил.
\vs p158 3:3 \ublistelem{2.}\bibnobreakspace Подтверждение о том, что Бесконечный Дух удовлетворен полнотой пришествия Михаила в подобии смертной плоти на Урантии. Вселенский представитель Бесконечного Духа, ближайший сподвижник Михаила в Спасограде и его вездесущая соратница, в данном случае говорила через Отца\hyp{}Мелхиседека.
\vs p158 3:4 \pc Иисус с радостью воспринял сие свидетельство об успехе его земной миссии, представленное вестниками Вечного Сына и Бесконечного Духа, но заметил, что Отец его не дал знак, что его пришествие на Урантию завершилось; лишь незримое присутствие Отца свидетельствовало через Персонализированного Настройщика: «Сей есть Сын мой возлюбленный; его слушайте». И это было сказано словами, чтобы их услышали и три апостола.
\vs p158 3:5 После этого небесного посещения Иисус постарался узнать волю своего Отца и решил довести пришествие в подобии смертной плоти до естественного конца. Именно в этом был смысл преображения для Иисуса. Для трех апостолов оно явилось тем событием, после которого Учитель вступил на завершающий этап своего земного пути как Сын Божий и Сын Человеческий.
\vs p158 3:6 После официальной встречи с Гавриилом и Отцом\hyp{}Мелхиседеком Иисус непринужденно общался с этими своими Сынами служения и говорил с ними о делах вселенной.
\usection{4. Мальчик\hyp{}эпилептик}
\vs p158 4:1 Во вторник утром перед завтраком Иисус и его спутники вернулись в лагерь апостолов. Подходя к лагерю, они увидели довольно много людей, окружавших апостолов, и вскоре уже могли расслышать долетавшие до них громкие возгласы спорщиков. В этой толпе человек в пятьдесят, включая девятерых апостолов, полемизировали две примерно равные по численности группы --- иерусалимские книжники и верующие ученики, которые шли вслед за Иисусом и его сподвижниками во время их путешествия из Магадана.
\vs p158 4:2 Хотя толпа спорила по многим вопросам, но в основном спор был о неком жителе Тивериады, пришедшем в поисках Иисуса днем раньше. У этого человека, Иакова из Сафеда, был единственный сын, мальчик лет четырнадцати, который жестоко страдал от эпилепсии. Кроме этой нервной болезни мальчиком овладел один из тех блуждающих, вредных и мятежных срединников, которые в то время присутствовали на земле и никому не подчинялись, так что юноша\hyp{}эпилептик был к тому же одержим демоном.
\vs p158 4:3 Почти две недели несчастный отец, младший чиновник Ирода Антипы, искал Иисуса на западных границах владений Филиппа, чтобы упросить его вылечить больного сына. Лагерь апостолов он нашел лишь около полудня того дня, когда Иисус с тремя их товарищами был на горе.
\vs p158 4:4 Девять апостолов были крайне удивлены и сильно взволнованы, когда этот человек и около сорока других людей, искавших Иисуса, неожиданно набрели на них. В момент прихода этой группы девять апостолов, или по крайней мере большинство из них, обуреваемые старыми искушениями, выясняли, кто из них в грядущем царстве будет самым великим, горячо обсуждая возможные должности, которые займет каждый из апостолов. Они просто не могли полностью избавиться от давно лелеемой ими мысли о материальной стороне миссии Мессии. И теперь, когда сам Иисус благосклонно отнесся к высказанному ими убеждению в том, что он, действительно, был Избавителем --- или по крайней мере признал факт своей божественности, --- что сейчас в момент разлуки с Учителем, могло быть естественнее разговоров об этих надеждах и амбициях, которые занимали в их сердцах первое место? Этими спорами они и были заняты, когда Иаков из Сафеда и люди, искавшие с ним Иисуса, натолкнулись на них.
\vs p158 4:5 Андрей сделал шаг вперед, чтобы поприветствовать этого отца и его сына, и спросил: «Кого вы ищите?» Иаков сказал: «Добрый человек, мне нужен ваш Учитель. Я ищу исцеления для моего больного сына. И хочу, чтобы Иисус изгнал беса, овладевшего моим чадом». И отец стал рассказывать апостолам, что его сын так сильно болен, что много раз из\hyp{}за этих жестоких припадков был на грани смерти.
\vs p158 4:6 Пока апостолы слушали, Симон Зилот и Иуда Искариот подошли к отцу ближе и сказали: «Мы можем исцелить его; тебе незачем ждать возвращения Учителя. Мы --- посланцы царства и более этого не скрываем. Иисус --- Избавитель, и нам даны ключи от царства». Тем временем Андрей и Фома совещались в стороне. Нафанаил же и остальные с изумлением наблюдали за происходящим; всех ошеломила неожиданная смелость, если не самоуверенность, Симона и Иуды. Тогда отец сказал: «Если дано вам вершить такие дела, то молю вас, скажите слова, которые снимут с чада моего это ярмо». Тогда Симон выступил вперед и, положив свою руку на голову ребенка, посмотрел ему прямо в глаза и приказал: «Выйди из него, ты, дух нечистый; во имя Иисуса повинуйся мне». Но с мальчиком случился сильнейший припадок; и книжники смеялись над апостолами, ставшими посмешищем, и разочарованные верующие терпели насмешки этих недоброжелателей.
\vs p158 4:7 Андрей был глубоко огорчен этой неблагоразумной попыткой и ее печальным исходом. Он отозвал апостолов в сторону для совещания и молитвы. Поразмыслив, и будучи уязвленным поражением и унижением, которое все они испытали, Андрей предпринял вторую попытку изгнать демона, но и его усилия оказались неудачными. Андрей честно признал поражение и попросил отца остаться переночевать с ними и дождаться возвращения Иисуса, сказав: «Возможно, такой выйдет, если только Учитель лично прикажет ему».
\vs p158 4:8 Итак, пока Иисус спускался с горы с веселыми и восторженными Петром, Иаковом и Иоанном, их девять собратьев так же, как и они, не спали, пребывая в смятении и удрученном состоянии униженных. Это были удрученные и подавленные люди. Но Иаков из Сафеда не отступил. Хотя никто не мог ему сказать, когда вернется Иисус, он решил дожидаться возвращения Учителя.
\usection{5. Иисус исцеляет мальчика}
\vs p158 5:1 Когда пришел Иисус, девять апостолов, приветствуя его, испытали огромное облегчение и сильно приободрились, увидев веселые и воодушевленные лица Петра, Иакова и Иоанна. Все бросились приветствовать Иисуса и своих трех братьев. Пока они обменивались приветствиями, к ним приблизилась толпа, и Иисус спросил: «О чем вы спорили, когда мы подходили к вам?» Однако прежде чем смущенные и униженные апостолы смогли ответить на вопрос Учителя, несчастный отец больного мальчика выступил вперед и, упав на колени к ногам Иисуса, сказал: «Учитель, у меня есть сын, единственное мое дитя, который одержим злым духом. Он не только вопиет в ужасе и с пеной у рта, как мертвый, падает во время припадков, но часто сей овладевший им злой дух повергает его в судороги, а иногда бросает в воду и даже в огонь. Скрежеща зубами, и с многочисленными ушибами дитя мое гибнет. Его жизнь хуже смерти; мать его и я опечалены сердцем и сокрушаемся духом. Вчера около полудня, ища тебя, я набрел на лагерь твоих учеников, и пока мы тебя ждали, они пытались изгнать этого демона, но не смогли. Сделаешь ли ты это для нас, исцелишь ли моего сына, Учитель?»
\vs p158 5:2 Выслушав этот рассказ, Иисус коснулся коленопреклоненного отца и, велев ему встать, испытующе посмотрел на апостолов, стоящих рядом. Затем Иисус сказал всем им: «О, поколение неверное и развращенное. Доколе буду терпеть вас? Доколе буду с вами? Когда наконец поймете, что дела веры совершаются не по велению неверия, полного сомнений?» Затем, обращаясь к смущенному отцу, Иисус сказал: «Приведи своего сына». И когда Иаков подвел мальчика к Иисусу, тот спросил: «Давно ли мальчик страдает этой болезнью?» Отец ответил: «От самого младенчества». И когда они говорили, с юношей случился страшный припадок и он упал перед ними, скрежеща зубами и с пеной у рта. После жестоких судорог он лежал как мертвый. Тогда отец снова пал на колени к ногам Иисуса и, умоляя Учителя, сказал: «Если можешь исцелить его, прошу тебя, сжалься над нами и избавь от этой болезни». Услышав эти слова, Иисус посмотрел на огорченное лицо отца и сказал: «Не сомневайся в силе любви Отца моего, но лишь в искренности и глубине веры твоей. Кто действительно верует, тому все возможно». И тогда Иаков произнес навсегда запечатлевшиеся слова веры, смешанной с сомнением: «Верую, Господи. Умоляю тебя, помоги моему неверию».
\vs p158 5:3 Услышав эти слова, Иисус сделал шаг вперед и, взяв мальчика за руку, сказал: «Делаю это согласно воле Отца моего и во славу веры живой. Сын мой, встань! Дух непокорный, выйди из него и больше не возвращайся». И, вложив руку мальчика в руку отца, сказал: «Ступайте своей дорогой. Отец исполнил желание души твоей». И все, кто присутствовал при этом, даже враги Иисуса, изумились увиденному.
\vs p158 5:4 Трое апостолов, недавно испытавших восторг от увиденного и пережитого в момент преображения, были поистине разочарованы столь скорым возвращением к сей картине поражения и смущения своих собратьев\hyp{}апостолов. Однако с этими двенадцатью посланцами царства так было постоянно. Возвышение на их жизненном поприще часто сменялось унижением.
\vs p158 5:5 Действительно произошло исцеление от двух болезней: телесного заболевания и духовного недуга. И с того часа мальчик выздоровел навсегда. Когда Иаков и его вернувшийся к жизни сын ушли, Иисус сказал: «Сейчас идем в Кесарию Филиппову. Быстро собирайтесь». И они молча пошли на юг, и толпа последовала за ними.
\usection{6. В саду Кельса}
\vs p158 6:1 На ночь они остановились у Кельса; и в тот же вечер после трапезы и отдыха, когда двенадцать апостолов собрались вокруг Иисуса в саду, Фома сказал: «Учитель, хотя нам, оставшимся ожидать вас, до сих пор неведомо, что произошло на горе и чему столь великой радостью радовались наши братья, бывшие с тобой, мы просим тебя поговорить с нами о нашей неудаче и дать наставления об этом, тем более что произошедшее на горе не может быть сейчас открыто нам».
\vs p158 6:2 И Иисус, отвечая Фоме, сказал: «Все, что слышали ваши братья, в свое время откроется и вам. Однако я покажу вам, в чем причина неудачи того, что вы столь неблагоразумно попытались сделать. Вчера, пока Учитель ваш со своими спутниками, собратьями вашими, восходил вон на ту гору, дабы полнее узнать волю Отца и испросить более щедрого дара мудрости для наилучшего исполнения этой божественной воли, вы, оставшиеся здесь ожидать нас с наставлением, как обрести духовное понимание, и вместе с нами молиться о более полном откровении воли Отца, не сумели проявить имеющуюся в вас веру, но вместо этого впали в искушение, поддавшись своим старым пагубным наклонностям искать для себя привилегированных мест в царствие небесном --- в материальном и временном царстве, в ожидании которого вы упорствуете. И таких ошибочных представлений вы придерживаетесь, несмотря на неоднократные заявления о том, что царство мое не от мира сего.
\vs p158 6:3 Как только ваша вера начинает постигать личность Сына Человеческого, ваше эгоистическое желание мирского возвышения возвращается и вы снова рассуждаете между собой о том, кто будет большим в царстве небесном, в царстве, которого в том виде, в каком вы себе его представляете, не существует и не будет существовать никогда. Разве не говорил я вам, что тот, кто хочет стать большим в царстве духовного братства Отца моего, должен стать меньшим в своих собственных глазах, тем самым сделавшись слугою братьев своих? Духовное величие заключается в понимающей любви, которая суть Богоподобна, а не в наслаждении от проявления материальной силы ради собственного возвеличивания. В том, что вы пытались совершить и в чем потерпели столь полную неудачу, ваша цель была не чиста. Побуждение ваше не было божественным. Ваш идеал не был духовным. Ваше стремление не было альтруистичным. Ваши действия были основаны не на любви, и цель, которую вы пытались достигнуть, вовсе не была волей Отца Небесного.
\vs p158 6:4 Сколько времени потребуется, чтобы вы наконец поняли, что вам не дано ускорять ход установившихся природных явлений, кроме тех случаев, когда на то есть воля Отца? Что не дано вам вершить духовные дела при отсутствии духовной силы? И что ничего подобного вы делать не можете, даже при наличии потенциальной на то возможности, если нет третьего необходимого человеческого фактора, личного опыта обладания живой верой? Неужели для стремления к духовным реалиям царства вам всегда будут нужны материальные проявления? Неужели без зримого представления необычных деяний вы не можете осознать духовную суть моей миссии? Когда же можно будет рассчитывать на то, что вы останетесь верными высшим и духовным реалиям царства, несмотря на внешнюю видимость всех материальных проявлений?»
\vs p158 6:5 Сказав это двенадцати апостолам, Иисус добавил: «А теперь идите спать, ибо завтра мы вернемся в Магадан и там обсудим нашу миссию в городах и селениях Десятиградия. Подводя же итоги пережитого нами за день, позвольте мне каждому из вас объявить сказанное братьям вашим на горе; да укоренятся слова эти глубоко в сердцах ваших: ныне Сын Человеческий вступает на последний этап своего пути. Мы подошли к тому, чтобы приступить к трудам, которые в конечном итоге приведут к великому и окончательному испытанию веры и преданности вашей, когда я буду предан в руки людей, ищущих моей смерти. Запомните, что я говорю вам: Сын Человеческий будет убит, но воскреснет».
\vs p158 6:6 Апостолы легли спать с печальными мыслями. Они были в смятении и не могли понять сказанных слов. Хотя они не решались что\hyp{}либо спрашивать о том, что сказал Иисус, после его воскресения они вспомнили все.
\usection{7. Возражение Петра}
\vs p158 7:1 В среду ранним утром Иисус и двенадцать апостолов покинули Кесарию Филиппову и отправились в Магаданский лес близ Вифсаиды\hyp{}Юлии. Той ночью апостолы почти не спали, поэтому встали рано и были готовы идти. Даже флегматичные близнецы Алфеевы, и те были потрясены словами о смерти Иисуса. Направляясь на юг, они сразу за водами Мером подошли к дороге на Дамаск, и Иисус, желая избежать встречи с книжниками и другими людьми, которые, как он знал, вскоре пойдут следом за ними, решил идти в Капернаум по дороге на Дамаск, проходящей через Галилею. Поступил же он так, поскольку знал, что те, кто следовал за ним, пойдут по дороге вдоль восточного берега Иордана, ибо они считали, что Иисус и апостолы побоятся идти по землям Ирода Антипы. Иисус старался избежать встречи со своими недругами и толпой, которая следовала за ним, так как хотел в тот день побыть наедине с апостолами.
\vs p158 7:2 Они шли по Галилее и остановились в тени подкрепиться, когда было далеко за полдень. После еды Андрей, обращаясь к Иисусу, сказал: «Учитель, мои собратья не понимают твоих глубокомысленных высказываний. Мы до конца уверовали в то, что ты --- Сын Бога, и вот теперь мы слышим странные слова о том, что ты покинешь нас, слова о смерти. Мы не понимаем их смысла. Может быть, ты говоришь с нами притчами? Мы просим тебя, скажи нам обо всем прямо и без иносказаний».
\vs p158 7:3 Отвечая Андрею, Иисус сказал: «Братья мои, оттого, что вы признались, что считаете меня Сыном Бога, я и вынужден начать открывать вам истину о конце земного пришествия Сына Человеческого. Вы упорно придерживаетесь веры в то, что я Мессия, и не желаете расстаться с мыслью о том, что Мессия должен восседать на престоле в Иерусалиме; вот почему я не перестаю говорить вам, что Сыну Человеческому вскоре надлежит идти в Иерусалим, много пострадать, быть отверженным книжниками, старейшинами и первосвященниками, а затем быть убитым и воскреснуть из мертвых. Не притчу я рассказываю вам, а говорю истину, чтобы вы были готовы к этим событиям, когда они внезапно нагрянут на нас». И когда он еще говорил, Симон Петр стремительно бросился к нему, положил руку на плечо Учителя и сказал: «Учитель, мы не смеем спорить с тобой, но я объявляю во всеуслышание, что ничего подобного с тобой никогда не случится».
\vs p158 7:4 Говорил же так Петр, потому что любил Иисуса; однако человеческая природа Учителя распознала в этих словах доброжелательной любви едва уловимый намек на искушение изменить своей устремленности довести земное пришествие до конца, согласно воле своего Райского Отца. А поскольку он распознал опасность, которая заключалась в том, что советы пусть даже любящих и верных ему друзей, могли сбить его с выбранного пути, то, обращаясь к Петру и другим апостолам, сказал: «Отойди от меня. От тебя исходит дух врага, искусителя. Когда ты говоришь так, ты не на моей стороне а на стороне врага нашего. И таким образом превращаешь любовь ко мне в камень преткновения, мешающий мне исполнить волю Отца. Думай не о путях человеческих, но о воле Бога».
\vs p158 7:5 После того, как апостолы оправились от первого потрясения, вызванного резким упреком Иисуса, и перед тем, как продолжили свой путь, Иисус сказал им еще: «Если кто хочет идти за мною, пусть перестанет думать о себе, пусть каждый день исполняет свои обязанности и следует за мной. Ибо кто хочет из любви к себе сберечь свою жизнь, тот ее потеряет, а кто потеряет жизнь свою ради меня и евангелия, тот спасет ее. Какая польза человеку, если он приобретет весь мир, а душу свою потеряет? Что даст человек в обмен за жизнь вечную? Не стыдитесь меня и слов моих в этом грешном и лицемерном поколении, как и я не устыжусь вас, когда во славе явлюсь пред Отцом моим в присутствии всех воинств небесных. Тем не менее многие из вас, ныне стоящих предо мною, не познают смерти, пока не увидят царство Божье, пришедшее в силе».
\vs p158 7:6 Таким образом Иисус разъяснил двенадцати апостолам мучительный и противоречивый путь, по которому они должны были пройти, если хотят следовать за ним. Каким потрясением стали такие слова для этих галилейских рыбаков, которые упорно продолжали грезить о земном царстве с почетным положением для самих себя! Однако этот решительный призыв взволновал их верные сердца, и ни один из них не захотел покинуть Учителя. Иисус не посылал их на битву одних, а вел за собой. Он просил их лишь об одном --- смело следовать за ним.
\vs p158 7:7 Двенадцать апостолов постепенно стали осознавать, что Иисус говорил им о возможности своей смерти. Они лишь смутно поняли сказанное им о своей смерти, а его утверждение о воскресении из мертвых вообще не запечатлелось в их сознании. Со временем Петр, Иаков и Иоанн, вспоминая о пережитом ими на горе преображения, пришли к более полному пониманию некоторых из этих вопросов.
\vs p158 7:8 За все время общения двенадцати апостолов со своим Учителем они лишь несколько раз видели такой горящий взор и слышали такие гневные упреки, какие в этот раз были сказаны Петру и остальным. Иисус был всегда терпим к их человеческим недостаткам, но не тогда, когда он сталкивался с тем, что угрожало его стремлению всецело исполнить волю Отца в конце своего земного пути. Апостолы были буквально ошеломлены, изумлены и испуганы. Они не находили слов, чтобы выразить свою печаль. Но они начали постепенно понимать, что предстояло вынести Учителю и что через эти испытания им надо пройти вместе с ним, однако реальность грядущих событий они осознали лишь спустя много времени после того, как услышали эти первые намеки о надвигавшейся трагедии его последних дней.
\vs p158 7:9 По дороге через Капернаум Иисус и двенадцать апостолов молча шли к своему лагерю в Магаданском лесу. В течение дня они не беседовали с Иисусом, но много говорили между собой, в то время как Андрей разговаривал с Учителем.
\usection{8. В доме Петра}
\vs p158 8:1 В сумерки они вошли в Капернаум и по улицам, где обычно бывало мало народа, пошли прямо к дому Симона Петра поужинать. Пока Давид Зеведеев готовился переправить их через озеро, они расположились в доме Петра, и Иисус, глядя на него и других апостолов, спросил: «О чем вы так серьезно говорили между собой сегодня по пути?» Апостолы замолчали, потому что многие из них продолжали начатое у горы Ермон обсуждение того, какое положение они займут в будущем царстве; кто будет большим и т.д. Иисус же, зная, что занимало их мысли в тот день, подозвал одного из детей Петра и, поставив ребенка посреди них, сказал: «Истинно, истинно говорю вам, если только не обратитесь и не станете, как это дитя, мало достигнете в царстве небесном. Кто же умалится и станет, как малый сей, тот станет большим в царстве небесном. И кто принимает такое дитя, тот меня принимает. И кто принимает меня, тот принимает Пославшего меня. Если хотите быть первыми в царстве, старайтесь донести эти благие истины братьям вашим во плоти. Но кто соблазнит одного из малых сих, тому лучше было бы, если бы повесили ему камень на шею и бросили в море. Если то, что вы делаете руками вашими, или то, что вы видите глазами вашими, мешает распространению царства, пожертвуйте этими возлюбленными идолами, ибо лучше войти в царство без многого из того, что вы возлюбили в жизни, чем прилепиться к этим идолам и оказаться вне царства. Более же всего смотрите, не презирайте ни одного из малых сих, ибо ангелы их всегда видят лица воинств небесных».
\vs p158 8:2 Когда Иисус кончил говорить, они вошли в лодку и поплыли через озеро к Магадану.
