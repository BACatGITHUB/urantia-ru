\upaper{96}{Яхве --- Бог евреев}
\author{Мелхиседек}
\vs p096 0:1 Строя свое представление о Божестве, человек вначале включает в него всех богов, затем всех чужих богов подчиняет божеству своего племени и, наконец, исключает всех богов, кроме одного Бога окончательной и верховной ценности. Евреи всех богов синтезировали в свое наиболее возвышенное представление о Господе, Боге Израиля. Точно так же индусы объединили свои разнообразные божества в «единое духовное начало богов», изображенное в Риг\hyp{}Ведах, а жители Месопотамии своих богов свели к более централизованному представлению о Бел\hyp{}Мардуке. Эти монотеистические идеи стали созревать во всем мире вскоре после появления Махивенты Мелхиседека в Салиме в Палестине. Однако представление Мелхиседека о Божестве было непохожим на понятие эволюционной философии включения, подчинения и исключения; оно было основано исключительно на \bibemph{творческой силе} и весьма скоро оказало влияние на высшие представления о божестве жителей Месопотамии, Индии и Египта.
\vs p096 0:2 \pc Религия Салима почиталась кенитами и несколькими другими ханаанскими племенами как традиция. И среди целей воплощения Мелхиседека была такая: религия единого Бога должна укрепиться настолько, чтобы подготовить путь для земного пришествия Сына этого единого Бога. Ведь Михаил едва ли мог прийти на Урантию, пока там не существовало народа, верующего в Отца Всего Сущего, народа, в среде которого он бы мог явиться.
\vs p096 0:3 Религия Салима сохранялась у кенитов в Палестине как символ их веры, и эта религия, в том виде, в каком ее позднее заимствовали иудеи, сначала подверглась влиянию нравственных учений египтян, позднее --- вавилонской теологической мысли и, наконец, иранских представлений о добре и зле. Фактически иудейская религия основана на договоре между Авраамом и Махивентой Мелхиседеком; в эволюционном плане она --- порождение множества уникальных ситуативных обстоятельств, однако в культурном плане она многое позаимствовала в религии, морали и философии всего Леванта. Именно через еврейскую религию многое из морали и религиозной мысли Египта, Месопотамии и Ирана и передалось народам Запада.
\usection{1. Представления о Божестве у семитов}
\vs p096 1:1 Древние семиты считали, что дух пребывает во всем. Для них существовали духи животного и растительного миров; духи года, повелитель потомства; духи огня, воды и воздуха; целый пантеон духов, которых следовало боятся и почитать. И учения Мелхиседека о Творце Всего Сущего до конца так и не разрушило веру в эти подчиненные духи или в богов природы.
\vs p096 1:2 Переход иудеев от политеизма через генотеизм к монотеизму не был неразрывным и непрерывным концептуальным развитием. В эволюции своих представлений о Божестве они пережили множество отступлений назад, в то время в любую эпоху у различных групп верующих семитов существовали разные представления о Боге. Время от времени применительно к их представлениям о Боге использовались многочисленные названия, и во избежание путаницы дадим этим различным именам Божества определения в соответствии с их местом в эволюции еврейской теологии.
\vs p096 1:3 \ublistelem{1.}\bibnobreakspace \bibemph{Яхве} был богом южно\hyp{}палестинских племен, которые связывали это понятие о божестве с горой Хорив, Синайским вулканом. Яхве был всего лишь одним из сотен и тысяч богов природы, которые привлекали внимание и требовали почитания семитских племен и народов.
\vs p096 1:4 \pc \ublistelem{2.}\bibnobreakspace \bibemph{Эл\hyp{}Элион.} На протяжении веков после пребывания Мелхиседека в Салиме \bibemph{его} учение о Божестве сохранялось в различных версиях, однако, как правило, обозначалось словом Эл\hyp{}Элион, что означает Всевышний Бог небес. Многие семиты, включая и прямых потомков Авраама, в разные времена поклонялись и Яхве, и Эл\hyp{}Элиону.
\vs p096 1:5 \pc \ublistelem{3.}\bibnobreakspace \bibemph{Эл\hyp{}Шаддаи.} Трудно объяснить, что означало Эл\hyp{}Шаддаи. Это представление о Боге было смешанным и происходило от учений Аменемопа в «Книге Мудрости», модифицированных учением Эхнатона об Атоне и подвергшихся дальнейшему влиянию со стороны учений Мелхиседека, заключенных в понятии об Эл\hyp{}Элионе. Однако когда понятие об Эл\hyp{}Шаддаи глубоко проникло в умы иудеев, оно оказалось сильно окрашенным верованиями в Яхве, распространенными среди жителей пустыни.
\vs p096 1:6 Одной из идей, господствующих в религии этой эпохи, было представление египтян о божественном Провидении, учение о том, что материальное процветание является наградой за служение Эл\hyp{}Шаддаи.
\vs p096 1:7 \pc \ublistelem{4.}\bibnobreakspace \bibemph{Эл.} При всей путанице в терминологии и неясности понятий многие благочестивые верующие честно старались поклоняться всем этим развивавшимся представлениям о божественном, в результате чего и возник обычай обращаться к сему смешанному Божеству как к Элу. Причем это название заключало в себе и многих других бедуинских богов природы.
\vs p096 1:8 \pc \ublistelem{5.}\bibnobreakspace \bibemph{Элохим.} В Кише и Уре долго сохранялись шумерско\hyp{}халдейские группы, которые учили основанному на традициях времен Адама и Мелхиседека представлению о триедином Боге. Эта доктрина была перенесена в Египет, где такой Троице поклонялись, называя ее Элохим, или в единственном числе --- Элоа. Философские круги Египта и более поздние александрийские учителя иудейского происхождения учили сему единству множественных Богов, и многие из советников Моисея во время исхода верили в эту Троицу. Однако понятие о тройственном Элохиме по\hyp{}настоящему так и не стало частью иудейской теологии до тех пор, пока евреи не подпали под политическое влияние вавилонян.
\vs p096 1:9 \pc \ublistelem{6.}\bibnobreakspace \bibemph{Разные имена.} Семиты не любили произносить имя своего Божества, поэтому время от времени прибегали к использованию таких многочисленных имен, как Дух Бога, Господь, Ангел Господень, Всемогущий, Святой, Всевышний, Адонаи, Древний Дней, Господь Бог Израиля, Творец неба и земли, Кириос, Иах, Господь Саваоф и Отец небесный.
\vs p096 1:10 \pc \bibemph{Иегова ---} это термин, который использовался в последнее время для обозначения завершенного представления о Яхве, в конечном итоге сформировавшегося за долгую историю иудеев. Однако имя Иегова вошло в употребление лишь спустя полторы тысячи лет после времен Иисуса.
\vs p096 1:11 \pc Вплоть до второго тысячелетия до нашей эры гора Синай была периодически активным вулканом, отдельные извержения которого случались и во время пребывания израильтян в этом месте. Огонь и дым в сочетании с оглушительными сотрясениями, связанными с извержением сей вулканической горы, --- все это впечатляло и устрашало бедуинов окружающих мест, внушая им великий страх перед Яхве. Позднее сей дух горы Хорив стал богом семитов\hyp{}иудеев, и они, в конце концов, стали считать его верховным над всеми остальными богами.
\vs p096 1:12 Жители Ханаана давно поклонялись Яхве, и хотя многие из кенитов в большей или меньшей степени верили в Эл\hyp{}Элиона, главного бога салимской религии, большинство жителей Ханаана в той или иной степени продолжали поклоняться старым божествам своих племен. Они едва ли хотели отказываться от своих национальных божеств ради международного, если не сказать межпланетарного, Бога. И не были склонны верить во всемирное божество, а потому эти племена продолжали поклоняться своим племенным божествам, включая и Яхве, а также серебряного и золотого тельцов, которые символизировали представления пастухов\hyp{}бедуинов о духе Синайского вулкана.
\vs p096 1:13 Поклоняясь своим богам, сирийцы верили и в Яхве иудеев, ибо их пророки сказали сирийскому царю: «Их боги есть боги гор; поэтому они были сильнее нас; но дайте нам сразиться с ними на равнине, и мы обязательно будем сильнее их».
\vs p096 1:14 По мере развития человеческой культуры младшие боги подчинялись верховному божеству, и обращение к великому Юпитеру сохранилось лишь как восклицание. Монотеисты считают своих подчиненных богов духами, демонами, парками, нереидами, эльфами, домовыми, гномами, привидениями\hyp{}плакальщицами и дурным глазом. Иудеи прошли через генотеизм и долго верили в существование иных, нежели Яхве, богов, однако они все больше считали, что эти чужие божества подчинены Яхве. Евреи допускали реальность Чемоша, бога Амореев, но утверждали, что он подчинен Яхве.
\vs p096 1:15 Из всех теорий, созданных смертными о Боге, идея Яхве претерпела наибольшее развитие. Ее постепенную эволюцию можно сравнить лишь с метаморфозой представления о Будде в Азии, которое в конечном итоге привело к понятию о Вселенском Абсолюте так же, как представление о Яхве в конце концов привело к идее об Отце Всего Сущего. Однако в историческом плане следует понимать, что, хотя евреи и изменили свое воззрение на Божество с представления о племенном боге горы Хорив на представление о любящем и милосердном Отце\hyp{}Творце более позднего времени, они не изменили его имени и продолжали это совершенствовавшееся представление о Божестве называть Яхве.
\usection{2. Семитские народы}
\vs p096 2:1 Семиты Востока были хорошо организованной конницей с хорошим руководством, которые захватили восточные районы плодородных земель и объединились там с вавилонянами. Халдеи, жившие близ Ура, были одними из самых развитых восточных семитов. Финикийцы же были высшей и хорошо организованной группой смешанных семитов, которые удерживали западную часть Палестины вдоль берега Средиземного моря. В расовом отношении семиты относились к самым смешанным народам Урантии, которые обладали наследственными факторами почти всех девяти рас мира.
\vs p096 2:2 Семиты\hyp{}арабы снова и снова с боем прокладывали себе дорогу на север Обетованной земли, в страну, где «течет молоко и мед», но лучше организованные и обладавшие более высокой цивилизацией северные семиты и хеттеи столь же часто изгоняли их прочь. Позднее, во время необычайно жестокого голода, эти кочующие бедуины в большом количестве пришли в Египет в качестве наемников на египетских общественных работах, но оказались в положении простых и бесправных тружеников долины Нила, подвергающихся горькому опыту порабощения тяжелым ежедневным трудом.
\vs p096 2:3 Лишь по истечении дней Махивенты Мелхиседека и Авраама некоторые племена семитов за свои необычные религиозные верования были названы детьми Израиля, а позднее иудеями, евреями и «избранным народом». Авраам не был расовым отцом всех иудеев; он не был даже прародителем всех бедуинов\hyp{}семитов, которых удерживали в плену в Египте. Верно, его потомки, выйдя из Египта, сформировали ядро более позднего еврейского народа, однако подавляющее большинство мужчин и женщин, объединившихся в кланы Израиля, в Египте никогда не были. Они были просто кочевниками, решившими подчиниться руководству Моисея, когда дети Авраама и их соратники\hyp{}семиты шли из Египта через северную Аравию.
\vs p096 2:4 \pc Учение Мелхиседека об Эл\hyp{}Элионе, Всевышнем, и завете о божественной милости через веру ко времени порабощения египтянами семитских народов, которым вскоре предстояло сформировать иудейскую нацию, было во многом забыто. Однако на протяжении сего периода плена эти аравийские кочевники сохраняли оставшуюся у них традиционную веру в Яхве как божество своей расы.
\vs p096 2:5 Яхве поклонялось более ста отдельных арабских племен, и, если не считать составляющую понятия Мелхиседека об Эл\hyp{}Элионе, которое сохранялось у наиболее образованных классов Египта, включая и смешанные семьи евреев и египтян, религия рядовых пленных рабов\hyp{}евреев была видоизмененной версией старого ритуала магии и жертв, приносимых Яхве.
\usection{3. Несравненный Моисей}
\vs p096 3:1 Начало эволюции иудейских представлений и идеалов Верховного Творца восходит к исходу семитов из Египта под руководством великого вождя, учителя и организатора Моисея. Его мать принадлежала к царской семье Египта, а отец был чиновником\hyp{}семитом, ведающим связью между правительством и пленными бедуинами. Таким образом, Моисей обладал качествами, полученными из высших расовых источников; его предки были настолько смешанными, что его невозможно отнести ни к одной расовой группе. Если бы Моисей не был человеком такого смешанного типа, то никогда не проявил бы той необыкновенной разносторонности и умения приспосабливаться, которые позволили ему управлять разноликой массой, в конечном итоге примкнувшей к тем бедуинам\hyp{}семитам, что под его руководством бежали из Египта в Аравийскую пустыню.
\vs p096 3:2 Несмотря на искушения культуры Нильского царства, Моисей решил разделить свой жребий с народом своего отца. Во время, когда сей великий организатор составлял свои планы окончательного освобождения народа своего отца, пленники\hyp{}бедуины едва ли имели религию, достойную названия; фактически у них не было истинного представления о Боге и надежды в этом мире.
\vs p096 3:3 \pc Ни один лидер никогда не пытался преобразовать и возвысить более жалкую, подавленную, угнетенную и невежественную группу человеческих существ. Однако наследственные качества этих рабов содержали скрытые возможности развития, а для создания отряда умелых организаторов существовало достаточное количество образованных лидеров, воспитанных Моисеем при подготовке ко дню восстания и сражению за свободу. Эти лучшие люди использовались в качестве надсмотрщиков над своим народом и благодаря влиянию Моисея на египетских правителей получили некоторое образование.
\vs p096 3:4 Моисей пытался вести дипломатические переговоры о свободе своих собратьев\hyp{}семитов. Он и его брат достигли с царем Египта договора, который давал им возможность мирно покинуть долину Нила и уйти в Аравийскую пустыню. Иудеи должны были получить скромную плату деньгами и вещами за свою долгую службу в Египте. Со своей стороны, они согласились поддерживать дружественные отношения с фараонами и не вступать ни в какие военные союзы против Египта. Однако позднее царь счел возможным отречься от этого договора, выдвинув в качестве извиняющей его причины то, что его шпионы обнаружили измену в среде рабов\hyp{}бедуинов. Он утверждал, что те добивались свободы, чтобы уйти в пустыню и организовать нападение кочевников на Египет.
\vs p096 3:5 Но Моисей не отчаивался; он дождался подходящего момента, и менее чем через год, когда армии Египта были всецело заняты, одновременно отражая сильные удары ливийцев с юга и греческое морское вторжение с севера, этот бесстрашный организатор вывел своих соотечественников из Египта, совершив эффектный ночной побег. Этот рывок к свободе был тщательно спланирован и искусно осуществлен. И он евреям удался, несмотря на то, что их отчаянно преследовал фараон с небольшим отрядом египтян, которые все погибли от руки оборонявшихся беглецов, оставив много трофеев, к которым добавилось и то, что награбили многочисленные беглые рабы, пока шли вперед к дому своих предков в пустыне.
\usection{4. Провозглашение Яхве}
\vs p096 4:1 Эволюционируя и возвысившись учение Моисея оказало влияние почти на половину всего мира и продолжает делать это даже в двадцатом веке. Хотя Моисей был хорошо знаком с более передовой религиозной философией Египта, рабы\hyp{}бедуины мало знали о подобных учениях, зато никогда и ни в коей мере не забывали бога горы Хорив, которого их предки называли Яхве.
\vs p096 4:2 Моисей слышал об учениях Махивенты Мелхиседека и от отца, и от матери, причем сходство их религиозных убеждений объясняло необычный союз между женщиной царского происхождения и мужчиной, принадлежащим к порабощенной расе. Тесть Моисея был кенитом\hyp{}поклонником Эл\hyp{}Элиона, однако родители освободителя верили в Эл\hyp{}Шаддаи. Моисей, таким образом, был воспитан как почитатель Эл\hyp{}Шаддаи; благодаря влиянию своего тестя стал поклонником Эл\hyp{}Элиона и ко времени расположения евреев лагерем у горы Синай после бегства из Египта сформулировал новое и расширенное понимание Божества (проистекающее из всех его прежних верований), которое мудро решил провозгласить своему народу как расширенное представление о древнем племенном боге Яхве.
\vs p096 4:3 Моисей пытался научить этих бедуинов представлению об Эл\hyp{}Элионе, однако перед уходом из Египта пришел к убеждению, что те никогда не поймут его учение до конца. Поэтому Моисей сознательно пошел на компромиссное решение, приняв их племенного бога пустыни в качестве единого и единственного бога своих последователей. Специально Моисей не учил тому, что иные народы и нации не могут иметь других богов, но решительно утверждал, что Яхве превыше всех богов --- особенно для иудеев. Однако Моисея всегда беспокоило затруднительное положение, в котором он оказывался, пытаясь представить этим невежественным рабам свое новое и более высокое понимание Божества, скрываемого под личиной древнего слова Яхве, которое всегда символизировал золотой телец племен бедуинов.
\vs p096 4:4 \pc То, что Яхве был богом беглых иудеев, и объясняет, почему они так долго пребывали перед святой горой Синай и почему получили там десять заповедей, которые Моисей провозгласил во имя Яхве, бога Хорива. Во время сего продолжительного пребывания перед Синаем религиозные обряды впервые формировавшегося иудейского поклонения подверглись дальнейшему совершенствованию.
\vs p096 4:5 \pc Видимо, Моисей никогда бы не добился успеха в учреждении своего отчасти более передового ритуального поклонения и сохранении единства своих последователей на протяжении четверти века, если бы не страшное извержение Хорива, происходившее в течение третьей недели их пребывания и поклонения у ее подножья. «Гора Яхве была охвачена огнем, и восходил от нее дым, как дым из печи, и вся гора сильно колебалась». Ввиду этого катаклизма неудивительно, что Моисей смог внушить своим собратьям учение, согласно которому их Бог был «могучим, ужасным, пожирающим огнем, страшным и всесильным».
\vs p096 4:6 Моисей провозгласил Яхве Господом, Богом Израиля, который отметил иудеев в качестве своего избранного народа; он строил новую нацию и мудро придал своим религиозным учениям национальный характер, говоря своим последователям, что Яхве ставит перед ними трудную задачу, что он «Бог ревнитель». Тем не менее, он старался расширить их представление о божественном, когда учил, что Яхве --- «Бог духов всякой плоти», и когда говорил: «Вечный Бог --- твое прибежище, и ты под мышцами вечными» Моисей учил, что Яхве --- Бог, соблюдающий завет; что он «не оставит тебя и не погубит тебя и не забудет завет отцам твоим, потому что Господь любит тебя и не забудет клятву, которою клялся отцам твоим».
\vs p096 4:7 Представив Яхве «Богом истинным и без порока, справедливым и праведным на всех путях своих», Моисей совершил героическое усилие, дабы возвысить Яхве до величия верховного Божества. И все же, несмотря на это возвышенное учение, ограниченное понимание его последователей вынуждало говорить о Боге как о существе в человеческом образе, подверженном приступам злобы, гнева и жестокости, говорить даже, что он мстителен и легко поддается влиянию человеческого поведения.
\vs p096 4:8 Под влиянием учений Моисея племенной природный бог Яхве стал Господом, Богом Израиля, который последовал за своим народом в пустыню и даже в изгнание, где вскоре и стал Богом всех народов. Более позднее пленение, поработившее евреев в Вавилоне, окончательно освободило развивавшееся представление о Яхве, позволив ему играть монотеистическую роль Бога всех наций.
\vs p096 4:9 Наиболее уникальная и поразительная особенность религиозной истории иудеев связана с этой непрерывной эволюцией представления о Божестве от примитивного бога горы Хорив до учений их сменявших друг друга духовных лидеров и далее до высокого уровня развития, выразившегося в учениях о Божестве двух Исай, которые провозгласили величественную концепцию любящего и милосердного Отца\hyp{}Творца.
\usection{5. Учение Моисея}
\vs p096 5:1 Моисей прекрасно сочетал в себе качества военачальника, организатора общества и религиозного учителя. Между временами Махивенты и Иисуса он был самой значительной личностью, мировым учителем и лидером. Моисей пытался ввести множество реформ в Израиле, о которых нет письменных упоминаний. За время одной человеческой жизни он вывел из рабства и состояния нецивилизованного кочевья многоязычную толпу так называемых иудеев и положил основание последующему рождению нации и увековечению расы.
\vs p096 5:2 О великом деле Моисея существует так мало записей потому, что во время исхода у иудеев не было письменности. Записи о временах и деяниях Моисея были составлены на основе преданий, сохранявшихся после смерти великого вождя более тысячи лет.
\vs p096 5:3 Многие шаги, сделанные Моисеем и возвысившие его идеи над религией египтян и окружающих левантийских племен, были обусловлены традициями кенитов времен Мелхиседека. Без учений Махивенты, обращенных к Аврааму и его современникам, иудеи бы вышли из Египта в безнадежной тьме. Моисей и его тесть Иофор собрали остатки традиций времен Мелхиседека, и этими учениями, соединенными с учениями египтян, руководствовался Моисей в создании более совершенной религии и ритуалов израильтян. Моисей был организатором; он выбрал лучшее из религии и нравов Египта и Палестины и, связав эти обычаи с традициями учений Мелхиседека, создал ритуальную систему иудейского поклонения.
\vs p096 5:4 \pc Моисей верил в Провидение; он глубоко проникся учениями Египта о сверхъестественной власти Нила и других элементов природы. Он обладал великим видением Бога, но был до конца искренним, когда учил иудеев, что, если они будут покорны Богу, «он возлюбит тебя, благословит тебя и размножит тебя. Он размножит плод чрева твоего и плод земли твоей --- хлеб, вино, елей и стада твои. Благословен будешь больше всех народов, и Господь Бог твой отдалит от тебя всякую немощь и никаких лютых болезней египетских не наведет на тебя». Моисей даже сказал: «Помни Господа Бога твоего, ибо он дает тебе силу приобретать богатство». «Будешь давать взаймы многим народам, а сам не будешь брать взаймы. Будешь господствовать над многими народами, а они над тобою не будут господствовать».
\vs p096 5:5 \pc Однако поистине жалко было смотреть, как великий ум Моисея пытается приспособить свое возвышенное представление об Эл\hyp{}Элионе, Всевышнем, к пониманию невежественных и неграмотных иудеев. Своим собравшимся лидерам он громогласно возвещал: «Господь Бог ваш есть единственный Бог, и нет никого, кроме него», а смешанной толпе объявлял: «Кто подобен Богу твоему среди всех богов?» Моисей смело и отчасти успешно выступил против фетишей и идолопоклонства, заявив: «Вы не видели никакого лика, в день, когда Бог ваш говорил к вам на Хориве среди огня». Он также запретил делать изображения любого рода.
\vs p096 5:6 Моисей боялся провозгласить милость Яхве, предпочитая держать свой народ в страхе перед правосудием Божиим, и говорил: «Господь, Бог ваш, есть Бог богов и Владыка владык, Бог великий, сильный и страшный, который не смотрит на лица». И опять\hyp{}таки Моисей пытался усмирить мятежные кланы, когда заявил, что «Бог твой убивает, когда ты непослушен ему; он исцеляет и дает жизнь, когда ты послушен ему». Но Моисей учил эти племена, что они станут избранным народом Бога лишь при условии, что они «соблюдут все его заповеди и будут исполнять законы его».
\vs p096 5:7 В эти древние времена иудеев мало учили о милосердии Бога. Они узнавали о Боге как о «Всемогущем; Господь есть муж брани, Бог сражений, славный в силе, наголову разбивающий своих врагов». «Господь, Бог твой, ходит среди стана твоего, чтоб избавлять тебя». Израильтяне думали о своем Боге как о Боге, любящем их, который, однако, «ожесточил сердце фараона» и «проклял их врагов».
\vs p096 5:8 Хотя Моисей позволил детям Израиля мельком взглянуть на всемирное и милосердное Божество, в целом в их повседневном представлении Яхве был Богом, но лишь ненамного лучшим племенных богов соседних народов. Их представление о Боге было примитивным, грубым и антропоморфным; когда Моисей умер, эти племена бедуинов быстро вернулись к полуварварским представлениям о своих старых богах Хорива и пустыни. Расширенное и более возвышенное осознание Бога, которое Моисей иногда являл своим лидерам, вскоре сошло на нет, а большая часть народа вернулась к поклонению своим фетишам --- золотым тельцам, символу Яхве палестинских пастухов.
\vs p096 5:9 \pc Когда Моисей передал руководство иудеями Иисусу Навину, Моисей уже собрал тысячи состоявших в родстве потомков Авраама, Нахора, Лота и других представителей родственных племен и сплотил их в самостоятельную и отчасти самоуправляющуюся нацию воинов\hyp{}пастухов.
\usection{6. Представления о Боге после смерти Моисея}
\vs p096 6:1 После смерти Моисея его возвышенное представление о Яхве быстро исказилось. Иисус Навин и вожди Израиля продолжали придерживаться моисеевых традиций премудрого, милосердного и всемогущего Бога, однако простой народ быстро вернулся к старому представлению о Яхве времен жизни в пустыне. И эта деградация представления о Божестве продолжалась со все большей силой при последующем правлении племенных шейхов, или так называемых Судей.
\vs p096 6:2 Обаяние удивительной личности Моисея не давало угаснуть в сердцах его последователей вдохновляющему и все более и более расширявшемуся представлению о Боге; однако, достигнув плодородных земель Палестины, они быстро превратились из кочующих пастухов в оседлых и довольно\hyp{}таки мирных фермеров. Причем эта эволюция жизненных привычек и изменение религиозных воззрений потребовали той или иной перемены в их представлениях о природе их Бога Яхве. Во времена начала превращения сурового, грубого, требовательного и грозного пустынного бога Синая в появившееся позднее понятие о Боге любви, справедливости и милосердия, иудеи почти забыли возвышенные учения Моисея. Они почти утратили всякое представление о монотеизме и, можно сказать, потеряли свою возможность стать народом, который служил бы жизненно важным звеном в духовной эволюции Урантии, группой, которая сохранила бы учения Мелхиседека о Боге до времени воплощения совершившего пришествие Сына этого всеобщего Отца.
\vs p096 6:3 Иисус Навин отчаянно пытался сохранить представление о высшем Яхве в умах соплеменников, благодаря чему было провозглашено: «Как я был с Моисеем, так буду и с тобою; не отступлю от тебя и не оставлю тебя». Иисус Навин считал необходимым проповедовать суровое евангелие своему неверующему народу, народу, слишком готовому верить в свою старую и туземную религию, но не желавшему идти вперед в религии веры и праведности. Сутью учения Иисуса Навина стали слова: «Яхве --- Бог святой, Бог ревнитель; он не потерпит беззакония вашего и грехов ваших». Высочайшее представление этой эпохи изображало Яхве как «Бога силы, суда и справедливости».
\vs p096 6:4 Однако даже в эту мрачную эпоху время от времени появлялись учителя\hyp{}одиночки, провозглашавшие Моисеево представление о божественном: «Вы, дети порока, не можете служить Господу, ибо он Бог святый». «Человек смертный праведнее ли Бога? Человек чище ли Творца своего?». «Можешь ли ты исследованием найти Бога? Можешь ли совершенно постигнуть Всемогущего? Вот, Бог велик, и мы не можем познать его. Касаясь Всемогущего, не можем отыскать его».
\usection{7. Псалтырь и Книга Иова}
\vs p096 7:1 Под руководством своих шейхов и священников иудеи более или менее утвердились в Палестине. Однако вскоре они стали возвращаться к закоснелым верованиям жизни в пустыне и прониклись порочным влиянием менее передовых ханаанских религиозных обычаев. Они стали идолопоклонниками и развратились, а их представление о Божестве опустилось гораздо ниже египетских и месопотамских представлений о Боге, которых придерживались кое\hyp{}какие сохранившиеся салимские группы и которые записаны в некоторых псалмах и в так называемой Книге Иова.
\vs p096 7:2 \pc Псалтырь является трудом двадцати или более авторов; многие из них были написаны египетскими и месопотамскими учителями. В те времена, когда Левант поклонялся богам природы, все\hyp{}таки существовало немало верующих в верховенство Эл\hyp{}Элиона, или Всевышнего.
\vs p096 7:3 Ни в одном собрании религиозных писаний нет такого богатства религиозного рвения и вдохновенных идей о Боге, как в Псалтыри. И было бы очень полезно, если бы при внимательном чтении этого чудесного собрания богословской литературы внимание уделяли источнику и хронологии каждого отдельного гимна хвалы и поклонения, помня при этом, что ни одно другое собрание не охватывает собой столь огромного периода времени. Псалтырь представляет собой запись различных представлений о Боге, которых придерживались верующие последователи салимской религии во всем Леванте, и охватывает собой весь период от Аменемопа до Исайи. В Псалтыри Бог изображен во всех аспектах представлений о нем --- от грубого понятия о божестве племени до предельно расширенного идеала более поздних иудеев, рисующего Яхве любящим правителем и милосердным Отцом.
\vs p096 7:4 При таком рассмотрении эти псалмы --- самый ценный и полезный сборник благочестивых мыслей, когда\hyp{}либо собранный человеком вплоть до времен двадцатого века. Дух богопочитания, присущий этому собранию гимнов, намного превосходит дух богопочитания, присущий всем остальным священным книгам мира.
\vs p096 7:5 \pc Многогранное изображение Божества, представленное в Книге Иова, является произведением более двадцати религиозных учителей Месопотамии, живших на протяжении почти трех столетий. И читая возвышенное представление о божественном, которое можно найти в этом собрании месопотамских верований, вы осознаете, что именно в окрестностях халдейского Ура в мрачные времена в Палестине понимание реального Бога сохранилось лучше всего.
\vs p096 7:6 В Палестине часто осознавали, что Бог мудр и вездесущ, но редко понимали, что он благ и милосерден. Яхве того времени «посылает злых духов, дабы те овладели душами его врагов»; он благоприятствует своим собственным и послушным детям и проклинает и сурово осуждает всех остальных. «Он разрушает замыслы коварных и улавливает мудрецов их же лукавством».
\vs p096 7:7 Только в Уре возвышался голос, дабы провозгласить милосердие Бога, говоря: «Он будет молиться Богу, и найдет милость его, и с радостью будет взирать на лицо его, ибо Бог даст человеку божественную праведность». Таким образом, из Ура проповедуется спасение, божественная милость, обретаемые благодаря вере: «Он умилосердится над кающимся и скажет: „Освобожу его от сошествия в преисподнюю, ибо я нашел искупление“. Если кто скажет: „Я согрешил и извратил праведное, и это не принесло мне пользы“, Бог избавит душу его от преисподней, и он увидит свет». Со времен Мелхиседека Левантийский мир не слышал такого звучного и радостного послания, как это прекрасное учение Елиуя, урского пророка и священника салимских верующих, то есть остатка бывшей колонии Мелхиседека в Месопотамии.
\vs p096 7:8 И так остатки салимских миссионеров в Месопотамии поддерживали свет истины на протяжении периода дезорганизации иудейских народов до появления первого из долгой череды учителей Израиля, которые, не останавливаясь, воздвигали храм осознания, пока не достигли идеала Отца Творца Всего Сущего, высшей точки эволюции представления о Яхве.
\vsetoff
\vs p096 7:9 [Представлено Мелхиседеком Небадона.]
