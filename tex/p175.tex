\upaper{175}{Последняя беседа в храме}
\author{Комиссия срединников}
\vs p175 0:1 Во вторник днем в начале третьего Иисус в сопровождении одиннадцати апостолов, Иосифа Аримафейского, тридцати греков и нескольких других учеников пришел в храм и начал свою последнюю речь, произнесенную во дворах этого священного здания. Эта беседа должна была стать его последним обращением к еврейскому народу и его последним обвинением яростным врагам, мечтающим его уничтожить, --- книжникам, фарисеям, саддукеям и верховным правителям Израиля. Все время до полудня у разных людей была возможность задавать Иисусу вопросы; после полудня в этот день никто его уже ни о чем не спрашивал\ldots
\vs p175 0:2 Когда Учитель начал говорить, во дворе храма была тишина и порядок. Менялы и торговцы не осмеливались больше входить в храм с тех пор, как накануне Иисус и возбужденная толпа выгнали их оттуда. Прежде чем начать беседу, Иисус ласково посмотрел на слушателей, которым так скоро предстояло услышать его прощальное обращение, предлагающее милость человечеству и в то же время осуждающее лживых учителей и фанатичных правителей евреев.
\usection{1. Беседа}
\vs p175 1:1 «Все это долгое время я был с вами, ходил по стране и возвещал любовь Отца к детям человеческим, и многие увидели свет и через веру вошли в царство небесное. В связи с этим учением и проповедованием Отец совершил много чудесных деяний, вплоть до воскрешения мертвого. Многие больные и страждущие сделались здоровыми, потому что верили; но и возвещение истины и исцеления болезней так и не открыли глаза тем, кто отказывается видеть свет, кто полон решимости отвергнуть это евангелие царства.
\vs p175 1:2 В полном согласии с волей моего Отца, я и мои апостолы делали все возможное, чтобы жить в мире с нашими братьями, следовать разумным требованиям закона Моисея и традициям Израиля. Мы упорно стремились к миру, но правители Израиля не желают его. Отвергая истину Бога и свет небес, они встают на сторону заблуждений и тьмы. Не может быть мира между светом и тьмой, между жизнью и смертью, между истиной и заблуждениями.
\vs p175 1:3 Многие из вас решились поверить в мои учения и уже обрели радость и свободу от осознания сыновства по отношению к Богу. И вы засвидетельствуете, что я предлагал это же сыновство по отношению к Богу всей еврейской нации, даже тем самым людям, которые сейчас стремятся меня уничтожить. Но и теперь мой Отец принял бы этих ослепленных учителей и этих лицемерных вождей, если бы только они обратились к нему и приняли его милосердие. Даже сейчас этому народу еще не поздно принять слово небес и приветствовать Сына Человеческого.
\vs p175 1:4 Мой Отец долгое время милостиво относился к этому народу. Из поколения в поколение мы посылали наших пророков, чтобы учить и предостерегать их, и из поколения в поколение они убивали этих посланных небом учителей. И теперь ваши своевольные первосвященники и упрямые правители продолжают делать то же самое. Как Ирод казнил Иоанна, так же и вы собираетесь теперь уничтожить Сына Человеческого.
\vs p175 1:5 Пока есть вероятность, что евреи обратятся к моему Отцу и будут стремиться к спасению, Бог Авраама, Исаака и Иакова будет простирать к вам свои милостивые руки; но когда вы переполните чашу своей нераскаянностью и когда окончательно отвергнете милость моего Отца, эта нация будет предоставлена сама себе и вскоре придет к бесславному концу. Этот народ был призван стать светом мира, явить духовную славу народа, знающего Бога, но вы настолько отступили от исполнения ваших божественных привилегий, что ваши руководители готовы совершить величайшее безумие всех времен, окончательно отвергнув дар Бога всем людям и на все века --- откровение любви Отца Небесного ко всем его созданиям на земле.
\vs p175 1:6 И когда вы отвергнете это откровение Бога человеку, царство небесное будет отдано другим народам, тем, которые примут его с удовольствием и радостью. От имени Отца, пославшего меня, я серьезно предупреждаю вас, что вы вот\hyp{}вот утратите свою роль мировых знаменосцев вечной истины и стражей божественного закона. Сейчас я предлагаю вам последнюю возможность откликнуться и раскаяться, выказать свое намерение всем сердцем устремиться к Богу и, подобно маленьким детям, через искреннюю веру вступить в дающее уверенность и спасение царствие небесное.
\vs p175 1:7 Мой Отец долго трудился для вашего спасения, и я пришел на землю, чтобы жить среди вас и лично указать вам путь. Многие из евреев и самарян и даже язычников поверили в евангелие царства, но те, кто первыми должны были бы откликнуться и принять свет небесный, упорно отказывались поверить в откровение божественной истины --- в Бога, открывающегося в человеке, и человека, возвышенного к Богу.
\vs p175 1:8 В этот день мои апостолы стоят здесь перед вами в молчании, но скоро вы услышите, как зазвучат их голоса, взывающие к спасению и к единению с царством небесным в качестве сынов живого Бога. И сейчас я призываю моих учеников и верующих в евангелие царства, равно как незримых вестников рядом с ними, в свидетели того, что я еще раз предложил Израилю и его правителям избавление и спасение. Но все вы видите, как пренебрегают милостью Отца и как отвергают вестников истины. Тем не менее, я предупреждаю вас, эти книжники и фарисеи по\hyp{}прежнему восседают на месте Моисея, и поэтому призываю вас сотрудничать с этими старейшинами Израиля, пока Всевышние, которые правят царствами человеческими, окончательно не уничтожат эту нацию и не разрушат место пребывания этих правителей. От вас не требуется объединяться с ними в их намерениях уничтожить Сына Человеческого, но во всем, что касается мира в Израиле, вы должны подчиняться им. Во всех этих вопросах поступайте так, как они велят вам, и соблюдайте суть закона, но не следуйте примеру их дурных деяний. Помните, вот в чем грех этих правителей: ибо они говорят и не делают. Вы хорошо знаете, как эти предводители возлагают тяжкие бремена на ваши плечи, бремена неудобоносимые, а сами не хотят и перстом двинуть, чтобы помочь вам носить эти тяжкие бремена. Они подавили вас обрядами и поработили традициями.
\vs p175 1:9 Кроме того, эти эгоцентричные правители любят совершать добрые деяния напоказ. Они увеличивают размеры своих филактерий и делают шире кайму одежд своих. Они желают предвозлежать на пиршествах и председать в синагогах. Они жаждут хвалебных приветствий на рыночных площадях и желают, чтобы все люди называли их «учитель». И при том, что они добиваются от людей всех этих почестей, они же тайно завладевают домами вдов и получают доход от служб в святом храме. Для виду эти лицемеры на людях долго молятся и раздают милостыню, чтобы привлечь внимание своих соплеменников.
\vs p175 1:10 Хотя вам следует воздавать почести своим правителям и чтить своих учителей, вы не должны называть ни одного человека духовным Отцом, ибо есть тот, кто является вашим Отцом, и это Бог. Не должны вы и помыкать своими братьями по царству. Помните, я учил вас, что тот, кто будет большим из вас, должен стать слугой для всех. Если вы осмелитесь возвышать себя перед Богом, вы наверняка будете унижены; но всякий, кто воистину унижает себя, обязательно будет возвышен. В своей повседневной жизни стремитесь не к прославлению себя, но к славе Бога. Разумно подчиняйте свою собственную волю воле Отца Небесного.
\vs p175 1:11 Не поймите мои слова превратно. Я не таю зла на этих первосвященников и правителей, которые даже сейчас стремятся меня уничтожить; я не испытываю неприязни к этим книжникам и фарисеям, отвергающим мои учения. Я знаю, что многие из вас тайно верят, и знаю, что вы открыто заявите о своей верности царству, когда придет мой час. Но как оправдаются ваши раввины, если они делают вид, что разговаривают с Богом, а затем позволяют себе отвергнуть и уничтожить того, кто пришел открыть Отца мирам?
\vs p175 1:12 Горе вам, книжники и фарисеи, лицемеры! Вы хотели бы затворить двери царства небесного для искренних людей, если они не достаточно просвещены в области ваших учений. Вы сами отказываетесь войти в царство и в то же время делаете все, что в ваших силах, чтобы не дать войти всем желающим. Вы заслонили двери спасения и боретесь со всеми, кто хотел бы в них войти.
\vs p175 1:13 Горе вам, книжники и фарисеи, лицемеры! ибо вы обходите море и сушу, дабы обратить в веру хоть одного, а когда это случится, вы не успокоитесь, пока не сделаете его вдвое хуже, чем он был, пока оставался сыном язычников.
\vs p175 1:14 Горе вам, первосвященники и правители, завладевающие имуществом бедных и облагающие тяжкими податями тех, кто хотел бы служить Богу так, как, по их мнению, предписывал Моисей! Вы, отказывающиеся проявлять милосердие, можете ли вы надеяться на милосердие в грядущих мирах?
\vs p175 1:15 Горе вам, лживые учителя, слепые проводники! Чего можно ожидать от нации, когда слепые ведут слепых? И те, и другие оступятся и попадут в яму погибели.
\vs p175 1:16 Горе вам, которые притворяются, давая клятву! Вы обманщики, поскольку учите, что человек может поклясться храмом и нарушить свою клятву, но что всякий, кто поклялся золотом храма, тот повинен. Вы все безумны и слепы. Вы даже непоследовательны в своем обмане, ибо что больше: золото или храм, освящающий золото? Вы также учите, что если человек поклянется жертвенником, то это ничего; но если кто поклянется даром, который на нем, тогда его следует признать должником. И снова вы слепы к истине, ибо что больше: дар или жертвенник, освящающий дар? Как можете вы оправдать такое лицемерие и обман перед Богом на небе?
\vs p175 1:17 Горе вам, книжники и фарисеи и все прочие лицемеры, которые следят за тем, чтобы дать десятину с мяты, аниса и тмина и в то же время пренебрежительно относятся к более весомому в законе --- вере, милосердию и суду! Разумеется, и одно надлежит делать, но и другим не пренебрегать. Воистину, вы слепые проводники и немые учителя; вы оцеживаете комара, а верблюда поглощаете.
\vs p175 1:18 Горе вам, книжники, фарисеи и лицемеры, что тщательно очищаете внешность чаши и блюда, но внутри остается грязь вымогательства, неумеренности и обмана. Вы духовно слепы. Разве не сознаете вы, насколько лучше было бы сначала очистить внутренность чаши, а затем то, что расплескивается, само очистило бы ее внешность? Вы, грешные нечестивцы! вы делаете так, чтобы отправление вашей религии внешне соответствовало букве закона Моисея в вашем толковании, тогда как ваши души погрязли в беззаконии и закоренели в убийстве.
\vs p175 1:19 Горе всем вам, отвергающим истину и пренебрегающим милосердием! Многие из вас подобны побеленным гробницам, которые снаружи кажутся красивыми, а внутри полны костей мертвых и всякой нечистоты. Так и вы, сознательно отвергающие совет Бога, по наружности кажетесь людям святыми и праведными, но внутри ваши сердца исполнены лицемерия и беззакония.
\vs p175 1:20 Горе вам, лживые поводыри нации! Вон там вы воздвигли памятник замученным пророкам прошлого и в то же время строите планы, как уничтожить того, о ком они говорили. Вы украшаете могилы праведников и льстите себе, что если бы жили во дни своих отцов, то не убивали бы пророков; а затем, вопреки таким мыслям о собственной праведности, вы собираетесь убить того, о ком говорили пророки, Сына Человеческого. Таким образом, вы сами против себя свидетельствуете, что вы --- греховные сыновья тех, кто убивал пророков. Дополняйте же меру отцов ваших!
\vs p175 1:21 Горе вам, дети зла! Иоанн справедливо назвал вас змеиными отродьями, и я спрашиваю, как можете вы избежать приговора, вынесенного вам Иоанном?
\vs p175 1:22 Но даже сейчас я предлагаю вам от имени Отца моего милость и прощение; даже сейчас я протягиваю любящую руку вечного братства. Мой Отец посылал к вам мудрецов и пророков; одних вы преследовали, других убивали. Затем появился Иоанн, возвещая приход Сына Человеческого, и его вы уничтожили после того, как многие поверили в его учение. И теперь вы готовитесь еще пролить невинную кровь. Разве вы не понимаете, что наступит страшный день расплаты, когда Судья всей земли потребует от этого народа ответа за то, как они отвергали, преследовали и уничтожали посланцев неба? Разве вы не понимаете, что должны будете ответить за всю эту праведную кровь, от первого убитого пророка и до времен Захарии, убитого между храмом и жертвенником? И если вы продолжите творить зло, ответ этот придется держать уже этому поколению.
\vs p175 1:23 О, Иерусалим и дети Авраама, избивающие пророков и камнями побивающие посланных вам учителей, даже сейчас я собрал бы ваших детей, как птица собирает птенцов своих под крылья, но вы не хотите!
\vs p175 1:24 А теперь я покидаю вас. Вы услышали мою весть и приняли решение. Те, кто поверили в мое евангелие, даже и теперь надежно пребывают в царстве Бога. Вам, решившим отвергнуть дар Бога, я говорю, что вы больше не услышите моего учения в храме. Я завершил здесь деяния свои. Смотрите, теперь я ухожу со своими детьми, а оставляется вам дом ваш пуст!»
\vs p175 1:25 И затем Учитель знаком велел своим последователям удалиться из храма.
\usection{2. Статус отдельных евреев}
\vs p175 2:1 Тот факт, что духовные руководители и религиозные учителя еврейского народа некогда отвергли учения Иисуса и даже замыслили осуществить его жестокую казнь, ни коим образом не повлиял на статус какого\hyp{}либо конкретного еврея перед Богом. И это не должно побуждать тех, кто признает себя последователем Христа, относиться с предубеждением к смертному собрату\hyp{}еврею. Евреи как нация, как общественно\hyp{}политическая группа, сполна заплатили ужасной ценой за отвержение Принца Мира. Давно уже они перестали нести человеческим народам мира духовный факел божественной истины, но это не является обоснованной причиной для того, чтобы конкретные потомки этих евреев далекого прошлого подвергались тем преследованиям, которые обрушивали на них нетерпимые, недостойные и фанатичные люди, считающие себя последователями Иисуса из Назарета, который и сам по рождению был евреем.
\vs p175 2:2 Много раз эта абсурдная и нехристосовская ненависть к современным евреям и их преследование приводили к страданиям и смерти невинных и безобидных евреев, чьи предки во времена Иисуса всем сердцем приняли его евангелие и вскоре, не дрогнув, умерли за ту истину, в которую они так беззаветно верили. Какое содрогание и ужас должны испытывать взирающие свыше небесные существа при виде того, как считающие себя последователями Иисуса позволяют себе преследовать, изводить и даже убивать потомков Петра, Филиппа, Матфея и других палестинских евреев, которые так достойно отдали свои жизни, став первыми великомучениками евангелия царства небесного!
\vs p175 2:3 Как жестоко и абсурдно заставлять невинных детей страдать за грехи их предков, о проступках которых они совершенно не ведали и за которые никоим образом не могут быть ответственны! И совершать такие греховные поступки во имя того, кто учил своих учеников любить даже своих врагов! В ходе этого рассказа о жизни Иисуса потребовалось описать, как именно отдельные его соплеменники\hyp{}евреи отвергли его и замыслили осуществить его постыдную казнь; но мы хотели бы предупредить всех, читающих сие повествование, что это историческое описание никоим образом не служит оправданием несправедливой ненависти и не оправдывает того пристрастного чувства, которое очень многие считающие себя христианами испытывали по отношению к конкретным евреям многие столетия. Верующие в царство, следующие учению Иисуса, должны перестать относиться к конкретному еврею как к виновнику отвержения и распятия Иисуса. Отец и его Сын\hyp{}Творец никогда не переставали любить евреев. Бог не взирает на лица, и спасение дается евреям точно так же, как и неевреям.
\usection{3. Роковой совет синедриона}
\vs p175 3:1 В восемь часов вечера в этот вторник был созван роковой совет синедриона. И раньше много раз этот верховный суд еврейской нации неофициально приговаривал Иисуса к смерти. Много раз этот высокий правящий орган решал положить конец его деятельности, но никогда до этого они не принимали решение арестовать его и добиться его смерти любой ценой. Незадолго до полуночи в этот вторник 4 апреля 30 года н.э. синедрион в тогдашнем его составе официально и \bibemph{единогласно} проголосовал за вынесение смертного приговора и Иисусу, и Лазарю. Так они ответили на последний призыв Учителя, с которым тот в храме обратился к правителям евреев всего за несколько часов до того, и это отражало их чувство сильного негодования по поводу последнего решительного обвинения Иисуса в адрес этих самых первосвященников и нераскаявшихся саддукеев и фарисеев. Вынесение смертного приговора Сыну Бога (еще до суда над ним) было ответом синедриона на последнее предложение небесной милости, которое было оказано еврейской нации как таковой.
\vs p175 3:2 С этого времени евреи были оставлены, чтобы закончить свое краткое и непродолжительное существование как единой нации в полном соответствии с их чисто человеческим статусом в ряду наций Урантии. Израиль отверг Сына того Бога, который заключил договор с Авраамом, и замысел сделать детей Авраама носителями света истины для мира рухнул. Божественный договор был расторгнут, и скоро настал конец еврейской нации.
\vs p175 3:3 Рано утром на следующий день людям синедриона был отдан приказ схватить Иисуса, но с наказом, что он не должен быть схвачен на людях. Им велели выработать план, как схватить его тайно, желательно --- внезапно и ночью. Понимая, что в тот день (в среду) он может не прийти снова учить в храме, этим людям синедриона велели «доставить его в высокий еврейский суд в четверг до полуночи».
\usection{4. Ситуация в Иерусалиме}
\vs p175 4:1 В конце последней проповеди Иисуса в храме апостолы снова пришли в замешательство и ужас. Перед тем, как Учитель начал свое беспощадное обличение еврейских правителей, в храм вернулся Иуда, так что вторую половину последней проповеди Иисуса слышали все двенадцать апостолов. К сожалению, Иуда Искариот не мог слышать предлагающей милость первой половины этого прощального обращения. Он не слышал этого последнего предложения милости еврейским правителям потому, что все еще совещался с некой группой своих саддукейских родственников и друзей, с которыми он был во время обеда, и советовался о том, как лучше порвать с Иисусом и собратьями\hyp{}апостолами. Именно слушая завершающую часть обличения Учителем еврейских руководителей и правителей, Иуда окончательно и бесповоротно принял решение покинуть евангелическое движение и умыть руки, отрекшись от всего этого предприятия. Тем не менее, он вышел из храма вместе со всеми апостолами, дошел с ними до Масличной горы, где вместе со всеми выслушал ту пророческую беседу о разрушении Иерусалима и конце еврейской нации, и остался в тот вечер во вторник с ними в новом лагере возле Гефсимании.
\vs p175 4:2 \P\ Услышав, как от своего милостивого призыва к еврейским руководителям Иисус резко перешел к внезапному и резкому упреку, граничащему с жестким осуждением, толпа была ошеломлена и озадачена. В ту ночь, пока синедрион заседал, вынося Иисусу смертный приговор, и пока на Масличной горе Иисус в кругу своих апостолов и некоторых учеников предрекал гибель еврейской нации, весь Иерусалим по\hyp{}серьезному и втихомолку обсуждал один\hyp{}единственный вопрос: «Что они сделают с Иисусом?»
\vs p175 4:3 \P\ В доме у Никодима собрались более тридцати видных евреев, тайно веривших в царство, и обсуждали, какую линию поведения избрать в случае, если произойдет открытый разрыв с синедрионом. Все присутствующие договорились открыто признать свою верность Учителю тотчас же, как услышат о его аресте. И именно так они впоследствии и поступили.
\vs p175 4:4 Саддукеи, которые в тот момент руководили и преобладали в синедрионе, жаждали избавиться от Иисуса по следующим причинам:
\vs p175 4:5 \ublistelem{1.}\bibnobreakspace Они боялись, что возросшее благоволение, с которым массы относились к нему, ставит под угрозу существование еврейской нации ввиду возможного вмешательства Римских властей.
\vs p175 4:6 \P\ \ublistelem{2.}\bibnobreakspace Его решительные тщания по преобразованию храма прямо ударили по их доходам; очищение храма затронуло непосредственно их кошельки.
\vs p175 4:7 \P\ \ublistelem{3.}\bibnobreakspace Они чувствовали себя ответственными за сохранение общественного порядка и боялись последствий дальнейшего распространения странного и нового учения Иисуса о братстве людей.
\vs p175 4:8 \P\ У фарисеев были другие поводы желать казни Иисуса. Они боялись его, потому что:
\vs p175 4:9 \ublistelem{1.}\bibnobreakspace Он возражал против их традиционной власти над людьми. Фарисеи были сверхконсервативны и крайне негодовали по поводу этих, как они полагали, радикальных нападок на их закрепленный законом престиж религиозных учителей.
\vs p175 4:10 \P\ \ublistelem{2.}\bibnobreakspace Они считали, что Иисус нарушал закон; что он проявил полное пренебрежение и к субботе, и ко многим другим требованиям закона и обряда.
\vs p175 4:11 \P\ \ublistelem{3.}\bibnobreakspace Они обвиняли его в богохульстве, потому что он говорил о Боге как о своем Отце.
\vs p175 4:12 \P\ \ublistelem{4.}\bibnobreakspace А теперь они были взбешены его последней беседой и тем жестким осуждением, которое он высказал им в этот день в храме в заключительной части прощальной речи.
\vs p175 4:13 \P\ В этот вторник, приняв официальное постановление о смерти Иисуса и отдав приказы о его аресте, около полуночи синедрион прервал свой совет, назначив на десять часов следующего утра встречу в доме первосвященника Каиафы, чтобы сформулировать обвинения, по которым Иисуса следовало привлечь к суду.
\vs p175 4:14 Небольшая группа саддукеев фактически предложила избавиться от Иисуса, убив его, но фарисеи категорически отказались одобрить такой образ действий.
\vs p175 4:15 \P\ Таковы были в этот знаменательный день в Иерусалиме настроения людей, а в это время сонмище небесных существ парило над этим важным местом действия на земле, желая что\hyp{}то сделать, чтобы помочь своему любимому Владыке, но бессильные действовать, потому что их решительно сдерживали возглавляющие их руководители.
