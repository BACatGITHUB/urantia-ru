\upaper{174}{Во вторник утром в храме}
\author{Комиссия срединников}
\vs p174 0:1 Во вторник около семи часов утра Иисус встретился в доме Симона с апостолами, женским отрядом и примерно двумя дюжинами других видных учеников. На этой встрече он попрощался с Лазарем, дав ему наставление, которое очень скоро сподвигло его убежать в находящуюся в Перее Филадельфию, где позже он включился в миссионерское движение, имеющее свой центр в этом городе. Иисус попрощался и с состарившимся Симоном и дал женскому отряду последние советы, поскольку он никогда уже больше официально не обращался к ним\ldots
\vs p174 0:2 В это утро он персонально каждому издвенадцати апостолов дал наставление. Андрею он сказал: «Не пугайся надвигающихся событий. Укрепляй дух своих собратьев и не допускай, чтобы они видели тебя удрученным». Петру он сказал: «Не полагайся ни на силу плоти, ни на сталь оружия. Опирайся на духовный фундамент вечных твердынь.» Иакову он сказал: « Помни, внешность обманчива, не пасуй. Оставайся твердым в своей вере, и скоро ты убедишься в реальности того, во что веришь». Иоанну он сказал: «Будь кротким; люби даже своих врагов; будь терпимым. И помни, что я многое доверял тебе». Нафанаилу он сказал: «Не суди по внешнему проявлению; оставайся твердым в своей вере, когда будет казаться, что все пропало; будь преданным своей миссии посланника царства». Филиппу он сказал: «Будь непреклонен перед лицом надвигающихся сейчас событий. Оставайся непоколебимым, даже когда не видно будет пути. Будь верен своей клятве, данной при освящении». Матфею он сказал: «Не забывай о милосердии, принявшем тебя в царство. Не позволяй ни одному человеку обманом лишить тебя твоей вечной награды. Как ты противостоял склонностям человеческой природы, так и впредь стремись быть непоколебимым». Фоме он сказал: «Как бы это ни было тяжело, сейчас ты должен идти, полагаясь на веру, а не на зрение. Не сомневайся в том, что я в состоянии завершить труд, который начал, и что со временем я увижу всех своих верных посланников в царстве в мире ином». Близнецам Алфеевым он сказал: «Не позволяйте вещам, которые вы не можете понять, сокрушить вас. Доверяйте зову сердца своего и не полагайтесь на сильных мира сего, ни на меняющееся настроение народа. Будьте верны своим братьям». А Симону Зелоту он сказал: «Симон, тебя может сокрушить разочарование, но твой дух возвысится над тем, что может лечь на тебя тяжким бременем. Чему тебе не удалось научиться у меня, тому научит тебя мой дух. Стремись к подлинным реалиям духа и перестань прельщаться нереальными и материальными тенями». Иуде же Искариоту он сказал: «Иуда, я любил тебя и молился, чтобы ты любил братьев своих. Не уставай делать добро; и я хотел бы предостеречь тебя --- остерегайся скользких дорожек лести и ядовитых стрел насмешек».
\vs p174 0:3 И закончив эти приветствия, он отправился с Андреем, Петром, Иаковом и Иоанном в Иерусалим, тогда как другие апостолы стали обустраивать Гефсиманский лагерь, в который они должны были отправиться в эту ночь и который сделали своим пристанищем на все оставшееся время жизни Учителя во плоти. На полпути вниз по склону Масличной горы Иисус сделал остановку и больше часа беседовал с четырьмя апостолами.
\usection{1. Божественное прощение}
\vs p174 1:1 Несколько дней Петр и Иаков были заняты обсуждением разных точек зрения относительно учения Учителя о прощении греха. Они договорились изложить проблему Иисусу, и Петр воспользовался остановкой как подходящей возможностью получить совет Учителя. Поэтому Симон Петр прервал разговор, касавшийся различий между хвалой и почитанием и задал вопрос: «Учитель, у нас с Иаковом нет единодушия по поводу твоих учений, касающихся прощения греха. Иаков заявляет, что ты учишь, что Отец прощает нас еще до того, как мы его просим, а я утверждаю, что прощению должны предшествовать покаяние и исповедь. Кто из нас прав? что ты скажешь?»
\vs p174 1:2 После короткого молчания Иисус многозначительно посмотрел на всех четверых и ответил: «Братья мои, вы заблуждаетесь в своих взглядах, потому что не понимаете природу близких и исполненных любви отношений между созданием и Творцом, между человеком и Богом. Вы не осознаете той исполненной понимания любви, которую мудрый родитель питает к своему незрелому и иногда ошибающемуся ребенку. Вряд ли умным и любящим родителям приходится когда\hyp{}либо прощать обычного и нормального ребенка. Понимание и любовь действенно предотвращают все те отчужденияя в отношениях, которые позже приходится сглаживать и раскаянием ребенка, и прощением со стороны родителей.
\vs p174 1:3 Часть каждого отца живет в его ребенке. Отец имеет преимущество и превосходство в понимании всех вопросов, связанных с отношениями между ребенком и родителем. Отец, обладая большей родительской зрелостью, более зрелым опытом старшего партнера, способен видеть незрелость ребенка. В случае земного ребенка и небесного Отца божественный родитель обладает бесконечностью и божественностью сострадания и способности к исполненному любви пониманию. Божественное прощение неизбежно; оно присуще Богу и неотъемлемо от его безграничного всепонимания, от его абсолютного знания всего, что касается ошибочных суждений и неправильных решений ребенка. Божественное правосудие так вечно справедливо, что оно неизменно воплощает исполненное понимания милосердие.
\vs p174 1:4 Когда мудрый человек будет понимать внутренние побуждения своих ближних, он будет любить их. А когда вы любите брата своего, вы уже простили его. Эта способность понимать природу человека и прощать его видимые прегрешения --- Божественна. Если вы мудрые родители, именно так вы будете любить и понимать своих детей, даже прощать их, когда вас, казалось бы, разделило временное непонимание. Ребенок, будучи незрелым и недостаточно полно понимающим всю глубину отношений между ребенком и отцом, часто должен испытывать чувство вины и отчуждения от полного одобрения отца, но истинный отец никогда не чувствует такого отчуждения. Грех --- это опыт, свойственный сознанию создания; он не является частью сознания Бога.
\vs p174 1:5 Ваша неспособность или нежелание прощать своих ближних служит мерой вашей незрелости, вашего неумения развить в себе настоящее чувство сострадания, способности к пониманию и любви. Вы таите злобу и лелеете чувство мести в в полном соответствии с вашим незнанием внутренней природы и истинных стремлений ваших детей и собратьев. Любовь --- это следствие божественных и внутренних жизненных устремлений. Она опирается на понимание, питается альтруистическим служением и совершенствуется мудростью».
\usection{2. Вопросы еврейских правителей}
\vs p174 2:1 В понедельник вечером состоялся совет синедриона и еще примерно пятидесяти руководителей, избранных из числа книжников, фарисеев и саддукеев. Совет пришел к общему мнению, что схватить Иисуса на людях было бы опасно из\hyp{}за любви к нему простого народа. Большинство также сошлось во мнении, что прежде, чем схватить его и предать суду, следует приложить решительные усилия для его дискредитации в глазах народа. Поэтому было назначено несколько групп ученых мужей, которые на следующее утро должны были находиться в храме и пытаться сложными вопросами поймать его в ловушку или прочими способами стараться поставить его в затруднительное положение на глазах у людей. Фарисеи, саддукеи и даже иродиане --- все, наконец, объединились в этом стремлении дискредитировать Иисуса в глазах собравшегося на Пасху народа.
\vs p174 2:2 Во вторник утром, когда Иисус пришел во двор храма и начал учить, он успел произнести лишь несколько слов, как вперед вышли несколько молодых учащихся из академий, подготовленных для этой цели, и один из них от лица всех обратился к Иисусу: «Учитель, мы знаем, что ты праведный учитель, и знаем, что ты возвещаешь пути истины и что ты служишь только Богу, ибо ты не боишься никого из людей, и что ты не взираешь на лица. Мы еще учимся и хотели бы знать истину по одному вопросу, который волнует нас; наше затруднение таково: позволительно ли нам давать подать кесарю? Давать или не давать»? Иисус, понимая их лицемерие и хитрость, сказал им: «Зачем вы пришли таким образом искушать меня? Покажите мне деньги, которыми платится подать, и я отвечу вам». И когда они дали ему динарий, он посмотрел на него и сказал: «Чье изображение и надпись на этой монете?» И когда они ответили ему: «Кесаря», Иисус сказал: «Отдавайте кесарево кесарю, а Божье Богу».
\vs p174 2:3 Когда он так ответил этим молодым книжникам и их иродианским сообщникам, они удалились, а люди, даже саддукеи, радостно восприняли их поражение. Даже молодые люди, пытавшиеся поймать его в ловушку, чрезвычайно восхитились неожиданной проницательностью ответа Учителя.
\vs p174 2:4 В предыдущий день правители стремились запутать его перед толпой в вопросах духовного права, а потерпев неудачу, они попытались теперь вовлечь его в пагубную для него дискуссию о светской власти. И Пилат, и Ирод в это время были в Иерусалиме, и враги Иисуса полагали, что, если бы он осмелился посоветовать не платить дань кесарю, они могли бы сразу же обратиться к римским властям и обвинить его в подстрекательстве к мятежу. С другой стороны, они правильно рассчитали, что, если бы он недвусмысленно посоветовал платить дань, такое заявление сильно задело бы национальную гордость еврейских слушателей, отвратив, тем самым, добрую волю и любовь масс.
\vs p174 2:5 Весь этот план врагов Иисуса был расстроен, поскольку существовало хорошо известное постановление синедриона, установленное для руководства евреями, рассеянными среди нееврейских наций, согласно которому «право чеканить монету влечет за собой право взимать налоги». Таким образом Иисус избежал их ловушки. Ответить на их вопрос «нет» было бы равносильно подстрекательству к мятежу; ответить «да» значило бы уязвить глубоко коренящиеся национальные чувства того времени. Учитель не уклонился от вопроса; он просто проявил мудрость, дав двусмысленный ответ. Иисус никогда не был уклончив, но он всегда проявлял мудрость при общении с теми, кто стремился извести и уничтожить его.
\usection{3. Саддукеи и воскресение}
\vs p174 3:1 Прежде, чем Иисус смог начать учение, еще несколько человек вышли вперед, чтобы задать ему вопрос, на этот раз группа ученых и хитрых саддукеев. Говоривший от их лица приблизился к Иисусу и сказал: «Учитель, Моисей сказал, что если женатый человек умрет бездетным, то брат его должен взять его жену и восстановить семя брату своему. Случилось так, что некий человек, у которого было шесть братьев, умер бездетным; его жену взял следующий за ним по возрасту брат, но тоже вскоре умер, не оставив детей. Подобным же образом, его жену взял второй брат, но тоже умер, не оставив потомства. И так далее, пока все шестеро его братьев не умерли, взяв ее и не оставив детей. А затем, после них всех, умерла и сама женщина. Теперь вот что мы хотели бы спросить у тебя: чьей женой она будет после воскресения, если имели ее женой все семеро этих братьев?»
\vs p174 3:2 Иисус, равно как и присутствовавшие люди, знал, что саддукеи лукавили, когда задавали этот вопрос, потому что было маловероятно, чтобы такой случай действительно мог произойти; кроме того, этот обычай, по которому братья умершего человека стремились произвести за него детей, в то время среди евреев практически не применялся, хотя и не был отменен. Тем не менее, Иисус снизошел до ответа на этот злонамеренный вопрос. Он сказал: «Вы все заблуждаетесь, задавая такие вопросы, потому что вы не знаете ни Писания, ни живой силы Бога. Вы знаете, что чада века сего женятся и выходят замуж, но, похоже, вы не понимаете, что те, кто признаются достойными того, чтобы достигнуть миров грядущих через воскресение праведных, не женятся и не отдаются в жены. Те, кто познали воскресение из мертвых, более похожи на ангелов небесных и никогда не умирают. Эти воскресшие --- вечные сыны Бога; они дети света, воскресшие к движению вперед к вечной жизни. И даже ваш Отец Моисей понимал это, ибо услышал, как Отец при купине сказал: „Я \bibemph{есть} Бог Авраама, Бог Исаака, Бог Иакова“. Итак, вместе с Моисеем я заявляю, что мой Отец --- Бог не мертвых, но живых. В нем все вы живете, производите потомство и обладаете своей конечной жизнью».
\vs p174 3:3 Когда Иисус кончил отвечать на эти вопросы, саддукеи удалились, а некоторые фарисеи настолько забылись, что стали восклицать: «Правда, правда, Учитель, ты хорошо ответил этим неверующим саддукеям». Саддукеи больше не осмеливались задавать ему вопросы, а простые люди восхитились мудростью его учения.
\vs p174 3:4 \P\ В своем споре с саддукеями Иисус сослался только на Моисея, потому что эта религиозно\hyp{}политическая секта признавала значимость только пяти так называемых Книг Моисея; они не признавали, что учения пророков приемлемы в качестве основы для религиозных догматов. В своем ответе Учитель, хотя и ясно подтвердил факт продолжения существования созданий через воскресение, однако ни в коей мере не одобрил веру фарисеев в воскресение человеческого тела в буквальном смысле. Иисус хотел особо подчеркнуть тот факт, что Отец сказал: «Я \bibemph{есть} Бог Авраама, Исаака и Иакова», а не «я \bibemph{был} » их Богом.
\vs p174 3:5 Саддукеи думали подвергнуть Иисуса уничтожающему воздействию \bibemph{насмешек,} прекрасно зная, что открытое преследование наверняка вызвало бы рост симпатий к нему в народе.
\usection{4. Великая заповедь}
\vs p174 4:1 Другой группе саддукеев было поручено задавать Иисусу скользкие вопросы об ангелах, но когда они увидели, что стало с их товарищами, пытавшимися сбить его с толку вопросами о воскресении, то мудро решили промолчать; они удалились, не задав ни одного вопроса. Заранее подготовленный план объединившихся фарисеев, книжников, саддукеев и иродиан предусматривал весь день задавать такие ставящие в затруднительное положение вопросы в надежде тем самым дискредитировать Иисуса в глазах народа и в то же время реально воспрепятствовать тому, чтобы у него было время для возвещения своих нарушающих покой учений.
\vs p174 4:2 Затем вышла вперед, чтобы задавать провокационные вопросы, очередная группа фарисеев, и один из них, подав знак Иисусу, сказал: «Учитель, я законник, и хотел бы спросить тебя, какая, по твоему мнению, наибольшая заповедь?» Иисус ответил: «Есть одна заповедь, и она наибольшая из всех, и заповедь эта такова: „Слушай, Израиль! Господь Бог наш есть Господь единый; и возлюби Господа Бога твоего всем сердцем твоим, и всей душою твоею, и всем разумением твоим, и всей крепостью твоею!“. Сие есть первая и наибольшая заповедь. А вторая заповедь подобна этой первой; воистину, она проистекает прямо из нее, и она такова: „Возлюби ближнего своего, как самого себя“. Иной, большей этих, заповеди нет; на этих двух заповедях утверждается весь закон и пророки».
\vs p174 4:3 Когда этот законник понял, что Иисус ответил не только в соответствии с высшей идеей еврейской религии, но что он ответил мудро и с точки зрения собравшейся толпы, он счел, что разумнее будет открыто похвалить ответ Учителя. Поэтому он сказал: «Правда, Учитель, ты хорошо сказал, что Бог един и нет другого, кроме него; и что любить его всем сердцем, разумом и всею крепостью, а также любить своего ближнего как самого себя --- это первая и наибольшая заповедь; и мы согласны, что этой великой заповеди следует придавать гораздо больше значения, чем всем всесожжениям и жертвоприношениям». Когда законник произнес эти благоразумные слова, Иисус посмотрел на него и сказал: «Друг мой, я вижу, недалеко ты от царства Бога».
\vs p174 4:4 \P\ Иисус сказал правду, говоря, что этот законник «недалеко от царства», потому что в тот же вечер он отправился в лагерь Учителя возле Гефсимании, открыто принял веру в евангелие царства и был крещен Иосией, одним из учеников Авенира.
\vs p174 4:5 \P\ Присутствовали и намеревались задавать вопросы еще две или три группы книжников и фарисеев, но их или обезоружил ответ Иисуса законнику, или остановило фиаско всех предпринимавших попытки поймать его в ловушку. После этого ни один человек больше не осмелился задать ему публично еще вопросы.
\vs p174 4:6 Коль скоро вопросов больше не последовало и близился час полудня, Иисус не стал продолжать свое учение, но удовольствовался просто тем, что задал вопрос фарисеям и их сподвижникам. Иисус сказал: «Раз вы больше не задаете вопросов, я хотел бы задать вам один вопрос. Что вы думаете о Спасителе? Чей он сын?» После короткой паузы один из книжников ответил: «Мессия --- сын Давида». И поскольку Иисус знал, что было много споров, даже среди его собственных учеников, сын ли он Давида или нет, он задал следующий вопрос: «Если Спаситель воистину сын Давида, как же получается, что в Псалме, который вы приписываете Давиду, он сам, по вдохновению, говорит: „Сказал Господь Господу моему: сиди одесную меня, доколе положу врагов твоих в подножие ног твоих“. Если Давид называет его Господом, как же тогда он может быть его сыном?» Хотя правители, книжники и первосвященники не дали ответа на этот вопрос, они в то же время воздержались и от того, чтобы продолжить задавать вопросы с целью поставить его в затруднительное положение. Они так никогда и не ответили на этот вопрос, поставленный перед ними Иисусом, но после смерти Учителя попытались избежать затруднений, изменив толкование этого Псалма таким образом, будто в нем говорится об Аврааме, а не о Мессии. Другие пытались выйти из затруднительного положения, отрицая, что Давид был автором этого так называемого Мессианского Псалма.
\vs p174 4:7 Незадолго до этого фарисеи порадовались тому, как Учитель заставил умолкнуть саддукеев; теперь саддукеи были в восторге от неудачи фарисеев; но такое соперничество было лишь кратковременным; они быстро забыли о своих освященных временем разногласиях, объединившись в усилии остановить учения и деятельность Иисуса. Но на протяжении всех этих событий простые люди слушали его с удовольствием.
\usection{5. Вопрошающие греки}
\vs p174 5:1 Около полудня, когда Филипп закупал провизию для нового лагеря, который в тот день создавался возле Гефсимании, к нему обратилась делегация чужеземцев, группа верующих греков из Александрии, Афин и Рима, и один из них сказал апостолу: «Нам указали на тебя те, кто тебя знают; так что мы обращаемся к тебе, господин, с просьбой увидеть Иисуса, твоего Учителя». Таким образом, Филипп был застигнут врасплох, случайно встретив на рыночной площади этих вопрошающих видных греческих неевреев и, поскольку Иисус вполне определенно велел всем двенадцати не заниматься никаким учением народа в течение пасхальной недели, он был несколько озадачен, как поступить в этом случае. Он был в замешательстве также и потому, что эти люди были чужеземными неевреями. Если бы они были евреями или живущими поблизости знакомыми неевреями, он бы так сильно не колебался. Он поступил следующим образом: попросил этих греков остаться на том же месте. Когда он поспешил прочь, те сочли, что он отправился на поиски Иисуса, но на самом деле он поспешил в дом Иосифа, где, как он знал, обедали Андрей и другие апостолы, и, вызвав Андрея, объяснил цель своего прихода, а затем в сопровождении Андрея вернулся к дожидающимся грекам.
\vs p174 5:2 Поскольку Филипп практически закупил все что нужно, то он вместе с Андреем и греками вернулся в дом Иосифа, где их принял Иисус; и они сидели неподалеку, пока тот говорил с апостолами и с некоторыми из ведущих учеников, собравшимися за трапезой. Иисус сказал:
\vs p174 5:3 \P\ «Мой Отец послал меня в этот мир, чтобы открыть детям человеческим его исполненную любви доброту, но те, к кому я сначала пришел, отказались принять меня. Воистину, правда, что многие из вас сами уверовали в мое евангелие, но дети Авраама и их руководители готовы отвергнуть меня, а сделав так, они отвергнут Того, кто послал меня. Я свободно возвестил людям евангелие спасения; я рассказал им о сыновстве, сопряженном с радостью, свободой и жизнью, более духовно богатой. Мой Отец совершил много чудесных деяний среди этих скованных страхом сынов человеческих. Но, воистину, Пророк Исайя говорил об этих людях, когда писал: „Господи, кто уверовал в наши учения? И кому открыт Господь?“. Воистину, вожди моего народа умышленно ослепили глаза свои, чтобы не видеть, и ожесточили свои сердца, чтобы не верить и не быть спасенными. Все эти годы я стремился исцелить их от их неверия, чтобы они могли получить от Отца вечное спасение. Я знаю, что не все обманули мои ожидания; некоторые из вас действительно поверили в мою весть. В этой комнате сейчас добрых два десятка людей, которые некогда были членами синедриона или занимали высокое положение в руководящих органах нации, хотя некоторые из вас все\hyp{}таки избегают открыто признать истину, чтобы не быть отлученными от синагоги. Некоторые из вас подвержены искушению любить славу человеческуюболее, нежели славу Божию. Но я вынужден проявлять снисходительность, поскольку опасаюсь за безопасность и верность даже немногих из тех, кто так долго был возле меня и жил рядом со мной.
\vs p174 5:4 Здесь, на званом обеде, как я вижу, собралось примерно одинаковое число евреев и неевреев, и я хотел бы обратиться к вам как к первой и последней такой группе, которую я могу наставить в делах царства прежде, чем отправлюсь к моему Отцу».
\vs p174 5:5 Эти греки исправно посещали учения Иисуса в храме. В понедельник вечером в доме у Никодима они провели совещание, продолжавшееся до рассвета, и тридцать из них решили войти в царство.
\vs p174 5:6 Иисус стоял перед ними в этот момент и почувствовал окончание одной диспенсации и начало другой. Обратив свое внимание на греков, Учитель сказал:
\vs p174 5:7 \P\ «Тот, кто верит в это евангелие, не в меня верует, но в Пославшего меня. Когда вы смотрите на меня, вы видите не только Сына Человеческого, но пославшего меня. Я --- свет мира, и всякий, кто поверит в мое учение, больше не будет жить во тьме. Если вы, неевреи, услышите меня, вы получите слова жизни и тотчас вступите в радостную свободу истины сыновства у Бога. Если мои соотечественники, евреи, решат отвергнуть меня и отказаться от моих учений, я не стану судить их, ибо я пришел не судить мир, но предложить ему спасение. Тем не менее, те, кто отвергают меня и отказываются принять мое учение, будут судимы в свое время моим Отцом и теми, кого он назначил судить тех, кто отвергают дар милосердия и истину спасения. Помните, все вы, что я говорю не от себя самого, но правдиво возвестил вам то, что Отец повелел мне открыть детям человеческим. И слова, которые Отец повелел мне сказать миру, --- это слова божественной истины, непреходящего милосердия и вечной жизни.
\vs p174 5:8 Но я заявляю и евреям, и неевреям, что пришел час прославиться Сыну Человеческому. Вы хорошо знаете, что если пшеничное зерно, падши в землю, не умрет, то останется одно; а если умрет в доброй почве, то снова прорастает к жизни и принесет много плодов. Тот, кто эгоистично любит свою жизнь, стоит перед опасностью потерять ее; но тот, кто готов пожертвовать свою жизнь ради меня и евангелия, в избытке обретет жизнь на земле и на небесах, жизнь вечную. Если вы действительно будете следовать за мной даже после того, как я уйду к Отцу моему, тогда вы станете моими учениками и искренними служителями своих смертных собратьев.
\vs p174 5:9 Я знаю, что близится мой час, и я обеспокоен. Я чувствую, что мой народ намерен отвергнуть царство, но я рад принять этих стремящихся к истине неевреев, которые пришли сегодня сюда, вопрошая о пути света. Тем не менее, сердце мое болит за мой народ, и душа моя полна скорби оттого, что простирается передо мной. Что сказать мне, глядя вперед и видя, что скоро постигнет меня? Сказать ли мне: Отче, избавь меня от этого ужасного часа? Нет! Именно для этой цели и на сей час пришел я в этот мир. Вместо этого я, говорю и я молюсь, чтобы вы присоединились ко мне: Отче, да прославится имя твое; да исполнится воля твоя».
\vs p174 5:10 Когда Иисус сказал это, перед ним появился Персонализированный Настройщик, пребывавший с ним в его жизни до времени крещения, и когда он сделал заметную паузу, этот теперь могущественный дух представляющий Отца обратился к Иисусу из Назарета и сказал: «Я прославил свое имя в твоих пришествиях много раз и еще прославлю».
\vs p174 5:11 Хотя собравшиеся здесь евреи и неевреи не услышали никакого голоса, они не могли не заметить, что Учитель сделал паузу в своей речи на то время, пока ему шло сообщение из некоего сверхчеловеческого источника. Все они, все кто был подле него сказали: «Ангел ему говорил».
\vs p174 5:12 Затем Иисус продолжил: «Все это произошло не для меня, но для вас. Я знаю наверняка, что Отец примет меня и примет мою миссию ради вас, но нужно, чтобы вы приободрились и были готовы к жестокому испытанию, которое скоро предстоит. Я хочу вас уверить, что победа, в конце концов, увенчает наши объединенные усилия дать свет миру и свободу человечеству. Старый порядок выносит себе приговор; я низверг князя этого мира; и все люди станут свободными через свет духа, который я изолью на всякую плоть после того, как вознесусь к моему Отцу Небесному.
\vs p174 5:13 И сейчас я заявляю вам, что когда я вознесен буду от земли и ваших жизней, привлеку всех людей к себе и в братство Отца моего. Вы верили, что Спаситель будет жить на земле вечно, но я заявляю, что Сын Человеческий будет отвергнут людьми и вернется к Отцу. Лишь недолго пробуду я с вами; лишь недолго среди погруженного во тьму поколения будет живой свет. Идите же, пока у вас есть этот свет, чтобы надвигающаяся тьма и смятение не объяли вас. Ходящий во тьме не знает, куда идет; но если вы предпочтете идти при свете, то воистину все вы станете свободными сыновьями Бога. А теперь все вы идите со мной, мы вернемся в храм, и я скажу прощальные слова первосвященникам, книжникам, фарисеям, саддукеям, иродианам и пребывающим во тьме правителям Израиля».
\vs p174 5:14 Сказав это, Иисус направился через узкие улицы Иерусалима обратно к храму. Только что они услышали, как Учитель сказал, что это будет его прощальная проповедь в храме, и они шли за ним в молчании и глубокой задумчивости.
