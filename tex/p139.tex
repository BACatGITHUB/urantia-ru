\upaper{139}{Двенадцать апостолов}
\author{Комиссия срединников}
\vs p139 0:1 Красноречивым свидетельством притягательной силы и праведности земной жизни Иисуса является то, что, хотя он неоднократно разбивал надежды своих апостолов и развенчивал каждое их притязание на личное возвышение, лишь один из них оставил его.
\vs p139 0:2 Апостолы узнавали от Иисуса о царстве небесном, а Иисус от них многое узнавал о царстве людей, человеческой природе, в том виде как она существует на Урантии и других эволюционирующих мирах со временем и пространством. Эти двенадцать человек являли собой множество различных типов человеческого характера, и они не были сформированы \bibemph{одинаковым} воспитанием. В жилах многих из этих рыбаков\hyp{}галилеян текла в основном нееврейская кровь, что было следствием насильственного обращения нееврейского населения Галилеи столетием раньше.
\vs p139 0:3 \pc Ошибочно считать апостолов совершенно невежественными и необразованными. Все они, кроме близнецов Алфеевых, закончили синагогальные школы и были достаточно искушены в иудейских писаниях и во многих знаниях того времени. Семеро были выпускниками капернаумских синагогальных школ, а во всей Галилее не было еврейских школ лучше.
\vs p139 0:4 В тех случаях, когда ваши записи называют этих вестников царства «невежественными и необразованными», это сделано для того, чтобы показать, что они были непрофессионалами, не искушенными в знаниях раввинов и необученными методам раввинского толкования Писания. Им не хватало так называемого высшего образования. В нынешнее время их определенно сочли бы необразованными, а в некоторых кругах общества даже некультурными. Несомненно одно: они не получили одинакового строгого и стандартного образования. С юности они обретали свой собственный жизненный опыт.
\usection{1. Андрей Первозванный}
\vs p139 1:1 Андрей, предводитель отряда апостолов царства, родился в Капернауме. Он был старшим ребенком в семье, где было пятеро детей --- он сам, его брат Симон и трое сестер. Его отец, который к тому времени умер, был партнером Зеведея, они вялили рыбу в Вифсаиде, рыбацкой гавани Капернаума. Когда Андрей стал апостолом, он не был женат, но жил со своим женатым братом Симоном Петром. Оба были рыбаками и партнерами Иакова и Иоанна, сыновей Зеведеевых.
\vs p139 1:2 В 26 году н.э. Андрей был избран апостолом, тогда ему было 33 года; он был на целый год старше Иисуса и самым старшим из апостолов. У него была прекрасная родословная и он был самым способным из двенадцати. За исключением ораторского мастерства он обладал равными со своими сотоварищами способностями почти во всем, что только можно себе представить. Иисус никогда не давал Андрею прозвища, братского имени. Однако, подобно тому, как апостолы вскоре стали называть Иисуса Учителем, так и к Андрею стали обращаться словом, равнозначным слову «Глава».
\vs p139 1:3 \pc Андрей был хорошим организатором, но еще лучшим руководителем. Он принадлежал к близкому кругу из четырех апостолов, но так как Андрей был назначен Иисусом главой группы апостолов, он, исполняя эти обязанности, был вынужден оставаться со своими собратьями, в то время как остальные трое наслаждались самым тесным общением с Учителем. До самого конца Андрей оставался главой апостольского отряда.
\vs p139 1:4 Хотя Андрей никогда не был сильным проповедником, он прекрасно вел индивидуальную работу, став первым миссионером царства, ибо как первозванный апостол немедленно привел к Иисусу своего брата Симона, который впоследствии стал одним из величайших проповедников царства. Андрей наиболее последовательно поддерживал политику Иисуса использовать программу индивидуальной работы как средство подготовки двенадцати в качестве вестников царства.
\vs p139 1:5 Учил ли Иисус апостолов наедине или проповедовал народу, Андрей, как правило, был в курсе того, что происходило; он был толковым распорядителем и умелым руководителем. Он быстро принимал решения по всем вопросам, о которых ему сообщали, если не считал, что проблема выходит за пределы его компетенции; в этом случае он обращался с ней непосредственно к Иисусу.
\vs p139 1:6 \pc Андрей и Петр были очень разными по характеру и темпераменту людьми, но, к чести их сказать, они прекрасно ладили друг с другом. Андрей никогда не завидовал ораторским способностям Петра. Не часто можно увидеть, чтобы старший по возрасту человек, подобный Андрею, оказывал бы столь сильное влияние на младшего и одаренного брата. Андрей и Петр, похоже, никогда ни в малейшей степени не завидовали способностям и успехам друг друга. Поздно вечером в день Пятидесятницы, когда, во многом благодаря энергичной и вдохновенной проповеди Петра, две тысячи душ обратилось к царству, Андрей сказал своему брату: «Я бы этого сделать не смог, но я рад, что у меня есть брат, который сумел сделать это». На что Петр ответил: «Если бы ты не привел меня к Учителю и упорно не \bibemph{удерживал} бы рядом с ним, меня бы здесь не было и я бы не сделал этого». Андрей и Петр были исключением из правила, доказав, что даже братья могут мирно жить вместе и вместе плодотворно трудиться.
\vs p139 1:7 После Пятидесятницы Петр стал знаменит, но старшего Андрея никогда не раздражало то, что всю оставшуюся жизнь его представляли как «брата Симона Петра».
\vs p139 1:8 \pc Из всех апостолов Андрей лучше всех судил о людях. Он знал, что в сердце Иуды Искариота зреет тревога, даже когда никто не подозревал, что с их казначеем что\hyp{}то не так; однако о своих опасениях он не сказал никому. Великое служение Андрея царству заключалось в советах, которые он давал Петру, Иакову и Иоанну относительно выбора первых миссионеров, посланных возвещать евангелие, а также в наставлении этих первых лидеров относительно организации управления царством. Андрей обладал великим даром обнаруживать скрытые ресурсы и нераскрывшиеся таланты молодых людей.
\vs p139 1:9 Очень скоро после вознесения Иисуса Андрей начал писать личные воспоминания о многих высказываниях и деяниях своего ушедшего Учителя. После смерти Андрея с его личных воспоминаний были сделаны копии, которые широко распространялись среди первых учителей христианской церкви. Эти частные заметки Андрея впоследствии редактировались, исправлялись, переделывались и дополнялись до тех пор, пока из них не получилось достаточно последовательное повествование о жизни Учителя на земле. Последние из этих нескольких переделанных и исправленных копий уничтожил пожар в Александрии почти через сто лет после того, как оригинал был написан первозванным из двенадцати апостолов.
\vs p139 1:10 Андрей был проницательным человеком, человеком логической мысли и твердых решений, главное свойство характера которого заключалось в его великом постоянстве. Но его характеру недоставало энтузиазма; много раз ему не удавалось ободрить своих товарищей рассудительной похвалой. И эта сдержанность в похвале достойных успехов своих друзей происходила из его отвращения к лести и неискренности. Андрей был одним из тех разносторонних, умеренных, обязанных всем самим себе и успешных людей, которые делают скромные дела.
\vs p139 1:11 \pc Каждый из апостолов любил Иисуса, но справедливости ради надо отметить, что каждого из двенадцати привлекала к нему какая\hyp{}то определенная черта его личности, которая была этому апостолу особенно симпатична. Андрей восхищался Иисусом за его постоянную искренность и подлинное достоинство. Однажды узнав Иисуса, люди становились одержимы желанием познакомить с ним своих друзей; они действительно хотели, чтобы весь мир узнал его.
\vs p139 1:12 \pc Когда впоследствии апостолы в конце концов были изгнаны из Иерусалима, Андрей путешествовал по Армении, Малой Азии и Македонии и после того, как обратил многие тысячи к царству, был в конце концов арестован и распят в Патрае в Ахаи. Прошло целых два дня прежде чем этот крепкий человек скончался на кресте, и в течение этих трагических часов он продолжал неустанно возвещать благую весть о спасении в царстве небесном.
\usection{2. Симон Петр}
\vs p139 2:1 Когда Симон присоединился к апостолам, ему было тридцать лет. Он был женат, имел троих детей и жил в Вифсаиде, неподалеку от Капернаума. Вместе с ним жили его брат Андрей и мать его жены. И Петр, и Андрей занимались рыболовством вместе с сыновьями Зеведеевыми.
\vs p139 2:2 Учитель какое\hyp{}то время знал Симона еще до того, как Андрей представил его как второго из апостолов. Когда Иисус дал Симону имя Петр, то сделал это с улыбкой; оно должно было стать своего рода прозвищем. Симон был хорошо известен всем своим друзьям как непостоянный и импульсивный человек. Верно, позднее Иисус придал этому данному с легкой руки прозвищу новый и значимый смысл.
\vs p139 2:3 \pc Симон Петр был порывистым человеком, оптимистом. Он вырос, позволяя себе не сдерживать своих сильных эмоций; он постоянно попадал в сложное положение, потому что всегда говорил не подумав. Беспечность этого рода была источником непрестанных неприятностей для всех его друзей и товарищей и являлась причиной многочисленных мягких упреков, со стороны Учителя. Единственной причиной, почему Петр не попал в еще большую беду из\hyp{}за своей привычки говорить не подумав, было то, что он очень рано приучил себя о многих своих планах и замыслах рассказывать своему брату Андрею до того, как решался публично выступить со своими предложениями.
\vs p139 2:4 Петр был прекрасным оратором, красноречивым и наделенным драматическим даром. Он также был прирожденным и пламенным вождем людей, у него был быстрый ум, лишенный, однако, глубокой рассудительности. Он задавал много вопросов, больше, чем все апостолы вместе взятые, и хотя большинство этих вопросов были хорошими и затрагивали суть дела, многие из них были непродуманны и глупы. Петр не обладал глубоким умом, но свои возможности знал достаточно хорошо. Поэтому он был человеком быстрых решений и стремительных действий. В то время, как другие, увидев Иисуса на берегу, в изумлении говорили между собой, Петр бросился в воду и поплыл к берегу, чтобы встретить Учителя.
\vs p139 2:5 \pc Черта, которой Петр восхищался в Иисусе больше всего, была его неземная мягкость. Петр не уставал наблюдать терпеливость Иисуса. Он никогда не забывал урок о прощении грешника не только до семи, но до семидесяти семи раз. Он много думал об этих впечатлениях от всепрощающего характера Учителя во время тех темных и мрачных дней, что сразу последовали за его бездумным и неумышленным отречением от Иисуса на дворе первосвященника.
\vs p139 2:6 \pc Симон Петр был страшно переменчив, он мог неожиданно броситься из одной крайности в другую. Сначала он отказался позволить Иисусу умыть его ноги, а затем, услышав возражения Учителя, умолял умыть себя всего. Но, в конце концов, Иисус знал, что недостатки Петра шли от ума, а не от сердца. Он являл собой одно из самых необъяснимых сочетаний смелости и трусости, когда\hyp{}либо существовавших на земле. Великой силой его характера была преданность, дружба; Петр по\hyp{}настоящему и истинно любил Иисуса. И все же, несмотря на эту неистовую преданность, он был настолько изменчив и непостоянен, что позволил поддразниваниям служанки подвигнуть его отречься от своего Господа и Учителя. Петр мог противостоять преследованиям и любой форме прямого нападения, но слабел и отступал перед насмешкой. Он был храбрым солдатом перед лицом лобовой атаки, но становился трусом, сжавшимся от страха, когда его заставало врасплох нападение с тыла.
\vs p139 2:7 Петр первый из апостолов Иисуса стал на защиту трудов Филиппа среди самарян, а Павла среди неевреев; тем не менее, позднее в Антиохии он отступил от неевреев, столкнувшись с насмешками еврейских последователей Иисуса, что лишь навлекло на него решительное осуждение Павла.
\vs p139 2:8 Он был первым из апостолов, кто от всего сердца признал соединение в Иисусе человеческого и божественного и первым --- не считая Иуды --- отрекшимся от него. Петр был не столько мечтателем, сколько не любил спускаться с облаков исступленного восторга и энтузиазма театральных проявлений в простой и прозаический мир реальности.
\vs p139 2:9 Следуя за Иисусом, и в прямом и в переносном смысле, он либо возглавлял шествие, либо тащился в хвосте --- «следовал издали». Но он был лучшим проповедником из двенадцати; не считая Павла, он больше, чем кто\hyp{}либо другой сделал для установления царства и ради того, чтобы на протяжении жизни одного поколения послать его вестников во все концы земли.
\vs p139 2:10 После своих поспешных отречений от Учителя он собрался с силами и под сочувствующим и понимающим руководством Андрея вновь первым вернулся к рыбацким сетям, тогда как апостолы медлили, пытаясь узнать, что же должно случиться после распятия. Когда он совершенно убедился, что Иисус простил его, и узнал, что он снова принят в окружение Учителя, огни царства загорелись в его душе так ярко, что он стал великим и спасительным светом для тысяч пребывавших во тьме.
\vs p139 2:11 \pc Оставив Иерусалим еще до того, как Павел стал духовным вождем нееврейских христианских церквей, Петр много путешествовал, посещая все церкви от Вавилона до Коринфа. Он даже побывал во многих церквях, которые создал Павел, и служил им. Хотя Петр и Павел обладали совершенно разными характерами, получили различное образование, и даже придерживались разных теологических взглядов, они вместе дружно трудились на благо создания церквей в более поздние годы своей жизни.
\vs p139 2:12 \pc Слог и учение Петра до некоторой степени переданы в проповедях, частично записанных Лукой, и в Евангелии от Марка. Его энергичный стиль еще лучше передан в его письме, известном как Первое Послание Петра; по крайней мере это было так, пока его не изменил один из последователей Павла.
\vs p139 2:13 Однако Петр упорно совершал одну и ту же ошибку, пытаясь убедить евреев, что Иисус все\hyp{}таки был действительно и истинно еврейским Мессией. Вплоть до дня своей смерти Симон Петр так до конца и не разобрался в понятиях: Иисус как еврейский Мессия, Христос как искупитель мира и Сын Человеческий, как откровение Бога, любящего Отца всего человечества.
\vs p139 2:14 \pc Жена Петра была очень способной женщиной. Многие годы она достойно трудилась как член женского отряда, а когда Петр был изгнан из Иерусалима, сопровождала его во всех его путешествиях к церквям, а также во всех его миссионерских экспедициях. И в день, когда ее прославленный муж отдал свою жизнь, она была брошена на растерзание диким зверям на арене в Риме.
\vs p139 2:15 \pc И так сей человек Петр, близкий друг Иисуса, один из близкого круга апостолов, пошел из Иерусалима и возвещал евангелие царства с силой и славой, пока его служение не было исполнено до конца; он считал себя удостоившимся высоких почестей, когда те, кто заточил его в неволю, сообщили ему, что он должен умереть так же, как умер его Учитель, --- на кресте. И так Симон Петр был распят в Риме.
\usection{3. Иаков Зеведеев}
\vs p139 3:1 Иакову, старшему из двух апостолов\hyp{}сыновей Зеведеевых, которых Иисус прозвал «сынами громовыми», было тридцать лет, когда он стал апостолом. Он был женат, имел четверых детей и жил рядом со своими родителями на окраине Капернаума, в Вифсаиде. Он был рыбаком и занимался своим ремеслом вместе со своим младшим братом Иоанном и совместно с Андреем и Симоном. Иаков и его брат Иоанн пользовались тем преимуществом, что знали Иисуса дольше всех остальных апостолов.
\vs p139 3:2 \pc У этого способного апостола был противоречивый характер; казалось, он действительно обладал двумя натурами, каждая из которых питалась сильными чувствами. Особенно неистовым он становился, когда был полон негодования. Стоило его хоть немного рассердить, и он становился вспыльчивым, а когда буря утихала, всегда пытался оправдать и объяснить свой гнев тем, будто он был исключительно проявлением праведного негодования. Если не считать этих периодических всплесков гнева, личность Иакова очень походила на личность Андрея. Он не обладал рассудительностью или способностью проникать в сущность человеческой природы, что было свойственно Андрею, но зато был намного лучшим публичным оратором. После Петра и не считая Матфея Иаков был лучшим публичным оратором среди двенадцати.
\vs p139 3:3 Хотя Иаков ни в коем случае не был человеком, легко поддающимся переменам настроения, он, тем не менее, в один день мог быть тихим и молчаливым, и очень хорошим оратором и рассказчиком в другой. Обычно он свободно говорил с Иисусом, но среди двенадцати апостолов порой целыми днями не произносил ни слова. Его великой слабостью были эти приступы необъяснимого молчания.
\vs p139 3:4 Выдающейся чертой личности Иакова была его способность видеть все стороны того или иного положения. Из двенадцати апостолов он ближе всех подошел к пониманию смысла и значения учения Иисуса. Вначале он тоже недостаточно хорошо уяснил смысл учения Иисуса, но еще до того, как они закончили свое обучение, он лучше всех постиг послание Учителя. Иаков умел понимать широту человеческой натуры; он прекрасно ладил с разносторонним Андреем, импульсивным Петром и своим замкнутым братом Иоанном.
\vs p139 3:5 Хотя Иаков и Иоанн испытывали определенные трудности, пытаясь работать вместе, но отрадно было наблюдать, как они ладили между собой. Им не удавалось добиться такого же единства, как Андрею и Петру, но они преуспели в этом намного больше, чем обычно можно ожидать от двух братьев и особенно таких своевольных и решительных братьев. Но, каким бы странным это ни могло показаться, эти двое сыновей Зеведеевых были намного более терпимы друг к другу, чем к посторонним. Они питали большую любовь друг к другу и всегда были дружны. Именно эти «сыны громовы» хотели призвать огонь с небес, чтобы покарать самарян, осмелившихся выказать неуважение к их Учителю. Однако безвременная кончина Иакова сильно переменила горячий нрав его младшего брата Иоанна.
\vs p139 3:6 \pc Характерной чертой Иисуса, которой Иаков восхищался больше всего, была доброжелательность Учителя. Участливое отношение Иисуса к малому и великому, богатому и бедному очень нравился ему.
\vs p139 3:7 \pc Иаков Зеведеев был уравновешенным мыслителем и составителем планов. Так же, как и Андрей, он был одним из наиболее рассудительных в группе апостолов. Он был энергичным человеком, но никогда не спешил. Он являлся прекрасным противовесом Петру.
\vs p139 3:8 Он был скромным и простым человеком, повседневным служителем, и после того, как в какой\hyp{}то мере понял настоящее значение царства, не искал себе особой награды. Даже читая рассказ о том, как мать Иакова и Иоанна попросила, чтобы ее сыновьям была оказана честь сидеть по правую и левую руку Иисуса, необходимо помнить, что просила об этом именно мать. Следует также признать: когда Иаков и Иоанн дали понять, что готовы к подобной ответственности, они догадывались об опасностях, сопутствовавших восстанию, которое, как они предполагали, Учитель поднимет против власти Рима, и были готовы платить за это. Когда Иисус спросил, готовы ли они выпить чашу, они ответили, что готовы. И, что касается Иакова, это было в прямом смысле так --- он выпил чашу вместе с Учителем, став первым из апостолов, испытавшим мученичество, и раньше других принял смерть от меча Ирода Агриппы. Иаков, таким образом, был первым из двенадцати, пожертвовавшим своей жизнью в боевом строю царства. Ирод Агриппа боялся Иакова больше, чем всех остальных апостолов. Иаков действительно часто бывал спокоен и молчалив, но когда затрагивались и оспаривались его убеждения, он был храбр и решителен.
\vs p139 3:9 \pc Иаков прожил свою жизнь как должно, и когда пришел конец, держался с таким достоинством и с такой стойкостью, что даже его обвинитель и доносчик, который присутствовал на суде и его казни, был тронут настолько, что бежал с места смерти Иакова, чтобы присоединиться к последователям Иисуса.
\usection{4. Иоанн Зеведеев}
\vs p139 4:1 Когда Иоанн стал апостолом, ему было двадцать четыре года и он был самым молодым из двенадцати. Он был холост и жил со своими родителями в Вифсаиде; он был рыбаком и трудился со своим братом Иаковом вместе с Андреем и Петром. Еще до того, как стать апостолом, и после этого Иоанн действовал как личный представитель Иисуса в отношениях с семьей Учителя и продолжал исполнять эту обязанность до тех пор, пока была жива Мария, мать Иисуса.
\vs p139 4:2 Иоанн, как самый младший из двенадцати и столь тесно связанный с Иисусом в его семейных делах, был очень дорог Учителю, однако было бы неправдой сказать, что он являлся «учеником, которого любил Иисус». Едва ли такую великодушную личность, как Иисус, можно заподозрить в фаворитизме, в большей любви к одному из своих апостолов, чем к остальным. То, что Иоанн являлся одним из трех личных помощников Иисуса, придавало дополнительные оттенки этой ошибочной идее, не говоря уже о том, что Иоанн вместе со своим братом Иаковом знал Иисуса дольше, чем остальные.
\vs p139 4:3 \pc Петр, Иаков и Иоанн были назначены личными помощниками Иисуса через какое\hyp{}то время после того, как стали апостолами. Вскоре после избрания двенадцати, назначая Андрея руководителем группы, Иисус сказал ему: «А теперь я желаю, чтобы ты двум или трем из твоих товарищей поручил быть со мной и оставаться рядом со мной, поддерживать меня и служить моим ежедневным нуждам». И Андрей решил, что лучше всего для исполнения этой особой обязанности будет избрать именно трех первозванных апостолов. Для столь благословенного служения он был бы рад предложить самого себя, но Учитель уже дал ему поручение; поэтому он немедленно распорядился, чтобы Петр, Иаков и Иоанн были рядом с Иисусом.
\vs p139 4:4 \pc Иоанн Зеведеев обладал множеством прекрасных черт характера, однако одна из них была не особенно привлекательной и заключалась в его чрезмерном, но, как правило, хорошо скрываемом тщеславии. Вследствие продолжительного общения с Иисусом его характер во многом значительно изменился. И это тщеславие сильно убавилось, но когда Иоанн состарился и почти впал в детство, подобная самооценка до определенной степени возобладала вновь, так что, руководя Натаном в написании Евангелия, которое теперь носит имя Иоанна, старый апостол без стеснения назвал себя «учеником, которого любил Иисус». Ввиду того, что Иоанн ближе всех остальных смертных подошел к тому, чтобы стать приятелем Иисуса и что он являлся его избранным личным представителем в столь многих делах, неудивительно, что он стал смотреть на себя как на «ученика, которого Иисус любил», ведь Иоанн совершенно точно знал, что он был учеником, которому Иисус так часто доверял.
\vs p139 4:5 Самой положительной чертой характера Иоанна была его надежность; он был исполнителен и храбр, верен и предан. Его величайшей слабостью было это свойственное ему тщеславие. Он был самым младшим членом семьи своего отца и самым молодым в группе апостолов. Быть может, он был немного избалован; возможно, ему потакали чуть больше, чем нужно. Но Иоанн более поздних лет был личностью совершенно иного типа и сильно отличался от того самовлюбленного и самоуправного молодого человека, который вступил в ряды апостолов Иисуса, когда ему было двадцать четыре года.
\vs p139 4:6 \pc Иоанн больше всего ценил у Иисуса любовь и бескорыстие Учителя; эти черты произвели на него такое сильное впечатление, что вся последующая жизнь Иоанна была подчинена чувству любви и братской преданности. Он говорил о любви и писал о любви. Этот «сын громов» стал «апостолом любви»; и в Ефесе, когда престарелый епископ больше не мог проповедовать, стоя на кафедре, и его вносили в церковь на кресле, когда в конце службы у него просили сказать верующим несколько слов, в течение многих лет его единственное высказывание было таково: «Дети мои малые, любите друг друга».
\vs p139 4:7 \pc За исключением случаев, когда давал себя знать нрав Иоанна, он был человеком немногословным. Он много думал, но говорил мало. С возрастом его характер смягчился, стал более управляемым, но Иоанн так и не преодолел своего нежелания говорить и так и не сумел полностью побороть свою молчаливость. Однако он обладал даром замечательного и творческого воображения.
\vs p139 4:8 \pc Характер Иоанна имел еще одну сторону, о присутствии которой у такого тихого и погруженного в себя человека, догадаться было нельзя. Он был отчасти фанатичен и чрезмерно нетерпим. В этом отношении он и Иаков были очень похожи --- они оба хотели навлечь огонь с неба на головы непочтительных самарян. Когда Иоанн встретил неких странников, учивших во имя Иисуса, то сразу запретил им делать это. Однако он не был единственным из двенадцати, кому были свойственны подобного рода самомнение и сознание собственного превосходства.
\vs p139 4:9 То, как жил Иисус, не имея дома, оказало на жизнь Иоанна огромное влияние, ибо он знал, как преданно Иисус заботился о своей матери и своей семье. Иоанн также глубоко сочувствовал Иисусу, так как близкие Иисуса не сумели его понять, и Иоанн сознавал, что они постепенно отдаляются от него. Вся эта ситуация в сочетании с постоянным подчинением Иисусом своих малейших желаний воле Отца Небесного и его повседневная жизнь, полная упования, производила на Иоанна такое сильное впечатление, что стала причиной заметных и глубоких перемен в его характере, перемен, проявлявшихся на протяжении всей его последующей жизни.
\vs p139 4:10 Иоанн обладал хладнокровным и дерзким мужеством, которое было лишь у немногих из апостолов. Он был единственным апостолом, который оставался рядом с Иисусом в ночь его ареста и осмелился сопровождать своего Учителя в самую пасть смерти. Он был неотступно рядом вплоть до последнего земного часа и честно исполнял свой долг в отношении матери Иисуса и был готов получить такие дополнительные наставления, какие могли быть даны ему в последние мгновения смертного существования Учителя. Несомненно одно: Иоанн был человеком, на которого можно было положиться во всем. Когда двенадцать апостолов трапезничали, Иоанн, как правило, сидел по правую руку от Иисуса. Он был первым из двенадцати, кто по\hyp{}настоящему и полностью поверил в воскресение, и первым, кто узнал Учителя, когда тот пришел на берег моря после своего воскресения.
\vs p139 4:11 Этот сын Зеведеев был тесно связан с Петром на ранних этапах христианского движения, став одним из главных сторонников Иерусалимской церкви. Он был правой рукой Петра в день Пятидесятницы.
\vs p139 4:12 Спустя несколько лет после мученичества Иакова, Иоанн женился на вдове своего брата. Последние двадцать лет его жизни о нем заботилась любившая его внучка.
\vs p139 4:13 Иоанн несколько раз сидел в тюрьме и в течение четырех лет находился в изгнании на острове Патмос, пока к власти в Риме не пришел новый император. Если бы Иоанн не был осторожен и прозорлив, то его несомненно убили бы, как убили его более откровенного брата Иакова. Шли годы, и Иоанн вместе с Иаковом, братом Господним, выступая перед гражданскими судьями, научились применять метод мудрого примирения. Они поняли, что «кроткий ответ отвращает гнев». Они также учились представлять церковь как «духовное братство, посвященное общественному служению человечеству», а не как «царство небесное». Они учили о служении, исполненном любви, а не о правящей силе --- царстве и царе.
\vs p139 4:14 Находясь во временном изгнании на Патмосе, Иоанн написал Книгу Откровения, которую теперь вы имеете в крайне сокращенном и искаженном виде. Эта Книга Откровения содержит уцелевшие фрагменты великого откровения, значительные части которого были утеряны, а другие изъяты после написания ее Иоанном. Она сохранилась лишь в отрывочном и извращенном виде.
\vs p139 4:15 Иоанн много путешествовал, непрестанно трудился и после того, как стал епископом азиатских церквей, поселился в Ефесе. В Ефесе в возрасте девяноста девяти лет он руководил своим сотоварищем Натаном в написании так называемого «Евангелия от Иоанна». Из всех двенадцати апостолов Иоанн Зеведеев в конце концов стал выдающимся теологом. Он умер естественной смертью в Ефесе в 103 году н.э., когда ему был сто один год.
\usection{5. Филипп Любопытный}
\vs p139 5:1 Филипп был избран в апостолы пятым по счету и призван, когда Иисус и его первые четыре апостола шли со свидания с Иоанном от Иордана в Кану Галилейскую. Поскольку Филипп жил в Вифсаиде, он какое\hyp{}то время кое\hyp{}что знал об Иисусе, однако до того дня в Иорданской долине, когда Иисус сказал ему: «Следуй за мной», у него и в мыслях не было, что Иисус --- по\hyp{}настоящему великий человек. Некоторое влияние на Филиппа также оказало то, что Андрей, Петр, Иаков и Иоанн приняли Иисуса как Избавителя.
\vs p139 5:2 Филиппу было двадцать семь лет, когда он присоединился к апостолам; к этому времени он был женат, но детей еще не имел. Прозвище, которое ему дали апостолы, подчеркивало его «любопытство». Филипп всегда хотел иметь доказательства чего бы то ни было. Похоже, он никогда глубоко не вникал ни в одно утверждение. Вовсе не означало, что он был глуп, просто ему не хватало воображения. Этот недостаток воображения был великой слабостью характера Филиппа. Он был обыкновенным и заурядным человеком.
\vs p139 5:3 \pc Когда апостолам указали их служение, Филиппа сделали экономом; в его обязанности входило следить за тем, чтобы у них всегда было достаточно пищи. И экономом он был хорошим. Самой сильной чертой характера Филиппа была его методическая тщательность; он был и точен, и последователен.
\vs p139 5:4 Филипп происходил из семьи, в которой было семеро детей, три мальчика и четыре девочки. Среди детей он был вторым по старшинству и после воскресения Иисуса окрестил всю свою семью в царство. Семья Филиппа занималась рыболовством. Его отец был очень способным человеком, глубокомысленным, однако мать его происходила из весьма посредственной семьи. Филипп не был человеком, от которого можно было ожидать больших свершений, но зато он был человеком, который умел делать малые дела так, словно они были большими, делать их хорошо и как следует. Лишь несколько раз за четыре года ему не удалось добыть достаточно пищи, которой хватило бы для всех. Даже многочисленные потребности, возникавшие при чрезвычайных обстоятельствах, сопровождавших жизнь, которой они жили, редко заставали его неподготовленным. Хозяйство семьи апостолов управлялось разумно и умело.
\vs p139 5:5 Сильной чертой характера Филиппа была его неизменная надежность; слабой чертой его характера было крайне слабо развитое воображение, отсутствие способности к двум прибавить два и получить четыре. Его подход к абстрактным понятиям был математическим, ему недоставало творческого воображения. Определенные виды воображения у него почти полностью отсутствовали. Он был самым обычным и средним человеком. Среди масс, приходивших послушать, как учит и проповедует Иисус, подобных мужчин и женщин было великое множество, и они получали огромное утешение, видя, что такой же, как они, человек возведен в почетную должность в советах Учителя; то, что подобный им человек занял высокое положение в делах царства, придавало им смелости. И Иисус, терпеливо слушая глупые вопросы Филиппа и столь много раз уступая просьбам своего эконома «доказать ему», многое узнал о том, как функционируют некоторые человеческие умы.
\vs p139 5:6 Тем качеством Иисуса, которым столь постоянно восхищался Филипп, была неизменная щедрость Учителя. Филипп не мог найти в Иисусе хоть чего\hyp{}нибудь мелочного, скупого или скаредного, и восхищался этой постоянной и неизменной щедростью.
\vs p139 5:7 \pc В личности Филиппа было мало запоминающегося. О нем часто говорили как о «Филиппе из Вифсаиды, города, где живут Андрей и Петр». Он был почти лишен проницательности и не был способен понять огромные возможности данной ситуации. Он не был пессимистом; он был просто прозаичен. Ему также в значительной степени не хватало духовного понимания. Он, не задумываясь, мог прервать Иисуса на середине одного из самых замечательных рассуждений Учителя, чтобы задать откровенно глупый вопрос. Но Иисус никогда не упрекал его за подобное безрассудство; он был терпелив к нему и учитывал его неспособность понять более глубокий смысл учения. Иисус хорошо знал: если он хотя бы раз упрекнет Филиппа за эти надоедливые вопросы, то тем самым не только ранит эту честную душу, но подобный упрек причинит ему столько боли, что он больше никогда не будет чувствовать себя свободным задавать вопросы. Иисус знал, что в его мирах пространства живут бессчетные миллиарды похожих недостаточно сообразительных смертных, и хотел ободрить их всех, чтобы они смотрели на него и всегда могли без стеснения прийти к нему со своими вопросами и проблемами. В конце концов Иисуса действительно больше интересовали глупые вопросы Филиппа, чем проповедь, которую он мог бы произносить. Иисуса в высшей степени интересовали \bibemph{люди,} все типы людей.
\vs p139 5:8 Эконом апостолов не был хорошим публичным оратором, но он умел очень убедительно говорить и с успехом вел индивидуальную работу. Сбить его с толку было непросто; он был тружеником, упорным во всем, за что бы ни брался. Он обладал этим великим и редким даром умения сказать: «Идем». Когда Нафанаил, первый человек, которого обратил Филипп, захотел поспорить о достоинствах и недостатках Иисуса из Назарета, убедительный ответ Филиппа был таков: «Пойди и посмотри». Он не был безапелляционным проповедником, который призывал своих слушателей «идти» --- делать то и делать это. Во всех ситуациях, возникавших в его работе, он говорил: «Пойди» --- «Пойди со мной; я покажу тебе путь». А это всегда эффективный метод во всех формах и на всех этапах учения. Даже родители могут поучиться у Филиппа и \bibemph{не} говорить своим детям: «Иди и делай то и иди и делай это», но говорить: «Пойди с нами, и мы покажем тебе и научим тебя, как делать лучше».
\vs p139 5:9 Неспособность Филиппа приспособиться к новой ситуации полностью проявилась, когда в Иерусалиме к нему подошли греки и сказали: «Господин, мы желали бы видеть Иисуса». Любому еврею, обратившемуся с подобной просьбой, Филипп сказал бы: «Пойдем». Но эти люди были иностранцами, и Филипп не мог вспомнить ни одного наставления своих начальников в отношении подобных вопросов; поэтому единственное, что он смог придумать, было посоветоваться с главой апостолов Андреем, после чего они вместе отвели обратившихся с просьбой греков к Иисусу. Подобно тому, идя в Самарию, проповедуя и крестя верующих, как наставлял его Учитель, Филипп воздерживался от возложения рук на тех, кого он обращал, в знак принятия Духа Истины. Это делалось Петром и Иоанном, которые вскоре пришли из Иерусалима, чтобы понаблюдать за его работой на благо матери\hyp{}церкви.
\vs p139 5:10 Филипп прошел через тяжелые времена смерти Учителя, участвовал в реорганизации двенадцати и был первым, кто отправился обращать души к царству среди ближайших соседей евреев и добился наибольших успехов в своей работе на благо самарян и во всех своих последующих трудах на благо евангелия.
\vs p139 5:11 \pc Жена Филиппа, которая была активным членом женского отряда, тесно сотрудничала со своим мужем в его евангелической работе после их бегства от преследований в Иерусалиме. Его жена была бесстрашной женщиной. Она стояла у основания креста Филиппа, ободряя его провозглашать благую весть даже своим убийцам, и когда силы оставили его, стала рассказывать историю о спасении через веру в Иисуса и замолчала лишь тогда, когда разгневанные евреи набросились на нее и до смерти забили камнями. Их старшая дочь Лия продолжила их дело, позднее став знаменитой пророчицей Иераполя.
\vs p139 5:12 \pc Филипп, в прошлом эконом двенадцати апостолов, был в царстве могущественным человеком и обращал к нему души, куда бы он ни шел; в конце концов за свою веру он был распят и похоронен в Иераполе.
\usection{6. Нафанаил Честный}
\vs p139 6:1 Нафанаила, шестого и последнего из апостолов, избранных самим Учителем, привел к Иисусу его друг Филипп. Нафанаил вместе с Филиппом участвовал в нескольких деловых предприятиях и вместе с ним шел, чтобы увидеть Иоанна Крестителя, когда они встретились с Иисусом.
\vs p139 6:2 Когда Нафанаил присоединился к апостолам, ему было двадцать пять лет, и в группе он был вторым по молодости. Он был младшим из семерых детей в семье, был холост и являлся единственной опорой престарелых и немощных родителей, вместе с которыми жил в Кане; его братья и сестры были женаты и замужем, либо умерли, и никто из них там не жил. Нафанаил и Иуда Искариот были двумя самыми образованными людьми среди двенадцати апостолов. Нафанаил собирался стать купцом.
\vs p139 6:3 \pc Иисус сам не давал прозвища Нафанаилу, но вскоре двенадцать стали говорить о нем словами, которые свидетельствовали о его честности и искренности. В нем «не было никакого лукавства». И это было его великой добродетелью; он был и честен, и искренен. Слабой чертой характера Нафанаила была его гордость; он очень гордился своей семьей, своим городом, своей репутацией и своей нацией, что в общем достойно похвалы, если в этом не заходить слишком далеко. Однако во многих из своих предубеждений Нафанаил был склонен доходить до крайностей. Ему было свойственно, не выслушав людей, судить о них в соответствие со своими личными взглядами. Еще не видев Иисуса, он, тем не менее, не замедлил задать вопрос: «Из Назарета может ли быть что доброе?» Но Нафанаил не был упрям, даже если и был горд. Однажды посмотрев в лицо Иисуса, он быстро изменил свои взгляды.
\vs p139 6:4 Во многих отношениях среди двенадцати апостолов Нафанаил был странным гением. Он был апостолом\hyp{}философом и апостолом\hyp{}мечтателем, но мечтателем весьма практического свойства. Периоды глубоких философских раздумий сменялись у него периодами редкого и тонкого юмора; в хорошем настроении он становился рассказчиком, лучшим из двенадцати. Иисус получал огромное наслаждение, слушая рассуждения Нафанаила и о важном, и о пустяках. Нафанаил все серьезнее и серьезнее относился к Иисусу и к царству, но никогда не принимал всерьез самого себя.
\vs p139 6:5 Все апостолы любили и уважали Нафанаила, и он прекрасно ладил со всеми из них, кроме Иуды Искариота. Иуда не считал, что Нафанаил относится к своему апостольству достаточно серьезно, и однажды осмелился тайно подойти к Иисусу и пожаловаться на него. Иисус сказал: «Будь осторожен, Иуда; не переоценивай свое положение. Кто из нас имеет право судить своего брата? Воля Отца отнюдь не в том, чтобы дети его занимались в жизни только серьезными делами. Позволь мне повторить: я пришел, чтобы братья мои во плоти могли иметь радость, веселье и жизнь с избытком. Поэтому иди, Иуда, и делай хорошо то, что тебе велели, но предоставь брату твоему Нафанаилу самому отвечать за себя перед Богом». И память об этом и о многих других случаях долго сохранялась в самозаблуждающемся сердце Иуды Искариота.
\vs p139 6:6 Множество раз, когда Иисус уходил и был на горе с Петром, Иаковом и Иоанном, и отношения между апостолами становились натянутыми и запутанными, когда даже Андрей пребывал в сомнении относительно того, что сказать своим безутешным братьям, Нафанаил снимал напряжение небольшой порцией философии или вспышкой юмора; и притом тонкого юмора.
\vs p139 6:7 В обязанности Нафанаила входило заботиться о семьях двенадцати апостолов. Он часто отсутствовал на апостольских советах, ибо, услышав о том, что с кем\hyp{}нибудь из его подопечных случалась болезнь или что\hyp{}нибудь из ряда вон выходящее, он, не теряя времени, спешил к тому дому. Двенадцать апостолов чувствовали себя спокойно, зная, что благополучие их семей находится в надежных руках Нафанаила.
\vs p139 6:8 \pc Нафанаил больше всего чтил Иисуса за его терпимость. Ему никогда не надоедало размышлять о широте взглядов и щедром сочувствии Сына Человеческого.
\vs p139 6:9 \pc Отец Нафанаила (Варфоломей) умер вскоре после Пятидесятницы, после чего апостол пошел в Месопотамию и Индию, возвещая благую весть царства и крестя верующих. Его братья так и не узнали, что сталось с их бывшим философом, поэтом и юмористом. Однако в царстве он был тоже великим человеком и многое сделал для распространения учений своего Учителя, хотя и не участвовал в создании христианской церкви, возникшей впоследствии. Умер Нафанаил в Индии.
\usection{7. Матфей Левий}
\vs p139 7:1 Седьмой апостол, Матфей, был избран Андреем. Матфей принадлежал к семье сборщиков налогов, или мытарей, но сам был сборщиком таможенных податей в Капернауме, где и жил. Ему был тридцать один год, он был женат и имел четверых детей. Матфей был человеком среднего достатка, единственным в апостольском корпусе, располагавшим какими\hyp{}то средствами. Он был хорошим коммерсантом, очень общительным человеком и был одарен способностью находить друзей и ладить с самыми разными людьми.
\vs p139 7:2 \pc Андрей назначил Матфея финансовым представителем апостолов. Он был своего рода финансовым агентом и представителем по связям с общественностью апостольской организации. Он был глубоким знатоком человеческой природы и очень сильным пропагандистом. Он --- личность, которую трудно точно описать, однако он был крайне серьезным учеником и все больше и больше верил в миссию Иисуса и реальность царства. Иисус так и не дал Левию прозвища, но его собратья\hyp{}апостолы обычно говорили о нем как о «добытчике денег».
\vs p139 7:3 Сильной чертой характера Левия была его беззаветная преданность делу. То, что он, мытарь, был принят Иисусом и его апостолами, явилось причиной огромной благодарности со стороны бывшего сборщика налогов. Однако потребовалось какое\hyp{}то, хоть и незначительное время, чтобы остальные апостолы, и особенно Симон Зилот и Иуда Искариот, примирились с присутствием среди них мытаря. Слабостью Матфея были его недальновидные и материалистические взгляды на жизнь. Однако по прошествии месяцев он продвинулся далеко вперед во всех этих вопросах. Конечно же он по необходимости отсутствовал во время наиболее ценных наставлений, ибо его обязанностью было восполнение казны.
\vs p139 7:4 Больше всего Матфей ценил готовность Учителя прощать. Он не переставал говорить, что в деле обретения Бога необходима лишь вера. Ему всегда нравилось говорить о царстве как об «этом деле обретения Бога».
\vs p139 7:5 \pc Хотя Матфей был человеком с определенным прошлым, он, тем не менее, производил на всех прекрасное впечатление и со временем его сподвижники стали гордиться работой мытаря. Он был одним из апостолов, кто делал подробные записи того, о чем говорил Иисус, и эти записи легли в основу последующего повествования Исадора о словах и деяниях Иисуса, которые стали известны как Евангелие от Матфея.
\vs p139 7:6 Великая и полезная жизнь Матфея, коммерсанта и сборщика таможенных налогов из Капернаума, послужила тому, что тысячи тысяч других коммерсантов, чиновников и политиков в последующие века могли тоже услышать зовущий голос Учителя, который говорит: «Следуй за мной». Матфей, действительно, был расчетливым политиком, но он был чрезвычайно верен Иисусу и в высшей степени предан делу заботы о том, чтобы вестники грядущего царства были достаточно обеспечены материально.
\vs p139 7:7 Присутствие Матфея среди апостолов служило тому, чтобы врата царства были широко открыты для массы впавших в уныние и отвергнутых душ, которые давно уже считали, что им невозможно получить религиозное утешение. Отвергнутые и отчаявшиеся мужчины и женщины собирались толпами, чтобы послушать Иисуса, и он ни разу не отверг никого из них.
\vs p139 7:8 \pc Матфей получал пожертвования, которые без всякого принуждения давали ему верующие последователи и непосредственные слушатели учений Учителя, однако он никогда прямо не просил денег у народных масс. Всю свою финансовую работу он вел тихо и индивидуально и большую часть денег собирал среди наиболее состоятельного класса заинтересованных верующих. Практически все свое скромное состояние он отдал на благо работы Учителя и его апостолов, но об этой щедрости не узнал никто, кроме Иисуса, которому об этом было известно все. Матфей не решался открыто делать свой вклад в казну апостолов, ибо боялся, что Иисус и его соратники могут посчитать его деньги грязными; поэтому он много давал от имени других верующих. В течение первых месяцев, когда Матфей понимал, что его присутствие среди них было в большей или меньшей степени испытанием, он чувствовал сильное искушение рассказать им, что часто насущный хлеб дают им его деньги, но не поддался ему. Когда проявления презрения к мытарю становились явными, Левий сгорал от желания показать им свою щедрость, но всегда сдерживал себя.
\vs p139 7:9 Когда денег на неделю было меньше, чем в принципе требовалось, Левий часто много брал из своих собственных средств. Также, глубоко заинтересовавшись учением Иисуса, он иногда предпочитал остаться и послушать наставления, хотя и знал, что восполнять недостаток собранных средств придется лично ему. И все же Левию очень хотелось, чтобы Иисус мог узнать, что много денег поступало из его кармана! Он не сознавал, что Учитель знает об этом все. Все апостолы умерли, так и не узнав, что Матфей был их благотворителем до такой степени, что, когда после начала преследований пошел возвещать евангелие царства, он был практически без гроша.
\vs p139 7:10 \pc Когда эти гонения вынудили верующих оставить Иерусалим, Матфей отправился на север, проповедуя евангелие царства и крестя верующих. Его бывшие соратники\hyp{}апостолы ничего не знали о нем, но он продолжал идти, проповедуя и крестя, через Самарию, Капподокию, Галатию, Вифинию и Фракию. Именно во Фракии в Лисимахии некие неверующие евреи вступили в сговор с римскими солдатами, чтобы те убили его. И сей духовно возрожденный мытарь умер победителем с верой в спасение, о которой все больше узнавал из учений Учителя на протяжении последних лет своего пребывания на земле.
\usection{8. Фома Дидымус}
\vs p139 8:1 Фома был восьмым апостолом, и его избрал Филипп. Позднее он стал известен как «Фома неверующий», но его собратья\hyp{}апостолы едва ли считали его неверующим. Верно, у него был логический, скептический ум, но он обладал той отважной преданностью, которая не позволяла тем, кто знал его близко, считать его ограниченным скептиком.
\vs p139 8:2 Когда Фома присоединился к апостолам, ему было двадцать девять лет, он был женат и имел четверых детей. В прошлом Фома был плотником и каменщиком, но позднее стал рыбаком и жил в Тарихее, расположенной на западном берегу Иордана, там, где река вытекает из Галилейского моря, и считался самым незаурядным жителем этого небольшого селения. Он был мало образован, но обладал острым, рассудительным умом и был сыном прекрасных родителей, которые жили в Тивериаде. Из двенадцати апостолов один Фома обладал поистине аналитическим умом; в группе апостолов он был настоящим ученым.
\vs p139 8:3 Ранняя жизни Фомы было неудачной; его родители не были вполне счастливы в браке, и это отразилось во взрослой жизни Фомы. Он вырос со сварливым и придирчивым характером. Даже его жена радовалась тому, что он присоединился к апостолам; мысль о том, что ее все видящий в мрачном свете муж большую часть времени будет вне дома, приносила ей облегчение. Фома был также склонен к подозрительности, отчего мирно уживаться с ним было весьма трудно. Вначале Петр был очень недоволен Фомой и жаловался своему брату Андрею, что Фома «зол, противен и всегда подозрителен». Однако чем лучше соратники узнавали Фому, тем больше он им нравился. Они поняли, что он был в высшей степени честным и беззаветно преданным человеком. Он был совершенно искренним и безусловно правдивым, но он же был от природы въедливым и, когда вырос, стал настоящим пессимистом. Его аналитический ум был чрезвычайно подозрителен. Он уже почти потерял веру в людей, когда связал себя с двенадцатью апостолами и, таким образом, соприкоснулся с возвышенным характером Иисуса. Эта связь с Учителем сразу же начала изменять весь его характер и производить великие перемены в его внутреннем отношении к людям.
\vs p139 8:4 Великая сила Фомы была в его превосходном аналитическом уме, сочетавшемся с непоколебимой смелостью, которой он наполнялся, стоило ему только принять решение. Его великая слабость была в том, что он всех подозревал и во всем сомневался, и этот недостаток за всю свою жизнь во плоти он преодолеть так и не смог.
\vs p139 8:5 В организации двенадцати апостолов Фоме поручалось составлять и планировать маршруты, и он весьма искусно управлял работой и передвижениями апостольского корпуса. Он был хорошим распорядителем, превосходным коммерсантом, но ему мешали частые перемены настроения; в один день он был одним человеком, а на следующий становился другим. Когда Фома присоединился к апостолам, он был склонен к меланхолии, однако общение с Иисусом и апостолами в значительной степени избавило его от этой болезненной интроспекции.
\vs p139 8:6 Иисус любил общаться с Фомой и имел с ним много продолжительных бесед. Его присутствие среди апостолов служило великим утешением всем честным склонным к сомнению людям и помогло многим смятенным умам войти в царство, даже если они не могли до конца понять все духовные и философские аспекты учений Иисуса. То, что Фома принадлежал к двенадцати, служило убедительным подтверждением того, что Иисус любил даже искренних сомневающихся.
\vs p139 8:7 \pc Другие апостолы чтили Иисуса за какую\hyp{}либо особую или выдающуюся черту его богатой личности, но Фома почитал Учителя за его в высшей степени уравновешенный характер. Фома все больше и больше восхищался и чтил того, кто был столь полон любви и милосерден и вместе с тем столь справедлив и беспристрастен; столь тверд, но никогда не упрям; столь спокоен, но всегда небезразличен; кто так стремился помочь и так сострадал, но никогда не вмешивался в чужие дела и не навязывал свою волю; кто был столь силен и одновременно нежен; кто был столь уверен, но никогда не был груб или невежлив; кто был столь мягок, но никогда не был нерешителен; кто был столь чист и столь невинен и одновременно мужественен, настойчив и силен; кто был столь истинно отважен, но никогда не был безрассуден или отчаян; кто так любил природу, но был настолько свободен от всякой склонности поклоняться ей; кто был столь весел и так любил шутить, но был настолько лишен легкомыслия и фривольности. Именно эта неповторимая гармоничность личности так очаровывала Фому. Вероятно, из двенадцати апостолов он обладал высочайшим интеллектуальным пониманием Иисуса и способностью оценить его личность.
\vs p139 8:8 \pc В советах двенадцати апостолов Фома был всегда осторожен, в первую очередь отстаивая политику безопасности, однако если его консерватизм отвергался или отклонялся, он всегда был первым, кто бесстрашно приступал к осуществлению выбранной программы. Он снова и снова выступал против какого\hyp{}то плана как безрассудного и самонадеянного; он спорил до самого конца, но когда Андрей ставил предложение на голосование и после того, как двенадцать апостолов принимали решение делать то, против чего он столь энергично возражал, Фома первым говорил: «Пойдем!» Он умел проигрывать. Он не держал злобы и не пестовал уязвленные чувства. Он снова и снова возражал против того, чтобы Иисус подвергал себя опасности, однако когда Учитель решался пойти на такой риск, Фома всегда сплачивал апостолов бесстрашными словами: «Вставайте, друзья, пойдем и умрем с ним».
\vs p139 8:9 В некоторых отношениях Фома был похож на Филиппа; он тоже хотел, чтобы «ему показали», однако внешние проявления сомнений у него опирались на совершенно иные интеллектуальные доводы. Фома был аналитиком, а не просто скептиком. Во всем, что касалось физической смелости, он был самым храбрым из двенадцати.
\vs p139 8:10 \pc Иногда Фома бывал чрезвычайно подавлен; порой бывал печален и удручен. Утрата сестры\hyp{}близнеца в возрасте девяти лет причинила ему много страданий и плохо сказалась на его характере в дальнейшей жизни. Когда Фома впадал в уныние, избавиться от него ему иногда помогал Нафанаил, иногда Петр и нередко один из близнецов Алфеевых. К сожалению, в периоды самого угнетенного состояния он всегда старался избежать прямого общения с Иисусом. Но Учитель знал об этом все и относился к своему апостолу с понимающим сочувствием, когда тот страдал от депрессии и мучился от сомнений.
\vs p139 8:11 Иногда Фома получал от Андрея разрешение уединиться на день, другой. Однако он вскоре осознал, что подобный путь неразумен; он быстро понял, что в период уныния лучше всего с головой уходить в работу и держаться ближе к своим товарищам. Однако, что бы ни происходило в его эмоциональной жизни, он продолжал быть апостолом. Когда действительно приходило время двигаться вперед, именно Фома всегда говорил: «Пойдем!»
\vs p139 8:12 Фома являет собой великий пример человека, который, сомневаясь, не отступает и побеждает. У него был великий ум; он не был придирчивым критиком. Он мыслил логически и являл собой пробный камень для Иисуса и своих соратников\hyp{}апостолов. Если бы Иисус и его дело не были истинными, то оно не смогло бы от начала до конца удержать такого человека, как Фома. Фома обладал острым и уверенным чувством \bibemph{правды.} При первом же признаке обмана или мошенничества Фома бы оставил их всех. Ученые не могут до конца понять все об Иисусе и его труде на земле, но с Учителем и его соратниками\hyp{}людьми жил и работал человек, чей ум был умом истинного ученого; это был Фома Дидымус, и он верил в Иисуса из Назарета.
\vs p139 8:13 \pc Фома пережил тяжелое время во дни суда и распятия. На какое\hyp{}то время он погрузился в бездну отчаяния, но, воспрянув духом, примкнул к апостолам и был вместе с ними, чтобы приветствовать Иисуса у Галилейского моря. На какое\hyp{}то время он поддался своей полной сомнений депрессии, но в конце концов мобилизовал свою веру и мужество. После Пятидесятницы он дал мудрый совет апостолам и, когда преследования рассеяли верующих, отправился на Кипр, затем на Крит, потом на северный берег Африки и на Сицилию, проповедуя благую весть и крестя верующих. И Фома продолжал проповедовать и крестить до тех пор, пока не был арестован представителями римского правительства и предан смерти на Мальте. Всего за несколько недель до своей смерти он начал писать о жизни и учениях Иисуса.
\usection{9 и 10. Иаков и Иуда Алфеевы}
\vs p139 9:1 Сыновья Алфея Иаков и Иуда, рыбаки\hyp{}близнецы, жившие неподалеку от Хересы, были девятым и десятым апостолами, которых избрали Иаков и Иоанн Зеведеевы. Им было по двадцать шесть лет, они были женаты, и у Иакова было трое детей, а у Иуды --- двое.
\vs p139 9:2 \pc Немногое можно сказать об этих двух простых рыбаках. Они любили своего Учителя, и Иисус любил их, но они никогда не прерывали его рассуждения вопросами. Они очень мало понимали в философских дискуссиях или теологических спорах своих товарищей\hyp{}апостолов, но радовались тому, что были причислены к подобной группе могущественных людей. Эти два человека почти не отличались своей наружностью, умственными качествами и глубиной духовного понимания. Все, что можно сказать об одном, можно отнести и к другому.
\vs p139 9:3 Андрей поручил им работу по наблюдению за порядком в толпе. Они были главными стражами порядка в часы проповедей и фактически слугами и посыльными двенадцати апостолов. Они помогали Филиппу собирать запасы, относили деньги семьям вместо Нафанаила и всегда были готовы протянуть руку помощи любому из апостолов.
\vs p139 9:4 Масса простого люда чувствовала великое воодушевление, видя, что двое подобных им удостоились места среди апостолов. Само принятие этих двух ничем не выделявшихся близнецов в апостолы послужило обращению великого множества робко верующих к царству. И простолюдины также более благосклонно относились к тому, что ими официально управляют и руководят стражники порядка, очень похожие на них самих.
\vs p139 9:5 У Иакова и Иуды, которых также называли Фаддеус и Леббеус, не было ни сильных, ни слабых черт характера. Прозвища, данные им учениками, добродушно указывали на их посредственность. Они были «наименьшими из всех апостолов»; они знали это и этим довольствовались.
\vs p139 9:6 \pc Иаков Алфеев особенно любил Иисуса за простоту Учителя. Эти близнецы не могли понять ум Иисуса, но они сознавали узы любви, связывающие их с сердцем их Учителя. Их умы не были умами высокого порядка; в каком\hyp{}то отношении их можно даже со всем уважением назвать глупыми, но в их духовных натурах они обрели подлинный опыт. Они верили в Иисуса; они были сыновьями Бога и соратниками в царстве.
\vs p139 9:7 Иуду Алфеева притягивала к Иисусу непритязательная скромность Учителя. Подобная скромность, сочетавшаяся с такими личными достоинствами, была для Иуды крайне привлекательной. То, что Иисус всегда приказывал молчать о своих необычных поступках, производило огромное впечатление на это простое дитя природы.
\vs p139 9:8 \pc Близнецы были добродушными и простодушными помощниками, и все их любили. Иисус принял этих двух молодых людей, обладавших одним\hyp{}единственным талантом, на почетные места в своем личном окружении в царстве, потому что в мирах пространства существуют бессчетные миллионы других таких же простых и порабощенных страхом душ, которые он также желает видеть среди своих активных и верующих причастников и причастников исходящего от него Духа Истины. Иисус не взирал на малость, а лишь на зло и на грех. Иаков и Иуда были \bibemph{малыми,} но они вместе с тем были \bibemph{верными.} Они были просты и невежественны, но вместе с тем были сердечны, добры и щедры.
\vs p139 9:9 И какой же благодарной гордостью исполнились эти скромные люди в день, когда Учитель отказался принять некого богатого человека как евангелиста, пока тот не продаст свое имение и не поможет бедным. Когда народ услышал об этом и увидел близнецов среди советников Иисуса, то убедился, что он не взирает на лица. Однако только божественно установленное --- царство небесное --- могло быть построено на таком посредственном человеческом фундаменте!
\vs p139 9:10 За все время своего общения с Иисусом близнецы всего один раз или дважды решились задать ему вопрос открыто. Однажды Иуда не выдержал и задал Иисусу вопрос, когда Учитель сказал об открытом явлении себя миру. Иуда был немного разочарован тем, что среди двенадцати апостолов больше не должно было быть секретов, и решился спросить: «Но, Учитель, являя так себя миру, как ты отметишь нас особыми проявлениями своей доброты?»
\vs p139 9:11 \pc Близнецы верно служили до самого конца, до темных дней суда, распятия и отчаяния. В своих сердцах они не утратили веру в Иисуса и (не считая Иоанна) были первыми, кто поверил в его воскресение. Но они не могли постичь установления царства. Вскоре после распятия своего Учителя они вернулись к своим семьям и сетям; их дело было сделано. Они не были способны дальше участвовать в более сложных сражениях за царство. Но они жили и умерли, сознавая, что были прославлены и благословлены четырьмя годами близкого и личного общения с Сыном Бога, полновластным создателем вселенной.
\usection{11. Симон Зилот}
\vs p139 11:1 Симон Зилот, одиннадцатый апостол, был избран Симоном Петром. Он был способным человеком, хорошего происхождения и жил со своей семьей в Капернауме. Когда он примкнул к апостолам, ему было двадцать восемь лет. Он был пламенным агитатором, но человеком, который много говорил, не подумав. До того, как обратил все свое внимание к патриотической организации зилотов, он был купцом в Капернауме.
\vs p139 11:2 \pc Симону Зилоту было поручено заботиться о развлечениях и отдыхе апостольской группы, и он был весьма умелым организатором досуга и веселья двенадцати.
\vs p139 11:3 Сила Симона была в его несомненной преданности. Когда апостолы встречали мужчину или женщину, которые пребывали в нерешительности относительно обращения к царству, они посылали за Симоном. Этому полному энтузиазма стороннику спасения через веру в Бога обычно требовалось всего минут пятнадцать, чтобы рассеять все сомнения и покончить со всей нерешительностью, увидеть, что еще одна душа перешла в «свободу веры и радость спасения».
\vs p139 11:4 Великая слабость Симона была в его склонности к материализму. Он не мог быстро измениться, став из еврейского националиста духовным интернационалистом. Четыре года были слишком коротким сроком, чтобы могло произойти подобное интеллектуальное и эмоциональное преобразование, но Иисус был всегда по отношению к нему терпелив.
\vs p139 11:5 \pc Той чертой Иисуса, которой Симон особенно восхищался, было спокойствие Учителя, его уверенность, уравновешенность и необъяснимое самообладание.
\vs p139 11:6 \pc Хотя Симон был неистовым революционером, бесстрашным зачинщиком агитации, он постепенно усмирял свой вспыльчивый нрав, пока не стал сильным и умелым проповедником «мира на земле и доброй воли среди людей». Симон был великим спорщиком; ему нравилось спорить. И когда приходилось иметь дело с законническими рассуждениями образованных евреев или интеллектуальными софизмами греков, такое задание всегда поручалось Симону.
\vs p139 11:7 По своей природе Симон был бунтарем, а по воспитанию --- человеком, борющимся с традиционными верованиями, но Иисус сумел обратить его к более высоким представлениям о царствие небесном. Симон всегда отождествлял себя с протестующей партией, но теперь он присоединился к партии прогресса, партии неограниченного и вечного совершенствования в духе и истине. Симон был человеком исключительно верным, человеком горячей личной преданности и очень любил Иисуса.
\vs p139 11:8 \pc Иисус не боялся связывать себя с коммерсантами, тружениками, пессимистами, философами, скептиками, мытарями, политиками и патриотами.
\vs p139 11:9 \pc У Учителя было много бесед с Симоном, но он так и не сумел полностью превратить в интернационалиста этого страстного еврейского националиста. Иисус часто говорил Симону, что хотеть улучшения социального, экономического и политического порядка совершенно правильно, но к этому всегда добавлял: «Это --- отнюдь не дело царства небесного. Мы должны посвятить себя исполнению воли Отца. Наше дело --- быть посланцами духовного руководства свыше, и мы не должны думать ни о чем ином, кроме выражения воли и природы божественного Отца, который властвует над всем и чьим доверием мы облечены». Все это Симону было трудно осознать, но постепенно он начал в какой\hyp{}то мере понимать смысл учения Учителя.
\vs p139 11:10 \pc После рассеяния, вызванного преследованиями в Иерусалиме, Симон временно отошел от дел. Он был буквально сломлен. Как патриот\hyp{}националист он отдал всего себя учению Иисуса; теперь же все было потеряно. Он был в отчаянии, но через несколько лет воспрянул духом и стал возвещать евангелие царства.
\vs p139 11:11 Он пошел в Александрию и, двигаясь к верховьям Нила, проник в самое сердце Африки, повсюду проповедуя евангелие и крестя верующих. И так он трудился до тех пор, пока не состарился и одряхлел. Он умер и был похоронен в сердце Африки.
\usection{12. Иуда Искариот}
\vs p139 12:1 Иуда Искариот, двенадцатый апостол, был избран Нафанаилом. Он родился в Кериоте, небольшом городке в южной Иудее. Когда Иуда был мальчиком, его родители переселились в Иерихон, где он жил и занимался разными делами своего отца, пока не заинтересовался проповедью и деяниями Иоанна Крестителя. Родители Иуды были саддукеями, и когда их сын примкнул к последователям Иоанна, отреклись от него.
\vs p139 12:2 \pc Когда Нафанаил встретил Иуду в Тарихее, тот пытался найти работу на предприятии, где вялили рыбу, в нижней части Галилейского моря. Когда Иуда присоединился к апостолам, ему было тридцать лет, и он был холост. Из двенадцати он был, возможно, самым образованным человеком и единственным иудеем в апостольской семье Учителя. У Иуды не было качеств, которые выделяли бы его как сильную личность, хотя у него и было много внешне заметных признаков культуры и привитых воспитанием привычек. Он был добропорядочным мыслителем, но не всегда мыслителем истинно \bibemph{честным.} Иуда по\hyp{}настоящему не понимал самого себя, и по отношению к самому себе не был действительно честным.
\vs p139 12:3 Андрей назначил Иуду казначеем двенадцати, на должность, которая ему в высшей степени подходила, и до времени предательства своего Учителя он исполнял свои обязанности честно, верно и максимально эффективно.
\vs p139 12:4 \pc У Иисуса не было такой черты, которой Иуда бы особенно восхищался в привлекательной и крайне обаятельной личности Учителя. Иуда так и не смог подняться над своими свойственными жителям Иудеи предубеждениями против своих товарищей\hyp{}галилеян; в уме он даже критиковал многое в Иисусе. Этот самодовольный иудей в душе решался осуждать того, кого одиннадцать из апостолов считали совершенным человеком, и «тем кто милее всех и главнее среди десятка тысяч». Он действительно придерживался точки зрения, согласно которой Иисус был нерешителен и отчасти боялся заявить о своей силе и власти.
\vs p139 12:5 \pc Иуда был хорошим коммерсантом. Чтобы управлять делами такого идеалиста, как Иисус, требовался такт, способности и терпение, а также исключительная преданность, не говоря уже о необходимости бороться с беспорядочными методами ведения дел некоторыми из его апостолов. Иуда, действительно, был замечательным распорядителем, дальновидным и способным финансистом. Он был сторонником упорядоченности. Никто из двенадцати никогда не критиковал Иуду. Насколько они видели, Иуда Искариот был несравненным казначеем, ученым человеком, верным (хотя иногда и критически настроенным) апостолом и во всех отношениях успешным человеком. Апостолы любили Иуду; он действительно был одним из них. Вероятно, он \bibemph{верил} в Иисуса, однако мы сомневаемся, действительно ли он \bibemph{любил} Учителя всем сердцем. Дело Иуды подтверждает истинность изречения: «Есть путь, который кажется человеку прямым, но конец --- смерть». Вполне возможно стать жертвой незаметного обольщения, которое заключается в приятном приспособлении к путям греха и смерти. Будьте уверены, что Иуда с финансовой точки зрения был всегда верен своему Учителю и своим братьям\hyp{}апостолам. Деньги никогда не могли стать причиной его предательства Учителя.
\vs p139 12:6 Иуда был единственным сыном неразумных родителей. В детстве его нежили и пестовали; он был избалованным ребенком. Он вырос, имея преувеличенные представления о собственной значимости. У него было извращенное представление о справедливости и склонность к ненависти и подозрительности. Он был мастером извращенного толкования слов и поступков своих друзей. Всю свою жизнь Иуда воспитывал в себе привычку мстить тем, кто, по его мнению, обращался с ним дурно. У него было извращенное представление о ценностях и преданности.
\vs p139 12:7 \pc Для Иисуса Иуда был испытанием в вере. С самого начала Учитель полностью понимал слабость этого апостола и хорошо знал об опасностях, связанных с принятием его в число апостолов. Однако такова природа Сынов Бога --- давать всем сотворенным существам равные возможности для спасения и вечной жизни. Иисус хотел, чтобы не только смертные этого мира, но и наблюдатели бесчисленных иных миров знали: если существуют сомнения относительно искренности и честности в преданности того или иного создания царству, то неизменная практика Судей людей заключается в том, чтобы все же принять вызывающего сомнения кандидата. Врата вечной жизни широко раскрыты для всех; «всякий желающий может войти»; и для входящего кроме \bibemph{веры} нет ни преград, ни условий.
\vs p139 12:8 Именно по этой причине Иисус и позволил Иуде идти до самого конца и всегда делал все возможное, чтобы изменить и спасти этого слабого и запутавшегося апостола. Однако, когда свет не принимают честно и не живут в согласии с ним, он в душе постепенно становится тьмой. Интеллектуально Иуда возрастал в понимании учений Иисуса о царстве, но в отличие от других апостолов не продвинулся в развитии у себя духовных качеств. В духовном опыте он не сумел добиться удовлетворительных личных результатов.
\vs p139 12:9 \pc Иуда все больше и больше погружался в личные разочарования и в конце концов стал жертвой обиды. Его чувства бывали часто уязвлены, и он сделался непомерно подозрительным в отношении своих лучших друзей и даже Учителя. Вскоре он стал одержим идеей расквитаться, пойти на все, чтобы отомстить за себя, да, даже предать своих товарищей и своего Учителя.
\vs p139 12:10 Однако эти коварные и опасные мысли не приобрели законченной формы до того дня, когда благодарная женщина открыла дорогой сосуд с благовонием у ног Иисуса. Иуде это показалось расточительным, и когда его открытый протест был столь решительно здесь же в присутствие всех отвергнут, Иуда не выдержал. Это событие пробудило скопившиеся ненависть, обиду, злобу, все предубеждение, всю ревность и всю жажду мести, и он решился расквитаться со всеми; однако все зло своего естества он направил на невинную личность во всей отвратительной драме его злополучной жизни только лишь потому, что Иисус оказался главным действующим лицом в эпизоде, который отметил переход Иуды из прогрессивного царства света в избранное им самим вместилище тьмы.
\vs p139 12:11 Учитель многократно и тайно, и открыто предупреждал Иуду о том, что он может совершить ошибку, однако божественные предостережения, как правило, бесполезны, когда речь идет о человеческой природе, исполненной злобы. Иисус сделал все возможное, сообразное с нравственной свободой человека, чтобы предотвратить выбор Иудой неправильного пути. Великое испытание, наконец, наступило. Сын негодования потерпел поражение; он поддался отвратительному и мерзкому диктату гордого и мстительного ума, полного самомнения, и стремительно погрузился в смятение, отчаяние и порочность.
\vs p139 12:12 Затем Иуда принял участие в низком и позорном заговоре, направленном на предательство своего Господа и Учителя, и быстро привел гнусный план в действие. Во время исполнения своих порожденных злобой замыслов предательской измены он испытывал минуты сожаления и стыда, и в эти светлые моменты, пытаясь внутренне оправдать себя, малодушно думал, что Иисус, возможно, проявит свою силу и в последний момент освободит себя.
\vs p139 12:13 Когда отвратительное и грешное дело было сделано, этот предатель\hyp{}смертный, который, недолго думая, продал своего друга за тридцать серебряников, чтобы удовлетворить свою вожделенную страсть к мщению, поспешно совершил финальный акт в драме бегства от реальностей смертного бытия --- самоубийство.
\vs p139 12:14 Одиннадцать апостолов были потрясены и ошеломлены. Иисус же смотрел на предателя только с жалостью. Миры не смогли простить Иуду, и его имени стали чуждаться во всей необъятной вселенной.
