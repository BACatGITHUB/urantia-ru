\upaper{160}{Родан Александрийский}
\author{Комиссия срединников}
\vs p160 0:1 Утром в воскресенье 18 сентября Андрей объявил, что на предстоящую неделю никакой работы не планируется. Все апостолы, кроме Нафанаила и Фомы, отправились домой навестить свои семьи или побыть с друзьями. В течение недели Иисус наслаждался практически полным покоем; Фома же и Нафанаил были заняты дискуссиями с неким греческим философом из Александрии по имени Родан. Этот грек недавно стал учеником Иисуса благодаря наставлениям одного из сподвижников Авенира, который проповедовал в Александрии. На тот момент Родан пытливо искал соответствия между своей философией жизни и новым религиозным учением Иисуса и пришел в Магадан, надеясь, что Учитель обсудит с ним эти проблемы. Он также желал получить первичную достоверную версию евангелия либо от Иисуса, либо от кого\hyp{}нибудь из его учеников. Хотя Учитель отказался принять участие в такой беседе с Роданом, он его любезно принял и сразу же распорядился, чтобы Нафанаил и Фома, выслушав все, что тот хотел сказать, в ответ рассказали ему об евангелии.
\usection{1. Греческая философия Родана}
\vs p160 1:1 В понедельник рано утром Родан начал цикл из десяти выступлений перед Нафанаилом, Фомой и двумя дюжинами верующих, случайно оказавшихся в Магадане. Эти беседы в обобщенном виде и изложенные на современном языке, выдвигают на обсуждение следующие идеи.
\vs p160 1:2 \P\ В жизни человека присутствуют три великие движущие силы --- побуждения, желания и соблазны. Сильным характером, могучей личностью можно стать, лишь направив естественные жизненные побуждения в русло общественного искусства жизни, трансформировав сиюминутные желания в те высшие стремления, что способны обеспечить вечные свершения; банальный же соблазн бытия следует перенести из области заурядных, устоявшихся идей в высшие сферы неизведанных идей и неоткрытых идеалов.
\vs p160 1:3 Чем больше усложняется цивилизация, тем труднее становится искусство жизни. Чем стремительнее преобразования в общественном бытие, тем сложнее становится задача становления характера. Чтобы прогресс продолжался, человечество через каждые десять поколений должно заново учиться искусству жизни. Если же человек станет настолько изобретательным, что процесс усложнения организации общества пойдет еще быстрее, то искусством жизни придется овладевать заново через меньшие промежутки времени, возможно, каждому новому поколению. Если эволюция искусства жизни начнет отставать от технической эволюции бытия, человечество быстро вернется к простому жизненному побуждению --- устремится к удовлетворению сиюминутных желаний. Человечество, таким образом, останется неразвитым, а общество не сумеет раскрыть все свои возможности.
\vs p160 1:4 Социальная зрелость --- это такой уровень развития, на котором человек готов отказаться от удовлетворения простых преходящих и сиюминутных желаний ради тех высших устремлений, достижение которых приносит еще более полное удовлетворение потому, что постепенно приближает к непреходящим целям. Однако истинным признаком социальной зрелости является готовность людей отказаться от права жить мирной и сытой жизнью в соответствии с необременительными нормами, характерными для установившихся верований и традиционных идей, ради тревожного и требующего усилий устремления к неизведанным возможностям достижения высших целей и возвышенных духовных реалий.
\vs p160 1:5 Животные прекрасно реагируют на жизненные побуждения, но лишь человек может овладеть искусством жизни, хотя большинству людей присуще только животное стремление жить. Животным свойственно лишь это слепое и инстинктивное стремление; человек же способен возвыситься над подобными стремлениями к естественному. Человек может выбрать жить на высоком уровне интеллектуального искусства или даже на уровне небесной радости и духовного экстаза. Животные не думают о смысле жизни, а потому никогда не испытывают нравственных страданий и не совершают самоубийств. Самоубийства среди людей доказывают, что эти существа поднялись над чисто животным уровнем существования и более того, свидетельствуют о том, что исследовательские усилия таких человеческих существ не привели к высокому уровню жизненного опыта. Животные ничего не знают о смысле жизни; человек же не только обладает способностью распознавать ценности и постигать смысл, но и осознает смыслы смыслов --- осознает своим разумом свое собственное понимание.
\vs p160 1:6 Когда люди решаются отказаться от жизни в которой они руководствуются природными желаниями ради жизни, полной смелого творчества и неопределенной логики, им следует быть готовыми переносить проистекающие из такой жизни эмоциональные травмы --- конфликты, несчастья и сомнения --- по крайней мере до достижения ими определенного уровня интеллектуальной и эмоциональной зрелости. Упадок духа, беспокойство и праздность однозначно свидетельствуют о нравственной незрелости. Человеческое общество решает две проблемы: достижение зрелости индивидуумом и достижение зрелости расой. Зрелый человек быстро начинает смотреть на всех остальных смертных с чувствами симпатии и терпимости. Зрелый человек смотрит на незрелых людей с любовью и, помня о том, что родители относятся к своим детям терпимо.
\vs p160 1:7 Успешная жизнь есть не что иное, как искусство владеть надежными методами решения обычных проблем. В качестве первого шага в решении любого вопроса следует определить суть затруднения, выделить проблему и беспристрастно изучить ее природу и степень тяжести. Мы совершаем большую ошибку, когда отказываемся разобраться в жизненных проблемах если они вызывают в нас глубокий страх. Более того, когда осознание нами наших трудностей вынуждает нас поубавить свое долго лелеямое самомнение, признаться в зависти или же расстаться с глубоко укоренившимися предубеждениями, средний человек предпочитает цепляться за старые иллюзии безопасности и ставшее привычным ложное чувство уверенности. Только смелая личность готова честно признать и бесстрашно смотреть на то, что обнаруживает беспристрастный и логический ум.
\vs p160 1:8 Мудрое и эффективное решение любой проблемы требует, чтобы ум освободился от предвзятости, страстей и остальных исключительно личных предрассудков, которые могут влиять на беспристрастное исследование действительных факторов, характеризующих проблему, которую необходимо решать. Решение жизненных проблем требует смелости и искренности. Только честные и смелые люди способны по приводящему в недоумение и сбивающему с толку лабиринту жизни идти туда, куда ведет логика бесстрашного ума. И это раскрепощение ума и души невозможно без движущей силы разумного энтузиазма, граничащего с религиозным рвением. Ведь чтобы заставить человека идти к цели, путь к которой преграждают сложные материальные проблемы и многочисленные интеллектуальные опасности, требуется соблазн великого идеала.
\vs p160 1:9 Даже если вы во всеоружии и готовы столкнуться лицом к лицу со сложными жизненными ситуациями, вы едва ли можете рассчитывать на успех, пока не обретете ту мудрость ума и то личное обаяние, которые обеспечат вам сердечную поддержку ваших собратьев и их сотрудничество с вами. Ни в мирской, ни в религиозной работе не надейтесь на большой успех, пока не постигнете, как убеждать ваших собратьев, как стать лидером среди людей. Такт и терпение вам просто необходимы.
\vs p160 1:10 \P\ Однако величайшему из всех методов решения проблем я научился у Иисуса, вашего Учителя. Я говорю о том, что он столь последовательно совершает и чему столь верно учил вас, --- об уединенном, полном почитания размышлении. В этом обыкновении Иисуса столь часто уединяться для общения с Отцом Небесным следует видеть способ не только обретения силы и мудрости для разрешения заурядных жизненных ситуаций, но и получения энергии для решения высших проблем морального и духовного свойства. Однако даже правильные пути решения проблем не могут компенсировать недостатков, присущих личности, или возместить отсутствие стремления к истинной праведности.
\vs p160 1:11 На меня производит неизгладимое впечатление обыкновение Иисуса уединяться и размышлять в это время над жизненными проблемами; искать новые источники мудрости и энергии чтобы иметь силы для служения обществу; стимулировать и усугубить устремление к верховной цели жизни, действительно всецело подчиняя свою личность осознанию соприкосновения с божественностью; стремиться по\hyp{}новому и лучшим образом приспосабливаться к постоянно изменяющимся ситуациям жизненного опыта; осуществлять те жизненно важные преобразования и исправления своих личных позиций, которые существенны для углубленного понимания всего достойного и реального; и делать все это только во славу Бога --- преисполняясь искренностью любимой молитвы вашего Учителя: «Не моя воля, но твоя да будет».
\vs p160 1:12 Такой способ богопочитания, принятый у вашего Учителя дает отдых, обновляющий ум; озарение, вдохновляющее душу; отвагу, позволяющую смело смотреть проблемам в лицо; самопонимание, уничтожающее расслабляющий страх; и сознание союза с божественностью, вооружающее человека уверенностью, которая позволяет ему решаться быть Богоподобным. Отдых почитания, или духовное общение, практикуемое Учителем, ослабляет напряжение, устраняет конфликты и значительно повышает все возможности личности. Вся эта философия плюс евангелие царства и есть новая религия, как я ее понимаю.
\vs p160 1:13 \P\ Предубеждение ослепляет душу и не дает ей распознать истину; избавиться от предубеждения можно лишь благодаря искренней преданности души поклонению причине, охватывающей и включающей всех его собратьев. Предубеждение неразрывно связано с эгоизмом. Предубеждение можно устранить, лишь отказавшись от своекорыстия и заменив его поиском удовлетворения, какое приносит служение делу, которое не только больше собственного «я», но и больше всего человечества --- исканием Бога, познанием божественности. Зрелости личности человек достигает тогда, когда его устремления неустанно направлены на воплощение высочайших и истинно божественных ценностей.
\vs p160 1:14 В непрерывно изменяющемся мире, в условиях развивающегося общественного устройства невозможно сохранять установившиеся и утвердившиеся цели предназначения. Состояние стабильности личности доступно лишь тем, кто открыл для себя живого Бога и избрал его вечной целью бесконечного познания. Таким образом, чтобы перенести свою цель из времени в вечность, с земли в Рай, из человеческого в божественное, человеку необходимо возродиться, обратиться и родиться вновь; стать заново сотворенным чадом божественного духа; обрести вход в братство царства небесного. Все философии и религии, не достигшие этих идеалов, незрелы. Философия же, которой учу я, совместно с евангелием, которое проповедуете вы, являет собой новую религию зрелости, идеал всех будущих поколений. И сие истинно, поскольку наш идеал окончателен, непогрешим, вечен, универсален, абсолютен и бесконечен.
\vs p160 1:15 Моя философия побудила меня искать реалии истинного достижения, цель зрелости. Однако мое побуждение было лишено силы; моим поискам недоставало движущей энергии; моему устремлению не хватало определенности в выборе направления. Но эти недостатки с избытком возместило новое евангелие Иисуса с его более глубоким пониманием, возвышенными идеалами и устойчивостью целей. Теперь без сомнений и опасений я могу искренне вступить на вечный путь.
\usection{2. Искусство жизни}
\vs p160 2:1 Существуют лишь два способа, посредством которых смертные могут сосуществовать, --- это материальный, или животный, способ и способ духовный, или человеческий. С помощью сигналов и звуков животные способны в какой\hyp{}то степени общаться друг с другом. Однако подобными формами общения не передать значения, ценности или идеи. Одно из отличий человека от животных заключается в том, что человек может общаться с себе подобными посредством \bibemph{символов,} которые наиболее точно обозначают и идентифицируют значения, ценности, идеи и даже идеалы.
\vs p160 2:2 Поскольку животные не могут обмениваться идеями, личность у них развиться не может. У человека же личность развивается, поскольку он способен общаться со своими собратьями, обмениваясь идеями и идеалами.
\vs p160 2:3 Именно эта способность общаться и передавать значения и составляет человеческую культуру и позволяет человеку через систему общественных связей строить цивилизации. Знание и мудрость накапливаются благодаря способности человека передавать их последующим поколениям. Вследствие этого образуется культурная деятельность расы: искусство, наука, религия и философия.
\vs p160 2:4 Общение между человеческими существами с помощью символов предопределяет возникновение общественных союзов. Из всех общественных союзов наиболее эффективным является семья, и в особенности \bibemph{двое родителей.} Личная привязанность --- вот духовные узы, которые удерживают вместе эти материальные союзы. Подобные прочные отношения возможны также между двумя лицами одного пола, что прекрасно подтверждается примерами преданности, характерными для настоящей дружбы.
\vs p160 2:5 Эти основанные на дружбе и взаимной любви связи облагораживают и приучают человека жить в обществе, ибо они способствуют и содействуют следующим существенным факторам высших уровней искусства жизни:
\vs p160 2:6 \ublistelem{1.}\bibnobreakspace \bibemph{Взаимное самовыражение и самопонимание.} Многие благородные человеческие порывы гибнут, потому что никто о них ничего не знал. Истинно, нехорошо человеку быть одному. Для развития человеческого характера необходима известная степень одобрения и определенная мера признания. Без настоящей любви домашних характер ни одного ребенка не может по\hyp{}настоящему нормально развиться. Характер есть нечто большее, нежели ум и мораль. Из всех общественных связей, определяющих развитие характера, наиболее эффективной и идеальной является полная любви и взаимопонимания дружба между мужчиной и женщиной, скрепленная узами осознанного разумного брака. Супружество с его разнообразными обязанностями лучше всего способствует проявлению тех драгоценных порывов и высших побуждений, которые необходимы для становления сильного характера. Утверждая это, я не колеблясь восхваляю семейную жизнь, ибо Учитель ваш в качестве краеугольного камня сего нового евангелия царства мудро избрал отношения между отцом и чадом. И подобная ни с чем не сравнимая община родственников, союз мужчины и женщины, скрепленные высочайшими идеалами времени, являет собой столь ценный и приносящий удовлетворение опыт, что ради обладания ими можно заплатить любую цену и пойти на любые жертвы.
\vs p160 2:7 \P\ \ublistelem{2.}\bibnobreakspace \bibemph{Единение душ --- выражение, мудрости.} У любого человеческого существа рано или поздно формируются определенные понятия об этом мире и определенное видение мира иного. Теперь же через общение личностей возможно объединение этих точек зрения на временное и вечное бытие. Благодаря этому ум одного, заимствуя многое из понимания другого, увеличивает свои духовные ценности. Таким способом, объединяя свой духовный потенциал, люди обогащают душу. Более того, таким же образом человек приобретает возможность избавиться от постоянной угрозы стать жертвой искаженного представления, предвзятости точки зрения и узости суждения. Страха, зависти и тщеславия можно избежать лишь при тесном общении с другими умами. Обращаю ваше внимание на то, что Учитель никогда не посылает вас в одиночку трудиться ради расширения царства, а всегда вдвоем. А поскольку мудрость --- это сверхзнание, значит, построенная на союзе мудрости общественная группа, большая или малая, совместно пользуется знанием каждого.
\vs p160 2:8 \P\ \ublistelem{3.}\bibnobreakspace \bibemph{Страсть к жизни.} Изоляция ведет к истощению заряда энергии души. Общение с собратьями необходимо для возобновления интереса к жизни и обязательно для поддержания мужества сражаться в битвах являющихся следствием восхождения на более высокие уровни человеческой жизни. Дружба усиливает радость и прославляет достижения нашей жизни. Любовь и близость между людьми лишает страдание печали, а нужду --- большей части присущей ей горечи. Присутствие друга делает все более привлекательным и усугубляет всякую добродетель. С помощью разумных символов человек может стимулировать и повышать способность своих друзей к пониманию. Именно эта сила и возможность взаимного развития воображения является одной из самых славных черт человеческой дружбы. Сознанию искренней приверженности общему делу, взаимной верности космическому Божеству присуща великая духовная сила.
\vs p160 2:9 \P\ \ublistelem{4.}\bibnobreakspace \bibemph{Усиленная защита против всякого зла.} Общение личностей и взаимная любовь --- вот надежная защита от зла. Трудности, печаль, разочарование и поражение в одиночку переносятся гораздо больнее и еще больше лишают мужества. Общение не превращает зло в добро, но помогает, сильно уменьшая боль. Ваш Учитель сказал: «Блаженны скорбящие» --- да, если рядом друг, который может утешить. В сознании того, что ты живешь ради благополучия других и что эти другие точно так же живут ради твоего благополучия и успеха, скрыта великая сила. В одиночестве человек увядает. Видя лишь преходящие сиюминутные дела, люди неизбежно падают духом. Настоящее вне связи с прошлым и будущим становится раздражающе тривиальным. Лишь взгляд на цикл вечности может вдохновить человека сделать все от него зависящее и заставить лучшее в нем максимально проявить себя. Когда же человек делает все, на что способен, он максимально бескорыстен и живет на благо других, своих собратьев, существующих во времени и в вечности.
\vs p160 2:10 \P\ Повторяю, подобное вдохновляющее и облагораживающее общение идеально проявляется в супружеских отношениях людей. Верно, многого можно достигнуть и вне брака и многие, очень многие браки ни в коей мере не смогли принести эти нравственные и духовные плоды. Слишком часто в брак вступают те, кто ищет иные ценности, которые уступают этим высшим достижениям человеческой зрелости. Идеальный брак должен основываться на чем\hyp{}то более стабильном, нежели переменчивость чувств и непостоянство полового влечения; он должен основываться на подлинной и взаимной личной преданности. Таким образом, если вы сможете создать подобные надежные и эффективные небольшие единицы человеческого содружества, то тогда, когда они предстанут в совокупности, мир увидит великую и славную общественную структуру, зрелую цивилизацию смертных. Такой расе, возможно, и удастся начать реализовывать идеал вашего Учителя: «Мир на земле и добрая воля среди людей». Хотя подобное общество не будет совершенно или полностью свободно от зла, оно, по крайней мере, приблизится к стабильности, которая присуща зрелости.
\usection{3. Соблазны зрелости}
\vs p160 3:1 Стремление к зрелости вынуждает работать, а работа требует энергии. Где же взять силу, чтобы исполнить все это? Все физическое можно считать само собой разумеющимся, однако Учитель хорошо сказал: «Не хлебом единым жив человек». Получив в дар нормальное тело и относительно хорошее здоровье, мы должны дальше искать именно те соблазны, которые будут действовать как стимул для пробуждения дремлющих в человеке духовных сил. Иисус учил нас: Бог живет в человеке; как же тогда убедить человека проявить эти сокрытые в душе силы божественности и бесконечности? Как убедить людей дать Богу свободу, чтобы он обновил наши собственные души, и в то же время просветил, возвысил и благословил другие бесчисленные души? Как мне лучше всего разбудить эти скрытые и устремленные к добру силы, спящие в ваших душах? В одном я уверен: эмоциональное возбуждение вовсе не идеальный духовный стимул. Возбуждение не увеличивает запасы энергии, а истощает силы и ума и тела. Откуда же берется энергия, необходимая для того, чтобы вершить сии великие дела? Посмотрите на вашего Учителя. Вот и сейчас он в горах обретает силу, тогда как мы здесь нашу энергию расходуем. Полная разгадка этой проблемы кроется в духовном общении, в почитании. С человеческой точки зрения, это --- вопрос сочетания размышлений и расслабления. Размышления обеспечивают соприкосновение ума с духом; расслабление же определяет способность к духовной восприимчивости. Именно эта замена слабости на силу, страха на смелость, собственного ума на волю Бога и составляет суть почитания. По крайней мере, так это видит философ.
\vs p160 3:2 При частом повторении этих переживаний они превращаются в привычки --- дающие силу и полные почитания привычки, --- и такие привычки в конечном итоге формируют духовный характер, а подобный характер в конце концов воспринимается собратьями человека как \bibemph{зрелая личность.} Действия эти трудны и требуют много времени, однако, став привычными, сразу же приносят покой и экономят время. Чем сложнее общество и чем сильнее соблазны цивилизации, тем насущнее становится для знающих Бога людей необходимость формировать подобные защитные повседневные привычки, предназначенные сохранить и приумножить их духовные силы.
\vs p160 3:3 Еще одно условие, обязательное для достижения зрелости, --- сотрудничество общественных групп, стремление приспособиться к постоянно изменяющемуся окружению. Незрелый индивидуум вызывает враждебное к себе отношение своих собратьев; зрелый же человек добивается от своих товарищей сердечного сотрудничества и тем самым во много раз приумножает плоды своих жизненных усилий.
\vs p160 3:4 Моя философия говорит мне: бывают времена, когда я, если понадобится, должен сражаться, защищая свое представление о праведности, но я не сомневаюсь: Учитель с его более зрелым типом личности легко и изящно одержит равную победу с помощью своего более совершенного и привлекательного метода --- тактом и терпением. Когда мы сражаемся за правду, то слишком часто оказывается, что поражение терпят и победитель и побежденный. Только вчера я слышал, как Учитель сказал: «Мудрый человек, пытаясь войти через закрытую дверь, не станет дверь ломать, но будет искать ключ, который ее откроет». Слишком часто мы ввязываемся в драку только затем, чтобы убедить самих себя, что нам не страшно.
\vs p160 3:5 Сие новое евангелие царства оказывает великую службу искусству жизни, ибо оно дает человеку новое и более ценное побуждение к высшей жизни. Оно выдвигает новую и более высокую цель предназначения, верховную жизненную миссию. И эти новые понятия о вечной и божественной цели бытия сами по себе являются превосходными стимулами, порождающими то лучшее, что есть в высшей природе человека. На каждой вершине интеллектуального озарения следует искать отдых для ума, силу для души и общение с Богом --- для духа. С таких высот жизни человек способен возвыситься над материальными раздражителями низших уровней мышления, а именно: тревогой, ревностью, завистью, местью и гордыней незрелой личности. Эти высоко поднявшиеся души избавляют себя от множества конфликтов мелочной жизни и, таким образом, обретают свободу для достижения осознания высших потоков духовной мысли и небесного общения. Однако цель жизни следует ревностно оберегать от искушения стремиться к легким и преходящим результатам; более того, ее следует вынашивать так, чтобы она стала неуязвимой для угрозы разрушающего фанатизма.
\usection{4. Уравновешенность зрелости}
\vs p160 4:1 Всецело устремив себя к достижению вечных реалий, необходимо также заботиться о потребностях временной жизни. Хотя цель наша --- дух, плоть --- это наша действительность. Порой, по воле случая, необходимое для жизни само попадает нам в руки, однако, как правило, мы за него должны разумно трудиться. В жизни есть две главные задачи: добывание средств к временному существованию и обретение жизни вечной. Причем даже для идеального решения проблемы добывания средств существования требуется религия. Обе эти проблемы сугубо личные. Ведь на самом деле истинная религия в отрыве от человека не существует.
\vs p160 4:2 \P\ Потребности временной жизни, с моей точки зрения, таковы:
\vs p160 4:3 \ublistelem{1.}\bibnobreakspace Хорошее физические здоровье.
\vs p160 4:4 \ublistelem{2.}\bibnobreakspace Четкое и чистое мышление.
\vs p160 4:5 \ublistelem{3.}\bibnobreakspace Способности и умение.
\vs p160 4:6 \ublistelem{4.}\bibnobreakspace Богатство --- материальные блага жизни.
\vs p160 4:7 \ublistelem{5.}\bibnobreakspace Способность стойко переносить поражения.
\vs p160 4:8 \ublistelem{6.}\bibnobreakspace Культура --- образование и мудрость.
\vs p160 4:9 \P\ Даже физические проблемы здоровья тела и работоспособности лучше всего решаются, когда на них смотрят с религиозной точки зрения учения нашего Учителя, согласно которому тело и разум человека --- это место пребывания дара Богов, духа Бога, который становится духом человека. Разум человека, таким образом, делается посредником между материальными вещами и духовными реалиями.
\vs p160 4:10 \P\ Чтобы получить свою долю желаемого в жизни, необходимы умственные способности. Предположение, будто неуклонное исполнение повседневной работы принесет в награду изобилие, --- полностью ошибочно. За исключением редких и непредвиденных случаев приобретения богатства, материальные вознаграждения временной жизни текут по определенным строго организованным каналам, и лишь те, кто имеет доступ к этим каналам, могут рассчитывать на хорошее вознаграждение за свои временные усилия. Бедность --- вот вечный удел всех, кто ищет изобилия в изолированных и индивидуальных каналах. Мудрое планирование, таким образом, становится неотъемлемой частью мирского процветания. Успех требует не только прилежания в работе, но и того, чтобы человек действовал как часть какого\hyp{}нибудь одного из каналов материального изобилия. Если вы неблагоразумны, то можете посвятить всю жизнь своему поколению, но материальной награды не получить; если же вы случайно получили поток изобилия, то можете купаться в роскоши, хотя и не сделали ничего достойного для ваших собратьев.
\vs p160 4:11 Способность --- это то, что вы наследуете, а умение --- то, что приобретаете. Для того, кто не умеет что\hyp{}нибудь делать хорошо, профессионально, жизнь нереальна. Умение --- это один из реальных источников удовлетворения в жизни. Способность же предполагает дар предвидения, предусмотрительности. Не обольщайтесь искушающими наградами, если они получены нечестным путем; будьте готовы трудиться ради будущей прибыли в награду за честное старание. Мудрый человек способен отличать цели от средств; иначе скрупулезное планирование будущего может привести к разрушению высокой цели, ради которой оно предпринималось. Как искатели удовольствия вы должны стремиться всегда быть как потребителями, так и производителями.
\vs p160 4:12 Тренируйте вашу память, чтобы в ней как священный вклад сохранялись придающие силу и достойные эпизоды жизни, которые при желании сможете вспоминать к своему удовольствию или в назидание самим себе. Именно так, для себя и в себе, стройте и сохраняйте заповедные галереи красоты, доброты и художественного великолепия. Однако самым прекрасным из всех воспоминаний является драгоценная память о великих моментах благородной дружбы. И все эти сокровища памяти при освобождающем прикосновении духовного почитания оказывают драгоценное и возвышающее воздействие.
\vs p160 4:13 Однако если вы не научитесь благородно проигрывать, жизнь для вас превратится в тяжкое бремя существования. В поражении есть свое искусство, которым возвышенные души овладевают всегда; переживая поражение, вы должны оставатся бодрым и неустрашимым перед лицом разочарований. Никогда не бойтесь признать поражение. Не пытайтесь его прятать за обманчивыми улыбками и сияющим оптимизмом. Всегда трубить о победе --- звучит хорошо; однако конечные результаты при этом плачевны. Такой метод ведет прямо к созданию воображаемого мира и неизбежному краху и полному разочарованию.
\vs p160 4:14 Успех может порождать смелость и способствовать уверенности, однако мудрость приходит лишь с опытом приспособления к последствиям поражения. Человек, предпочитающий реальности радужные иллюзии, мудрым не станет никогда. Мудрости достигают лишь те, кто смотрит действительности в лицо и приспосабливает ее к идеалам. Мудрость объемлет и идеал и действительность и тем самым спасает своих приверженцев от двух бесплодных крайностей философии --- не дает стать человеком, чей идеализм исключает действительность, и материалистом, лишенным духовного кругозора. Те робкие души, что могут продолжать жизненную борьбу, лишь непрерывно погружаясь в ложные иллюзии успеха, обречены испытать неудачу и потерпеть поражение, когда полностью очнутся от мира грез их собственного воображения.
\vs p160 4:15 В этом деле столкновения с неудачей и адаптации к поражению верховное влияние оказывает дальновидность религии. Неудача --- это просто элемент обучения --- культурный эксперимент в овладении мудростью --- в опыте ищущего Бога человека, который вступил на вечную стезю исследования вселенной. Для таких людей поражение --- всего лишь новое орудие для достижения более высоких уровней вселенской реальности.
\vs p160 4:16 С точки зрения вечности, жизненный путь человека, ищущего Бога, может оказаться великим успехом, даже если дело всей его временной жизни может казаться сокрушающей неудачей --- при условии, что каждая жизненная неудача приносит урожай культуры, основанной на мудрости и духовном знании. Не делайте ошибку, путая знание, культуру и мудрость. В жизни они между собой связаны, но выражают совершенно разные духовные ценности; мудрость всегда выше знания и постоянно возвеличивает культуру.
\usection{5. Религия идеала}
\vs p160 5:1 Вы сказали мне, что ваш Учитель смотрит на истинную человеческую религию как на индивидуальный опыт с духовными реалиями. Я же рассматривал религию как опыт реакции человека на нечто такое, что он считает достойным уважения и приверженности всего человечества. В этом смысле религия символизирует нашу высшую приверженность тому, что олицетворяет собой высокие представления об идеалах реальности и самые лучшие устремления наших умов к вечным возможностям духовного достижения.
\vs p160 5:2 Если люди воспринимают религию в родовом, национальном или расовом смысле, то это бывает потому, что они не склонны считать тех, кто находится вне их группы, полноценными людьми. Объект же нашего религиозного поклонения мы всегда рассматриваем как нечто достойное почитания всеми людьми. Религия никогда не бывает вопросом простой интеллектуальной веры или же философских рассуждений; религия --- это везде и вечно способ отношения к жизненным ситуациям; это тип поведения. Религия объемлет мышление, чувства и благоговейные действия в отношении некой реальности, которую мы считаем достойной всеобщего поклонения.
\vs p160 5:3 Если что\hyp{}то в вашем жизненном опыте превратилось в религию, то, очевидно, вы уже стали активным евангелистом этой религии, поскольку считаете верховную концепцию вашей религии достойной почитания всего человечества, всех разумных существ вселенной. Если же вы не активный и не проповедующий евангелист вашей религии, значит, вы самообольщаетесь, ибо то, что вы в этом случае называете религией, --- всего лишь или традиционная вера, или же просто интелектуальная философская теория. Если ваша религия --- это духовный опыт, то объектом вашего почитания должны быть всемирная духовная реальность и идеал ваших одухотворенных представлений. Все религии, основанные на страхе, эмоциях, традиции и философии, я называю интеллектуальными религиями; религии же, основанные на подлинном духовном опыте, называю истинными. Объект религиозного поклонения может быть материальным или духовным, истинным или ложным, реальным или нереальным, человеческим или божественным. Следовательно, и религии могут быть как благими, так и порочными.
\vs p160 5:4 Мораль и религия --- не обязательно одно и то же. Система морали, включившая в себя объект почитания, может стать религией. Религия же, утратившая всеобъемлющую привлекательность как предмет веры и верховной приверженности, может переродиться в философскую систему или же моральный кодекс. Вещь, существо, состояние или порядок бытия либо возможность достижения того, что составляет верховный идеал религиозной преданности и является объектом религиозного поклонения приверженцев, есть Бог. Независимо от имени, присвоенного такому идеалу духовной реальности, это --- Бог.
\vs p160 5:5 Социальные особенности истинной религии заключаются в том, что она неизменно стремится обратить человека и преобразовать мир. Религия подразумевает существование неосознанных идеалов, намного превосходящих принятые нормы этики и морали, содержащиеся даже в лучших общественных системах наиболее зрелых институтов цивилизации. Религия стремится к неоткрытым идеалам, неведомым реалиям, надчеловеческим ценностям, божественной мудрости и истинно духовным достижениям. Только истинная религия может все это; все же остальные верования недостойны этого названия. Без верховного и возвышенного идеала вечного Бога нет подлинно духовной религии. Религия без такого Бога --- это человеческое изобретение, человеческий институт безжизненных интеллектуальных верований и бессмысленных эмоциональных обрядов. Религия в качестве своего объекта приверженности может выдвигать некий великий идеал. Однако подобные идеалы нереальности недостижимы; такая концепция иллюзорна. Единственные идеалы, которых человек может достигнуть --- это божественные реалии бесконечных ценностей, присущих духовному факту вечного Бога.
\vs p160 5:6 Само слово «Бог», сама идея Бога в отличие от идеала Бога может стать частью любой религии, какой бы пустой или ложной она ни была. И эта идея Бога может стать чем угодно; все зависит от того, какой смысл вкладывают в нее те, кто ее исповедует. Низшие религии формируют свои идеи о Боге согласно естественному состоянию человеческого сердца; высшие же требуют, чтобы человеческое сердце изменялось согласно требованиям идеалов истинной религии.
\vs p160 5:7 \P\ Религия Иисуса превосходит все прежние концепции идеи почитания, ибо он не только изображает своего Отца как идеал бесконечной реальности, но и однозначно утверждает, что этот божественный источник ценностей и вечный центр вселенной истинно и лично достижим для всякого смертного творения, решившего войти в царство небесное на земле и тем самым признающего принятие сыновства по отношению к Богу и братство людей. Я утверждаю, что это --- высшая концепция религии, которую мир когда\hyp{}либо знал, и объявляю, что выше нее не будет ничего и никогда, ибо сие евангелие охватывает собой бесконечность реалий, божественность ценностей и вечность всемирных достижений. Подобное представление являет собой достижение опыта идеализма на верховном и предельном уровнях.
\vs p160 5:8 Совершенные идеалы религии вашего Учителя не только заинтересовали меня, они взволновали меня так глубоко, что я открыто признаюсь в своей вере в его утверждение, что эти идеалы духовных реалий достижимы; что мы с вами можем стать на долгий и вечный путь, помня о его заверении в том, что в конечном итоге мы обязательно прибудем к вратам Рая. Братья мои, я --- верующий; я вступил на сей путь и вместе с вами пойду по этой вечной дороге. Учитель говорит, что он пришел от Отца и что он покажет нам путь. Я полностью убежден: он говорит правду. Я совершенно уверен: вне Отца Всего Сущего нет достижимых идеалов реальности или совершенных ценностей.
\vs p160 5:9 Поэтому я пришел почитать не просто Бога бытия, но Бога возможности всякого бытия в будущем. Следовательно, и ваша преданность верховному идеалу, если идеал этот реален, должна быть преданностью сему Богу прошлых, настоящих и будущих вселенных вещей и существ. И иного Бога нет, ибо никакого другого Бога быть не может. Все остальные боги есть плод воображения, иллюзии смертного разума, издержки ложной логики и обольщающие идеалы тех, кто их сотворил. Да, у вас может быть религия без этого Бога, но это еще ничего не значит. И если вы словом «Бог» пытаетесь подменить реальность сего идеала Бога живого, то лишь вводите себя в заблуждение, ставя идею на место идеала, на место божественной реальности. Такие верования --- не более чем религии богатого воображения.
\vs p160 5:10 В учениях Иисуса я вижу религию в ее лучшем проявлении. Сие евангелие позволяет нам искать истинного Бога и его находить. Однако готовы ли мы заплатить цену сего вхождения в царство небесное? Готовы ли родиться заново? Быть заново сотворенными? Готовы ли подвергнуться сему ужасному и суровому процессу саморазрушения и возрождения души? Разве не сказал Учитель: «Кто хочет жизнь свою сберечь, тот ее потеряет. Не думайте, что я пришел принести мир; не мир пришел я принести, но душевную борьбу». Верно, заплатив за посвящение воле Отца, мы испытаем великий мир при условии, что продолжим идти этими духовными путями посвященной жизни.
\vs p160 5:11 Ныне мы истинно оставляем соблазны известного порядка бытия и всецело посвящаем наши поиски соблазнам неизвестного и неизведанного порядка бытия в будущей жизни странствий в духовных мирах где господствует высший идеализм божественной реальности. И ищем те символы смысла, посредством которых сможем передать нашим собратьям сии понятия реальности идеализма религии Иисуса и будем беспрестанно молиться о том дне, когда все человечество возрадуется общему для всех видению этой верховной истины. Итак, если обобщить наши представления об Отце, каковы они в наших сердцах, то можно сказать так: Бог есть дух; своим же собратьям мы передаем его как: Бог есть любовь.
\vs p160 5:12 Религия Иисуса требует живого и духовного опыта. Другие религии могут заключаться в традиционных верованиях, эмоциональных чувствах, философском сознании или во всем этом вместе; учение же Учителя требует достижения подлинных уровней реального духовного совершенствования.
\vs p160 5:13 Сознание, присущее порыву уподобиться Богу, не есть истинная религия. Стремление почитать Бога не есть истинная религия.Осознанное убеждение отречься от себя и служить Богу не есть истинная религия. Мудрость рассуждения о том, что эта религия --- лучшая из всех, не есть религия как личный и духовный опыт. Истинная религия говорит о предназначении и реальности обретения, так же, как и о реальности и идеализме того, что искренне принимается верой. И все это, благодаря откровению Духа Истины, должно стать для нас личным.
\vs p160 5:14 \P\ И так закончились рассуждения греческого философа, одного из величайших философов своего народа, философа, который поверил в евангелие Иисуса.
