\upaper{172}{Вход в Иерусалим}
\author{Комиссия срединников}
\vs p172 0:1 1 марта 30 года н.э. в пятницу днем в начале пятого Иисус и апостолы пришли в Вифанию. Их ожидали Лазарь, его сестры и их друзья; но поскольку каждый день множество людей приходили поговорить с Лазарем о его воскрешении, Иисусу сообщили, что он остановится у верующего соседа по имени Симон, который после смерти отца Лазаря был самым уважаемым жителем этого города.
\vs p172 0:2 В тот же день вечером Иисус принял многих, и простые люди из Вифании и Виффагии делали все возможное, чтобы он почувствовал их радушие. Хотя многие полагали, что Иисус, полностью пренебрегая вынесенным синедрионом смертельным приговором, направляется теперь в Иерусалим, чтобы провозгласить себя царем евреев, вифанийская семья --- Лазарь, Марфа и Мария --- более глубоко осознавали, что Учитель был царем особого рода; они смутно догадывались, что это, быть может, его последнее посещение Иерусалима и Вифании.
\vs p172 0:3 Первосвященники были осведомлены, что Иисус остановился в Вифании, но сочли за лучшее даже и не пытаться схватить его, пока он среди друзей; они решили дождаться его прихода в Иерусалим. Иисус обо всем этом знал, но сохранял величественное спокойствие; его друзья никогда не видели его более спокойным и сердечным; даже апостолы были изумлены его безразличием к тому, что синедрион призвал все еврейство предать его в их руки. В ту ночь, пока Учитель спал, апостолы по двое охраняли его, и многие из них были вооружены мечами. Рано утром на следующий день они были разбужены сотнями паломников, несмотря на субботний день пришедших из Иерусалима, чтобы увидеть Иисуса и Лазаря, воскрешенного им из мертвых.
\usection{1. Суббота в Вифании}
\vs p172 1:1 Паломники, приходившие из\hyp{}за границ Иудеи, равно как и еврейские власти, постоянно задавали вопрос: «Как вы думаете? Иисус придет на праздник?» Поэтому, когда люди услышали, что Иисус в Вифании, они обрадовались, первосвященники же и фарисеи были несколько озадачены. Они были довольны тем, что он окажется в их власти, но в то же время были в некотором замешательстве от его смелости; они помнили, что во время предыдущего его посещения Вифании был воскрешен из мертвых Лазарь, и Лазарь становился серьезной проблемой для врагов Иисуса.
\vs p172 1:2 За шесть дней до Пасхи, в субботний вечер вся Вифания и Виффагия собрались, чтобы ознаменовать приход Иисуса всеобщим пиром в доме у Симона. Ужин был устроен в честь Иисуса и Лазаря; и это был вызов синедриону. Марфа руководила приготовлением и подачей на стол пищи; ее сестра Мария была среди женщин, наблюдавших со стороны, поскольку участие женщины в званом пиру противоречило обычаю евреев. Там были и люди синедриона, но они боялись схватить Иисуса, пока он находился среди своих друзей.
\vs p172 1:3 Иисус беседовал с Симоном об Иисусе Навине, своем тезке, и рассказывал, как тот Иисус и израильтяне пришли в Иерусалим через Иерихон. Объясняя легенду о падении стен Иерихона, Иисус сказал: «Меня не заботят такие стены из кирпича и камня; но я хотел бы сделать так, чтобы стены из предрассудков, себялюбивой праведности и ненависти рухнули перед проповедью любви Отца ко всем людям».
\vs p172 1:4 Пир проходил очень весело и был вполне обычным за исключением того, что апостолы были необычайно сосредоточенными. Иисус был исключительно весел и играл с детьми перед тем, как подойти к столу.
\vs p172 1:5 \pc Все шло своим чередом, и лишь незадолго до окончания пира Мария, сестра Лазаря отошла от женщин, наблюдавших за пиром, и, приблизившись к тому месту, где, как почетный гость, возлежал Иисус, открыла большой алебастровый сосуд с очень редким и драгоценным миро и, умастив голову Учителя, помазала им его ноги и отерла их своими волосами. Весь дом наполнился благоуханием от мира, и все присутствующие были изумлены поступком Марии. Лазарь ничего не сказал, но когда некоторые люди зароптали, выражая негодование, что столь драгоценное миро используется таким образом, Иуда Искариот подошел к месту, где возлежал Андрей, и сказал: «Почему бы это миро не продать и деньги не пожертвовать на пропитание бедных? Тебе следует поговорить с Учителем, чтобы он осудил такую расточительность».
\vs p172 1:6 Иисус, который знал, что они думали, и слышал, что они говорили, возложил руку на голову Марии, стоявшей подле него на коленях, обратил на нее взор, исполненный доброты, и сказал: «Оставьте ее в покое, все вы. Почему вы ее укоряете за это, хотя она сделала доброе дело от всего сердца? Вам, которые ропщут и говорят, что это миро следовало продать и раздать деньги бедным, я могу сказать, что бедные всегда будут с вами, так что вы можете служить им в любой момент, когда вам захочется; но я не всегда буду с вами; я скоро ухожу к моему Отцу. Эта женщина сберегла это миро, чтобы умастить им мое тело, когда его будут хоронить и теперь, когда в ожидании моей смерти она сочла за благо использовать это миро, не следует лишать ее этого удовольствия. Этот поступок Марии --- укор всем вам, ибо своим действием она доказывает веру в то, что я говорил о моей смерти и вознесении к моему Отцу на небеса. Эта женщина не будет осуждена за то, что она сделала в этот вечер; напротив, говорю я вам, что в грядущие века везде, где будет проповедоваться это евангелие, по всему свету будут помнить о ней и говорить о том, что она сделала».
\vs p172 1:7 Именно из\hyp{}за этого упрека, который Иуда Искариот воспринял как укор лично ему, он принял окончательное решение постараться отомстить за свои оскорбленные чувства. Много раз такие мысли возникали у него подсознательно, но теперь он решился обдумать эти грешные мысли явно и осознанно. Кстати, многие сочувствовали подобному его отношению к поступку Марии, поскольку цена этого миро была равна сумме, зарабатываемой одним человеком за один год --- достаточной, чтобы приобрести хлеб для пяти тысяч человек. Но Мария любила Иисуса; она приобрела это драгоценное миро, чтобы набальзамировать им его мертвое тело, ибо верила его словам, когда он предупреждал их, что должен умереть, и если она передумала, решив преподнести Учителю этот дар, пока он еще жив, нельзя было отказать ей в этом праве.
\vs p172 1:8 И Лазарь, и Марфа знали, что Мария долго копила деньги, чтобы купить этот сосуд с ароматическим нардовым миро, они искренне одобряли, что она поступает в подобном вопросе так, как велит ей сердце, поскольку они были состоятельными людьми и легко могли позволить себе преподнести такой дар.
\vs p172 1:9 Когда первосвященники услышали об этом обеде, устроенном в Вифании в честь Иисуса и Лазаря, они стали обсуждать между собой, что делать с Лазарем. И вскоре они решили, что Лазарь тоже должен умереть. Они справедливо заключили, что бесполезно было бы казнить Иисуса и оставить в живых Лазаря, которого он воскресил из мертвых.
\usection{2. В воскресное утро с апостолами}
\vs p172 2:1 В это воскресное утро в прекрасном саду у Симона Учитель собрал вокруг себя двенадцать апостолов и дал им последние наставления перед входом в Иерусалим. Он сказал им, что произнесет, вероятно, много речей и много еще будет учить прежде, чем вернется к Отцу, но посоветовал апостолам воздерживаться от публичной деятельности в Иерусалиме во время Пасхи. Он велел им оставаться поблизости от него и «смотреть и молиться». Иисус знал, что многие из его апостолов и ближайших последователей даже и тогда имели при себе мечи, спрятанные под одеждой, но не упомянул об этом.
\vs p172 2:2 Наставления, данные этим утром, включали в себя также краткий обзор их служения со дня их посвящения возле Капернаума и вплоть до нынешнего дня, когда они собирались войти в Иерусалим. Апостолы слушали молча; они не задавали никаких вопросов.
\vs p172 2:3 В тот день рано утром Давид Зеведеев передал Иуде деньги, вырученные от продажи имущества, оставшегося от лагеря возле Пеллы, а Иуда, в свою очередь, передал большую часть этих денег на хранение Симону, у которого они жили, на случай чрезвычайных обстоятельств, которые могли возникнуть в связи с их входом в Иерусалим.
\vs p172 2:4 После совещания с апостолами Иисус разговаривал с Лазарем и повелел ему стараться не приносить свою жизнь в жертву мстительности синедриона. Следуя именно этому указанию, Лазарь бежал в Филадельфию несколько дней спустя, когда члены синедриона послали людей, чтобы схватить его.
\vs p172 2:5 Так или иначе, все последователи Иисуса ощущали надвигающийся кризис, но полностью осознать всю его серьезность не позволяло им необычайно радостное и исключительно хорошее настроение Учителя.
\usection{3. Отбытие в Иерусалим}
\vs p172 3:1 Вифания находилась примерно в двух милях от храма, и в это воскресенье днем в половине второго Иисус собрался отправиться в Иерусалим. Он испытывал чувство глубокой привязанности к Вифании и живущим в ней простым людям. Назарет, Капернаум и Иерусалим отвергли его, но Вифания его приняла, уверовала в него. И именно в этом небольшом селении, где почти каждый мужчина, женщина и ребенок были верующими, он решил совершить самое грандиозное деяние периода своего земного пришествия, воскрешение Лазаря. Он воскресил Лазаря не для того, чтобы жители могли уверовать, а потому, что они уже верили.
\vs p172 3:2 Все утро Иисус думал о своем входе в Иерусалим. До этого он всегда старался пресекать все попытки народа провозглашать его Мессией, но теперь ситуация изменилась; он приближался к концу своей жизни во плоти, синедрионом было вынесено постановление о его смерти, и не было никакого вреда в том, что его ученикам было бы позволено свободно выразить свои чувства, как этого можно было ожидать, если бы он решил войти в город торжественно и открыто.
\vs p172 3:3 Иисус решил совершить этот торжественный вход в Иерусалим не как последний шаг, чтобы привлечь к себе народ, и не как последний рывок к власти. Он сделал это и не только для того, чтобы удовлетворить человеческие желания своих учеников и апостолов. Иисус не питал никаких иллюзий, свойственных беспочвенным мечтателям; он хорошо знал, каков должен быть исход этого посещения.
\vs p172 3:4 Решив торжественно войти в Иерусалим, Учитель оказался перед необходимостью избрать надлежащий способ для осуществления такого решения. Иисус перебрал в уме все многочисленные, но в той или иной мере противоречащие друг другу так называемые мессианские пророчества, и только лишь одно из них представлялось подходящим для того, чтобы действовать в соответствии с ним. В большинстве из этих пророчеств говорилось о смелом и решительном царе, сыне и наследнике Давида, который на время избавит весь Израиль от ига иноземного господства. Но в писании были слова, которые иногда связывали с Мессией те, кто придерживался в основном духовного понимания его миссии, и которыми, по мнению Иисуса, с полным основанием можно было руководствоваться во время его планируемого входа в Иерусалим. Это были слова из книги пророка Захарии, и они гласили: «Ликуй от радости, О дочь Сиона; торжествуй, О дочь Иерусалима. Смотри, царь твой грядет к тебе. Он праведный и несет спасение. Он кроткий, едет, сидя на ослице и молодом осле, сыне подъяремной».
\vs p172 3:5 \pc Царь\hyp{}воитель всегда въезжал в город верхом на коне; царь, стремящийся к миру и дружбе, всегда въезжал верхом на осле. Иисус не хотел въезжать в Иерусалим на коне, но соглашался мирно и с добрыми намерениями, как Сын Человеческий, въехать верхом на осле.
\vs p172 3:6 \pc Иисус долгое время прямо учил своих апостолов и учеников тому, что его царство не от мира сего, что оно имеет исключительно духовную природу; но все его усилия донести это до их сознания оказались тщетными. Теперь, обращаясь к символам, он хотел добиться того, что ему не удалось сделать путем непосредственного и личного обучения. Соответственно, сразу после полуденной трапезы Иисус позвал Петра и Иоанна, велел им отправиться в Виффагию, соседнее селение, находящееся чуть в стороне от главной дороги на небольшом расстоянии к северо\hyp{}западу от Вифании, и сказал: «Идите в Виффагию, и когда дойдете до пересечения дорог, вы найдете привязанного там молодого осла. Отвяжите его и приведите сюда. Если кто\hyp{}нибудь спросит вас, зачем вы это делаете, скажите просто: „Он нужен Учителю“». И когда эти два апостола отправились в Виффагию, как велел Учитель, они нашли молодого осла, привязанного рядом с ослицей прямо на улице возле углового дома. Когда Петр стал отвязывать осленка, подошел хозяин и спросил, почему они это делают, а когда Петр ответил ему так, как велел Иисус, человек сказал: «Если ваш Учитель --- Иисус из Галилеи, то пусть он берет осленка». И они вернулись, приведя этого осленка с собой.
\vs p172 3:7 К этому времени вокруг Иисуса и его апостолов собрались несколько сот паломников. С утра здесь останавливались люди, идущие на празднование Пасхи. Тем временем Давид Зеведеев и некоторые из его бывших вестников поспешили в Иерусалим и энергично распространяли среди находившихся вокруг храма толп паломников весть о том, что Иисус из Назарета скоро торжественно войдет в город. Соответственно, несколько тысяч этих паломников дружно ринулись приветствовать пророка и чудотворца, о котором было столько разговоров и которого некоторые считали Мессией. Выйдя из Иерусалима, эти люди встретили Иисуса и толпу, идущую в город, когда те только миновали выступ Масличной горы и начали спускаться вниз к городу.
\vs p172 3:8 Когда процессия вышла из Вифании, в празднично настроенной толпе учеников, верующих и паломников, многие из которых пришли из Галилеи и Переи, царил величайший энтузиазм. Прямо перед выходом появились двенадцать женщин из первоначальной женской апостольской группы в сопровождении некоторых из своих сподвижниц и присоединились к этой необыкновенной процессии, с радостным оживлением направившейся в сторону города.
\vs p172 3:9 Перед тем, как двинуться в путь, близнецы Алфеевы накрыли осла своими плащами и держали его, пока на него садился Учитель. Когда процессия двигалась к вершине Масличной горы, празднично настроенные люди постилали на землю свою одежду и приносили ветви растущих поблизости пальм, чтобы в знак почтения выстлать ковер, по которому ступал бы осел, везущий на себе царственного Сына, обетованного Мессию. По дороге к Иерусалиму радостная толпа стала петь или, точнее, громко кричать в унисон псалом «Осанна сыну Давидову; благословен грядущий во имя Господне. Осанна в вышних. Благословенно царство, нисходящее с небес».
\vs p172 3:10 В пути Иисус был беспечален и радостен до тех пор, пока они не дошли до вершины Масличной горы, откуда открывалась панорама города и башен храма; там Учитель остановил шествие, и когда они увидели, что он плачет, наступила полнейшая тишина. Глядя на огромную толпу, идущую из города приветствовать его, Учитель сказал взволнованно и печально: «О, Иерусалим, если бы ты только узнал, именно ты, хотя бы в сей твой день, что служит к миру твоему и что ты мог так легко обрести! Но это великолепие скоро будет скрыто от твоих глаз. Ты собираешься отвергнуть Сына Мира и отвернуться от евангелия спасения. Скоро придут на тебя дни, когда враги твои обложат тебя окопами и окружат тебя со всех сторон; они полностью разрушат тебя и не оставят камня на камне. И все это постигнет тебя за то, что ты не узнал времени твоего божественного посещения. Ты собираешься отвергнуть дар Бога, и все люди отвергнут тебя».
\vs p172 3:11 Когда он кончил говорить, они стали спускаться вниз с Масличной горы и вскоре соединились с толпой паломников, вышедших из Иерусалима, которые размахивали пальмовыми ветвями, кричали «осанна!» и иным образом выражали ликование и свое расположение. Учитель не планировал, что эти толпы выйдут из Иерусалима и будут встречать его; это было сделано другими людьми. Он никогда не придумывал никаких внешних эффектов.
\vs p172 3:12 Вместе с массой людей, вышедших приветствовать Учителя, пришли также многие фарисеи и другие его враги. Это внезапное и неожиданное бурное выражение народных симпатий привело их в такое смятение, что они побоялись схватить его, поскольку это могло бы вызвать открытый бунт населения. Они чрезвычайно боялись настроений многочисленных паломников, которые много слышали об Иисусе и многие из которых верили в него.
\vs p172 3:13 По мере приближения к Иерусалиму толпа становилась все более возбужденной, настолько, что некоторые фарисеи, подойдя к Иисусу, сказали: «Учитель, тебе следовало бы сделать замечание твоим ученикам и призвать их вести себя более благопристойно». Иисус ответил: «Вполне естественно, что эти дети приветствуют Сына Мира, которого отвергли первосвященники. Бесполезно останавливать их, иначе эти придорожные камни возопили бы вместо них».
\vs p172 3:14 Фарисеи поспешили вперед, обогнали процессию и, присоединившись к Синедриону, который собрался в это время в храме на совет, сказали своим сподвижникам: «Смотрите, все, что мы делаем, бесполезно; этот Галилеянин смешал наши планы. Народ без ума от него; если мы не остановим этих невеж, весь мир пойдет за ним».
\vs p172 3:15 В действительности, неправильно было бы усматривать глубокий смысл в этом поверхностном и спонтанном взрыве народного энтузиазма. Этот радушный прием, хотя он действительно был радостным и искренним, не свидетельствовал о каких\hyp{}либо подлинно глубоких убеждениях в сердцах этой празднично настроенной толпы. Позже на той же неделе эти самые толпы ровно с такой же готовностью быстро отвергли Иисуса, когда синедрион занял твердую и решительную позицию по отношению к нему и когда они сами были разочарованы, поняв, что Иисус не собирается устанавливать царство в соответствии с их издавна лелеемыми надеждами.
\vs p172 3:16 Но весь город был сильно взбудоражен, настолько, что все спрашивали: «Кто этот человек?» И толпа отвечала: «Это Иисус, пророк из Назарета Галилейского».
\usection{4. Посещение храма}
\vs p172 4:1 Пока близнецы Алфеевы возвращали осла его владельцу, Иисус и остальные десять апостолов, отделившись от своих ближайших сподвижников, отправились бродить по храму, наблюдая приготовления к Пасхе. Никаких попыток преследования Иисуса не было, поскольку синедрион чрезвычайно опасался народа, и это было одной из причин того, что Иисус позволил толпам так бурно приветствовать его. Апостолы плохо понимали, что это было единственно возможным человеческим действием, способным предотвратить арест Иисуса сразу же после его входа в город. Учитель стремился дать жителям Иерусалима всех сословий, равно как и десяткам тысяч паломников, пришедших на Пасху, еще одну, последнюю возможность услышать евангелие и принять, если они пожелают, Сына Мира.
\vs p172 4:2 И теперь, когда близился вечер и народ отправился на поиски пропитания, Иисус и его ближайшие последователи остались одни. Какой это был необычный день! Апостолы были задумчивы, но молчаливы. Никогда за все годы своего общения с Иисусом они не переживали такого. Они немного посидели возле сокровищницы, глядя, как люди вносят свои пожертвования: помногу клали в ящик для пожертвований богатые, и все отдавали, кто что может, в соответствии с размерами своей собственности. Наконец, пришла совсем плохо одетая бедная вдова, и они увидели, как она бросила в отверстие две лепты (мелкие медные монеты). И тогда Иисус сказал, привлекая внимание апостолов к этой вдове: «Хорошенько обратите внимание на то, что вы увидели. Эта бедная вдова внесла больше остальных, потому что эти пожертвовали совсем небольшие для них суммы --- от избытка своего, а эта бедная женщина, хоть и сама пребывает в нужде, отдала все, что имела, все пропитание свое».
\vs p172 4:3 Вечерело, они ходили в молчании по храму, и после того, как Иисус еще раз посмотрел на эти знакомые места и, вспомнив все, что он испытывал во время предыдущих посещений, не исключая и самые первые, он сказал: «Давайте пойдем отдыхать в Вифанию». Иисус вместе с Петром и Иоанном отправились в дом к Симону, а остальные апостолы остановились на ночлег у их друзей в Вифании и Виффагии.
\usection{5. Настроение апостолов}
\vs p172 5:1 В этот воскресный вечер Иисус шел по дороге в Вифанию впереди апостолов. Не проронив ни единого слова, они дошли до дома Симона, где и расстались. Никогда никакие двенадцать человек не испытывали таких разнообразных и необъяснимых чувств, как те, что переполняли сейчас умы и души этих посланцев царства. Эти стойкие галилеяне пребывали в смущении и замешательстве; они не знали, чего ждать дальше; они были слишком удивлены, чтобы испытывать ощутимый страх. Они ничего не знали о планах Учителя на предстоящий день и не задавали никаких вопросов. Они отправились на ночлег, хотя все, кроме близнецов, спали плохо. Но в эту ночь они не несли вооруженную охрану Иисуса в доме у Симона.
\vs p172 5:2 Андрей был сильно озадачен и пришел почти в полное замешательство. Он был единственным из апостолов, кто и не думал о том, что означают бурные приветствия народа. Он был слишком озабочен мыслями о своих обязанностях главы апостольской группы, чтобы серьезно задумываться о значении и значимости громких славословий толпы. Андрей постоянно следил за некоторыми из своих сподвижников, в частности, за Петром, Иаковом, Иоанном и Симоном Зилотом, которых, как он опасался, в период радостного волнения могли слишком захлестнуть эмоции. Весь этот день и последующие дни Андрея терзали серьезные сомнения, но он ни разу не сказал о своих недобрых предчувствиях товарищам\hyp{}апостолам. Он был озабочен настроем некоторых из них, которые, как он знал, были вооружены мечами; но он не знал, что и его собственный брат Петр имел при себе это оружие. Так что шествие в Иерусалим произвело на Андрея относительно слабое впечатление; он был слишком занят обязанностями, связанными с его должностью, и что\hyp{}то другое не могло его сильно отвлечь.
\vs p172 5:3 Симон Петр поначалу был совершенно опьянен этим всенародным выражением восторга; но когда они вечером вернулись в Вифанию, он несколько успокоился. Петр терялся в догадках, что собирается делать Учитель. Он был ужасно разочарован тем, что Иисус не сделал никакого заявления на волне этого народного восторга. Петр не мог понять, почему Иисус не стал говорить перед народом, когда они пришли в храм, или хотя бы не позволил одному из апостолов обратиться к толпе. Петр был великим проповедником, и ему неприятно было видеть, что они упустили такой подходящий случай выступить перед восприимчивой и полной энтузиазма толпой потенциальных слушателей. Ему очень хотелось прямо там, в храме обратиться к этой толпе с проповедью царства; но Учитель недвусмысленно повелел не заниматься ни обучением, ни проповедованием в Иерусалиме в эту пасхальную неделю. Впечатляющее шествие в город оказало слишком сильное воздействие на Симона Петра; но к ночи он успокоился и был невыразимо печален.
\vs p172 5:4 Для Иакова Зеведеева это воскресенье было днем растерянности и глубокого замешательства; он не мог уяснить смысл происходящего; он не мог понять намерений Учителя, который допустил этот шквал восторженных приветствий, а затем, когда они дошли до храма, отказался сказать людям хотя бы пару слов. Когда процессия двигалась вниз с Масличной горы к Иерусалиму и особенно когда их встретили тысячи паломников, вышедших приветствовать Учителя, Иакова раздирали противоречивые чувства, с одной стороны, восторга и радости от того, что он видел, а с другой стороны, сильного страха по поводу того, что произойдет, когда они дойдут до храма. А позже он был удручен и охвачен чувством разочарования, когда Иисус слез с осла и стал неторопливо прохаживаться по дворам храма. Иаков не мог понять, почему не была использована такая великолепная возможность провозгласить царство. Ночью его разум оказался во власти тревожной и ужасающей неопределенности.
\vs p172 5:5 Иоанн Зеведеев был близок к пониманию того, почему Иисус так поступил; по крайней мере, он отчасти осознал духовную значимость этого, так называемого, триумфального входа в Иерусалим. В тот момент, когда толпа двигалась к храму и Иоанн видел своего Учителя, сидящего верхом на осленке, он вспомнил, как Иисус цитировал однажды отрывок из Писания --- слова Захарии, описывающие приход Мессии как человека, идущего с миром и въезжающего в Иерусалим на осле. Когда Иоанн обдумал эту цитату из Писания, он начал понимать символическую значимость этой торжественной процессии. По крайней мере, он осознал значение этой цитаты из Писания в достаточной степени, чтобы в какой\hyp{}то мере порадоваться этому эпизоду и избежать чрезмерного чувства подавленности из\hyp{}за явно безрезультатного окончания этого триумфального шествия. Иоанн от природы имел склад ума, тяготеющий к символическому мышлению и восприятию.
\vs p172 5:6 \pc Филипп был совершенно выбит из колеи неожиданностью и стихийностью происходящего. Спускаясь с Масличной горы он не мог в достаточной степени собраться с мыслями, чтобы прийти к какому\hyp{}то определенному мнению о значении этого всеобщего проявления чувств. В некотором смысле он был рад происходящему, поскольку воздавались почести его Учителю. К тому времени, когда они дошли до храма, он был сильно обеспокоен мыслью, что Иисус может попросить его накормить эту толпу, так что поведение Иисуса, спокойно ушедшего от толпы, которое стало причиной такого болезненного разочарования большинства апостолов, вызвало у Филиппа чувство сильного облегчения. Толпа иногда оказывалась тяжелым испытанием для управляющего хозяйством апостольской группы. Избавившись от личных опасений насчет материальных потребностей толпы, Филипп поддержал Петра, выражавшего разочарование по поводу того, что ничего не было предпринято для учения народа. В ту ночь Филипп стал обдумывать все эти переживания и невольно стал подвергать сомнению всю идею царства; он искренне не понимал, что происходит, но никому не высказывал своих сомнений; он слишком любил Иисуса. Он обладал огромной личной верой в Учителя.
\vs p172 5:7 \pc Нафанаил, если не говорить о символических и пророческих аспектах, ближе всех подошел к пониманию причин, по которым Учитель заручился поддержкой толп паломников, пришедших на Пасху. Раньше, чем они дошли до храма, он рассудил, что без такого впечатляющего входа в Иерусалим Иисус был бы схвачен членами синедриона и брошен в тюрьму тотчас же, как только осмелился бы войти в город. Поэтому он ничуть не был удивлен, что Учитель никак не использовал восторженно приветствовавшие его толпы после того, как вошел в пределы стен города и произвел, тем самым, настолько сильное впечатление на еврейских правителей, что они воздержались немедленно взять его под стражу. Понимая истинную причину того, почему Учитель вошел в город именно таким образом, в дальнейшем Нафанаил воспринимал происходящее с большим самообладанием и был в меньшей степени, чем другие апостолы, обеспокоен и разочарован последующим поведением Иисуса. Нафанаил очень верил в способность Иисуса понимать людей, равно как и в его прозорливость и искусность в разрешении сложных ситуаций.
\vs p172 5:8 \pc Матфей сначала пришел в замешательство от одного вида этой грандиозной процессии. Он не улавливал смысла того, что видели его глаза, до тех пор, пока тоже не вспомнил то место из книги Захарии, где пророк говорит о радости, охватившей Иерусалим потому, что пришел его царь, едущий на осле и несущий спасение. Когда процессия двигалась в направлении города и затем стала приближаться к храму, Матфея охватил восторг; он был уверен, что, когда Учитель дойдет до храма во главе этой кричащей массы народа, произойдет что\hyp{}то необыкновенное. Когда один из фарисеев стал осмеивать Иисуса, говоря: «Смотрите все, смотрите, кто едет сюда, царь евреев верхом на осле!», Матфей не ударил его, лишь призвав на помощь все свое самообладание. В тот вечер по пути обратно в Вифанию ни один из двенадцати апостолов не чувствовал себя более подавленным. Подобно Симону Петру и Симону Зилоту, он испытал сильнейшее нервное потрясение и к ночи был в состоянии полного истощения. Но утром Матфей существенно приободрился; в конце концов, он всегда умел не падать духом, несмотря на неудачи.
\vs p172 5:9 \pc Из всех двенадцати апостолов Фома был смущен и озадачен в наибольшей степени. Почти все время он просто шел вместе со всеми, пристально глядя на это зрелище, и искренне недоумевал, какими мотивами руководствуется Учитель, участвуя в таком необычном шествии. В глубине души он считал происходящее несколько несерьезным, если не сказать глупым. Он никогда не видел, чтобы Иисус делал что\hyp{}либо подобное, и затруднялся объяснить его странное поведение в этот воскресный день. К тому моменту, когда они дошли до храма, Фома пришел к заключению, что это проявление симпатий народа имело целью напугать синедрион настолько, чтобы они не осмелились немедленно схватить Учителя. По дороге обратно в Вифанию Фома много думал, но ничего не говорил. Ближе к ночи прозорливость Учителя, устроившего этот шумный вход в Иерусалим, начала обретать в его глазах несколько забавную привлекательность, и такой взгляд на вещи сильно его приободрил.
\vs p172 5:10 Для Симона Зилота начало этого воскресенья предвещало великий день. Он грезил об удивительных деяниях в Иерусалиме в последующие несколько дней, и в этом был прав, но Симон мечтал об установлении у евреев новой государственной власти с Иисусом на троне Давида. Симону виделось, как националисты сразу примутся за дело, как только царство будет провозглашено, а сам он станет во главе собирающихся войск нового царства. По дороге с Масличной горы ему даже представлялось, что члены синедриона и все их приспешники будут мертвы в этот же день еще до захода солнца. Он действительно верил, что случится нечто великое. Из всей толпы он был самым громогласным. К пяти часам дня он был молчаливым, подавленным и разочарованным апостолом. Он так никогда полностью и не вышел из депрессии, которая охватила его в результате пережитого в тот день потрясения; по крайней мере, он еще долго испытывал ее и после воскресения Учителя.
\vs p172 5:11 Для близнецов Алфеевых это был великолепный день. Они весь день были счастливы, и, поскольку их не было во время тихого посещения храма, им в основном удалось избежать всеобщего упадка настроения. Они не могли понять подавленности апостолов после возвращения в Вифанию в тот вечер. Этот день навсегда остался в памяти близнецов как день, в который они, находясь на земле, были ближе всего к небу. Это был самый счастливый день за все время их служения в качестве апостолов. И память о восторге, испытанном в этот воскресный день, помогала им преодолевать все горести этой насыщенной событиями недели вплоть до часа распятия. Именно так, по их представлениям, и должен был войти в город царь; они испытывали восторг непрерывно на протяжении всего торжественного шествия. Они полностью одобряли все, что видели, и долго вспоминали об этом.
\vs p172 5:12 Из всех апостолов наиболее неодобрительно воспринял этот торжественный вход в Иерусалим Иуда Искариот. Он был раздражен упреком, высказанным ему накануне Учителем в связи с поступком Марии, умастившей ноги Иисуса на пиру в доме у Симона. Иуде было противно это зрелище. Оно казалось ему несерьезным, чтобы не сказать нелепым. Когда этот мстительный апостол в этот воскресный день взирал на все происходящее, ему казалось, что Иисус больше похож на шута, чем на царя. Он сильно негодовал по поводу самого этого шествия. Он разделял взгляды греков и римлян, которые испытывали презрение к любому, кто согласился бы ехать верхом на осле или осленке. К тому времени, когда триумфальная процессия вошла в город, Иуда был уже практически близок к решению полностью отбросить идею такого царства; он почти твердо решил отказаться от таких нелепых попыток установить царство небесное. Но затем он вспомнил воскрешение Лазаря и многое другое и решил остаться пока с апостолами, по крайней мере, еще на один день. Кроме того, он нес сумку и не стал бы бросать апостолов, оставив у себя их деньги. В ту ночь, возвращаясь в Вифанию, все апостолы были в равной степени удручены и молчаливы, и его поведение не показалось странным.
\vs p172 5:13 Огромное влияние на Иуду оказали насмешки его друзей саддукеев. Ничто так сильно не повлияло на его окончательное решение покинуть Иисуса и своих товарищей апостолов, как один эпизод, произошедший в то время, когда Иисус был у ворот города: один видный саддукей (друг семьи Иуды) устремился к нему с торжествующе\hyp{}насмешливым видом и, похлопывая его по спине, сказал: «Отчего такой обеспокоенный вид, друг мой? Будь веселей и вместе со всеми нами радостно приветствуй этого Иисуса из Назарета, царя евреев, въезжающего в ворота Иерусалима верхом на осле». Иуда никогда не отступал перед гонениями, но он не переносил такого рода насмешек. К давно питаемому чувству мести теперь добавилась эта пагубная боязнь насмешек, это ужасное и чудовищное чувство стыда за своего Учителя и своих товарищей апостолов. В душе этот посвященный посланник царства уже был отступником; ему оставалось лишь найти какой\hyp{}нибудь благовидный предлог для открытого разрыва с Учителем.
