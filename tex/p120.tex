\upaper{120}{Пришествие Михаила на Урантии}
\author{Мантутия Мелхиседек}
\vs p120 0:1 Я, Мелхиседек, руководитель комиссии по откровению, назначенный Гавриилом надзирать за подготовкой нового описания жизни Михаила на Урантии в образе смертного, уполномочен рассказать вам о некоторых событиях, непосредственно предшествовавших прибытию Сына\hyp{}Творца на Урантию для последнего этапа его пришествий в сотворенной вселенной. Пришествия Сына\hyp{}Творца в образах сотворенных им существ различных планов бытия, то есть проживание тех жизней, которые он предназначает для своих разумных созданий, --- часть той цены, которую каждый Сын\hyp{}Творец должен заплатить за полное и верховное владычество в сотворенной им самим вселенной со всем, что в ней есть.
\vs p120 0:2 До событий, о которых я собираюсь поведать, произошло шесть пришествий Михаила из Небадона в образах сотворенных им разумных существ, принадлежащих шести различным чинам сотворенной вселенной. Теперь он готов был воплотиться на Урантии в облике смертного человека --- представителя наинизшего чина сотворенных им и наделенных разумом и волей существ, --- и в этом пришествии сделать последний шаг к обретению вселенского владычества, согласно установлениям божественных Райских Правителей вселенной вселенных.
\vs p120 0:3 В каждом из своих пришествий Михаил не только обретал конечный опыт одной из групп сотворенных им самим существ, но также обретал весьма важный опыт сотворчества с Раем, все более становясь благодаря этому владыкой созданной им вселенной. В любой момент на протяжении предыдущего времени существования его локальной вселенной Михаил мог бы заявить о своем личном владычестве как Сын\hyp{}Творец и как Сын\hyp{}Творец мог бы управлять своей вселенной по собственному усмотрению. В этом случае Иммануил и сотрудничающие с ним Райские Сыны покинули бы вселенную. Но Михаил не хотел править Небадоном самовластно как Сын\hyp{}Творец. Он стремился через сотворчество, в согласии с волей Райской Троицы взойти к столь высокому вселенскому статусу, чтобы управлять своей вселенной и вершить ее дела с совершенной проницательностью и мудростью, какие когда\hyp{}нибудь станут присущи возвышенному правлению Верховного Существа. Он стремился не к тому совершенству, которое достижимо в правлении Сына\hyp{}Творца, а к высшему совершенству правления, которое воплощает в себе вселенскую мудрость и божественный опыт Верховного Существа.
\vs p120 0:4 Таким образом, эти семь пришествий Михаила в семи различных чинах его вселенских созданий имели двойную цель. Во\hyp{}первых, он обретал понимание своих созданий, которое необходимо каждому Сыну\hyp{}Творцу на пути достижения им полного владычества. В любой момент Сын\hyp{}Творец может стать полноправным владыкой сотворенной им вселенной, но верховным представителем Райской Троицы он становится лишь после семи пришествий к созданиям своей вселенной. Во\hyp{}вторых, Михаил стремился представить максимальную власть Райской Троицы, насколько это возможно в непосредственном личном управлении вселенной. Каждое из его семи пришествий было связано с добровольным подчинением различным сочетаниям аспектов воли лиц Райской Троицы. В первом пришествии Михаил подчинялся соединенной воле Отца, Сына и Духа; во втором пришествии --- воле Отца и Сына; в третьем пришествии --- воле Отца и Духа; в четвертом пришествии --- воле Сына и Духа; в пятом пришествии --- воле Бесконечного Духа; в шестом пришествии --- воле Вечного Сына; наконец, в седьмом и последнем пришествии на Урантию --- воле Отца Всего Сущего.
\vs p120 0:5 Таким образом, личная власть Михаила опирается на соединение божественной воли вселенских Творцов во всех ее семи аспектах и на лично пережитый и познанный опыт различных созданий его вселенной. Это дает его правлению наиболее возможную силу и авторитет и в то же время полностью исключает произвол. Его сила неограничена, поскольку она исходит из лично пережитого им союза с Райскими Божествами; его авторитет неоспорим, поскольку основан на обладании опытом разумных существ его вселенной. Его владычество является верховным, поскольку воплощает в одно и то же время семеричную точку зрения Райских Божеств и точку зрения созданий материального мира, существующих во времени и пространстве.
\vs p120 0:6 \pc Определив место и время своего последнего пришествия, выбрав планету, на которой должно было состояться это уникальное событие, Михаил, как и накануне прежних своих пришествий, встретился, чтобы обсудить его, с Гавриилом, а затем предстал перед своим старшим братом и советником Рая Иммануилом. Все полномочия управления вселенной, которые не были прежде переданы Гавриилу, Михаил оставил теперь попечительству Иммануила. Принимая на время воплощения Михаила на Урантии заботу о вселенной, Иммануил дал ему напутствие, которым Михаил должен был руководствоваться во время своей жизни в человеческом облике на Урантии.
\vs p120 0:7 \pc В связи с этим следует иметь в виду, что Михаил выбрал осуществление последнего своего пришествия в образе смертного человека, подчиняясь воле Райского Отца. Сыну\hyp{}Творцу не нужны были бы никакие напутствия, если бы он совершал это воплощение с единственной целью --- достигнуть владычества над вселенной, но другая его задача --- откровение Верховного --- включала действие в согласии с разными аспектами воли Райских Божеств. Его конечное личное владычество должно было охватывать собой все семеричные аспекты божественной воли, как они кульминируют в Верховном. Поэтому и предыдущие шесть раз Михаила наставляли личные представители Райских Божеств или их соединений. Теперь его наставником был Объединяющий Дней --- посол Райской Троицы в локальной вселенной Небадон, действующий от имени Отца Всего Сущего.
\vs p120 0:8 \pc Готовность могущественного Сына\hyp{}Творца еще раз добровольно подчиниться воле Райских Божеств, на этот раз --- воле Отца Всего Сущего, давала непосредственные преимущества и вознаграждалась сполна. Благодаря своему решению о таком сотрудническом подчинении Михаил мог испытать в этом воплощении не только природу смертного человека, но и волю Райского Отца. Далее, во время этого особого пришествия Михаил мог не беспокоиться о своей вселенной, не только потому, что Иммануил в его отсутствие, управляя ею, пользовался всеми полномочиями Отца, но и потому, что Древние Дней сверхвселенной гарантировали на данное время безопасность его мира.
\vs p120 0:9 \pc Такова была ситуация во время знаменательного события, когда Иммануил напутствовал Михаила накануне седьмого пришествия. Из этого напутствия Иммануила правителю вселенной, впоследствии ставшему Иисусом из Назарета (Христом\hyp{}Михаилом) на Урантии, мне позволено привести следующие выдержки.
\usection{1. Напутствие к седьмому пришествию}
\vs p120 1:1 «Мой брат\hyp{}Творец, скоро увижу я твое седьмое и последнее пришествие. Ты добросовестно и в совершенстве выполнил свои задачи в предыдущих шести; не сомневаюсь, что такой же успех ожидает тебя и в этом последнем пришествии на пути к верховному владычеству. До сих пор ты появлялся на планетах своих пришествий в образах зрелых существ того или иного чина, согласно твоему выбору. Теперь же ты явишься на Урантию, объятую хаосом и раздираемую конфликтами планету, которую ты избрал, не взрослым зрелым человеком, а беспомощным младенцем. Это, мой друг, новый и неизведанный для тебя опыт. Тебе предстоит сполна, от начала до конца испытать жизнь смертного, но и столь же полным будет и твое просветление Творца, воплотившегося в тварное существо.
\vs p120 1:2 В каждом из предыдущих пришествий ты добровольно предавал себя воле трех Райских Божеств и их божественных соединений. Из семи аспектов воли Верховного ты испытал все, кроме личной воли твоего Райского Отца. Теперь, когда ты в седьмом пришествии добровольно вступил в подчинение воле твоего Отца, я, Его личный представитель, принимаю неограниченные полномочия по управлению твоей вселенной.
\vs p120 1:3 Сходя на Урантию, ты добровольно лишаешь себя всякой внепланетной поддержки и специальной помощи, которая могла бы быть оказана твоими творениями. Как благополучие сотворенных тобой сынов Небадона во время их жизни во вселенной целиком зависит от тебя, так и ты должен стать целиком и полностью зависим от своего Райского Отца во всех нераскрытых превратностях жизни на Урантии. Завершив это воплощение, ты воистину познаешь полное смысла и глубокое значение веры и надежды, достижения которых неизменно требуешь от своих творений в их сокровенных отношениях с тобой как Творцом и Отцом их локальной вселенной.
\vs p120 1:4 Пусть во время воплощения на Урантии у тебя будет лишь одна забота: сохранение неразрывной связи между тобой и твоим Райским Отцом; ведь именно благодаря совершенству этой связи между вами мир, в котором ты воплотишься, да и вся твоя вселенная, получат новое, более полное откровение твоего и моего Отца, Единого Отца Всего Сущего. Таким образом, ты должен целиком отдать себя жизни на Урантии. Я полностью отвечаю за безопасность и стабильность управления твоей вселенной с момента твоего добровольного ухода от власти до того, когда ты возвратишься к нам Вселенским Владыкой, признанным Раем, и снова получишь власть из моих рук --- уже не власть наместника, оставляемую мне, а верховную власть и полномочия в своей вселенной.
\vs p120 1:5 Чтобы ты был уверен, что я в силах исполнить все обещанное (хорошо зная, что я сам по себе являюсь гарантией Рая в том, что мое слово будет исполнено), я извещаю тебя, что мне только что был передан указ Древних Дней на Уверсе, который предотвращает все духовные опасности в Небадоне на период твоего добровольного воплощения. С момента, как ты отступишься от своего сознания, воплощаясь в смертного человека, и до твоего возвращения к нам высшим и неограниченным владыкой созданной и устроенной тобою вселенной, --- ничего серьезного не может случиться в Небадоне. В течение твоего воплощения я следую воле Древних Дней о мгновенном, автоматическом и безусловном уничтожении всякого существа в Небадоне, совершившего восстание или собирающегося его совершить. Брат мой, силы Рая, которые неотъемлемо присутствуют во мне и которым указ Уверсы придает законную силу, гарантируют безопасность твоей вселенной и всех ее верных созданий на время твоего пришествия. Ты можешь, исполняя свою миссию, отбросить все заботы, кроме одной: передать с новой глубиной откровение нашего Отца разумным существам твоей вселенной.
\vs p120 1:6 Напоминаю тебе, что, как и во время прежних твоих пришествий, я управляю Небадоном как твой брат и доверенное лицо. Я властвую от твоего имени и действую так, как действовал бы наш Райский Отец, в соответствии с твоим ясно высказанным пожеланием, чтобы именно так я поступал от твоего имени. Всю эту власть, все полномочия я возвращу тебе в любой момент, когда ты сочтешь, что для этого пришла пора. Твое пришествие от начала до конца совершенно добровольно. Как смертный человек ты не имеешь никакой небесной мощи, но в любой момент, когда пожелаешь, ты сможешь вернуть себе всю неограниченную силу и власть над вселенной. Но если ты пожелаешь вернуть свою власть и силу, то единственно по \bibemph{личным причинам,} поскольку я являюсь и живым и верховным гарантом благополучного управления твоей вселенной в согласии с волей Отца. Во время твоего отсутствия в Спасограде, в Небадоне исключены бунты, какие происходили в нем уже трижды. Древние Дней распорядились, чтобы в течение этого периода всякий мятеж в Небадоне нес в себе семя собственного уничтожения.
\vs p120 1:7 Итак, на то время, пока ты отсутствуешь среди нас, будучи в этом последнем необычайном воплощении, я даю слово (вместе с Гавриилом) добросовестно управлять твоей вселенной. Поручая тебе эту миссию божественного откровения и обретения опыта совершенного человеческого понимания, я действую от имени твоего и моего Отца. Выслушай же некоторые советы, которые помогут тебе в земной жизни и будут обретать смысл по мере того, как ты станешь осознавать божественную миссию своей жизни в плотском образе».
\usection{2. Условия пришествия}
\vs p120 2:1 «1. Согласно обычаям и в соответствии с принятым в Сынограде порядком --- определенным установлениями, исходящими от Вечного Сына, пребывающего в Раю, --- я подготовил все условия для твоего немедленного пришествия в облике смертного человека, по планам, которые составил ты сам, а Гавриил передал на мое попечение. Ты вырастешь на Урантии как обычный ребенок, получишь человеческое образование --- оставаясь постоянно в подчинении воле твоего Райского Отца, --- проживешь жизнь на Урантии, как сам определил, завершишь свой планетарный путь и приготовишься взойти к Отцу своему, чтобы получить от него верховное владычество в своей вселенной.
\vs p120 2:2 \ublistelem{2.}\bibnobreakspace Помимо твоей миссии на земле и вселенского откровения, но в естественной гармонии с этим, советую тебе, после того как ты в достаточной мере осознаешь свою божественную природу, взять на себя еще одну задачу: практически положить конец бунту Люцифера в системе Сатании и сделать это, будучи \bibemph{Сыном Человеческим ---} смертным существом, в слабости своей получившим силу благодаря основанному на вере подчинению воле Отца своего. Я предлагаю тебе милостиво выполнить то, что ты неоднократно отказывался совершить самовластно, когда только зарождался этот греховный и ничем не оправданный бунт. На мой взгляд, ты достигнешь вершины в своем пришествии, если вернешься к нам Сыном Человеческим, Планетарным Принцем Урантии, и в то же время Сыном Божьим, верховным владыкой своей вселенной. Как смертный человек, низшее разумное создание Небадона осуди богоборческие притязания Калигастии и Люцифера и, находясь в положении самого низшего из собственных созданий, навсегда положи конец измышлениям этих падших детей света. Ты неуклонно отказывался использовать свои возможности творца, чтобы ниспровергнуть мятежников; именно теперь, будучи подобным низшим своим созданиям, тебе надлежит вырвать власть из рук этих падших Сынов. И вся твоя локальная вселенная увидит раз и навсегда справедливость твоих действий, которые ты совершаешь в смертной плоти, из милосердия отказавшись предпринять их в полноте власти. Так, своим пришествием установив возможность владычества Верховного в Небадоне, ты на деле положишь конец всем последствиям всех прошлых восстаний, суд над которыми еще не совершен, независимо от того, как давно они произошли. Тогда все распри, угрожающие твоей вселенной, будут, в сущности, искоренены. И после получения тобой верховного владычества подобные вызовы твоей власти уже никогда не смогут повториться ни в одном из миров Небадона, твоего великого личного творения.
\vs p120 2:3 \ublistelem{3.}\bibnobreakspace Когда ты успешно преодолеешь раскол на Урантии, как это, несомненно, произойдет, советую тебе принять от Гавриила титул „Планетарный Принц Урантии“ в знак вечного признания твоей вселенной твоего последнего пришествия, и в дальнейшем предпринять все возможное, что согласуется с задачами твоего пришествия, для искупления смуты и страданий, вызванных на Урантии предательством Калигастии и последующим срывом Адама.
\vs p120 2:4 \ublistelem{4.}\bibnobreakspace Как ты пожелал, Гавриил и все остальные, принимающие участие в судьбе Небадона, помогут тебе завершить последнее пришествие диспенсационным судом, за которым последует окончание эпохи, воскрешение спящих в посмертии и начало нового пришествия Духа Истины.
\vs p120 2:5 \ublistelem{5.}\bibnobreakspace На планете своего воплощения я советую тебе взаимодействовать с современными тебе людьми в основном как учитель. Главное внимание направь на то, чтобы освободить и вдохновить духовную природу человека. Затем, стремись к просветлению темного человеческого ума, исцелению человеческих душ, освобождению человеческого разума от вековых страхов. И наконец, по своей человеческой мудрости служи физическому благополучию и материальному комфорту своих собратьев\hyp{}людей. Живи так, чтобы твоя совершенная религиозная жизнь вдохновляла и наставляла твою вселенную.
\vs p120 2:6 \ublistelem{6.}\bibnobreakspace В мире своего пришествия сделай отчужденного бунтом человека духовно свободным. На Урантии продолжай способствовать владычеству Верховного во всей сотворенной тобой вселенной. В этом пришествии, приняв образ из плоти и крови, ты испытаешь полное просветление Творца пространства и времени, благодаря опыту одновременного существования и как человеческая сущность и согласно воле Райского Отца. В твоей краткой земной жизни воли конечного творения и бесконечного Творца станут одним целым --- так же, как они сливаются в процессе эволюции Божественного в Верховном Существе. Излей Дух Истины на планету твоего пришествия, и тогда каждый нормальный смертный на этой изолированной сфере сможет непосредственно и со всей полнотой воспринять служение Настройщика Мысли, который является индивидуализированным присутствием нашего Райского Отца.
\vs p120 2:7 \ublistelem{7.}\bibnobreakspace Все свои действия на планете воплощения согласуй с тем, что ты проживаешь эту человеческую жизнь для обучения и наставления всей своей вселенной. Ты даруешь эту жизнь пришествия в образе \bibemph{смертного человека} Урантии, но предназначена \bibemph{эта жизнь,} чтобы духовно вдохновить каждое человеческое и сверхчеловеческое разумное существо, которое обитало, обитает или будет обитать в любом мире, который был создан, может быть создан или будет создан в любой части огромной галактики, которой ты управляешь. Ты не должен стремиться к тому, чтобы своей земной жизнью в облике смертного человека служить \bibemph{примером} для смертных Урантии, которые будут жить там одновременно с тобой, или для грядущих поколений на Урантии, или даже для всех грядущих поколений людей всех миров твоей вселенной. Скорее, твоя жизнь должна стать источником \bibemph{вдохновения} для всех жизней во всех мирах Небадона, во всех поколениях грядущих веков.
\vs p120 2:8 \ublistelem{8.}\bibnobreakspace Великая миссия, которую тебе предстоит осознать и осуществить в смертном воплощении, основана на твоем решении прожить жизнь, целиком подчиненную воле твоего Райского Отца, и благодаря этому явить \bibemph{откровение Бога,} твоего Отца, во плоти и прежде всего созданиям из плоти. Вместе с тем ты дашь новое расширенное \bibemph{толкование} Отца нашего и для сверхчеловеческих существ всего Небадона. Одновременно со служением нового откровения и углубленного раскрытия Райского Отца для человеческого и других типов разума, ты по\hyp{}новому откроешь Богу человека. Своей одной короткой земной жизнью покажешь всему Небадону невиданные доселе трансцендентные возможности знающего Бога человека на его кратком смертном пути, а разумным сверхчеловеческим существам Небадона дашь новое и просветленное \bibemph{понимание} человека и превратностей его планетарной жизни. Сойдя на Урантию в образе смертного и ведя жизнь человека своего времени и поколения, ты должен действовать так, чтобы продемонстрировать всей вселенной идеал совершенного метода верховного свершения в твоей гигантской вселенной: успех Бога, который ищет человека и находит его, и феномен человека, который ищет Бога и находит его; и сверши все это к взаимному удовлетворению обоих, за одну короткую жизнь во плоти.
\vs p120 2:9 \ublistelem{9.}\bibnobreakspace Я предупреждаю, ты должен всегда помнить, что, являясь фактически обыкновенным человеком, потенциально ты остаешься Сыном\hyp{}Творцом Райского Отца. Творческие атрибуты твоей личной божественности вместе с тобой прибудут из Спасограда на Урантию, хотя ты будешь жить и действовать на этой планете как Сын Человеческий. В любой момент после прибытия твоего Настройщика Мысли ты сможешь по собственной воле прекратить воплощение. До прибытия и принятия твоего Настройщика Мысли твоя личная идентичность охраняется мною. После этого, в соответствии со все большим осознанием природы, характера и значения своей миссии, тебе следует воздерживаться от любых не свойственных обычному человеку волеизъявлений, достижений и проявлений силы, ввиду того, что твои возможности Творца останутся и у твоей смертной личности, будучи неотделимыми от твоего личного присутствия. На твоем земном пути не должны сказаться никакие сверхчеловеческие влияния, кроме воли Райского Отца, если только ты не примешь сознательно волевого решения, которое завершится определенным выбором, исходящим из всей полноты твоей личности.
\usection{3. Дальнейшие советы и рекомендации}
\vs p120 3:1 «Теперь, брат мой, оставляя тебя, я дал тебе общее напутствие но прежде чем ты будешь готов покинуть нас для предстоящего воплощения, прими еще некоторые советы от нас с Гавриилом, касающиеся деталей твоей жизни на Урантии.
\vs p120 3:2 . Стремясь достигнуть идеала твоей земной жизни, уделяй также некоторое внимание тому, чтобы осуществить и показать пример в определенных практических вещах, непосредственно полезных для окружающих.
\vs p120 3:3 \pc . В семейных отношениях отдавай предпочтение общепринятым обычаям своего времени и среды. Веди семейную и общественную жизнь в согласии с традициями людей, которых ты избрал как свое окружение.
\vs p120 3:4 \pc . Что касается отношения к социальному устройству, то мы советуем тебе направить свои усилия преимущественно на духовное возрождение и интеллектуальную эмансипацию. Избегай любого вмешательства в экономическую структуру и политические обязательства. Посвяти себя прежде всего возвышенной религиозной жизни на Урантии.
\vs p120 3:5 \pc . Ни при каких обстоятельствах, и никоим образом тебе не следует вмешиваться в нормальную поступательную эволюцию рас на Урантии. Однако этот запрет никак не ограничивает тебя в деятельности, направленной на то, чтобы создать и оставить после себя стабильную усовершенствованную систему \bibemph{позитивной религиозной этики.} Как Сын, с пришествием которого происходит диспенсация, ты наделен определенными дополнительными возможностями, способствующими повышению \bibemph{духовного} и \bibemph{религиозного} уровня людей.
\vs p120 3:6 \pc . Если ты сочтешь это нужным, ты можешь идентифицироваться с какими\hyp{}то из существующих на Урантии религиозных и духовных движений, но любыми возможными путями избегай формального учреждения организованного культа, определенной религии, возникновения изолированной по этическому принципу группировки смертных существ. Твоя жизнь и твое учение должны стать общим достоянием всех религий и всех людей.
\vs p120 3:7 \pc . Советуем тебе, чтобы не способствовать возникновению новых религиозных догм или форм непрогрессивных религиозных верований, не оставлять после себя на планете никаких писаний. Не пиши на прочных материалах; воспрещай своим спутникам делать какие\hyp{}либо изображения твоего земного облика. Сделай так, чтобы к моменту твоего ухода на планете не осталось от тебя ничего, могущего стать предметом поклонения.
\vs p120 3:8 \pc . Живя нормальной, обычной жизнью нормального индивидуума мужского пола на Урантии, тебе, может быть, стоит воздержаться от вступления в супружеские отношения, хотя в них самих по себе нет ничего недостойного для тебя или несовместимого с условиями и задачами твоего пришествия. Однако я должен напомнить тебе, что в одном из установлений Спасограда запрещается воплощенному Райскому Сыну оставлять после себя человеческое потомство на планете.
\vs p120 3:9 \pc . Во всех остальных вопросах предстоящей человеческой жизни мы вверяем тебя руководству твоего Настройщика Мысли, водительству всегда пребывающего божественного духа, направляющего человека, и здравому смыслу собственного расширяющегося человеческого сознания, дар которого ты наследуешь. Такое соединение атрибутов творения и Творца позволит тебе прожить совершенную человеческую жизнь, которая, правда, не обязательно покажется совершенной любому человеку любого поколения на любой планете (и менее всего --- на Урантии); однако обитатели более развитых миров твоей вселенной увидят ее совершенство и верховную насыщенность.
\vs p120 3:10 Твой и мой Отец, всегда во всем поддерживавший нас доныне, да ведет он и поддерживает тебя и да будет с тобой с первого момента, когда ты покинешь нас и откажешься от своего личностного сознания, во все время постепенного возвращения к тебе сознания твоей божественной природы, воплощенной в человеческом облике, и затем всегда на всем твоем земном пути, до тех пор, пока ты не освободишься от плоти и не взойдешь к Нему, чтобы сесть по правую руку Отца. Когда я снова увижу тебя в Спасограде, мы будем приветствовать тебя как верховного и неограниченного властителя сотворенной тобой вселенной, в которой ты прошел свое служение и достиг полного познания.
\vs p120 3:11 Отныне от твоего имени правлю я. На срок твоего седьмого пришествия в образе смертного на Урантии я принимаю полномочия правителя Небадона. Тебе, Гавриил, я поручаю хранить безопасность будущего Сына Человеческого до тех пор, пока он в силе и славе не возвратится к нам как Сын Человеческий и Сын Божий. И ты будешь под началом моим, Гавриил, до возвращения Михаила».
\separatorline
\vs p120 3:12 И затем сразу же в присутствии всего Спасограда Михаил исчез от нас, и мы больше не видели его на обычном его месте вплоть до возвращения к нам высшим и полновластным владыкой вселенной, после завершения пришествия на Урантию.
\usection{4. Воплощение --- два становятся одним}
\vs p120 4:1 Итак, самоотверженное служение, которое Сыну Бога предстояло осуществить в воплощении Сына Человеческого, на все время подчинившего себя «воле Райского Отца», должно было навсегда опровергнуть, посрамить и лишить иллюзий недостойных чад Михаила, обвинивших его, своего Творца\hyp{}отца, в эгоистическом стремлении к господству и опустившихся до измышления, что Сын\hyp{}Творец удерживается у власти автократически и самодержавно благодаря бездумной преданности рабских созданий обманутой вселенной.
\vs p120 4:2 \pc Но не совершите ошибки: Христос\hyp{}Михаил, будучи воистину двойственного происхождения, не был двойственной личностью. Он не был Богом \bibemph{в соединении} с человеком, а был Богом, \bibemph{воплощенным} в человеке. Он был таким всегда, от начала до конца. В этом его непостижимом состоянии менялось только одно --- возрастающее понимание и принятие его человеческим разумом его собственной природы --- бытия Бога и человека.
\vs p120 4:3 Христос\hyp{}Михаил не становился постепенно Богом. Бог ни в какой момент земной жизни Иисуса не становился человеком. Иисус был человек \bibemph{и} Бог --- всегда и на веки вечные. Этот Бог и этот человек были и есть \bibemph{одно,} так же, как Райская Троица трех сущностей есть в действительности \bibemph{одно} Божество.
\vs p120 4:4 Никогда не забывайте, что высшей духовной задачей пришествия Михаила являлось расширенное и углубленное \bibemph{откровение Бога.}
\vs p120 4:5 \pc У смертных Урантии различное представление о чудесах; для нас же, граждан локальной вселенной, на свете существует совсем немного чудес, и среди них нет чуда таинственнее, чем воплощенные пришествия Райских Сынов. Когда в материальном мире, по\hyp{}видимому путем естественных процессов, появляется Божественный Сын --- это чудо, то есть действие универсальных законов, находящихся за пределами нашего понимания. Иисус из Назарета был воплощенным чудом.
\vs p120 4:6 В этом необычайном событии Бог Отец явил себя \bibemph{как обычно:} нормальным, естественным, надежным путем божественного действия.
