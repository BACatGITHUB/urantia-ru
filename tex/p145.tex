\upaper{145}{Четыре богатых событиями дня в Капернауме}
\author{Комиссия срединников}
\vs p145 0:1 Иисус и апостолы прибыли в Капернаум в четверг вечером 13 января. Как обычно, они обосновались в доме Зеведея в Вифсаиде. Теперь, когда Иоанна Крестителя уже не было в живых, Иисус готовился отправиться в первое путешествие по Галилее с открытыми и публичными проповедями. Весть о том, что Иисус вернулся, быстро разнеслась по всему городу, и рано утром на следующий день Мария, мать Иисуса, поспешила уйти в Назарет, навестить своего сына Иосифа.
\vs p145 0:2 Среду, четверг и пятницу Иисус провел в доме Зеведея, наставляя своих апостолов перед их первым продолжительным путешествием с публичными проповедями. Он также принимал и учил многих искренне интересующихся, как по одному, так и группами. Через Андрея он договорился о своем выступлении в синагоге в будущую субботу.
\vs p145 0:3 В пятницу поздно вечером Иисуса тайно посетила его сестренка Руфь. Они вместе почти час провели в лодке, бросив якорь на небольшом расстоянии от берега. Ни одно человеческое существо, кроме Иоанна Зеведеева, не знало об этой встрече; он же был предупрежден о том, чтобы никому ничего не рассказывать. Руфь была единственным членом семьи Иисуса, стойко и непоколебимо верившим в божественность его земной миссии с ранних времен пробуждения ее духовного сознания, в течение всего его богатого событиями служения, его смерти, воскресения и вознесения; и она в конце концов перешла в миры иные, ни разу не усомнившись в сверхъестественном характере миссии своего отца\hyp{}брата во плоти. Со стороны земного семейства Иисуса Руфь была его главным утешением в тяжком испытании во время суда над ним, отвержения и распятия.
\usection{1. Улов рыбы}
\vs p145 1:1 Утром в пятницу той же недели, когда Иисус учил у берега моря, народ стеснил его так близко к краю воды, что он знаком попросил рыбаков, находившихся в лодке неподалеку, прийти ему на помощь. Войдя в лодку, он продолжал учить собравшуюся толпу еще более двух часов. Эта лодка называлась «Симон»; раньше это была рыбачья лодка Симона Петра и была построена собственными руками Иисуса. В то самое утро лодкой пользовались Давид Зеведеев с двумя помощниками\hyp{}рыбаками, которые только что подплыли к берегу после безуспешного ночного лова на озере. Когда Иисус попросил их прийти ему на помощь, они промывали и чинили свои сети.
\vs p145 1:2 Закончив учить народ, Иисус сказал Давиду: «Раз ты потерял время, помогая мне, позволь теперь и мне с тобой потрудиться. Давай порыбачим; отплыви вон на то глубокое место и закинь сети свои для лова». Но Симон, один из помощников Давида, ответил: «Бесполезно, Учитель. Мы трудились всю ночь и ничего не поймали; впрочем, по твоему приказанию мы отплывем и закинем сети». И Симон по знаку своего хозяина Давида согласился следовать советам Иисуса. Подплыв к месту, показанному Иисусом, они закинули сети и поймали такое множество рыбы, что испугались, как бы не прорвались их сети, и стали звать своих товарищей на берегу, чтобы те пришли им на помощь. Когда же рыбой наполнили все три лодки, так что они погрузились почти до края бортов, этот Симон припал к коленам Иисуса и сказал: «Отойди от меня, Учитель, ибо я человек грешный». Симон и все, кто имел отношение к случившемуся, изумлялись улову рыбы. С того дня Давид Зеведеев, этот Симон и их товарищи оставили свои сети и следовали за Иисусом.
\vs p145 1:3 Однако в этом улове никоим образом не было ничего чудесного. Иисус внимательно изучал природу; он был опытным рыбаком и знал повадки рыбы в Галилейском море. На сей раз он просто направил тех людей к месту, где в это время дня обычно можно было найти рыбу. Однако последователи Иисуса всегда считали это чудом.
\usection{2. Днем в синагоге}
\vs p145 2:1 В следующую субботу во время дневной службы в синагоге Иисус произнес свою проповедь о «Воле Отца Небесного». Утром Симон Петр проповедовал о «Царстве». В четверг на вечернем собрании в синагоге учил Андрей; его темой был «Новый путь». Именно в это время в Капернауме в Иисуса уверовало больше людей, чем в любом другом городе на земле.
\vs p145 2:2 В этот субботний день, уча в синагоге, Иисус, согласно обычаю, взял первый текст из закона и прочел из Книги Исхода: «Служите Господу, Богу вашему, и он благословит хлеб ваш и воду вашу, и вся болезнь будет отвращена от тебя». Второй текст он выбрал из Пророков и прочел из Книги Исайи: «Восстань, светись, ибо пришел свет твой, и слава Господня взошла над тобою. Тьма может покрыть землю и мрак --- народ, но дух Господень восстанет над тобой и божественная слава явится с тобою. Даже неевреи придут к свету сему и многие умы великие склонятся пред сиянием света сего».
\vs p145 2:3 Эта проповедь была попыткой Иисуса объяснить, что религия --- \bibemph{это личный опыт.} Среди прочего Учитель сказал:
\vs p145 2:4 «Вы хорошо знаете, что хотя добросердечный отец любит свою семью как целое, он относится к ней как к группе из\hyp{}за своей сильной любви к каждому отдельному члену этой семьи. Вы больше не должны подходить к Отцу Небесному как чадо Израиля, но как \bibemph{дитя Бога.} Как группа вы, действительно, дети Израиля, но как индивидуумы каждый из вас --- дитя Божье. Я пришел не затем, чтобы открыть Отца детям Израиля, но для того, чтобы дать сие знание Бога и откровение о его любви и милосердии как подлинный личный опыт каждому отдельному верующему. Все пророки учили вас, что Яхве печется о своем народе, что Бог любит Израиль. Я же пришел к вам затем, чтобы провозгласить более великую истину, ту, которую понимали и многие из последних пророков, --- что Бог любит \bibemph{вас ---} каждого из вас --- как индивидуумов. У всех поколений ваших была национальная или расовая религия; теперь же пришел я, чтобы дать вам религию личную.
\vs p145 2:5 Впрочем, и эта идея не нова. Многие из духовно мыслящих среди вас знают сию истину, поскольку некоторые из пророков учили вас этому. Разве не читали вы в Писании, где пророк Иеремия говорит: „В те дни уже не будут говорить: Отцы ели кислый виноград, а у детей на зубах --- оскомина. Каждый будет умирать за свою собственную порочность; кто будет есть кислый виноград, у того на зубах и оскомина будет. Вот, наступают дни, когда я заключу новый завет с народом моим, не такой завет, какой я заключил с отцами их, когда вывел их из земли египетской, но по\hyp{}новому. На сердцах их напишу закон мой. Я буду им Богом, а они будут моим народом. В тот день не будет говорить человек ближнему своему: Знаешь ли ты Господа? Нет! Ибо они все узнают меня лично, от самого последнего до самого великого“.
\vs p145 2:6 Разве вы не читали эти обетования? Разве не верите вы Писанию? Неужели вы не понимаете, что слова пророка исполнились в том, что вы видите уже сегодня? Не призывал ли вас Иеремия сделать религию делом сердца, связать себя с Богом как отдельно взятые люди? Разве пророк не говорил вам, что Бог небес будет искать сердце каждого из вас отдельно? И не предупреждали ли вас, что лукаво естественное сердце человеческое и часто крайне грешное?
\vs p145 2:7 Разве не читали вы также в Писании, где Иезекииль учил отцов ваших, что религия должна стать сущностью вашего личного опыта? Не употребляйте более пословицу, которая гласит: „Отцы ели кислый виноград, а у детей на зубах --- оскомина“. Говорит Господь Бог: „Живу я --- вот, все души мои; как душа отца, так и душа сына. Лишь душа согрешающая умрет“. Уже тогда Иезекииль предвидел сегодняшний день, когда, говоря от имени Бога, сказал: „Дам вам сердце новое и дух новый вложу в вас“.
\vs p145 2:8 Больше не бойтесь, что Бог накажет нацию за грехи одного человека; не накажет Отец Небесный и одного из верующих детей своих за грехи нации, хотя отдельный член любой семьи часто должен терпеть материальные последствия ошибок семьи и коллективных проступков. Неужели вы не понимаете, что надежда на лучшую нацию --- или лучший мир --- связана с прогрессом и просвещением индивидуума?»
\vs p145 2:9 Затем Учитель поведал, что Отец Небесный хочет, чтобы его дети на земле после того, как познают сию духовную свободу, начали то вечное восхождение по Райскому пути, которое заключается в том, что создание сознательно откликается на божественный призыв пребывающего в нем духа --- найти Творца, узнать Бога и стараться уподобиться ему.
\vs p145 2:10 \pc Эта проповедь оказала апостолам великую помощь. Все они еще полнее осознали, что евангелие царства --- это весть отдельно взятому человеку, а не нации.
\vs p145 2:11 Хотя жители Капернаума были знакомы с учением Иисуса, их поразила его проповедь в этот субботний день. Он действительно учил как власть имеющий, а не как книжники.
\vs p145 2:12 \pc Как только Иисус кончил говорить, с находившимся среди собравшихся молодым человеком, который был сильно взволнован его словами, случился страшный эпилептический припадок и он громко закричал. В конце припадка, когда к нему возвращалось сознание, он в смутном состоянии проговорил: «Что тебе до нас Иисус Назарянин? Ты святой Божий; не погубить ли нас ты пришел?» Иисус велел народу молчать и, взяв молодого человека за руку, сказал: «Очнись», и тот немедленно пришел в сознание.
\vs p145 2:13 Этим молодым человеком отнюдь не овладел нечистый дух или демон; он был жертвой обыкновенной эпилепсии. Однако молодого человека учили, что его беда была следствием того, что им овладел злой дух. Он верил этому учению и поэтому все, что он думал и что говорил о своем недуге, соответствовало этой вере. Все люди верили, что подобные явления были непосредственно вызваны присутствием нечистых духов. Поэтому они думали, что Иисус изгнал демона из этого человека. Но Иисус в то время отнюдь не избавил его от эпилепсии. По\hyp{}настоящему этот человек был исцелен лишь позднее, после захода солнца в тот день. Долгое время спустя после дня Пятидесятницы Апостол Иоанн, последний из писавших о деяниях Иисуса, избегал вообще упоминать об этих так называемых актах «изгнания злых духов», поступая так потому, что случаев одержимости злым духом после Пятидесятницы никогда не было.
\vs p145 2:14 Вследствие этого заурядного случая по всему Капернауму быстро разнеслась молва о том, что Иисус изгнал демона из человека и чудом исцелил его в синагоге в конце своей дневной проповеди. Суббота была как раз тем временем, когда подобные поразительные слухи распространялись особенно быстро. Этот слух разнесли также по всем мелким селениям вокруг Капернаума, и многие из народа поверили ему.
\vs p145 2:15 \pc Приготовление пищи и домашнюю работу в большом доме Зеведея, где обосновались Иисус и двенадцать апостолов, большей частью вели жена Симона Петра и ее мать. Дом Петра был близко от дома Зеведея; и поскольку мать жены Петра уже несколько дней болела малярией, Иисус со своими друзьями зашел туда по пути из синагоги. Теперь же случилось так, что приблизительно в то самое время, когда Иисус стоял над этой больной женщиной, держа ее за руку, поглаживая ее лоб и говоря слова утешения и ободрения, лихорадка оставила ее. Иисус еще не успел объяснить своим апостолам, что в синагоге никакого чуда не было; и под действием этого события, столь свежо и живо запечатлевшегося в их памяти, а также под влиянием воспоминания о воде и вине в Кане они ухватились за это совпадение как за следующее чудо, и некоторые из них бросились разносить весть о нем по всему городу.
\vs p145 2:16 Аманта, теща Петра, страдала от малярийной лихорадки. В то время Иисус отнюдь не исцелил ее чудом. Ее исцеление произошло лишь несколько часов спустя после захода солнца в связи с необычайным событием, которое случилось на переднем дворе дома Зеведея.
\vs p145 2:17 \pc Эти случаи типичны в том смысле, что ищущее чудес поколение и народ, верящий в чудеса, неизменно хватались за все подобного рода совпадения как предлог для объявления о том, что Иисус совершил новое чудо.
\usection{3. Исцеление при заходе солнца}
\vs p145 3:1 Ко времени, когда Иисус и его апостолы незадолго до конца этого богатого событиями субботнего дня приготовились вкусить свою вечернюю пищу, весь Капернаум и его окрестности пришли в возбуждение из\hyp{}за этих якобы чудесных исцелений; и все, кто болел или страдал от недугов, стали готовиться идти к Иисусу либо к тому, чтобы их туда понесли их друзья, как только зайдет солнце. Согласно еврейскому учению, в священные часы субботы не допускалось путешествовать даже в поисках исцеления.
\vs p145 3:2 Поэтому, как только солнце скрылось за горизонтом, великое множество больных мужчин, женщин и детей отправилось к дому Зеведея в Вифсаиде. Один человек двинулся в путь со своей парализованной дочерью, как только солнце скрылось за домом его соседа.
\vs p145 3:3 События всего дня подготовили сцену для этого необычного действа при закате солнца. Ведь даже текст, использованный Иисусом во время его дневной проповеди, намекал на то, что болезнь должна быть изгнана; и он говорил столь беспрецедентно сильно и властно! Его послание было настолько убедительным! Он не взывал к человеческой власти, он обращался прямо к сознанию и душам людей. Не прибегая к логике, законническим софизмам или к хитроумным рассуждениям, он мощно, прямо и лично воззвал к сердцам своих слушателей.
\vs p145 3:4 \pc Эта суббота была великим днем в земной жизни Иисуса, да и в жизни вселенной. Для всей локальной вселенной небольшой еврейский город Капернаум фактически стал настоящей столицей Небадона. И горстка евреев в капернаумской синагоге не была единственными существами, слышавшими то важнейшее заключительное утверждение в проповеди Иисуса: «Ненависть --- это тень страха, а месть --- маска трусости». Не могли слушатели Иисуса забыть и его слова, гласившие: «Человек есть сын Бога, а не дитя дьявола».
\vs p145 3:5 \pc Вскоре после заката солнца, когда Иисус и апостолы все еще сидели за столом, жена Петра услышала голоса на переднем дворе и, подойдя к двери, увидела, что там собирается множество больных, а дорога из Капернаума заполнена толпой идущих получить исцеление от рук Иисуса. Увидев такое, она сразу пошла и сообщила об этом своему мужу, а тот --- Иисусу.
\vs p145 3:6 Когда Учитель вышел из главного входа дома Зеведея, перед его глазами предстала масса пораженных недугами и страждущих людей. Он смотрел почти на тысячу больных и недомогающих человеческих существ; по крайней мере таково было число собравшихся перед ним. Не все из присутствовавших были больны; некоторые пришли, помогая своим близким в этой попытке получить исцеление.
\vs p145 3:7 Вид этих пораженных недугами смертных, мужчин, женщин и детей, в огромной степени страдающих вследствие ошибок и проступков его собственных доверенных Сынов из вселенской администрации, особенно тронул человеческое сердце Иисуса и воззвал к божественному милосердию сего великодушного Сына\hyp{}Творца. Но Иисус хорошо знал, что он никогда не сможет построить прочное духовное движение на фундаменте чисто материальных чудес. Воздерживаться от проявления своих исключительных прав творца было его последовательной политикой. Со времен Каны сверхъестественное или чудесное не сопровождало его учение; и все же эта масса страждущих тронула его отзывчивое сердце и с великой силой воззвала к его сострадательной любви.
\vs p145 3:8 Голос с переднего двора воскликнул: «Учитель, скажи слово, верни наше здоровье, исцели наши болезни и спаси наши души». Не успели прозвучать эти слова, как огромная свита серафимов, физических контролеров, Носителей Жизни и срединников, которая всегда сопровождала сего воплотившегося Творца вселенной, приготовилась действовать с творческой силой при первом же сигнале своего Владыки. Это был один из тех моментов земного пути Иисуса, когда божественная мудрость и человеческое сострадание в суждении Сына Человеческого соединились настолько, что он искал спасения, взывая к воле своего Отца.
\vs p145 3:9 Когда Петр стал умолять Учителя обратить внимание на сей крик о помощи, Иисус, посмотрев на толпу страждущих, ответил: «Я пришел в мир открывать Отца и устанавливать его царство. Ради этой цели я жил своей жизнью до этого часа. Поэтому, если такова будет воля Пославшего меня и если это не противоречит моему посвящению делу провозглашения евангелия царства небесного, я желал бы видеть моих детей здоровыми и\ldots », --- но следующие слова Иисуса потонули в шуме и криках.
\vs p145 3:10 Иисус передал ответственность за это решение об исцелении на усмотрение своего Отца. Очевидно, воля Отца не была против, ибо едва успели прозвучать слова Учителя, как сонмы небесных личностей, служивших под командованием Персонализированного Настройщика Мыслей Иисуса, пришли в сильное движение. Великая свита спустилась в гущу сей пестрой толпы больных смертных и в одно мгновение 683 мужчины, женщины и ребенка выздоровели и совершенно исцелились от своих телесных недугов и других физических расстройств. Ни до этого дня, ни после него на земле никогда не видели ничего подобного. Для тех же из нас, кто присутствовал, видя эту творческую волну исцеления, --- это было действительно потрясающее зрелище.
\vs p145 3:11 \pc Но среди всех существ, изумленных этим внезапным и неожиданным всплеском сверхъестественного исцеления, больше всех удивлен был Иисус. В момент, когда его человеческий интерес и сочувствие сосредоточились на развернувшейся перед ним картине страдания и боли, он упустил из виду, что в своей человеческой памяти должен держать предостерегающее предупреждение своего Персонализированного Настройщика относительно невозможности ограничить элемент времени в сфере исключительных творческих прав Сына\hyp{}Творца при определенных условиях и в определенных обстоятельствах. Иисус желал видеть этих страждущих смертных здоровыми, если это не нарушит волю его Отца. Персонализированный Настройщик Иисуса сразу же постановил, что подобный акт творческой энергии в данное время не преступит воли Райского Отца и этим решением --- ввиду предшествующего этому решению изъявления Иисусом желания дать исцеление --- творческий акт \bibemph{был осуществлен.} То, чего желает \bibemph{Сын\hyp{}Творец,} и то, что Отец \bibemph{велит, ---} СОВЕРШАЕТСЯ. За всю последующую земную жизнь Иисуса не было другого такого же массового исцеления.
\vs p145 3:12 \pc Как и следовало ожидать, молва об этом исцелении при заходе солнца в Вифсаиде близ Капернаума разнеслась по всей Галилее, Иудее и далеко за их пределами. Опасения Ирода снова пробудились, и он послал наблюдателей, чтобы те доносили о деяниях и учениях Иисуса и выяснили, кто он --- бывший плотник из Назарета или воскресший из мертвых Иоанн Креститель.
\vs p145 3:13 Главным образом из\hyp{}за этой непреднамеренной демонстрации исцеления телесных недугов впредь на всем своем оставшемся земном пути Иисус стал врачом в той же мере, что и проповедником. Правда, он продолжал учить, но его личный труд в основном заключался в служении больным и бедствующим, тогда как делом его апостолов стали публичные проповеди и крещение верующих.
\vs p145 3:14 Но большинство из тех, кто получил сверхъестественное или творческое телесное исцеление во время этого действия божественной энергии при закате солнца, не извлекли постоянной духовной пользы из сего необычайного проявления милосердия. Немногие получили истинное наставление благодаря этому служению телу, однако это поразительное извержение вневременного творческого исцеления не продвинуло духовное царство в сердцах людей.
\vs p145 3:15 Чудеса исцеления, время от времени сопровождавшие миссию Иисуса на земле, не были частью его замысла провозглашения царства. Они были случайностями, присущими пребыванию на земле божественного существа, наделенного почти неограниченными творческими правами в сочетании с беспрецедентным соединением божественного милосердия и человеческого сочувствия. Однако так называемые чудеса доставляли Иисусу немало хлопот, ведь они приводили к совершенно ненужной публичности, и создавали излишнюю шумиху.
\usection{4. Тем же вечером}
\vs p145 4:1 Весь вечер после этого великого потока исцелений радостная и счастливая толпа наводняла дом Зеведея, и апостолы Иисуса настроились на высочайший лад эмоционального восторга. С человеческой точки зрения, это был, вероятно, величайший из всех великих дней их общения с Иисусом. Никогда прежде и никогда потом их надежды не возносились до подобных высот уверенного ожидания. О том, что пробил час, когда царство должно было быть провозглашено в \bibemph{силе,} Иисус сказал им всего несколько дней назад, когда они были еще в пределах Самарии, и вот глаза их видели то, что казалось им исполнением этого обещания. Видение того, чему предстояло произойти, если это удивительное проявление целительной силы было всего лишь началом, приводило их в состояние восторга. Остававшиеся у них сомнения относительно божественности Иисуса были изгнаны. Они буквально потеряли голову попав под влияние сводящего с ума очарования.
\vs p145 4:2 Однако, когда они стали искать Иисуса, то найти его не смогли. Учитель пребывал в сильном смятении от происшедшего. Эти исцеленные от разных болезней мужчины, женщины и дети не уходили до позднего вечера, надеясь дождаться возвращения Иисуса, чтобы поблагодарить его. Проходили часы, а Иисус оставался в уединении, и апостолы не могли понять поведение Учителя; если бы не это продолжавшееся отсутствие, их радость была бы полной и совершенной. Когда же Иисус вернулся к ним, было уже поздно, и практически все исцеленные разошлись по домам. Иисус отказался от поздравлений и поклонения двенадцати апостолов и других, оставшихся приветствовать его, сказав лишь: «Не тому радуйтесь, что Отец мой властен исцелять тело, а тому, что он обладает могуществом спасать душу. Пойдем спать, ибо завтра нам должно быть в том, что принадлежит Отцу».
\vs p145 4:3 И снова двенадцать разочарованных, недоумевающих и опечаленных сердцем людей пошли спать; никто из них, кроме близнецов, не спал хорошо этой ночью. Не успел Учитель сделать нечто, ободряющее души и радующее сердца своих апостолов, как тут же, казалось, разрушает их надежды и полностью уничтожает основы их смелости и энтузиазма. Эти сбитые с толку рыбаки смотрели друг другу в глаза, но в них была лишь одна мысль: «Мы не понимаем его. Что все это значит?»
\usection{5. Рано утром в воскресенье}
\vs p145 5:1 Почти не спал в ту ночь и Иисус. Он понимал, что мир полон физических страданий и наводнен материальными трудностями, и размышлял о великой опасности быть вынужденным посвящать заботе о больных и страждущих столь много времени, что его служение телесному будет препятствовать его миссии установления духовного царства в сердцах людей или, по крайней мере, подчинит ее себе. Из\hyp{}за этих и других похожих мыслей, занимавших смертный ум Иисуса всю ночь, он поднялся в то воскресное утро задолго до рассвета и отправился в одиночестве к одному из своих излюбленных мест для общения с Отцом. В это раннее утро Иисус молился о мудрости и рассудительности, которые не позволили бы его человеческому сочувствию, слившемуся с его божественным милосердием, перед лицом страдания смертных взывать к нему настолько, что все его время было бы занято служением телесному в ущерб духовному. Хотя Иисус не хотел совсем отказаться от служения больным, он, тем не менее, знал, что ему надлежит также заниматься более важным трудом духовного учения и религиозного воспитания.
\enlargethispage{\baselineskip}%
\vs p145 5:2 Иисус так часто уходил молиться в горы потому, что не было уединенных помещений для его личных молитв.
\vs p145 5:3 В ту ночь Петр не мог уснуть; поэтому очень рано, вскоре после того, как Иисус ушел молиться, разбудил Иакова и Иоанна, и они втроем пошли искать Учителя. Проведя в поисках более часа, они нашли Иисуса и попросили его поведать им причину своего странного поведения. Они желали узнать, почему он, казалось, был обеспокоен мощным излиянием исцеляющего духа, в то время как все люди были вне себя от радости и его апостолы так ликовали.
\vs p145 5:4 Иисус более четырех часов старался объяснить этим трем апостолам, что же произошло. Он учил их о том, что случилось, и объяснял опасности подобных проявлений. Иисус поведал им причину, по которой он пришел молиться. Он попытался разъяснить своим личным сподвижникам подлинные причины, почему царство нельзя было построить на совершении чудес и исцелении тела. Но они не смогли понять его учение.
\vs p145 5:5 Тем временем ранним воскресным утром около дома Зеведея начали собираться новые толпы страждущих душ и любопытных. Они шумно требовали встречи с Иисусом. Андрей и другие апостолы были в таком замешательстве, что пока Симон Зилот говорил с собравшимися, Андрей с несколькими своими товарищами пошел искать Иисуса. Увидев Иисуса с тремя апостолами, он сказал: «Учитель, почему ты оставил нас с толпой одних? Смотри, все люди ищут тебя; никогда прежде столько людей не стремились к твоему учению. Вот и сейчас дом окружен теми, кто пришел из ближних и дальних мест благодаря твоим великим делам. Не вернешься ли ты с нами, чтобы служить им?»
\vs p145 5:6 Услышав это, Иисус ответил: «Разве не учил я тебя, Андрей, и сих остальных, что моя миссия на земле --- открывать Отца и что мое послание --- провозглашение царства небесного? Почему же тогда, ты хотел бы, чтобы я отступил от своего дела ради ублажения любопытных и потворства тем, кто ищет чудес и знамений? Разве не были мы среди людей сих все эти месяцы, и разве собирались они в толпы, чтобы услышать благую весть царства? Зачем же теперь они осаждают нас? Не более ли из\hyp{}за исцеления их физических тел, чем в результате принятия духовной истины ради спасения душ их? Когда людей привлекают к нам необычные проявления, то многие из них приходят, ища отнюдь не истины и спасения, а исцеления своих телесных недугов и избавления от материальных трудностей.
\vs p145 5:7 Все это время я был в Капернауме и в синагогах и у берега моря возвещал евангелие царства всем имевшим уши, чтобы слышать, и сердца, чтобы принять истину. Воля Отца моего отнюдь не в том, чтобы я вернулся с вами угождать сим любопытным и занялся служением телесному в ущерб духовному. Я посвятил вас проповедовать евангелие и служить больным, но я не могу посвятить себя исцелениям в ущерб моему учению. Нет, Андрей, я с вами не вернусь. Иди и скажи людям, чтобы они верили в то, чему я учил их, и радовались свободе сыновей Бога, а сами приготовьтесь отправиться в другие города Галилеи, где путь для проповеди благой вести царства уже готов. Ради этой цели я и пришел от Отца. Посему иди, приготовься к нашему немедленному уходу, а я подожду вас здесь».
\vs p145 5:8 Когда Иисус кончил говорить, Андрей и его собратья\hyp{}апостолы печально побрели к дому Зеведея, распустили собравшуюся толпу и быстро приготовились к путешествию, как велел им Иисус. Итак, в воскресенье после полудня 18 января 28 года н.э. Иисус и апостолы отправились в свое первое путешествие с по\hyp{}настоящему публичными и открытыми проповедями по городам Галилеи. Во время этого первого путешествия они проповедовали евангелие царства во многих городах, но Назарет не посетили.
\vs p145 5:9 В то же воскресенье после полудня, вскоре после того, как Иисус и его апостолы ушли в Риммон, повидаться с ним в дом Зеведея зашли его братья Иаков и Иуда. Около полудня того дня Иуда разыскал своего брата Иакова и настоял на том, чтобы они пошли к Иисусу. Однако ко времени, когда Иаков согласился идти с Иудой, Иисус уже ушел.
\vs p145 5:10 Апостолы очень не хотели уходить от того великого внимания к ним, которое возникло в Капернауме. Петр подсчитал, что в царство можно было окрестить не менее тысячи верующих. Иисус их терпеливо выслушал, но вернуться не согласился. Какое\hyp{}то время все молчали, но затем Фома обратился к своим собратьям\hyp{}апостолам и сказал: «Пойдем! Учитель сказал. Неважно, что мы не можем до конца понять тайны царства небесного, зато мы точно знаем одно: мы идем за учителем, который не ищет славы для себя самого». И они скрепя сердце пошли проповедовать благую весть в городах Галилеи.
