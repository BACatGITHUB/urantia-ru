\upaper{193}{Последние явления и вознесение}
\author{Комиссия срединников}
\vs p193 0:1 Шестнадцатое моронтийное явление Иисуса произошло в пятницу 5 мая около девяти часов вечера на дворе дома Никодима. В этот вечер иерусалимские верующие впервые после воскресения попытались собраться вместе. Среди собравшихся были: одиннадцать апостолов, отряд женщин, их сподвижники и примерно пятьдесят других наиболее выдающихся учеников Учителя, среди них и несколько греков. Эти верующие уже более получаса непринужденно общались, когда неожиданно явился Учитель в моронтийном состоянии, видимый всем, и тотчас стал наставлять их. Иисус сказал:
\vs p193 0:2 \pc «Мир вам. Это наиболее представительная группа верующих --- апостолов и учеников, как мужчин, так и женщин, --- которой я являлся с момента моего освобождения от плоти. Ныне я призываю вас свидетельствовать, что я заранее предупреждал вас, что мое пребывание среди вас должно подойти к концу; я говорил вам, что вскоре должен буду вернуться к Отцу. А затем ясно сказал вам, как первосвященники и правители евреев предадут меня на смерть, и что я восстану из могилы. Почему же тогда, позвольте вас спросить, вы столь смущены этими событиями, когда они произошли? И почему так изумились, когда на третий день я восстал из гробницы? Вы не смогли поверить мне, потому что слушали мои слова, не понимая их смысла.
\vs p193 0:3 Теперь же внимайте моим словам, чтобы не совершить снова ошибку, которая заключается в том, что мое учение вы слушаете умом, а сердцем не понимаете его смысла. С начала моего пребывания в качестве одного из вас я учил, что моя единственная цель --- открыть Отца Небесного его детям на земле. Я прожил открывающее Бога пришествие, чтобы путь познания Бога могли пройти и вы. Я открыл Бога как вашего небесного Отца; я открыл вас как сынов Бога на земле. То, что Бог любит вас, своих сынов, является фактом. Благодаря вере в мое слово, этот факт становится вечной и живой истиной в ваших сердцах. Когда же, благодаря живой вере, вы становитесь божественно сознающими Бога, тогда и рождаетесь от духа как дети света и жизни и даже как дети вечной жизни, в которой вы вознесетесь во вселенную вселенных и познаете опыт обретения Отца в Раю.
\vs p193 0:4 Я призываю вас всегда помнить, что ваша миссия среди людей состоит в том, чтобы провозглашать евангелие царства --- реальность отцовства Бога и истину о сыновстве человека. Возвещайте всю истину благой вести, а не только часть спасительного евангелия. Опыт моего воскресения отнюдь не меняет вашего послания. Сыновство по отношению к Богу, которое дает вера, --- по\hyp{}прежнему спасительная истина евангелия царства. Вы должны идти, проповедуя любовь Бога и служение человека. Вот то, что более всего нужно знать миру: люди есть сыновья Бога и через веру могут действительно осознать и ежедневно испытывать сию возвышающую истину. Мое пришествие должно помочь всем людям узнать, что они --- дети Бога, однако подобного знания будет недостаточно, если они лично не сумеют верой осознать спасительную истину, которая заключается в том, что они --- подлинные духовные сыны вечного Отца. Главное в евангелии царства --- любовь Отца и служение его детей на земле.
\vs p193 0:5 Здесь между собой вы делитесь сведениями о том, что я воскрес из мертвых, но это неудивительно. Я имею власть отдать свою жизнь и взять ее снова; Отец дает такую власть Райским Сынам. Вы же в сердцах ваших должны быть взволнованы знанием о том, что вскоре после того, как я покинул новую гробницу Иосифа, умершие вступили на путь вечного восхождения. Мою жизнь во плоти я прожил, чтобы показать, как вы через служение любви можете стать открывающими Бога для своих собратьев\hyp{}людей, как и я, любя вас и служа вам, стал открывающим Бога для вас. Я жил среди вас как Сын Человеческий, чтобы вы и все остальные люди могли узнать, что все вы на самом деле сыны Бога. Поэтому теперь идите по всему миру и всем людям проповедуйте это евангелие царства небесного. Любите всех людей, как я любил вас; служите своим собратьям, как я вам служил. Даром получили, даром отдавайте. Оставайтесь же здесь в Иерусалиме, лишь пока я иду к Отцу и пока не пошлю вам Духа Истины. Он наставит вас, расширяя ваше понимание истины; я же пойду с вами по всему миру. Я с вами всегда; мир мой оставляю вам».
\vs p193 0:6 \pc Кончив говорить с ними, Учитель стал для них невидим. Эти верующие разошлись только на рассвете; всю ночь они провели вместе, горячо обсуждая сказанное Учителем и размышляя обо всем, что случилось с ними. Иаков Зеведеев и другие апостолы тоже рассказали им о своем опыте общения с Учителем в моронтийном состоянии в Галилее и поведали, как он трижды являлся им.
\usection{1. Явление в Сихаре}
\vs p193 1:1 3 мая в субботу около четырех часов после полудня недалеко от колодезя Иаковлева в Сихаре Учитель явился Нальде и почти семидесяти пяти верующим самарянам. Верующие привыкли собираться в этом месте, рядом с которым Иисус говорил Нальде о воде живой. В этот день, только они закончили обсуждения известий о воскресении, им неожиданно явился Иисус, говоря:
\vs p193 1:2 \pc «Мир вам. Вы радуетесь, узнав, что я есть воскресение и жизнь, но это вам ничего не даст, если вы не родитесь сначала от вечного духа и тем самым через веру не станете обладателями дара вечной жизни. Если вы --- верующие сыновья моего Отца, то никогда не умрете и не погибнете. Евангелие царства учило вас, что все люди --- сыновья Бога. И эта благая весть о любви Отца Небесного к своим детям на земле должна разнестись по всему миру. Пришло время, когда вы будете поклоняться Богу не на горе Гаризим и не в Иерусалиме, но там, где вы есть, и такими, какие вы есть, в духе и в истине. Именно вера ваша спасает ваши души. Спасение --- это дар Бога всем, кто верит, что они --- его сыны. Однако не впадайте в заблуждение; хотя спасение есть бескорыстный дар Бога и даруется всем, кто принимает его верой, за ним следует опыт приношения плодов сей духовной жизни, которой живут во плоти. Принятие учения об отцовстве Бога предполагает, что вы тоже бескорыстно примете связанную с ним истину о братстве людей. Если же человек --- ваш брат, значит, он больше, нежели ваш ближний, которого Отец требует от вас любить, как самих себя. Вашего брата, который является членом вашей семьи, вы будете любить не только родственной любовью, но и будете служить ему, как служили бы самим себе. Будете же так любить вашего брата и так служить ему потому, что и вас, являющихся моими братьями, я так любил и так служил вам. Идите в мир и рассказывайте эту благую весть всем созданиям каждой расы, каждого племени и каждого народа. Дух же мой пойдет перед вами, и я буду с вами всегда».
\vs p193 1:3 \pc Явление Учителя крайне изумило этих самарян, и они поспешили отправиться в близлежащие города и селения, где разнесли весть о том, что видели Иисуса и что он говорил с ними. Было же это семнадцатым моронтийным явлением Учителя.
\usection{2. Явление в Финикии}
\vs p193 2:1 Восемнадцатое моронтийное явление Учителя было в Тире во вторник 16 мая чуть раньше девяти часов вечера. Иисус снова явился, когда собрание верующих закрывалось те уже собирались расходиться, и сказал:
\vs p193 2:2 \pc «Мир вам. Вы радуетесь, узнав, что Сын Человеческий воскрес из мертвых, потому что, таким образом, знаете, что и вы и братья ваши тоже переживут смерть, которую претерпевают смертные. Однако такое преодоление смерти зависит от того, родились ли вы от духа поиска истины и отыскания Бога. Хлеб жизни и вода живая даются лишь тем, кто алчет истины и жаждет праведности --- алчет и жаждет Бога. То, что мертвые воскресают, --- отнюдь не евангелие царства. Эти великие истины и вселенские факты все связаны с евангелием, ибо являются частичным следствием веры в благую весть, и заключаются в последующем опыте тех, кто благодаря вере стали на деле и в истине вечными сынами вечного Бога. Отец мой послал меня в мир, дабы всем людям возвестить это спасение сыновства. Поэтому и я посылаю вас в мир проповедовать сие спасение сыновства. Спасение есть бескорыстный дар Бога, однако те, кто рожден от духа, сразу начнут приносить плоды духа в полном любви служении таким же, как они, созданиям. Плоды же божественного духа, приносимые в жизнях рожденных от духа и знающих Бога смертных, таковы: любовное служение; бескорыстная преданность; доблестная верность; искренняя справедливость; просвещенная честность; неумирающая надежда; полное доверия упование; милосердное служение; неизменная доброта; прощающая терпимость и прочный мир. Если же те, кто называет себя верующим, не приносят в своих жизнях этих плодов божественного духа, значит, они мертвы; Духа Истины в них нет; они --- бесполезные ветви на живой лозе виноградной и вскоре будут удалены. От детей веры мой Отец требует, чтобы они приносили много духовного плода. Поэтому, если вы не плодоносны, он окопает ваши корни и отсечет ваши бесплодные ветви. Восходя к небу в царстве Бога, вы должны все больше и больше приносить плодов духа. В царство вы можете войти, как дети, но Отец требует, чтобы вы вырастали в благодати до полного достижения духовной зрелости. Когда же вы пойдете в мир, чтобы всем народам сообщить благую весть сего евангелия, я пойду перед вами, и мой Дух Истины пребудет в ваших сердцах. Мир мой оставляю вам».
\vs p193 2:3 \pc Затем Учитель стал невидим. На следующий день те, кто нес эту весть, пошли из Тира в Сидон и даже в Антиохию и Дамаск. Иисус был с этими верующими, когда жил во плоти, и они быстро узнавали его, когда он начинал их учить. Хотя его друзья не всегда распознавали его в моронтийном облике, когда он делался видимым, они тем не менее всегда без труда узнавали его личность, когда он обращался к ним.
\usection{3. Последнее явление в Иерусалиме}
\vs p193 3:1 Рано утром в четверг 18 мая Иисус совершил свое последнее явление на земле как моронтийная личность. Когда одиннадцать апостолов собирались садиться завтракать в комнате наверху в доме Марии Марк, Иисус явился им и сказал:
\vs p193 3:2 \pc «Мир вам. Я просил вас оставаться здесь в Иерусалиме до времени, когда я вознесусь к Отцу, и даже до времени, когда я пошлю вам Дух Истины, который вскоре изольется на всякую плоть и дарует вам силу с неба». Симон Зилот перебил Иисуса, спросив: «Значит, Учитель, ты восстановишь царство, и мы увидим славу Бога, явленную на земле?» Выслушав вопрос Симона, Иисус ответил: «Симон, ты продолжаешь придерживаться своих прежних представлений о еврейском Мессии и материальном царстве. Но, когда дух снизойдет на вас, ты примешь духовную силу и вскоре пойдешь, проповедуя всему миру евангелие царства. Как Отец послал меня в мир, так и я вас посылаю. И хочу, чтобы вы друг друга любили и доверяли друг другу. Иуды с вами больше нет, потому что его любовь охладела и потому что он отказался доверять вам, своим верным братьям. Разве не читали вы в Писании, где написано: „Не хорошо быть человеку одному. Никто не живет для себя?“ А также где сказано: „Кто хочет иметь друзей, тот и сам должен быть дружелюбным“? И разве не посылал я вас учить по двое, чтобы не было вам одиноко и вы не испытали бы вред и страдания уединения? Вы также хорошо знаете, что, когда я был во плоти, то не позволял себе долго оставаться одному. С самого начала нашего общения двое или трое из вас всегда постоянно были рядом со мной либо совсем недалеко, даже когда я общался с Отцом. Поэтому верьте и доверяйте друг другу. Это же тем более необходимо, поскольку сегодня я собираюсь оставить вас в мире одних. Час настал; я собираюсь идти к Отцу».
\vs p193 3:3 \pc Кончив говорить, Иисус подал им знак идти вместе с ним и повел их на Масличную гору, где и простился с ними, готовясь покинуть Урантию. Путешествие к Масличной горе было торжественным. Со времени, когда они покинули комнату наверху, до времени, когда Иисус остановился с ними на Масличной горе, никто из них не проронил ни слова.
\usection{4. Причины падения Иуды}
\vs p193 4:1 В начале прощального послания своим апостолам Учитель упомянул об утрате Иуды и указал, что трагическая судьба их совершившего предательство сотоварища есть серьезное предостережение об опасности обособления в обществе и в братстве. Для верующих этой и грядущих эпох, возможно, будет полезно коротко рассмотреть причины падения Иуды в свете замечаний Учителя и с учетом знаний, накопившихся в последующие века.
\vs p193 4:2 Вспоминая об этой трагедии, мы понимаем, что Иуда сбился с пути в первую очередь потому, что был ярко выраженной замкнутой личностью, личностью, ушедшей в себя и отгородившейся от обычных общественных связей. Он упорно отказывался доверять своим собратьям\hyp{}апостолам или свободно общаться с ними. Однако то, что он был личностью замкнутого типа, само по себе не причинило бы Иуде такого вреда, если бы не еще одно обстоятельство: Иуда не сумел возрасти в любви и вырасти в духовной благодати. А кроме того, словно затем, чтобы еще более ухудшить положение, упорно таил злобу и пестовал таких психологических врагов, как месть и вообще стремление с кем\hyp{}нибудь «поквитаться» за все свои разочарования.
\vs p193 4:3 Это неудачное сочетание индивидуальных особенностей и умственных наклонностей вкупе погубили благонамеренного человека, который не сумел преодолеть эти пороки любовью, верой и доверием. То, что Иуда совсем не обязательно должен был сбиться с пути, доказано примерами Фомы и Нафанаила, которые оба страдали от того же рода подозрительности и чрезмерно развитых индивидуалистических наклонностей. Даже Андрей и Матфей, и те имели множество склонностей такого же свойства; однако все эти люди со временем стали любить Иисуса и своих собратьев\hyp{}апостолов не меньше, а больше. Они возрастали в благодати и в познании истины. Они все больше и больше полагались на своих братьев и постепенно развивали в себе способность доверять своим друзьям. Иуда же доверять своим братьям упорно отказывался. Когда скопище противоречивых чувств побуждало его искать отдушину в самовыражении, он неизменно обращался за советом и принимал неразумные утешения своих недуховных родственников или тех случайных знакомых, которые были либо безразличны, либо по\hyp{}настоящему враждебны к благополучию и совершенствованию духовных реальностей царства небесного, одним из двенадцати посвященных посланников на земле которого он был.
\vs p193 4:4 Иуда потерпел поражение в своей земной борьбе из\hyp{}за следующих личных наклонностей и слабостей характера.
\vs p193 4:5 \ublistelem{1.}\bibnobreakspace Он был человеком замкнутого типа. Был крайне индивидуалистичен и решил стать личностью сугубо замкнутой и необщительной.
\vs p193 4:6 \ublistelem{2.}\bibnobreakspace В детстве жизнь его была излишне беззаботной. Он терпеть не мог, когда ему возражали. Всегда рассчитывал на победу и не умел проигрывать.
\vs p193 4:7 \ublistelem{3.}\bibnobreakspace Относиться к разочарованию по\hyp{}философски он так и не научился. Вместо того, чтобы принять разочарование как обычную и заурядную особенность человеческого бытия, он неизменно прибегал к обвинению во всех своих личных трудностях и разочарованиях кого\hyp{}нибудь одного либо всех своих товарищей вместе взятых.
\vs p193 4:8 \ublistelem{4.}\bibnobreakspace Он был склонен таить обиду и всегда наслаждался мыслью об отмщении.
\vs p193 4:9 \ublistelem{5.}\bibnobreakspace Он не любил смотреть фактам в лицо и в своем отношении к жизненным ситуациям был нечестен.
\vs p193 4:10 \ublistelem{6.}\bibnobreakspace Он не любил обсуждать свои личные проблемы с ближайшими товарищами и отказывался говорить о своих трудностях со своими настоящими друзьями и теми, кто его истинно любил. За все годы общения с Учителем он ни разу не пришел к нему с чисто личной проблемой.
\vs p193 4:11 \ublistelem{7.}\bibnobreakspace Он так и не понял, что подлинным воздаянием за возвышенную жизнь в конце концов являются духовные награды, которые не всегда даются во время этой короткой жизни во плоти.
\vs p193 4:12 \pc Вследствие постоянной личной обособленности Иуды его беды множились, печали возрастали, тревоги усиливались, а отчаяние углублялось, становясь почти нестерпимым.
\vs p193 4:13 Хотя у этого эгоцентричного сверхиндивидуалистичного апостола было множество душевных, эмоциональных и духовных проблем, его основной недостаток заключался в том, что как личность он был необщителен. Обладал подозрительным и мстительным умом. По темпераменту --- был угрюм и злопамятен. В чувствах --- лишен любви и не умел прощать. В общественном плане был недоверчив и почти абсолютно замкнут. В духе --- стал высокомерен и исполнен эгоистических амбиций. В жизни --- пренебрегал теми, кто его любил, а в смерти --- был лишен друзей.
\vs p193 4:14 Итак, эти психические особенности в совокупности с влиянием зла объясняют, почему благонамеренный и в иных отношениях некогда искренне верующий в Иисуса человек даже после нескольких лет близкого общения с его преобразующей личностью бросил своих собратьев, отрекся от священного дела, отказался от своего святого призвания и предал своего божественного Учителя.
\usection{5. Вознесение Учителя}
\vs p193 5:1 Утром в четверг 18 мая почти в половине восьмого часа Иисус со своими одиннадцатью молчаливыми и несколько растерянными апостолами достиг западного склона Масличной горы. С этого места, расположенного приблизительно в двух третях пути от подножия к вершины горы, они могли видеть весь Иерусалим, и Гефсиманию. Теперь, перед тем как покинуть Урантию, Иисус приготовился сказать свое последнее прощальное слово апостолам. Когда же он стоял перед ними, они без всякого на то указания встали вокруг него на колени, и Учитель сказал:
\vs p193 5:2 \pc «Я повелел вам дожидаться в Иерусалиме, когда вам будет дарована сила с неба. Ныне я собираюсь покинуть вас; я готовлюсь вознестись к Отцу моему, и скоро, очень скоро мы пошлем в этот мир где я какое\hyp{}то время обретался Дух Истины; когда же он придет, вы начнете новое провозглашение евангелия царства, сначала в Иерусалиме, а потом и в самых отдаленных уголках мира. Любите людей любовью, которой я вас любил, и служите своим смертным собратьям, как служил вам я. Духовными плодами жизней ваших побуждайте души верить в истину, что человек есть сын Бога и что все люди --- братья. Помните все, чему я учил вас, и жизнь, которую я среди вас прожил. Любовь моя укрывает вас, дух мой снизойдет на вас, и мир мой всегда будет с вами. Прощайте».
\vs p193 5:3 \pc Сказав это, Учитель в моронтийном состоянии стал для них невидимым. Это же так называемое вознесение ничем не отличалось от других случаев, когда он становился невидим для глаз смертных в течение сорока дней его моронтийной жизни на Урантии.
\vs p193 5:4 Учитель отправился в Эдентию через Иерусем, где Всевышние под наблюдением Райского Сына освободили Иисуса из Назарета от моронтийного состояния и по духовным каналам вознесения вернули его к статусу Райского сына и верховного владыки на Спасограде.
\vs p193 5:5 Этим утром в семь часов сорок пять минут Иисус в моронтийном состоянии стал невидим для своих одиннадцати апостолов, чтобы начать вознесение одесную своего Отца и там принять официальное подтверждение полноты своего владычества во вселенной Небадон.
\usection{6. Петр созывает собрание}
\vs p193 6:1 По указанию Петра, Иоанн Марк и другие пошли собирать лучших учеников в доме Марии Марк. К десяти тридцати сто двадцать самых лучших учеников Иисуса, живших в Иерусалиме, собрались, чтобы услышать рассказ о прощальном послании Учителя и узнать о его вознесении. Среди этих людей была и Мария, мать Иисуса. Она пришла назад в Иерусалим вместе с Иоанном Зеведеевым, когда апостолы возвратились после своего последнего пребывания в Галилее. Вскоре после Пятидесятницы Мария вернулась в Вифсаиду в дом Саломеи. Иаков, брат Иисуса, также присутствовал на этом собрании, первом совещании учеников Учителя, созванном после окончания его планетарной жизни.
\vs p193 6:2 Симон Петр решил выступить от имени своих собратьев\hyp{}апостолов и взволнованно поведал о последней встрече одиннадцати со своим Учителем и удивительно трогательно живописал, прощальные слова Учителя и его вознесение. То было собрание, подобного которому в этом мире еще не случалось никогда. Эта часть собрания продолжалась чуть менее часа. Затем Петр объяснил, что надо решить кому передать обязанности Иуды Искариота, и что в собрании объявляется перерыв, чтобы апостолы могли сделать выбор между Матиасом и Иустом, выдвинутыми на эту должность.
\vs p193 6:3 Затем одиннадцать апостолов спустились вниз по лестнице, и уже там решили просто бросить жребий, дабы определить, кто из этих людей станет апостолом и будет служить вместо Иуды. Жребий выпал Матиасу, и его объявили новым апостолом. Он был надлежащим образом введен в свою должность, а затем назначен казначеем. Однако в последующих делах апостолов Матиас участия почти не принимал.
\vs p193 6:4 \pc Вскоре после Пятидесятницы близнецы вернулись в свои дома в Галилее. Симон Зилот перед тем, как идти проповедовать евангелие, на какое\hyp{}то время отошел от дел. Фома пребывал в беспокойстве недолгое время, а потом возобновил свое учение. Нафанаил все больше расходился с Петром в вопросе проповеди об Иисусе вместо провозглашения прежнего евангелия царства. К середине следующего месяца эти разногласия так обострились, что Нафанаил ушел в Филадельфию, чтобы встретиться с Авениром и Лазарем; и, проведя там более года, пошел в земли за Месопотамией, проповедуя евангелие, как сам его понимал.
\vs p193 6:5 В результате из первоначально избранных двенадцати апостолов участвовали в первом провозглашении евангелия в Иерусалиме лишь шестеро: Петр, Андрей, Иаков, Иоанн, Филипп и Матфей.
\vs p193 6:6 \pc Около полудня апостолы вернулись к своим братьям в комнату наверху и объявили, что новым апостолом был избран Матиас. Затем Петр призвал всех верующих к молитве --- молитве о том, чтобы они могли приготовиться к принятию дара духа, который обещал им послать Учитель.
