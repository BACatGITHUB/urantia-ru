\upaper{178}{Последний день в лагере}
\vs p178 0:1 Четверг, свой последний свободный день на земле в качестве божественного Сына, пришедшего во плоти, Иисус собирался провести со своими апостолами и несколькими верными и преданными учениками. В это прекрасное утро, вскоре после часа, посвященного завтраку, Учитель отвел их в уединенное место, находившееся недалеко от лагеря, и там наставлял их в ранее не известных им истинах. Хотя в тот день в предвечерние часы он наставлялапостолов и о другом, эта речь, состоявшаяся в четверг еще до наступления полудня, стала его прощальным обращением к совместно жившей в лагере группе апостолов и избранных учеников, как евреев, так и неевреев. Присутствовали все апостолы, кроме Иуды. Петр и другие апостолы обратили внимание на его отсутствие, и некоторые из них подумали, что Иисус отослал Иуду в город уладить какое\hyp{}нибудь дело, например, позаботиться о деталях предстоящего празднования ими Пасхи. В лагерь Иуда вернулся лишь спустя несколько часов пополудни, незадолго до того, как Иисус повел апостолов в Иерусалим вкусить от Тайной Вечери.
\usection{1.\bibnobreakspace Беседа о сыновстве и гражданстве}
\vs p178 1:1 Почти два часа беседовал Иисус приблизительно с пятьюдесятью своими доверенными последователями и отвечал на многочисленные вопросы, касающиеся отношений признающих себя сынами Бога к гражданству земных правительств. Эту беседу, включая ответы Иисуса на вопросы, можно обобщить и изложить на современном языке следующим образом:
\vs p178 1:2 \P\ Царства мира сего, будучи материальными, могут часто считать необходимым применять физическую силу для исполнения своих законов и поддержания порядка. В царстве же небесном истинно верующие к применению физической силы прибегать не будут. Царство небесное, будучи духовным братством рожденных от духа сынов Бога, может распространяться лишь силой духа. Это различие в образе действия указывает на отношения, существующие между царством верующих и царствами мирского управления, и никоим образом не аннулирует право социальных групп верующих поддерживать порядок в своих рядах и применять наказания к распущенным и недостойным членам.
\vs p178 1:3 Между сыновством в духовном царстве и гражданском участии в светском или гражданском управлении нет никакой несовместимости. Отдавать кесарю кесарево, а Богу божье --- долг верующего. Между этими двумя требованиями, одно из которых материального, а другое --- духовного плана, не может быть никаких противоречий, если не окажется, что кесарь позволяет себе узурпировать исключительные права Бога и требует, чтобы воздавалось и духовное, и верховное почитание. В подобном случае вы должны почитать только Бога и одновременно стараться просветить подобных заблуждающихся правителей, тем самым ведя и их к признанию Отца Небесного. Воздавать духовное почитание земным правителям нельзя; нельзя использовать физические возможности земных правительств (члены которых, не исключено, когда\hyp{}нибудь станут верующими) в деле расширения миссии духовного царства.
\vs p178 1:4 С точки зрения совершенствующейся цивилизации, сыновство в царстве должно помочь вам стать идеальными гражданами царств мира сего, ведь братство и служение --- краеугольные камни евангелия царства. Зов любви, исходящий из духовного царства, должен стать действенным разрушителем порывов ненависти у неверующих и воинственно настроенных граждан земных царств. Однако думающие только о материальном и пребывающие во тьме сыновья никогда не увидят исходящий от вас духовный свет истины, если вы не будете воздействовать непосредственнона них путем бескорыстного общественного служения, которое в жизненном опыте каждого верующего человека в отдельности является естественным результатом принесения плодов духа.
\vs p178 1:5 Как смертные и материальные люди вы действительно являетесь гражданами земных царств и должны быть хорошими гражданами, став же заново рожденными духовными сынами, вы будете еще лучшими гражданами. Как просвещенные верой и освобожденные духом сыны царства небесного вы стоите перед двойной ответственностью --- долгом перед человеком и долгом перед Богом, одновременно добровольно принимая на себя третье и священное обязательство: служение братству верующих, знающих Бога.
\vs p178 1:6 Вам не следует поклоняться вашим временным правителям и вы не должны пользоваться мирской властью для расширения духовного царства; однако вы должны являть праведное служение, с равной любовью служа и верующим и неверующим. В евангелии царства живет могучий Дух Истины, и я вскоре изолью сей дух на всякую плоть. Плоды же духа, ваше искреннее и полное любви служение, являются мощным общественным рычагом, вызволяющим народы из тьмы, и сей Дух Истины станет для вас точкой опоры, приумножающей силы.
\vs p178 1:7 В своих отношениях с неверующими светскимиправителями проявляйте мудрость и демонстрируйте прозорливость. Будьте благоразумны и покажите, что вы умеете сглаживать небольшие разногласия и устранять мелкие недоразумения. Всеми возможными способами --- во всем, что не затрагивает вашу духовную преданность правителям вселенной, --- старайтесь жить в мире со всеми людьми. Всегда будьте мудры, как змеи, и просты, как голуби.
\vs p178 1:8 Став просвещенными сыновьями царства, вы будете лучшими подданными светского государства; подобно тому, веруя в сие евангелие царства небесного, члены земных правительств станут лучшими правителями в гражданских делах. Позиция бескорыстного служения человеку и разумного почитания Бога сделает всех верующих царства лучшими гражданами мира, в то время как позиция честного гражданства и искренней преданности мирским обязанностям поможет таким гражданам намного легче воспринять духовный призыв к сыновству в царстве небесном.
\vs p178 1:9 Пока члены земных правительств играют рольрелигиозных диктаторов, вы, верующие в сие евангелие царства, можете рассчитывать только на беды, преследования и даже на смерть. Однако сам свет, который вы несете в мир, а равно и то, как вы будете страдать и умирать за сие евангелие царства, в конце концов просветят весь мир и приведут к отделению политики от религии. Настойчивая проповедь сего евангелия царства однажды принесет всем народам новое и поразительное освобождение, интеллектуальную независимость и религиозную свободу.
\vs p178 1:10 В условиях надвигающихся преследований со стороны ненавидящих сие евангелие радости и свободы, вы преуспеете, а царство расцветет. Однако смертельной опасности вы подвергнетесь в последующие времена, когда большинство людей будет хорошо говорить о верующих царства и многие в верхах номинально примут евангелие царства небесного. Научитесь быть верными царству даже во времена мира и процветания. Не искушайте ангелов, хранящих вас, чтобы те не направили вас на пути бед, как требует того дисциплина, происходящая от любви к вам и необходимая для спасения ваших ленивых душ.
\vs p178 1:11 Помните, что вы призваны проповедовать сие евангелие царства --- верховное желание исполнять волю Отца в сочетании с верховной радостью достижения через веру сыновства по отношению к Богу --- и ничему не позволяйте мешать вам посвятить себя исполнению этого долга. Пусть пойдут на благо всему человечеству переполняющее вас и полное любви служение, просвещающее интеллектуальное общение и возвышающая общественная деятельность; но ни одному из сих гуманных трудов, ни даже всем им вместе взятым, не давайте занять место провозглашения евангелия. Эти чрезвычайно важные служения являются побочными общественными продуктами еще более грандиозных служений и преобразований, осуществляемых в сердце верующего царства живым Духом Истины и личным осознанием того, что вера рожденного от духа человека дарует уверенность в живом родстве с вечным Богом.
\vs p178 1:12 Не пытайтесь нести истину или утверждать праведность властью гражданских правительств либо силой светских законов. Вы всегда можете трудиться, дабы вразумить людей, но никогда не смейте принуждать их. Не забывайте великий закон человеческой справедливости, которому я учил вас в позитивной форме: как хотите, чтобы с вами поступали люди, так поступайте и вы с ними.
\vs p178 1:13 Когда верующий царства призван служить гражданскому правительству, пусть как временный подданный такого царства исполняет подобное служение, хотя такой верующий в своем служении должен проявлять все обыкновенные качества гражданина, еще больше усиленные духовным просвещением, происходящим от облагораживающего общения ума смертного человека с пребывающим в нем духом вечного Отца. Если же неверующий может лучше исполнять обязанности гражданского служащего, то вам следует серьезно задаться вопросом: а не мертвы ли корни истины в вашем сердце от недостатка воды живой, суть которой --- нераздельность духовного общения и общественного служения. Сознание сыновства по отношению к Богу оживляет все жизненное служение каждого мужчины, каждой женщины и каждого ребенка, ставших обладателями такого мощного стимулятора для всех сил, присущих человеческой личности.
\vs p178 1:14 Нельзя быть пассивными мистиками или бледными аскетами; нельзя становиться мечтателями и никчемными людьми, безвольно полагающимися на то, что воображаемое Провидение обеспечит вас даже предметами первой необходимости. Вы действительно должны быть милосердны в общении с заблуждающимися смертными, терпеливы в сношениях с невежественными людьми и терпимы, когда вас умышленно раздражают; однако вы должны быть храбрыми, отстаивая праведность, сильными, распространяя истину, и агрессивными, проповедуя сие евангелие царства даже до края земли.
\vs p178 1:15 Сие евангелие царства есть живая истина. Я уже говорил вам, что оно подобно закваске в тесте или горчичному зерну; теперь же объявляю: оно подобно семени живого существа, которое из поколения в поколение, оставаясь тем же самым живым семенем, неизбежно раскрывается в новых проявлениях и должным образом адаптируется к специфическим потребностям и условиям жизни каждого последующего поколения. Откровение, данное мною вам, есть \bibemph{живое откровение,} и я желаю, чтобы оно, согласно законам духовного роста, приумножения и адаптивного развития, приносило подобающие плоды в каждом отдельно взятом человеке и в каждом поколении. Из рода в род сие евангелие должно показывать все возрастающую энергию и являть все большие глубины духовной силы. Нельзя допустить, чтобы оно стало просто священным воспоминанием или простым преданием обо мне и о временах, в которые мы живем ныне.
\vs p178 1:16 Не забывайте: мы не посягали непосредственно на личности или на власть восседающих на седалище Моисеевом; мы лишь предлагали им новый свет, который они столь энергично отвергли. Мы подвергали их резкой критике, осуждая лишь их духовную неверность тем самым истинам, учителями и охранителями которых они себя называют. Мы вступали в конфликт с этими авторитетными лидерами и признанными правителями лишь тогда, когда они прямо становились на пути проповеди евангелия царства сыновьям человеческим. Вот и сейчас не мы нападаем на них, а они ищут нашей погибели. Не забывайте, что вы призваны идти и проповедовать только благую весть. Не обличайте старое, но искусно вносите закваску новой истины в среду старых верований. Пусть Дух Истины делает свое дело сам. Пусть спор возникает лишь тогда, когда презревшие истину вынуждают вас к нему. Когда же своевольный неверующий ополчится на вас, без колебаний становитесь на защиту истины, которая вас спасла и освятила.
\vs p178 1:17 Какими бы ни были превратности жизни, всегда помните, что вы должны любить друг друга. Не боритесь с людьми, даже с неверующими. Будьте милосердны даже к тем, кто презрительно оскорбляет вас. Покажите себя лояльными гражданами, честными ремесленниками, достойными похвалы соседями, верными родственниками, понимающими родителями и искренне верующими в братство в царстве Отца. И дух мой пребудет на вас отныне и до конца света.
\vs p178 1:18 \P\ Когда Иисус закончил свое учение, был уже почти час дня, и они сразу пошли в лагерь, где Давид и его товарищи приготовили для них обед.
\usection{2.\bibnobreakspace После полуденной трапезы}
\vs p178 2:1 Не многие из слушателей Учителя смогли усвоить хотя бы часть произнесенной им до полудня речи. Из всех слушавших Иисуса наиболее понятливыми оказались греки. Даже одиннадцать апостолов, и тех смутили его высказывания о будущих политических царствах и сменяющих друг друга поколениях верующих царства. Наиболее преданные последователи Иисуса не могли увязать надвигающийся конец его земного служения с этими высказываниями о длительной евангельской деятельности в будущем. Некоторые из этих верующих евреев начинали чувствовать приближение величайшей трагедии земли, но не могли увязать подобную надвигающуюся катастрофу ни с веселым и беззаботным настроением Учителя, ни с его дополуденной речью, в которой он неоднократно намекал на будущие дела царства небесного, простирающиеся на долгие времена и охватывающие отношения со многими и сменяющими друг друга мирскими царствами на земле.
\vs p178 2:2 К полудню этого дня все апостолы и ученики узнали о поспешном бегстве Лазаря из Вифании. Они стали ощущать зловещую решимость еврейский правителей убить Иисуса и искоренить его учение.
\vs p178 2:3 Давид Заведеев благодаря работе своих тайных агентов в Иерусалиме был полностью информирован о планах арестовать и убить Иисуса. Он все знал о роли Иуды в этом заговоре, но так и не открыл этого ни другим апостолам, ни кому\hyp{}либо из учеников. Вскоре после обеденной трапезы Давид отвел Иисуса в сторону и, набравшись смелости, спросил, знает ли Иисус об этом, --- но так и не смог договорить свой вопрос. Учитель взял его за руку и, прервав, сказал: <<Да, Давид, я знаю об этом все и знаю, что и ты знаешь, однако, смотри, ничего никому не рассказывай. Не сомневайся в сердце своем: воля Бога в конце концов восторжествует>>.
\vs p178 2:4 Этот разговор с Давидом был прерван прибытием вестника из Филадельфии, принесшего известие о том, что Авенир узнал о замысле убить Иисуса и спрашивает, следует ли ему отправиться в Иерусалим. Гонец поспешно вернулся в Филадельфию с известием Авениру: <<Продолжай свое дело. Если я и покину тебя во плоти, то затем лишь, чтобы вернуться в духе. Я не оставлю тебя. И буду с тобой до конца>>.
\vs p178 2:5 Приблизительно в это время Филипп подошел к Учителю и спросил: <<Учитель, приближается время Пасхи, где велишь нам приготовить ее?>> Выслушав вопрос Филиппа, Иисус ответил: <<Ступай и приведи Петра с Иоанном, и я дам вам указания относительно вечерней трапезы, которую мы будем вместе вкушать этой ночью. Что же касается Пасхи, об этом вам придется подумать после того, как сначала сделаем это>>.
\vs p178 2:6 Услышав, что Учитель обсуждает с Филиппом эти вопросы, Иуда подошел ближе, чтобы подслушать их разговор. Но Давид Заведеев, находившийся рядом, вышел вперед и вовлек Иуду в беседу, а Филипп, Петр и Иоанн отошли поговорить с Учителем в сторону.
\vs p178 2:7 Иисус сказал трем апостолам: <<Тотчас ступайте в Иерусалим, и когда войдете в ворота, встретится вам человек, несущий кувшин с водой. Он заговорит с вами, и тогда идите за ним. Когда же он приведет вас к некоему дому, войдите за ним и спросите у доброго человека дома того: >>Где горница, в которой Учителю вкушать вечернюю трапезу со своими апостолами?<< Когда же спросите об этом, сей хозяин дома покажет вам большую комнату, устланную и приготовленную для нас>>.
\vs p178 2:8 Придя к городу, апостолы встретили у ворот человека с кувшином воды и, следуя за ним, пришли к дому Иоанна Марка, где отец мальчика встретил их и показал им верхнюю комнату, приготовленную к вечерней трапезе.
\vs p178 2:9 Случилось же это благодаря взаимопониманию, достигнутому между Учителем и Иоанном Марком после полудня днем раньше, когда они были одни в горах. Иисус хотел быть уверенным, что сможет спокойно провести свою последнюю трапезу с апостолами, и, полагая, что если Иуда заранее узнает об их месте встречи, то поможет врагам схватить его, тайно обо всем договорился с Иоанном Марком. Таким образом, Иуда не знал о месте предстоящей встречи до тех пор, пока позднее не пришел туда вместе с Иисусом и другими апостолами.
\vs p178 2:10 \P\ У Давида Заведеева было к Иуде множество деловых вопросов, так что помешать Иуде пойти за Петром, Иоанном и Филиппом, чего тот очень хотел, было несложно. Когда Иуда выдал Давиду на покупку снеди необходимую сумму, Давид спросил его: <<Иуда, не будет ли правильно при возникших обстоятельствах дать мне немного денег вперед, на предстоящие нужды?>> Немного подумав, Иуда ответил: <<Да, Давид, я думаю, это вполне разумно. Действительно, учитывая тревожное положение в Иерусалиме, будет лучше всего, если я передам тебе все деньги. Против Учителя готовится заговор, и если со мной что\hyp{}нибудь случится, вы не будете стеснены в средствах>>.
\vs p178 2:11 Так Давид получил всю апостольскую казну наличными и расписки о всех деньгах, отданных в рост. Апостолы же до следующего вечера об этой передаче денег ничего не знали.
\vs p178 2:12 \P\ Было около половины пятого, когда три апостола возвратились и сообщили Учителю, что для вечерней трапезы все готово. Учитель тотчас приготовился вести двенадцать апостолов по тропе в сторону Вифанийской дороги, а оттуда --- в Иерусалим. Это было последнее путешествие, которое он совершил со всеми двенадцатью апостолами.
\usection{3.\bibnobreakspace В пути на вечерю}
\vs p178 3:1 Стараясь избежать встречи с толпами, пересекавшими Кедронскую долину по пути из Гефсиманского сада в Иерусалим и обратно, Иисус и двенадцать апостолов пошли по западному склону Масличной горы, направляясь к дороге, ведущей из Вифании в город. Подойдя к месту, где Иисус предыдущим вечером задержался для беседы о разрушении Иерусалима, они невольно остановились и молча смотрели на раскинувшийся внизу город. Поскольку к месту они подошли раньше времени и так как Иисус не хотел идти через город до захода солнца, Учитель сказал своим сподвижникам:
\vs p178 3:2 \P\ <<Садитесь и отдыхайте, пока я буду говорить с вами о том, чему вскоре предстоит произойти. Все эти годы я жил с вами как с братьями, уча вас истине о царстве небесном и открывая вам его тайны. Отец же мой в связи с моей миссией на земле действительно совершил много чудесного. Вы были свидетелями всему этому и участвовали в опыте сотрудничества с Богом. Вы же будете свидетельствовать, что в течение некоторого времени я предупреждал вас, что вскоре должен буду вернуться к делу, совершить которое мне поручено Отцом; я также ясно говорил вам, что должен буду оставить вас в мире продолжать дело царства. Именно с этой целью я и отметил вас в горах Капернаума. Теперь же будьте готовы пережитое вами со мной разделить с остальными. Как послал меня Отец в этот мир, так и я собираюсь послать вас представлять меня и закончить начатое мной дело.
\vs p178 3:3 На город сей вы смотрите в печали, потому что слышали мои слова, говорящие о конце Иерусалима. Я предупредил вас, чтобы вам не погибнуть при его разрушении и, таким образом, не задержать провозглашение евангелия царства. Подобно тому, я предостерегаю вас: будьте осторожны и без нужды не подвергайте себя опасности, когда придут брать Сына Человеческого. Я должен уйти; вам же надлежит остаться и свидетельствовать о сем евангелии, когда меня не будет с вами, так же, как я повелел Лазарю бежать от гнева человеческого, дабы он мог жить, являя славу Бога. Если воля Отца такова, что я должен умереть, то ничто из того, что вы можете сделать, не способно расстроить божественный план. Берегитесь, чтобы и вас не убили. Да будут души ваши мужественны, защищая евангелие силой духа, но не дайте ввести себя в заблуждение и не предпринимайте неразумных попыток защитить Сына Человеческого. В защите руки человеческой я не нуждаюсь; воинства небесные и сейчас рядом со мной; но я решил исполнить волю Отца моего небесного и, стало быть, мы должны подчиниться тому, что вскоре посетит нас.
\vs p178 3:4 Когда же увидите этот город разрушенным, не забывайте, что вы уже вступили в вечную жизнь бесконечного служения в постоянно приближающемся царстве небесном, и даже в царстве неба небес. Вы должны знать, что во вселенной Отца моего и моей обителей много и что там детей света ожидает откровение городов, строителем которых является Бог, а также миров, где образ жизни есть праведность и радость в истине. Я принес вам царство небесное сюда, на землю, но я заявляю вам, что все из вас, верой в него входящие и остающееся в нем благодаря живому служению истины, непременно взойдут в миры в верхах и воссядут со мной в духовном царстве Отца моего. Однако прежде вам надлежит препоясаться и завершить дело, начатое вами вместе со мной. Вы должны сначала пройти через многие беды и претерпеть много печалей --- испытаниям этим мы подвергаемся уже сейчас --- закончив же ваше дело на земле, войдете в радость мою так же, как я на земле завершил дело Отца моего и готов вернуться в его объятия>>.
\vs p178 3:5 \P\ Кончив говорить, Учитель поднялся, и все они, идя за ним, стали спускаться с Масличной горы в город. Никто из апостолов, кроме троих, не знал, куда они направляются, идя по узким улочкам в сгущающейся тьме. Многие толкали их, но никто их не узнал и не догадался, что мимо проходит Сын Человеческий, идущий на последнее смертное свидание со своими избранными посланцами царства. Не знали и апостолы, что один из них уже вошел в сговор, дабы предать Учителя в руки его врагов.
\vs p178 3:6 Все это время Иоанн Марк шел следом за ними до самого города, а после того, как они прошли ворота, побежал по другой улице, так что, когда они пришли, он уже ожидал их, дабы приветствовать в доме своего отца.
