\upaper{66}{Планетарный Принц Урантии}
\author{Мелхиседек}
\vs p066 0:1 Приход Сына\hyp{}Ланонандека в обычный мир означает, что воля, способность выбирать путь продолжения вечного существования, развилась в разуме примитивного человека. Но на Урантию Планетарный Принц прибыл спустя почти полмиллиона лет после возникновения человеческой воли.
\vs p066 0:2 Калигастия, Планетарный Принц, прибыл на Урантию примерно пятьсот тысяч лет назад, одновременно с появлением шести цветных, или сангикских рас. Во время прибытия Принца на земле было почти полмиллиарда примитивных человеческих существ, которые были рассеяны по Европе, Азии и Африке. Центр Принца, основанный в Месопотамии, располагался примерно в середине мирового расселения человеческих существ.
\usection{1. Принц Калигастия}
\vs p066 1:1 Калигастия был Сыном\hyp{}Ланонандеком, номер 9344 вторичного чина. У него уже был опыт управления делами локальной вселенной в целом, а в последнее время он, в частности, управлял локальной системой Сатании.
\vs p066 1:2 До времен правления Люцифера в Сатании Калигастия был членом совета Носителей Жизни в Иерусеме. Люцифер возвысил Калигастию, включив его в свою личную свиту, и он последовательно занимал пять должностей, вполне достойно исполняя обязанности, требующие чести и доверия.
\vs p066 1:3 \pc Калигастия очень рано стал стремиться получить полномочия Планетарного Принца, но неоднократно, когда его запрос поступал для утверждения в советы созвездия, он не получал согласия Отцов Созвездия. Калигастия, по\hyp{}видимому, особенно жаждал быть назначенным планетарным правителем в десятичный мир, или мир видоизмененной жизни. Ему несколько раз отказывали в просьбе, прежде чем он наконец был назначен на Урантию.
\vs p066 1:4 Калигастия отправился из Иерусема в доверенное ему мировое владение, имея завидную характеристику, удостоверяющую его лояльность и преданность делу процветания вселенной своего происхождения и пребывания, несмотря на явное проявление нетерпеливости, в сочетании со склонностью не соглашаться по каким\hyp{}то незначительным вопросам с установленным порядком.
\vs p066 1:5 Я присутствовал в Иерусеме, когда блестящий Калигастия отправился из столицы системы. Ни один из принцев планет не начинал карьеру мирового управления с более богатым практическим опытом и с лучшими перспективами, чем Калигастия в тот насыщенный событиями день полмиллиона лет назад. Ясно одно: в то время как я выполнял свое задание передать рассказ об этом событии по сетям трансляции локальной системы, я ни на одно мгновение не допускал даже в малейшей степени мысли о том, что этот благородный Ланонандек так скоро предаст свой священный долг заботиться о планете и так ужасно опозорит честное имя своего возвышенного чина во вселенском сыновстве. Я действительно относил Урантию к числу пяти или шести самых счастливых планет во всей Сатании, поскольку мировыми делами на ней руководит такой опытный, блестящий и оригинальный ум. Я тогда не предполагал, что Калигастия постепенно впадает в самолюбование; я тогда еще совершенно не понимал, до какой степени его обуяла гордыня.
\usection{2. Штат Принца}
\vs p066 2:1 Планетарный Принц Урантии был послан для выполнения своей миссии не один, его сопровождал обычный отряд ассистентов и административных помощников.
\vs p066 2:2 Во главе группы был Далигастия, сподвижник\hyp{}ассистент Планетарного Принца. Далигастия был также вторичным Сыном\hyp{}Ланонандеком и имел номер 319 407 этого чина. Перед назначением сподвижником Калигастии он был в ранге ассистента.
\vs p066 2:3 Планетарный штат включал большое число ангельских соратников и сонм других небесных созданий, посланных для того, чтобы способствовать интересам и процветанию человеческих рас. Но с вашей точки зрения, наиболее заслуживающей внимания группой были облеченные в плоть члены свиты Принца --- иногда называемые как \bibemph{сотня Калигастии.}
\vs p066 2:4 \pc Эта сотня рематериализированных членов штата Принца была выбрана Калигастией из более чем 785 000 граждан Иерусема, прошедших путь восхождения, которые добровольно вызвались участвовать в событиях на Урантии. Каждый член этой избранной сотни был с разных планет, и ни один из них не был с Урантии.
\vs p066 2:5 Эти иерусемские добровольцы были перенесены серафическим транспортом прямо из столицы системы на Урантию, и по прибытии они оставались внутри серафимов до момента, когда их можно было снабдить личностными формами двойственной природы, специально предназначенными для служения на планете, а именно: телами из плоти и крови, которые были также связаны с жизненными контурами системы.
\vs p066 2:6 \pc Незадолго до прибытия этой сотни иерусемских граждан два руководящих Носителя Жизни, постоянно пребывающих на Урантии и уже закончившие составление своих планов, испросили разрешения в Иерусеме и Эдентии трансплантировать жизненную плазму ста избранных потомков из ветви Андона и Фонты в материальные объекты, из которых будут созданы тела членов свиты Принца. Запрос был удовлетворен в Иерусеме и одобрен в Эдентии.
\vs p066 2:7 Соответственно, пятьдесят мужчин и пятьдесят женщин, потомков Андона и Фонты, представители лучших сохранившихся, ветвей этой уникальной расы, были выбраны Носителями Жизни. За одним или двумя исключениями, эти андониты, способствовавшие улучшению расы, были незнакомы друг другу. Их собрали из разных отдаленных от планетарного центра Принца мест благодаря координированному руководству Настройщиков Мыслей и серафическому ведению. Здесь сто человеческих существ были переданы комиссии очень искусных добровольцев из Авалона, которые и руководили материальной экстракцией части жизненной плазмы этих потомков Андона. Это жизненное вещество было затем введено в материальные тела, созданные для ста членов сотни иерусемской свиты Принца. В это время эти вновь прибывшие граждане столицы системы оставались в состоянии сна в серафическом транспорте.
\vs p066 2:8 \pc Эти преобразования, вместе с непосредственным созданием специальных тел для сотни Калигастии, положили начало многочисленным легендам, многие из которых впоследствии стали отождествлять с более поздними преданиями, касающимися водворения на планете Адама и Евы.
\vs p066 2:9 Весь ход реперсонализации, начиная с момента прибытия серафических транспортов, несущих сотню добровольцев Иерусема, и до того, как они стали сознательными, троичными существами этого мира, занял ровно десять дней.
\usection{3. Даламатия --- город Принца}
\vs p066 3:1 Центр Планетарного Принца был расположен в районе Персидского залива тех дней, где в более позднее время располагалась Месопотамия.
\vs p066 3:2 В те дни климат и ландшафт Месопотамии были во всех отношениях благоприятными для деятельности штата Принца и их ассистентов и очень отличались от условий, которые стали преобладать в более поздние времена. Чтобы побудить примитивных обитателей создать определенные зачатки культуры и цивилизации, была необходима природная среда с благоприятным климатом. Одной из величайших задач тех веков было постепенное превращение человека из охотника в скотовода, позволяющая надеяться, что со временем он станет миролюбивым, оседлым фермером.
\vs p066 3:3 \pc Центр Планетарного Принца на Урантии ничем не отличался от таких же станций на молодой и развивающейся сфере. Это был простой, но красивый город, окруженный стеной высотой в сорок футов. Этот мировой центр культуры назывался Даламатией в честь Далигастии.
\vs p066 3:4 Город был разделен на десять секторов, в центрах которых располагались обители десяти советов облеченной в плоть свиты. В самом центре города стоял храм невидимого Отца. Административное здание Принца и его сподвижников было разделено на двенадцать залов, расположенных непосредственно вокруг храма.
\vs p066 3:5 Все строения Даламатии были одноэтажными, за исключением центров советов, которые были двухэтажными, и храма всеобщего Отца --- небольшого трехэтажного здания.
\vs p066 3:6 Город представлял собой прекрасный пример применения наилучшего строительного материала тех древних времен --- кирпича. Камень или дерево использовались очень мало. Строительство домов и деревенская архитектура окружающих народов впоследствии были значительно усовершенствованы, следуя примеру Даламатии.
\vs p066 3:7 \pc Возле центра Принца обитали группы человеческих существ разного цвета кожи. И первые ученики школ Принца были набраны из этих ближних племен. Хотя эти ранние школы Даламатии были элементарными, они обеспечивали всем, что было необходимо мужчинам и женщинам этого первобытного времени.
\vs p066 3:8 Облеченный в плоть штат Принца постоянно собирал вокруг себя наиболее развитых личностей окружающих племен и, обучив и вдохновив этих учеников, отправлял назад как учителей и лидеров своих народов.
\usection{4. Первоначальный период}
\vs p066 4:1 Прибытие штата Принца произвело колоссальное впечатление. Хотя понадобилась почти тысяча лет для распространения этой новости, учение и поведение ста новых жителей Урантии чрезвычайно влияли на племена, жившие вокруг Месопотамского центра. И многое в вашей последующей мифологии отображает в искаженном виде предания тех древних времен, когда члены штата Принца были реперсонализированы на Урантии как сверхлюди.
\vs p066 4:2 Положительному влиянию внепланетарных учителей серьезно препятствовали представления смертных о них как о богах, но, кроме способа своего появления на земле, первая сотня Калигастии --- пятьдесят мужчин и пятьдесят женщин --- не прибегала ни к сверхъестественным методам, ни к сверхчеловеческим манипуляциям.
\vs p066 4:3 Но, тем не менее, облеченный в плоть штат был сверхчеловеческим. Они начали свою миссию на Урантии как необычные троичные существа:
\vs p066 4:4 \pc \ublistelem{1.}\bibnobreakspace Они обладали телесностью и были в определенной мере людьми, поскольку в них была подлинная жизненная плазма одной из человеческих рас --- андонитская жизненная плазма Урантии.
\vs p066 4:5 Эти сто членов штата Принца были поровну разделены по половому признаку в соответствии с их прежним статусом смертных. Каждый член этой группы был способен стать одним из родоначальников какого\hyp{}нибудь нового вида физических существ, но исчерпывающие инструкции обязывали их производить потомство только в определенных условиях. Обычно облеченный в плоть штат Планетарного Принца порождал своих преемников за какое\hyp{}то время до прекращения своей особой службы на планете. Как правило, это происходит во время или вскоре после прибытия Планетарного Адама и Евы.
\vs p066 4:6 Поэтому эти особые существа не имели практически никакого представления о том, какого типа материальное создание может получиться от их сексуального союза. И они никогда и не узнали этого; прежде чем наступил этот этап в осуществлении их работы в мире, весь ее ход был нарушен восстанием, и те, кто впоследствии все\hyp{}таки стали родителями, были уже изолированы от потоков жизни системы.
\vs p066 4:7 По цвету кожи и языку эти материализованные члены свиты Калигастии соответствовали андонитской расе. Так же, как и смертные этого мира, они зависели от пищи, однако воссозданные тела этой группы вполне довольствовались вегетарианской пищей. Это было одной из причин, которая определила их пребывание в теплом регионе, изобилующем фруктами и орехами. Практика придерживаться вегетарианской диеты восходит к временам сотни Калигастии, этот обычай широко распространился и повлиял на особенности питания многих окружающих племен, групп, произошедших от эволюционных рас, когда\hyp{}то питавшихся исключительно мясом.
\vs p066 4:8 \pc \ublistelem{2.}\bibnobreakspace Сотня состояла из материальных, но сверхчеловеческих созданий, которые были воссозданы на Урантии как уникальные мужчины и женщины особого высокого порядка.
\vs p066 4:9 Члены этой группы, пока обладали в Иерусеме статусом предварительного гражданства, не слились еще со своими Настройщиками Мысли; а когда они вызвались добровольцами и были приняты на планетарную службу во взаимодействии с нисходящими чинами сыновства, их Настройщики были отделены. Но все\hyp{}таки эти граждане Иерусема были сверхчеловеческими существами --- они обладали душами, прошедшими восхождение. Во время смертной жизни во плоти душа находится в эмбриональном состоянии; она рождается (воскресает) в моронтийной жизни и растет, проходя через последовательные моронтийные миры. И души первой сотни Калигастии развились через прогрессивный опыт миров\hyp{}обителей до статуса гражданина Иерусема.
\vs p066 4:10 В соответствии с полученными инструкциями члены свиты не участвовали в половом размножении, но они старательно изучали свое собственное телосложение и тщательно исследовали все мыслимые фазы интеллектуальной (разум) и моронтийной (душа) связи. И во время тридцать третьего года их пребывания в Даламатии, задолго до того, как была окончена стена, номер второй и номер седьмой группы Дана неожиданно открыли феномен, сопутствующий связи их моронтийных личностей (предположительно несексуальной и нематериальной); и результатом этого события стало появление первого срединника первого рода. Это новое существо было полностью видимым для планетарного штата и для их небесных сподвижников, но было невидимо для мужчин и женщин различных человеческих племен. По указанию Планетарного Принца весь облеченный плотью штат начал создавать подобные существа, и все добились успеха, следуя инструкциям пары из группы Дана. Так свита Принца в конечном итоге дала существование новому отряду из 50 000 срединников первого рода.
\vs p066 4:11 Эти создания срединного типа были исключительно полезны в выполнении дел мирового центра. Они были невидимы для человеческих существ, но примитивные обитатели Даламатии были извещены об этих невидимых полудухах; и на протяжении веков они являли собой весь духовный мир для развивающихся смертных.
\vs p066 4:12 \pc \ublistelem{3.}\bibnobreakspace Сотня Калигастии обладала личным бессмертием, или неумиранием. Через их материальные формы циркулировали компоненты\hyp{}противоядия жизненных потоков системы; и если бы они не потеряли контакт с жизненными контурами из\hyp{}за бунта, то, безусловно, дожили бы до прибытия следующего Сына Бога, или до того времени, когда бы им снова позволили продолжить прерванное путешествие к Хавоне и к Раю.
\vs p066 4:13 Эти противоядные компоненты жизненных потоков Сатании извлекались из плодов древа жизни, кустарника Эдентии, который был передан на Урантию Всевышними Норлатиадека во время прибытия Калигастии. В дни Даламатии это дерево росло в центральном внутреннем дворе храма невидимого Отца, и именно плоды этого древа жизни позволяли материальным, а значит, смертным существам штата Принца жить неограниченно долго, пока у них был доступ к плодам.
\vs p066 4:14 Не обладая никакой ценностью для эволюционирующих рас, эта сверхпища была достаточной для дарования продолжительности жизни сотне Калигастии и ста модифицированным андонитам, взаимодействующим с ней.
\vs p066 4:15 \pc В этой связи следует объяснить, что когда сто андонитов предоставили свою человеческую зародышевую плазму членам штата Принца, Носители Жизни подключили их смертные тела к контурам системы; и поэтому они могли жить вместе с членами штата, век за веком, избегая физической смерти.
\vs p066 4:16 Со временем ста андонитам было сообщено об их вкладе в новые формы своих высших собратьев, и именно эти сто детей племен Андона содержались в центре как персональные помощники облеченного в плоть штата Принца.
\usection{5. Организация сотни}
\vs p066 5:1 Сотня была организована для работы в десяти автономных советах по десять членов в каждом. Когда два или более из этих десяти советов встречались на объединенной сессии, на таких общих собраниях председательствовал Далигастия. Эти десять групп были организованы следующим образом:
\vs p066 5:2 \pc \ublistelem{1.}\bibnobreakspace \bibemph{Совет по пище и материальному благополучию.} Этой группой руководил Анг. Этот умелый отряд заботился о пище, воде, одежде и материальном прогрессе человеческого рода. Они учили рыть колодцы, контролировать источники и проводить ирригацию. Тех, кто жил в высоких широтах и на севере, они учили улучшенным методам обработки шкур для использования в качестве одежды, а позднее учителями искусства и науки было введено ткачество.
\vs p066 5:3 Огромные результаты были достигнуты в методах хранения пищи. Пища сохранялась приготовлением на огне, вялением и копчением; она, таким образом, стала самой ранней собственностью. Человек был обучен предусматривать опасные последствия голода, от которого периодически вымирало население мира.
\vs p066 5:4 \pc \ublistelem{2.}\bibnobreakspace \bibemph{Группа по одомашниванию и использованию животных.} Этот совет занимался проблемами селекции и разведения тех животных, которые лучше всего были приспособлены для помощи человеческим существам в перевозке и тяжестей, и их самих, пригодны для пищи, а позднее и для помощи при обработке почвы. Этот искусный отряд возглавлял Бон.
\vs p066 5:5 Было приручено несколько типов полезных животных, одни из них уже вымерли, а другие остаются домашними животными и в настоящее время. Человек долгое время жил вместе с собакой, а голубой человек уже добился успеха в приручении слона. Корова была так улучшена тщательным разведением, что стала ценным источником пищи; масло и сыр стали обычными составляющими стола человека. Люди были обучены использовать быка для перевозки тяжестей, но лошадь была одомашнена много позднее. Члены этого отряда впервые научили людей использовать колесо для облегчения тяги.
\vs p066 5:6 В те дни впервые был приручен почтовый голубь, которого брали в дальние путешествия, чтобы отправить сообщения или призывы о помощи. Группа Бона успешно обучала огромных фандоров как пассажирских птиц, но они вымерли более тридцати тысяч лет назад.
\vs p066 5:7 \pc \ublistelem{3.}\bibnobreakspace \bibemph{Советники, укротители хищных животных.} Первобытный человек не только старался одомашнить определенных животных, он должен был также научиться защищаться от нападения остальной части враждебного животного мира. Эту группу возглавлял Дан.
\vs p066 5:8 Древняя стена города была предназначена для защиты от диких зверей, а также неожиданных нападений враждебных племен. Безопасность людей, живших вне города и в лесах, зависела от убежищ на деревьях, от каменных хижин и от ночных костров. Поэтому вполне естественно, что эти учителя проводили много времени, наставляя своих учеников, как улучшать человеческие жилища. Огромный прогресс в укрощении животных был достигнут благодаря разработке более совершенных методов охоты и использованию ловушек.
\vs p066 5:9 \pc \ublistelem{4.}\bibnobreakspace \bibemph{Факультет распространения и сохранения знания.} Эта группа организовывала и направляла чисто образовательные процессы в эти древние века. Ее возглавлял Фад. Методы образования Фада состояли в том, что он руководил занятиями людей и одновременно обучал их улучшенным методам труда. Фад составил первый алфавит и ввел письменность. Этот алфавит содержал 25 букв. В качестве писчего материала древние люди использовали древесную кору, глиняные дощечки, каменные пластинки, разновидность пергамента, сделанного из разбитой молотком кожи, немного похожий на бумагу материал, сделанный из гнезд ос. Библиотека Даламатии, уничтоженная вскоре после предательства Калигастии, состояла более чем из двух миллионов отдельных записей и была известна как «дом Фада».
\vs p066 5:10 Голубой человек проявил склонность к алфавитной письменности и достиг огромного прогресса в этом направлении. Красный человек предпочитал письменность в виде картинок, тогда как желтые расы стремились к использованию символов для слов и понятий, почти таких же, как и те, которые они используют сегодня. Но впоследствии и алфавит, и многое другое было утеряно для мира во время беспорядков, сопровождавших бунт. Отступничество Калигастии разрушило надежды мира на всемирный язык, по крайней мере на неимоверно долгое время.
\vs p066 5:11 \pc \ublistelem{5.}\bibnobreakspace \bibemph{Комиссия по производству и торговле.} Этот совет был занят опекой над производством в племенах и содействием торговле между различными мирными группами. Ее лидером был Нод. Этим отрядом поощрялась каждая форма примитивного производства. Они непосредственно способствовали росту уровня жизни, снабжая примитивных людей многими понравившимися им предметами потребления. Они сильно увеличили торговлю качественной солью, произведенной советом по науке и искусству.
\vs p066 5:12 Среди этих продвинутых групп, обученных в школах Даламатии, был введен в практику первый коммерческий кредит. В центре обмена кредитов они получали жетоны, которые принимались вместо подлинных объектов натурального обмена. Мир не превзошел эти методы бизнеса в течение сотен тысяч лет.
\vs p066 5:13 \pc \ublistelem{6.}\bibnobreakspace \bibemph{Колледж религии откровения.} Этот орган действовал медленно. Цивилизация Урантии буквально выковывалась между наковальней нужды и молотом страха. Но эта группа добилась значительного прогресса в своем стремлении заменить страх перед созданием (поклонение призракам) страхом перед Творцом еще до того, как их труды впоследствии были прерваны беспорядками, сопровождавшими раскол. Главой этого совета был Хэп.
\vs p066 5:14 Никто из свиты Принца не явил бы откровение, которое могло бы усложнить процесс эволюции; откровение было явлено ими только тогда, когда силы эволюции совершенно истощились. Но Хэп пошел навстречу желаниям обитателей города установить форму религиозной службы. Его группа дала даламатийцам семь молитвенных песнопений, повседневную фразу\hyp{}восхваление и впоследствии научила их «молитве Отцу», а именно:
\vs p066 5:15 \pc «Отец всего сущего, чьего Сына мы чтим, обрати свой взгляд на нас благосклонно. Избавь нас от всех страхов, кроме страха перед тобой. Дай нам радовать наших божественных учителей и навсегда вложи истину в наши уста. Избавь нас от насилия и гнева; дай нам уважение к нашим старшим и тому, что принадлежит нашим соседям. Дай нам в этом году зеленые пастбища и плодовитые стада, чтобы это радовало наши сердца. Мы молимся, чтобы скорее пришел обещанный реализатор подъема, и мы будем выполнять твою волю в этом мире, как другие выполняют ее в иных мирах».
\vs p066 5:16 \pc Хотя возможности штата Принца были ограничены естественными ресурсами и обычными методами улучшения расы, они все же обещали адамический дар новой расы, которая станет целью последующего эволюционного роста после достижения вершины биологического развития.
\vs p066 5:17 \pc \ublistelem{7.}\bibnobreakspace \bibemph{Хранители здоровья и жизни.} Этот совет занимался внедрением санитарии и развитием примитивной гигиены и управлялся Лутом.
\vs p066 5:18 Его члены во многом учили тому, что в значительной степени было утеряно впоследствии во время беспорядков и было открыто вновь только в двадцатом веке. Они учили человечество, что приготовление пищи, кипячение и прожаривание продуктов, есть средство, позволяющее избежать заболеваний; что такое приготовление пищи значительно уменьшает детскую смертность и облегчает раннее отлучение от груди.
\vs p066 5:19 Много ранних учений хранителей здоровья Лута сохранились среди земных племен до дней Моисея, хотя и претерпели сильные искажения и значительные изменения.
\vs p066 5:20 Большим препятствием на пути распространения гигиены среди этих невежественных народов было то обстоятельство, что истинные возбудители болезней были слишком малы, чтобы видеть их невооруженным глазом, а также то, что все народы относились к огню с суеверным почтением. Потребовались тысячи лет, чтобы убедить их сжигать отбросы. До этого времени их учили закапывать свой гниющий мусор. Большим достижением санитарии этой эпохи было распространение знания о свойствах солнечного света, дающих здоровье и разрушающих болезни.
\vs p066 5:21 До прибытия Принца купание было исключительно религиозной церемонией. Было действительно трудно уговорить примитивных людей мыть свое тело в качестве меры, полезной для здоровья. Лут в конце концов убедил религиозных учителей включить очищение водой в церемонию очищения, проводимую еженедельно, во время полуденных обрядов, поклонения Отцу всего сущего.
\vs p066 5:22 Эти хранители здоровья также старались ввести рукопожатие взамен обмена слюной и выпивания крови, как знака личной дружбы и залога групповой преданности. Но как только влияние учений этих высших лидеров ослабло, примитивные народы довольно быстро вернулись к прежним разрушающим здоровье и вызывающим болезни обычаям, полным невежества и суеверия.
\vs p066 5:23 \pc \ublistelem{8.}\bibnobreakspace \bibemph{Планетарный совет по искусству и науке.} Этот отряд много сделал для улучшения производственных технических приемов труда древнего человека и для развития его представлений о красоте. Их лидером был Мек.
\vs p066 5:24 Искусство и наука пребывали в упадке во всем мире, но даламатийцев обучали основам физики и химии. Развивались гончарное ремесло и все виды декоративно\hyp{}прикладного искусства, и идеалы человеческой красоты поднялись на более высокий уровень. Но в музыке прогресс наметился только после того, как прибыла фиолетовая раса.
\vs p066 5:25 Эти первобытные люди не соглашались экспериментировать с энергией пара, невзирая на неоднократные настояния своих учителей; они никогда не смогли пересилить свой большой страх перед взрывной энергией находящегося под давлением пара. Их, однако, в конце концов склонили работать с металлами и огнем, хотя кусок раскаленного докрасна металла в древнего человека вселял страх.
\vs p066 5:26 Мек очень много сделал для того, чтобы развить культуру андонитов и искусство голубого человека. Смешение голубого человека с племенем Андона дало артистически одаренный тип, и многие из этих людей стали прекрасными скульпторами. Они не работали по камню или мрамору, но их творения из обожженной глины украшали сады Даламатии.
\vs p066 5:27 Огромный прогресс был достигнут в домашнем искусстве, которое в значительной степени было утрачено в долгие и темные века бунта, и уже никогда больше не обнаруживалось, вплоть до настоящего времени.
\vs p066 5:28 \pc \ublistelem{9.}\bibnobreakspace \bibemph{Правители развитых племенных отношений.} Этой группе была вверена работа по развитию человеческого общества до уровня государственности. Их руководителем был Тут.
\vs p066 5:29 Эти лидеры во многом способствовали межплеменным бракам. Они поощряли ухаживание и брак, совершаемый после должного обдумывания и знакомства. Чисто военные танцы были усовершенствованы и стали служить полезным целям общества. Было введено много игр соревновательного характера, но древние народы были серьезными людьми; юмор мало радовал эти ранние племена. Очень немногие обычаи пережили последующее разобщение из\hyp{}за планетарного бунта.
\vs p066 5:30 Тут и его сподвижники настойчиво способствовали племенным союзам миролюбивого характера, регулировали и гуманизировали приемы ведения войн, координировали межплеменные отношения и улучшали управление племенами. В окрестностях Даламатии развилась более совершенная культура, и эти привлекательные общественные отношения очень положительно влияли на отдаленные племена. Но паттерн цивилизации, преобладавшей в центре Принца, значительно отличался от варварского общества, развивающегося в других местах, точно так же, как в двадцатом веке общество Кейптауна, в Южной Африке совершенно непохоже на примитивную культуру бушменов\hyp{}карликов, живущих к северу от него.
\vs p066 5:31 \pc \ublistelem{10.}\bibnobreakspace \bibemph{Верховный суд племенной координации и кооперации рас.} Этот верховный совет возглавлял Ван; это была судебная инстанция, которая рассматривала все обращения, поступившие из других девяти специальных комиссий, занятых руководством делами человечества. У этого совета были широкие функции, ему вверялись все вопросы, касающиеся земных дел, которые не были специально поручены остальным группам. Члены этого избранного отряда были утверждены Отцами Созвездия в Эдентии перед тем, как они были уполномочены принять на себя функции верховного суда Урантии.
\usection{6. Правление Принца}
\vs p066 6:1 Уровень культуры мира измеряется социальным наследием его исконного населения, а степень распространения культуры полностью определяется способностью его обитателей постигать новые и более совершенные идеи.
\vs p066 6:2 Рабское следование традициям дает стабильность и сотрудничество благодаря сентиментальной связи прошлого с настоящим, но в то же время оно душит инициативу и порабощает творческие силы личности. Весь мир был загнан в тупик связанными с традициями нравами, когда прибыла сотня Калигастии и начала провозглашать новое евангелие индивидуальной инициативы среди социальных групп того времени. Но это благотворное правление было так скоро прервано, что расы никогда полностью не освободились от рабского следования обычаям; мода по\hyp{}прежнему имеет чрезмерное влияние на Урантии.
\vs p066 6:3 Сотня Калигастии --- выходцы из миров\hyp{}обителей Сатании --- хорошо знали искусство и культуру Иерусема, но такое знание было почти бесполезным на варварской планете, населенной примитивными людьми. Эти мудрые существа были осмотрительны, они осознавали, что невозможно осуществить \bibemph{неожиданную} трансформацию или массовый подъем примитивных рас того времени. Они хорошо понимали значение медленной эволюции человеческих видов и мудро воздерживались от любых радикальных попыток изменить образ жизни людей на земле.
\vs p066 6:4 Каждая из десяти планетарных комиссий \bibemph{медленно} и естественно способствовала развитию порученных им проблем. Их задача состояла в том, чтобы привлечь лучшие умы окружающих племен и, после обучения, отправить их назад к их народам как эмиссаров социального подъема.
\vs p066 6:5 Эмиссары никогда не посылались к чужой расе, если только народы специально не просили об этом. Те, кто способствовали духовному подъему и развитию данного племени или расы, всегда были уроженцами этого племени или расы. Сотня не предпринимала попыток внедрять привычки и нравы даже высшей расы в другое племя. Они всегда терпеливо трудились, чтобы развить и улучшить проверенные временем нравы каждой расы. Простые народы Урантии принесли свои социальные обычаи в Даламатию не для того, чтобы поменять их на новые и лучшие привычки, но чтобы возвысить их в общении с более высокой культурой и с высшими умами. Процесс был медленным, но очень эффективным.
\vs p066 6:6 Учителя Даламатии стремились внести сознательный социальный отбор в чисто природную селекцию биологической эволюции. Они не перестраивали человеческое общество, но они заметно ускорили его нормальную и естественную эволюцию. Их побуждением был прогресс путем эволюции, а не революция путем откровения. Человеческой расе потребовались века для обретения той малой религии и морали, которые она имела, а эти сверхлюди хорошо знали, что нельзя, чтобы человечество лишилось и этих немногих достижений из\hyp{}за смятения и страха, которые всегда возникают, когда просветленные и высшие существа стараются возвысить отсталые расы чрезмерным учением и чрезмерным просвещением.
\vs p066 6:7 Когда христианские миссионеры отправляются в сердце Африки, где сыновья и дочери, как правило, остаются под наблюдением и управлением родителей на протяжении всей их жизни, они лишь вносят смятение и разрушают всяческие авторитеты, когда стремятся на протяжении жизни одного поколения искоренить этот обычай, обучая тому, что дети должны быть свободны от всяких родительских ограничений, после того, как достигнут возраста двадцати одного года.
\usection{7. Жизнь в Даламатии}
\vs p066 7:1 Центр Принца, хотя и изысканно красивый и созданный, чтобы внушать благоговение примитивным людям того времени, был в целом скромным. Здания были не особенно большими, поскольку прибывшие учителя стремились поощрять возможное развитие сельского хозяйства путем введения животноводства. Земли, внутри городских стен было достаточно, для того, чтобы пасти скот и разводить сады для снабжения примерно двадцатитысячного населения.
\vs p066 7:2 Интерьеры центрального храма почитания и десяти особняков советов, где заседали руководители\hyp{}сверхлюди, были по\hyp{}настоящему прекрасными творениями искусства. И несмотря на то, что жилые здания были образцами опрятности и чистоты, все было очень простым и в какой\hyp{}то степени примитивным по сравнению с достижениями последующих времен. В этом центре культуры не применялись технологии, которые изначально не были свойственны Урантии.
\vs p066 7:3 Облеченный в плоть штат Принца заведовал простыми и образцовыми жилищами, которые они содержали как дома, предназначенные, чтобы вдохновлять и благоприятно воздействовать на обучающихся наблюдателей, временно проживающих в социальном центре мира, в этом центре образования.
\vs p066 7:4 \pc Определенный уклад семейной жизни и жизни одной семьей в одном жилище, на относительно постоянном месте берет свое начало из этих времен Даламатии и определялся главным образом именно примером и наставлениями сотни и ее учеников. Дом как ячейка общества никогда не имел успеха до тех пор, как сверхмужчины и сверхженщины Даламатии не подвели человечество к тому, чтобы любить и учитывать в своих планах и внуков и детей внуков. Дикий человек любит своего ребенка, но цивилизованный человек любит также своего внука.
\vs p066 7:5 Штат Принца жил вместе как отцы и матери. Правда, они не имели своих собственных детей, но в пятидесяти образцовых домах Даламатии всегда проживало не менее пятисот приемных детей, собранных из высших семейств андонитских и сангикских рас; многие из этих детей были сиротами. Их сверхродители приучали их к дисциплине и воспитывали; и проведя три года в школах Принца (они начинали между тринадцатью и пятнадцатью годами), они созревали для брака и были готовы принять на себя полномочия эмиссаров Принца в испытывающих нужду племенах своих рас.
\vs p066 7:6 \pc Фад руководил планом обучения в Даламатии, который осуществлялся в ремесленных школах, где ученики обучались практической деятельности, благодаря которой учились выполнять повседневные полезные дела. Эта система образования не исключала мышление и эмоции в развитии характера; но первое место в ней отводилось физическому труду. Обучение было индивидуальным и коллективным. Учеников обучали и мужчины, и женщины, и действующие совместно пары. Первая половина обучения в этой группе была раздельной для девочек и мальчиков, вторая --- совместной. Индивидуально их обучали приемам ручного труда а учили жизни в коллективе, объединяя в группы, или классы. В них воспитывали братское отношение к группам более молодых людей, более старших и совсем взрослых, учили работать вместе со своими ровесниками. Их знакомили и с такими объединениями, как семейные группы, игровые команды и школьные классы.
\vs p066 7:7 Среди более поздних групп обучавшихся в Месопотамии для работы со своими расами, были андониты с нагорий западной Индии, и представители красной и голубой рас; позднее было также принято небольшое число представителей желтой расы.
\vs p066 7:8 \pc Хэп дал этим древним расам моральный закон. Этот кодекс был известен как «Путь Отца» и состоял из следующих семи наставлений:
\vs p066 7:9 \ublistelem{1.}\bibnobreakspace Вы не будете бояться или служить иному Богу, кроме Отца Всего Сущего.
\vs p066 7:10 \ublistelem{2.}\bibnobreakspace Вы не будете оказывать неповиновение Сыну Отца, правителю мира, и проявлять неуважение к его сверхчеловеческим сподвижникам.
\vs p066 7:11 \ublistelem{3.}\bibnobreakspace Вы не будете говорить неправду перед судьями людей.
\vs p066 7:12 \ublistelem{4.}\bibnobreakspace Вы не будете убивать мужчин, женщин или детей.
\vs p066 7:13 \ublistelem{5.}\bibnobreakspace Вы не будете воровать добро или скот вашего соседа.
\vs p066 7:14 \ublistelem{6.}\bibnobreakspace Вы не будете прикасаться к жене вашего друга.
\vs p066 7:15 \ublistelem{7.}\bibnobreakspace Вы не будете проявлять неуважение к вашим родителям или к старцам племени.
\vs p066 7:16 \pc Он был законом Даламатии почти триста тысяч лет. И многие из камней, на которых был начертан этот закон, сейчас лежат под водами у берегов Месопотамии и Персии. Вошло в привычку каждый день недели повторять наизусть одно из этих наставлений, и как приветствие, и как благодарственную молитву во время еды.
\vs p066 7:17 \pc Время в те дни измерялось лунным месяцем, этот период охватывал двадцать восемь дней. За исключением дня и ночи это было единственное исчисление времени, известное древним людям. Семидневная неделя была введена учителями Даламатии и происхождение ее связано с тем, что семь было одной четвертью двадцати восьми. Значение числа семь в сверхвселенной, без сомнения, предоставило им возможность ввести напоминание о духовном в обычное исчисление времени. Но недельный период не имеет естественного происхождения.
\vs p066 7:18 \pc Местность вокруг города в радиусе ста миль была довольно плотно населена. Непосредственно в окрестностях города сотни выпускников школ Принца были заняты животноводством или проводили в жизнь то, чему их научили штат и их многочисленные человеческие помощники. Немногие занимались земледелием и садоводством.
\vs p066 7:19 Человечество не было приговорено к тяжелому земледельческому труду в качестве наказания за предполагаемый грех. «В поте лица своего будешь ты добывать хлеб свой насущный» не было наказанием, объявленным человеку за его участие в безумиях бунта Люцифера под руководством вероломного Калигастии. Обработка почвы присуща процессу становления развивающейся цивилизации в эволюционирующих мирах, и это была основная установка всего учения Планетарного Принца и его штата на протяжении трехсот тысяч лет, которые прошли между их прибытием на Урантию и теми трагическими днями, когда Калигастия связал свою судьбу с бунтовщиком Люцифером. Обработка земли --- это не проклятие; скорее, это высшее благоволение для всех, кому таким образом разрешено наслаждаться самой человечной из всех человеческих профессий.
\vs p066 7:20 К началу бунта в Даламатии было почти шесть тысяч постоянного населения. Сюда входили и постоянные учащиеся, но не посетители и наблюдатели, число которых всегда превышало тысячу. Вы можете получить лишь очень приблизительное представление об изумительном прогрессе тех отдаленных времен; почти все эти замечательные достижения человечества тех дней были стерты ужасным хаосом и крайней духовной темнотой, которые последовали за катастрофой обмана и призыва Калигастии к бунту.
\usection{8. Несчастья Калигастии}
\vs p066 8:1 Оглядываясь назад на долгую карьеру Калигастии, мы находим лишь одну выдающуюся черту в его поведении, которая могла бы привлечь внимание; он был чрезмерным индивидуалистом. Он был склонен принимать участие почти в каждой партии протеста и обычно симпатизировал тем, кто хотя бы даже в очень мягкой форме выражал свое критическое отношение. Мы обнаружили, что он, если находился под чьим\hyp{}нибудь руководством, почти сразу же начинал проявлять беспокойство и в определенной мере отвергать все формы контроля. Немного обижаясь на советы вышестоящих и отчасти проявляя своенравие перед руководством свыше, он, тем не менее, при проверках всегда был лояльным по отношению к правителям вселенной и послушен указам Отцов Созвездия. За ним не было замечено никакой реальной вины до его постыдного предательства Урантии.
\vs p066 8:2 Следует отметить, что и Люцифера и Калигастию любовно предупреждали и терпеливо указывали на их критические тенденции, постепенное усугубление их гордыни и связанного с ней преувеличенного чувства собственного достоинства. Но все эти попытки помочь были неверно истолкованы как незаслуженная критика и неоправданное вмешательство в личную свободу. И Калигастия, и Люцифер сочли, что их дружественно настроенные руководители действуют, исходя из очень предосудительных мотивов, хотя именно эти мотивы начинали преобладать в их собственных болезненных мыслях и ошибочном планировании. Они судили о своих бескорыстных руководителях с позиции своего собственного усугубляющегося эгоизма.
\vs p066 8:3 \pc С момента прибытия принца Калигастии планетарная цивилизация развивалась вполне обычным образом почти триста тысяч лет. Несмотря на то, что Урантия была планетой с разновидностями жизни и в силу этого объектом многочисленных отклонений и необычных колебаний эволюции, она весьма удовлетворительно двигалась по планетарному пути до момента бунта Люцифера и одновременного предательства Калигастии. И эта катастрофическая ошибка, и более поздний провал Адама и Евы в выполнении их планетарной миссии резко изменили всю последующую историю.
\vs p066 8:4 Во время восстания Люцифера Принц Урантии участвовал в злодеянии, ввергнув таким образом планету в состояние длительного смятения. Вслед за этим он был лишен суверенной власти благодаря совместным действиям правителей созвездий и других властителей вселенной. Он разделил неизбежные превратности судьбы изолированной Урантии до времени прибытия Адама на планету и в какой\hyp{}то мере способствовал провалу плана развития смертных рас, который должен был быть реализован благодаря притоку жизненной крови новой фиолетовой расы --- потомков Адама и Евы.
\vs p066 8:5 Способность павшего Принца вмешиваться в дела людей сильно уменьшилась благодаря пришествию на планету, в облик смертного, Махивенты Мелхиседека в дни Авраама; и впоследствии, во время жизни Михаила во плоти, совершивший предательство Принц был окончательно лишен всей власти на Урантии.
\vs p066 8:6 \pc Доктрина личностного дьявола на Урантии, хотя и исходит в какой\hyp{}то мере из планетарного присутствия предательского и порочного Калигастии, тем не менее, абсолютно несостоятельна в тех постулатах, где утверждается, что «дьявол» может влиять на нормальный человеческий разум вопреки его свободному и естественному выбору. Даже до пришествия Михаила на Урантию ни Калигастия, ни Далигастия никогда не были способны притеснять смертных или принудить любую нормальную личность совершить что\hyp{}либо против человеческой воли. Свободная человеческая воля является верховной в моральных вопросах; даже пребывающий Настройщик Мысли отказывается принуждать человека хоть к одной мысли или хоть одному делу против выбора собственной воли человека.
\vs p066 8:7 И теперь этот бунтовщик этого мира, лишенный хоть какой\hyp{}либо возможности навредить своим прежним подчиненным, ожидает окончательного вынесения решения Древних Дней Уверсы относительно всех, кто принимал участие в восстании Люцифера.
\vsetoff
\vs p066 8:8 [Представлено Мелхиседеком Небадона.]
