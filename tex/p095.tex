\upaper{95}{Учение Мелхиседека в Леванте}
\vs p095 0:1 Как Индия дала начало многим религиям и философиям восточной Азии, так и Левант был родиной вероисповеданий Западного мира. Салимские миссионеры распространились по всей юго\hyp{}западной Азии, в Палестине, Месопотамии, Египте, Иране и Аравии, повсюду возвещая благую весть --- евангелие Махивенты Мелхиседека. В некоторых из этих стран их проповедь принесла плоды; в других --- их принимали с переменным успехом. Иногда их неудачи происходили от недостатка мудрости, иногда --- вследствие обстоятельств, над которыми они были не властны.
\usection{1. Религия Салима в Месопотамии}
\vs p095 1:1 К 2000 году до н.э. религии Месопотамии были на грани утраты учения Сифитов и в большой степени находились под влиянием первобытных верований двух вторгшихся групп: бедуинов\hyp{}семитов, которые просачивались в страну через западную пустыню, и варваров\hyp{}всадников, пришедших с севера.
\vs p095 1:2 Однако обычай древних Адамических народов чтить седьмой день недели никогда полностью не исчезал в Месопотамии. Только во времена Мелхиседека седьмой день рассматривался как самый несчастливый день. На него падало множество запретов: в этот дурной день запрещалось отправляться в путешествие, готовить пищу, разводить огонь. Евреи принесли в Палестину множество запретов Месопотамии, которые они связывали с характерным для Вавилона соблюдением седьмого дня, Шабаттума.
\vs p095 1:3 Хотя салимские учителя много сделали для того, чтобы облагородить и возвысить религии Месопотамии, им не удалось привести различные народы к неизменному признанию единого Бога. Более полутораста лет такие учения доминировали, а затем они постепенно уступили место более древней вере во множество богов.
\vs p095 1:4 Салимские учителя значительно уменьшили число богов Месопотамии, причем число главных богов было в какой\hyp{}то момент сведено к семи, это были: Бел, Шамаш, Набу, Ану, Эа, Мардук и Син. Во время наибольшего успеха нового учения они провозгласили трех из этих богов высшими по отношению ко всем остальным, это была вавилонская триада: Бел, Еа и Ану, боги земли, моря и неба. Тем не менее, в различных областях выдвигались другие триады, причем все они напоминали учение Андитов и Шумеров о троице и основывались на вере салимитов в знак Мелхиседека, изображающий три круга.
\vs p095 1:5 Салимские учителя никогда не могли полностью побороть популярность Иштар, матери богов, олицетворявшей дух сексуального плодородия. Они многое сделали, чтобы облагородить поклонение этой богине, но вавилоняне и их соседи так никогда и не смогли полностью освободиться от своих завуалированных форм сексуализированного поклонения. Во всей Месопотамии был распространен обычай, когда каждая молодая женщина, по крайней мере, один раз в своей жизни, отдавалась объятиям незнакомца; это считали знаком преданности, которого требует Иштар, и полагали, что плодородие в значительной степени зависит от этой сексуальной жертвы.
\vs p095 1:6 \P\ Первоначально учение Мелхиседека устанавливалось вполне успешно, пока Набодад, руководитель школы в Кише, не решил энергично атаковать широко распространенную практику храмового разврата. Но салимские миссионеры потерпели неудачу в своем стремлении осуществить эту социальную реформу, и крах этой попытки привел к провалу всех их более важных духовных и философских учений.
\vs p095 1:7 Это поражение салимского евангелия немедленно повлекло за собой интенсивный рост культа Иштар, который уже проник в Палестину под именем Ашторет, в Египет под именем Изиды, в Грецию --- Афродиты и в северные племена под именем Астарты. И именно в связи с этим возрождением поклонения Иштар вавилонские священники снова обратились к созерцанию звезд; астрология Месопотамии испытала свой последний великий подъем, в моду вошло предсказывание судьбы, и в течение столетий духовенство вырождалось все больше и больше.
\vs p095 1:8 Мелхиседек предупреждал своих последователей, что следует учить о существовании единого Бога, Отца и Создателя всего сущего, проповедовать только евангелие о приобретении божественного благоволения через одну лишь веру. Но ошибка учителей новой истины часто состоит в том, что они пытаются сделать слишком много, пытаются подменить медленный эволюционный процесс стремительной революцией. В Месопотамии миссионеры Мелхиседека предъявляли слишком высокие для живущих там людей требования к уровню нравственности; они попытались сделать слишком много, и их благородное дело потерпело поражение. Им было поручено проповедовать некое определенное евангелие, возвещать истину о реальности Отца Всего Сущего; но они оказались вовлеченными в казавшееся важным дело реформирования нравов и это увело их великую миссию с правильного пути и, в конце концов, привело ее к краху и забвению.
\vs p095 1:9 Салимский центр в Кише прекратил свое существование за время жизни одного поколения, и пропаганда веры в единого Бога угасла практически по всей Месопотамии. Но остатки салимских школ продолжали существовать. Разбросанные там и сям небольшие группы продолжали сохранять свою веру в единого Творца и вести борьбу против идолопоклонства и аморальности месопотамских священников.
\vs p095 1:10 \P\ Именно салимские миссионеры в период, следующий за отказом от их учения, написали многие псалмы Ветхого Завета, начертав их на камне, который иудейские священники более поздних времен нашли во время вавилонского пленения и впоследствии включили их в собрание гимнов, авторство которых приписывалось евреям. Эти великолепные псалмы из Вавилона не были написаны в храмах Бела\hyp{}Мардука, они были созданием потомков более древних салимских миссионеров, и они разительно контрастируют с разнородным набором магических текстов вавилонских священников. Книга Иова достаточно хорошо отражает учение салимской школы в Кише, да и во всей Месопотамии.
\vs p095 1:11 Многое из религиозной культуры Месопотамии пришло в древнееврейскую литературу и древнееврейский ритуал богослужения из Египта, благодаря трудам Аменемопа и Эхнатона. Египтяне прекрасно сохранили учения о социальных обязанностях, восходящие к более древним обитателям Месопотамии --- Андитам, и как много из этого было утеряно вавилонянами, занявшими позднее долину Евфрата.
\usection{2. Древняя египетская религия}
\vs p095 2:1 Оригинальные учения Мелхиседека, в действительности, наиболее глубоко укоренились в Египте, откуда впоследствии достигли Европы. Эволюционирующая религия нильской долины периодически обогащалась в результате прихода высших родов Нодитов, Адамитов и более поздних Андических народов долины Евфрата. Время от времени многие египетские гражданские руководители были шумерами. Как Индия в это время была прибежищем для самой разнообразной смеси рас мира, так и Египет взрастил тип религиозной философии, в котором удивительно перемешались всяческие представления, которые только можно было найти на Урантии, и из долины Нила эта философия проникла во многие части мира. Евреи в своем представлении о сотворении мира много позаимствовали от вавилонян, но свое понятие о божественном Провидении они взяли у египтян.
\vs p095 2:2 Скорее, благодаря политическим и моральным тенденциям, чем философским и религиозным, Египет оказался более благосклонным к салимскому учению, чем Месопотамия. В Египте вождь каждого племени после того, как завоевывал себе трон, стремился сохранить свою династию, провозглашая своего племенного бога изначальным божеством и создателем всех других богов. Так египтяне постепенно привыкли к мысли о существовании сверхбога, что было первым шагом на пути к более поздней доктрине Божества --- создателя всего сущего. В Египте идея монотеизма в течение многих столетий то усиливалась, то ослабевала, причем вере в единого Бога всегда отдавалось предпочтение, но она никогда полностью не могла взять верх над развивающимися представлениями политеизма.
\vs p095 2:3 Веками народы Египта поклонялись богам природы; точнее, каждое из нескольких десятков племен имело своего особого бога --- одно поклонялось быку, другое --- льву, третье --- барану и так далее. Однако раньше у племен были тотемы, точно так же, как у туземцев Америки.
\vs p095 2:4 \P\ В какой\hyp{}то момент египтяне заметили, что мертвые тела, помещенные в гробницы, сделанные не из кирпича, сохраняются долго --- бальзамируются --- благодаря воздействию песка, насыщенного содой, в то время как тела, погребенные в кирпичных склепах, разлагаются. Эти наблюдения привели к опытам, в результате которых появился обычай бальзамировать мертвых. Египтяне верили, что сохранность тела помогает его переходу в будущую жизнь. Чтобы в отдаленном будущем можно было бы правильно установить личность человека после распада тела, они помещали в могиле вместе с телом погребальную статую --- изображение умершего, вырезая его на крышке гроба. Изготовление этих погребальных статуй привело к значительному совершенствованию египетского искусства.
\vs p095 2:5 Веками египтяне верили, что гробницы являются гарантией сохранности тела и его последующей благоприятной жизни в посмертии. Хотя магические обряды, сопровождавшие жизнь человека от колыбели до могилы, были обременительными, их дальнейшая эволюция в высшей степени эффективно способствовала избавлению египтян от религии гробниц. Священники вырезали на гробах заклинания, которые, как полагали, служили гарантией того, что «в потустороннем мире сердце человека не будет у него отнято». Вскоре различные варианты этих магических текстов были собраны и хранились под названием Книги Мертвых. Но в долине Нила магический ритуал давно стал вторгаться в области, связанные с совестью и характером человека, причем в такой степени, какая не часто достигалась в ритуалах того времени. И впоследствии люди в своих надеждах на спасение полагались скорее на эти моральные и этические идеалы, чем на тщательное убранство гробниц.
\vs p095 2:6 \P\ Прекрасный пример суеверия того времени --- всеобщая вера в действенность слюны как лекарства, представление, которое родилось в Египте и распространилось оттуда в Аравию и Месопотамию. В легендарном сражении Гора и Сета юный бог потерял глаз, но после того, как Сет был побежден, мудрый бог Тот вернул этот глаз, плюнув на рану и излечив ее.
\vs p095 2:7 \P\ Египтяне долгое время верили, что мерцание звезд на ночном небе означает спасение душ избранных умерших; остальные люди, как они думали, в посмертии поглощаются солнцем. В определенный период времени поклонение солнцу стало разновидностью почитания предков. Наклонный входной коридор великой пирамиды был направлен непосредственно на Полярную звезду, так чтобы душа царя, поднявшись из гробницы, могла идти прямо к неизменным и вечным созвездиям неподвижных звезд, предполагаемой обители царей.
\vs p095 2:8 Когда косые лучи солнца проникали на землю через толщу облаков, верующие были убеждены, что это спускается небесная лестница, по которой душа царя или других праведников может вознестись. «Царь Пепи опустил луч свой как лестницу под ноги свои, чтобы подняться по ней к матери своей».
\vs p095 2:9 Когда явился во плоти Мелхиседек, у египтян была религия, значительно превосходящая религии окружающих народов. Они верили, что лишенная телесной оболочки душа, если только она должным образом защищена магическими заклинаниями, может противостоять вмешательству духов зла и найти дорогу к судилищу Озириса, где, «если она не виновна в убийстве, грабеже, вероломстве, прелюбодействе, воровстве и эгоизме», будет принята в царство блаженства. Если же душа была взвешена на весах и обнаружился недостаток, она будет отправлена в ад к Пожирательнице. И это было относительно прогрессивное представление о будущей жизни по сравнению с верованиями многих соседних племен.
\vs p095 2:10 Представление о грядущем суде за грехи плотской жизни на земле было перенесено в иудейскую теологию из Египта. Слово суд появляется во всей Псалтири только один раз, и этот самый псалом был написан египтянином.
\usection{3. Эволюция понятий морали}
\vs p095 3:1 Хотя культура и религия Египта была унаследована, главным образом, от Андитов Месопотамии и была, в значительной степени, передана последующим цивилизациям через посредство иудеев и греков, все\hyp{}таки много, очень много черт социального и этического идеализма египтян в долине Нила появилось в результате чисто эволюционного процесса развития. Несмотря на то, что во многом истины и культура были заимствованы у Андитов, в Египте нравственная культура эволюционировала как результат чисто человеческого развития в большей степени, чем это происходило вследствие подобного естественного процесса в других областях до пришествия Михаила.
\vs p095 3:2 Эволюция морали зависит не только от откровения. Идеи высокой морали могут проистекать и из человеческого опыта. Человек может даже развить представление о духовных ценностях и обрести космическое прозрение из опыта своей личной жизни, потому что в нем пребывает божественный дух. Такая естественная эволюция совести и характера усиливалась также в результате периодического прибытия учителей истины, в древние времена --- из второго Эдема, а позднее --- из центра Мелхиседека в Салиме.
\vs p095 3:3 За тысячи лет до того, как салимское евангелие проникло в Египет, его нравственные вожди учили справедливости и чистоте, учили остерегаться жадности. За три тысячи лет до того, как было написано священное писание иудеев, девиз египтян гласил: «Замечателен тот человек, чьим образцом является праведность, кто идет по ее пути». Они учили доброте, выдержке и благоразумию. Послание одного из великих учителей той эпохи гласило: «Поступайте правильно и справедливо со всеми». Правда\hyp{}Справедливость\hyp{}Праведность были египетской триадой того времени. Среди всех человеческих религий Урантии ни одна никогда не превзошла общественные идеалы и моральное величие былого гуманизма нильской долины.
\vs p095 3:4 На почве этих развивающихся этических представлений и нравственных идеалов расцвели уцелевшие учения салимской религии. Понятия добра и зла нашли немедленный отклик в сердцах людей, которые верили, что «Жизнь дается миролюбивому, а смерть --- виновному». «Миролюбивый делает то, что вызывает любовь; виновный делает то, что вызывает ненависть». Веками обитатели нильской долины жили в соответствии с этими этическими и социальными нормами даже прежде того, как они восприняли более поздние идеи правого и неправого --- хорошего и дурного.
\vs p095 3:5 \P\ Египет отличался интеллектуальностью и нравственными качествами, но не чрезмерной духовностью. За шесть тысяч лет среди египтян появилось только четыре великих пророка. В течение некоторого времени египтяне слушались Аменемопа; Охбана они убили; без особого энтузиазма они приняли Эхнатона и то --- на краткий период жизни одного поколения; Моисея они отвергли. И опять же скорее политические, чем религиозные, факторы привели к тому, что Аврааму, а впоследствии --- Иосифу было легко сделать, чтобы салимское учение о едином Боге стало пользоваться большим влиянием во всем Египте. Но когда салимские миссионеры впервые вступили в Египет, они столкнулись с тем, что эта высокая этическая культура эволюции сочетается с модифицированными моральными нормами иммигрантов из Месопотамии. Эти древние учителя нильской долины первыми провозгласили, что совесть есть наказ Бога, голос Божества.
\usection{4. Учение Аменемопа}
\vs p095 4:1 В свое время в Египте появился учитель, которого многие называли «сыном человеческим», а другие --- Аменемопом. Этот провидец превозносил совесть как наивысший критерий правильного и неправильного, учил о наказании за грехи и провозглашал спасение, достижимое в результате обращения к солнечному божеству.
\vs p095 4:2 Аменемоп учил, что богатство и удача есть дар Бога, и это представление наложило глубокий отпечаток на появившуюся позднее иудейскую философию. Этот благородный учитель верил, что сознание присутствия Бога во всех ситуациях является определяющим фактором поведения всего сущего; что каждый момент нужно переживать, осознавая присутствие Бога и ответственность перед ним. Учения этого мудреца были впоследствии переведены на древнееврейский и стали священной книгой этого народа задолго до того, как был написан Ветхий Завет. Главный наказ этого мужа касается наставлений своему сыну, в нем он учит его быть честным, справедливым и беспристрастным на постах, которые доверяет ему государство, и эти благородные чувства далекого прошлого сделали бы честь любому современному государственному деятелю.
\vs p095 4:3 Этот мудрый человек с берегов Нила учил, что «богатство приделывает себе крылья и улетает», т.е. что все земное мимолетно. Замечательной его молитвой была молитва о «спасении от страха». Он призывал всех от «слов человеческих» обратиться к «делам Божиим». По сути, он учил, что Человек предполагает, а Бог располагает. Его учения, переведенные на древнееврейский, определили философию Книги Притчей Соломоновых в Ветхом Завете. Переведенные на греческий, они наложили отпечаток на всю последующую эллинистическую религиозную философию. Экземпляр Книги Мудрости был у более позднего александрийского философа Филона.
\vs p095 4:4 Аменемоп пытался сохранить этику эволюции и мораль откровения, и в своих сочинениях он передал их другим --- и иудеям, и грекам. Он не был величайшим религиозным учителем своего времени, но был самым влиятельным в том смысле, что он наложил отпечаток на последующую мысль двух жизненно важных звеньев в развитии Западной цивилизации --- на иудеев, в среде которых достигла своей высшей точки Западная религиозная вера, и на греков, философия которых получила в Европе наивысшее развитие.
\vs p095 4:5 \P\ В Книге Притчей Соломоновых, главы пятнадцатая, семнадцатая, двадцатая и с главы двадцать второй, стих семнадцатый до главы двадцать четвертой, стих двадцать второй взяты почти дословно из Книги Мудрости Аменемопа. Первый псалом Псалтири был написан Аменемопом, и в нем представлена суть будущего учения Эхнатона.
\usection{5. Удивительный Эхнатон}
\vs p095 5:1 Учение Аменемопа постепенно утрачивало свое влияние на умы египтян, в то время как под влиянием египетского врача\hyp{}салимита женщина из царского рода уверовала в учение Мелхиседека. Эта женщина убедила своего сына Эхнатона, фараона Египта, принять учение о едином Боге.
\vs p095 5:2 Со времени исчезновения Мелхиседека во плоти человека вплоть до этого периода ни одно человеческое существо не имело столь ясного представления о салимской религии откровения, как Эхнатон. В некоторых отношениях этот юный египетский царь был одним из самых удивительных личностей в человеческой истории. Во время все нарастающего духовного упадка в Месопотамии, в Египте он сохранял живым учение об Эл Элионе, едином Боге, тем самым поддерживая источник монотеистической философии, столь жизненно важный для религиозной подготовки будущего пришествия Михаила. И в признание этого подвига --- среди других причин --- младенец Иисус был увезен в Египет, где некоторые духовные последователи Эхнатона увидели его и в какой\hyp{}то мере осознали определенные аспекты его божественной миссии на Урантии.
\vs p095 5:3 Моисей, величайшая фигура между Мелхиседеком и Иисусом, был для мира общим даром иудейского народа и египетского царского рода; и если бы Эхнатон обладал многосторонностью и талантом Моисея, если бы его политический гений был подстать его удивительному дару религиозного лидерства, тогда Египет стал бы великой монотеистической нацией своего времени; и если бы это случилось, вполне вероятно, что Иисус большую часть своей смертной жизни прожил бы в Египте.
\vs p095 5:4 Никогда во всей истории ни один царь не проводил столь методично процесс обращения целой нации из политеизма в монотеизм, как это делал этот удивительный Эхнатон. С совершенно поразительной решительностью этот юный правитель порвал с прошлым, изменил свое имя, покинул свою столицу, построил совершенно новый город и создал для всего народа новое искусство и новую литературу. Но он шел слишком быстро; он строил слишком много, больше того, что могло устоять после его смерти. Кроме того, ему не удалось обеспечить материальную стабильность и процветание своего народа, и все это неблагоприятным образом сказалось на судьбе его религиозного учения, когда впоследствии потоки несчастья и гнета нахлынули на египтян.
\vs p095 5:5 Если бы этот человек поразительного дара отчетливого ясного видения и необычайной целеустремленности обладал политической прозорливостью Моисея, он мог бы изменить всю историю эволюции религии и откровения истины в Западном мире. Во время своей жизни он смог обуздать активность священников, доверие к которым он сильно подорвал, но те продолжали исповедовать свой культ втайне, и, как только молодой царь отошел от власти, тотчас начали действовать и не замедлили свалить все дальнейшие беды Египта на установление монотеизма во время его царствования.
\vs p095 5:6 Эхнатон очень мудро пытался внедрить монотеизм под маской поклонения богу\hyp{}солнцу. К этому решению прийти к поклонению Отцу Всего Сущего, заменив культ всех богов единым культом поклонения солнцу, его подвигнул совет врача\hyp{}салимита. Эхнатон принял распространенную доктрину существовавшей тогда веры в Атона относительно Божественного отцовства и материнства и создал религию, которая признавала глубоко личную сокровенную связь между Богом и человеком, его почитающим.
\vs p095 5:7 Эхнатон был достаточно мудр, чтобы поддерживать внешнюю сторону поклонения Атону, богу\hyp{}солнцу, в то же время он привел своих собратьев к завуалированному поклонению Единому Богу, создателю Атона и верховного Отца всего сущего. Этот молодой царь\hyp{}учитель был плодовитым писателем, автором повествования под названием «Единый Бог», книги из тридцати одной главы, которую священники полностью уничтожили, когда вернулись к власти. Эхнатон написал также сто тридцать семь гимнов, двенадцать из которых сохранились в Псалтири Ветхого Завета, авторство которой приписывается иудеям.
\vs p095 5:8 \P\ Верховным словом религии Эхнатона в повседневной жизни было слово «праведность», и он быстро распространил понятие правого деяния на область как международной, так и национальной этики. Это было поколение поразительного личного благочестия, в котором наиболее умные мужчины и женщины отличались истинным стремлением найти Бога и познать его. В то время социальное положение или богатство не давали ни одному египтянину никакого преимущества перед лицом закона. Семейная жизнь в Египте делала многое, чтобы сохранить и приумножить нравственную культуру, она была вдохновляющим образцом для замечательной семейной жизни евреев, проживающих в Палестине в более поздние времена.
\vs p095 5:9 Роковой слабостью евангелия Эхнатона была его величайшая истина, учение, что Атон есть не только создатель Египта, но также и «всего мира, людей и животных, всех чужеземных стран, даже Сирии и Куша, помимо этой страны --- Египта. Он ставит все на свои места и дает всем им необходимое». Это представление о Божестве было высоким и возвышенным, но в нем отсутствовал элемент национализма. Подобные проявления интернационализма в религии не могли повысить моральный дух египетской армии на поле сражений, но были эффективным оружием в руках священников против молодого царя и его новой религии. Он обладал понятием Бога, далеко превосходящим это понятие у позднейших иудеев, но оно было слишком прогрессивным, чтобы служить целям создателя нации.
\vs p095 5:10 \P\ Хотя монотеистический идеал померк с уходом Эхнатона, идея единого Бога продолжала существовать в умах многих групп. Зять Эхнатона согласился со священниками и вернулся к поклонению старым богам, изменив имя на Тутанхамона. Столица возвратилась в Фивы, и священники наживались на земле, получив в свое распоряжение, в конечном счете, одну седьмую всего Египта; и вскоре один из этого самого ордена священников осмелился захватить корону.
\vs p095 5:11 Но священники не могли полностью побороть волну монотеизма. Все больше и больше они были вынуждены объединять своих богов и принимать нескольких богов за одного, имеющего несколько имен; семья богов все больше сокращалась. Эхнатон связывал пламенеющий на небе диск с Богом\hyp{}творцом, и эта идея продолжала вспыхивать в сердцах многих людей, даже священников, долгое время после того, как умер молодой реформатор. Никогда в Египте и в мире идея монотеизма полностью не умерла в сердцах людей. Она просуществовала даже до прибытия Творца\hyp{}Сына того самого божественного Отца, единого Бога, поклоняться которому Эхнатон так ревностно призывал весь Египет.
\vs p095 5:12 Слабость доктрины Эхнатона состояла в том, что он провозгласил слишком передовую религию, так что только образованные египтяне могли всецело воспринять его учение. Рядовые сельские труженики никогда в действительности не понимали его евангелия и, следовательно, были готовы возвратиться вместе со священниками к прежнему почитанию Изиды и ее супруга Озириса, который, как полагали, чудесным образом воскрес после мучительной смерти от рук Сета, бога тьмы и зла.
\vs p095 5:13 Учение о бессмертии для всех людей было слишком прогрессивным для египтян. Воскресение было обещано только царям и богатым; поэтому они так тщательно бальзамировали и хранили в гробницах свои тела до судного дня. Но демократичность спасения и воскресения, как учил тому Эхнатон, в конце концов победила, и даже дошло до того, что впоследствии египтяне верили даже в жизнь после смерти бессловесных животных.
\vs p095 5:14 \P\ Хотя попытка этого египетского правителя навязать своему народу поклонение единому Богу потерпела поражение, следует отметить, что отзвуки его деятельности раздавались в течение столетий и в Палестине, и в Греции и Египет, таким образом, стал посредником для передачи эволюционирующей культуры Нила, соединенной с религией откровения Евфрата, всем последующим народам Запада.
\vs p095 5:15 Слава этой великой эры развития морали и духовного роста в нильской долине быстро угасала приблизительно в то время, когда складывалась иудейская нация, и после пребывания евреев в Египте эти бедуины унесли с собой многое из этих учений и сохранили многое из доктрин Эхнатона в религии своего народа.
\usection{6. Салимские доктрины в Иране}
\vs p095 6:1 Некоторые миссионеры Мелхиседека прошли через Месопотамию к великому Иранскому плоскогорью. Более пятисот лет салимские учителя преуспевали в Иране, и вся нация склонялась к тому, чтобы принять религию Мелхиседека, когда смена правителей внезапно вызвала ожесточенное их преследование, которое, практически, положило конец монотеистическим учениям салимского культа. Учение о завете Авраама, в сущности, вымерло в Персии, когда в то великое столетие морального возрождения, в шестом веке до Рождества Христа, появился Зороастр, чтобы возродить едва теплящиеся огоньки салимского евангелия.
\vs p095 6:2 Этот основатель новой религии был сильным и смелым юношей, который в своем первом путешествии в Ур в Месопотамии познакомился с преданиями о бунте Калигастии и Люцифера --- наряду со многими другими преданиями, --- которые все произвели сильное впечатление на его религиозную натуру. Соответственно, увидев в Уре сон, он составил план возвращения в свой северный дом, чтобы предпринять преобразование религии своего народа. Он усвоил иудейскую идею Бога справедливости, моисеево понятие божественности. В его сознании было ясное представление о верховном Боге, и он отбросил всех других богов как бесов, низведя их до положения демонов, о которых он слышал в Месопотамии. Он знал историю о Семи Духах\hyp{}Мастерах по преданию, существующему в Уре, и, соответственно, он создал плеяду семи верховных богов во главе с Ахурамаздой. Этих низших богов он представил как идеальные образы Справедливого Закона, Благой Мысли, Благородного Правления, Святого Характера, Здоровья и Бессмертия.
\vs p095 6:3 И эта новая религия была религией действия, работы, а не молитв и ритуалов. Ее Богом было существо верховной мудрости, покровитель цивилизации; это была воинствующая религиозная философия, которая отваживалась сражаться со злом, бездействием и отсталостью.
\vs p095 6:4 Зороастр не учил поклоняться огню, но стремился использовать пламя как символ чистого и мудрого Духа всемирного верховного господства. (Хотя правда и то, что более поздние его последователи и почитали, и поклонялись этому символическому огню.) В конце концов, после того, как иранский принц был обращен в эту новую религию, она распространялась при помощи меча. И Зороастр героически погиб в бою за то, что было, как он верил, «истиной Господа света».
\vs p095 6:5 \P\ Зороастризм --- единственная урантийская вера, которая сохранила учение Даламатии и Эдема о Семи Духах\hyp{}Мастерах. Хотя зороастризму и не удалось развить представление о Троице, он, в определенном смысле, приблизился к представлению о Семеричном Боге. Первоначальный зороастризм не был чистым дуализмом; хотя ранние учения и изображали зло как неразрывно связанное во времени с добром, с позиции вечности, оно, безусловно, было погружено в предельную реальность добра. Только в более поздние времена укрепилась вера, что добро и зло борются на равных началах.
\vs p095 6:6 Еврейские предания о небесах и аде, а также учение о дьяволе в том виде, как оно было представлено в иудейских священных писаниях, хотя и было основано на существовавших преданиях о Люцифере и Калигастии, проистекало, главным образом, из зороастризма во времена, когда евреи находились под политическим и культурным господством персов. Зороастр, как и египтяне, учил о «судном дне», но он связывал это событие с концом мира.
\vs p095 6:7 Зороастризм заметно повлиял даже на те религии в Персии, которые появились после него. Когда иранские священники пытались уничтожить учение Зороастра, они возродили древний культ Митры. И этот культ Митры распространился по всему Леванту и областям Средиземноморья, оказавшись в течение некоторого времени современником и иудаизма и христианства. Таким образом, учение Зороастра последовательно наложило отпечаток на три великие религии: на иудаизм, христианство, а через них --- и на магометанство.
\vs p095 6:8 \P\ Но слишком далеко отстоят друг от друга возвышенные учения и благородные псалмы Зороастра и современные извращения его благовествований парсами с их вечным страхом перед мертвыми, соединенным с постоянной склонностью к софизмам, тогда как Зороастр никогда не снисходил до того, чтобы поощрять софизмы.
\vs p095 6:9 Этот великий человек был одним из тех уникальных личностей, появившихся в шестом столетии до Рождества Христа для того, чтобы спасти свет Салима от полного и окончательного уничтожения. Как бы слабо он ни горел, но он указывал человеку в его мире, окутанном мраком, путь света, ведущий к вечной жизни.
\usection{7. Салимское учение в Аравии}
\vs p095 7:1 Учение Мелхиседека о едином Боге установилось в Аравийской пустыне сравнительно недавно. Как в Греции, так и в Аравии, салимские миссионеры потерпели неудачу из\hyp{}за того, что неправильно поняли наставления Мелхиседека относительно чрезмерных организационных мер. Но они полагали, что его предостережение, направленное против любых попыток распространять евангелие при помощи военной силы и гражданского принуждения, их не касается.
\vs p095 7:2 Ни даже в Китае, ни в Риме учение Мелхиседека не потерпело столь полного поражения, как в этой пустынной области, расположенной так близко от самого Салима. Спустя долгое время после того, как большинство народов Востока и Запада стали, соответственно, буддистами и христианами, Аравия продолжала находится в том же состоянии, что и в течение предыдущих тысячелетий. Каждое племя поклонялось своему древнему фетишу, и многие отдельные роды имели своих домашних богов. Продолжалась долгая борьба между вавилонской Иштар, иудейским Яхве, иранским Ахурой и христианским Отцом Господа Иисуса Христа. Но ни одно из понятий не было способно полностью вытеснить остальные.
\vs p095 7:3 По всей Аравии тут и там существовали роды и кланы, которые придерживались смутного представления о едином Боге. Такие группы бережно хранили традиции Мелхиседека, Авраама, Моисея и Зороастра. Существовали многочисленные центры, которые могли бы откликнуться на евангелие Иисуса, но христианские миссионеры в пустынных землях представляли собой группу суровых и непримиримых людей в противоположность способным к компромиссам и нововведениям миссионерам, действующим в странах Средиземноморья. Если бы последователи Иисуса более серьезно отнеслись к его призыву «идти во все страны и проповедовать евангелие» и если бы они были более милосердны в этой проповеднической деятельности, менее строги во второстепенных социальных требованиях, которые они сами же и вводили, тогда многие страны, и среди них --- Аравия, с радостью приняли бы простое евангелие сына плотника.
\vs p095 7:4 Несмотря на тот факт, что великий левантийский монотеизм не смог пустить корни в Аравии, эта пустынная страна оказалась способной породить веру, которая, хотя и содержала меньшие социальные требования, но была, тем не менее, монотеистической.
\vs p095 7:5 Существовал только один фактор племенного, расового или национального характера, объединяющий первобытные и разрозненные верования людей пустыни, --- это было странное всеобщее уважение, которое почти все аравийские племена выказывали по отношению к некоему фетишу --- черному камню в некоем храме в Мекке. Этот фактор всеобщей связи и почитания привел впоследствии к установлению религии ислама. Чем был Яхве, дух вулкана, для евреев\hyp{}семитов, тем стал камень Каабы для их арабских родичей.
\vs p095 7:6 Сила ислама заключалась в четком и ясном представлении Аллаха как единого и единственного Бога; его слабость состояла в использовании военной силы для его распространения и униженном положении женщин. Но ислам непоколебимо придерживался своего представления о Едином Всемирном Боге всех, «который знает видимое и невидимое. Он милосерд и сострадателен». «Поистине Бог неисчерпаем в своей доброте ко всем людям». «И когда я болен, это он излечит меня». «Всякий раз, когда трое ведут беседу, четвертым в ней присутствует Бог», ибо не есть ли он «первый и последний, видимый и невидимый»?
\vs p095 7:7 7[Представлено Мелхиседеком Небадона.]
