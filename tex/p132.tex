\upaper{132}{Пребывание в Риме}
\author{Комиссия срединников}
\vs p132 0:1 Гонод привез Тиберию приветственные послания индийских раджей, поэтому на третий день после прибытия в Рим Иисус и двое индусов предстали перед римским правителем. Мрачный император в этот день был необычно приветлив, долго беседовал с тремя путешественниками и, когда те ушли, сказал стоявшему справа помощнику об Иисусе: «Будь у меня царственная осанка и величественные манеры этого человека, я был бы настоящим императором, не так ли?»
\vs p132 0:2 \pc В Риме Ганид ежедневно несколько часов отдавал учебе и знакомству с достопримечательностями города. Гоноду же предстояло совершить массу сделок, и, чтобы сын вырос достойным наследником его обширной торговли, он решил, что настала пора познакомить мальчика с деловым миром. В Риме находилось множество граждан Индии, поэтому часто Гонода в качестве переводчика сопровождал один из его собственных служащих. Благодаря этому Иисус целыми днями бывал свободен, что позволяло ему хорошо изучить город, населенный двумя миллионами жителей. Иисуса встречали на форуме, в центре политической, законодательной и деловой жизни. Он часто ходил к Капитолию и, глядя на величественный храм, посвященный Юпитеру, Юноне и Минерве, размышлял о рабстве невежества, в котором живут римляне. Немало времени провел Иисус и на Палатинском холме, где располагалась резиденция императора, храм Аполлона, греческая и латинская библиотеки.
\vs p132 0:3 \pc К той поре в Римскую империю уже входили весь юг Европы, Малая Азия, Сирия, Египет и северо\hyp{}запад Африки; ее населяли граждане всех стран Восточного полушария. Желание изучить эту космополитическую массу смертных Урантии и пообщаться с ней было главной причиной, почему Иисус отправился в это путешествие.
\vs p132 0:4 В Риме Иисус многое узнал о людях, однако наиболее ценным из многообразия всего того, что с ним произошло за шесть месяцев пребывания в городе, стало общение с религиозными лидерами столицы империи и то влияние, которое он на них оказал. Уже к концу первой недели жизни в Риме Иисус познакомился с наиболее знаменитыми учителями киников, стоиков и мистериальных культов, в частности, с митраистами. Неизвестно, понимал ли Иисус, что евреи отвергнут его миссию, но почти с полной уверенностью можно сказать: Иисус предвидел, что вскоре его посланники прибудут в Рим, чтобы возвестить царство небесное, а потому удивительнейшим образом подготавливал все к тому, чтобы их весть была лучше встречена и воспринята. Для этого он избрал пятерых наиболее выдающихся стоиков, одиннадцать киников, шестнадцать виднейших приверженцев мистериальных культов и почти шесть месяцев значительную часть свободного времени отдавал близкому общению с этими религиозными лидерами. Метод его наставлений был таков: Иисус не только ни разу не осудил их ошибок, но даже не упоминал о недостатках их учений. Всякий раз Иисус находил истину в том, чему они учили, а затем ее всячески выявлял, подкрепляя примерами, чтобы спустя самое непродолжительное время эта истина укоренилась в сознании и вытеснила заблуждение. Таким образом, эти мужчины и женщины, которых учил Иисус, были готовы к тому, чтобы впоследствии воспринять дополняющие и сходные истины в учениях первых христиан\hyp{}миссионеров. То, что учение проповедников евангелия было быстро воспринято, послужило мощным импульсом к распространению христианства в Риме, а оттуда по всей империи.
\vs p132 0:5 Чтобы лучше понять значение этого замечательного деяния, достаточно вспомнить, что из тридцати двух религиозных лидеров, которых учил Иисус, только деятельность двоих не была плодотворной; тридцать же сыграли важнейшую роль в установлении христианства в Риме, а некоторые из них во многом способствовали превращению главного храма Митры в первую христианскую церковь города. Нам, наблюдающим за человеческими деяниями как бы из\hyp{}за кулис и в свете прошедших девятнадцати столетий, видны три неоценимых фактора, которые подготовили почву для быстрого распространения христианства в Европе, а именно:
\vs p132 0:6 \ublistelem{1.}\bibnobreakspace Избрание и утверждение Симона\hyp{}Петра в качестве апостола.
\vs p132 0:7 \ublistelem{2.}\bibnobreakspace Беседа Иисуса со Стефаном в Иерусалиме, смерть которого привела к обращению Савла Тарсянина.
\vs p132 0:8 \ublistelem{3.}\bibnobreakspace Предварительная подготовка тридцати римлян для будущего водительства новой религией в Риме и других частях империи.
\vs p132 0:9 \pc Ни Стефан, ни кто\hyp{}либо из тридцати избранных Иисусом так и не поняли, что однажды им довелось беседовать с человеком, чье имя легло в основу их религиозного учения. Встречи Иисуса с первыми тридцатью двумя учениками были сугубо личными. Трудясь на их благо, книжник из Дамаска, никогда не встречался более, чем с тремя, очень редко более, чем с двумя, и почти всегда учил только одного из них. Величайший труд религиозного воспитания ему удалось совершить потому, что эти не отягощенные традициями мужчины и женщины не были жертвами предвзятого мнения о будущем развитии религии.
\vs p132 0:10 В последующие годы Петр, Павел и другие учителя христианства в Риме не раз слышали о книжнике из Дамаска, который, предшествуя им, столь явно (но им думалось, непреднамеренно) приготовил путь для проповеди ими нового евангелия. Хотя Павел не догадывался, кем был дамасский книжник, незадолго до смерти благодаря сходству различных описаний, он пришел к выводу, что «книжник из Дамаска» и «изготовитель шатров из Антиохии» --- одно лицо. Однажды во время проповеди в Риме, услышав описание книжника, Симон\hyp{}Петр подумал, что тот мог быть Иисусом, однако апостол отбросил эту мысль, будучи уверен (ибо так он думал), что Учитель никогда не был в Риме.
\usection{1. Истинные ценности}
\vs p132 1:1 В самом начале пребывания в столице империи Иисус встретился с главой стоиков Ангамоном, и их беседа продолжалась всю ночь. Впоследствии Ангамон стал ближайшим другом Павла и оказался одним из главных приверженцев христианской церкви в Риме. Вот изложенная современным языком суть того, чему учил Иисус Ангамона:
\vs p132 1:2 \pc «Критерии истинных ценностей следует искать в духовном мире и на божественных уровнях вечной реальности. Всякий смертный, идущий по пути восхождения, должен рассматривать все более низкие и материальные критерии как преходящие, неполные и низшие. Взгляд ученого как такового ограничен исследованием взаимосвязи материальных фактов. Формально он не имеет права называть себя ни материалистом, ни идеалистом, ибо, поступая так, перестает быть истинным ученым; ведь все и любые подобные попытки утверждать свое мировосприятие составляют самую суть философии.
\vs p132 1:3 Если нравственное прозрение и духовные достижения человечества не увеличатся пропорционально, росту материалистической культуры то ее неудержимое наступление может в конце концов превратиться в угрозу цивилизации. Чисто материалистическая наука таит в себе потенциал разрушения всех научных стремлений, ибо такой подход предвещает полнейший крах цивилизации, утратившей понятие о моральных ценностях и переставшей стремиться к духовным достижениям.
\vs p132 1:4 Ученые\hyp{}материалисты и крайние идеалисты обречены на вечное противостояние. Однако это не касается тех ученых и идеалистов, у которых есть общие критерии высших нравственных ценностей и духовных ориентиров. Во все времена ученые и религиозные деятели обязаны понимать, что они стоят перед судом человеческих потребностей. В своем стремлении оправдать собственное существование величайшей преданностью делу прогресса человечества они должны избегать всякой вражды между собой. Если же так называемые наука и религия любого времени окажутся ложными, то им следует либо очиститься, либо уступить место материальной науке и духовной религии более истинным и более достойным».
\usection{2. Добро и зло}
\vs p132 2:1 Признанный глава римских киников Мардус стал большим другом книжника из Дамаска. День за днем он беседовал с Иисусом и каждую ночь внимал его непревзойденному учению. Одна из наиболее важных дискуссий Иисуса с Мардусом была посвящена ответу на поставленный прямодушным киником вопрос о добре и зле. Вот слова Иисуса, переложенные на язык двадцатого века:
\vs p132 2:2 \pc Брат мой, добро и зло --- всего лишь слова, обозначающие относительные уровни человеческого понимания наблюдаемой вселенной. Если ты ленив этически и социально безразличен, то в качестве критерия добра можешь взять современные нормы, принятые в обществе. Если ты духовно ленив и остановился в нравственном развитии, то критериями добра можешь считать религиозные обычаи и традиции своих современников. Однако душа, преодолевающая время и устремленная к вечности, должна делать настоящий и личностный выбор между добром и злом, которые определяются истинными ценностями духовных критериев, установленных божественным духом, который Отец Небесный послал в сердце каждого человека. Этот пребывающий в сердце дух и есть критерий спасения личности.
\vs p132 2:3 Доброта, подобно истине, всегда относительна и неизменно противоположна злу. Осознание этих качеств доброты и истины позволяет совершенствующимся человеческим душам принимать личностные решения в ситуациях выбора, которые ведут к вечной жизни.
\vs p132 2:4 Человек, духовно слепой, логически следующий диктату науки, общественным нормам и религиозной догме, подвергает себя великой опасности утратить нравственную свободу и потерять свободу духовную. Такая душа обречена на участь интеллектуального попугая, общественного автомата и раба религиозных авторитетов.
\vs p132 2:5 Доброта всегда устремлена к новым уровням возрастающей свободы нравственного самосознания и духовного роста личности --- к открытию пребывающего в ней Настройщика и к отождествлению себя с ним. Всякий опыт хорош, если он усиливает восприятие красоты, укрепляет нравственную волю и углубляет понимание истины, если он увеличивает способность любить ближних и служить им, если он возвышает духовные идеалы и объединяет верховные устремления человека с вечным замыслом Настройщика, живущего в нем, ибо все это прямо способствует росту желания исполнять волю Отца, а следовательно, и божественной потребности найти Бога и уподобляться ему.
\vs p132 2:6 \pc Поднимаясь по вселенским ступеням совершенствования создания, ты увидишь возрастание доброты и умаление зла в точном соответствии со своими способностями воспринимать добро и познавать истину. Однако способность впадать в заблуждение и чинить зло не исчезнет совсем, пока человеческая душа, идущая по пути восхождения, не достигнет конечных духовных уровней.
\vs p132 2:7 Доброта --- это живой, относительный, постоянно развивающийся и всегда личный опыт, неизменно связанный с процессом познания истины и красоты. Доброта зиждется на признании позитивных истинных ценностей духовного уровня, которые в человеческом опыте должны противостоять своей отрицательной противоположности --- призракам потенциального зла.
\vs p132 2:8 \pc Человек, не дошедший до уровней Рая, скорее стремится к доброте, нежели ей обладает, она для него пока только цель, а не опыт достигнутого. Однако, даже стремясь обрести праведность, человек уже испытывает все большее удовлетворение от частичного обретения доброты. Присутствие в мире добра и зла само по себе служит убедительным доказательством бытия и реальности нравственной воли человека, его личности, которая, таким образом, не только распознает эти ценности, но и проявляет способность выбирать между ними.
\vs p132 2:9 Ко времени достижения Рая способность смертного, идущего по пути восхождения, отождествлять собственное «я» с истинными ценностями духа возрастает до такой степени, что он достигает совершенства обладания светом жизни. Подобная совершенная духовная личность становится настолько полностью, божественно и духовно объединенной с позитивными и верховными качествами доброты, истины и красоты, что, когда такой праведный дух озарит всепроникающее сияние божественного света бесконечных Райских Правителей, исчезает даже возможность появления и тени потенциального зла. В каждой из подобных духовных личностей доброта уже не является частичной, она не имеет более противоположности, перестает быть относительной и становится божественно завершенной и духовно полной, достигая чистоты и совершенства Верховного.
\vs p132 2:10 \bibemph{Возможность} зла необходима для морального выбора, но из этого не следует, что актуальность зла необходима. Ведь тень --- только относительно реальна. Настоящее зло отнюдь не необходимо как личный опыт. В сферах нравственного совершенствования на низших уровнях духовного развития потенциальное зло может служить адекватным стимулом для принятия решения. Оно становится реальностью личного опыта только тогда, когда наделенное нравственной волей сознание выбирает его.
\usection{3. Истина и вера}
\vs p132 3:1 Греческий еврей Набон был наиболее выдающимся среди лидеров митраистов, последователей главного мистериального культа в Риме. Набон много беседовал с книжником из Дамаска, и слова об истине и вере, сказанные им во время одной из вечерних встреч, произвели на первосвященника митраизма неизгладимое впечатление. Набон думал, будто ему удастся обратить Иисуса, и даже предлагал ему вернуться в Палестину в качестве проповедника Митры, ибо он мало понимал, что, наоборот, Иисус готовит из него одного из первых обращенных к евангелию царства. Вот учение Иисуса, изложенное на современном языке:
\vs p132 3:2 \pc Истину нельзя определить словами, а только самой жизнью. Истина всегда больше знания. Знание ограничено наблюдаемыми фактами, истина же превосходит подобные материальные уровни, ибо она общается с мудростью и заключает в себе такие не поддающиеся оценке вещи, как человеческий опыт и даже духовная и жизненная реальности. Знание рождается наукой, мудрость --- истинной философией, а истина --- религиозным опытом духовной жизни. Знание оперирует фактами, мудрость --- отношениями, истина же --- ценностями реальности.
\vs p132 3:3 Человек склонен кристаллизовать науку, сводить философию к формулировкам и догматизировать истину, ибо, приспосабливаясь к поступательной жизненной борьбе, он ленив умом и к тому же страшится неизвестного. Обычный человек не спешит менять привычный способ мышления и образ жизни.
\vs p132 3:4 Истина, данная в откровении, истина, открытая лично, --- вот высшее наслаждение души человека; она --- общее творение материального сознания и пребывающего в нем духа. Вечное спасение такой понимающей истину и любящей красоту души дается жаждой делать добро, которая устремляет смертного к единой цели --- исполнять волю Отца, найти Бога и уподобляться ему. Между подлинным знанием и истиной противоречий нет. Они возникают лишь между знанием и верованиями человека --- верованиями, окрашенными предрассудками, искаженными страхом, подчиненными боязнью столкнуться с новыми фактами научных открытий и духовного прогресса.
\vs p132 3:5 Однако истина не доступна человеку без веры, ибо мысли человека, его мудрость, этика и идеалы никогда не превосходят его веры, его высшей надежды. Всякая же истинная вера основана на глубоком размышлении, искренней самокритике и нравственном сознании, не знающем компромисса. Вера --- это вдохновение одухотворенного, творческого воображения.
\vs p132 3:6 Вера дает свободу сверхчеловеческой деятельности божественной искры, этого бессмертного живущего в сознании человека начала, который и является потенциалом вечного спасения. Растения и животные выживают во времени, передавая из поколения в поколение идентичные частицы себя. Человеческая же душа (личность) преодолевает смерть плоти, отождествляя себя с этой бессмертной пребывающей в ней искрой божественного, которая служит увековечению человеческой личности на новом, более высоком уровне непрерывно совершенствующегося бытия во вселенной. Бессмертный дух --- это семя, сокрытое в душе человека. Второе проявление души есть первое из череды духовных и устремленных к совершенству существований, которые прекращаются лишь тогда, когда эта божественная сущность достигает источника своего существования --- личного источника всего бытия, Бога, Отца Всего Сущего.
\vs p132 3:7 Человеческая жизнь продолжается --- не прекращается --- благодаря своему вселенскому назначению --- задаче отыскания Бога. Движимая верой душа человека не может остановиться, не дойдя до этой цели своего предназначения; и, однажды достигнув этой божественной цели, она уже не может погибнуть, ибо, уподобившись Богу, становится вечной.
\vs p132 3:8 \pc Духовная эволюция --- это опыт все более уверенного и сознательного выбора добра и одновременно постоянное и пропорциональное уменьшение вероятности зла. Когда выбор в пользу добра становится окончательным, а способность понимать истину достигает полноты, наступает совершенство красоты и святости, праведность которых навечно устраняет саму возможность появления даже понятия потенциального зла. Подобная знающая Бога душа, действуя на столь высоком духовном уровне божественной доброты, уже не допускает и тени сомнения --- зла.
\vs p132 3:9 Присутствие Райского духа в сознании человека --- вот обетование откровения и верный залог вечного божественного совершенствования для каждой души, стремящейся к отождествлению с бессмертной частицей духа Отца Всего Сущего, которая пребывает в ней.
\vs p132 3:10 Прогресс во вселенной характеризуется возрастающей свободой личности, ибо эта свобода связана с последовательным достижением все более высоких уровней самосознания, а следовательно, и добровольного самоограничения. Достижение совершенства духовного самоограничения равносильно полноте вселенской свободы и свободы личной. Вера воспитывает и укрепляет душу человека на начальном этапе, когда у него еще нет устойчивых ориентиров в безграничной вселенной; в то время в молитве замечательно сливаются порывы творческого воображения и устремления верующей души, пытающейся отождествить себя с духовным идеалом божественного присутствия, которое пребывает в человеке и связано с ним.
\vs p132 3:11 \pc Набона глубоко поразили слова Иисуса, как и вообще все беседы с ним. Открывшиеся истины зажгли его сердце, и он оказал огромную помощь проповедникам евангелия Иисуса, когда те появились в Риме.
\usection{4. Личное служение}
\vs p132 4:1 Когда Иисус жил в Риме, то не весь свой досуг посвящал подготовке мужчин и женщин, которые станут в будущем адептами грядущего царства. Значительную часть времени он отдавал глубокому изучению человеческих рас и классов, представители которых жили в этом самом большом и космополитическом городе мира. Общаясь с людьми, Иисус преследовал двойную цель: во\hyp{}первых, ему хотелось узнать, как они относятся к жизни, которой живут во плоти, а во\hyp{}вторых, он стремился сказать или совершить что\hyp{}нибудь такое, что сделало бы эту жизнь богаче и достойнее. Религиозное учение Иисуса в это время не отличалось от более позднего, когда он стал учителем двенадцати апостолов и проповедником для народных масс.
\vs p132 4:2 Главным в речах Иисуса всегда были: реальность любви Небесного Отца, истина о его милосердии и благая весть о том, что человек благодаря вере становится сыном Бога любви. Обычно Иисус общался с людьми, вовлекая их в разговор расспросами, и, как правило, эти беседы начинались с вопросов, которые задавал он, а заканчивались вопросами собеседников. Иисус с равным искусством учил, и когда спрашивал, и когда отвечал, но обычно больше других от него узнавали те, кому он говорил меньше. Максимальную пользу его личное служение приносило самым угнетенным, самым растерявшимся и удрученным смертным, которые получали огромное облегчение, благодаря возможности излить душу сочувствующему и понимающему слушателю, каким и даже более того был Иисус. Когда эти сбившиеся с пути люди рассказывали о своих бедах, он всегда давал немедленные и практические советы, помогающие разрешить их реальные трудности, но не пренебрегал Иисус и словами, которые успокаивалии и утешали сразу. Он всегда говорил отчаявшимся о любви Бога и всевозможными способами учил, что они --- дети любящего их Небесного Отца.
\vs p132 4:3 Таким образом, живя в Риме, Иисус лично вступил в теплые и вселяющие уверенность отношения с более чем пятьюстами жителями города. Благодаря этому сам Иисус многое узнал о человеческих расах, чего он не смог бы сделать не только в Иерусалиме, но даже в Александрии. Шесть месяцев, проведенные в Риме, он всегда считал одним из самых плодотворных и информативных периодов своей земной жизни.
\vs p132 4:4 Как и следовало ожидать, такой разносторонний и энергичный человек не мог жить в мировой метрополии, не привлекая внимания людей, которые хотели воспользоваться его служением на пользу какому\hyp{}нибудь делу, а чаще чтобы помочь проекту какого\hyp{}либо учения, социальных реформ или религиозного движения. Иисус получил больше дюжины подобных предложений и использовал каждое из них как возможность с помощью нужных слов или любезной услуги сообщить какую\hyp{}нибудь облагораживающую душу мысль, ведь он любил помогать даже в мелочах всем людям.
\vs p132 4:5 \pc С римским сенатором Иисус говорил о политике и государственных делах, и эта единственная беседа произвела на законодателя такое впечатление, что всю оставшуюся жизнь он (к сожалению, напрасно) убеждал коллег изменить политику управления страной, отказавшись от идеи, что правительство кормит и поддерживает народ, в пользу другой, что, напротив, народ поддерживает правительство. С богатым рабовладельцем Клавдием он провел вечер, говоря о человеке как о сыне Бога, и на следующий день этот человек, Клавдий, отпустил на свободу сто семьдесят рабов. С греком\hyp{}врачом Иисус во время трапезы говорил о том, что у пациентов есть не только тело, но также разум и душа; это побудило способного лекаря попытаться более всеобъемлюще служить своим собратьям. И вообще Иисус встречался с самыми разными людьми и беседовал с ними обо всем. Единственным местом, куда Иисус не ходил, были публичные бани. Сюда он отказывался идти с друзьями из\hyp{}за царившего там разврата.
\vs p132 4:6 \pc Вот что сказал Иисус римскому воину, с которым прогуливался вдоль Тибра: «Будь храбр сердцем и тверд рукой. Не бойся поступать справедливо и будь великодушен, чтобы проявлять милосердие. Подчиняй низменное в себе возвышенному в себе, как сам подчиняешься командирам. Почитай добро и превозноси истину. Выбирай красивое вместо уродливого. Люби своих собратьев и всем сердцем стремись к Богу, ибо Бог --- твой Небесный Отец».
\vs p132 4:7 \pc Оратору форума Иисус говорил: «Прекрасно твое красноречие, твоя логика восхитительна, у тебя приятный голос, но твое учение едва ли истинно. О, если бы ты только познал вдохновляющее удовлетворение от осознания Бога своим духовным Отцом. Тогда бы ты смог использовать силу своего красноречия, чтобы вызволить сограждан из темноты и освободить их от рабства невежества». Человек, с которым говорил Иисус, был Маркус. Услышав проповедь Петра в Риме, он стал продолжателем дела апостола. Когда же распяли Симона\hyp{}Петра, именно Маркус, бросив вызов преследователям, продолжал открыто проповедовать новое евангелие.
\vs p132 4:8 \pc Встретив ложно обвиненного бедняка, Иисус вместе с ним пошел в суд и, получив особое разрешение выступать в защиту несчастного, произнес прекрасную речь, в которой сказал и такие слова: «Правосудие делает нацию великой. Чем больше величие нации, тем сильнее она заботится о том, чтобы от несправедливости не страдал даже самый смиреннейший гражданин. Позор любой нации, в которой правосудие служит одним богачам и властителям! Священный долг судьи --- оправдывать невиновных так же, как и карать преступников. От беспристрастности, справедливости и честности суда зависит прочность нации. Гражданское правление основано на правосудии так же, как истинная религия --- на милосердии». Суд возобновил дело и, выслушав свидетелей, освободил заключенного. Из всех дел личного служения Иисуса это больше других походило на публичное выступление.
\usection{5. Советы богачу}
\vs p132 5:1 Некий богатый человек, римский гражданин и стоик, серьезно заинтересовался учением Иисуса, с которым его познакомил Ангамон. После нескольких задушевных бесед этот богатый гражданин спросил у Иисуса, как бы тот поступил с богатством, если бы оно у него было. Иисус ответил ему: «Материальное богатство я бы употребил для улучшения материальной жизни, так же как знания, мудрость и духовное служение использовал бы для обогащения интеллектуальной, облагораживания общественной и совершенствования духовной. А еще материальное достояние я бы использовал в качестве мудрого и эффективного вложения ресурсов одного поколения на благо следующего и грядущих поколений».
\vs p132 5:2 Однако ответ Иисуса не вполне удовлетворил богатого человека, и он решился спросить еще: «Но как, по\hyp{}твоему, должен распорядиться своим богатством человек в моем положении? Могу ли я им пользоваться сам или мне его следует раздать?» Иисус понял, что богатый действительно желает больше узнать об истине, своем долге перед Богом и перед людьми, а потому продолжал: «Мой добрый друг, я вижу, ты настоящий искатель мудрости и действительно возлюбил истину. Поэтому я согласен изложить тебе мой взгляд на решение твоих проблем, связанных с той ответственностью, которую налагает на тебя богатство. Я делаю это потому, что ты \bibemph{попросил} у меня совета, но, давая его, я говорю отнюдь не о том, что принадлежит другим состоятельным людям. Мой совет предназначен только тебе и потому одному тебе может служить руководством. Если ты честно желаешь рассматривать свое состояние как полученный тобою на попечение и на самом деле хочешь стать мудрым и умелым распорядителем накопленного тобою богатства, то я советую тебе следующим образом проанализировать источники, из которых оно происходит: спроси себя, откуда твое богатство, и сделай все, чтобы дать честный ответ. А чтобы исследовать источники твоего огромного состояния тебе было проще, я советовал бы тебе запомнить следующие десять способов накопления материальных благ:
\vs p132 5:3 \ublistelem{1.}\bibnobreakspace Наследство --- богатство, полученное от родителей и других предков.
\vs p132 5:4 \ublistelem{2.}\bibnobreakspace Открытие --- богатство, извлеченное из недр земли.
\vs p132 5:5 \ublistelem{3.}\bibnobreakspace Торговля --- богатство, полученное в виде честной прибыли от обмена материальными благами и торговли ими.
\vs p132 5:6 \ublistelem{4.}\bibnobreakspace Не справедливое --- богатство, добытое нечестной эксплуатацией порабощенных ближних.
\vs p132 5:7 \ublistelem{5.}\bibnobreakspace Проценты --- доход, полученный от честного и справедливого вложения капитала.
\vs p132 5:8 \ublistelem{6.}\bibnobreakspace Гениальность --- богатство, приобретенное благодаря творческому и изобретательскому дарованию.
\vs p132 5:9 \ublistelem{7.}\bibnobreakspace Случай --- богатство, полученное от щедрот ближних или благодаря жизненным обстоятельствам.
\vs p132 5:10 \ublistelem{8.}\bibnobreakspace Воровство --- богатство, добытое несправедливостью, нечестностью, воровством или мошенничеством.
\vs p132 5:11 \ublistelem{9.}\bibnobreakspace Попечительство --- богатство, доверенное ближними для исполнения какой\hyp{}то особой цели в настоящее время или в будущем.
\vs p132 5:12 \ublistelem{10.}\bibnobreakspace Труд --- богатство, добытое собственным трудом, честная и справедливая награда за ежедневный физический и умственный труд.
\vs p132 5:13 \pc Итак, мой друг, если ты хочешь перед Богом и людьми быть заслуживающим доверия и справедливым распорядителем своего огромного состояния, тебе следует разделить его приблизительно на десять основных частей, а затем с каждой из них поступить так, как того требует честное и мудрое понимание законов правосудия, справедливости, честности и истинной пользы; тогда Бог на небесах не осудит тебя, если иной раз в неоднозначной ситуации ты и ошибешься в пользу милосердного и бескорыстного сочувствия горю страждущих жертв несчастных обстоятельств смертной жизни. Если же ты искренне сомневаешься, что в той или иной ситуации поступаешь честно и справедливо, то да будут твои решения в пользу тех, кто нуждается и страдает от незаслуженных тягот».
\vs p132 5:14 После многочасового обсуждения этих вопросов в ответ на просьбу богатого человека дать ему более подробные разъяснения Иисус решил дополнить свой совет и, по существу, сказал следующее: «Хорошо, я дам тебе еще несколько советов, как относиться к богатству, но прошу тебя помнить, что они касаются одного тебя и только тебя. Все, что я скажу, я скажу от себя и скажу тебе как пытливому другу. Но при этом заклинаю, не указывай другим, как им относиться к своему богатству. Вот что я тебе посоветую:
\vs p132 5:15 \pc 
\vs p132 5:16 \pc \ublistelem{2.}\bibnobreakspace Всякому, кто получил богатство благодаря сделанному открытию, следует помнить о скоротечности человеческой жизни и, следовательно, позаботиться о том, чтобы разделить открытие с как можно большим числом своих собратьев. Сделавшему открытие нельзя отказать в вознаграждении за приложенные усилия, однако он не должен своекорыстно претендовать на все преимущества и всю пользу, которые дает обнаружение сокрытых природой ресурсов.
\vs p132 5:17 \pc \ublistelem{3.}\bibnobreakspace Пока люди совершают дела путем обмена и торговли, они имеют право на честную и законную прибыль. Каждый торговец заслуживает платы за свои услуги, а купец имеет право на прибыль. Справедливая торговля и честность по отношению к людям, занимающимся коммерческой деятельностью, создают множество способов накопления богатства путем извлечения прибыли, причем обо всех этих способах следует судить, опираясь на высочайшие принципы законности, честности и справедливости. Честный торговец без колебания может брать такую же прибыль, какую с радостью отдал бы сам за аналогичную услугу. Хотя подобного рода богатство, когда дело ведется в больших масштабах, и отличается от дохода, полученного индивидуально, все же такое честно накопленное состояние наделяет его обладателя существенной долей права голоса во время принятия решения о его последующем распределении.
\vs p132 5:18 \pc \ublistelem{4.}\bibnobreakspace Никогда смертный, знающий Бога и желающий исполнять Его волю, не унизит себя, употребляя богатство для угнетения других людей. Ни один благородный человек не будет копить богатство и увеличивать власть, которую дают деньги, путем порабощения или нечестной эксплуатации братьев по плоти. Богатство становится нравственным проклятием и духовным позором, если оно добывалось потом и кровью угнетенных. Такое богатство следует вернуть ограбленным, их детям или детям их детей. Прочную цивилизацию нельзя построить, лишая работников положенной платы.
\vs p132 5:19 \pc \ublistelem{5.}\bibnobreakspace Владелец богатства, нажитого честно, имеет право на прибыль в виде процентов. Пока люди дают и берут в долг, такой доход можно считать честным, если ссужаемый капитал приобретен законно. Прежде, чем претендовать на проценты, сначала очисти свой капитал. Не будь мелочен и жаден, иначе превратишься в ростовщика. Никогда не позволяй себе быть настолько эгоистичным, чтобы, пользуясь властью денег, получить несправедливое преимущество перед ближними, оказавшимися в нужде. Не поддавайся соблазну взять с попавшего в беду собрата больше положенного.
\vs p132 5:20 \pc \ublistelem{6.}\bibnobreakspace Если тебе посчастливилось получить богатство благодаря твоему гениальному озарению или твое состояние слагалось из вознаграждений за дар изобретать, не претендуй на большее, чем тебе положено. Гений имеет обязательства и перед предками, и перед потомками; точно так же он несет обязательства перед своей расой, народом и условиями, благодаря которым он сделал свои открытия; гению следует помнить и то, что, трудясь и работая над своим изобретением, он пользовался помощью других людей. Однако в равной степени было бы несправедливо отказать гению в вознаграждении, хотя едва ли людям когда\hyp{}нибудь удастся выработать правила и нормы, одинаково применимые для решения всех проблем, связанных со справедливым распределением богатства. Поэтому прежде всего тебе следует понять, что любой человек есть твой брат, и если ты действительно хочешь поступать с ним, как ты хотел бы, чтобы он поступал с тобой, то обычные правила справедливости, искренности и честности подскажут тебе правильное и беспристрастное решение любой проблемы материального вознаграждения и социальной справедливости.
\vs p132 5:21 \pc \ublistelem{7.}\bibnobreakspace Кроме заслуженной и законной платы, положенной за управление состоянием, никто не имеет права претендовать на богатство, которое по воле времени и случая попало в его руки. Такое случайно полученное богатство надлежит рассматривать, до некоторой степени как средства, которые следует потратить на нужды своей социальной или экономической группы. Обладателю такого богатства необходимо предоставить основное право решать, как наиболее эффективно и мудро распределить такие даром полученные ресурсы. Цивилизованный человек не будет всегда считать все, чем он управляет, своей личной и частной собственностью.
\vs p132 5:22 \pc \ublistelem{8.}\bibnobreakspace Если какая\hyp{}то часть твоего состояния получена путем сознательного мошенничества или накоплена бесчестными и несправедливыми путями, если оно --- результат недостойных сделок с ближними, то без промедления верни эти неправедные доходы законным владельцам. Доведи это дело до конца и, таким образом, освободи свое состояние от нечестных денег.
\vs p132 5:23 \pc \ublistelem{9.}\bibnobreakspace Попечительство над состоянием одного человека на благо других --- важная и священная ответственность. Не злоупотребляй и не рискуй вверенным тебе богатством и возьми лишь ту его долю, какую позволил бы себе взять любой честный человек.
\vs p132 5:24 \pc \ublistelem{10.}\bibnobreakspace Та часть твоего состояния, которая заработана твоим собственным умственным и физическим трудом --- если работа была выполнена честно и добросовестно, --- принадлежит тебе по праву. Никто не может отрицать твоего права обладать и пользоваться ей, если это не наносит ущерба твоим ближним».
\vs p132 5:25 \pc Когда Иисус изложил все свои советы, богатый римлянин поднялся с ложа и, пожелав ему доброй ночи, дал обещание: «Мой добрый друг, я понял, что ты --- человек величайшей мудрости и добродетели. Завтра же я начну распоряжаться моим богатством, как ты советовал мне».
\usection{6. Общественное служение}
\vs p132 6:1 В Риме произошел и такой трогательный случай, когда Творец Вселенной потратил несколько часов, чтобы вернуть потерявшегося ребенка встревоженной матери. Мальчик заблудился, и Иисус нашел его горько плачущим. Иисус и Ганид, шедшие в библиотеку, увидели это и решили отвести мальчика домой. Ганид навсегда запомнил слова Иисуса: «Знаешь, Ганид, большинство людей похожи на этого заблудившегося ребенка. Большую часть своего времени они плачут от страха и страдают от печали, когда на самом деле покой и безопасность совсем близко, так же как этот мальчик был совсем рядом с домом. Поэтому все, кто знает путь истины и обладает уверенностью, что знает Бога, должны считать не долгом, а привилегией возможность помогать ближним, ищущим гармонии в жизни. Разве не радуемся сверх меры мы, помогая вернуть сына матери? Так и те, кто ведет людей к Богу, испытывают верховную радость служения людям». С того самого дня до конца естественной жизни Ганид всегда искал заблудившихся детей, которым он мог помочь найти свой дом.
\vs p132 6:2 \pc У женщины, матери пятерых детей, погиб муж от несчастного случая, и она овдовела. Иисус рассказал Ганиду о том, как от несчастного случая погиб его отец, и они часто ходили утешать ту женщину и ее детей, Ганид же просил у отца денег, чтобы кормить и одевать их. Иисус и Ганид не переставали помогать нуждающимся, пока не нашли работу старшему сыну, чтобы он сам помогал в заботах о семье.
\vs p132 6:3 \pc В тот вечер Гонод, выслушав рассказ о том, как Иисус и его сын заботились о несчастных, добродушно сказал Иисусу: «Я собираюсь сделать из сына ученого или делового человека, а ты взялся делать из него философа и филантропа». На это Иисус с улыбкой отвечал: «А не сделать ли нам его всем этим одновременно. Тогда он сможет получать от жизни в четыре раза больше удовольствия, ибо, слушая мелодию человеческих душ, сможет узнавать в ней не одну, а четыре ноты». «Теперь я вижу, --- отвечал Гонод, --- ты --- настоящий философ. Ты должен написать книгу для будущих поколений». Иисус отвечал: «Не книгу, ибо моя цель --- прожить жизнь в этом поколении и на благо всех поколений. Я\ldots » Но тут Иисус прервал себя и обратился к Ганиду: «Сын мой, нам пора идти спать».
\usection{7. Путешествия в окрестностях Рима}
\vs p132 7:1 Иисус, Гонод и Ганид совершили пять путешествий к достопримечательностям, расположенным вокруг Рима. По пути к северным Итальянским озерам Иисус долго говорил с Гонодом о невозможности передать знания о Боге человеку, который не желает его знать. В дороге им случайно встретился беззаботный язычник, и Ганид удивился, что Иисус не стал, как обычно, привлекать его к разговору, который постепенно дошел бы до обсуждения духовных вопросов. Когда Ганид спросил Учителя, почему он не проявил интереса к язычнику, Иисус ответил:
\vs p132 7:2 \pc «Ганид, этот человек не алчет истины. Он вполне доволен собой и пока не готов просить о помощи, глаза его разума еще не открыты, чтобы воспринять свет для души. Этот человек еще не созрел для жатвы спасения; ему потребуется немало времени, чтобы пройти через трудности и испытания жизни, которые приготовят его к восприятию мудрости и высшего знания. Либо, если бы он мог жить с нами, мы своим примером могли бы открыть ему небесного Отца, и он, привлеченный той жизнью, которой мы живем как сыновья Бога, стал бы интересоваться им. Нельзя открыть Бога тем, кто его не ищет; нельзя привести тех, кто не хочет этого, к радости спасения. Человек должен взалкать истины, благодаря жизненному опыту, либо к Небесному Отцу его может привести другой смертный, но вначале он сам, глядя на тех, кто уже знает Бога, должен этого захотеть. Если мы знаем Бога, то наше главное дело на земле --- жить так, чтобы Отец открывал себя в наших жизнях, благодаря чему все, кто ищет Бога, узнали бы Отца и могли бы просить нас помочь им больше узнать о Боге, который таким образом проявляет себя в нашей жизни.»
\vs p132 7:3 \pc Во время посещения Швейцарии высоко в горах Иисус целый день говорил с отцом и сыном о буддизме. Ганид не раз задавал Иисусу прямые вопросы о Будде, но всегда получал более или менее уклончивые ответы. Теперь же в присутствии сына подобный вопрос задал Гонод и получил на него прямой ответ. «Я действительно хотел бы знать, что ты думаешь о Будде», --- сказал он. Иисус отвечал:
\vs p132 7:4 «Ваш Будда был намного лучше вашего буддизма. Будда был великим человеком и даже пророком своего народа, но он был пророком\hyp{}сиротой --- я имею в виду, что он рано потерял способность видеть своего духовного Отца --- Отца Небесного. Его жизненный путь трагичен. Он пытался жить и учить как посланник Бога, но без Бога. Будда направил свой корабль спасения в безопасную бухту, к самому входу в гавань спасения смертных, но здесь из\hyp{}за неправильных навигационных карт добрый корабль сел на мель. Здесь он покоится уже многие века и, не двигаясь с места, почти безнадежно остается на мели. Здесь же все эти годы остаются многие ваши соотечественники. Они живут совсем близко от безопасных вод покоя, но отказываются войти в них потому, что величественному судну доброго Будды не повезло и оно оказалось на мели у самого входа в бухту. Народы, исповедующие буддизм, никогда не войдут в эту гавань, пока не оставят философское судно своего пророка и не проникнутся его благородным духом. Если бы ваш народ оставался верен духу Будды, вы бы давно уже вошли в бухту духовного спокойствия, покоя души и уверенности в спасении.
\vs p132 7:5 Понимаешь, Гонод, Будда познал Бога в духе, но явно не сумел открыть его умом; евреи нашли Бога умом, но в основном не познали его духом. Сегодня буддисты увязли в философии без Бога, а мой народ порабощен страхом перед Богом и не ведает спасительной философии жизни и свободы. У вас --- философия без Бога, у евреев --- Бог, но почти без связанной с ним философии жизни. Будда не сумел прозреть Бога как Духа и Отца и не смог придать своему учению нравственной энергии и духовной движущей силы, которыми должна обладать всякая религия, если ее цель --- изменить расу и возвеличить нацию».
\vs p132 7:6 «Учитель, --- воскликнул Ганид, --- давай вместе создадим новую религию, которая будет достаточно хороша для Индии и достаточно велика для Рима. Тогда, возможно, мы сумеем убедить евреев принять ее вместо Яхве». Иисус ответил: «Ганид, религии не создаются. Они формируются в течение длительного времени, когда откровения Бога вспыхивают на земле, освещая ее жизнями людей, открывающих Бога своим ближним». Однако ни отец, ни сын не поняли значения пророческих слов Иисуса.
\vs p132 7:7 \pc В эту ночь Ганид не мог уснуть. Он долго беседовал с отцом и в заключение сказал: «Знаешь, отец, я иногда думаю, что Иешуа --- пророк». На это Гонод лишь сонно ответил: «Сын мой, есть и другие\ldots »
\vs p132 7:8 С того самого дня до конца естественной жизни Ганид продолжал работать над созданием собственной религии. Огромное влияние на него оказали широта взглядов Иисуса, его честность и терпимость. Во всех беседах с Иисусом о философии и религии юноша ни разу не испытал чувства обиды и не почувствовал антагонизма по отношению к Иисусу.
\vs p132 7:9 \pc Прекрасное зрелище для небесных существ созерцать, как индийский мальчик предлагает Творцу вселенной создать новую религию! И хотя Ганид не сознавал этого, но они ее действительно создавали; они создавали новую религию на все времена, создавали именно тогда и именно там. Это был новый путь спасения, откровение Бога человеку, явленное в Иисусе и через него. То, что мальчик хотел делать больше всего на свете, он уже делал, о том не ведая. Так было, так есть и так будет всегда. Все, к чему искренне и бескорыстно стремится просвещенное воображение мыслящего человека, обращенного к духовному учению и руководству, обязательно приобретает действенную творческую силу в точном соответствии со степенью преданности смертного человека божественному делу исполнения воли Отца. Если человек действует вместе с Богом, могут совершаться и совершаются великие дела.
