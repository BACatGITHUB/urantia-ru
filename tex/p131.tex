\upaper{131}{Религии мира}
\author{Комиссия срединников}
\vs p131 0:1 Во время посещения Иисусом, Гонодом и Ганидом Александрии молодой человек тратил много времени и отцовских денег на составление собрания учений мировых религий о Боге и его отношениях со смертным человеком. Ганид пользовался услугами более шестидесяти ученых переводчиков для того, чтобы собрать эти выдержки о Божествах из религиозных доктрин мира. И следует заметить, что все учения, отражающие монотеизм, в целом прямо или косвенно произошли от проповедей посланников Махивенты Мелхиседека, которые были посланы им из Салема, чтобы распространить учение о едином Боге --- о Всевышнем --- по всей земле.
\vs p131 0:2 Ниже представлены извлечения из рукописи Ганида, которую он составил в Александрии и в Риме и которая хранилась в Индии на протяжении сотен лет после его смерти. Он распределил этот материал по десяти разделам, как он и приводится ниже.
\usection{1. Кинизм}
\vs p131 1:1 Остатки учения последователей Мелхиседека, кроме тех, которые продолжаются в иудаизме, лучше всего сохранились в доктринах киников. Сборник Ганида содержал следующие из них:
\vs p131 1:2 \P\ «Бог есть верховный; он есть Всевышний на небе и на земле. Бог есть завершенный круг вечности, и он правит вселенной вселенных. Он --- единственный творец неба и земли. Когда он задумывает вещь, она существует. Наш Бог --- единый Бог, и он сострадателен и милосерден. Все высокое, святое, истинное и красивое подобно нашему Богу. Всевышний есть свет неба и земли; он есть Бог востока, запада, севера и юга.
\vs p131 1:3 Даже если исчезнет земля, сияющий лик Верховного пребудет в величии и славе. Всевышний есть первый и последний, начало и конец всего. Есть только один этот Бог, и имя ему --- Истина. Жизнь Бога в нем самом, и он чужд всякого гнева и вражды; он бессмертен и бесконечен. Наш Бог всемогущ и щедр. У него множество проявлений, но мы поклоняемся только самому Богу. Богу ведомо все --- все наши тайны и провозглашения; ему ведомо также, чего каждый из нас заслуживает. Его могущество равно всему сущему.
\vs p131 1:4 Бог есть податель мира и верный защитник всех, кто боится его и верит ему. Он дает спасение всем, кто ему служит. Все творение существует во власти Всевышнего. Его божественная любовь берет начало в святости его власти, и благодеяние рождается из мощи его величия. Всевышний утвердил единение тела и души и одарил человека своим собственным духом. То, что делает человек, должно иметь конец, но то, что делает Творец, длится вечно. Мы получаем знания из человеческого опыта, но мудрость мы извлекаем из созерцания Всевышнего.
\vs p131 1:5 Бог изливает дождь на землю, он заставляет солнце освещать зерно, пускающее ростки, и он посылает нам обильный урожай благ этой жизни и вечное спасение в грядущем мире. Наш Бог обладает великой властью; имя ему --- Прекрасный, и природа его непостижима. Когда ты болен, не кто иной, как Всевышний исцеляет тебя. Бог исполнен доброты по отношению ко всем людям; у нас нет друга, подобного Всевышнему. Его милосердие наполняет все, и его доброта окружает все души. Всевышний неизменен; он наш помощник во всякой нужде. Где бы ты ни обратился к молитве, там лик Всевышнего и приклоненный слух нашего Бога. Ты можешь спрятаться от людей, но не от Бога. Бог не далек от нас; он вездесущ. Бог пребывает везде и живет в сердце человека, который боится его святого имени. Творение пребывает в Творце, и Творец пребывает в своем творении. Мы ищем Всевышнего и затем находим его в своем сердце. Ты идешь искать дорогого друга, а потом находишь его в своей душе.
\vs p131 1:6 Человек, знающий Бога, смотрит на всех людей как на равных; они его братья. Тот, кто эгоистичен и пренебрегает своими братьями по плоти, расплачивается за это опустошением. Тот же, кто любит своих братьев и имеет чистое сердце, узрит Бога. Бог никогда не забывает искренности. Он направит честное сердце к истине, потому что Бог --- это истина.
\vs p131 1:7 В жизнях ваших побеждайте грех и преодолевайте зло любовью к живой истине. Во всех ваших отношениях с людьми отвечайте добром на зло. Господь Бог полон милосердия и любви; он прощает грехи. Давайте будем любить Бога, ибо он первым полюбил нас. Любовью Бога и его милосердием мы будем спасены. Бедные люди и богатые люди --- братья. Бог есть их Отец. Зло, которое ты не хотел бы, чтобы причинили тебе, не делай другому.
\vs p131 1:8 Во всякое время взывай к его имени, и если ты веришь в его имя, твоя молитва будет услышана. Какая высокая честь --- поклоняться Всевышнему! Все миры и вселенные поклоняются Всевышнему. И всеми твоими молитвами возноси благодарность --- и восходи к почитанию. Молитвенное почитание избегает зла и не позволяет грешить. Давайте всегда славить имя Всевышнего. Человек, который ищет убежище во Всевышнем, скрывает свои недостатки от вселенной. Когда ты предстоишь перед Богом с чистым сердцем, ты не боишься всего творения. Всевышний подобен любящему отцу и матери; он и в самом деле любит нас, своих детей на земле. Наш Бог простит нас и направит наши стопы на пути спасения. Он возьмет нас за руку и приведет нас к себе. Бог спасает тех, кто вверяет себя ему; он не принуждает человека служить своему имени.
\vs p131 1:9 Если вера во Всевышнего проникла в твое сердце, ты будешь свободен от страха во все дни твоей жизни. Не огорчайся, видя благополучие безбожника; не бойся тех, кто замышляет зло; пусть твоя душа отвернется от зла и полностью доверится Богу спасения. Усталая душа скитающегося смертного обретает вечный покой в руках Всевышнего; мудрый человек алчет божественного объятия; земное дитя жаждет безопасности в руках Отца Всего Сущего. Благородный человек ищет того высокого положения, в котором душа смертного сочетается с духом Верховного. Бог справедлив: те плоды, которых мы не получаем от наших садов в этом мире, мы получим в следующем».
\usection{2. Иудаизм}
\vs p131 2:1 Кениты Палестины спасли многое из учения Мелхиседека, и из этих рукописей, сохраненных и измененных евреями, Иисус и Ганид сделали следующие выписки:
\vs p131 2:2 \P\ «Вначале Бог сотворил небо и землю и все внутри них. И вот, все, что он сотворил, было очень хорошо. Господь, он есть Бог: нет ничего кроме него в небесах наверху и на земле внизу. Поэтому ты должен любить Господа Бога твоего всем сердцем твоим, и всей душой твоей, и всей силой твоей. Как воды покрывают море, так и земля будет полна знанием Бога. Небеса провозглашают славу Божию, и небесный свод являет дело рук его. День за днем произносит речь; ночь за ночью являет знание. Нет речи или языка, в котором не был бы слышен их голос. Велика работа Господа, и в мудрости сделал он всякую вещь; величие Господа непостижимо. Он знает число звезд, и каждую он зовет по имени.
\vs p131 2:3 Велика сила Господа и бесконечно понимание его. Говорит Господь: „также, как небеса выше земли, так и мои пути выше, чем ваши пути, и мои мысли выше, чем ваши“. Богу ведомы все самые сокровенные тайны, ибо свет пребывает в нем. Бог милосерден и добр; он долготерпелив и обилен в доброте и правде. Бог благ и справедлив; он направит мысли смиренного. Изведай и убедись, что Бог благ! Благословен человек, верящий Богу. Бог наше прибежище и сила, самая верная помощь в горе.
\vs p131 2:4 Милосердие Бога испокон веков и во веки веков с теми, кто боится его и его правды, вплоть до детей наших детей. Бог милостив и полон сострадания. Бог добр ко всем, и его кроткая милость на всех его творениях; он исцеляет сокрушенных сердцем и врачует их раны. Куда пойду я от духа Божия? Куда убегу от божественного присутствия? Так говорит Возвышенный и Величественный, обитающий в вечности, чье имя Святой: „Я пребываю в высоком и священном месте; а также с тем, у кого сокрушенное сердце и смиренный дух“ Никто не может укрыться от нашего Бога, так как он наполняет небо и землю. Да возрадуются небеса и возвеселится земля. Пусть говорят все народы: Царстует Господь! Благодарите Господа, ибо милость его во веки веков.
\vs p131 2:5 Небеса возвещают правоту Бога, и все народы видели славу его. Бог создал нас, а не мы сами создали себя. Мы народ его, овцы стада его. Милосердие его длится вечно, и правда его простирается на все поколения. Наш Бог правит всеми народами. Да исполнится земля славы его! Да восхвалят люди Господа за его доброту и за его чудесные дары детям человеческим!
\vs p131 2:6 Бог создал человека почти божественным и увенчал его любовью и милосердием. Господь знает путь праведника, но путь безбожника обраnится в ничто. Страх Божий есть начало мудрости; знание Верховного есть понимание. Говорит Всемогущий Бог: „Предстань пред лицом моим и будь совершен!“ Не забывай, что гордость идет впереди разрушения, а надменный дух --- впереди падения. Тот, кто управляет своим духом, могущественнее того, кто покоряет город. Говорит Господь Бог Святой: „В возвращении к твоему духовному покою спасение твое, в спокойствии и доверии обретешь силу“. Надеющиеся на Господа восполнят свои силы; на крыльях, подобных орлиным, воспарят они вверх. Они будут бежать, и не утомятся, они будут ходить, и не ослабеют. Господь даст тебе отдых от твоего страха. Говорит Господь: „Не бойся, ибо я с тобой. Не унывай, ибо я Бог твой. Я дам тебе силу; я помогу тебе; да, я буду поддерживать тебя правой рукой моей праведности“.
\vs p131 2:7 Бог есть Отец наш; Господь есть наш спаситель. Бог сотворил все силы небесные и охраняет их всех. Праведность его подобна горам, и кара его подобна великой бездне. Он заставляет нас пить из реки его наслаждений, и в его свете мы увидим свет. Хорошо благодарить Господа и петь хвалы Всевышнему; проявлять исполненную любовью доброту утром и божественную преданность каждый вечер. Царство Бога --- вечное царство, и владычество его длится все поколения. Господь --- мой пастырь; я не буду нуждаться. Он пасет меня на зеленых пастбищах и ведет меня к спокойным водам. Он возрождает мою душу. Он ведет меня по тропам праведности. Да, даже если я пойду через долину сени смертной, я не убоюсь зла, потому что Бог со мной. Конечно, доброта и милосердие будут следовать за мной во все дни моей жизни, и я буду пребывать в доме Бога вечно.
\vs p131 2:8 Ягве есть Бог моего спасения; поэтому на божественное имя возложу я свою веру. Я буду верить в Бога всем моим сердцем; не на мое собственное понимание буду я полагаться. Во всех делах моих я буду признавать его, и он направит пути мои. Господь правдив, он верен своему слову с теми, кто служит ему; праведный будет жить своей верой. Если ты поступаешь нехорошо, это потому. что грех стоит у дверей; люди пожинают плоды зла, которое они вспахали, и греха, который посеяли. Не угнетайся из\hyp{}за делающих зло. Если ты порочен в сердце своем, Господь не услышит тебя; если ты грешишь против Бога, ты также вредишь своей собственной душе. Бог вынесет дела каждого человека на суд, со всеми тайными деяниями, добрые они или злые. Как человек помышляет в сердце своем, таков и он сам.
\vs p131 2:9 Господь рядом с теми, кто взывает к нему искренне и правдиво. Плач может продолжаться всю ночь, но утром приходит радость. Веселое сердце исцеляет подобно лекарству. Ни в одной доброй вещи Бог не откажет тому, кто ходит прямыми путями. Бойся Бога и соблюдай его заповеди, ибо в этом и есть весь долг человека. Так говорит Господь, создавший небо и землю: „Нет Бога, кроме меня, праведного Бога и спасителя. Положитесь на меня и спаситесь, все концы земли. Если ты ищешь меня, ты найдешь меня, если ты ищешь меня всем сердцем“. Кроткие наследуют землю и насладятся изобилием мира. Сеющий же беззаконие пожнет бедствие; сеющие ветер пожнут бурю.
\vs p131 2:10 „Приди, и давай рассудим вместе“ --- сказал Господь. --- „Если твои грехи будут, как пурпур, они станут белыми, как снег. Даже если они будут алеть, как румянец, они станут, как руно“. Но нет мира для грешника; твои собственные грехи не дают добру приблизиться к тебе. Бог есть здоровье моего тела и веселье моей души. Вечный Бог есть моя сила; он есть место нашего пребывания, и под ним вечные руки. Господь рядом с сокрушенными сердцем; он спасает тех, чей дух подобен духу ребенка. Много горестей у праведного, но Бог избавляет его от них от всех. Вверь путь свой Господу --- доверься ему --- и он исполнит его. Пребывающий под кровом Всевышнего покоится под сенью Всемогущего.
\vs p131 2:11 Возлюби ближнего своего как самого себя; не держи зла ни на кого; не делай никому того, что ты сам ненавидишь. Люби брата твоего, ибо Господь сказал: „Я буду любить детей моих крепко“. Дорога праведности подобна сияющему свету, который светит все ярче и ярче до совершенного дня. Мудрые воссияют подобно блеску небесного свода, и обратившие многих на путь праведности --- как звезды во веки веков. Пусть безнравственный оставит свой путь зла, а неправедный --- свои мятежные мысли. Говорит Господь: „Пусть они вернутся ко мне, и я в изобилии дарую им прощение“.
\vs p131 2:12 Говорит Бог, творец неба и земли: „Великий покой уготован тем, кто почитает закон мой. Вот заповеди мои: ты должен любить меня всем твоим сердцем; ты не должен иметь других богов, кроме меня; ты не должен упоминать имени моего всуе; помни день субботний, чтобы святить его; почитай отца и мать твоих; ты не должен убивать; ты не должен прелюбодействовать; ты не должен красть; ты не должен лжесвидетельствовать; ты не должен желать чужого имущества“.
\vs p131 2:13 И всем любящим Господа больше всего на свете и любящим ближних своих как самих себя, Бог неба говорит: „Я спасу тебя от могилы, я выкуплю тебя от смерти. Я буду милостив к детям твоим и справедлив к ним. Разве не говорил я о моих творениях на земле --- вы сыновья Бога живого? И разве не любил вас вечной любовью? Разве я не звал вас стать такими, как я, и вечно пребывать со мной в Раю?“»
\usection{3. Буддизм}
\vs p131 3:1 Ганид был потрясен, узнав, насколько близко буддизм подошел к тому, чтобы стать великой и прекрасной религией без Бога, без личного и вселенского Божества. Тем не менее, он нашел несколько рукописей, относящихся к более ранним верованиям, которые отражали некоторое влияние учения посланников Мелхиседека, продолжавших свою работу в Индии вплоть до времен Будды. Иисус и Ганид собрали следующие выдержки из буддийской литературы:
\vs p131 3:2 \P\ «Из чистого сердца радость устремится к Бесконечному; все мое существо будет в мире с этой радостью, не знающей смерти. Мою душу наполнило довольство, и мое сердце переполнено блаженством мирной веры. Во мне нет страха; я свободен от беспокойства. Я пребываю в безопасности, и мои враги не могут встревожить меня. Я доволен плодами своего доверия. Я обнаружил, что приближение к Бессмертному легко доступно. Я молюсь о вере, которая поддерживала бы меня в длительном путешествии; я знаю, что вера, ниспосланная свыше, не обманет моих ожиданий. Я знаю, что мои братья станут благоденствовать, если проникнутся верой Бессмертного, той самой верой, которая дарует скромность, честность, мудрость, храбрость, знание и стойкость. Давайте оставим печаль и отвергнем страх. Давайте с верой обратимся к истинной праведности и подлинному мужеству. Давайте научимся размышлять о справедливости и милосердии. Вера есть истинное богатство человека; это дар добродетели и славы.
\vs p131 3:3 Нечестивость презренна; грех презренен. Зло, содержится ли оно в мыслях, или выражается в делах, унижает. Боль и печали следуют вслед за злом так же, как пыль следует за ветром. Счастье и спокойствие ума следуют за чистыми мыслями и добродетельной жизнью, как тень за веществом материальных предметов. Зло есть плод неправильного направления мыслей. Это зло --- видеть грех там, где нет греха, и не видеть греха там, где он есть. Зло есть путь ложных доктрин. Те, кто избегают зла, видя вещи такими, как они есть, обретают радость, постигая истину. Положи конец своему несчастью, возненавидев грех. Когда взираешь на Возвышенного, отвернись от греха всем своим сердцем. Не оправдывай зла; не извиняй греха. Прикладывая усилия, чтобы исправить прошлые грехи, ты обретаешь силу противостоять им в будущем. Раскаяние рождает сдержанность. Не оставляй вины, в которой ты не исповедовался бы Возвышенному.
\vs p131 3:4 Бодрость и радость --- награда за дела, сделанные хорошо и во славу Бессмертного. Никто не может украсть у тебя свободу твоего собственного разума. Когда вера твоей религии освободила твое сердце, когда ум устойчив и незыблем, как гора, тогда покой души будет изливаться спокойно, подобно полноводной реке. Уверенные в том, что есть спасение, навсегда свободны от похоти, зависти, ненависти, обольщений богатства. Хотя вера есть энергия лучшей жизни, ты, тем не менее, должен настойчиво работать для своего спасения. Если ты убежден в своем конечном спасении, тогда удостоверься, что ты искренне стремишься быть праведным во всем. Развивай верность сердца, идущую изнутри, и так приди к тому, чтобы насладиться блаженством вечного спасения.
\vs p131 3:5 Ни один верующий, упорствующий в лени, праздности, ничтожестве, бездеятельности, бесстыдстве и эгоизме, не может надеяться, что достигнет просветления вечной мудрости. Но человек благоразумный, заботящийся о других, размышляющий, пылкий и искренний --- даже если он еще живет на земле --- может достичь верховнго просветления покоя и свободы божественной мудрости. Запомни, всякое деяние получит свою награду. Зло приводит к печали, а грех заканчивается болью. Веселье и счастье есть результат праведной жизни. Даже творящий зло наслаждается милосердием до времени, когда его злые дела окончательно созреют, но время полного сбора урожая злых дел неизбежно наступит. Пусть никто не думает легко о грехе, говоря про себя: „Наказание за неправедные дела не настигнет меня“. Суд мудрости таков: то, что ты делаешь, будет сделано тебе. Несправедливость, совершенная тобой по отношению к твоим братьям, вернется к тебе. Создание не может избегнуть последствия своих дел.
\vs p131 3:6 Глупец сказал в сердце своем: „Зло не застигнет меня врасплох“; но безопасность приходит лишь тогда, когда душа жаждет укорения, а ум ищет мудрости. Мудрый человек --- это благородная душа, которая полна дружелюбия среди врагов, спокойна среди буйствующих и щедра среди жадных. Любовь к себе подобна сорной траве в прекрасном поле. Себялюбие ведет к горю; бесконечные заботы убивают. Укрощенный ум приносит счастье. Он --- величайший воитель, кто побеждает и покоряет себя. Тот, кто сдержан во всем, хорош. Только тот является превосходной личностью, кто уважает добродетель и исполняет свой долг. Не позволяй гневу и ненависти властвовать над тобой. Не говори ни о ком резко. Довольство есть величайшее богатство. То, что дано мудро, хорошо сохраняется. Не делай другим того, что ты не хочешь, чтобы было сделано тебе. Плати добром за зло, преодолевай зло добром.
\vs p131 3:7 Праведная душа должна быть более желанной, чем власть над всей землей. Бессмертие есть цель искренности, а смерть --- конец безрассудной жизни. Те, кто искренни, не умрут, а безрассудные уже мертвы. Благословенны те, кто способны познать состояние бессмертия. Те, кто мучают живых, едва ли найдут счастье после смерти. Бескорыстные возносятся на небо, где наслаждаются блаженством бесконечной свободы и продолжают совершенствоваться в благородной щедрости. Каждый смертный, который мыслит праведно, говорит благородно и действует бескорыстно, будет наслаждаться добродетелью не только здесь, в этой краткой жизни, но после гибели тела будет продолжать наслаждаться радостью и на небе».
\usection{4. Индуизм}
\vs p131 4:1 Посланники Мелхиседека несли с собой учение о едином Боге повсюду, куда бы они ни направились. Многое из этой монотеистической доктрины оказалось, вместе с другими и некоторыми существовавшими ранее концепциями, воплощенным в последующих учениях индуизма. Иисус и Ганид сделали следующие выдержки:
\vs p131 4:2 \P\ «Он есть великий Бог, во всех отношениях верховный. Он есть Господь, который объемлет все сущее. Он создатель и управитель вселенной вселенных. Бог есть единственный Бог; он один и сам по себе; он --- единственный. И этот Бог есть наш Создатель и последний удел нашей души. Верховный излучает неописуемое сияние; он есть Свет Света. Каждое сердце и каждый мир освещен этим божественным светом. Бог --- наш защитник: он стоит рядом со своими созданиями, и те кто учатся узнавать его, становятся бессмертными. Бог есть великий источник энергии; он есть Великая Душа. Он повелитель всех вселенных. Этот единый Бог --- любящий Бог, великолепный и восхитительный. Наш Бог обладает верховной властью и пребывает в верховной обители. Его подлинный Лик вечен и божественен; он есть изначальный Господь небес. Все пророки приветствуют его, и он являет себя нам. Мы поклоняемся ему. О Верховный, источник всех созданий, Господь творения и правитель вселенной, открой нам, твоим созданиям, силу, посредством которой ты пребываешь везде! Бог создал небо и звезды; он светел, чист и существует сам по себе. Его вечное знание --- божественная мудрость. Вечный непроницаем для зла. Так как вселенная произошла от Бога, он правит ею по своим предначертаниям. Он есть причина творения, и поэтому всякая вещь порождена им.
\vs p131 4:3 Бог есть верное прибежище для каждого человека, оказавшегося в нужде; Бессмертный заботится обо всем человечестве. Спасение Бога незыблемо и доброта его милосердна. Он есть любящий защитник и благословенный хранитель. Говорит Господь: „Я обитаю в их собственных душах как лампада мудрости. Я величие величия и доброта добра. Там, где двое или трое собираются вместе, там и я с ними“. Создание не может избегнуть присутствия Создателя. Господом сочтен каждый вздох смертного; и мы поклоняемся этому божественному Сущему как нашему неразлучному спутнику. Он преобладает надо всем, он щедрый, вездесущий и бесконечно добрый. Господь есть наш правитель, прибежище и высший руководитель, и его изначальный дух обитает в каждой смертной душе. Вечный Свидетель порока и добродетели пребывает в сердце человеческом. Давайте предадимся долгому размышлению об обожаемом и божественном Животворящем; пусть его дух всецело управляет нашими мыслями. Веди нас из этого нереального мира в реальный! Веди нас от темноты к свету! Будь нам проводником от смерти к бессмертию!
\vs p131 4:4 С сердцами, очищенными чистым от всякой ненависти, давайте почитать Вечного. Наш Бог есть Господь молитвы; он слышит плач своих детей. Пусть все люди подчиняют свою волю ему, Непоколебимому. Давайте возрадуемся щедрости Господа молитвы. Сделай молитву твоим сокровенным другом и почитание прибежищем души своей. „Если только ты будешь почитать меня любовью“ --- говорит Вечный, --- „Я дам тебе мудрость постичь меня, ибо почитание меня есть добродетель, общая для всех созданий“. Бог рассеивает мрак светом и силой для тех, кто слаб. Бог есть наш друг сильный, и мы уже не имеем страха. Мы славим имя вовеки непобедимого Победителя. Мы почитаем его, потому что он есть верный и вечный помощник человека. Бог есть наш верный руководитель и неизменный наставник. Он великий Отец неба и земли, обладающий неограниченной энергией и бесконечной мудростью. Его великолепие величественно и его красота божественна. Он есть верховное прибежище вселенной и неизменный страж вечного закона. Наш Бог есть Господь жизни и Утешитель всех людей; он любит людей и помогает отчаявшимся. Он тот, кто дарует нам жизнь и Добрый Пастырь человеческой паствы. Бог есть наш отец, брат и друг. И мы жаждем узнать Бога внутри себя.
\vs p131 4:5 Мы научились добиваться веры стремлением наших сердец. Мы достигли мудрости, обуздывая наши чувства, и с помощью мудрости мы испытали мир в Верховном. Тот, кто полон веры, почитает истинно, когда его внутреннее я поглощено Богом. Наш Бог носит небеса как мантию; он также обитает в шести других бескрайних вселенных. Он верховный для всех и во всем. Мы молим Господа простить нас за все наши прегрешения против наших братьев; и мы простим нашему ближнему то зло, которое он сделал нам. Наш дух питает отвращение ко всему злу, поэтому, о Господи, освободи нас от всех пут греха. Мы молимся Богу как утешителю, защитнику и избавителю --- тому, кто любит нас.
\vs p131 4:6 Дух Хранителя Вселенной входит в душу обыкновенного создания. Тот человек мудр, кто почитает Единого Бога. Стремящиеся к совершенству дожны истинно знать Господа Верховного.Тот, кому ведома блаженная защита Верховного, никогда не знает страха, ибо Верховный говорит тем, кто служит ему: „Не бойся, ибо я с тобой“. Бог провидения есть Отец наш. Бог есть истина. И Бог желает, чтобы его создания поняли его --- достигли полного знания истины. Истина вечна; она поддерживает вселенную. Нашим верховным стремлением должно быть единство с Верховным. Великий Управитель есть создатель всего сущего --- все развивается из него. И вот суть долга: пусть никто не делает другому того, чего бы он не хотел для себя; не держи зла на другого, не ударь того, кто ударил тебя, побеждай гнев милосердием, изничтожай ненависть благожелательностью. И все это мы должны исполнять потому, что Бог есть добрый друг и милостивый отец, прощающий нам все наши земные проступки.
\vs p131 4:7 Бог есть наш Отец, земля есть наша мать, а вселенная есть наша родина. Без Бога душа --- пленница; знание Бога освобождает душу. Через размышления о Боге, через союз с ним приходит избавление от соблазнов зла и окончательное спасение от материальных оков. Когда человек свернет пространство, как кусок кожи, тогда придет конец злу, потому что человек найдет Бога. О Боже, спаси нас от тройной погибели ада --- вожделения, гнева и алчности! О душа, подготовься к духовной борьбе бессмертия. Когда придет конец смертной жизни, без колебаний покинь это тело ради лучшей и красивейшей формы и проснись в обители Верховного и Бессмертного, где нет страха, печали, голода, жажды и смерти. Знать Бога --- значит разорвать узы смерти. Знающая Бога душа всплывает на поверхности вселенной подобно тому, как сливки всплывают на молоке. Мы почитаем Бога, создателя всего, Великую Душу, вечно пребывающую в сердцах своих созданий. И тем, кто знает, что Бог возведен на престол сердца человеческого, предначертано стать такими, как он --- бессмертными. Зло должно остаться в этом мире, но добродетель следует за душой на небеса.
\vs p131 4:8 Только злой человек говорит: Вселенная не имеет ни истины, ни правителя; она предназначена только для утоления наших страстей. Такие души введены в заблуждение ограниченностью своего разума. Итак, они предаются утолению своих страстей и лишают свои души радостей добродетели и удовольствий праведности. Что может быть более великого, чем испытать избавление от греха? Человек, который видел Верховного, бессмертен. Друзья человека, живущие во плоти, не могут пережить смерть; только добродетель сопровождает человека, когда он совершает свой путь, восходя к радостным и залитым солнцем полям Рая».
\usection{5. Зороастризм}
\vs p131 5:1 Заратустра лично общался с потомками более ранних посланников Мелхиседека, и их доктрина единого Бога стала основной идеей религии, которую он основал в Персии. Не считая иудаизма, ни одна религия тех дней не содержала в большей степени этих учений Салема. Из рукописей этой религии Ганид сделал следующие выдержки:
\vs p131 5:2 \P\ «Все вещи исходят от Единого Бога и принадлежат Единому богу --- премудрому, благому, праведному, святому, великолепному и исполненному славы. Он, наш Бог, есть источник всякого света. Он есть Создатель, Бог всех благих целей и защитник справедливости во вселенной. Мудрость жизни состоит в том, чтобы действовать в согласии с духом истины. Бог всевидящ, и он знает и злые дела грешных, и добрые дела праведных; все зрит сверкающее око нашего Бога. Его прикосновение --- исцеляющее прикосновение. Господь есть всемогущий благодетель. Бог простирает свою щедрую руку как праведным, так и грешным. Бог основал мир и предопределил воздаяние за добро и за зло. Всемудрый Бог обещал вечность набожным душам, чьи помыслы чисты и дела праведны. Каковы твои верховные желания, таков будешь и ты сам. Мудрость, как солнце, озаряет тех, кто различает Бога во вселенной.
\vs p131 5:3 Воздавай хвалу Богу следуя воле Мудрого. Почитай Бога света радостным хождением по путям, указанным откровением его религии. Есть только один Верховный Бог, Господь Света. Мы почитаем его, сотворившего воды, растения, животных, землю и небо. Наш Бог есть Господь, самый благодетельный. Мы почитаем самого прекрасного, щедрого Бессмертного, обладающего вечным светом. Бог дальше всех от нас и в то же время ближе всех, ибо он пребывает в наших душах. Наш Бог есть божественный и святейший Дух Рая, и все же он более друг человеку, чем самые лучшие друзья из созданий. Бог --- величайший из всех помощников в величайшем из всех дел --- в познании его самого. Бог есть наш самый обожаемый и праведный друг; он есть наша мудрость, жизнь и сила души и тела. Через наши благие мысли мудрый Создатель даст нам исполнить его волю, достигая тем самым всего, что есть божественное совершенство.
\vs p131 5:4 Господи, научи нас прожить эту жизнь во плоти, готовясь к грядущей жизни духа. Говори с нами, Господи, и мы исполним твои повеления. Научи нас путям добра, и мы будем верно следовать по ним. Дозволь нам достигнуть единения с тобой. Мы знаем, что та религия истинна, которая ведет к праведности. Бог есть наша мудрая природа, лучшая мысль и праведное дело. Да одарит нас Бог единством с божественным духом и бессмертием в нем самом!
\vs p131 5:5 Эта религия Мудрого очищает верующего от всякого злого помысла и греховного дела. Я склоняюсь перед Богом неба в раскаянии, если я оскорбил его мыслью, словом или делом --- намеренно или ненамеренно, и я возношу молитвы о милости и хвалу за прощение. Я знаю, что, когда я исповедуюсь, грех будет снят с моей души, если я не собираюсь совершить злое дело снова. Я знаю, что прощение разрушает оковы греха. Совершающие зло будут наказаны, а следующие истине получат радость блаженства вечного спасения. Осени нас своей благодатью и снизошли спасительную силу нашим душам. Мы молим о милости, потому что стремимся к совершенству; мы хотим быть как Бог».
\usection{6. Судуанизм (джайнизм)}
\vs p131 6:1 Третья группа религиозных верующих, сохранивших в Индии доктрину единого Бога --- наследие учения Мельхиседека, --- была известна в те дни как судуанисты. Позже эти верующие стали известны как последователи джайнизма. Они учили:
\vs p131 6:2 \P\ «Господь Небес верховен. Совершающие грех не поднимутся ввысь, тем же, которые ходят путями праведности, найдется место на небе. Нам обеспечена грядущая жизнь, если мы знаем истину. Душа человека может подняться до высоты небес, чтобы развить там свою истинную духовную природу и так достичь совершенства. Небесное состояние освобождает человека от порабощения грехом и приводит его к окончательному блаженству. Праведный человек уже изведал освобождение от греха и все связанные с ним страдания. Собственное „Я“ есть непобедимый враг человека и проявляется в четырех величайших страстях человека: гневе, гордости, хитрости и жадности. Величайшая победа человека --- это победа над самим собой. Когда человек взирает на Бога с мольбой о прощении, и когда он обретает смелость радоваться такой свободе, он тем самым освобождается от страха. Человек должен пройти свой жизненный путь, обращаясь со своими собратьями так, как он хотел бы, чтобы они обращались с ним».
\usection{7. Синто}
\vs p131 7:1 Только недавно рукописи, относящиеся к этой дальневосточной религии, были помещены в Александрийскую библиотеку. Это была одна из мировых религий, о которой Ганид никогда не слышал. Эта вера тоже испытала влияние более ранних учений Мелхиседека, как это видно из следующих извлечений:
\vs p131 7:2 \P\ «Говорит Господь: „Все вы получаете мою божественную силу; все люди радуются моей милосердной помощи. Я радуюсь приумножению праведных на земле. И в красотах природы, и в добродетелях людей Принц Неба стремится открыть себя и явить свою праведную природу. Поскольку раньше люди не знали моего имени, я явил свое присутствие, родившись в мире в зримом образе, и вынес унижения, чтобы люди не забыли моего имени. Я творец неба и земли; солнце, и луна, и все звезды повинуются моей воле. Я правитель всех созданий на земле и в четырех морях. Хотя я великий и верховный, я все же откликнусь на молитву даже самого ничтожного из людей. Если какое\hyp{}либо создание поклоняется мне, я услышу его молитву и исполню желание его сердца“.
\vs p131 7:3 „Каждый раз, когда человек поддается тревоге, он отступает на шаг от водительства духа своего сердца“. Гордыня омрачает Господа. Если хочешь получить небесную помощь, отставь свою гордость; и капля гордости затмевает спасительный свет, как затмило бы его большое облако. Если ты не имеешь истины внутри себя, бесполезно молиться за то, что снаружи. „Если я слышу твои молитвы, то это потому, что ты предстал передо мной с чистым сердцем, свободным от лжи и лицемерия, с душой, которая отражает истину, как зеркало. Если ты хочешь достичь бессмертия, оставь мир и приди ко мне“».
\usection{8. Таоизм}
\vs p131 8:1 Посланцы Мелхиседека проникли далеко вглубь Китая; и доктрина единого Бога сделалась частью ранних учений некоторых китайских религий; одной из них, продолжавшей существовать дольше других и содержавшей многие монотеистические истины, был таоизм, и Ганид выбрал следующее из учений его основателя:
\vs p131 8:2 \P\ «Как чист и спокоен Верховный, и при этом как силен и могуч, как глубок и необъясним! Бог небес есть почитаемый нами предтеча всех вещей. Если ты знаешь Вечного, ты просвещен и мудр. Если ты не знаешь Вечного, тогда невежество выливается в зло и таким образом возникают греховные страсти. Это удивительное Существо было до того, как появились небеса и земля. Он поистине духовен; он один и он неизменен. Он воистину матерь земли, все творение движется вокруг него. Этот Великий отдает себя людям и тем дает им возможность становиться лучше и спасаться. Даже если человек мало знает, он все\hyp{}таки может ходить путями Верховного; он может подчиняться воле небес.
\vs p131 8:3 Все добрые дела истинного служения исходят от Верховного. Каждая вещь зависит от Великого Источника жизни. Великий Верховный не ищет похвалы за свои дары. Его власть превосходит все, но он остается скрыт от нашего взгляда. Он непрестанно меняет свои свойства, совершенствуя свои создания. Небесный Разум нетороплив и терпелив в своих замыслах, но уверен в их осуществлении. Верховный покрывает всю вселенную и всю ее поддерживает. Как велики и могущественны исходящие от него влияния и притягательная сила! Истинная доброта подобна воде; она все благословляет и не причиняет зла никому. И, подобно воде, настоящая доброта ищет самые низкие места, даже такие, которых другие избегают, и это потому, что она сродни Верховому. Верховный создает все вещи, питая их природу и совершенствуя их дух. И это тайна, как Верховный воспитывает, защищает и улучшает создание, не принуждая его. Он ведет и руководит, но не неволит. Он помогает развитию, но не господствует.
\vs p131 8:4 Мудрый человек делает свое сердце всеобъемлющим. Небольшое знание --- опасно. Те, кто стремятся к величию, должны учиться смирять себя. В сотворении Верховный стал матерью мира. Знать свою мать --- значит признавать свое сыновство. Тот человек мудр, который рассматривает все части с позиции целого. Веди себя по отношению к каждому человеку так, как если бы ты был на его месте. Вознаграждай несправедливость добротой. Если ты любишь людей, они потянутся к тебе --- и тебе не составит труда привлечь их.
\vs p131 8:5 Великий Верховный заполняет собой все; он слева и он справа; он поддерживает все творение и пребывает во всех истинных существах. Ты не можешь найти Верховного; но ты и не можешь пойти туда, где бы его не было. Если человек осознает зло своих путей и раскаивается в грехе от всего сердца, тогда может он снискать прощение; он может избежать наказания; вместо бедствия он получит благословение. Верховный есть охраняющее прибежище для всего творения; он защитник и спаситель человечества. Если ты каждый день ищешь его, ты найдешь его. Так как он может прощать грехи, он действительно дороже всего для всех людей. Всегда помни, что Бог награждает человека не за то, что он делает, а за то, каков он есть. Поэтому ты должен оказывать помощь своим собратьям без мысли о воздаянии. Делай добро без мысли о пользе для себя.
\vs p131 8:6 Те, кому известны законы Вечного, мудры. Незнание божественного закона есть несчастье и бедствие. Разум знающих законы Бога свободен от предрассудков. Если ты знаешь Вечного, даже когда твое тело исчезнет, душа продолжит существование в духовном служении. Ты поистине мудр, когда осознаешь свою ничтожность. Если ты пребудешь в свете Вечного, ты насладишься просветлением Верховного. Посвятившие себя служению Верховному испытывают радость в поисках Вечного. Когда человек умирает, его дух начинает свой длинный полет, совершая великий путь домой».
\usection{9. Конфуцианство}
\vs p131 9:1 Из великих мировых религий даже наименее признающая Бога подтверждает монотеизм посланников Мелхиседека и их верных последователей. Краткое изложение Ганидом конфуцианства состояло в следующем:
\vs p131 9:2 \P\ «То, что определено Небом, безошибочно. Истина реальна и божественна. Все происходит из Неба, и Великое Небо не делает ошибок. Небо назначило множество подвластных ему сил, чтобы помогать учить и возвышать низшие существа. Велик, очень велик Единый Бог, управляющий человеком с небес. Величественна сила Бога, и внушает страх его приговор. Но этот великий Бог даровал нравственное чувство даже многим низшим народам. Щедрость Неба никогда не иссякает. Благоволение есть лучший дар Неба человеку. Небо даровало свое величие душе человека; добродетели человека есть плод этого дара Небесного величия. Великое Небо видит все и сопричастно ко всем делам человека. И мы правы, когда называем Великое Небо нашим Отцом и нашей Матерью. Если мы таким образом являемся слугами наших божественных прародителей, мы можем тогда с доверием молиться Небу. Давайте во всякое время и во всем испытывать благоговейный трепет перед величием Неба. Мы признаем, о Боже, Всевышний и державный Властитель, что суд остается за тобой и что вся милость исходит из божественного сердца.
\vs p131 9:3 С нами Бог; поэтому нет страха в сердцах наших. Если найдется во мне какая\hyp{}нибудь добродетель, это проявление Неба, которое пребывает во мне. Но это Небо внутри меня часто многого требует от моей веры. Если со мною Бог, я решил избавиться от сомнений в сердце моем. Вера должна быть кратчайшим путем к истине, и я не понимаю, как человек может жить без этой доброй веры. Радость и горе не происходят с человеком без причины. Небо поступает с душой человека согласно своей цели. Когда ты обнаруживаешь свою неправоту, без колебаний признай ошибку и не медли с исправлением.
\vs p131 9:4 Мудрый человек занят поисками истины, а не только средств к существованию. Достичь совершенства Неба есть цель человека. Совершенному человеку дано управлять собой, и он свободен от беспокойства и страха. С тобою Бог; пусть не будет сомнений в твоем сердце. За каждое доброе дело воздается. Совершенный человек не ропщет на Небо и не испытывает недовольства людьми. Не делай другим того, чего ты не хочешь, чтобы сделали тебе. Пусть сострадание будет частью всякого наказания; прилагай все усилия, чтобы сделать наказание благодеянием. Таков путь Великого Неба. В то время как все существа должны умереть и вернуться к земле, дух благородного человека возносится вверх и восходит к свету славы совершенного сияния».
\usection{10. «Наша религия»}
\vs p131 10:1 Завершив нелегкую работу по созданию этого собрания учений мировых религий, имеющих отношение к Райскому Отцу, Ганид поставил себе целью сформулировать то, что в его понимании являлось изложением веры в Бога, к которой он пришел под влиянием учения Иисуса. Этот молодой человек имел обыкновение говорить об этих верованиях как о «нашей религии». Вот его записи:
\vs p131 10:2 \P\ «Господь Бог наш --- единый Господь, возлюби Господа Бога твоего всем сердцем твоим, и всем разумом твоим, и в то же время не щади усилий, чтобы возлюбить ближнего твоего, как самого себя. Этот единый Бог есть наш небесный Отец, и в Нем заключено все, и через свой дух он пребывает в каждой искренней человеческой душе. И мы, являющиеся детьми Бога, должны учиться, как вверять ему, истинному Творцу, наши души. С нашим небесным Отцом все возможно. Поскольку он есть Творец, создавший все вещи и всех существ, не может быть иначе. Хотя мы не можем видеть Бога, мы можем знать его. И проживая каждый день по воле нашего Отца на небе, мы можем открыть его нашим собратьям.
\vs p131 10:3 Божественные богатства природы Бога должны быть бесконечно глубоки и вечно мудры. Мы не можем отыскать Бога с помощью знания, но мы можем узнать его в наших сердцах из личного опыта. В то время как его справедливость может не быть понята, его милость может быть ниспослана самым ничтожным существам на земле. В то время как Отец наполняет собой всю вселенную, он живет и в наших сердцах. Разум человека является человеческим, смертным, но дух его --- божественный, бессмертный. Бог не только всемогущ, но и всеведущ. Если наши земные родители, имея склонность ко злу, знают, как любить своих детей и давать им даяния благие, тем более благой Отец на небесах лучше знает, как мудро любить своих детей на земле и даровать им соответствующие блага.
\vs p131 10:4 Отец на небесах не позволит ни одному из своих детей на земле погибнуть, если это дитя жаждет обрести Отца и действительно страстно стремится быть похожим на него. Наш Отец любит даже злых и всегда добр к неблагодарным. Если бы только больше человеческих существ могло знать о доброте Бога, они бы, безусловно, встали на путь раскаяния в своих злых делах и отказа от всех известных грехов. Все благое исходит от Отца света, в котором нет ни непостоянства, ни тени изменчивости. Дух истинного Бога присутствует в сердце человека. Он подразумевает, что все люди --- братья. Когда люди начинают искать Бога, это свидетельствует о том, что Бог нашел их, и они находятся в поисках знания о нем. Мы живем в Боге и Бог пребывает в нас.
\vs p131 10:5 Я не буду долее довольствоваться верой в то, что Бог есть Отец всего моего народа, отныне я буду верить в то, что он есть также и мой Отец. Я всегда буду стараться почитать Бога опираясь на Духа Истины, который есть мой помощник, с тех пор как я действительно узнал Бога. Но прежде всего я собираюсь явить свое почитание Богу тем, что научусь, как исполнять волю Бога на земле; то есть я намерен не щадить своих сил, чтобы обращаться с каждым из моих смертных собратьев именно так, как, по\hyp{}моему, Бог бы хотел, чтобы с ними обращались. И когда мы проведем нашу жизнь во плоти таким образом, мы можем многое просить у Бога, и он исполнит желания нашего сердца, чтобы мы могли быть наилучшим образом подготовлены к служению нашим собратьям. И все это преданное служение детям Бога увеличивает способность получать и испытывать небесные радости, возвышенные удовольствия, которые несет нам служение небесного духа.
\vs p131 10:6 Я буду каждый день благодарить Бога за его несказанные дары; я буду восхвалять его за его удивительные деяния для детей человеческих. Для меня он --- Всемогущий, Творец, Сила и Милость, но, более всего, он мой духовный Отец, и, как его земной сын, я когда\hyp{}нибудь встречусь с ним. А мой наставник сказал, что, ища его, я стану таким, как он. Верой в Бога я достиг мира с ним. Эта наша новая религия исполнена радости и порождает бесконечное счастье. Я уверен, что я буду веровать до самой смерти и что я обязательно получу венец вечной жизни.
\vs p131 10:7 Я учусь удостоверять все вещи и твердо держаться тех, которые хороши. Я буду делать для моих собратьев все то, что я бы хотел, чтобы люди делали для меня. Благодаря этой новой вере я знаю, что человек может стать сыном Бога, но иногда это ужасает меня, когда я начинаю задумываться над тем, что все люди --- братья, но это должно быть истиной. Я не вижу, как я могу наслаждаться отцовством Бога, в то время как я отказываюсь принять братство людей. Всякий, кто призывает имя Господа, будет спасен. Если это истина, то все люди должны быть моими братьями.
\vs p131 10:8 Впредь я буду делать мои добрые дела втайне; также я буду молиться в основном в одиночестве. Я не буду судить, чтобы не быть несправедливым к моим собратьям. Я собираюсь научиться любить моих врагов; я еще не умею уподобиться Богу. Хотя я вижу Бога и в этих других религиях, я считаю, что в «нашей религии» он более прекрасен, любящ, милосерд, личностен и позитивен. Но более всего это великое и прекрасное Существо есть мой духовный Отец; я его дитя. И никакими другими средствами, кроме моего искреннего желания быть похожим на него, я не смогу найти его и вечно служить ему. Наконец у меня есть религия с Богом, чудесным Богом, и он есть Бог вечного спасения».
