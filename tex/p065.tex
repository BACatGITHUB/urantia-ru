\upaper{65}{Контроль эволюции свыше}
\vs p065 0:1 Основа эволюционирующей материальной жизни --- жизни до разума --- это формируется Мастерами\hyp{}Физическими Контролерами и наделяющими жизнь семью Духами\hyp{}Мастерами с активной помощью назначенных Носителей Жизни. Как результат скоординированных действий этого троичного творчества появляется способность организма к существованию разума --- материальные механизмы для разумной реакции на стимулы внешней среды и, позднее, на внутренние стимулы --- влияния, имеющие происхождение в самом разуме организма.
\vs p065 0:2 \P\ Таким образом, есть три отчетливых уровня создания жизни и эволюции:
\vs p065 0:3 \ublistelem{1.}\bibnobreakspace Область физической энергии --- создание вместилища разума.
\vs p065 0:4 \ublistelem{2.}\bibnobreakspace Служение разуму духов\hyp{}помощников, --- оказывающее влияние на духовные способности.
\vs p065 0:5 \ublistelem{3.}\bibnobreakspace Обретение разумом смертного человека дара духа, --- достигающее кульминации в пришествии Настройщика Мысли.
\vs p065 0:6 \P\ Механические\hyp{}необучаемые уровни реакций организма на среду являются сферой физических контролеров. Духи\hyp{}помощники разума активизируют и регулируют способные к адаптации, или немеханические\hyp{}обучаемые типы разума, те, у которых механизмы реакций организмов обладают способностью обучаться посредством опыта. И подобно тому, как духи\hyp{}помощники манипулируют таким образом потенциалами разума, так и Носители Жизни осуществляют (по собственному усмотрению) всеобъемлющий контроль над связанными со средой аспектами эволюционных процессов, вплоть до момента появления человеческой воли --- способности познать Бога и возможности выбирать почитание его.
\vs p065 0:7 Это объединенное действие Носителей Жизни, физических контролеров и духов\hyp{}помощников обусловливает ход эволюции органической жизни в населенных мирах. И поэтому эволюция --- на Урантии или где бы то ни было еще --- всегда целенаправленна и никогда не случайна.
\usection{1. Функции Носителей Жизни}
\vs p065 1:1 Носители Жизни наделены потенциалами метаморфозы личности, а этим обладают лишь немногие отряды созданий. Эти Сыны локальной вселенной способны функционировать в трех различных фазах существования. Они обычно исполняют свои обязанности как Сыны срединной фазы, которая является их исходным состоянием. Но на такой стадии существования Носители Жизни не могут действовать в электрохимических сферах как производители физических энергий и материальных частиц в единицы живого бытия.
\vs p065 1:2 Носители Жизни способны функционировать и функционируют на трех следующих уровнях:
\vs p065 1:3 \ublistelem{1.}\bibnobreakspace Физический уровень электрохимии.
\vs p065 1:4 \ublistelem{2.}\bibnobreakspace Обычная средняя фаза квази\hyp{}моронтического существования.
\vs p065 1:5 \ublistelem{3.}\bibnobreakspace Продвинутый полудуховный уровень.
\vs p065 1:6 \P\ Когда Носители Жизни готовы имплантировать жизнь, и ими уже выбраны места для этого, они призывают комиссию архангелов трансмутации Носителей Жизни. Эта группа состоит из десяти чинов различных личностей, включая физических контролеров и их сподвижников, которыми руководит глава архангелов, действующий в пределах своей компетенции по установлению Гавриила и с разрешения Древних Дней. Когда эти существа соответствующим образом включены в контур, они могут произвести такие модификации в Носителях Жизни, которые немедленно позволят им действовать на физических уровнях электрохимии.
\vs p065 1:7 После того как паттерны жизни сформулированы и материя должным образом организована, сверхматериальные силы, участвующие в распространении жизни, тотчас же становятся активными, и возникает жизнь. После этого Носители Жизни немедленно возвращаются в свою нормальную срединную стадию личностного существования --- в состояние, в котором они могут манипулировать единицами жизни и направлять эволюционирующие организмы, даже если они лишаются всякой способности формировать --- создавать --- новые паттерны живой материи.
\vs p065 1:8 После того как органическая субстанция прошла определенный эволюционный путь и в наивысших эволюционирующих организмах появилась свободная воля человеческого типа, Носители Жизни должны либо покинуть планету, либо принять клятву самоотречения; это означает, что они должны дать обет воздерживаться от всяческих попыток влиять в дальнейшем на ход органической эволюции. И когда такие клятвы добровольно принесены теми Носителями Жизни, которые предпочли остаться на планете как будущие советчики тех, кому будет поручено воспитывать эти новые развивающиеся создания, обладающие волей, тогда созывается комиссия двенадцати, возглавляемая главой Вечерних Звезд, уполномоченного действовать Владыкой Системы и с разрешения Гавриила. И только тогда эти Носители Жизни немедленно переходят в третью фазу личностного существования --- полудуховный уровень бытия. И я действую на Урантии в этой третьей фазе существования со времен Андона и Фонты.
\vs p065 1:9 Мы ждем того времени, когда вселенная будет установлена в свете и жизни, и мы, возможно, обретем четвертую стадию бытия, в которой мы будем полностью духовными. Но нам никогда не было открыто, каким образом мы можем достичь этого желанного и продвинутого состояния.
\usection{2. Панорама эволюции}
\vs p065 2:1 История восхождения человека от морской водоросли до царя земли --- это поистине роман биологической борьбы и выживания разума. Изначальными предками человека в буквальном смысле были слизь и ил океанского дна в застойных и тепловодных заливах и лагунах протяженной береговой линии древних внутренних морей, те самые воды, в которых Носители Жизни осуществили три независимые имплантации жизни на Урантии.
\vs p065 2:2 Очень немногие виды ранних типов морской растительности, участвовавшие в этих эпохальных изменениях, которые привели к возникновению промежуточных организмов, подобных животным, все еще существуют сегодня. Стоящие между растениями и животными, губки как раз и являются одним из сохранившихся организмов раннего типа, через которые и произошел \bibemph{постепенный} переход от растений к животным. Эти ранние переходные формы, хотя и не идентичны современным губкам, но очень на них похожи; они были поистине пограничными организмами --- не растение, не животное --- но именно они в конечном итоге привели к развитию настоящих животных форм жизни.
\vs p065 2:3 Бактерии, простые растительные организмы очень примитивной природы, весьма мало изменились со времени зарождения жизни; они даже частично регрессировали из\hyp{}за своего паразитического поведения. Многие из грибов также являют собой пример ретроградного движения в эволюции, являясь растениями, которые утратили свою способность производить хлорофилл, став более или менее паразитическими. Большинство болезнетворных бактерий и их дополнительные вирусные тела, на самом деле принадлежат к этой же группе изменившихся паразитических грибов. В течение прошедших веков все обширное царство растительной жизни развилось от тех же предков, от которых произошли и бактерии.
\vs p065 2:4 Вскоре появился, и появился \bibemph{внезапно,} более высокий одноклеточный тип животной жизни. И с этих отдаленных времен амеба, типичный одноклеточный животный организм, дошла до настоящего времени почти не изменившись. Она живет сегодня почти так же, как и тогда, когда она была последним и величайшим результатом эволюции жизни. Это крошечное создание и ее протозоа\hyp{}сородичи являются для животных созданий тем же, чем бактерии являются для царства растений, они демонстрируют способность выживания на первых ранних стадиях эволюции и дифференциации жизни, одновременно с \bibemph{неспособностью к последующему развитию.}
\vs p065 2:5 Вскоре ранние одноклеточные животные типы объединились в сообщества, сначала по типу вольвокса, а немного спустя --- по линии гидры и медузы. Еще позднее эволюционировали морские звезды, морские лилии, морские ежи, голотурии, многоножки, насекомые, пауки, ракообразные и близко родственные группы земляных червей и пиявок, за которыми вскоре последовали моллюски --- устрица, осьминог и улитка. Сотни и сотни видов возникали и исчезали; упомянуты только те, которые выжили в долгой, долгой борьбе. Эти непрогрессирующие экземпляры, вместе с появившимся позднее семейством рыб, сегодня представляют неизменившиеся виды ранних и низших животных --- те ветви древа жизни, которые не смогли прогрессировать.
\vs p065 2:6 \P\ Мир был подготовлен к появлению первых позвоночных животных, рыб. Из семейства рыб произошли две уникальные модификации: лягушка и саламандра. И именно с лягушки началась та серия прогрессивных дифференциаций животной жизни, высшей точкой которой, наконец, стал сам человек.
\vs p065 2:7 Лягушка --- один из самых ранних сохранившихся предков человеческой расы, но она также остановилась в развитии, существуя сегодня почти в том же виде, что и в те отдаленные времена. Лягушка --- единственный предковый вид ранних рас, живущий сейчас на земле. У человеческой расы нет сохранившихся предков между лягушкой и эскимосом.
\vs p065 2:8 \P\ Лягушки произвели рептилий --- огромное семейство животных, практически вымершее, но которое, перед исчезновением, дало начало всему семейству птиц и многочисленным отрядам млекопитающих.
\vs p065 2:9 Вероятно, наибольший качественный скачок во всей эволюции до человека произошел, когда рептилия стала птицей. Все современные типы птиц --- орлы, утки, голуби и страусы --- все они давным\hyp{}давно произошли от огромных рептилий.
\vs p065 2:10 Царство рептилий, произошедшее от семейства лягушек, сегодня представлено четырьмя сохранившимися отрядами: два непрогрессирующих --- змеи и ящерицы, вместе с их близкими сородичами --- аллигаторами и черепахами; один частично прогрессирующий --- семейство птиц, и один --- предки млекопитающих и предки по прямой линии человека. Но хотя все рептилии давно вымерли, массивность этих исчезнувших существ откликнулась эхом в слоне и мастодонте, а их необычные формы увековечены в прыгающих кенгуру.
\vs p065 2:11 \P\ На Урантии возникло только четырнадцать типов, рыбы были последним, и после птиц и млекопитающих новых классов не развилось.
\vs p065 2:12 \P\ От маленьких подвижных плотоядных динозавров\hyp{}рептилий, имевших сравнительно крупный мозг, \bibemph{неожиданно} выделились плацентарные млекопитающие. Эти млекопитающие развивались быстро и во многих различных направлениях, дав происхождение не только обычным современным сухопутным разновидностям, но и морским типам, таким как киты и тюлени, а также обитателям воздуха, таким как летучие мыши.
\vs p065 2:13 Таким образом человек эволюционировал от высших млекопитающих, произошедших преимущественно из \bibemph{западной имплантации} жизни в древние восточно\hyp{}западные защищенные моря. \bibemph{Восточная} и \bibemph{центральная группы} живых организмов рано и успешно прогрессировали в направлении до\hyp{}человеческих уровней животного существования. Но по прошествии веков в восточном центре развития жизни не был достигнут удовлетворительный уровень интеллектуального до\hyp{}человеческого статуса, ибо центр неоднократно нес невосполнимые потери высших типов зародышевой плазмы и вследствие этого навсегда утратил способность восстановить потенциальные возможности для развития человека.
\vs p065 2:14 Поскольку качество способности разума к развитию в этой восточной группе было столь явно ниже, чем в остальных двух группах, Носители Жизни с согласия своих руководителей так изменили окружающую среду, чтобы в дальнейшем ограничить эти низшие дочеловеческие линии эволюционирующей жизни. Внешне устранения этих низших групп созданий казались случайными, но на самом деле это была вполне целенаправленная акция.
\vs p065 2:15 Впоследствии в процессе эволюции интеллекта лемурные предки человека в Северной Америке стали значительно более умственно развиты, чем их сородичи в остальных регионах; именно это и подвигло их мигрировать из западного центра имплантации жизни через Берингов перешеек и далее по береговой линии в юго\hyp{}западную Азию, где они продолжали развиваться преимущественно под влиянием определенных черт центральной группы жизни. Человек, таким образом, сформировался из определенных западных и центральных жизненных линий, именно в регионе между центральной и ближневосточной части Азии.
\vs p065 2:16 Именно так имплантированная на Урантии жизнь эволюционировала до ледникового периода, когда впервые появился и начал свой насыщенный событиями планетарный путь сам человек. И это появление первобытного человека на земле во время ледникового периода отнюдь не было случайным; оно совершилось по плану. Холодный и суровый климат ледниковой эры во всех отношениях наилучшим образом отвечал цели формирования человеческого существа выносливого типа с даром необыкновенной выживаемости.
\usection{3. Содействие эволюции}
\vs p065 3:1 Вряд ли возможно объяснить современному разумному человеку многие из странных и явно гротескных событий раннего этапа эволюции. Все эти, казалось бы, странные изменения живых существ всегда происходили строго в соответствии с задуманным, но нам не разрешено произвольно вмешиваться в развитие паттернов жизни после того, как однажды они начали действовать.
\vs p065 3:2 \P\ Носители Жизни могут использовать любые возможные природные ресурсы, могут воспользоваться любыми или всеми случайными обстоятельствами, которые ускорят развитие эксперимента над жизнью, но нам не разрешается непосредственно вмешиваться или произвольно манипулировать характером или течением эволюции как растительной, так и животной.
\vs p065 3:3 Вас проинформировали, что смертные Урантии в конечном итоге появились в процессе развития примитивной лягушки, и что эта восходящая линия, которая потенциально присутствовала в единственной лягушке, едва не исчезла при определенных обстоятельствах. Но не следует делать вывод, что эволюция человечества прекратилась бы из\hyp{}за несчастного случая в этой точке. В тот же самый момент мы наблюдали и заботились о не менее чем тысяче различных и расположенных далеко друг от друга мутирующих линиях жизни, которые можно было бы направить в различные паттерны до\hyp{}человеческого развития. Эта конкретная предковая лягушка --- наша третья попытка, две предыдущих жизненные линии исчезли несмотря на все наши усилия их сохранить.
\vs p065 3:4 \P\ Даже если бы Андон и Фонта умерли, не произведя потомство, это задержало бы человеческую эволюцию, но не остановило бы ее. После появления Андона и Фонты и до того, как потенции животной жизни к человеческой мутации были исчерпаны, эволюционировало не менее семи тысяч перспективных линий, которые могли достичь в той или иной форме человеческого типа развития. И многие из этих лучших линий впоследствии были ассимилированы различными ветвями распространяющегося человеческого вида.
\vs p065 3:5 Задолго до прибытия на планету Материальных Сына и Дочери, реализаторов биологического подъема, человеческие потенции эволюционирующих животных видов были исчерпаны. Этот биологический статус животной жизни открыт Носителям Жизни явлением третьей фазы мобилизации духов\hyp{}помощников разума, которая автоматически происходит и сопутствует исчерпанию способности всей животной жизни дать начало мутантным потенциям до\hyp{}человеческих видов.
\vs p065 3:6 \P\ Человечество на Урантии должно решать свои проблемы развития видов смертного человека с тем человеческим материалом, которым оно располагает --- новых рас из до\hyp{}человеческих источников в будущие времена не появится. Но этот факт не препятствует возможности достичь намного более высоких уровней человеческого развития благодаря разумной заботе об эволюционном потенциале, который все еще присутствует в расах смертных людей. То же что мы, Носители Жизни, делаем для выхаживания и сохранения жизненных линий до появления человеческой воли, человек должен сделать сам для себя после того, как это событие свершилось и мы прекратили активно участвовать в эволюции. Вообще, судьба эволюции человека находится в его собственных руках, и научный подход рано или поздно заменит беспорядочное действие неконтролируемой природной селекции и случайного выживания.
\vs p065 3:7 И в продолжение темы содействия эволюции хотелось бы подчеркнуть, что однажды в отдаленном будущем, когда вы, может быть, будете прикреплены к отрядам Носителей Жизни, вам не раз представится прекрасный случай вносить свои предложения и любые возможные улучшения в планы и технику управления и трансплантации жизни. Будьте терпеливы! Если у вас есть хорошие идеи, если вам известны лучшие методы управления любой частью любой области вселенной, в грядущем у вас определенно будет возможность представить их вашим сподвижникам и собратьям\hyp{}администраторам.
\usection{4. События на Урантии}
\vs p065 4:1 Нельзя не учитывать и то, что Урантия была поручена нам как мир эксперимента над жизнью. На этой планете мы сделали шестидесятую попытку модифицировать и, если возможно, улучшить адаптированные для Сатании Небадонские формы жизни, и зафиксировано, что мы в конечном итоге создали многочисленные удачные модификации стандартных форм жизни. Конкретно: на Урантии мы выработали и успешно продемонстрировали не менее двадцати восьми деталий модификации жизни, которые будут использоваться во всем Небадоне во все будущие времена.
\vs p065 4:2 Но нигде в мире установление жизни никогда не бывает экспериментом в том смысле, что делается попытка осуществить что\hyp{}либо непроверенное или неизвестное. Эволюция жизни --- это процесс всегда прогрессивный, дифференциальный и изменчивый, но никогда не случайный, неконтролируемый, и никогда в полном смысле слова экспериментальный.
\vs p065 4:3 \P\ Многие аспекты человеческой жизни представляют многочисленные свидетельства, что феномен существования смертных был разумно спланирован, что органическая эволюция --- не просто космическая случайность. Поврежденная живая клетка способна выделять определенные химические вещества, которые позволяют так стимулировать и активизировать нормальные клетки, что последние немедленно начинают секрецию некоторых веществ, облегчающих процессы заживления раны; и в то же время эти нормальные и неповрежденные клетки начинают делиться --- фактически начинают работу по созданию новых клеток, чтобы заменить любые соседние клетки, которые могли быть разрушены вследствие несчастного случая.
\vs p065 4:4 Это химическое действие и реакция, участвующие в заживлении раны и размножении клеток, описываются формулой, охватывающей более ста тысяч фаз и черт возможных химических реакций и биологических последствий, которую выбрали Носители Жизни. Более полумиллиона определенных экспериментов было проведено Носителями Жизни в их лабораториях, прежде чем они, наконец, остановились на этой формуле для эксперимента с жизнью на Урантии.
\vs p065 4:5 Когда ученые Урантии больше узнают об этих целебных веществах, они станут более эффективно лечить раны и косвенно больше узнают о том, как предотвратить некоторые серьезные болезни.
\vs p065 4:6 С момента установления жизни на Урантии Носители Жизни усовершенствовали этот лечебный метод, который был введен в другом мире Сатании, где позволяет в большей степени облегчить боль и лучше контролировать способность деления ассоциированных нормальных клеток.
\vs p065 4:7 \P\ В эксперименте над жизнью Урантии было много уникальных моментов, но самые выдающиеся события --- это появление Андонической расы до эволюции шести цветных народов и затем одновременное появление Сангикских мутантов в одной семье. Урантия --- это первый мир в Сатании, где шесть цветных рас произошли от единственной человеческой семьи. Они обычно формируются в различных линиях от независимых мутаций в до\hyp{}человеческой животной ветви и, как правило, появляются на земле по одной и последовательно в течение долгого времени, начиная с красного человека, затем друг за другом расы всех цветов вплоть до синего.
\vs p065 4:8 Другим значительным отклонением от установленного порядка вещей было позднее прибытие Планетарного Принца. Как правило, принц появляется на планете примерно во время развития воли; и если бы следовали этому плану, Калигастия должен был бы прийти на Урантию уже при жизни Андона и Фонты, а не на пятьсот тысяч лет позднее, одновременно с возникновением шести Сангикских рас.
\vs p065 4:9 В обычном обитаемом мире явление Планетарного Принца произошло бы по запросу Носителей Жизни во время, или немного позднее появления Андона и Фонты. Но Урантия была выбрана как планета модифицированной жизни, и по предварительному решению, двенадцать наблюдателей\hyp{}Мелхиседеков были посланы как советчики Носителей Жизни и как наблюдатели за планетой до последующего прибытия Планетарного Принца. Эти Мелхиседеки пришли в то время, когда Андон и Фонта приняли свое решение, которое позволило Настройщикам Мысли пребывать в их смертных умах.
\vs p065 4:10 \P\ На Урантии усилия Носителей Жизни по улучшению форм жизни Сатании привели к созданию многих, на первый взгляд, бесполезных промежуточных форм жизни. Но уже полученных преимуществ достаточно, чтобы оправдать видоизменения стандартных форм жизни на Урантии.
\vs p065 4:11 Мы намеревались обеспечить раннее проявление воли в эволюционирующей жизни на Урантии, и мы добились успеха. Обычно воля не возникает до тех пор, пока цветные расы не просуществуют уже достаточно долгое время, как правило, она впервые появляется среди высших типов красного человека. Ваш мир --- это единственная планета в Сатании, где человеческий тип воли появился в расе, предшествующей цветным.
\vs p065 4:12 Стремясь обеспечить комбинацию и симбиоз наследственных факторов, которые в конечном итоге дали начало млекопитающим предкам человеческой расы, мы столкнулись с необходимостью допустить существование сотен и тысяч других, сравнительно бесполезных комбинаций и сочетаний наследственных факторов. Многие из этих на вид странных побочных продуктов наших усилий, без сомнения, привлекут ваше пристальное внимание, когда вы углубитесь в изучение планетарного прошлого, и я могу хорошо понять, какими странными, с ограниченной человеческой точки зрения, будут казаться некоторые из этих вещей.
\usection{5. Превратности эволюции жизни}
\vs p065 5:1 Источником огорчения для Носителей Жизни было то, что нашим особым усилиям модифицировать интеллектуальную жизнь на Урантии так помешали трагические обстоятельства, лежащие вне нашей компетенции: предательство Калигастии и срыв Адама и Евы.
\vs p065 5:2 Но на протяжении всего биологического процесса наше наибольшее разочарование было связано со значительным и неожиданным регрессом определенных примитивных форм растительной жизни до протохлорофильных уровней паразитических бактерий. Эта случайность в эволюции растительной жизни стала причиной многих тяжелых болезней у высших млекопитающих, особенно у более уязвимого человеческого вида. Когда мы столкнулись с этой запутанной ситуацией, то в чем\hyp{}то недооценили возникшие трудности, потому что знали, что последующее смешение с Адамической жизненной плазмой настолько усилит защитные свойства образующейся смешанной расы, что сделает ее практически невосприимчивой ко всем болезням, вызываемым растительными организмами. Но все наши надежды не оправдались из\hyp{}за несчастного срыва Адама.
\vs p065 5:3 Вселенная вселенных, включая этот маленький мир, называемый Урантия, управляется не только для того, чтобы вызвать наше одобрение, или для нашего удобства, и уж тем более не для того, чтобы удовлетворять наши прихоти и наше любопытство. Мудрые и всесильные существа, которые отвечают за управление вселенной, без сомнения, точно знают, чего они хотят; и поэтому Носителям Жизни подобает и разумным смертным надлежит терпеливо ждать и всеми силами способствовать правлению мудрости, царствованию власти и развитию прогресса.
\vs p065 5:4 Есть, конечно, определенное воздаяние за несчастья, например пришествие Михаила на Урантию. Но несмотря на все подобные обстоятельства, последующие небесные руководители этой планеты абсолютно уверены в окончательном торжестве эволюции человеческой расы и в конечном счете одобряют наши оригинальные планы и паттерны жизни.
\usection{6. Эволюционные методы жизни}
\vs p065 6:1 Невозможно одновременно точно определить правильное положение и скорость движущегося объекта; любая попытка определить одно, неизбежно вызывает неопределенность другого. С тем же типом парадокса сталкивается и смертный человек, когда проводит химический анализ протоплазмы. Химик может определить химический состав \bibemph{мертвой} протоплазмы, но не может распознать физическое строение либо динамическое функционирование \bibemph{живой} протоплазмы. Ученый всегда будет все ближе и ближе подходить к секретам жизни, но никогда не откроет их, и только потому, что он должен убить протоплазму для того, чтобы анализировать ее. Мертвая протоплазма весит столько же, сколько живая протоплазма, но это не одно и то же.
\vs p065 6:2 \P\ Живые создания и существа обладают врожденным даром приспосабливаться. В каждой \bibemph{живой} растительной или животной клетке, в каждом \bibemph{живом} организме --- материальном или духовном --- есть ненасытное, постоянно возрастающее стремление достичь абсолютного приспособления к среде, адаптации организма к постоянно развивающемуся процессу жизни. Эти перманентные усилия всех живых существ свидетельствуют об их врожденном стремлении к усовершенствованию.
\vs p065 6:3 Наиболее важным шагом в эволюции растительной жизни было развитие способности продуцировать хлорофилл, а вторым величайшим достижением было эволюционное превращение споры в сложное семя. Спора более эффективна как фактор размножения, но в ней отсутствуют потенции разнообразия и разносторонности, присущие семени.
\vs p065 6:4 Один из наиболее полезных и сложных эпизодов в эволюции высших типов животных состоял в развитии способности железа в циркулирующих кровяных клетках выполнять двойную функцию: переносить кислород и выводить двуокись углерода. И это свойство красных кровяных телец иллюстрирует, как эволюционирующие организмы способны адаптировать свои функции к варьирующей или изменяющейся среде. Высшие животные, включая человека, снабжают свои ткани кислородом за счет действия железа в красных кровяных тельцах, которое несет кислород живым клеткам и так же эффективно выносит двуокись углерода. Но и другие металлы также могут выполнять те же функции. Каракатица использует для этой цели медь, а асцидия --- ванадий.
\vs p065 6:5 Продолжительность таких биологических настроек иллюстрируется эволюцией зубов у высших млекопитающих Урантии; их число достигало тридцати шести у отдаленных предков человека, а затем началось адаптивное изменение до тридцати двух у древнего человека и его близких родственников. Сейчас человеческий вид медленно стремится к двадцати восьми. Процесс эволюции и адаптации по\hyp{}прежнему активно продолжается на этой планете.
\vs p065 6:6 Но многие на первый взгляд загадочные приспособления живых организмов обусловлены чисто химическими реакциями, то есть это всецело физические процессы. В каждый момент времени в кровяном потоке каждого человеческого организма существует возможность протекания до 15 000 000 химических реакций между выделенными гормонами дюжины беспротоковых желез.
\vs p065 6:7 \P\ Низшие формы растительной жизни полностью реагируют на физическую, химическую и электрическую окружающую среду. Но по мере восхождения по лестнице жизни, один за другим начинают служение разуму семь духов\hyp{}помощников, и разум становится все более приспособленным, творческим, координирующим и доминирующим. Способность животных адаптироваться к воздуху, воде и суше --- это не сверхъестественное дарование, это суперфизическое приспособление.
\vs p065 6:8 Физика и химия сами по себе не могут объяснить, как из первобытной протоплазмы ранних морей возникло человеческое существо. Способность учиться, память и дифференцированная реакция на среду --- это дар разума. Законы физики не поддаются обучению они непреложны и неизменны. Химические реакции не изменяются образованием; они единообразны и постоянны. Вне присутствия Неограниченного Абсолюта, электрические и химические реакции предсказуемы. Но разум может обогащаться посредством опыта, может обучаться, вырабатывая определенные привычки, манеру поведения в ответ на одни и те же воздействия.
\vs p065 6:9 Не обладающие интеллектом организмы реагируют на воздействие среды, но те организмы, которые реагируют на служение разума, могут приспосабливать саму среду и умело влиять не нее.
\vs p065 6:10 Физический мозг и связанная с ним нервная система обладают врожденной способностью отвечать на служение разума так же, как развивающийся разум личности обладает определенной врожденной способностью духовной восприимчивости и, таким образом, содержит возможности духовного развития и свершений. Интеллектуальная, социальная, моральная и духовная эволюции зависят от служения разуму семи духов\hyp{}помощников и их сверхфизических сподвижников.
\usection{7. Уровни эволюционирующего разума}
\vs p065 7:1 Семь духов\hyp{}помощников разума --- разносторонне одаренные служители разума для низших интеллектуальных существ в локальной вселенной. Этому чину разума оказывается содействие из центров локальной вселенной или какого\hyp{}то связанного с ним мира, но определяющее влияние на низшие функции разума оказывают столицы системы.
\vs p065 7:2 В эволюционирующем мире многое, очень многое, зависит от работы этих семи помощников. Но они --- служители разума и не связаны с физической эволюцией --- этой сферой действия Носителей Жизни. Тем не менее, полная интеграция этих духовных даров с установленной и естественной методикой развивающейся системы Носителей Жизни является причиной неспособности смертных распознавать в феномене разума что\hyp{}либо, кроме влияния природы и результата естественных процессов, хотя вы от случая к случаю сами оказываетесь в противоречивой ситуации, когда приходится объяснять все относящееся к естественным реакциям разума его материальностью. И если бы Урантия управлялась в большем соответствии с первоначальными планами, ваше внимание еще меньше бы привлекалось к феномену разума.
\vs p065 7:3 Семь духов\hyp{}помощников больше похожи на контур, чем на сущность, и в обычных мирах они соединены в контур с другими помощниками, функционирующими во всей локальной системе. На планетах экспериментов с жизнью, однако, они относительно изолированы. И на Урантии, из\hyp{}за уникальной природы ее паттернов жизни, низшие помощники испытывали намного больше трудностей в контакте с эволюционирующими организмами, чем в случае более стандартизированного типа дара жизни.
\vs p065 7:4 Опять же в среднем эволюционирующем мире семь духов\hyp{}помощников намного лучше синхронизированы с прогрессирующими стадиями развития животных, чем на Урантии. За единственным исключением помощники испытывали самые большие трудности в контактах с эволюционирующими разумами организмов Урантии, намного больше, чем они когда\hyp{}либо испытывали за все их функционирование во всей вселенной Небадона. В этом мире развилось много форм пограничных феноменов --- запутанных комбинаций механических\hyp{}необучаемых и немеханических\hyp{}обучаемых типов реакций организмов.
\vs p065 7:5 Семь духов\hyp{}помощников не устанавливают контактов с чисто механической стороной реакций организмов на среду. Такие предшествующие разуму реакции живых организмов относятся к чисто энергетическим сферам центров мощи, физическим контролерам и их сподвижникам.
\vs p065 7:6 Приобретение потенциала способности \bibemph{обучаться} посредством опыта обозначает начало функционирования духов\hyp{}помощников, и они действуют начиная с самого низшего разума примитивных и невидимых созданий до высочайших типов на шкале эволюции человеческих существ. Они являются источником и паттерном поведения, которое в противном случае казалось бы более или менее загадочным, и не до конца понятных быстрых реакций разума на материальную окружающую среду. Долго должны эти верные и всегда надежные помощники продолжать свое предварительное служение, пока, наконец, животный разум не приобретет человеческие уровни духовной восприимчивости.
\vs p065 7:7 Помощники действуют исключительно в процессе эволюции познающего на опыте разума до достижения уровня шестой фазы, духа почитания. На этом уровне происходит неминуемое перекрывание служения --- феномен высшего нисходит вниз, чтобы скоординироваться с низшим в преддверии последующего достижения более высоких уровней развития. И кроме того, дополнительное духовное служение сопутствует действиям седьмого, и последнего помощника, духа мудрости. На протяжении служения духовного мира индивидуум никогда не испытывает резкой смены в сотрудничестве духов; эти изменения всегда постепенны и взаимны.
\vs p065 7:8 Всегда следует разделять сферы физической (электрохимической) и ментальной реакции на воздействие среды, и в свою очередь все они должны различаться как явления, отличные от духовных действий. Сферы физического, ментального и духовного тяготения являются различными областями космической реальности, невзирая на их тесные взаимоотношения.
\usection{8. Эволюция во времени и пространстве}
\vs p065 8:1 Время и пространство неразрывно связаны; это исходная связь. Задержка во времени неизбежна при наличии определенных космических условий.
\vs p065 8:2 Если покажется странным, что так много времени затрачивается на эволюционные изменения в развитии жизни, то я хотел бы сказать, что мы не можем заставить жизнь развертываться быстрее, чем позволяют физические метаморфозы планеты. Мы должны ждать естественного, физического развития планеты; мы абсолютно не контролируем геологическую эволюцию. Если позволят физические условия, мы можем произвести полную эволюцию жизни за значительно более короткое время, чем один миллион лет. Но мы все находимся под юрисдикцией Верховных Правителей Рая, а в Раю время не существует.
\vs p065 8:3 Индивидуальная мера измерения времени --- это продолжительность нашей жизни. Все создания, таким образом, находятся в зависимости от времени, поэтому они считают эволюцию очень продолжительным процессом. Для тех из нас, чьи продолжительность жизни не ограничивается временным существованием, эволюция не кажется таким уж протяженным действием. В Раю, где времени не существует, все это является \bibemph{настоящим} в разуме Бесконечности и действиях Вечности.
\vs p065 8:4 Поскольку эволюция разума зависит от и задерживается медленным развитием физических условий, то и духовный прогресс зависит от ментальных возможностей и неизменно задерживается замедленным развитием интеллекта. Но это не означает, что духовная эволюция определяется образованием, культурой или мудростью. Душа может развиваться вне зависимости от ментальной культуры, но не в отсутствии ментальной способности и желания --- выбора продолжения существования в посмертии и решения достигнуть постоянно возрастающего совершенства --- исполнять волю небесного Отца. Хотя продолжение существования в посмертии может не зависеть от обладания знанием и мудростью, прогресс наверняка зависит от них.
\vs p065 8:5 \P\ В космических эволюционных лабораториях разум всегда доминирует над материей и дух всегда коррелирует с разумом. Неудача в синхронизации и координации этих разнообразных даров может вызвать задержку во времени, но если индивидуум действительно знает Бога и желает найти его и стать подобным ему, тогда продолжение существования гарантировано независимо от препятствий времени. Физический статус может препятствовать разуму, и ментальное извращение может сдерживать духовные достижения, но ни одно из этих препятствий не может победить искренний выбор воли, сделанный от души.
\vs p065 8:6 Когда физические условия созрели, может иметь место \bibemph{внезапная} ментальная эволюция; когда ментальный статус благоприятен, может произойти \bibemph{внезапная} духовная трансформация; когда духовные ценности получают соответствующее признание, тогда космический смысл становится видимым и личность все более освобождается от препятствий времени и избавляется от ограничений пространства.
\vs p065 8:7 [Представлено Носителем Жизни Небадона, постоянно находящимся на Урантии.]
