\upaper{181}{Последние предостережения и предупреждения}
\author{Комиссия срединников}
\vs p181 0:1 После прощального разговора с одиннадцатью апостолами Иисус непринужденно беседовал с ними, вспоминал множество событий, которые случались в жизни и группы, и отдельных людей. Наконец\hyp{}то этим галилеянам становилось ясно, что их друг и учитель собирается их покинуть, и они с надеждой ухватилась за обещание, что он скоро будет с ними опять, но они были настроены забывать, что это возвращение также будет коротким. Многие из апостолов и лучших учеников действительно думали, что это обещание вернуться на короткое время (краткий промежуток времени между воскресением и вознесением) указывало на то, что Иисус просто уходит на непродолжительное свидание со своим Отцом, после которого он вернется, чтобы установить царство. Именно такое толкование его учения соответствовало и их предвзятым верованиям, и их пылким надеждам. Поскольку таким образом согласовывались верования всей их жизни и надежды на исполнение желаний, им было нетрудно найти такое толкование слов Учителя, которое оправдало бы их страстные желания\ldots
\vs p181 0:2 После того, как закончилось обсуждение прощальной речи и она начала укладываться в их умах, Иисус снова призвал их к порядку и стал изъявлять свои последние предостережения и предупреждения.
\usection{1. Последние слова утешения}
\vs p181 1:1 Когда одиннадцать апостолов заняли свои места, Иисус встал и обратился к ним: «Пока я с вами во плоти, я среди вас или во всем мире могу быть единственно как человек. Освободившись же от сего облачения смертной природы, смогу вернуться как дух, пребывающий в каждом из вас и во всех остальных верующих в сие евангелие царства. Таким образом Сын Человеческий станет духовным воплощением в душах всех истинно верующих.
\vs p181 1:2 Когда я вернусь пребывать в вас и через вас трудиться, тогда смогу лучше направлять вас в этой жизни и проведу вас через многие обители в будущей жизни на небе небес. Жизнь в вечном творении Отца вовсе не бесконечный покой в праздности и эгоистическом наслаждении, но беспрестанное совершенствование в благодати, истине и славе. Каждая из многих, многих обителей в доме Отца моего --- это привал, чтобы подготовиться к следующей жизни, ожидающей вас впереди. Итак, дети света пойдут от свершения к свершению, пока не достигнут божественного состояния, в котором они духовно совершенны, как совершен во всем Отец.
\vs p181 1:3 Если хотите идти за мной, когда я покину вас, приложите все усилия, дабы жить согласно духу моих учений и идеалу моей жизни --- исполняйте волю Отца моего. Сие делайте, но не пытайтесь подражать моей естественной жизни во плоти, которой мне в силу необходимости пришлось жить в этом мире.
\vs p181 1:4 В этот мир меня послал Отец, но лишь немногие из вас решили полностью меня принять. Я изолью дух мой на всякую плоть, но не все люди решат принять этого нового учителя как руководителя и наставника души. Однако все, кто примет его, просветятся, очистятся и утешатся. И сей Дух Истины станет в них колодезем воды живой, текущей в жизнь вечную.
\vs p181 1:5 Теперь же, готовясь покинуть вас, хочу сказать вам слова утешения. Мир оставляю вам; мир мой даю вам. Дары же эти даю не так, как дает мир --- не мерою, --- но каждому кто сколько сможет принять. Да не смущается сердце ваше и да не устрашается. Я победил мир, и все вы через веру восторжествуете во мне. Я предупредил вас о том, что Сын Человеческий будет убит, но уверяю вас: перед тем, как идти к Отцу, я вернусь, хотя и ненадолго. Когда же вознесусь к Отцу, обязательно пошлю нового учителя быть с вами и пребывать в сердцах ваших. И когда увидите, что сбылось все сие, не ужасайтесь, но верьте, ибо вы знали об этом наперед. Великой любовью любил я вас и не оставил бы вас, но такова воля Отца. Мой час настал.
\vs p181 1:6 Не сомневайтесь ни в одной из этих истин даже тогда, когда вас рассеют гонения и удручат многие печали. Когда почувствуете, что вы в мире одни, я буду знать об одиночестве вашем так же, как и вы, когда рассеетесь каждый в свою сторону и оставите Сына Человеческого в руках врагов его, будете знать о моем одиночестве. Но я никогда не одинок; Отец мой всегда со мной. Даже в такое время я буду молиться о вас. О всем же этом я сказал вам, дабы вы имели мир и имели в изобилии. В этом мире вы будете иметь скорбь; но мужайтесь; я победил мир и показал вам путь в вечную радость и бесконечное служение».
\vs p181 1:7 \pc Иисус дает мир тем, кто вместе с ним исполняет волю Бога, мир, однако, совсем не похожий на радости и удовольствия сего материального мира. Неверующие материалисты и фаталисты могут надеяться познать лишь два вида мира и душевного покоя: они должны быть либо стоиками и с непоколебимой решимостью смело в лицо смотреть неизбежному и преодолеть наихудшее, либо --- оптимистами, постоянно предающимися той надежде, что вечно возникает в человеческом сердце, тщетно ищущей мира, который по\hyp{}настоящему не приходит никогда.
\vs p181 1:8 Определенная доля и стоицизма, и оптимизма полезна для жизни на земле, но ни то, ни другое не имеет ничего общего с тем высшим миром, который Сын Человеческий дарует своим братьям во плоти. Мир, который дает Михаил своим детям на земле, есть тот самый мир, который наполнял его душу, когда он сам жил смертной жизнью во плоти в этом мире. Мир Иисуса есть радость и удовлетворенность познавшего Бога индивидуума, который одержал победу, до конца постигнув, как, живя смертной жизнью во плоти, исполнять волю Бога. Внутренний покой Иисуса был основан на абсолютной человеческой вере в реальность мудрого и милосердного попечительства божественного Отца. Иисус терпел беды на земле, он был даже ошибочно назван «мужем скорбей», однако во всех испытаниях этих и всегда он находил утешение в той вере, которая неизменно позволяла ему стремиться к цели своей жизни в полной уверенности, что он исполняет волю Отца.
\vs p181 1:9 Иисус был решителен, настойчив и абсолютно предан исполнению своей миссии, однако он не был бесчувственным и бессердечным стоиком; в своих жизненных треволнениях он всегда искал светлые стороны, но слепым и самообольщающимся оптимистом не был. Учитель знал обо всем, что ему предстояло, и был неустрашим. Даровав сей мир каждому из своих последователей, он мог со всем основанием сказать: «Да не смущается сердце ваше и да не устрашается».
\vs p181 1:10 Следовательно, мир Иисуса --- это мир и уверенность сына, полностью верующего, что его путь во времени и в вечности надежно и всецело принадлежит заботе и попечению всемудрого, вселюбящего и всемогущего духа Отца. И действительно, этот мир превосходит понимание смертного разума, но им в полной мере может обладать верующее человеческое сердце.
\usection{2. Прощальные личные предостережения}
\vs p181 2:1 Учитель окончил свои прощальные наставления и последние предостережения группе апостолов. Затем Иисус стал прощаться с каждым из них в отдельности, давая каждому личный совет и свое прощальное благословение. Апостолы по\hyp{}прежнему сидели вокруг стола, как сели вначале вкушать от Тайной Вечери, и по мере того, как Учитель, обходя вокруг стола, обращался к каждому из них, тот вставал.
\vs p181 2:2 \pc Иоанну Иисус сказал: «Ты, Иоанн, самый младший из братьев моих. Ты был мне очень близок, и хотя всех вас я люблю той же любовью, какую отец дарует сынам своим, ты был назначен Андреем одним из трех апостолов, кому надлежало всегда находиться рядом со мной. Помимо этого, ты замещал меня во многих вопросах, касающихся моей земной семьи, и должен поступать так же и впредь. Иоанн, я ухожу к Отцу в полной уверенности, что ты и дальше будешь заботиться о моих близких во плоти. Смотри, чтобы их сегодняшние заблуждения в отношении моей миссии никоим образом не помешали тебе проявлять к ним все сочувствие, на которое ты способен, давать им советы и помогать им, как, насколько тебе известно, поступал бы и я, если бы мне надлежало оставаться во плоти. Когда же все они увидят свет и полностью войдут в царство, хоть и все вы будете радостно приветствовать их, я поручаю тебе, Иоанн, приветствовать их вместо меня.
\vs p181 2:3 Теперь же, когда наступают последние часы моего земного пути, будь рядом со мной, чтобы я мог дать тебе поручение относительно моих родных. Что же касается дела, вверенного в мои руки Отцом, ныне оно завершилось, если не считать моей смерти во плоти, и я готов выпить сию последнюю чашу. Однако, что касается обязанностей, возложенных на меня моим земным отцом Иосифом, хоть я и исполнял их всю свою жизнь, но теперь должен положиться на тебя, поручив тебе действовать вместо меня во всех этих вопросах. Избрал же я тебя делать это для меня, Иоанн, потому, что ты самый молодой и, следовательно, вероятнее всего переживешь остальных апостолов.
\vs p181 2:4 Когда\hyp{}то мы называли тебя и твоего брата сынами грома. Ты начинал с нами упрямым и нетерпимым, но с тех пор, когда ты хотел, чтобы я низвел огонь на головы невежественных и безрассудных неверующих, ты сильно изменился. И тебе предстоит измениться еще больше. Ты должен стать апостолом новой заповеди, которую я дал вам в эту ночь. Посвяти свою жизнь учению братьев твоих, как любить друг друга так же, как возлюбил вас я».
\vs p181 2:5 Иоанн Зеведеев стоял в верхней комнате, и по щекам его текли слезы; он посмотрел Учителю в глаза и сказал: «Я так и сделаю Учитель, но как мне возлюбить братьев моих еще больше?» Тогда Иисус ответил: «Ты научишься больше любить своих братьев, научившись сначала больше любить Отца их небесного, и после того, как станешь воистину больше заботиться о их благоденствии во времени и в вечности. Всякой же подобной человеческой заинтересованности способствуют понимающее сочувствие, бескорыстное служение и безграничное прощение. Никто не должен относиться свысока к твоей молодости, но я призываю тебя всегда должное внимание уделять тому факту, что возраст зачастую равнозначен опыту и что ничто в делах людей не может заменить реальный жизненный опыт. Старайся жить мирно со всеми людьми, особенно со своими друзьями во братстве царства небесного. И всегда помни, Иоанн: не борись с душами, которые хочешь обратить к царству».
\vs p181 2:6 \pc И затем, проходя мимо своего места, Учитель на минуту задержался около места Иуды Искариота. Апостолы были весьма удивлены, потому что Иуда еще не вернулся, и очень хотели понять, почему столь печально было лицо Иисуса, когда тот стоял возле пустого места предателя. Однако никто из них, за исключением, быть может, Андрея, не допускал и малейшей мысли, что их казначей ушел предавать своего Учителя, как намекнул им чуть раньше во время вечери Иисус. Происходило столько необычного, что на какое\hyp{}то время они совершенно забыли слова Учителя о том, что один из них его предаст.
\vs p181 2:7 \pc Затем Иисус подошел к Симону Зилоту, который встал и выслушал такое предостережение: «Ты истинный сын Авраамов, тем не менее, как долго старался я сделать из тебя сына сего царства небесного. Я люблю тебя, и тебя любят все твои братья. Я знаю, что ты любишь меня, Симон, и что ты также любишь царство, но ты по\hyp{}прежнему настроен переделать это царство на свой собственный лад. Я прекрасно знаю, что в конце концов ты поймешь духовную суть и смыл моего евангелия и будешь самоотверженно трудиться, возвещая его, но я беспокоюсь о том, что может случиться с тобою, когда я уйду. Я буду рад узнать, что ты не дрогнешь; я буду счастлив, если смогу узнать, что, когда я уйду, ты не перестанешь быть моим апостолом и будешь достойно вести себя как посланец царства небесного».
\vs p181 2:8 Не успел Иисус окончить свое обращение к Симону Зилоту, как пламенный патриот, осушив слезы, воскликнул: «Учитель, за свою верность я не боюсь. Я отказался от всего ради того, чтобы посвятить свою жизнь установлению твоего царства на земле, и не отступлю. До сих пор мне удавалось пережить все разочарования, и я не оставлю тебя».
\vs p181 2:9 Тогда, положив руку на плечо Симона, Иисус сказал: «Воистину радостно слышать, что ты так говоришь, особенно в такое время, как это, однако, мой добрый друг, ты по\hyp{}прежнему не знаешь, о чем говоришь. Я ни минуты не сомневаюсь в твоей верности, в твоей преданности и знаю, что ты без колебаний пошел бы на бой и умер бы за меня, как поступили бы и все эти остальные» (и все апостолы энергично закивали в знак согласия), «но этого от тебя не потребуется. Я многократно говорил вам, что царство мое не от мира сего и что ученики мои не будут сражаться, дабы его установить. Я говорил тебе это множество раз, Симон, но ты отказываешься смотреть истине в лицо. Меня не беспокоит твоя верность мне и царству, однако что ты будешь делать, когда я уйду, и ты наконец очнешься и осознаешь, что не сумел понять смысл моего учения и должен изменить свои неправильные представления в соответствии с реальностью иного, духовного порядка вещей в царстве?»
\vs p181 2:10 Симон хотел было говорить дальше, но Иисус, остановив его движением руки, продолжал: «Среди моих апостолов нет никого искреннее и честнее сердцем, чем ты, но ни один из них после моего ухода не будет столь расстроен и не придет в такое же уныние, как ты. В этом твоем унынии дух мой пребудет с тобой и братья твои, не оставят тебя. Не забывай, чему я учил вас об отношении гражданства на земле к сыновству в духовном царстве Отца. Хорошо обдумывай то, что я говорил вам об отдаче кесарю кесарева, а Богу --- божьего. Посвяти свою жизнь, Симон, тому, чтобы показать, сколь достойно может смертный человек исполнять мое повеление относительно того, что в одно и то же время можно признавать временный долг перед гражданскими властями и духовное служение в братстве царства. Если Дух Истины будет наставлять тебя, то между требованиями гражданства на земле и требованиями сыновства на небе не будет никаких противоречий, если только мирские правители не осмелятся потребовать от вас почитания и поклонения, которые подобают одному Богу.
\vs p181 2:11 \pc Итак, Симон, когда окончательно увидишь все это и, избавившись от своей подавленности, пойдешь, активно возвещая сие евангелие, никогда не забывай, что я был с тобой даже во все время твоего уныния и буду с тобою впредь до самого конца. Ты всегда будешь моим апостолом и когда станешь смотреть духовным зрением и полнее подчинишь свою волю воле Отца Небесного, тогда будешь снова трудиться в качестве моего посланца, и никто из\hyp{}за того, что ты недостаточно быстро постигал истины, которым я учил тебя, не отнимет у тебя полномочий, которыми я тебя наделил. Итак, Симон, еще раз предупреждаю тебя: взявший меч от меча и погибнет, а трудящийся в духе получит жизнь вечную в грядущем царстве и радость и мир в том царстве, которое есть сейчас. Когда же работа, порученная рукам твоим, закончится на земле, тогда ты, Симон, воссядешь со мною в царстве моем на небе. Ты действительно увидишь царство, которого так желал, но не в этой жизни. Продолжай верить в меня, в то, что я открыл вам, и получишь дар жизни вечной».
\vs p181 2:12 \pc Окончив говорить с Симоном Зилотом, Иисус подошел к Матфею Левию и сказал: «Тебе более не придется заботиться о пополнении апостольской казны. Скоро, очень скоро все вы рассеетесь, и у тебя не будет возможности обратиться за утешением и поддежкой ни к одному из братьев твоих. Идя вперед и проповедуя евангелие царства, тебе придется найти себе новых сподвижников. В процессе обучения вашего я посылал вас парами; теперь же, когда я вас покидаю, ты, оправившись от потрясения, дойдешь один до края земли, возвещая эту благую весть: возрожденные верой смертные есть сыны Бога».
\vs p181 2:13 Затем сказал Матфей: «Однако, Учитель, кто пошлет нас и как мы узнаем, куда идти? Должен ли Андрей показать нам дорогу?» Иисус же ответил: «Нет, Левий, Андрей более не будет руководить вами в провозглашении евангелия. Он действительно останется вашим другом и советником до того дня, когда придет новый учитель, --- тогда Дух Истины поведет каждого из вас трудиться во имя расширения царства. Многие перемены произошли с тобой с того дня в таможне, когда ты впервые решил следовать за мной; но куда больше перемен произойдет прежде, чем ты обретешь понимание братства, где в братском союзе нееврей сидит рядом с евреем. Однако продолжай со всей своей настойчивостью обращать своих еврейских братьев, пока полностью не удовлетворишься, и тогда обратись со всей своей энергией к неевреям. В одном можешь быть уверен, Левий: ты завоевал доверие и любовь своих братьев; все они любят тебя». (И все апостолы кивнули в знак согласия со словами Учителя).
\vs p181 2:14 «Левий, мне многое известно о твоих тревогах, жертвах и трудах ради восполнения казны, о чем не знают братья твои, и я радуюсь тому, что, хотя тот, кто носил суму, отсутствует, посланец\hyp{}мытарь здесь на моей прощальной встрече с вестниками царства. Я молюсь, чтобы ты мог распознать смысл моего учения глазами духа. Когда же новый учитель придет в сердце твое, продолжай следовать туда, куда поведет тебя он, и пусть братья твои --- и даже весь мир --- видят, что может сделать Отец для ненавистного сборщика налогов, который осмелился следовать за Сыном Человеческим и верить в евангелие царства. С самого начала я любил тебя, Левий, как любил этих остальных галилеян. Поэтому, прекрасно зная, что ни Отец, ни Сын не взирают на лица, смотри, не делай никаких различий между теми, кто через твое служение станет верующим в евангелие. Итак, Матфей, служение всей своей будущей жизни посвяти тому, чтобы показать всем людям, что Бог не взирает на лица; что в глазах Бога и во братстве царства все люди равны, все верующие --- сыны Бога».
\vs p181 2:15 \pc Затем Иисус подошел к Иакову Зеведееву, который молча стоял, пока Учитель, обращаясь к нему, говорил: «Иаков, однажды, когда ты и твой младший брат пришли ко мне, ища привилегий и отличий в царстве, и я сказал вам, что подобные почести дарует Отец, я спросил вас, сможете ли вы пить мою чашу, и вы оба ответили, что можете. Даже если вы не могли того тогда и не можете этого теперь, вас вскоре приготовит к подобному служению то, что вам предстоит пережить. Таким поведением ты в то время разгневал своих братьев. И если они еще не простили тебя до конца, то простят, когда увидят, как ты пьешь мою чашу. Каким бы ни было твое служение, коротким ли, долгим ли, будь внутренне терпелив. Когда же придет новый учитель, позволь ему научить тебя проявлять сострадание и ту полную сочувствия терпимость, которая рождается от высшего доверия мне и совершенного подчинения воле Отца. И всей жизнью своей являй единство человеческой любови и божественного достоинства ученика, познавшего Бога и верующего в Сына. И все, живущие такой жизнью, откроют евангелие даже тем, как они умрут. Ты и твой брат Иоанн пойдете разными путями, и один из вас сможет сесть со мной в вечном царстве намного раньше другого. Если ты поймешь, что истинная мудрость --- это не только смелость, но и осторожность, это тебе очень поможет. Ты должен научиться быть не только агрессивным, но и проницательным. Наступят те верховные мгновения, когда мои ученики без колебаний положат жизни свои за сие евангелие, однако в обычных обстоятельствах будет намного лучше умиротворять гнев неверующих, чтобы вы могли жить и продолжать проповедовать благую весть. Настолько, насколько это в твоих силах, живи на земле долго, чтобы твоя многолетняя жизнь была плодотворной и обратила множество душ к царству небесному».
\vs p181 2:16 \pc Окончив говорить с Иаковом Зеведеевым, Иисус, обойдя вокруг стола, подошел к тому его краю, где сидел Андрей, и, посмотрев своему верному помощнику в глаза, сказал: «Андрей, ты преданно представлял меня как действующий глава посланцев царства небесного. Хоть иногда ты сомневался и порой проявлял опасную робость, тем не менее всегда был неподдельно справедлив и в высшей степени честен в общении со своими сотоварищами. С момента посвящения тебя и твоих братьев в вестники царства, вы всегда сами решали организационные вопросы общины, если не считать того, что я назначил тебя действующим главой этих избранных. Ни в каком другом мирском вопросе я не предпринимал ничего, дабы управлять вашими решениями или влиять на них. Поступал же я так для того, чтобы обеспечить руководство в управлении всеми вашими последующими совместными совещаниями. В моей вселенной и во вселенной вселенных Отца моего во всем, что касается духовных вопросов, к нашим братьям\hyp{}сыновьям относятся как к личностям, однако во всем, что касается взаимоотношений в группе, мы неизменно обеспечиваем четкое руководство. Наше царство --- это царство порядка, и где взаимодействуют два или больше наделенных волей творения, там всегда должно быть руководящее начало.
\vs p181 2:17 Андрей, ты был назначен мною руководить братьями твоими и, таким образом, был моим личным представителем, теперь же, поскольку я вскоре покину вас и уйду к Отцу, я освобождаю тебя от твоих обязанностей во всех текущих организационных вопросах и административных делах. Отныне над братьями твоими ты не можешь вершить иной власти, кроме той, которую ты заслужил как духовный лидер и которую, следовательно, твои братья свободно признают за тобой. С этого часа ты не можешь вершить власть над своими братьями, если только они не вернут тебе подобных полномочий своим конкретным правомерным установлением после того, как я уйду к Отцу. Однако такое освобождение от обязанностей руководителя сей группы никоим образом не умаляет твоего нравственного обязательства делать все от тебя зависящее, дабы твердой и любящей рукой удерживать твоих братьев вместе во времена будущих испытаний, в те дни, которые должны пройти между моей смертью во плоти и посланием нового учителя, который будет жить в ваших сердцах и который окончательно наставит вас истине в полной мере. Готовясь покинуть вас, я хочу освободить тебя от административной ответственности, имевшей начало и силу, когда я был среди вас и как один из вас. Впредь над вами и среди вас я буду вершить только духовную власть.
\vs p181 2:18 Если братья твои желают оставить тебя своим советником, я повелеваю тебе во всех вопросах, как мирских, так и духовных, делать все от тебя зависящее, дабы укрепить мир и гармонию среди всех искренне верующих в евангелие. Посвяти остаток своей жизни тому, чтобы практически упрочить братскую любовь среди твоих собратьев. Будь добр к моим братьям во плоти, когда они всецело уверуют в сие евангелие; проявляй полную любви и беспристрастную преданность грекам на Западе и Авениру --- на Востоке. Хотя апостолы мои вскоре рассеются по всем четырем сторонам земли, возвещать благую весть о спасении в сыновстве к Богу, ты должен удерживать их вместе в эти тяжелые времена, которые скоро наступят --- в период суровых испытаний, когда вы должны будете научиться верить в это евангелие без моего личного присутствия, терпеливо ожидая пришествия нового учителя, Духа Истины. Итак, Андрей, хоть, возможно, тебе и не выпадет вершить великие с точки зрения людей дела, довольствуйся тем, что ты будешь учителем и советником тех, кто совершает подобное. Свершай свое дело на земле до самого конца, и ты продолжишь это служение в вечном царстве, ибо не говорил ли я вам множество раз, что есть у меня и овцы не сего стада?»
\vs p181 2:19 \pc Затем Иисус подошел к близнецам Алфеевым и, встав между ними, сказал: «Дети мои малые, вы --- одна из трех пар братьев, которые решили следовать за мной. Все вы шестеро успешно и мирно трудились со своими братьями, но никому не удавалось делать это лучше вас. Впереди у нас трудные времена. Возможно, вы не поймете всего, что произойдет с вами и с вашими собратьями, но никогда не сомневайтесь, что некогда вы были призваны трудиться во благо царства. Какое\hyp{}то время не будет людей, которыми нужно управлять, но не стоит впадать в уныние; когда дело всей вашей жизни завершится, я приму вас на небе, где вы во славе будете рассказывать о своем спасении воинствам серафимов и множеству высоких Сынов Бога. Посвятите свои жизни возвышению простого труда. Покажите всем людям на земле и ангелам небесным, как радостно и смело может смертный человек, будучи призван на какое\hyp{}то время трудиться в особом служении Богу, затем вернуться к делам прежних дней. Если на какое\hyp{}то время ваш труд во внешних делах царства должен завершиться, значит, вы должны вернуться к своим прежним трудам, но уже просвещенные опытом сыновства по отношению к Богу и возвышенные пониманием того, что для познавшего Бога нет такого понятия, как простой труд или мирская работа. Для вас, трудившихся вместе со мной, все стало священным, а все земные труды --- служением самому Богу Отцу. Когда же услышите известия о деяниях ваших бывших сподвижников\hyp{}апостолов, радуйтесь вместе с ними и продолжайте свой повседневный труд как это делают те, кто ожидает Бога и, ожидая, служит. Вы были моими апостолами и будете ими всегда, и я буду помнить вас в грядущем царстве».
\vs p181 2:20 \pc Затем Иисус подошел к Филиппу, который стоя выслушал следующие наставления своего Учителя: «Ты задавал мне множество неразумных вопросов, Филипп, но я сделал все, чтобы ответить на каждый из них, и теперь хочу дать ответ на последний из подобных вопросов, возникших в твоем честнейшем, но бездуховном уме. Все это время, пока я шел к тебе, не спрашивал ли ты себя: „Что буду делать, если Учитель уйдет и оставит нас в мире одних?“ О маловерный! И все же у тебя почти столько же веры, сколько у многих из братьев твоих. Ты был хорошим экономом, Филипп. И подвел нас всего несколько раз, причем одним из этих промахов мы воспользовались, дабы явить славу Отца. Ты исполнил обязанности эконома почти до конца. Вскоре тебе предстоит вплотную заняться делом, к которому ты был призван, --- к проповеди сего евангелия царства. Филипп, ты всегда хотел, чтобы тебе показали, и очень скоро увидишь великое. Было бы намного лучше, если бы ты через веру увидел все это, однако поскольку даже в своей близорукости ты был искренен, ты увидишь исполнение моих слов. Когда же будешь благословлен духовным зрением, иди вперед к труду своему, посвятив жизнь делу наставления человечества в поисках Бога и исканию вечных реалий глазами духовной веры, а не глазами материального разума. Помни, Филипп, тебе предстоит великая миссия на земле, ибо мир полон тех, кто смотрит на жизнь такими же глазами, как и ты. Тебе предстоит великий труд; когда же он совершится в вере, тогда ты придешь ко мне в мое царство и я с великой радостью покажу тебе то, чего не видел глаз, не слышало ухо и не воспринимал смертный ум. Тем временем стань в царстве духа чадом малым и позволь мне как духу нового учителя вести тебя в духовном царстве вперед. Таким образом я смогу сделать для тебя много того, чего не сумел сделать, когда был с вами как смертный царства. И всегда помни, Филипп: видевший меня видел Отца».
\vs p181 2:21 \pc Затем Иисус подошел к Нафанаилу. Когда Нафанаил встал, Иисус велел ему сесть и, расположившись рядом с ним, сказал: «Нафанаил, став моим апостолом, ты научился жить, не считаясь с предрассудками и быть чрезвычайно терпимым. Но тебе предстоит еще научиться гораздо большему. Для своих собратьев ты был благословением, ибо всегда увещевал их своей постоянной искренностью. Когда я уйду, может случиться, что твоя прямота будет мешать хорошим отношениям с твоими собратьями, как со старыми, так и с новыми. Ты должен понять, что выражение, пусть даже благой мысли, должно соответствовать интеллектуальному уровню и духовному развитию слушателя. В деле царства искренность тогда полезна, когда сочетается с благоразумной осторожностью.
\vs p181 2:22 Если ты научишься работать со своими собратьями, то сможешь достигнуть более постоянных свершений, однако если ты решишь отправиться на поиски думающих так же, как ты, в этом случае посвяти свою жизнь доказательству того, что знающий Бога ученик может стать строителем царства даже тогда, когда он в мире один и полностью изолирован от своих собратьев. Я знаю, ты будешь верен до конца, и однажды с радостью приглашу тебя к расширенному служению моему царству на небе».
\vs p181 2:23 Затем Нафанаил задал Иисусу такой вопрос: «С тех пор, как ты впервые призвал меня к служению сего царства, я всегда слушал твое учение, но, честно говоря, так и не могу понять до конца смысл всего, что ты говоришь нам. Я не знаю, чего нам ждать дальше, и думаю, что большинство моих собратьев точно так же смущены, но не решаются признаться в своем смятении. Можешь ли ты мне помочь?» Иисус, положив руку на плечо Нафанаила, сказал: «Друг мой, неудивительно, что ты запутался, пытаясь понять смысл моего духовного учения, ведь тебе так мешают твои же предубеждения, идущие от еврейской традиции, и так сбивает с толку твое упорное стремление толковать евангелие согласно учениям книжников и фарисеев.
\vs p181 2:24 Я прожил свою жизнь среди вас и многому учил вас изустно. Я сделал все возможное, чтобы просветить умы ваши и раскрепостить ваши души, и что вы не сумели почерпнуть из моих учений и моей жизни, то должны теперь приготовиться получить из рук лучшего из учителей --- практического опыта. Во всем в этом новом опыте, ныне ожидающем вас, я пойду перед вами, и Дух Истины будет с вами. Не бойся; что теперь не можешь понять, то новый учитель, придя, будет открывать тебе всю твою оставшуюся жизнь на земле и на всем протяжении воспитания твоего в эпоху вечности».
\vs p181 2:25 Затем, обращаясь ко всем, Иисус сказал: «Не пугайтесь того, что не можете понять всего смысла евангелия. Вы всего лишь конечные, смертные люди; я же учил вас бесконечному, божественному и вечному. Будьте терпеливы и ободритесь, ибо впереди у вас вечность для продолжения последовательного обретения опыта быть совершенным, как совершен Отец ваш в Раю».
\vs p181 2:26 \pc Затем Иисус подошел к Фоме, который стоя слушал, как тот говорил: «Тебе, Фома, часто не хватало веры; впрочем, в периоды сомнений тебе всегда хватало смелости. Я хорошо знаю, что лжепророки и лжеучителя тебя не обольстят. Когда я уйду, братья твои станут больше ценить твое критическое отношение к новым учениям. Когда же в грядущие времена все вы рассеетесь до края земли, помни, что ты по\hyp{}прежнему мой посланец. Посвяти свою жизнь великому делу и покажи, как критический материальный ум человека может восторжествовать над инерцией интеллектуального сомнения, когда окажется перед лицом проявления живой веры, как она выражается в опыте рожденных от духа мужчин и женщин, приносящих плоды духа в своих жизнях и любящих друг друга, как любил вас я. Я рад, Фома, что ты присоединился к нам, и знаю, что после краткого периода смятения ты продолжишь служение царству. Твои сомнения смущали твоих братьев, но никогда не беспокоили меня. Я уверен в тебе и пойду перед тобою даже в самые отдаленные края земли».
\vs p181 2:27 \pc Затем Учитель подошел к Симону Петру, который встал, когда Иисус к нему обратился: «Петр, я знаю, что ты любишь меня и посвятишь свою жизнь публичному провозглашению сего евангелия царства евреям и неевреям, однако меня огорчает, что годы столь тесного общения со мной не многое сделали для того, чтобы ты прежде, чем говорить, думал. Что же должен ты пережить прежде, чем научишься сдерживать уста свои? Сколько же бед ты причинил нам своими бездумными разговорами и самонадеянной самоуверенностью! И ты обречен причинить себе еще больше бед, если не справишься с этим недостатком. Ты знаешь, что твои братья любят тебя, несмотря на эту слабость, и должен также понять, что недостаток этот никоим образом не ослабляет моей любви к тебе, но он уменьшает твою успешность и не прекращает причинять тебе беды. Однако то, что ты переживешь этой же ночью, несомненно, окажет тебе большую помощь. То же, что я сейчас говорю тебе, Симон Петр, я говорю и всем братьям твоим, собравшимся здесь: этой ночью вы все подвергнетесь великой опасности впасть в сомнение относительно меня. Вы знаете, что написано: „Поражен будет пастырь, и рассеются овцы“. Когда меня не станет, возникнет огромная опасность, что некоторые из вас уступят сомнениям и оступятся из\hyp{}за того, что случится со мной. Но я обещаю вам сейчас, что вернусь к вам ненадолго, а потом предварю вас в Галилее».
\vs p181 2:28 Тогда, положив руку на плечо Иисуса, Петр сказал: «Не важно, если и все братья мои соблазнятся о тебе, я обещаю не впадать в сомнение, что бы ты ни делал. Я пойду за тобой, и если потребуется, за тебя умру».
\vs p181 2:29 Пока Петр стоял перед своим Учителем, весь дрожа от волнения и переполнявшей его подлинной любви, Иисус посмотрел прямо в его увлажнившиеся глаза и сказал: «Истинно, истинно говорю тебе, Петр, что в эту ночь прежде, нежели дважды пропоет петух, трижды или четырежды отречешься от меня. Таким образом, чему ты не сумел научиться из мирного общения со мной, тому научишься через многие беды и многие печали. Когда же ты действительно усвоишь сей необходимый урок, тогда ты должен будешь укреплять дух своих братьев и жить дальше жизнью, посвященной проповеди сего евангелия, хотя, быть может, попадешь в темницу и, возможно, последуешь за мной, заплатив верховную цену полного любви служения, в строительстве царства Отца.
\vs p181 2:30 Однако помни мое обещание: когда воскресну, буду с вами какое\hyp{}то время прежде, чем уйду к Отцу. И даже этой ночью буду молить Отца, чтобы он укрепил каждого из вас к тому, что теперь уже совсем скоро вы должны будете пережить. Я люблю всех вас любовью, которой Отец возлюбил меня, и поэтому вы должны любить друг друга, как я вас любил».
\vs p181 2:31 \pc И затем, пропев гимн, они отправились в лагерь на Масличную гору.
