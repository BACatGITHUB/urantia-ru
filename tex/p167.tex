\upaper{167}{Посещение Филадельфии}
\author{Комиссия срединников}
\vs p167 0:1 Когда речь идет о посещениях Иисусом и апостолами различных поселений, где трудились семьдесят вестников, следует помнить, что в течение всего периода служения в Перее, с Иисусом, как правило, было только десять апостолов, поскольку было принято оставлять в Пелле для наставления масс хотя бы двух апостолов. Пока Иисус готовился идти в Филадельфию, в лагерь в Пелле вернулись Симон Петр и его брат Андрей, чтобы учить собравшихся там людей. Не было ничего необычного в том, что, когда Учитель покидал лагерь в Пелле, за ним следовало от трехсот до пятисот жителей лагеря. Когда же он подходил к Филадельфии, его сопровождало уже более шестисот последователей.
\vs p167 0:2 Последнее путешествие с проповедями по Десятиградию не сопровождалось никакими чудесами, и, если не считать десяти прокаженных, не было совершено никаких чудес и в этой миссии в Перее. То был период, когда евангелие возвещалось с силой, без чудес и, как правило без личного участия Иисуса, равно как и его апостолов.
\vs p167 0:3 \pc Иисус и десять апостолов пришли в Филадельфию в среду 22 февраля и четверг и пятницу провели, отдыхая с дороги и от своих недавних трудов. Ночью в ту пятницу Иаков выступил в синагоге, а уже на следующий вечер состоялась общая встреча последователей. Они на нем радовались распространению евангелия в Филадельфии и окрестных селениях. Вестники Давида принесли известия о дальнейшем распространении царства по всей Палестине и хорошие новости из Александрии и Дамаска.
\usection{1. Завтрак с фарисеями}
\vs p167 1:1 В Филадельфии жил некий очень богатый и влиятельный фарисей, который принял учения Авенира и пригласил Иисуса в свой дом в субботу на завтрак. Уже было известно, что Иисус в это время будет в Филадельфии; поэтому из Иерусалима и других мест пришло множество народа, среди которого было немало фарисеев. Таким образом, около сорока из них и несколько законников были приглашены на этот завтрак, устроенный в честь Учителя.
\vs p167 1:2 Пока Иисус у двери беседовал с Авениром, хозяин занял свое место, и после того в комнату вошел член синедриона, один из видных фарисеев в Иерусалиме, и по своей привычке прямо направился к почетному месту слева от хозяина. Однако это место предназначалось Учителю, а место справа --- Авениру, поэтому хозяин жестом предложил иерусалимскому фарисею сесть на четыре места левее; сановник был сильно оскорблен, потому что не удостоился почетного места.
\vs p167 1:3 Вскоре все сидели и наслаждались общением между собой, ибо большинство из присутствовавших были учениками Иисуса или доброжелательно относились к евангелию. Лишь враги Иисуса обратили внимание на то, что он не совершил ритуального омовения рук перед тем как сесть за стол. Авенир же умыл руки в начале трапезы, но не делал этого во время перемены блюд.
\vs p167 1:4 Ближе к концу трапезы с улицы вошел человек, давно страдавший хронической водянкой. Это был верующий, недавно крещенный сподвижниками Авенира. Он не просил Иисуса об исцелении, но Учитель прекрасно знал, что больной пришел на завтрак, надеясь тем самым избежать толпы, обычно теснившейся вокруг Учителя и, таким образом, получить большую возможность привлечь его внимание. Этот человек знал, что в то время почти не совершалось чудес; однако в сердце своем рассудил, что его плачевное состояние, скорее всего, вызовет сострадание Учителя. И не ошибся, ибо, когда он вошел в комнату, и Иисус, и самодовольный фарисей из Иерусалима заметили его. Фарисей не замедлил высказать свое возмущение тем, что такому человеку позволили войти. Иисус же посмотрел на больного и улыбнулся такой доброй улыбкой, что тот подошел ближе и сел на пол. Когда трапеза заканчивалась, Иисус окинул взором остальных гостей, а затем, значительно посмотрев на больного водянкой, сказал: «Друзья мои, учителя в Израиле и ученые законники, я хочу задать вам вопрос: законно врачевать больных и страждущих в субботу или нет?» Однако присутствовавшие слишком хорошо знали Иисуса, и они молчали и на его вопрос не отвечали.
\vs p167 1:5 \pc Тогда Иисус подошел к месту, где сидел больной и, взяв того за руку, сказал: «Встань и иди своим путем. Ты не просил об исцелении, но я знаю желание сердца твоего и веру души твоей». До того, как человек вышел из комнаты, Иисус вернулся на свое место и, обращаясь к сидевшим за столом, сказал: «Такие дела Отец мой свершает не для того, чтобы привлечь вас в царство, а затем, чтобы открыть себя тем, кто уже в царстве. Вы можете понять, что подобное этому творит отец, ибо кто из вас, имея любимое животное, которое упадет в колодец в субботу, тотчас не пойдет и не вытащит его?» А поскольку никто не хотел отвечать ему и так как хозяин дома, очевидно, одобрял происходившее, Иисус встал и сказал всем присутствовавшим: «Братья мои, когда позовут вас на брачный пир, не садитесь на первое место, чтобы не случился кто из званных почетнее вас и не пришлось бы хозяину подходить к вам и просить уступить место этому почетному гостю. В этом случае вам придется со стыдом занять более низкое место за столом. И когда позовут вас на пир, мудро будет, придя к праздничному столу, искать последнее место и занять его, так чтобы, оглядев гостей, хозяин мог сказать тебе: „Друг мой, почему сидишь на последнем месте? Пересядь выше“. Таким образом, этому человеку будет почет перед другими гостями. Не забывайте, всякий, возвышающий сам себя, унижен будет, а всякий, поистине унижающий себя, возвысится. Поэтому, когда делаешь обед или ужин, не зови всегда друзей твоих, ни братьев твоих, ни родственников твоих, ни соседей богатых, чтобы они в ответ могли позвать тебя на свои пиршества и ты бы, таким образом, получил воздаяние. Но когда делаешь пир, зови иногда бедных, увечных и слепых. Так блажен будешь в сердце твоем, ибо знал ты, что хромые и увечные не могут воздать тебе за твое служение, полное любви».
\usection{2. Притча о большом ужине}
\vs p167 2:1 Когда Иисус закончил свою речь за трапезным столом фарисея, один из присутствовавших законников, желая разрядить тишину, бездумно сказал: «Блажен, кто вкусит хлеба в царстве Божьем» --- эта поговорка была распространена в те дни. И тогда Иисус рассказал притчу, которую даже дружественно настроенный хозяин дома был вынужден принять близко к сердцу. Иисус сказал:
\vs p167 2:2 «Один правитель устроил большой ужин и, пригласив много гостей, перед ужином разослал своих слуг сказать всем званным: „Идите, ибо уже все готово“. И начали все, как бы сговорившись, извиняться. Первый сказал: „Я купил землю, и мне нужно пойти и посмотреть его; прошу тебя, извини меня“. Другой сказал: „Я купил пять пар волов и должен пойти принять их; прошу тебя, извини меня“. И третий сказал: „Я женился и потому не могу прийти“. Поэтому слуги возвратились и доложили о сем господину своему. Услышав это, хозяин дома сильно разгневался и, повернувшись к своим слугам, сказал: „Я приготовил сей пышный пир; откормленную скотину заколол, и все готово для гостей моих, но они отвергли мое приглашение: каждый из них пошел по делам земель своих и товаров своих, показав неуважение даже к слугам моим, которые позвали их на пир мой. Посему скорее пойдите по улицам и переулкам города, по большим и малым дорогам и приведите сюда нищих и отверженных, слепых и увечных, чтобы были гости на брачном пиру“. И сделали слуги, как приказал их господин, но и тогда осталось еще место для гостей. Тогда сказал господин слугам своим: „Пойдите по дорогам и переулкам и по округе всей и убедите тех, кого встретите, прийти, чтобы наполнился дом мой. Объявляю, что никто из тех, кто сначала был зван, не вкусит от ужина моего“. И сделали слуги, как приказал господин их, и наполнился дом».
\vs p167 2:3 \pc И, услышав эти слова, они разошлись; каждый пошел к своему месту. По крайней мере один из усмехавшихся фарисеев, присутствовавших в то утро, понял значение этой притчи, ибо крестился в тот день и публично исповедал свою веру в евангелие царства. Авенир же в ту ночь на общем собрании верующих произнес проповедь, основанную на этой притче.
\vs p167 2:4 На следующий день все апостолы занялись философствованием, пытаясь истолковать смысл притчи о большом ужине. Хотя Иисус с интересом выслушал все эти разнообразные толкования, он наотрез отказался что\hyp{}либо добавить, чтобы помочь им осмыслить притчу. И сказал лишь: «Пусть каждый сам для себя и в душе своей найдет ее смысл».
\usection{3. Женщина с духом немощи}
\vs p167 3:1 Авенир договорился о проповеди Учителя в синагоге в эту субботу; это был первый раз, когда Иисус появился в синагоге после того, как все они были закрыты для его учений указом синедриона. При завершении службы Иисус посмотрел на стоявшую перед ним пожилую женщину с подавленным выражением на лице и скорченным телом. Эта женщина уже давно была скована страхом, и вся радость ушла из ее жизни. Сойдя с кафедры, Иисус подошел к ней и, коснувшись плеча ее скорченного тела, сказал: «Женщина, если только будешь верить, сможешь полностью освободиться от своего духа немощи». И эта женщина, которая была скорчена и скована угнетенным состоянием более восемнадцати лет, поверила словам Учителя и тотчас же благодаря вере выпрямилась. Увидев, что выпрямилась, эта женщина возвысила свой голос и славила Бога.
\vs p167 3:2 Несмотря на то, что болезнь этой женщины была чисто душевной, а ее согбенное тело --- следствием угнетенного состояния ее ума, народ подумал, что Иисус исцелил настоящее физическое расстройство. Хотя прихожане филадельфийской синагоги дружественно относились к учениям Иисуса, главный начальник синагоги был недружественно настроенным фарисеем. А поскольку он разделял мнение прихожан, считая, что Иисус исцелил физическое расстройство, и негодовал, потому что Иисус осмелился сделать подобное в субботу, он встал перед собранием и сказал: «Разве нет шести дней, в которые люди должны делать все дела свои? В эти рабочие дни и приходите исцеляться, а не в день субботний».
\vs p167 3:3 Когда этот недружественно настроенный начальник сказал это, Иисус вернулся на кафедру и сказал: «Зачем лицемерить? Не отвязывает ли каждый из вас вола своего от яслей в субботу и не ведет ли поить? Если такое служение допустимо в день субботний, то не должно ли и эту женщину, дочь Авраамову, которая была скована злом вот уже восемнадцать лет, освободить от уз сих и отвести испить вод свободы и жизни хотя бы и в этот субботний день?» И поскольку женщина продолжала славить Бога, его хулитель был посрамлен и собрание радовалось ее исцелению вместе с ней.
\vs p167 3:4 За свое публичное осуждение Иисуса в эту субботу главный начальник был смещен, а на его место поставлен последователь Иисуса.
\vs p167 3:5 \pc Иисус часто избавлял подобных жертв страха от их духа немощи, угнетенного состояния ума и оков смятения. Но народ считал, что все подобные болезни были либо физическими расстройствами, либо следствием одержимости злыми духами.
\vs p167 3:6 \pc В воскресенье Иисус снова учил в синагоге, и многие крестились у Авенира в полдень того дня в реке, которая протекала южнее города. На следующий день Иисус и десять апостолов хотели отправиться в обратный путь к лагерю в Пелле, но им помешало прибытие одного из вестников Давида, который принес Иисусу срочное послание от его друзей из Вифании, что близ Иерусалима.
\usection{4. Весть из Вифании}
\vs p167 4:1 Глубокой ночью в воскресенье 26 февраля в Филадельфию прибыл гонец из Вифании и принес послание от Марфы и Марии, в котором было сказано: «Господи, тот, кого ты любишь, тяжело болен». Весть эта была передана Иисусу в конце вечернего совещания и как раз тогда, когда он прощался с апостолами перед тем как идти спать. Сначала Иисус ничего не ответил. Наступил один из тех странных моментов, периодов, когда он, казалось, общается с кем\hyp{}то, находящимся вне и вдали него самого. Затем, посмотрев вверх, Иисус обратился к посланнику так, чтобы его слышали апостолы, и сказал: «Эта болезнь действительно не к смерти. Не сомневайтесь, ей можно воспользоваться для прославления Бога и возвышения Сына».
\vs p167 4:2 \pc Иисус очень любил Марфу, Марию и брата их Лазаря; любил их горячей любовью. Его первым и человеческим порывом было тотчас идти им на помощь, но другая идея осенила его богочеловеческий ум. Он уже почти оставил надежду на то, что еврейские предводители (в Иерусалиме когда\hyp{}либо примут царство, но по\hyp{}прежнему любил свой народ, и теперь у него возник замысел, дающий книжникам и фарисеям Иерусалима еще одну возможность принять его учения; и решил, согласно воле Отца, сделать этот последний призыв к Иерусалиму самым выдающимся и знаменательным наглядным свершением всего своего земного пути. Евреи оставались верны представлению об Избавителе, творящем чудеса. И хотя Иисус отказывался снисходить до свершения материальных чудес или осуществления мирских проявлений политической власти, теперь он просил согласия Отца проявить свою до сих пор неявленную власть над жизнью и смертью.
\vs p167 4:3 \pc Евреи имели обыкновение хоронить своих мертвых в день их кончины; в таком теплом климате обычай этот был необходим. Часто случалось даже, что они относили в гробницу того, кто просто находился в коматозном состоянии, иной раз случалось, что на второй или даже на третий день такой человек выходил из гробницы. Однако евреи верили, что, хотя дух или душа и могут оставаться около тела два или три дня, но никогда не задерживаются дольше третьего дня; что к четвертому дню наступает полное разложение и что никто и никогда не возвращался из гробницы по истечении такого срока. Именно по этим причинам Иисус и задержался в Филадельфии на целых два дня перед тем, как собрался отправиться в Вифанию.
\vs p167 4:4 \pc Поэтому рано утром в среду он сказал своим апостолам: «Собирайтесь и пойдем опять в Иудею». Услышав, что Учитель сказал, апостолы на время удалились и держали совет друг с другом. Иаков задал тон беседы, и все согласились, что было бы просто неразумно позволить Иисусу опять идти в Иудею, и, вернувшись, все они объявили ему об этом. Иаков сказал: «Учитель, ты был в Иерусалиме несколько недель назад, и правители искали твоей смерти, а народ хотел побить тебя камнями. В тот раз ты дал этим людям возможность воспринять истину, и мы не позволим тебе опять идти в Иудею».
\vs p167 4:5 Тогда Иисус сказал: «Однако разве не понимаете вы, что во дне двенадцать часов, в которые уверенно можно делать дело? Если человек ходит днем, то не спотыкается, так как видит свет. Если же ночью ходит, может споткнуться, потому что нет света с ним. Пока продолжается день мой, я не боюсь войти в Иудею. Я сделаю еще одно великое дело для евреев; я предоставлю им еще одну возможность уверовать, причем даже на их условиях --- условиях внешней славы и зримого проявления силы Отца и любви Сына. А кроме того, разве не осознаете вы, что наш друг Лазарь уснул, но я хочу пойти и разбудить его ото сна?!»
\vs p167 4:6 Тогда один из апостолов сказал: «Учитель, если Лазарь уснул, то вернее всего выздоровеет». У евреев в то время был обычай говорить о смерти как о форме сна, но поскольку апостолы не поняли, что Иисус имел в виду то, что Лазарь покинул этот мир, он теперь сказал прямо: «Лазарь умер. И я радуюсь за вас --- даже если другие благодаря этому не спасутся --- что меня там не было, ибо у вас теперь будет еще одна причина уверовать в меня; благодаря же тому, чему вы станете свидетелями, вы все укрепитесь в подготовке к тому дню, когда я оставлю вас и уйду к Отцу».
\vs p167 4:7 Когда же они не смогли убедить его воздержаться от посещения Иудеи и когда некоторые из апостолов были не склонны даже сопровождать его, Фома обратился к своим собратьям и сказал: «Мы рассказали Учителю о наших опасениях, но он полон решимости идти в Вифанию. Я убежден, что это означает конец; они обязательно убьют его, но если таков выбор Учителя, то будем вести себя смело; пойдем и мы, умрем с ним». И так было всегда; в делах, требовавших разумного и твердого мужества, Фома был всегда главной опорой двенадцати апостолов.
\usection{5. По пути в Вифанию}
\vs p167 5:1 На пути в Иудею Иисуса сопровождали примерно пятьдесят человек --- друзей и недоброжелателей. В среду во время полуденной трапезы, он беседовал с апостолами и этими людьми об «Условиях Спасения» и в конце этой беседы рассказал притчу о фарисее и мытаре (сборщике налогов). Иисус сказал: «Как видите, Отец дает спасение детям человеческим, и это спасение --- бескорыстный дар всем имеющим веру принять сыновство в божественной семье. Человек не может ничего сделать, чтобы заслужить это спасение. Дела самодовольных не могут купить расположение Божье, и усердные молитвы на людях не искупят недостаток веры живой в сердце. Своим показным служением вы можете обмануть людей, но Бог смотрит в души ваши. То же, что я говорю вам, хорошо видно на примере двух человек, которые пришли молиться в храм, один из них был фарисей, а другой --- мытарь. Фарисей стоял и молился сам в себе: „Боже! Благодарю тебя, что я не таков, как прочие люди, грабители, невежественные, несправедливые, прелюбодеи или как этот мытарь. Пощусь два раза в неделю, даю десятую часть всего, что приобретаю“. Мытарь же, стоя вдали, не смел даже поднять глаз на небо, но, ударяя себя в грудь, говорил: „Боже! Будь милостив ко мне грешнику“. Сказываю вам, что мытарь, а не фарисей пошел домой с благоволением Божьим, ибо всякий, возвышающий сам себя, будет унижен, а унижающий себя возвысится».
\vs p167 5:2 \pc В ту ночь в Иерихоне недружественно настроенные фарисеи попытались заманить Учителя в ловушку, искушая его на беседу о браке и разводе, как уже однажды делали их собратья в Галилее, но Иисус искусно уклонился от их попыток вовлечь его в конфликт с их законами о разводе. Как притча о фарисее и мытаре показывает хорошее и дурное верование, так и их обычаи развода противопоставляли лучшим брачным законам еврейского кодекса позорную распущенность фарисейских толкований Моисеевых норм о разводе. Фарисей судил себя наименьшей мерой; мытарь же мерил себя по высочайшему идеалу. Набожность для фарисея была средством, стимулирующим самодовольную бездеятельность и гарантирующим ложную духовную безопасность; для мытаря же набожность была средством пробудить свою душу к осознанию нужды в покаянии, исповеди и принятию верой милосердного прощения. Фарисей искал справедливости, а мытарь --- милосердия. Закон вселенной гласит: просите, и получите; ищите, и найдете.
\vs p167 5:3 Хотя Иисус не дал втянуть себя в спор с фарисеями о разводе, он все\hyp{}таки возвестил позитивное учение о высочайших идеалах брака. Он превознес брак как самую идеальную и высочайшую форму человеческих отношений. Кроме того, он выразил серьезное порицание распущенных и несправедливых правил развода иерусалимских евреев, по которым в те времена допускалось, что человек может развестись со своей женой по самым пустячным причинам, таким как неумение готовить пищу, неправильное ведение домашнего хозяйства, или же просто потому, что он влюбился в более красивую женщину.
\vs p167 5:4 Фарисеи зашли даже так далеко, что учили, будто такой необременительный развод был особой привилегией, дарованной еврейскому народу и в частности фарисеям. Поэтому, хотя Иисус и отказался высказываться относительного брака и развода, он самым суровым образом осудил это постыдное глумление над брачными отношениями и отметил их несправедливость по отношению к женщинам и детям. Он никогда не одобрял какой бы то ни было обычай развода, который давал мужчине преимущества перед женщиной; Учитель одобрял лишь те учения, которые предоставляли женщине равные права с мужчиной.
\vs p167 5:5 Хотя Иисус не предлагал новых установлений, регулирующих вопросы брака и развода, он призывал евреев жить согласно их же собственным законам и высшим учениям. В своем стремлении усовершенствовать их обычаи, касавшиеся этих социальных институтов, он постоянно указывал на Писание. Таким образом, поддерживая высокие, идеальные представления о браке, Иисус искусно избегал столкновений с теми, кто задавал ему вопросы о социальных обычаях, либо записанных в их законах, либо представленных столь милыми их сердцу бракоразводными привилегиями.
\vs p167 5:6 Апостолам было очень трудно понять нежелание Учителя наставительно высказаться по научным, социальным, экономическим и политических проблемам. Они в принципе не понимали, что его земной путь был посвящен раскрытию исключительно духовных и религиозных истин.
\vs p167 5:7 В тот вечер, когда Иисус окончил свою речь о браке и разводе, позднее его апостолы наедине задали ему много дополнительных вопросов, и его ответы позволили им избавиться от многих неправильных представлений. В завершении этой беседы Иисус сказал: «Брак почетен и должен быть желаем всеми людьми. То, что Сын Человеческий исполняет свою земную миссию один, никоим образом не отрицает желательность брака. То, что я должен действовать именно так, является волей Отца, но сам же Отец повелел сотворить мужчину и женщину, и по божественной воле мужчина и женщина должны находить свое высочайшее служение и проистекающую из него радость в образовании семьи, рождении и воспитании детей, в сотворении которых родители становятся партнерами Творца неба и земли. Посему человек и должен оставить и отца своего, и мать и прилепиться к жене своей, и будут два одною плотью».
\vs p167 5:8 И таким образом Иисус избавил апостолов от многих беспокоящих их размышлений о браке и от неправильного понимания ими многих проблем развода; одновременно он немало сделал, дабы возвысить их идеалы общественного союза и укрепить их в уважительном отношении к женщине, детям и семье.
\usection{6. Благословение детей}
\vs p167 6:1 Произнесенное в ту ночь послание Иисуса о браке и благословенности детей распространилось по всему Иерихону, так что на следующее утро задолго до того, как Иисус и апостолы собрались уходить и даже еще до завтрака, к месту, где остановился Иисус, пришло множество матерей, которые принесли на руках или привели за руку своих детей, желая, чтобы он благословил малышей. Выйдя взглянуть на матерей с детьми, апостолы пытались отослать их, но женщины отказывались уходить, пока Учитель не возложит руки на детей и не благословит их. И когда апостолы стали громко упрекать матерей, Иисус, услышав шум, вышел и возмущенно упрекнул их, сказав: «Пустите детей, не препятствуйте им приходить ко мне, ибо таковых есть царство небесное. Истинно, истинно говорю вам, кто не примет царства Божьего, как дитя малое, тот едва ли войдет в него и не достигнет полной духовной зрелости».
\vs p167 6:2 И переговорив с апостолами, Учитель принял всех детей, возложил на них руки, а матерям сказал слова, вселяющие уверенность и надежду.
\vs p167 6:3 \pc Иисус часто беседовал с апостолами о небесных обителях и учил, что совершенствующиеся дети Бога должны взрастать в них духовно так же, как вырастают дети физически в этом мире. И как священное часто кажется обыденным, так и дети эти, и матери их в этот день мало сознавали, что наблюдавшие за ними разумные существа Небадона смотрели, как дети Иерихона играют с Творцом вселенной.
\vs p167 6:4 \pc Учение Иисуса существенно облегчило положение женщин в Палестине; и так было бы во всем мире, если бы его последователи не отошли столь далеко от того, чему он усердно учил их.
\vs p167 6:5 \pc В Иерихоне же в связи с обсуждением раннего религиозного воспитания детей в традициях божественного почитания, Иисус объяснил своим апостолам великое значение красоты как фактора, вызывающего стремление к почитанию, в особенности у детей. Учитель наставлением и примером учил ценности почитания Творца в естественном окружении творения. Он предпочитал общаться с Отцом Небесным среди деревьев и непритязательных тварей мира природы. Он наслаждался созерцанием Отца через вдохновляющую картину звездных миров Сынов\hyp{}Творцов.
\vs p167 6:6 Когда поклоняться Богу в природных кущах невозможно, человек должен сделать все от него зависящее, дабы создать прекрасные здания, святилища, полные подкупающей простоты и художественной красоты, так чтобы наряду с интеллектуальным стремлением к духовному общению с Богом пробуждались и возвышенные человеческие чувства. Истина, красота и святость --- вот мощные и эффективные помощники в истинном почитании. Однако духовному общению не способствует витиеватость и чрезмерное украшательство изысканных и нарочитых произведений человеческого искусства. Красота наиболее религиозна, когда она максимально проста и естественна. Как жаль, что дети получают первые впечатления о публичном поклонении в холодных помещениях, совершенно лишенных притягательной силы красоты и какого бы то ни было намека на веселье и вдохновляющую святость! Ребенок должен учиться поклонению на лоне природы, и только потом ходить со своими родителями в молитвенные дома, которые должны быть, по крайней мере, столь же привлекательны и художественно прекрасны, как и дом, в котором они живут каждый день.
\usection{7. Беседа об ангелах}
\vs p167 7:1 Когда, поднимаясь в горы, они направлялись из Иерихона в Вифанию, Нафанаил большую часть пути шел рядом с Иисусом, и их беседа о детях и царствие небесном постепенно подвела к разговору о служении ангелов. В конце концов Нафанаил задал Учителю такой вопрос: «Ввиду того, что первосвященник --- саддукей, и поскольку саддукеи не верят в ангелов, чему нам учить народ о небесных служителях?» Тогда, в частности, Иисус сказал:
\vs p167 7:2 \pc «Ангельские воинства --- это отдельный чин сотворенных существ; они во всем отличаются от материального чина смертных созданий и действуют как отдельная группа разумных существ вселенной. Ангелы не те создания, которые в Писании названы „Сынами Бога“, не являются они и прославленными духами смертных людей, продолжающих совершенствоваться в небесных обителях. Ангелы есть непосредственное творение и не воспроизводят сами себя. Ангельские воинства имеют лишь духовное родство с человечеством. По мере продвижения на пути к Райскому Отцу человек в определенное время пребывает в состоянии, аналогичном состоянию ангелов, однако смертный человек ангелом не становится никогда.
\vs p167 7:3 Ангелы никогда не умирают, как умирает человек. Ангелы бессмертны, если только не вовлекаются во грех, подобно тому, как некоторые из них вовлеклись в ложь Люцифера. Ангелы --- это духовные служители на небесах, и они не являются ни всемудрыми, ни всемогущими. Однако все верные ангелы истинно чисты и святы.
\vs p167 7:4 Разве не помнишь ты, как однажды я тебе сказал, что если бы духовные глаза твои были помазаны, ты бы увидел, как отверзлось небо, и узрел бы восходящих и нисходящих ангелов Божьих? Благодаря служению ангелов один мир и может поддерживать связь с другими мирами, ибо не говорил ли я тебе неоднократно, что есть у меня и другие овцы не сего стада. Ангелы эти --- не соглядатаи духовного мира, которые следят за тобой, а затем идут и рассказывают Отцу о помышлениях сердца твоего и доносят о делах плоти. Отец не имеет нужды в подобном служении, ибо его собственный дух живет внутри тебя. Но эти ангельские духи действуют, дабы сообщать одной части небесного творения о делах других, отдаленных частей вселенной. И многим из ангелов, действующих в правительстве Отца и во вселенных Сынов, поручено служение родам человеческим. Уча тебя тому, что многие из этих серафимов являются духами\hyp{}служителями, я не использую язык метафор или поэтические образы. И все это истинно, несмотря на трудности, которые ты испытываешь в понимании подобных вопросов.
\vs p167 7:5 Многие из этих ангелов участвуют в деле спасения людей, ибо не говорил ли я тебе о радости серафимов, когда хоть одна душа решает оставить грех и начать поиски Бога? Я даже говорил тебе о радости \bibemph{ангельского присутствия} на небесах и об одном грешнике кающемся и тем самым указал на существование иных и более высоких чинов небесных существ, которые, подобно тому, заинтересованы в духовном благоденствии и божественном совершенствовании смертного человека.
\vs p167 7:6 Кроме того, эти ангелы в значительной степени связаны со средствами, с помощью которых дух человека освобождается от сосуда плоти, а его душа вводится в небесные обители. Ангелы --- вот надежные небесные проводники души человека в течение того неизвестного и неопределенного периода времени, что лежит между смертью плоти и новой жизнью в духовных обителях».
\vs p167 7:7 \pc Иисус и дальше говорил бы с Нафанаилом о служении ангелов, но был прерван Марфой, которая узнала, что Учитель приближается к Вифании от друзей, заметивших как он поднимался по горам на востоке, и поспешила приветствовать его.
