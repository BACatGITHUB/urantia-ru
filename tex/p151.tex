\upaper{151}{Остановка и проповедь на берегу моря}
\author{Комиссия срединников}
\vs p151 0:1 К 10 марта все группы проповедников и учителей собрались в Вифсаиде. В четверг вечером и в пятницу многие из них отправились ловить рыбу, а в субботу они пошли в синагогу послушать речь пожилого еврея из Дамаска, о славе отца Авраама. Иисус большую часть этого субботнего дня провел один в горах. Вечером в эту субботу Учитель больше часа рассуждал перед всей группой о «значении несчастий и духовной ценности разочарований». Это было знаменательное событие, и его слушатели никогда не забывали полученного урока.
\vs p151 0:2 Иисус еще не совсем оправился от огорчения, вызванного его недавним отвержением в Назарете; апостолы замечали особую печаль, примешивающуюся к его обычному бодрому настроению. С ним в основном находились Иаков и Иоанн, Петр же был очень занят многочисленными обязанностями, связанными с благосостоянием и руководством новыми отрядами евангелистов. В течение этого времени ожидания перед тем, как отправиться в Иерусалим на Пасху, женщины ходили по домам, проповедуя евангелие и служа больным в Капернауме и окрестных городах и селениях.
\usection{1. Притча о сеятеле}
\vs p151 1:1 Примерно в это время Иисус, обучая массы народа, который так часто собирался вокруг него, впервые стал обращаться к притчам. Поскольку Иисус беседовал с апостолами и другими людьми до глубокой ночи, в это воскресное утро лишь очень немногие из группы встали к завтраку; поэтому он отправился к морю, сел один в лодку, старую рыбачью лодку Андрея и Петра, которая всегда была в его распоряжении, и размышлял о следующем шаге, который предстояло сделать в ходе работы по распространению царства. Но Учителю недолго суждено было оставаться в одиночестве. Очень скоро к нему стали приходить люди из Капернаума и близлежащих селений, и к десяти часам утра почти тысяча человек собрались на берегу возле лодки Иисуса и шумно пытались обратить на себя внимание. Петр к тому времени уже проснулся и, пробравшись к лодке, сказал Иисусу: «Учитель, поговорить ли мне с ними?» Но Иисус ответил: «Нет, Петр, я расскажу им историю». И тогда Иисус стал рассказывать притчу о сеятеле, одну из первых в череде таких притч, с помощью которых он учил толпы людей, следовавших за ним. В этой лодке была высокая скамейка, на которой он и сидел (поскольку, обучая, было принято сидеть), пока говорил с толпой, собравшейся на берегу. После того, как Петр проговорил несколько слов, Иисус сказал:
\vs p151 1:2 \P\ «Сеятель отправился сеять, и когда он сеял, оказалось так, что некоторые семена упали на обочину, где их затоптали и склевали птицы небесные. Другое семя упало на каменистую почву и тут же проросло, потому что земли здесь было совсем немного, но как только начало светить солнце, оно засохло, потому что не имело корней, которые сохраняли бы влагу. Какие\hyp{}то семена оказались среди колючек, которые выросли и заглушили семена, и те не дали зерна. Между тем другие семена упали в добрую почву, проросли и дали зерно --- одни в тридцать крат, другие шестьдесят крат, а некоторые и сто крат». И закончив рассказывать эту притчу, он сказал толпе: «Имеющий уши, да слышит».
\vs p151 1:3 \P\ Апостолы и те, кто были с ними, услышав каким образом Иисус учит людей, были сильно озадачены; они долго обсуждали это между собой, и затем вечером того же дня в саду у Зеведея Матфей сказал Иисусу: «Учитель, какой смысл в этих туманных историях, которые ты рассказываешь толпе? Почему ты говоришь притчами с теми, кто ищет истины?» И Иисус ответил:
\vs p151 1:4 «С терпением наставлял я вас все это время. Вам дано знать тайны царствия небесного, но не способной распознавать толпе и тем, кто стремится нас уничтожить, тайны царства отныне будут даваться в виде притч. И мы будем так поступать, чтобы истинно желающие войти в царство могли проникнуть в смысл учения и таким образом обрести спасение, планы же тех, кто слушает только чтобы заманить нас в ловушку, разрушатся оттого, что будут видеть, не видя, и слышать, не слыша. Дети мои, не постигли ли вы закон духа, гласящий, что тому, кто имеет, дано будет в изобилии; а не имеющий лишится даже того, что имеет. Поэтому отныне я часто буду говорить с людьми притчами, чтобы наши друзья и желающие знать истину могли найти то, что ищут, враги же наши и те, кто не любят истины, слушали бы, не понимая. Многие из этих людей не следуют по пути истины. Пророк, воистину, говорил обо всех таких невоспринимающих душах, когда сказал: „Ибо сердце этих людей очерствело, и уши их потеряли слух, и они закрыли глаза, чтобы не зреть истину, и не постичь ее в своих сердцах“».
\vs p151 1:5 Апостолы не до конца поняли значение слов Учителя. Пока Андрей и Фома продолжали беседовать с Иисусом, Петр и другие апостолы отошли в другое место сада, где начали горячо и долго спорить.
\usection{2. Толкование притчи}
\vs p151 2:1 Петр и собравшаяся вокруг него группа пришли к заключению, что притча о сеятеле была аллегорией, что каждая деталь имела скрытый смысл, и решили пойти к Иисусу и попросить объяснить ее. Поэтому Петр подошел к Учителю со словами: «Мы не можем проникнуть в смысл этой притчи и хотим, чтобы ты нам его объяснил, раз ты говоришь, что нам дано знать тайны царства». И когда Иисус это услышал, он сказал Петру: «Сын мой, я ничего не желаю утаивать от тебя, но сначала расскажи мне, о чем ты говорил; каково твое толкование притчи?»
\vs p151 2:2 После минутного молчания, Петр сказал: «Учитель, мы много беседовали об этой притче, и я пришел к такому толкованию: сеятель --- это проповедник евангелия; семя --- это слово Божее. Семя, упавшее на обочину, символизирует тех, кто не понимает учения евангелия. Птицы, которые склевали попавшие на каменистую почву семена, символизируют Сатану, или злой дух, похищающий посеянное в сердцах этих невежественных людей. Семя, упавшее на каменистую почву и мгновенно проросшее, символизирует тех поверхностных и бездумных людей, которые, услышав благую весть, принимают ее с радостью; но поскольку истина не пускает глубокие корни в их сознании, их преданность оказывается недолговечной перед лицом несчастий и преследований. Когда начинаются беды, эти верующие отступают перед ними; они поддаются искушениям. Семя, упавшее среди колючек, символизирует тех, кто с готовностью слушает слово, но позволяют мирским заботам и сбивающему с пути истинного богатству заглушить слово истины, и оно не приносит плодов. Семя же, которое пало на добрую почву, проросло и дало зерно, какое в тридцать крат, какое в шестьдесят крат, а какое и в сто крат, символизирует тех, кто, услышав истину, воспринял ее с разной степенью понимания --- в соответствии со своими различными интеллектуальными способностями --- и, таким образом, показывает эти различные уровни религиозного опыта».
\vs p151 2:3 Выслушав толкование притчи, предложенное Петром, Иисус спросил остальных апостолов, не хотят ли и они тоже предложить что\hyp{}либо. На это откликнулся только Нафанаил. Он сказал: «Учитель, хотя я считаю, что Симон Петр достаточно хорошо толкует притчу, я не вполне согласен с ним. Мое представление об этой притче таково: семя представляет собой евангелие царства, а сеятель олицетворяет вестников царства. Семена, упавшие на обочину на твердую почву, символизируют тех, кто слышал очень мало об евангелии, равно как и тех, кто безразличен к благой вести и чьи сердца очерствели. Птицы небесные, которые склевали упавшие на обочину семена, символизируют жизненные привычки, искушения злом и плотские страсти. Семя, упавшее на каменистую почву, это те эмоциональные души, которые быстро воспринимают новое учение и так же быстро отказываются от истины при столкновении с трудностями и реальностями жизни; им не хватает духовного восприятия. Семя, упавшее среди колючек, символизирует тех, кого привлекали истины евангелия; они готовы следовать его учению, но им мешает гордыня, ревность, зависть и заботы повседневного человеческого существования. Семя, упавшее на добрую почву, проросло и дало зерно, какое в тридцать крат, какое в шестьдесят крат, а какое и сто крат, есть естественная разностепенная способность постигать истину и откликаться на духовное учение, присущая мужчинам и женщинам, обладающим разным даром духовного озарения».
\vs p151 2:4 Когда Нафанаил закончил говорить, апостолы и их сподвижники вступили в серьезное обсуждение, и возник горячий спор, причем одни настаивали на правильности толкования Петра, а другие, которых было примерно столько же, стремились отстоять объяснение притчи, данное Нафанаилом. Тем временем Петр и Нафанаил удалились в дом, где каждый из них энергично и решительно пытался переубедить и изменить мнение другого.
\vs p151 2:5 Учитель подождал, пока страсти начнут утихать; затем он хлопнул в ладоши и подозвал их к себе. Когда они все снова собрались вокруг него, он сказал: «Прежде, чем я расскажу вам об этой притче, желает ли кто\hyp{}либо из вас что\hyp{}нибудь сказать?» После недолгого молчания заговорил Фома: «Да, Учитель, я хотел бы сказать несколько слов. Я помню, как некогда ты велел нам остерегаться именно этого. Ты наставлял нас, что в качестве примеров в наших проповедях нам следует приводить реальные истории, а не басни и что надо выбрать историю, наилучшим образом иллюстрирующую одну из главных и наиболее существенных истин, которой мы хотим научить людей, а выбрав таким образом историю, нам не следует пытаться придавать духовный смысл каждой незначительной детали при изложении этой истории. Я полагаю, что Петр и Нафанаил оба неправы в своих попытках толковать эту притчу. Я восхищен их способностью делать подобные вещи, но в то же время уверен, что все подобные попытки найти духовные аналогии для каждой детали реальной притчи могут привести лишь к сумятице и серьезному искажению истинного назначения такой притчи. Моя правота полностью подтверждается тем, что, мы, которые еще час назад были единодушны во взглядах, теперь разделены на две отдельные группы, придерживающиеся различных мнений относительно этой притчи, и настаиваем на этих мнениях настолько горячо, что это, на мой взгляд, может воспрепятствовать нашей способности в полной мере осознать ту великую истину, которую ты имел в виду, когда рассказал эту притчу толпе и затем попросил нас высказать свое мнение о ней».
\vs p151 2:6 Слова, сказанные Фомой, оказали на них всех умиротворяющее воздействие. Он заставил их вспомнить о том, чему учил их Иисус до того, и прежде, чем Иисус снова заговорил, поднялся Андрей и сказал: «Я убежден, что Фома прав, и хотел бы, чтобы он рассказал нам о том смысле, который он придает этой притчей о сеятеле». Когда Иисус знаком предложил Фоме говорить, тот сказал: «Братья мои, я не хотел продолжать это обсуждение, но если вы желаете, я скажу, что эта притча, по\hyp{}моему, была рассказана, чтобы научить нас одной великой истине. Она состоит в том, что проповедь евангелия царства, как бы честно и успешно не выполняли мы эту божественную миссию, будет иметь успех в разной степени; и что все такие различия в результатах прямо зависят от условий, связанных с обстоятельствами нашего служения, условий, которые почти или совсем не подвластны нам».
\vs p151 2:7 Когда Фома закончил говорить, большинство его товарищей\hyp{}проповедников были почти готовы согласиться с ним, даже Петр и Нафанаил собирались заговорить с ним, когда Иисус поднялся и сказал: «Хорошо, Фома; ты понял истинный смысл притч; но и Петр, и Нафанаил принесли вам всем не меньше добра тем, что так полно продемонстрировали опасность попыток превращать мои притчи в аллегорию. Вы можете позволить себе полет отвлеченных фантазий, но совершите ошибку, если попытаетесь высказать эти свои размышления, когда учите народ».
\vs p151 2:8 Теперь, когда напряжение прошло, Петр и Нафанаил поздравили друг друга со своими толкованиями, и прежде, чем лечь спать, каждый из апостолов, кроме братьев Алфеев, отважился высказать свое толкование притчи о сеятеле. Даже Иуда Искариот предложил вполне убедительное толкование. В своем кругу двенадцать апостолов часто пытались представлять притчи Учителя в виде аллегорий, но они никогда больше не относились к этим плодам фантазии серьезно. Это было очень полезным уроком для апостолов и их сподвижников, особенно в связи с тем, что впредь с тех пор Иисус все больше и больше использовал притчи при обучении народа.
\usection{3. Еще о притчах}
\vs p151 3:1 Апостолы были поглощены притчами до такой степени, что и весь следующий вечер был посвящен дальнейшему обсуждению притч. Иисус начал вечернюю беседу словами: «Возлюбленные мои, когда вы учите, вы всегда должны делать это так, чтобы ваше изложение истины проникло бы в умы и сердца людей, которые перед вами. Когда вы стоите перед массой людей с разными интеллектами и темпераментами, вы не можете говорить разные слова для каждого типа слушателей, но вы можете рассказать историю, чтобы донести до них ваше учение; и каждая группа, даже каждый отдельный человек сможет по\hyp{}своему истолковать вашу притчу в соответствии со своей собственной интеллектуальной и духовной одаренностью. Вы должны дать своему свету сиять, но делать это мудро и осторожно. Ни один человек, зажигая светильник, не накрывает его колпаком и не ставит под кровать; он ставит свой светильник на подставку, чтобы все могли видеть свет. Позвольте мне сказать вам, что нет ничего скрытого в царстве небесном, что бы не стало явным; и нет никаких тайн, которые в конце концов не сделались бы известными. В конечном счете, все выходит на свет. Думайте не только о массах и о том, как они слышат истину; обращайте внимание и на то, как вы сами ее слышите. Помните, как я много раз говорил вам: тому, кто имеет, будет дано больше, а не имеющий лишится даже того, что он думает, что он имеет».
\vs p151 3:2 \P\ Они продолжили обсуждать притчи, и последующие наставления относительно их толкования можно обобщить и изложить на современном языке следующим образом:
\vs p151 3:3 \ublistelem{1.}\bibnobreakspace Иисус не советовал использовать басни или аллегории при обучении истине евангелия. Он рекомендовал свободно пользоваться притчами, особенно притчами о природе. Он подчеркивал ценность использования \bibemph{аналогии,} существующей между природным и духовным мирами, как средства обучения истине. Он часто называл природное «нереальной и быстротечной тенью духовных реальностей».
\vs p151 3:4 \P\ \ublistelem{2.}\bibnobreakspace Иисус привел три или четыре притчи из Иудейских писаний, обращая внимание на то, что этот способ обучения не является совсем новым. Однако это стало почти новым методом обучения, в том виде, в котором он впредь стал использовать его.
\vs p151 3:5 \P\ \ublistelem{3.}\bibnobreakspace Наставляя апостолов ценности притч, Иисус обращал их внимание на следующие моменты:
\vs p151 3:6 Притча позволяет обращаться одновременно к совершенно разным уровням ума и духа. Притча будит воображение, требует проницательности и побуждает к критическому мышлению; она усиливает сочувствие, не вызывая антагонизма.
\vs p151 3:7 Притча ведет от известных вещей к пониманию неизвестного. Притча использует материальное и естественное, чтобы дать представление о духовном и сверхматериальном.
\vs p151 3:8 Притчи способствуют принятию беспристрастных моральных решений. Притча позволяет обходить многие предрассудки и ненавязчиво вносить в умы новую истину, и делает все это, практически не вызывая неприятия со стороны человека.
\vs p151 3:9 Чтобы отвергнуть истину, содержащуюся в аналогиях притчи, требуется сознательная интеллектуальная деятельность, прямо противоречащая собственным честным суждениям и справедливым решениям. Притча ведет к стимуляции мысли через посредство слуха.
\vs p151 3:10 Использование притчевой формы обучения позволяет учителю сообщать новые и даже поразительные истины, и во многом избегать при этом любых споров и внешнего столкновения с традицией и существующей властью.
\vs p151 3:11 Притча обладает также тем преимуществом, что она вызывает в памяти истину, которую усвоили, когда впоследствии человек сталкивается с теми знакомыми ему уже картинами.
\vs p151 3:12 \P\ Таким образом Иисус стремился объяснить своим последователям, почему он столь широко использует притчи при обучении народа.
\vs p151 3:13 \P\ К концу этого вечернего урока Иисус впервые прокомментировал притчу о сеятеле. Он сказал, что эта притча касалась двух вещей: во\hyp{}первых, она представляла картину его собственного служения вплоть до настоящего времени и предсказание того, что ему предстояло в оставшийся период жизни на земле. И во\hyp{}вторых, это было указание на то, что со временем может предстоять апостолам и другим вестникам царства в их служении последующим поколениям.
\vs p151 3:14 Иисус также прибегал к притчам как к наилучшему возможному способу опровергнуть попытки религиозных лидеров Иерусалима учить, будто вся его деятельность осуществлялась с помощью демонов и принца дьяволов. Обращение к природе противоречило подобному учению, поскольку люди в те дни рассматривали все природные явления как продукт непосредственного действия духовных сущностей и сверхъестественных сил. Он также остановился на этом методе обучения потому, что он позволял ему возвещать важнейшие истины тем, кто стремился узнать лучший путь, и в то же время давал меньше возможности его врагам находить поводы для выпадов и обвинений против него.
\vs p151 3:15 Прежде, чем отпустить группу на ночь, Иисус сказал: «Теперь я расскажу вам конец притчи о сеятеле. Мне хотелось бы узнать, как вы воспримете это: царство небесное тоже подобно человеку, бросающему добрые семена в землю; и пока он спал ночью, а днем занимался делами, семена проросли и растут, и хотя он не знал, как это произошло, растение дало плоды. Сначала был росток, потом колосок, потом полный зерна колос. А потом, когда зерно созрело, он взял серп, и жатва была закончена. Имеющий уши, да услышит».
\vs p151 3:16 Много раз апостолы обдумывали эти слова, но Учитель никогда больше не упоминал об этом дополнении к притче о сеятеле.
\usection{4. Другие притчи, рассказанные у моря}
\vs p151 4:1 На следующий день Иисус, сидя в лодке, снова учил людей, говоря: «Царство небесное подобно человеку, посеявшему добрые семена на своем поле; но пока он спал, пришел его враг, посеял сорную траву среди пшеницы и быстро скрылся. Поэтому, когда появились молодые ростки, которые позже должны были принести урожай, появились и сорняки. Тогда слуги этого хозяина пришли и сказали ему: „Хозяин, разве не посеял ты хорошие семена на своем поле? Откуда же в таком случае взялись сорняки?“ И он ответил своим слугам: „Это сделал враг“. Тогда слуги спросили своего хозяина: „Не хочешь ли ты послать нас, чтобы мы выпололи эти сорняки?“ Но он сказал им в ответ: „Нет, потому что, вырывая их, вы можете выдернуть также и пшеницу. Пусть они лучше растут все вместе до времени жатвы, когда я скажу жнецам: соберите сначала сорные травы, свяжите их в пучки и сожгите, а затем уберите пшеницу и сложите ее в мой амбар“».
\vs p151 4:2 \P\ После того, как люди задали ему несколько вопросов, Иисус рассказал еще одну притчу: «Царство небесное подобно горчичному зерну, которое человек посеял на своем поле. Сейчас горчичное зерно --- мельчайшее из зерен, но когда эта трава вырастает, то достигает огромных размеров и становится подобной дереву, так что даже птицы небесные могут прилетать и отдыхать на его ветвях».
\vs p151 4:3 \P\ «Царство небесное также подобно закваске, которую женщина взяла и замесила с тремя мерами муки, и таким образом получилось, что все тесто взошло».
\vs p151 4:4 \P\ «Царство небесное также подобно зарытому в поле кладу, который человек нашел. Обрадовавшись, отправился он продавать все, что имел, чтобы получить деньги и купить поле».
\vs p151 4:5 \P\ «Царство небесное также подобно купцу, ищущему прекрасные жемчужины; и найдя одну жемчужину величайшей ценности, он пошел и продал все, чем владел, чтобы иметь возможность купить эту необычайную жемчужину».
\vs p151 4:6 \P\ «Еще царство небесное подобно рыболовной сети, которая была заброшена в море, и в нее попала всевозможная рыба. И когда сеть была полна, рыбаки вытянули ее на берег и сели там сортировать рыбу, кладя в сосуд хорошую и выбрасывая плохую».
\vs p151 4:7 \P\ Много других притч рассказал Иисус народу. В сущности, с тех пор он редко учил людей каким\hyp{}либо иным способом. После того, как он рассказывал многочисленным слушателям различные притчи, он на вечерних занятиях более глубоко и подробно излагал свое учение апостолам и евангелистам.
\usection{5. Посещение Хересы}
\vs p151 5:1 Всю неделю толпа продолжала увеличиваться. В субботу Иисус поспешил удалиться в горы, но когда наступило воскресное утро, толпа вернулась. Иисус поговорил с ними вскоре после полудня, после проповеди Петра, а закончив, сказал апостолам: «Я устал от толп; давайте пересечем озеро, чтобы один день отдохнуть на другой стороне».
\vs p151 5:2 По пути через озеро они попали в один из яростных и внезапных штормов, которые обычны на Галилейском море, и больше всего в это время года. Это озеро расположено почти на семьсот футов ниже уровня моря и окружено высокими берегами, особенно на западе. От озера к горам идут крутые теснины, и поскольку в течение дня над озером поднимается нагретый воздух, после захода солнца на озеро часто с силой устремляется более холодный воздух из теснин. Эти бури быстро налетают и иногда точно так же быстро утихают.
\vs p151 5:3 В один из таких вечерних штормов и попала лодка, которая в этот воскресный вечер везла Иисуса на другую сторону озера. Три другие лодки, в которых находились некоторые из молодых евангелистов, плыли следом. Буря была жестокой, хотя и разразилась только над этой частью озера, а на западном берегу не было даже признаков шторма. Ветер был настолько сильным, что волны перехлестывали через борт лодки. Сильный ветер сорвал парус прежде, чем апостолы успели убрать его, и теперь, надеясь только на весла, они старательно гребли к берегу, до которого было чуть больше полутора миль.
\vs p151 5:4 Иисус тем временем спал под небольшим навесом на корме лодки. Учитель был очень усталым, когда они отплыли из Вифсаиды, и просил отвезти его на другую сторону озера именно для того, чтобы отдохнуть. Эти бывшие рыбаки были сильными и опытными гребцами, но буря была одной из самых жестоких из всех, которые когда\hyp{}либо настигали их. Хотя ветер и волны швыряли их лодку, как игрушечный кораблик, Иисус продолжал крепко спать. Петр греб правым веслом возле кормы. Когда лодка стала наполняться водой, он оставил весло, бросился к Иисусу и стал что есть силы трясти его, чтобы разбудить, и когда тот проснулся, Петр сказал: «Учитель, разве ты не знаешь, что мы попали в жестокий шторм? Если ты не спасешь нас, мы все погибнем».
\vs p151 5:5 Выйдя под дождь, Иисус сначала посмотрел на Петра, затем устремил взгляд в темноту на гребцов, сражающихся со стихией, снова перевел взгляд на Симона Петра, который был в сильном возбуждении и не вернулся еще к своему веслу, и сказал: «Почему вы все так испугались? Где ваша вера? Тихо, успокойтесь». Как только Иисус высказал этот упрек Петру и другим апостолам, едва он призвал Петра к спокойствию и умиротворению его встревоженной души, как взбудораженная атмосфера утихомирилась и наступил полнейший штиль. Грозные волны почти сразу же стихли, а темные тучи, излившись коротким ливнем, исчезли, и небесные звезды засияли над головой. Все это было, насколько мы можем судить, чистое совпадение; но апостолы, особенно Симон Петр, никогда не переставали считать этот эпизод чудом природы. В те времена люди особенно легко верили в чудеса природы, поскольку они были твердо уверены, что вся природа непосредственно управляется духовными силами и сверхъестественными существами.
\vs p151 5:6 Иисус ясно объяснил двенадцати апостолам, что он говорил с их встревоженными душами и обращался к их охваченными страхом умам, что он не приказывал стихиям подчиниться его слову, но все было напрасно. Последователи Учителя всегда упорно по\hyp{}своему толковали любые такие совпадения. Начиная с того дня они настойчиво утверждали, что Учитель имеет абсолютную власть над силами природы. Петр не уставал повторять, что «даже ветры и волны подчиняются ему».
\vs p151 5:7 Был уже поздний вечер, когда Иисус и его сподвижники добрались до берега, и поскольку была тихая и прекрасная ночь, все они отдыхали в лодках и вышли на берег лишь вскоре после рассвета на следующее утро. Когда все собрались вместе, всего около сорока человек, Иисус сказал: «Давайте поднимемся вон на те горы, побудем там несколько дней и поразмышляем над проблемами царства Отца».
\usection{6. Безумец из Хересы}
\vs p151 6:1 Хотя большая часть близлежащего восточного берега озера полого поднималась в сторону находящихся в отдалении гор, именно в этом месте возвышался крутой склон горы, и кое\hyp{}где берег отвесно обрывался в озеро. Указывая на склон близлежащей горы, Иисус сказал: «Давайте поднимемся на этот склон, позавтракаем и в каком\hyp{}нибудь укрытии отдохнем и поговорим».
\vs p151 6:2 Весь склон был покрыт пещерами, вырубленными в скале. Многие из этих ниш были древними гробницами. Примерно на середине склона, на небольшой относительно ровной площадке находилось кладбище маленькой деревни под названием Хереса. Когда Иисус и его сподвижники проходили мимо этого места погребения, к ним устремился безумец, живший в этих пещерах на склоне горы. Этот безумец был хорошо известен в этих местах, однажды его заковали в цепи и заключили в одну из пещер. Он давно уже разбил свои оковы и теперь свободно скитался среди могил и брошенных гробниц.
\vs p151 6:3 На этого человека, которого звали Амос, нападали приступы безумия. Бывали довольно длительные периоды, когда он находил себе одежду и вполне достойно вел себя среди своих соплеменников. Во время одного из таких периодов просветления ума он ходил в Вифсаиду, где услышал проповедь Иисуса и апостолов, и тогда он в какой\hyp{}то мере уверовал в евангелие царства. Но вскоре началась буйная фаза его болезни, и он убежал к могилам, где стонал, громко кричал и своим поведением приводил в ужас всех, кому случалось встретить его.
\vs p151 6:4 Когда Амос узнал Иисуса, то упал к его ногам и воскликнул: «Я знаю тебя, Иисус, но я одержим многими бесами и прошу тебя не причинять мне страданий». Этот человек действительно верил, что его периодические ментальные отклонения были вызваны тем, что в такие периоды в него вселялись злые или нечистые духи и властвовали над его разумом и телом. Его болезнь носила преимущественно эмоциональный характер --- мозг его не был сильно поврежден.
\vs p151 6:5 Иисус, глядя на человека, корчившегося у его ног, как животное, наклонился и, взяв его за руку, поднял на ноги и сказал ему: «Амос, ты не одержим дьяволом; ты уже слышал благую весть, что ты --- сын Бога. Я велю тебе преодолеть этот приступ болезни». И когда Амос услышал эти слова Иисуса, его сознание так преобразилось, что он тут же вновь обрел здравый разум и полный контроль над своими чувствами. К этому моменту собралась довольно большая толпа жителей близлежащей деревни, и эти люди вместе с подошедшими с гор свинопасами были удивлены, увидев этого безумца сидящим в здравом уме с Иисусом и его последователями и свободно беседующим с ними.
\vs p151 6:6 Когда свинопасы поспешили в деревню, чтобы рассказать новость об укрощении безумца, собаки бросились на небольшое и оставленное без присмотра стадо примерно из тридцати свиней и согнали большую часть из них с обрыва в море. И именно это случайное происшествие в связи с присутствием Иисуса и воображаемым чудесным исцелением безумца послужило основой легенды, что Иисус исцелил Амоса, изгнав из него легион бесов, и что эти бесы вселились в стадо свиней, заставив их тотчас очертя голову броситься с обрыва в море навстречу своей гибели. В тот же день свинопасы всем рассказали об этом случае, и вся деревня этому поверила. Амос особенно твердо поверил в эту историю; он видел, как свиньи бросались через край обрыва вскоре после того, как его больной ум успокоился; и он всегда верил, что они унесли с собой тех самых злых духов, которые так долго его мучили и причиняли страдания. Именно этим, в значительной степени, и было обусловлено то, что он окончательно исцелился. Так же верно, что все апостолы (кроме Фомы) поверили, что эпизод со свиньями был прямо связан с исцелением Амоса.
\vs p151 6:7 \P\ Иисусу так и не удалось отдохнуть. Большую часть этого дня он был окружен толпой людей, которых привлекло известие об исцелении Амоса и история о бесах, перешедших из безумца в стадо свиней. Так что, Иисус и его друзья успели отдохнуть всего лишь одну ночь, когда рано утром во вторник были разбужены делегацией этих неевреев\hyp{}свинопасов, которые пришли просить, чтобы он ушел из их мест. Их представитель сказал Петру и Андрею: «Рыбаки из Галилеи, уйдите от нас и уведите с собой своего пророка. Мы знаем, что он святой человек, но боги нашей страны не знают его, и нам угрожает опасность потерять много свиней. Нас охватил страх перед вами, так что мы умоляем вас уйти прочь». И когда Иисус услышал их, он сказал Андрею: «Давайте вернемся к себе».
\vs p151 6:8 Когда они собрались уходить, Амос стал просить Иисуса позволить ему пойти с ними, но Учитель не согласился. Иисус сказал Амосу: «Не забывай, что ты --- сын Бога. Возвращайся к своему народу и покажи им, какие великие вещи сотворил для тебя Бог». И Амос стал ходить повсюду и рассказывать, что Иисус изгнал из его больной души легион бесов и что эти злые духи вселились в стадо свиней и погубили их. И он продолжал это делать, пока не обошел все города Десятиградья с рассказом о том, какие чудесные вещи сотворил для него Иисус.
