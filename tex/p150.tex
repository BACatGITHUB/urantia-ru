\upaper{150}{Третье путешествие с проповедями}
\author{Комиссия срединников}
\vs p150 0:1 В воскресенье вечером 16 января 29 года н.э. Авенир с апостолами Иоанна достигли Вифсаиды и на следующий день начали совместное совещание с Андреем и апостолами Иисуса. Авенир и его сподвижники организовали свой центр в Хевроне, и для них стало привычным периодически приходить в Вифсаиду на эти совещания.
\vs p150 0:2 В числе многих вопросов, обсуждавшихся на этой совместной встрече, была практика помазания больного определенными видами масел в сочетании с молитвами об исцелении. И вновь Иисус отказывался участвовать в их обсуждениях или выражать свое мнение об их решении. Апостолы Иоанна всегда пользовались особым маслом для помазания в своем служении больным и страждущим, и они стремились утвердить это в качестве единой практики для обеих групп, но апостолы Иисуса отказывались связывать себя подобным правилом.
\vs p150 0:3 \pc Во вторник 18 января в доме у Зеведея в Вифсаиде к двадцати четырем апостолам присоединились выдержавшие испытание евангелисты, числом около семидесяти пяти человек, перед тем, как отправиться в третье проповедническое путешествие по Галилее. Эта третья миссия длилась семь недель.
\vs p150 0:4 Евангелисты были посланы группами по пять человек, тогда как Иисус и двенадцать апостолов большую часть времени путешествовали вместе, а апостолы по мере надобности по двое отправлялись крестить верующих. На протяжении почти трех недель Авенир и его сподвижники тоже трудились вместе с группами евангелистов, давая им советы и крестя верующих. Они посетили Магдалу, Тивериаду, Назарет и все основные города и селения в центральной и южной Галилее, все те места, где уже бывали раньше, и многие другие. Это было их последнее посещение Галилеи, за исключением ее северной части.
\usection{1. Отряд женщин --- евангелистов}
\vs p150 1:1 Из всех дерзновенных деяний, что совершил Иисус в своей земной жизни, самым удивительным было его неожиданное заявление, произнесенное вечером 16 января: «Завтра мы изберем десять женщин для служения делу царства». В начале двухнедельного отдыха, в течение которого апостолы и евангелисты отсутствовали в Вифсаиде, Иисус повелел Давиду призвать его родителей вернуться домой и отправить вестников, чтобы созвать в Вифсаиду десять благочестивых женщин, занимавшихся ведением хозяйства в предыдущем лагере и служивших в полевом лазарете. Все эти женщины слушали наставления, которые давались молодым евангелистам, но ни им самим, ни их учителям никогда бы не пришло в голову, что Иисус решится поручить женщинам учить евангелию царства и служить больным. Этими десятью женщинами, избранными и назначенными Иисусом, были: Сусанна, дочь бывшего хазана Назаретской синагоги; Иоанна, жена Хузы, управляющего Ирода Антипы; Елизавета, дочь богатого еврея из Тивериады и Сефориса; Марфа, старшая сестра Андрея и Петра; Рахиль, свояченица Иуды, брата Учителя по плоти; Насанта, дочь Елмана, сирийского лекаря; Милха, двоюродная сестра апостола Фомы; Руфь, старшая дочь Матфея Левия; Целта, дочь римского центуриона; и Агаман, вдова Дамаска. Впоследствии Иисус прибавил к этой группе еще двух женщин --- Марию Магдалину и Ревекку, дочь Иосифа Аримафейского.
\vs p150 1:2 Иисус поручил этим женщинам создать свою собственную организацию и велел Иуде выделить им средства для обеспечения всем необходимым и вьючными животными. Эти десять женщин избрали главой своей группы Сусанну, а казначеем --- Иоанну. В дальнейшем они обеспечивали себя всем сами; больше они никогда не обращались за помощью к Иуде.
\vs p150 1:3 В те дни, когда женщины не допускались даже на основной этаж синагоги (место их пребывания ограничивалось женской галереей), было совершенно поразительно, что они признаны полноправными учителями нового евангелия царства. То, что Иисус дал задание этим десятерым женщинам отправиться для проповеди евангелия и служения, было провозглашением эмансипации, освобождающим всех женщин на все времена; мужчина никогда больше не должен считать женщину духовно ниже себя. Это стало явным потрясением даже для двенадцати апостолов. Хотя они много раз слышали слова Учителя, что «в царстве небесном нет ни богатых, ни бедных, ни свободных, ни подневольных, ни мужчин, ни женщин, все одинаковые сыновья и дочери Бога», они были буквально потрясены, когда он предложил официально уполномочить этих десятерых женщин стать религиозными учителями и даже позволить им путешествовать вместе с ними. Вся страна была взбудоражена этим поступком, враги Иисуса извлекали огромную выгоду из этого шага, но женщины, верующие в благую весть, повсюду решительно поддержали своих избранных сестер и совершенно определенно выразили одобрение этого запоздалого признания роли женщины в религиозной деятельности. Это освобождение женщин, дающее им должное признание, практиковалось и апостолами сразу же после ухода Учителя, хотя последующие поколения вернулись к старым обычаям. В ранний период христианской церкви женщины\hyp{}учителя и служители назывались \bibemph{дьяконицами} и пользовались всеобщим признанием. Но для Павла, несмотря на то, что теоретически он все это признавал, это так и не стало органичной частью его собственных воззрений, и ему лично было трудно осуществлять эту политику на практике.
\usection{2. Остановка в Магдале}
\vs p150 2:1 Когда апостольская группа отправилась из Вифсаиды, женщины следовали позади. Во время совещаний они всегда группой сидели впереди, справа от выступающего. Все больше женщин уверовало в евангелие царства, и их желание лично побеседовать с Иисусом или с одним из апостолов становилось причиной затруднений и постоянно вызывало замешательство. Теперь все изменилось. Когда кто\hyp{}то из верующих женщин хотел увидеться с Учителем или поговорить с апостолами, они обращались к Сусанне и сразу же в сопровождении одной из двенадцати женщин\hyp{}евангелисток шли к Учителю или одному из его апостолов.
\vs p150 2:2 Именно в Магдале женщины впервые продемонстрировали, что их присутствие приносит пользу и доказали, что их избрание было мудрым поступком. Андрей установил для своих сподвижников довольно строгие правила, касающиеся личной работы с женщинами, особенно теми, которые пользовались сомнительной репутацией. Когда группа пришла в Магдалу, эти десять женщин\hyp{}евангелисток могли свободно заходить в вертепы разврата и проповедовать благую весть непосредственно их обитательницам. А при посещении больных эти женщины могли находиться рядом со своими страждущими сестрами во время служения. В результате служения этих десятерых женщин (позже известных как двенадцать женщин) в этом городе удалось привлечь, к царству Марию Магдалину. Из\hyp{}за несчастного стечения обстоятельств и вследствие отношения почтенного общества к женщинам, принявшим такие ошибочные решения, эта женщина оказалась в одном из гнусных мест Магдалы. Марфа и Рахиль доходчиво объяснили Марии, что двери царства открыты даже для таких, как она. Мария уверовала в благую весть, и на следующий день ее крестил Петр.
\vs p150 2:3 Мария Магдалина стала самым эффективным учителем евангелия из всех двенадцати евангелисток. Она была избрана для этого служения вместе с Ревеккой в Иотапате спустя примерно четыре недели после своего обращения в веру. Мария и Ревекка вместе с остальными евангелистками продолжали свою деятельность на протяжении всей оставшейся жизни Иисуса на земле, честно и действенно прилагая усилия к тому, чтобы просветить и помочь воспрянуть своим угнетенным сестрам; а когда разыгрался последний трагический эпизод драматической жизни Иисуса и все апостолы, кроме одного, разбежались, все эти женщины остались, и ни одна не отреклась и не предала его.
\usection{3. Суббота в Тивериаде}
\vs p150 3:1 Андрей по указанию Иисуса поручил женщинам проведение субботних служб апостольской группы. Это, конечно, означало, что их нельзя было проводить в новой синагоге. Женщины избрали Иоанну ответственной за это, и встреча состоялась в пиршественной зале нового дворца Ирода, пока Ирод находился в Перее в Юлии. Иоанна прочитала из Писания о деятельности женщин в религиозной жизни Израиля, приведя в пример Мириам, Девору, Есфирь и других.
\vs p150 3:2 \pc Поздно вечером Иисус провел со всей группой достопамятную беседу о «магии и суеверии». В те дни появление яркой и предположительно новой звезды считалось знаком того, что на земле родился великий человек. Поскольку незадолго до этого наблюдалась такая звезда, Андрей спросил у Иисуса, обоснованы ли эти представления. Пространно отвечая на вопрос Андрея, Учитель перешел к глубокому обсуждению вопроса о человеческих суевериях в целом. Сказанное тогда Иисусом можно изложить на современном языке следующим образом:
\vs p150 3:3 \ublistelem{1.}\bibnobreakspace Движение звезд на небесах никоим образом не связано с событиями человеческой жизни на земле. Астрономия --- это подобающий предмет изучения для науки, но астрология --- это собрание суеверных заблуждений, которым нет места в евангелии царства.
\vs p150 3:4 \ublistelem{2.}\bibnobreakspace Исследование внутренних органов недавно убитого животного не может позволить узнать что\hyp{}либо о погоде, будущих событиях или исходе человеческих дел.
\vs p150 3:5 \ublistelem{3.}\bibnobreakspace Духи умерших не возвращаются общаться со своими семьями или бывшими друзьями.
\vs p150 3:6 \ublistelem{4.}\bibnobreakspace Амулеты и реликвии бессильны излечить болезнь, предотвратить несчастье или оказать влияние на злых духов; вера во все такие материальные средства влияния на духовный мир есть не что иное, как грубый предрассудок.
\vs p150 3:7 \ublistelem{5.}\bibnobreakspace Бросание жребия, хотя оно и может быть удобным способом решения многих малосущественных проблем, не может служить средством для выяснения божественной воли. Его результаты --- просто материальная случайность. Единственное средство общения с духовным миром заключено в духовном даре человечества, в пребывающем в вас духе Отца вместе с излитым на вас духом Сына и вездесущим воздействием Бесконечного Духа.
\vs p150 3:8 \ublistelem{6.}\bibnobreakspace Гадание, колдовство и волшебство, равно как и иллюзии магии --- это предрассудки невежественных умов. Вера в магические числа, предзнаменования удачи и предвестия неудачи --- это абсолютно несостоятельный предрассудок.
\vs p150 3:9 \ublistelem{7.}\bibnobreakspace Толкование снов --- это в значительной степени суеверная и беспочвенная система невежественных и фантастических предположений. Евангелие царства не должно иметь ничего общего со жрецами\hyp{}прорицателями примитивных религий.
\vs p150 3:10 \ublistelem{8.}\bibnobreakspace Духи добра или зла не могут жить в материальных символах, сделанных из глины, дерева или металла; идолы --- это не что иное, как просто материал, из которого они сделаны.
\vs p150 3:11 \ublistelem{9.}\bibnobreakspace Деятельность чародеев, магов, волшебников и колдунов берет свое начало из предрассудков египтян, ассирийцев, вавилонян и древних хананеев. Амулеты и всевозможные магические действия бесполезны как средство обретения покровительства добрых духов или защиты от мнимых злых духов.
\vs p150 3:12 \ublistelem{10.}\bibnobreakspace Он обличил и осудил их веру в заклинания, магические испытания, чары, проклятия, знамения, мандрагору, узелки и все прочие виды невежественных и порабощающих суеверий.
\usection{4. Апостолы по двое отправляются в путь.}
\vs p150 4:1 На следующий вечер, собрав вместе своих двенадцать апостолов, апостолов Иоанна и недавно созданную группу женщин, Иисус сказал: «Вы сами видите, что урожай обилен, но жнецы малочисленны. Давайте же все будем молить Хозяина урожая, чтобы он послал побольше жнецов на свои поля. Я останусь, чтобы поддерживать и наставлять более молодых учителей, но я хотел бы, чтобы тем временем более опытные по двое за короткий срок прошли бы по всей Галилее, проповедуя евангелие царства, пока еще можно это делать спокойно и мирно». Потом он по своему усмотрению объединил апостолов в пары следующим образом: Андрей и Петр, Иаков и Иоанн Зеведеевы, Филипп и Нафанаил, Фома и Матфей, Иаков и Иуда Алфеевы, Симон Зилот и Иуда Искариот.
\vs p150 4:2 Иисус назначил дату встречи с двенадцатью апостолами в Назарете и на прощание сказал: «Во время этой миссии не ходите в города неевреев, не ходите и в Самарию, идите же к заблудшим овцам дома Израиля. Проповедуйте евангелие царства и возвещайте спасительную истину, что человек есть сын Бога. Помните, что ученик едва ли выше учителя, а слуга вряд ли важнее хозяина. Достаточно, чтобы ученик был равен учителю, а слуга стал подобен хозяину. Если некоторые люди осмелились назвать хозяина дома сподвижником Вельзевула, с большей готовностью воспримут они в качестве таковых и его домочадцев! Но вам не следует страшиться этих неверующих недругов. Я заявляю вам, что нет ничего тайного, что не стало бы явным; нет ничего скрытого, что не стало бы известным. То, чему я учил вас с глазу на глаз, проповедуйте с мудростью во всеуслышание. Что я открыл вам во внутренних покоях, то надлежит вам возвещать в должное время с крыш домов. И я говорю вам, друзья мои и ученики, не бойтесь тех, кто может убить тело, но не может разрушить душу; уповайте на Того, кто может поддержать тело и спасти душу.
\vs p150 4:3 Не продаются ли воробьи по грошу за пару? И, однако, я заявляю, что ни одного из них Бог не упускает из виду. Не знаете ли вы, что даже волосы на вашей голове все сочтены? Не бойтесь же; вы ценнее, чем великое множество воробьев. Не стыдитесь моего учения; идите, возвещая мир и добрую волю, но не обманывайтесь --- не всегда мир будет сопутствовать вашей проповеди. Я пришел, чтобы принести мир на землю, но когда люди отвергают мой дар, возникают разобщение и смута. Когда все в семье принимают евангелие царства, воистину мир царит в этом доме; но когда часть семьи входит в царство, а часть отвергает евангелие, такое несогласие может порождать только горе и печаль. Трудитесь ревностно для спасения всей семьи, чтобы человек не обрел врагов среди своих домочадцев. Но когда вы сделали все возможное для каждого в каждой семье, я заявляю вам, что не достоин царства тот, кто любит отца или мать больше, чем это евангелие».
\vs p150 4:4 Выслушав эти слова, двенадцать апостолов собрались в путь. И они не собирались все вместе вплоть до назначенного Учителем времени их общей встречи с Иисусом и остальными учениками в Назарете.
\usection{5. Что я должен делать, чтобы спастись?}
\vs p150 5:1 Однажды вечером в Сунеме после возвращения апостолов Иоанна в Хеврон и ухода апостолов Иисуса, когда Учитель занимался обучением трудившихся под руководством Иакова двенадцати молодых евангелистов и двенадцати женщин, Рахиль задала Иисусу такой вопрос: «Учитель, что следует нам отвечать, когда женщины спрашивают нас: что делать, чтобы спастись?» Когда Иисус услышал этот вопрос, он ответил:
\vs p150 5:2 \pc «Когда мужчины и женщины спрашивают, что делать, чтобы спастись, вам следует отвечать: веруйте в это евангелие царства; примите божественное прощение. Через веру осознайте пребывающий в вас дух Бога, приняв которого, вы становитесь сынами Бога. Не читали ли вы то место в Писании, где говорится: „В Господе имею я праведность и силу“. И еще --- где Отец говорит: „Праведность моя близка; спасение приблизилось, и руки мои обнимут народ мой“. „Душа моя возрадуется от любви Бога моего, ибо он облачил меня в одежды спасения и возложил на меня мантию его праведности“. Не читали ли вы также об Отце, что имя, которым „будут называть его, --- Господь нашей праведности“. „Уберите грязные лохмотья собственной праведности и облачите моего сына в мантию божественной праведности и вечного спасения“. Вовеки истинно, „живут верой“. Вход в царство Отца совершенно свободный, но развитие --- рост в благодати --- необходим для постоянного пребывания в нем.
\vs p150 5:3 Спасение --- дар Отца, открытый его Сынами. Принятие его вами через веру делает вас частью божественной природы, сыном или дочерью Бога. По вере вам воздается; через веру вы спасаетесь; с помощью той же самой веры вы постепенно продвигаетесь по пути божественного совершенства. По вере воздалось Аврааму и открыто ему было спасение через посредство учения Мелхиседека. На протяжении всех веков та же самая вера спасала сынов человеческих, но теперь от Отца пришел Сын, чтобы сделать спасение более реальным и доступным».
\vs p150 5:4 \pc Когда Иисус закончил говорить, радость охватила слушавших эти благодатные слова, и в последующие дни все они отправились возвещать евангелие царства с новой силой и удвоенной энергией и энтузиазмом. А еще больше радовались женщины, зная, что и они тоже участвуют в этих планах установления царства на земле.
\vs p150 5:5 В заключение Иисус сказал: «Нельзя купить спасение; нельзя заслужить праведность. Спасение --- это дар Бога, а праведность --- естественный результат духовного рождения и жизни в сыновстве в царстве. Не потому дается вам спасение, что вы живете праведной жизнью; но вы живете праведной жизнью потому, что вы уже спасены, приняв сыновство как дар Бога и служение в царстве как верховную радость в жизни на земле. Когда люди уверуют в это евангелие, которое является откровением доброты Бога, они придут к добровольному раскаянию во всех известных грехах. Осознание сыновства несовместимо с желанием грешить. Уверовавшие в царство жаждут праведности и стремятся к божественному совершенству».
\usection{6. Вечерние уроки}
\vs p150 6:1 В вечерних беседах Иисус говорил о многом. В оставшееся время этого путешествия --- до сбора всех в Назарете --- он обсудил «Любовь Бога», «Сны и видения», «Злобу», «Смирение и кротость», «Мужество и преданность», «Музыку и богопочитание», «Служение и послушание», «Гордость и самонадеянность», «Прощение в связи с раскаянием», «Мир и совершенство», «Злословие и зависть», «Зло, грех и искушение», «Сомнения и неверие», «Мудрость и почитание». В отсутствие старших апостолов эти новые группы мужчин и женщин свободнее себя чувствовали при обсуждении этих тем с Учителем.
\vs p150 6:2 Проведя два\hyp{}три дня с одной группой из двенадцати евангелистов, Иисус отправлялся к другой группе, зная от вестников Давида о местонахождении и передвижении их всех. Женщины, поскольку для них это было первое путешествие, значительную часть времени пребывали возле Иисуса. Благодаря вестникам, каждая из этих групп имела полные сведения об успехах путешествия, и новости, приходящие из других групп, постоянно вселяли бодрость в этих рассеянных по Галилее и разделенных друг от друга тружеников.
\vs p150 6:3 Еще до того, как они разделились, было решено, что в пятницу 4 марта двенадцать апостолов вместе с евангелистами и группой женщин соберутся в Назарете и встретятся с Учителем. Соответственно, с приближением этого времени разные группы апостолов и евангелистов начали двигаться в сторону Назарета. К середине дня Андрей и Петр, пришедшие последними, добрались до лагеря, подготовленного прибывшими раньше и расположенного в горах к северу от города. И в первый раз с тех пор, как началось его публичное служение, Иисус посетил Назарет.
\usection{7. Пребывание в Назарете}
\vs p150 7:1 Днем в эту пятницу Иисус ходил по Назарету совершенно незамеченным и не узнанным. Он прошел мимо дома, где прошло его детство, и плотницкой мастерской и провел полчаса на горе, где ему так нравилось бывать в детстве. С того дня, как Иоанн крестил его в Иордане, ни разу не захлестывал его душу такой поток человеческих эмоций. Спускаясь с горы, он услышал знакомые трубные звуки, возвещающие о заходе солнца, те же самые, что он слышал много\hyp{}много раз, пока рос в Назарете. Перед тем, как вернуться в лагерь, он прошел мимо синагоги, в которой в детстве учился, и в памяти всплыли множество приятных воспоминаний о днях детства. До этого в этот день Иисус уже послал Фому договориться с управителем синагоги относительно его проповеди на субботней утренней службе.
\vs p150 7:2 Жители Назарета никогда не пользовались репутацией благочестивых и праведно живущих людей. С годами это селение все больше подпадало под пагубное влияние аморальности близлежащего Сефориса. На протяжении всей юности и первых лет взрослой жизни Иисуса мнения относительно него в Назарете разделялись; осталось чувство обиды, когда он переселился в Капернаум. Хотя жители Назарета много слышали о деяниях их бывшего плотника, они были обижены тем, что он ни разу не посетил свое родное селение в ходе предыдущих проповеднических путешествий. Они много слышали о славе Иисуса, но большинство жителей злились из\hyp{}за того, что он вообще не совершил никаких великих деяний в городе своей юности. Многие месяцы люди в Назарете обсуждали Иисуса, и их мнения были, в целом, неблагоприятными для него.
\vs p150 7:3 Таким образом, Учитель оказался не в благодушной домашней обстановке, а в атмосфере, явной вражды и осуждения. Но это еще не все. Его враги, зная, что он должен провести эту субботу в Назарете, и предполагая, что он будет говорить в синагоге, наняли множество грубых и неотесанных людей, чтобы досаждать ему и всячески бесчинствовать.
\vs p150 7:4 Большинство старых друзей Иисуса, включая очень любившего его учителя\hyp{}хазана, умерли или покинули Назарет, а молодое поколение было склонно очень ревниво и отрицательно относиться к его славе. Они забыли прежнюю его преданность семье отца и резко порицали за то, что он не навещал своего брата и замужних сестер, живущих в Назарете. Это неприязненное чувство жителей еще усиливалось из\hyp{}за отношения к Иисусу со стороны его родственников. Еврейские ортодоксы позволили себе критиковать Иисуса даже за то, что в это субботнее утро он слишком быстро шел к синагоге.
\usection{8. Субботняя служба}
\vs p150 8:1 В эту субботу был прекрасный день, и весь Назарет, друзья и враги, пришли послушать, как этот бывший житель их города будет проповедовать в синагоге. Многим из сопровождавших его пришлось остаться за пределами синагоги; не хватило места для всех пришедших послушать его. В молодости Иисус часто говорил в этой синагоге, и в это утро, когда управитель синагоги вручил ему свиток священного писания для чтения отрывка из Писания, никто из присутствующих, казалось, не вспомнил, что это была та самая рукопись, которую он подарил синагоге.
\vs p150 8:2 Службы в этот день проходили точно так же, как и тогда, когда Иисус еще мальчиком присутствовал на них. Он поднялся на ораторское возвышение вместе с управителем синагоги, и служба началась с чтения двух молитв: «Благословен Господь, Царь мира, творящий свет и создающий темноту, творящий мир и создающий все; из милосердия дающий свет земле и живущим на ней в добродетели, день за днем и каждый день оживляющий свои творения. Хвала нашему Господу Богу за его славные творения и светила, дающие свет, созданные им во славу свою. Села. Благословен наш Господь Бог, создавший светила».
\vs p150 8:3 После короткой паузы они продолжили молиться: «Великой любовью возлюбил нас наш Господь Бог, и бесконечной жалостью пожалел нас Отец наш и Царь наш ради отцов наших, уверовавших в него. Ты научил их заповедям жизни; будь милостив к нам и научи нас. Открой наши глаза на закон; сделай наши сердца преданными твоим заповедям; соедини наши сердца в любви и страхе к имени твоему, и да не покроем мы себя позором, нигде и никогда. Ибо ты --- Бог, дающий спасение, и ты избрал нас из всех народов и языков и воистину приблизил нас к своему великому имени --- села, --- чтобы мы с любовью могли восхвалять твое единство. Благословен Господь, избравший и возлюбивший свой народ Израиль».
\vs p150 8:4 Затем прихожане произнесли Шема, еврейский символ веры. Этот ритуал заключался в повторении многочисленных отрывков из закона и символизировал возложение на себя верующими ярма царства небесного, в том числе ярма заповедей, действующих денно и нощно.
\vs p150 8:5 После этого последовала третья молитва: «Воистину ты --- Яхве, наш Бог и Бог отцов наших; наш Царь и Царь отцов наших; наш Спаситель и Спаситель отцов наших; наш Творец и твердыня нашего спасения; наш помощник и избавитель. Имя твое --- спокон веков, и нет Бога, кроме тебя. Новую песнь воспели избавленные тобой во имя твое у берега моря; все вместе восхвалили и признали тебя Царем и сказали: Яхве да царит во всем мире без конца. Благословен Господь, спасающий Израиль».
\vs p150 8:6 Затем руководитель синагоги занял место перед ковчегом, в котором хранилось священное писание, и начал читать девятнадцать хвалебных молитв, или славословий. Но в этот раз желательно было сократить службу, чтобы у знаменитого гостя осталось больше времени на его проповедь; поэтому прочитаны были только первое и последнее славословия. Первое было: «Благословен наш Господь Бог и Бог наших отцов, Бог Авраама, и Бог Исаака, и Бог Иакова; великий, могущественный и грозный Бог, проявляющий милосердие и доброту, творящий все сущее, помнящий о милостивых обещаниях, данных отцам, и в любви несущий спасение детям их детей во имя свое. О Царь, помощник, спаситель и щит! Благословен ты, О Яхве, щит Авраама».
\vs p150 8:7 Затем последовало последнее славословие: «О ниспошли народу твоему Израилю мир вовеки, ибо ты --- Царь и Господь всего мира. И ты считаешь за благо ниспосылать Израилю мир во все времена и в каждый час. Благословен ты, Яхве, ниспосылающий мир народу Израиля». Прихожане не смотрели на управителя синагоги, пока он читал славословия. После славословий он произнес в свободной форме молитву, подходящую к случаю, и когда она закончилась, все прихожане вместе произнесли аминь.
\vs p150 8:8 Потом хазан подошел к ковчегу, вынул свиток и вручил Иисусу для чтения Писания. Было принято, чтобы семь человек читали не меньше, чем по три стиха из закона, но на этот раз от этой практики отказались с тем, чтобы гость мог сам прочитать места из Писания по своему выбору. Иисус взял свиток, встал и начал читать из Второзакония: «Ибо эта заповедь, которую я даю тебе сегодня, не недоступна для тебя и не далека. Она не на небе, чтобы можно было говорить: кто взошел бы для нас на небо и принес бы ее нам, и дал бы нам услышать ее, и мы бы исполнили ее? И не за морем она, чтобы можно было говорить: кто сходил бы для нас за море и принес бы ее нам, и дал бы нам услышать ее, и мы бы исполнили ее? Нет, весьма близко к тебе это слово жизни, прямо рядом с тобой и в твоем сердце, чтобы ты мог знать и исполнять его».
\vs p150 8:9 Закончив чтение закона, он перешел к Исайе и начал читать: «Дух Господа снизошел на меня, потому что он миропомазал меня проповедовать благую весть бедным. Он послал меня возвестить об освобождении пленников и прозрении слепых, освободить подвергающихся гонениям и провозгласить год, угодный Господу».
\vs p150 8:10 Иисус закрыл книгу и, вернув ее управителю синагоги, сел и обратился к людям с речью. Он начал словами: «Сегодня это Писание исполнено». И затем Иисус почти пятнадцать минут говорил на тему «Сыновья и дочери Бога». Многим проповедь понравилась, и они дивились его благонравию и мудрости.
\vs p150 8:11 В синагоге было принято, чтобы после окончания официальной службы говоривший остался и ответил всем, кто желает задать вопросы. В соответствии с этим в это субботнее утро Иисус спустился к толпе, и люди стали протискиваться вперед, чтобы задать вопросы. В этой группе было много беспокойных личностей, которые были настроены на дурные дела, а в толпе сновали низкие люди, нанятые, чтобы доставлять неприятности Иисусу. Многие из учеников и евангелистов, оставшихся на улице, теперь протиснулись в синагогу и сразу почувствовали, что назревает конфликт. Они хотели увести Учителя, но он с ними не пошел.
\usection{9. Отвержение в Назарете}
\vs p150 9:1 Иисус оказался в синагоге окруженным большой толпой недругов, среди которых была лишь горстка его последователей, и в ответ на их грубые вопросы и ехидные подтрунивания он полушутя заметил: «Да, я сын Иосифа; я плотник, и меня не удивляет, что вы напоминаете мне о поговорке: „Лекарь сам себя лечит“ и вы сомневаетесь в моей способности сделать в Назарете то, что, как вы слышали, я делал в Капернауме; но я призываю вас вспомнить, что даже в Писании говорится, что „нет пророка в своем отечестве и среди своего народа“».
\vs p150 9:2 Но они теснили его и, показывая на него пальцами, говорили: «Ты думаешь, что ты лучше жителей Назарета; ты ушел от нас, но твой брат --- простой работник, и сестры твои по\hyp{}прежнему живут среди нас. Мы знаем твою мать, Марию. Где они сегодня? Мы слышим необыкновенные вещи о тебе, но видим, что ты не совершаешь никаких чудес, когда возвращаешься сюда». Иисус отвечал им: «Я люблю людей, живущих в городе, где я вырос, и я возрадовался бы при виде того, как все вы входите в царство небесное, но деяния Бога определяются не мной. Преображения благодати ниспосылаются в ответ на живую веру тех, кому они предназначены».
\vs p150 9:3 Иисус по\hyp{}хорошему договорился бы с толпой и обезоружил бы даже самых ярых своих врагов, если бы не опрометчивый поступок одного из его апостолов, Симона Зилота, который с помощью Нахора, одного из молодых евангелистов, собрал тем временем вместе друзей Иисуса, находившихся в толпе, и воинственно велел недругам Иисуса убираться прочь. Иисус долго учил апостолов, что кроткий ответ отвращает гнев, но его последователи не привыкли видеть, чтобы к их любимому наставнику, которого они сами так охотно называли Учителем, относились столь грубо и пренебрежительно. Для них это было слишком, и они дали выход своему горячему и сильному возмущению, что только разожгло это безбожное и дремучее сборище. Так что эти негодяи под предводительством наймитов схватили Иисуса и потащили из синагоги к уступу близлежащей отвесной скалы, откуда они вознамерились спихнуть его, чтобы он упал и разбился. Но когда они уже собирались столкнуть его со скалы, Иисус внезапно повернулся к схватившим его и, стоя лицом к ним, спокойно скрестил руки на груди. Он ничего не сказал, но его друзья были более чем удивлены, когда Иисус двинулся вперед и толпа, расступившись, позволила ему пройти, не причинив никакого вреда.
\vs p150 9:4 Ученики последовали за Иисусом в их лагерь, где подробно обо всем рассказали. И вечером в тот же день они приготовились, как повелел Иисус, отправиться на следующий день рано утром обратно в Капернаум. Этот бурный конец третьего проповеднического путешествия оказал отрезвляющее воздействие на всех последователей Иисуса. Они стали понимать некоторые вещи, которым их учил Иисус; они начали сознавать, что путь к царству лежит через многие печали и горькие разочарования.
\vs p150 9:5 Утром в это воскресенье они покинули Назарет и, следуя разными путями, в четверг 10 марта к полудню собрались, наконец, все в Вифсаиде. Теперь они представляли собой группу трезвых, серьезных, лишенных иллюзий проповедников евангелия истины, а не восторженный и всепобеждающий отряд триумфаторов\hyp{}крестоносцев.
