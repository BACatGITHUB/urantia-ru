\upaper{136}{Крещение и сорок дней}
\vs p136 0:1 Иисус начал свое публичное служение в момент наибольшего интереса в народе к проповеди Иоанна, в то время, когда еврейский народ Палестины с нетерпением ожидал появления Мессии. Между Иоанном и Иисусом была огромная разница. Иоанн был энергичным и серьезным работником, а Иисус --- спокойным и счастливым тружеником; за всю свою жизнь он спешил всего несколько раз. Иисус был целительным утешением миру и своего рода примером. Иоанн же утешением или примером был едва ли. Он проповедовал царство небесное, но сам вряд ли испытал счастье пребывающих в нем. Хотя Иисус говорил об Иоанне как о величайшем в ряду пророков древности, он также сказал, что больше Иоанна меньший из тех, кто увидел великий свет нового пути и, идя по нему, вошел в царство небесное.
\vs p136 0:2 Когда Иоанн проповедовал о грядущем царстве, главным в его послании были слова: <<Покайтесь! Бегите от грядущего гнева!>> Когда же проповедовать начал Иисус, то в его речах оставался призыв к покаянию, однако за этим посланием всегда следовало евангелие, благая весть о радости и свободе нового царства.
\usection{1.\bibnobreakspace Представления об ожидаемом Мессии}
\vs p136 1:1 У евреев было множество идей об ожидаемом избавителе, и каждая из этих различных школ мессианского учения могла указать на изречение в иудейских писаниях, которые доказывали ее правоту. В целом евреи считали, что их национальная история берет начало от Авраама, а своей кульминации достигнет с приходом Мессии, в новой эре царства Божиего. В древние времена они представляли себе этого избавителя как <<слугу Господа>>, потом как <<Сына Человеческого>>, а еще позднее некоторые из них пошли дальше и стали называть Мессию <<Сыном Божиим>>. Однако, какими бы именами ни называли его, <<семенем Авраамовым>> или <<сыном Давидовым>>, все соглашались в одном: он должен быть Мессией, <<помазанником>>. Таким образом, представление о Мессии эволюционировало от <<слуги Господа>> к <<сыну Давидову>>, затем к <<Сыну Человеческому>> и <<Сыну Божиему>>.
\vs p136 1:2 Во дни Иоанна и Иисуса наиболее просвещенные евреи выдвинули идею о грядущем Мессии как о совершенном и образцовом израильтянине, исполняющем в качестве <<слуги Господа>> тройное предназначение: пророка, священника и царя.
\vs p136 1:3 Евреи искренне верили: как Моисей освободил их отцов от египетского рабства благодаря сверхъестественным явлениям, так и грядущий Мессия освободит еврейский народ от римского господства, явив еще большие чудеса силы и необыкновенные проявления расового триумфа. Раввины собрали более пятисот изречений Писания, которые, несмотря на то, что они очевидно противоречили друг другу, по их утверждению, были пророчествами о грядущем Мессии. Но среди подробностей, касавшихся времени, способа и деятельности, они почти совсем упустили из вида личность обетованного Мессии. Их интересовало восстановление еврейской национальной славы --- мирское возвышение Израиля, --- а отнюдь не спасение мира. Понятно, почему Иисус из Назарета не мог удовлетворить материалистические представления о Мессии, каким его видело сознание евреев. Многие из предсказаний, считавшиеся евреями мессианскими, рассматривай они эти пророческие высказывания в ином свете, вполне естественно подготовили бы их умы к признанию Иисуса как завершителя одной эры и зачинателя новой и лучшей диспенсации милости и спасения для всех народов.
\vs p136 1:4 \P\ Евреи воспитывались в вере в учение о Шехине. Но этот известный национальный символ Божественного Присутствия был невидим в храме. Они верили, что пришествие Мессии приведет к его восстановлению. Они придерживались путаных представлений о расовом грехе и якобы греховной природе человека. Одни из них учили, что человеческий род проклят за грех Адама и что Мессия снимет это проклятие и вернет человеку расположение Божества. Другие утверждали: создавая человека, Бог вложил в него как доброе, так и злое начало; и когда он увидел результат такого сочетания, глубоко разочаровался и <<раскаялся Господь, что создал человека>>. Те, кто учили подобному, верили, что Мессия придет, дабы избавить человека от этого присущего ему зла.
\vs p136 1:5 Большинство евреев верили, что они продолжают томиться под римским игом за свои национальные грехи и равнодушие прозелитов\hyp{}неевреев. Еврейский народ еще не раскаялся чистосердечно; поэтому Мессия и откладывает свое пришествие. О покаянии много говорили; вот почему возымел такое сильное и немедленное действие призыв Иоанна <<Покайтесь и креститесь, ибо приблизилось царствие небесное>>. Ведь царство небесное для набожного еврея могло означать только одно: пришествие Мессии.
\vs p136 1:6 В пришествии Михаила была особенность, совершенно чуждая еврейскому пониманию Мессии; эта особенность --- соединение двух начал, человеческого и божественного. Евреи по\hyp{}разному представляли себе Мессию --- как совершенного человека, как сверхчеловека и даже как божество, но они никогда не считали, что в нем будет \bibemph{соединено} человеческое и божественное. Это и было великим камнем преткновения для первых учеников Иисуса. Они усвоили человеческое представление о Мессии как сыне Давидовом, предложенное первыми пророками; как Сыне Человеческом, согласно сверхчеловеческой идее Даниила и некоторых из более поздних пророков, и даже как Сыне Божием, каким изображали Мессию автор книги Еноха и некоторые из его современников; но они никогда ни на единое мгновение не верили в истинное представление о соединении в одной земной личности двух начал --- человеческого и божественного. Воплощение Творца в облике создания еще не было явлено. Оно явилось только в Иисусе, и мир ничего не знал о подобных вещах, пока Сын\hyp{}Творец не сделался плотью и не поселился среди смертных этого мира.
\usection{2.\bibnobreakspace Крещение Иисуса}
\vs p136 2:1 Иисус принял крещение, когда проповедь Иоанна достигла наивысшей силы и вся Палестина была охвачена ожиданием исполнения его послания --- <<приблизилось царствие Божие>> --- когда все еврейство было занято серьезным и глубоким самоанализом. Евреи обладали весьма сильным чувством расового единства. Они не только полагали, что грехи отца могут сказаться на его детях, но и твердо верили, что грех одного человека может стать проклятием для всей нации. Вот почему не все, кто пришли креститься к Иоанну, считали себя виновными в конкретных грехах, которые Иоанн осуждал. Многие благочестивые души крестились у Иоанна на благо Израиля. Они опасались, как бы какой\hyp{}нибудь грех, о котором они не ведали, не задержал пришествия Мессии. Чувствуя себя частью виновной и проклятой за грех нации, они омывались, надеясь явить тем самым, проявить расовое покаяние. Поэтому очевидно, что Иисус принял крещение от Иоанна отнюдь не в знак покаяния и не для прощения грехов. Принимая крещение от рук Иоанна, Иисус лишь следовал примеру многих набожных израильтян.
\vs p136 2:2 \P\ Когда Иисус из Назарета вошел в Иордан, чтобы креститься, он был смертным этого мира, достигшим вершины человеческого эволюционного восхождения во всем, что касается завоеваний разума и самоотождествления с духом. В тот день он стоял в Иордане как совершенный смертный эволюционирующих миров времени и пространства. Между смертным разумом Иисуса и пребывающим в нем Настройщиком духа, божественным даром Райского Отца, установилась совершенно синхронная инеограниченная связь. Со времен восхождения Михаила к главенству над вселенной такой же Настройщик пребывает во всех нормальных существах, живущих на Урантии; вся разница в том, что Настройщик Иисуса был предварительно подготовлен к этой особой миссии подобным же пребыванием в Махавенте Мельхиседеке, в другом надчеловеческом существе, воплотившемся в облике смертной плоти.
\vs p136 2:3 Обычно, когда смертный достигает столь высоких уровней совершенства личности, возникают те предварительные явления духовного подъема, которые завершаются окончательным слиянием зрелой души смертного с божественным Настройщиком, связанным с ним. Подобная же перемена, очевидно, должна была произойти и в личностном опыте Иисуса из Назарета в день, когда он со своими двумя братьями вошел в Иордан, чтобы креститься у Иоанна. Эта церемония стала финальным актом его чисто человеческой жизни на Урантии, и многие наблюдатели, надчеловеческие существа, приготовились стать свидетелями слияния Настройщика с разумом, в котором тот пребывал, но всем им суждено было испытать разочарование. Произошло нечто новое и даже более великое. Когда Иоанн возложил на Иисуса руки, чтобы крестить его, пребывающий в том Настройщик окончательно покинул совершенную человеческую душу Иешуа бен Иосифа. Через несколько мгновений сия божественная сущность вернулась из Божеграда как Персонализированный Настройщик, главный из подобных ему во всей локальной вселенной Небадон. Таким образом, Иисус увидел свой собственный прежний божественный дух возвращающимся к нему и нисходящим на него в персонализированном виде. Он услышал, как этот же самый происходивший из Рая дух теперь говорил, глаголя: <<Сей есть Сын Мой возлюбленный, в котором Мое благоволение>>. Иоанн с двумя братьями Иисуса также услышали эти слова. Ученики же Иоанна, стоявшие у края воды, этих слов не слышали, как не видели они и явления Персонализированного Настройщика. Персонализированного Настройщика зрили только глаза Иисуса.
\vs p136 2:4 \P\ Когда возвратившийся и теперь уже возвышенный Персонализированный Настройщик произнес эти слова, установилась полная тишина. И пока все четверо были в воде, Иисус, глядя на Настройщика, находившегося с ним рядом, молился: <<Отче мой, царствующий на небесах, да святится имя твое. Да приидет царство твое! Да будет воля твоя и на земле, как на небе>>. Когда Иисус произнес эту молитву, <<отверзлись небеса>>, и Сын Человеческий увидел данное ему теперь уже Персонализированным Настройщиком видение самого себя как Сына Божиего, кем он и был до пришествия на землю в облике смертной плоти и кем он станет вновь, когда его жизнь во плоти завершится. Это небесное видение зрил один Иисус.
\vs p136 2:5 Иоанн и Иисус слышали голос Персонализированного Настройщика, говорившего от имени Отца Всего Сущего, ибо этот Настройщик исходит от Райского Отца и подобен ему. Весь остаток земной жизни Иисуса этот Персонализированный Настройщик пребывал с ним во всех его трудах; Иисус находился в постоянном общении с этим возвышенным Настройщиком.
\vs p136 2:6 \P\ Принимая крещение, Иисус не каялся ни в каких преступлениях; он не исповедовался ни в каких грехах. Его крещение было посвящением исполнению воли Отца Небесного. Во время него он слышал ясный призыв своего Отца, решительное требование быть в том, что принадлежит Отцу, и на сорок дней уединился, чтобы обдумать связанные с этим многочисленные проблемы. Удалившись, таким образом, на время от активного личного общения со своими земными товарищами, Иисус, пребывая на земле и на Урантии, следовал той процедуре, которая всегда осуществляется в мирах моронтии, когда смертный, идущий по пути восхождения, сливается с внутренним присутствием Отца Всего Сущего.
\vs p136 2:7 В день, когда произошло крещение Иисуса, его чисто человеческая жизнь завершилась. Божественный Сын нашел своего Отца, Отец Всего Сущего нашел своего воплотившегося Сына, и они говорили друг с другом.
\vs p136 2:8 \P\ (Когда Иисус принял крещение, ему был почти тридцать один с половиной лет. Хотя Лука и говорит, что Иисус крестился в пятнадцатый год правления Тиберия кесаря, т.е. в 29 году н.э., поскольку Август умер в 14\hyp{}м, тем не менее, следует отметить, что Тиберий в течение двух с половиной лет до смерти Августа был соправителем Рима, и в его честь в октябре 11 года н.э. были отчеканены монеты. Поэтому пятнадцатым годом фактического правления Тиберия следует считать 26 год н.э.; в этот год и произошло крещение Иисуса. В этом же году Понтий Пилат стал прокуратором Иудеи.)
\usection{3.\bibnobreakspace Сорок дней}
\vs p136 3:1 Великое искушение своего пришествия во плоти Иисус выдержал перед своим крещением, когда в течение шести недель он освежался росой на горе Хермон. Там, на горе Хермон, как лишенный помощи смертный этого мира он встретился с принцем мира сего Калигастией, который претендовал на власть над Урантией, и победил его. В этот богатый событиями день, согласно вселенским записям, Иисус из Назарета стал Планетарным Принцем Урантии. Теперь же сей Принц Урантии, которому столь скоро предстояло быть провозглашенным высшим Владыкой Небадона, оправился в уединение на сорок дней, чтобы составить планы и определить, каким образом провозгласить новое царство Бога в сердцах людей.
\vs p136 3:2 После крещения Иисус вступил в сорокадневный период настройки себя в соответствии с изменившимися взаимоотношениями мира и вселенной, обусловленными персонализацией его Настройщика. Во время этого уединения в Перейских горах он выработал линию поведения, чтобы ей следовать, и методы, чтобы ими пользоваться в этой новой и изменившейся фазе земной жизни, в которую ему предстояло вступить.
\vs p136 3:3 Иисус отправился в уединение отнюдь не затем, чтобы поститься и подвергнуть испытанию свою душу. Он не был аскетом и явился, чтобы навсегда покончить с подобными представлениями о том, как приблизиться к Богу. Причины, побудившие его искать уединения, были совершенно иными, нежели те, что побуждали Моисея, Илию и даже Иоанна Крестителя. Иисус тогда всецело сознавал свое положение по отношению к созданной им вселенной, а также ко вселенной вселенных, управляемой Райским Отцом, его Отцом Небесным. Теперь он до конца восстановил в памяти обязанности, которые налагало на него его пришествие, а также напутствия, данные его старшим братом Иммануилом перед тем, как он принял свое урантийское воплощение. Теперь он полностью и ясно понимал все эти обширнейшиевзаимосвязи и желал на время удалиться, чтобы предаться спокойному размышлению, продумать планы и выбрать методы исполнения своей публичной деятельности на благо этого и всех иных миров своей локальной вселенной.
\vs p136 3:4 \P\ Странствуя в горах в поисках подходящего крова, Иисус встретился с главным распорядителем своей вселенной Гавриилом, Яркой и Утренней Звездой Небадона. Теперь Гавриил восстановил личную взаимосвязь с Сыном\hyp{}Творцом вселенной; это была их первая непосредственная встреча после того, как Михаил простился со своими сотоварищами в Спасограде, отправляясь на Эдентию для подготовки к пришествию на Урантию. По велению Иммануила и согласно воле Древних Дней Уверсы, Гавриил открыл теперь Иисусу, что опыт его пришествия на Урантии почти завершен в том, что касалось обретения совершенного владычества над своей вселенной и прекращения мятежа Люцифера. Первого он достиг в день своего крещения, когда персонализация его Настройщика доказала совершенство и полноту его пришествия в облике смертной плоти; второе стало историческим фактом в день, когда он спустился с горы Ермон и присоединился к ожидавшему его юноше Тиглату. Теперь Иисус узнал, что, по высочайшему указанию локальной вселенной и сверхвселенной, труд его пришествия завершен в той мере, в которой это касалось его личного статуса по отношению к верховной власти и к мятежу. Он уже получил тому прямое подтверждение из Рая в видении, данном ему при крещении, и в явлении персонализации пребывающего в нем Настройщика Мысли.
\vs p136 3:5 Когда Иисус оставался на горе и беседовал с Гавриилом, им явился лично Отец Созвездия Эдентии и сказал: <<Записи сделаны. Владычество Михаила № 611121 над его вселенной Небадон сохраняется полным и неизменным по правую руку Отца Всего Сущего. Я принес тебе от Иманнуила, твоего брата\hyp{}попечителя в воплощении на Урантии, весть об освобождении от пришествия. Ныне же или в любое другое время ты волен прекратить свое пришествие во плоти так, как тебе самому будет угодно, и вознестись, одесную Отца твоего, принять права владычества и взять на себя бразды совершенно заслуженного тобой неограниченного правления всем Небадоном. По указанию Древних Дней, я также свидетельствую о завершении записей сверхвселенной, относящихся к окончанию всякого грешного мятежа в твоей вселенной, и наделении тебя полной и неограниченной властью подавлять любую и каждую из подобных возможных вспышек в будущем. Формально ты совершил все, что должен был сделать на Урантии и во плоти смертного создания. Дальше свой путь ты волен избирать сам>>.
\vs p136 3:6 Когда Всевышний Отец Эдентии покинул их, Иисус еще долго беседовал с Гавриилом о благополучии вселенной и, передавая приветствия Иммануилу, заверил, что в действиях, которые он намерен предпринять на Урантии, он всегда будет помнить о совете, полученном им в связи с предваряющими его пришествие обязанностями, вмененными ему на Спасограде.
\vs p136 3:7 \P\ Все сорок дней уединения Иисуса Иаков и Иоанн, сыновья Заведеевы, искали его. Не один раз они были недалеко от места, где он находился, но его так и не нашли.
\usection{4.\bibnobreakspace Планы публичного служения}
\vs p136 4:1 День за днем, находясь в горах, Иисус строил планы на оставшееся время своего пришествия на Урантии. Прежде всего он решил, что не будет учить одновременно с Иоанном. Он решил оставаться в относительном уединении, пока действия Иоанна не достигнут своей цели, либо до тех пор, пока Иоанн не прекратит свою деятельность вследствие тюремного заключения. Иисус хорошо знал, что бесстрашная и бескомпромиссная проповедь Иоанна в конце концов вызовет опасения и враждебное отношение со стороны гражданских правителей. Учитывая рискованное положение, в котором находился Иоанн, Иисус решительно приступил к составлению программы своих публичных трудов на благо своего народа и мира, на благо всех обитаемых миров своей необъятной вселенной. Пришествие Михаила во плоти произошло на Урантии, но было для всех миров Небадона.
\vs p136 4:2 Первое, что сделал Иисус, обдумав общий план координирования своей программы с движением Иоанна, --- воссоздал в памяти наставления Иммануила. Он тщательно обдумал данный ему совет в отношении методов, которыми ему надлежало пользоваться в своих трудах, а также то обстоятельство, что он не должен оставить после себя на планете никаких долго сохраняющихся записей. Иисус больше никогда не писал ни на чем, кроме песка. Во время следующего посещения Назарета Иисус, к большой печали своего брата Иосифа, уничтожил все свои писания на досках, которые находились в мастерской и были развешены по стенам старого дома. Иисус тщательно обдумал и другой совет Иммануила, касавшийся его экономической, общественной и политической позиции по отношению к миру, каким он застанет его.
\vs p136 4:3 \P\ Иисус отнюдь не постился во время этих сорока дней уединения. Самый большой период, в течение которого он обходился без пищи, продолжался первые два дня, проведенные в горах, когда он был настолько погружен в раздумье, что совершенно забыл о еде. Но на третий день он отправился на поиски пищи. На протяжении этого времени он также не был искушаем никакими злыми духами или мятежными личностями, обитающими в этом мире либо прибывшими из иных миров.
\vs p136 4:4 \P\ В течение этих сорока дней последний раз человеческий и божественный умы держали совет, а точнее, совершалось первое реальное совместное действие этих двух умов, теперь сделавшихся одним. Результаты этого важного периода размышлений убедительно доказали, что божественный ум с триумфом и духовно возобладал над человеческим интеллектом. Отныне разум человека стал разумом Бога, и хотя собственное <<я>> человеческого разума присутствует всегда, всегда же подобный одухотворенный разум человека говорит: <<Не моя воля, но твоя да будет>>.
\vs p136 4:5 Плодами этого насыщенного событиями времени были отнюдь не видения изголодавшегося и изнемогшего ума; не были они и неясными и незрелыми видениями, которые впоследствии были описаны как <<искушения Иисуса в пустыне>>. Скорее, это было временем обдумывания всего богатого событиями и разнообразного пути пришествия на Урантии и тщательного вынашивания таких планов дальнейшего служения, которые не только принесли бы наибольшую пользу этому миру, но и внесли бы определенный вклад в улучшение всех остальных изолированных мятежом сфер. Иисус размышлял обо всем времени существования человека на Урантии со дней Андона и Фонты до срыва Адама и далее до служения Мельхиседека из Салима.
\vs p136 4:6 Гавриил напомнил Иисусу о том, что существуют два пути, идя по которым он может явить себя миру в случае, если он решит на время остаться на Урантии. И Иисусу было разъяснено, что его выбор в данном вопросе не будет связан ни с его владычеством над вселенной, ни с подавлением мятежа Люцифера. Вот какими были эти два пути служения миру:
\vs p136 4:7 \P\ \ublistelem{1.}\bibnobreakspace Его собственный путь --- путь, который мог показаться наиболее приятным и полезным с точки зрения сиюминутных потребностей этого мира и наставления его собственной вселенной.
\vs p136 4:8 \ublistelem{2.}\bibnobreakspace Путь Отца --- явление на долгие времена идеала жизни создания, каким он виделся высоким личностям Райских управителей вселенной вселенных.
\vs p136 4:9 Иисусу, таким образом, было ясно показано, что существуют два пути, по которым он мог направить остаток своей земной жизни. В свете сложившейся к тому моменту ситуации в пользу каждого из этих путей можно было привести определенные доводы. Сын Человеческий четко понимал, что его выбор между этими двумя способами поведения не будет иметь никакого отношения к принятию им права владычества над вселенной; вопрос этот уже был решен, а соответствующее решение закреплено в записях вселенной вселенных и ожидало лишь чтобы он лично их востребовал. Но Иисусу было дано понять, что его Райский брат Иммануил испытает великое удовлетворение, если он, Иисус, найдет возможным для себя закончить свой путь земного воплощения так же, как он столь благородно его начал, всегда подчиняясь воле Отца. В третий день этого уединения Иисус пообещал себе, что он вернется в мир, дабы закончить свой земной путь, и в ситуации, предполагающей выбор любого из двух путей, он всегда предпочтет волю Отца. Этому решению он оставался верен всю оставшуюся часть своей земной жизни. Вплоть до горького конца он неизменно подчинял свою волю владыки воле Отца Небесного.
\vs p136 4:10 \P\ Сорок дней, проведенные в горной глуши, отнюдь не были периодом великого искушения, а были временем великих решений Господа. В течение этих дней уединенного общения с самим собой и непосредственным присутствием своего Отца --- Персонализированным Настройщиком (у него не было больше личного серафима\hyp{}хранительницы) --- он принял одно за другим великие решения, которым предстояло определять его линию поведения на протяжении оставшейся части его земного пути. Впоследствии стало традицией описывать этот период уединения как великое искушение из\hyp{}за того, что его путали с отрывочными повествованиями о борениях на горе Ермон, а еще больше потому, что, по обычаю, все великие пророки и предводители человечества начинали свое общественное служение с подобных же предполагаемых периодов поста и молитвы. По своему обыкновению, Иисус всегда, когда ему предстояло принимать новые или серьезные решения, уединялся для общения со своим собственным духом, дабы узнать волю Бога.
\vs p136 4:11 \P\ Во всех планах на остаток земной жизни, которые вынашивал Иисус, его человеческое сердце терзалось между двумя противоположными линиями поведения:
\vs p136 4:12 \ublistelem{1.}\bibnobreakspace Он испытывал сильное желание добиться того, чтобы его народ --- и весь мир --- поверил в него и принял его новое духовное царство. А он хорошо знал представления своего народа о грядущем Мессии.
\vs p136 4:13 \ublistelem{2.}\bibnobreakspace Жить и работать таким образом, который, как он знал, одобрил бы его Отец, вести свою работу на благо иных нуждающихся миров и продолжать действовать во имя установления царства, открывать Отца и являть его божественную сущность --- любовь.
\vs p136 4:14 \P\ На протяжении всех этих богатых событиями дней Иисус жил в древней каменной пещере, в укрытии в склоне горы, неподалеку от которой находилось селение Бейт\hyp{}Адис. Он утолял жажду из небольшого ручья, который тек по склону горы рядом с этим укрытием.
\usection{5.\bibnobreakspace Первое великое решение}
\vs p136 5:1 На третий день после начала сего совета с собой и своим Персонализированным Настройщиком Иисусу было явлено видение собравшихся небесных воинств Небадона, посланных их предводителями ожидать изъявления воли их возлюбленного Владыки. Сие могущественное воинство состояло из двенадцати легионов серафимов и других разумных существ вселенной, явившихся соразмерным числом от каждого чина. Первое великое решение, принятое Иисусом в уединении, было связано с тем, воспользуется ли он помощью этих могучих личностей для осуществления программы своей публичной деятельности на Урантии.
\vs p136 5:2 Иисус решил, что он \bibemph{не} воспользуется помощью ни одной личности из сего огромного сообщества, пока не станет очевидным, что на то есть \bibemph{воля его Отца.} Однако, несмотря на это знаменательное решение, это небесное воинство оставалось с ним до конца его земной жизни постоянно готовое повиноваться малейшему волеизъявлению своего Владыки. Хотя Иисус своими человеческими глазами видел эти сопровождавшие его личности не постоянно, связанный с ним Персонализированный Настройщик видел их всегда и мог общаться с каждой из них.
\vs p136 5:3 \P\ Перед тем, как вернуться из сорокадневного уединения в горах, Иисус передал непосредственное командование над сопровождавшим его воинством обитателей вселенной своему Персонализированному Настройщику, и эти личности, избранные от каждого чина разумных существ вселенной, в течение более четырех лет времени на Урантии послушно и с почтением действовали под мудрым руководством этого опытного и возвышенного Персонализированного Таинственного Помощника. Принимая командование над этим могущественным сообществом, Настройщик, некогда бывший частью и сущностью Райского Отца, заверил Иисуса в том, что этим надчеловеческим силам ни в коем случае не будет позволено служить ему либо проявлять себя ни в связи с его земным служением, ни на благо его до тех пор, пока не станет очевидно, что Отец желает подобного вмешательства. Таким образом, Иисус одним великим решением добровольно лишил себя помощи всех надчеловеческих существ во всем, что было связано с оставшейся частью его земного пути, в том случае, если Отец сам не решит участвовать в каком\hyp{}либо определенном деянии или в каком\hyp{}либо эпизоде земных трудов Сына.
\vs p136 5:4 Принимая командование над воинством вселенной, сопровождавшим Христа\hyp{}Михаила, Персонализированный Настройщик сделал все, дабы объяснить Иисусу, что хотя подобное сообщество существ может быть ограниченно (вселенной властью переданной ему их Творцом) в своих действиях в \bibemph{пространстве,} такие ограничения не существуют применительно к их действию во \bibemph{времени.} Это обусловлено тем, что после персонализации Настройщики становятся вневременными существами. Поэтому Иисус был предупрежден, что в то время, как контроль Настройщика над живыми разумными существами, находящимися под его командованием, будет полным и совершенным во всем, что связано с пространством, подобных же совершенных ограничений во времени быть не может. Настройщик сказал: <<Как ты повелел мне, я буду запрещать этому сопровождающему тебя воинству вселенских разумных существ действовать хоть каким\hyp{}нибудь образом в твоем земном служении за исключением тех случаев, когда Райский Отец прикажет мне дать свободу подобным силам с тем, чтобы его божественная воля, подчиняться которой ты избрал, была исполнена, а также тогда, когда ты будешь действовать по своей собственной божественно\hyp{}человеческой воле в вопросах, затрагивающих отступление от естественного земного порядка вещей по отношению ко \bibemph{времени.} Во всех подобного рода случаях я бессилен, как бессильны и сотворенные тобой существа, собравшиеся здесь в совершенстве и единстве силы. Если соединившиеся в тебе божественное и человеческое начала однажды, возымеют такое желание, то угодные тебе распоряжения будут выполнены. Твое желание во всех подобного рода случаях заставит время сжаться, и вещь задуманная \bibemph{станет} существовать. В моей власти это составляет самое полное возможное ограничение, которое может быть наложено на твое потенциальное владычество. В моем самосознании времени не существует, и, следовательно, я не могу сдерживать сотворенные тобой существа ни в чем, связанным с ним>>.
\vs p136 5:5 \P\ Таким образом, Иисус был извещен о последствиях своего решения продолжать жить как человек среди людей. Единым решением он отстранил сопровождавшие его воинства различных разумных существ вселенной от участия в его дальнейшем публичном служении за исключением тех вопросов, которые были связаны только со \bibemph{временем.} Понятно, почему любые возможные сверхъестественные или предположительно сверхчеловеческие элементы, сопровождавшие служение Иисуса, были целиком и полностью связаны с упразднением времени, если только Отец Небесный не принимал особого иного решения. Ни чудеса, ни деяния милосердия, ни какие\hyp{}либо другие события, которые могли произойти в связи с оставшейся частью земных трудов Иисуса, ни в коем случае не могли выходить за пределы действия естественных законов, установленных и постоянно действующих в делах человека, живущего на Урантии, \bibemph{кроме} этого четко оговоренного аспекта \bibemph{времени.} Никакие ограничения, конечно, не могли быть наложены на проявление <<воли Отца>>. Упразднения времени в связи с ясно выраженным желанием сего будущего Владыки вселенной можно было избежать только прямым же и ясно выраженным актом \bibemph{воли} этого Богочеловека, в результате которого время в отношении данного акта или события \bibemph{не должно было ни сокращаться, ни упраздняться.} Во избежание возникновения очевидных \bibemph{чудес со временем} Иисусу необходимо было постоянно сознавать время. Любой пробел в осознании времени с его стороны, связанный с испытываемым им определенным желанием, был равносилен осуществлению задуманного в уме этого Сына\hyp{}Творца, причем без участия времени.
\vs p136 5:6 Михаил, благодаря бдительному контролю связанного с ним Персонализированного Настройщика, прекрасно смог ограничить свои личные земные деяния в пространственном аспекте, однако для Сына Человеческого было невозможно ограничить свой новый земной статус в качестве потенциального Владыки Небадона в аспекте \bibemph{времени.} Таково было действительное положение Иисуса из Назарета, когда он приступил к своему публичному служению на Урантии.
\usection{6.\bibnobreakspace Второе решение}
\vs p136 6:1 Определив линию своего поведения в отношении всех личностей всех классов сотворенных им разумных существ настолько, насколько это возможно было с учетом потенциала, присущего его новому статусу божественности, Иисус обратил теперь свои мысли на самого себя. Что он, полностью сознавая теперь себя создателем всех вещей и существ во вселенной, будет делать с этими правами творца в обыденных жизненных ситуациях, с которыми он немедленно столкнется, как только вернется в Галилею,чтобы возобновить свою деятельность среди людей? Фактически данная проблема уже возникла перед ним со всей неумолимостью здесь, в безлюдных горах, где он находился сейчас, в виде необходимости добывать пищу. На третий день его уединенных размышлений человеческое тело начало испытывать голод. Должен ли он отправиться на поиски пищи, как поступил бы любой обыкновенный человек, или же ему следует просто воспользоваться своими естественными творческими силами и произвести подходящую для себя пищу прямо здесь? И это великое решение Господа и преподносили вам как искушение --- как вызов, брошенный ему предполагаемыми врагами, которые предложили ему: <<\ldots скажи, чтобы камни сии сделались хлебами>>.
\vs p136 6:2 Иисус, таким образом, остановился на второй, последовательной линии поведения на время, оставшееся для его земных трудов. Отныне во всем, что касалось его личных потребностей и в целом даже его отношений с другими личностями, он намеренно выбирал путь обычного земного существования и решительно отказывался от поведения, выходящего за пределы установленных им же самим естественных законов, их нарушающего или ими пренебрегающего. Однако он не мог обещать себе, что процессы обесловленные этими естественными законами при определенных обстоятельствах не будет сильно \bibemph{ускоряться,} о чем он уже был предупрежден своим Персонализированным Настройщиком. В принципе Иисус решил, что дело всей его жизни должно быть спланировано и исполнено в соответствии с естественным законом и в полной гармонии с существующей организацией общества. Тем самым Господь выбрал жизненную программу, которая означала отказ от чудес и необычных явлений. Его решение было снова в пользу <<воли Отца>>; он опять передал все в руки своего Райского Отца.
\vs p136 6:3 Человеческая природа Иисуса указывала ему, что его первый долг --- это самосохранение; такова нормальная позиция естественного человека, живущего в мирах времени и пространства, и, следовательно, разумная реакция смертного Урантии. Но Иисус был связан не только с этим миром и населявшими его существами; он жил жизнью, которая должна была стать назиданием и воодушевлением для всего многообразия существ, живущих в необъятной вселенной.
\vs p136 6:4 До своего озарения при крещении он жил в совершенном подчинении воле и водительству Отца Небесного. Он определенно решил и дальше оставаться в подобной же полной зависимости смертного от воли Отца. Он вознамерился придерживаться неестественной линии поведения, решив не заботиться о самосохранении. Он избрал путь, продолжающий политику отказа защищать себя. Он сформулировал свои выводы в словах Писания, привычных его человеческому разуму: <<не хлебом одним будет жить человек, но всяким словом, исходящим из уст Божиих>>. Придя к этому заключению относительно такой потребности физической природы, как голод, Сын Человеческий окончательно отрешился от всех других желаний плоти и естественных побуждений человеческой природы.
\vs p136 6:5 Своей сверхчеловеческой силой он мог пользоваться ради блага других, но на благо себя самого --- никогда. Этой линии поведения он последовательно придерживался вплоть до самого конца, когда о нем глумливо сказали: <<Других спасал, а Себя Самого не может спасти>>, --- потому что он отказался от этого.
\vs p136 6:6 Евреи ожидали Мессию, который совершит чудеса, еще большие, чем Моисей, по преданию в пустыне добывший воду из камня и накормивший их отцовманной. Иисус знал, какого рода Мессию ждут его соотечественники, и обладал всей властью и исключительными возможностями, необходимыми для того, чтобы оправдать их самые оптимистические ожидания, однако он решил отказаться от подобного величественного плана могущества и славы. Иисус смотрел на такой путь совершения чудес, которых от него ждали, как на возврат к давно минувшим дням невежественного колдовства и недостойных деяний врачевателей\hyp{}варваров. Возможно, во имя спасения созданных им существ он смог бы ускорить действие естественного закона, но выходить за рамки своих же законов, будь то ради собственной пользы или для внушения благоговейного страха своим ближним, он не станет никогда. Это решение Господа было окончательным.
\vs p136 6:7 Иисус печалился о своем народе; он полностью понимал, что его привело к ожиданию будущего Мессии, к ожиданию времени, когда <<земля будет приносить плодов своих в десять тысяч раз больше, и на одной лозе будет тысяча веток, и каждая ветвь будет давать тысячу гроздей, и каждая гроздь будет давать тысячу виноградин, и каждая виноградина будет давать галлон вина>>. Евреи верили, что Мессия возвестит новую эру чудесного изобилия. Иудеев издавна воспитывали в традиции чудес и легенд о чудесах.
\vs p136 6:8 Иисус не был Мессией, который пришел приумножать хлеб и вино. Он пришел отнюдь не затем, чтобы служить только временным нуждам; он пришел, чтобы открыть Отца на небе его детям на земле, и вместе с тем он старался привести его земных детей к тому, чтобы они разделили его собственное искреннее стремление жить так, дабы исполнять волю Отца на небесах.
\vs p136 6:9 \P\ Этим решением Иисус из Назарета показал наблюдавшей за ним вселенной все безрассудство и грех использования божественных даров и возможностей, данных Богом, для собственного возвеличивания либо ради чисто эгоистических целей и стремления к личному прославлению. Таким был грех Люцифера и Калигастии.
\vs p136 6:10 Это великое решение Иисуса ярко изобразило истину о том, что эгоистическое удовлетворение и чувственное удовольствие, сами по себе и сами в себе, не способны принести счастье развивающимся человеческим существам. В жизни смертных есть более высокие ценности --- интеллектуальное совершенство и духовные достижения, --- которые намного важнее необходимого удовлетворения чисто физических желаний и потребностей человека. Природный дар человека, его талант и способности, должны, главным образом, служить развитию и облагораживанию его высших возможностей ума и духа.
\vs p136 6:11 Иисус, таким образом, открыл существам своей вселенной способ нового и лучшего пути, высшие нравственные ценности жизни и более глубоких духовных свершений совершенствующегося бытия человека в мирах пространства.
\usection{7.\bibnobreakspace Третье решение.}
\vs p136 7:1 После того, как Иисус принял решение относительно таких проблем, как пища и удовлетворение физических потребностей своего материального тела, забота о собственном здоровье и здоровье своих ближних, еще оставались и другие вопросы, ответ на которые предстояло найти. Какой будет его позиция, когда он столкнется с личной опасностью? Иисус решил прибегать к обычным мерам своей человеческой безопасности и принимать разумные меры предосторожности, чтобы избежать несвоевременного окончания своего служения во плоти, воздерживаясь, однако, от какого\hyp{}либо надчеловеческого вмешательства, когда наступит критический момент его земной жизни. Когда Иисус формулировал это решение, он сидел в тени дерева, которое росло на самом краю скалы, нависшей над пропастью. Он прекрасно понимал, что может броситься со скалы вниз и с ним ничего не случится, если он отменит свое первое решение не прибегать к помощи сотворенных им разумных небесных существ в деле своей жизни на Урантии и изменит второе решение, связанное с его подходом к вопросу самосохранения.
\vs p136 7:2 Иисус знал, что его соотечественники ожидают Мессию, который будет вне естественного закона. Ему было хорошо известно, что Писание гласит: <<Не приключится тебе зло, и язва не приблизится к жилищу твоему. Ибо ангелам своим заповедует о тебе --- охранять тебя на всех путях твоих. На руках понесут тебя, да не преткнешься о камень ногою твоею>>. Может ли подобного рода действия, такое пренебрежение законами гравитации, установленными его Отцом, быть оправдано, чтобы защитить самого себя от возможного вреда, либо, быть может, чтобы снискать доверие своего введенного в заблуждение и сбитого с толку народа? Но такой путь, как бы он ни потворствовал ждущим знамения евреям, отнюдь не будет откровением о его Отце, а лишь сомнительным нарушением установленных законов вселенной вселенных.
\vs p136 7:3 \P\ Понимая все это и зная, что Господь отказался действовать вопреки им же установленным законам природы во всем, что касалось его личного поведения, вы можете быть уверены, что он никогда не ходил по воде и не делал ничего такого, что могло бы противоречить материальному порядку управления миром; при этом, конечно, всегда следует помнить, что тогда еще не был найден способ, благодаря которому он мог быть полностью избавлен от недостатка контроля над элементом времени в вопросах, находившихся в ведении Персонализированного Настройщика.
\vs p136 7:4 Всю свою земную жизнь Иисус оставался неукоснительно верен этому решению. Не важно, насмехались ли над ним фарисеи, прося у него знамения, призывали ли его наблюдавшие казнь на Голгофе сойти с креста, он оставался непоколебим в своем решении, принятом на склоне горы в этот час.
\usection{8.\bibnobreakspace Четвертое решение}
\vs p136 8:1 Следующая великая проблема, с которой боролся сей Богочеловек и которую он в то время решил согласно воле Отца Небесного, затрагивала вопрос, следует ли ему воспользоваться своими надчеловеческими способностями, дабы привлечь внимание своих соотечественников и обрести их приверженность. Должен ли он воспользоваться своим вселенским могуществом ради потворства слабости, которую евреи питали ко всему показному и чудесному? Иисус решил, что он этого делать не будет. Он выбрал линию поведения, исключающую подобный метод привлечения внимания людей к его миссии. И он постоянно жил в соответствии с этим великим решением. Даже тогда, когда в многочисленных деяниях милосердного служения он допускал проявление сокращения времени, он всегда неизменно увещевал тех, к кому это служение было обращено, никому не рассказывать о благах, которые они получили. Он всегда отказывался отвечать на насмешливый вызов своих врагов <<показать им знамение>> в качестве доказательства и примера его божественности.
\vs p136 8:2 Иисус весьма мудро предвидел, что совершение чудес вызовет лишь поверхностную преданность, внушенную материальному разуму благоговейным страхом; подобные действия не дадут откровения о Боге и не приведут к спасению людей. Иисус отказался быть обыкновенным чудотворцем. Он принял решение всецело посвятить себя достижению единой цели --- установлению царства небесного.
\vs p136 8:3 \P\ На протяжении этого важного диалога Иисуса с самим собой в нем присутствовал человеческий элемент неуверенности и некоторого сомнения, ибо Иисус был не только Богом, но и человеком. Очевидно было, что евреи никогда не примут его как Мессию, если он не будет совершать чудеса. Кроме того, если он хотя бы раз согласится сделать нечто, противное естественным законам, то человеческий разум со всей определенностью будет знать, что он находится в подчинении разума истинно божественного. Будет ли подобная уступка божественного разума по отношению к полной сомнений природе разума человеческого сообразна <<воле Отца>>? Иисус решил, что не будет, и заключил, что присутствие Персонализированного Настройщика есть достаточное доказательство союза божественного с человеческим.
\vs p136 8:4 \P\ Иисус много путешествовал; он вспомнил Рим, Александрию и Дамаск. Ему были известны методы, которыми пользуются в этом мире, --- как люди путем компромисса и дипломатии добиваются своего в политике и в торговле. Воспользуется ли он этим знанием, чтобы поддержать свою миссию на земле? Нет! Он также решил, что в установлении царства не пойдет на компромисс со всей ученостью мира и с влиянием, которое дает богатство. Он снова решил полностью положиться на волю Отца.
\vs p136 8:5 Иисус знал о всех кратчайших путях, открытых для существа, обладающего его властью. Ему были известны многие способы, посредством которых он мог сразу сосредоточить на себе внимание нации и всего мира. Вскоре в Иерусалиме будут праздновать Пасху; город наполнится толпами паломников. Он может подняться на шпиль храма и на глазах изумленной толпы пойти по воздуху; как раз такого Мессию евреи и ждали. Но впоследствии ему придется разочаровать их, ведь он пришел отнюдь не затем, чтобы восстанавливать престол Давида. Ему было известно о пагубности метода Калигастии --- попыток форсировать естественный, постепенный и уверенный путь достижения божественной цели. И вновь Сын Человеческий смиренно встал на путь Отца, путь послушания его воле.
\vs p136 8:6 Иисус решил установить царство небесное в сердцах человечества естественным, обычным, тяжелым и трудным способом --- именно таким путем впоследствии пойдут все дети земли в своих трудах во имя увеличения и расширения этого небесного царствия. Ибо Сын Человеческий хорошо знал, что <<многими скорбями надлежит многим детям всех эпох войти в царство>>. Иисус теперь преодолевал великое искушение цивилизованного человека --- обладая властью, последовательно отказываться пользоваться ей в чисто эгоистических или личных целях.
\vs p136 8:7 \P\ Изучая жизнь и опыт Сына Человеческого, необходимо всегда помнить, что Сын Божий был воплощен в разуме человека, жившего в первом веке н.э., а отнюдь не в разуме смертного 20\hyp{}го или иного столетия. Этим мы хотим подчеркнуть, что человеческие дарования Иисуса были естественного происхождения. Он был продуктом наследственности факторов и влияния окружающей среды своего времени, своего воспитания и образования. Его человеческая природа была подлинной, естественной, целиком и полностью продиктованной и обусловленной существующим интеллектуальным уровнем, социальными и экономическими условиями того времени, в которых жило его поколение. Хотя в жизни сего Богочеловека всегда была возможность, что божественный разум возобладает над человеческим интеллектом, тем не менее, когда действовал его человеческий разум и при том действовал именно как человеческий, то делал он это так, как делал бы разум истинно смертного в условиях человеческого окружения тех дней.
\vs p136 8:8 \P\ Иисус показал всем мирам своей необъятной вселенной всю бессмысленность создания искусственных ситуаций с целью демонстрации произвольной власти или использования исключительных возможностей, чтобы усугубить нравственные ценности либо ускорить духовный прогресс. Иисус решил, что он не станет выполнять свою миссию на земле ради повторения разочарования, к которому привело правление Маккавеев. Он отказался унижать свои божественные дары ради приобретения незаслуженной популярности или завоевания политического признания. Он не допустит перерождения божественной и творческой энергии в национальное могущество или международное влияние. Иисус из Назарета отказался идти на компромисс со злом и в еще большей степени --- мириться с грехом. Господь с триумфом поставил преданность воле Отца выше всех других земных и временных соображений.
\usection{9.\bibnobreakspace Пятое решение}
\vs p136 9:1 Решив вопросы, касавшиеся его индивидуального отношения к естественному закону и духовной власти, Иисус обратил свое внимание на выбор методов, которыми он будет пользоваться, возвещая и устанавливая царство Божие. Иоанн уже начал этот труд; каким же образом ему продолжить его дело провозглашения царства? Как перенять эстафету миссии Иоанна? Как организовать своих последователей для эффективной работы и разумного сотрудничества? Иисус теперь готов был принять окончательное решение --- запретить в дальнейшем считать себя еврейским Мессией, по крайней мере, таким Мессией, каким его представляли себе народные массы того времени.
\vs p136 9:2 Евреи ожидали избавителя, который явится облеченным чудесным могуществом, дабы низвергнуть врагов Израиля и сделать евреев правителями мира, свободными от нужды и угнетения. Иисус понимал, что этой надежде не дано осуществиться никогда. Он знал, что царство небесное служит победе над злом в сердцах людей и что оно является исключительно духовной категорией. Он даже подумал, не целесообразно ли провозгласить духовное царство на фоне яркой и ослепительной демонстрации силы --- подобный путь был возможен для Михаила и всецело в его власти, --- но решил полностью отказаться от этого плана. Он не пойдет на компромисс с революционными методами Калигастии. Потенциально он уже завоевал мир, подчинив себя воле Отца, и решил закончить свое дело так же, как его начал, --- как Сын Человеческий.
\vs p136 9:3 Трудно представить себе, что случилось бы на Урантии, если бы сей Богочеловек, теперь потенциально обладающий всей полнотой власти на небе и на земле, в один прекрасный день решил бы развернуть знамя владычества и направить свои чудотворные батальоны боевым порядком! Но он не шел на компромисс. Он не станет служить злу, чтобы посредством этого, быть может, добиться почитания Бога. Он будет твердо придерживаться воли Отца. Он провозгласит наблюдающей за ним вселенной: <<Господу Богу твоему поклоняйся и Ему одному служи>>.
\vs p136 9:4 Шли дни, и Иисус все более ясно видел, как он должен открывать истину. Он сознавал, что путь, указанный Богом, не будет легким. Он начал понимать, что чаша оставшейся части его человеческой жизни, возможно, будет горька, но решил выпить ее.
\vs p136 9:5 Даже его человеческий ум прощается с престолом Давида. Шаг за шагом сей человеческий разум следует путем разума божественного. Человеческий ум еще задает вопросы, но неизменно получает божественные ответы как окончательное решение в этом сочетании жизни человека в миру и одновременного и безусловного подчинения вечной и божественной воле Отца.
\vs p136 9:6 Рим был господином Западного мира. Сын Человеческий, ныне пребывающий в уединении, принимающий эти божественные решения и имеющий в подчинении небесные воинства, являл собою последнюю возможность евреев добиться мирового господства; но этот рожденный на земле еврей, который обладал такой огромной мудростью и силой, отказался использовать дары своей вселенной и для возвеличивания себя самого, и для возведения на престол своего народа. Он видел <<царства мира сего>>, какими они были, и обладал властью, достаточной, чтобы получить их. Всевышние Эдентии передали все эти царства в его руки, но он не хотел их. Земные царства были ничтожными предметами и не могли заинтересовать Творца и Правителя вселенной. У него была только одна цель --- дать человеку более полное откровение о Боге, установить царство и правление Отца Небесного в сердцах человечества.
\vs p136 9:7 Мысль о брани, раздоре и убийстве была для Иисуса отвратительной; ни на что подобное он не пойдет. Он явится на землю как Принц Мира, дабы открыть Бога любви. Перед своими крещением он вновь отказался от предложения Зилотов возглавить их восстание против римских угнетателей. Теперь же он принял окончательное решение в отношении речений Писания, которым учила его мать, зечений, подобных этому: <<Господь сказал Мне; Ты Сын Мой; Я ныне родил Тебя; проси у меня, и дам народы в наследие тебе и пределы земли во владение Тебе; Ты поразишь их жезлом железным, как сосуд горшечника>>.
\vs p136 9:8 Иисус из Назарета пришел к заключению, что подобные изречения не имеют к нему никакого отношения. Наконец, человеческий разум Сына Человеческого окончательно избавился от всех подобных мессианских проблем и противоречий --- от иудейских писаний, родительского воспитания, учения хазана, еврейских чаяний и человеческих амбициозных устремлений; раз и навсегда он определил свой путь. Он вернется в Галилею и начнет мирно возвещать царство, а заботу о ежедневных частностях своего труда доверит Отцу (Персонализированному Настройщику).
\vs p136 9:9 \P\ Когда Иисус отказался применять материальные критерии для решения духовных проблем, когда он отказался самонадеянно пренебрегать естественными законами, то этими решениями он подал достойный пример каждой личности в каждом из миров огромной вселенной. Отказавшись от захвата временной власти в качестве прелюдии к духовной славе, он явил вдохновляющий пример нравственного благородства и верности вселенной.
\vs p136 9:10 \P\ Если, уходя в горы после крещения, Сын Человеческий и имел какие\hyp{}то сомнения относительно своей миссии и ее природы, то теперь, когда после сорока дней уединения и принятия решений он возвратился к своим ближним, у него их не было.
\vs p136 9:11 Иисус выработал программу установления царства Отца. Он не станет потакать физическому услаждению своего народа. Он не будет раздавать толпам хлеб, как это делалось в Риме и чему он совсем недавно был свидетелем. Он не будет привлекать внимания к себе чудотворством, хотя евреи ждут как раз такого освободителя. Он также не будет добиваться принятия духовного послания, являя политическую власть и мирское могущество.
\vs p136 9:12 Отвергнув эти методы повышение значения грядущего царства в глазах исполненных ожидания евреев, Иисус был уверен, что эти самые евреи со всей неизбежностью решительно отвергнут его притязания на власть и божественность. Зная все это, Иисус долго старался не допускать, чтобы его первые последователи называли его Мессией.
\vs p136 9:13 На протяжении всего своего публичного служения он сталкивался с тем, что ему постоянно приходилось иметь дело с тремя одними и теми же ситуациями: с громкими требованиями хлеба, с настойчивыми призывами совершать чудеса и с настойчивой просьбой своих последователей позволить им сделать его царем. Но Иисус ни разу не отступил от решений, которые он принял в эти дни уединения в Перейских горах.
\usection{10.\bibnobreakspace Шестое решение}
\vs p136 10:1 В последний день сего памятного уединения, перед тем, как спуститься с гор и присоединиться к Иоанну и его ученикам, Сын Человеческий принял свое последнее решение. И решение это он сообщил Персонализированному Настройщику в следующих словах: <<И во всех остальных делах, как и в принятых теперь решениях, обещаю тебе, что я буду покорен воле Отца моего>>. Сказав это, он начал спускаться с горы. И лицо его озаряло сияние духовной победы и нравственного подвига.
