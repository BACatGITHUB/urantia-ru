\upaper{142}{Пасха в Иерусалиме}
\author{Комиссия срединников}
\vs p142 0:1 В апреле Иисус и апостолы трудились в Иерусалиме, каждый вечер уходя из города, чтобы провести ночь в Вифании. Сам Иисус каждую неделю одну\hyp{}две ночи проводил в доме греческого еврея Флавия, куда побеседовать с ним тайно приходили многие выдающиеся евреи.
\vs p142 0:2 \P\ В первый день пребывания в Иерусалиме Иисус отправился к своему другу прежних лет Анне, родственнику Саломеи, жены Зеведея, и бывшему первосвященнику. Анна получал известия об Иисусе и его учениях, и когда Иисус явился в дом первосвященника, то принят был очень сдержанно. Почувствовав холодное отношение к себе Анны, Иисус стал сразу прощаться и, уходя, сказал: «Страх --- главный поработитель человека, а гордость --- его великая слабость; неужели ты предашь себя в кабалу этих двух разрушителей радости и свободы?» Но Анна ничего не ответил. Учитель больше не видел Анну до времени, когда тот вместе со своим зятем принял участие в суде над Сыном Человеческим.
\usection{1. Учение в храме}
\vs p142 1:1 Весь этот месяц Иисус или один из апостолов каждый день учили во храме. Когда толпы празднующих Пасху разрастались настолько, что все не могли присутствовать на проповеди в храме, апостолы вели множество учебных групп за пределами священных стен. Суть их послания была такова:
\vs p142 1:2 \ublistelem{1.}\bibnobreakspace Приблизилось царство небесное.
\vs p142 1:3 \ublistelem{2.}\bibnobreakspace Благодаря вере в отцовство Бога, вы можете войти в царство небесное и, таким образом, стать сынами Бога.
\vs p142 1:4 \ublistelem{3.}\bibnobreakspace Любовь есть правило жизни в царстве; это --- высшая преданность Богу и любовь к ближнему, как к самому себе.
\vs p142 1:5 \ublistelem{4.}\bibnobreakspace Подчинение воле Отца, приношение плодов духа в своей личной жизни есть закон царства.
\vs p142 1:6 \P\ Множество людей, пришедших праздновать Пасху, внимали учению Иисуса, и сотни людей радовались благой вести. Главные священники и правители евреев были весьма обеспокоены деяниями Иисуса и его апостолов и совещались между собой о том, что делать с ними.
\vs p142 1:7 Помимо наставлений в храме и возле него апостолы и другие верующие индивидуально работали со многими в пасхальной толпе. Эти проявившие интерес к учению мужчины и женщины разносили весть, содержавшуюся в послании Иисуса, с этого празднования Пасхи в самые отдаленные места Римской империи, а также на Восток. Так началось распространение евангелия царства по всему миру. Отныне дело Иисуса уже больше не ограничивалось только Палестиной.
\usection{2. Гнев Бога}
\vs p142 2:1 В Иерусалиме на пасхальных празднествах присутствовал некий Иаков, богатый торговец\hyp{}еврей с Крита; он пришел к Андрею с просьбой о встрече с Иисусом наедине. Андрей устроил эту тайную встречу с Иисусом в доме Флавия вечером на следующий день. Этот человек не мог понять учения Учителя и пришел, потому что желал полнее разузнать о царстве Божием. Иаков сказал Иисусу: «Но, Рабби, Моисей и древние пророки говорят нам, что Яхве --- Бог ревнитель, что он Бог великой ярости и грозного гнева. Пророки утверждают, что он ненавидит делающих зло и мстит тем, кто не повинуется его закону. Ты же и твои ученики учите нас, что Бог --- добрый и сострадающий Отец, который настолько любит всех людей, что с радостью примет их в этом новом царстве небесном, которое, как вы говорите, уже совсем близко».
\vs p142 2:2 \P\ Когда Иаков кончил говорить, Иисус ответил: «Иаков, ты правильно изложил учение древних пророков, которые учили детей своего поколения в свете своего времени. Наш Отец в Раю неизменен. Но представление о его природе со дней Моисея до времен Амоса и даже времен пророка Исайи расширилось и обогатилось. Ныне же пришел я во плоти, дабы открыть Отца в новой славе и показать его любовь и милосердие ко всем людям во всех мирах. По мере того, как евангилие этого царства будет распространяться в мире, неся с собой послание благой радости и доброй воли ко всем людям, между людьми всех национальностей начнут развиваться лучшие и более совершенные отношения. Со временем отцы и их дети станут больше любить друг друга, что приведет к лучшему пониманию любви Отца на небесах к его детям на земле. Запомни, Иаков, добрый и верный отец не только любит свою семью как целое --- просто как семью --- но он также предано любит каждого ее \bibemph{отдельного} члена и нежно заботится о нем».
\vs p142 2:3 После продолжительной беседы о природе Отца небесного Иисус сделал паузу и сказал: «Ты, Иаков, будучи отцом многих, хорошо знаешь истину моих слов». Иаков сказал: «Но, Учитель, кто сказал тебе, что я --- отец шестерых детей? Как ты узнал это обо мне?» И Учитель ответил: «Достаточно сказать, что Отец и Сын ведают обо всем, ибо они воистину всех видят. Любя детей своих как отец на земле, ты должен теперь как данность принимать любовь Отца небесного к \bibemph{тебе ---} не просто любовь ко всем детям Авраама, а к тебе, к твоей отдельно взятой душе».
\vs p142 2:4 \P\ Затем Иисус продолжал: «Пока дети твои совсем юны и незрелы и пока ты вынужден наказывать их, они могут подумать, что отец их сердит и исполнен возмущенного гнева. Незрелость их за наказанием не способна увидеть дальновидную и воспитующую любовь. Однако, когда те же дети станут взрослыми мужчинами и женщинами, не будет ли глупо с их стороны придерживаться этих прежних и неверных представлений о своем отце? Как мужчины и женщины они теперь должны будут увидеть во всех этих прошлых наказаниях любовь своего отца. Не должно ли по прошествии веков и человечество прийти к лучшему пониманию истинной природы Отца Небесного и его сущности, полной любви? Какая польза тебе от духовных озарений, явленных череде поколений, если ты продолжаешь смотреть на Бога, как смотрели на него Моисей и пророки? Говорю тебе, Иаков, в ярком свете этого часа ты должен видеть Бога так, как не видел его никто из живших прежде. Видя же Бога в таком свете, ты должен радоваться, что можешь войти в царство, где правит столь милосердный Отец, и впредь должен делать все, чтобы жизнь твоя была подчинена его воле, которой движет любовь».
\vs p142 2:5 Иаков ответил: «Верую, Рабби; и хочу, чтобы ты отвел меня в царство Отца».
\usection{3. Представление о Боге}
\vs p142 3:1 Двенадцать апостолов, большинство из которых слушали эту беседу о природе Бога, задали в ту ночь Иисусу много вопросов об Отце на небесах. Ответы Учителя на эти вопросы лучше всего могут быть кратко изложены на современном языке таким образом:
\vs p142 3:2 Иисус мягко попенял двенадцати апостолам и по существу сказал: «Разве не известны вам традиции Израиля, по которым можно судить о развитии представления о Яхве, и разве не знаете вы учения Писания, касающегося представления о Боге?» Затем Учитель стал рассказывать апостолам о развитии понятия о Божестве на всем протяжении эволюции еврейского народа. Он обратил их внимание на следующие фазы развития представления о Боге:
\vs p142 3:3 \ublistelem{1.}\bibnobreakspace \bibemph{Яхве ---} бог синайских кланов. Это было примитивное представление о Божестве, которое Моисей поднял до более высокого уровня Господа Бога Израиля. Отец на небесах всегда принимает искреннее почитание своих детей на земле, каким бы грубым их представление о Божестве ни было и какое бы имя не символизировало для них его божественную природу.
\vs p142 3:4 \P\ \ublistelem{2.}\bibnobreakspace \bibemph{Всевышний.} Это понятие об Отце Небесном было возвещено Аврааму Мелхиседеком и разнесено по свету из Салема теми, кто впоследствии поверил в это расширенное и углубленное представление о Божестве. Авраам и его брат покинули Ур из\hyp{}за введения культа поклонения солнцу; они стали последователями учения Мелхиседека о Эль\hyp{}Элионе --- Всевышнем Боге. Их представление о Боге было смешанным и соединяло в себе их прежние мессопотамские понятия с доктриной о Всевышнем.
\vs p142 3:5 \P\ \ublistelem{3.}\bibnobreakspace \bibemph{Эль Шаддаи.} В эти древние времена многие из иудеев поклонялись Эль Шаддаи, египетскому представлению о Боге небес, о котором они узнали во время своего плена в долине Нила. Много лет спустя со времен Мелхиседека эти три понятия о Боге объединились в одно, образовав доктрину о Божестве\hyp{}творце, Господе Боге Израиля.
\vs p142 3:6 \P\ \ublistelem{4.}\bibnobreakspace \bibemph{Элоим.} Со времен Адама сохранялось учение о Райской Троице. Разве не помните вы начало Писания, где сказано: «Вначале сотворили Боги небо и землю»? Данное утверждение указывает на то, что оно было записано в то время, когда понятие о Троице как о трех Богах в одном нашло место в религии наших предков.
\vs p142 3:7 \P\ \ublistelem{5.}\bibnobreakspace \bibemph{Верховный Яхве.} Ко временам Исайи эти верования о Боге развились до представления о Творце Всего Сущего, который одновременно и всемогущ, и всемилостив. Это развивающееся и расширяющееся понятие о Боге фактически вытеснило все предыдущие представления о Божестве в религии наших отцов.
\vs p142 3:8 \P\ \ublistelem{6.}\bibnobreakspace \bibemph{Отец Небесный.} И ныне мы знаем Бога как нашего Отца на небесах. Наше учение --- это религия, в которой верующий \bibemph{является} сыном Бога. Благая весть евангелия о царстве небесном заключается именно в этом. С Отцом сосуществуют Сын и Дух, и откровение о природе и служении сих Райских Божеств будет расширяться и сиять еще ярче во веки веков вечного духовного совершенствования сыновей Бога, идущих по пути восхождения. Во все времена и во все эпохи истинное почитание Бога любым человеком --- в плане его индивидуального духовного развития --- признается пребывающим в нем духом как почитание Отца Небесного.
\vs p142 3:9 \P\ Никогда еще апостолы не были так потрясены, как тогда, когда выслушали этот рассказ о развитии представления о Боге в умах евреев предыдущих поколений; они были настолько сбиты с толку, что не могли задавать вопросы. Они молча сидели перед Иисусом, и Учитель продолжал: «И вы знали бы эти истины, если бы читали Писание. Разве не читали вы во Второй Книге Царств, где сказано: „Гнев Господень возгорелся на израильтян, и возбудил он против них Давида сказать: пойди, исчисли Израиля и Иуду“? И в этом не было ничего удивительного, ведь во дни Самуила дети Авраама действительно верили, что Яхве сотворил и добро, и зло. Однако, при описании этих событий более поздним автором после того, как представление евреев о природе Бога расширилось, он не решился приписывать зло Яхве; поэтому он сказал: „И восстал сатана на Израиля, и возбудил Давида сделать счисление израильтян“. Разве вы не понимаете, что подобные утверждения в Писании ясно показывают, как из поколения в поколение продолжало развиваться представление о природе Бога?
\vs p142 3:10 И опять вы должны были увидеть, что понимание божественного закона совершенствовалось в точном соответствии с развитием представлений о божественном. Дети Израиля вышли из Египта во дни, когда более полное откровение о Яхве еще не было явлено; тогда у них были десять заповедей, которые служили им законом вплоть до времен, когда они расположились лагерем у Синая. Эти десять заповедей были таковы:
\vs p142 3:11 \ublistelem{1.}\bibnobreakspace Ты не должен поклоняться богу иному, кроме Господа; Он --- Бог ревнитель.
\vs p142 3:12 \ublistelem{2.}\bibnobreakspace Не делай себе богов литых.
\vs p142 3:13 \ublistelem{3.}\bibnobreakspace Не пренебрегай соблюдать праздник опресноков.
\vs p142 3:14 \ublistelem{4.}\bibnobreakspace Все первородное из людей или скота какой у тебя будет, мужеского пола, --- Господу.
\vs p142 3:15 \ublistelem{5.}\bibnobreakspace Шесть дней работай, а в седьмой покойся
\vs p142 3:16 \ublistelem{6.}\bibnobreakspace Соблюдай праздник жатвы первых плодов и праздник собирания плодов в конце года.
\vs p142 3:17 \ublistelem{7.}\bibnobreakspace Не изливай крови жертвы Моей на квасное.
\vs p142 3:18 \ublistelem{8.}\bibnobreakspace Жертва праздника Пасхи не должна быть оставлена до утра.
\vs p142 3:19 \ublistelem{9.}\bibnobreakspace Самые первые плоды земли твоей принеси в дом Господа Бога твоего.
\vs p142 3:20 \ublistelem{10.}\bibnobreakspace Не вари козленка в молоке матери его.
\vs p142 3:21 \P\ Затем, посреди громов и молний синайских Моисей дал им новые десять заповедей, которые, как вы все согласитесь, в большей степени соответствуют более совершенным представлениям о Яхве как Божестве. Разве не замечали вы никогда, что эти заповеди записаны в Писании дважды и что в первом случае в качестве причины соблюдения субботы приводится избавление от египетского рабства, тогда как в более позднем тексте более совершенные верования наших предков потребовали, чтобы причиной соблюдения субботы стало признание факта творения?
\vs p142 3:22 И потом, если помните, --- еще раз в великом духовном озарении дней Исайи --- эти десять запретительных заповедей были преобразованы в великий и утверждающий закон любви, в повеление возлюбить Бога превыше всего и любить своего ближнего, как самого себя. Этот высший закон любви к Богу и к человеку, я также объявляю вам как всеобъемлющий долг человека».
\vs p142 3:23 \P\ Когда Иисус закончил свою речь, никто не задал ему ни одного вопроса. Все разошлись, каждый к своему ночлегу.
\usection{4. Флавий и греческая культура}
\vs p142 4:1 Греческий еврей Флавий, не будучи обрезанным или крещеным, был прозелитом, допущенным лишь ко вратам храма; а поскольку он был большим любителем прекрасного в искусстве и скульптуре, то и дом, который он занимал во время проживания в Иерусалиме, представлял собой великолепное сооружение. Этот дом был изысканно украшен бесценными сокровищами, которые Флавий собирал во время своих странствий по миру. Когда он впервые подумал о том, чтобы пригласить Иисуса к себе в дом, он опасался, что Учитель огорчится, увидев так называемые изображения. Но Флавий был приятно удивлен, когда Иисус вошел в дом и вместо упреков за то, что у того в доме повсюду расставлены эти считавшиеся идолопоклонническими предметы, проявил огромный интерес ко всей коллекции и о каждой вещи задал множество вопросов ценителя и знатока, пока Флавий водил его из комнаты в комнату, показывая ему все свои любимые статуи.
\vs p142 4:2 Учитель видел, что хозяин сбит с толку его доброжелательным отношением к искусству; поэтому, когда они закончили осмотр всей коллекции, Иисус сказал: «Почему ты ожидаешь упреков за то, что ценишь красоту вещей, созданных моим Отцом, которым искусные человеческие руки придали форму? Разве должны все люди с неодобрением относиться к воссозданной красоте и изяществу из\hyp{}за того, что Моисей когда\hyp{}то пытался бороться с идолопоклонством и почитанием ложных богов? Я говорю тебе, Флавий, дети Моисея неверно поняли его и теперь делают ложных богов даже из его запрещений изображений и подобия вещей на небе и на земле. Но если Моисей и приучал к подобным ограничениям темные умы тех дней, то какое это имеет отношение к нашему времени, когда об Отце на небесах дано откровение как о всеобщем Духовном Правителе над всеми? Флавий, я объявляю: в грядущем царстве больше не будут учить: „Не поклоняйся тому и не поклоняйся этому“; и не будут более беспокоиться о соблюдении заповедей воздерживаться от того и стараться не делать этого, но все будут думать об одной высшей обязанности. Эта обязанность человека выражается в двух великих привилегиях: в искреннем почитании бесконечного Творца, Райского Отца и в служении любви, дарованном своим собратьям. Если ты возлюбил своего ближнего, как любишь самого себя, то действительно знаешь, что ты --- сын Бога.
\vs p142 4:3 В эпоху, когда моего Отца понимали недостаточно хорошо, попытки Моисея противостоять идолопоклонству были оправданы, однако в грядущие времена Отец будет явлен в жизни Сына; и это новое откровение о Боге сделает более ненужным спутывать Отца\hyp{}Творца с идолами из камня или изображениями из золота и серебра. Впредь разумные люди смогут наслаждаться сокровищами искусства, отнюдь не путая подобное материальное понимание красоты с поклонением и служением Отцу в Раю, Богу всех вещей и всех существ».
\vs p142 4:4 \P\ Флавий поверил всему, чему учил его Иисус. На следующий день он пошел в Вифанию за Иорданом и был крещен учениками Иоанна. Поступил же он так потому, что апостолы Иисуса еще не крестили верующих. Вернувшись в Иерусалим, Флавий устроил Иисусу великий праздник и пригласил шестьдесят своих друзей. И многие из этих гостей также уверовали в послание о будущем царстве.
\usection{5. Беседа об уверенности}
\vs p142 5:1 Одна из величайших проповедей, произнесенных Иисусом в храме во время этой пасхальной недели, была ответом на вопрос, заданный одним из его слушателей, человеком из Дамаска. Этот человек спросил Иисуса: «Но, Рабби, как нам точно узнать, что ты послан Богом и что мы истинно можем войти в царство, о котором ты и твои ученики объявляете, что оно близко?» Иисус ответил:
\vs p142 5:2 \P\ «Что до моей вести и учения моих последователей, то о них вы должны судить по их плодам. Если мы провозглашаем вам духовные истины, то о подлинности нашего послания будет свидетельствовать дух в ваших сердцах. Что же касается царства и вашей уверенности в том, что Отец Небесный примет вас, то позвольте спросить, какой же отец среди вас, который есть достойный и добросердечный отец, будет держать своего сына в беспокойстве или в неведении относительно его положения в семье или того места, которое он занимает в любящем сердце своего отца? Разве вы, земные отцы, находите удовольствие в том, чтобы мучить ваших детей неопределенностью относительно того места, которое они занимают в ваших сердцах, полных неизменной любви? Так и Отец ваш на небесах не оставляет своих детей, верою рожденных от духа, в полной сомнений неопределенности относительно их положения в царстве. Если вы принимаете Бога как своего Отца, значит, вы действительно и истинно сыны Бога. А раз вы сыны, значит, занимаете прочное положение и надежное место во всем, что касается вечного и божественного сыновства. Если вы верите моим словам, стало быть, верите в Пославшего меня; вера же в Отца укрепит ваше положение как небесных граждан. Если исполняете волю Отца Небесного, то непременно достигните вечной жизни развития в божественном царстве.
\vs p142 5:3 Верховный Дух будет свидетельствовать вашему духу о том, что вы --- воистину дети Бога. Если же вы, сыновья Бога, значит, вы рождены от духа Бога; всякий же рожденный от духа имеет в себе силу побороть все сомнения, и сие есть победа, одолевшая всю неуверенность, вера ваша.
\vs p142 5:4 Говоря о сих временах, пророк Исайя сказал: „Когда излиется на нас Дух свыше, тогда делом праведности будет мир, спокойствие и уверенность во веки“. Каждому, кто истинно верит сему евангелию, я сам буду залогом того, что он будет принят в вечные милости и в бесконечную жизнь в царстве моего Отца. Посему вы, слышащие это послание и верящие сему евангелию царства, --- сыновья Бога и имеете вечную жизнь; и свидетельством всему миру о том, что вы рождены от духа, служит ваша искренняя любовь друг к другу».
\vs p142 5:5 \P\ Толпа слушателей оставалась с Иисусом еще много часов, задавая ему вопросы и внимательно выслушивая его утешительные ответы. Учение Иисуса придало смелости даже апостолам и помогло им проповедовать евангелие царства с еще большей силой и уверенностью. Опыт, обретенный в Иерусалиме, стал для двенадцати великим вдохновителем. Это была их первая встреча со столь огромными толпами, и они усвоили множество ценных уроков, которые оказали им великую помощь в дальнейших трудах.
\usection{6. Посещение Никодима}
\vs p142 6:1 Однажды вечером в дом Флавия повидаться с Иисусом пришел некий Никодим, состоятельный человек, престарелый член Синедриона. Он многое слышал об учении этого галилеянина и поэтому однажды днем пошел послушать, как учит тот во дворах храма. Слушать, как учит Иисус, он ходил бы часто, но опасался, что будет замечен народом, присутствовавшим на его проповедях, ибо правители евреев были настолько несогласны с Иисусом, что ни один член Синедриона не пожелал бы открыто связывать себя с ним. Поэтому Никодим договорился с Андреем о тайной встрече с Иисусом в тот вечер после наступления темноты. Когда началась беседа, Петр, Иаков и Иоанн были в саду Флавия, но позднее вошли в дом, где разговор еще продолжался.
\vs p142 6:2 Принимая Никодима, Иисус не проявил какого\hyp{}либо особого уважения к нему; в разговоре с ним не было ни компромисса, ни чрезмерного старания убедить. Учитель не пытался оттолкнуть своего тайного гостя и не прибегал к сарказму. В своем обращении с необычным посетителем Иисус был спокоен, серьезен и исполнен чувства собственного достоинства. Никодим не был официальным представителем Синедриона; он пришел увидеться с Иисусом только из своего личного и искреннего интереса к учению Учителя.
\vs p142 6:3 После того, как Флавий представил его, Никодим сказал: «Рабби, мы знаем, что ты --- учитель, посланный Богом; ибо ни один человек не может учить так, как ты, если не будет с ним Бог. И я жажду больше узнать про твое учение о грядущем царстве».
\vs p142 6:4 Иисус ответил Никодиму: «Истинно, истинно говорю тебе, Никодим, если кто не родится свыше, не может увидеть царства Бога». Тогда Никодим возразил: «Как может человек родиться заново, будучи стар? Не может он в другой раз войти в утробу матери и родиться».
\vs p142 6:5 Иисус сказал: «И все же я говорю тебе: если человек не родится от духа, не может войти в царствие Бога. Рожденное от плоти есть плоть, а рожденное от духа есть дух. Не удивляйся тому, что я сказал тебе: должно тебе родиться свыше. Когда дует ветер, слышится шелест листьев, но ветра не видно и неизвестно, откуда он пришел и куда уходит, --- так бывает со всяким, рожденным от духа. Глазами плоти можно увидеть проявления духа, но увидеть сам дух нельзя».
\vs p142 6:6 Никодим возразил: «Не пойму --- как это может быть». Иисус сказал: «Может ли быть, что ты --- учитель Израилев, и этого не знаешь? Стало быть, те, кто знает о реальностях духа, должны открывать их тем, кто замечает лишь проявления материального мира. Но поверишь ли ты нам, если мы расскажем тебе о небесных истинах? Хватит ли у тебя смелости, Никодим, поверить в того, кто сошел с неба, даже в Сына Человеческого?»
\vs p142 6:7 Никодим сказал: «Но как мне начать овладевать этим духом, который должен переделать меня, готовя к вступлению в царство?» Иисус ответил: «Дух Отца Небесного уже пребывает в тебе. И если ты будешь следовать велениям этого духа свыше, то очень скоро начнешь смотреть глазами духа, и тогда, искренне избрав водительство духа, будешь рожден от духа, поскольку твоей единственной целью в жизни будет исполнение воли Отца твоего, который на небесах. Итак, родившись от духа и благополучно войдя в царство Бога, ты начнешь в своей каждодневной жизни приносить обильные плоды духа».
\vs p142 6:8 Никодим был совершенно искренен. Он был глубоко поражен, но ушел в смущении. Никодим достиг совершенства в саморазвитии, самоограничении и даже в высших моральных качествах. Он был благороден, себялюбив и альтруистичен; но он не знал, каким образом \bibemph{подчинить} свою волю воле божественного Отца так же, как малое дитя готово подчиняться руководству и наставлениям мудрого и любящего земного отца, и посредством этого в действительности стать сыном Бога, развивающимся сыном\hyp{}наследником вечного царства.
\vs p142 6:9 Но Никодим нашел в себе достаточно веры, чтобы войти в царство. Он слабо протестовал, когда его коллеги в Синедрионе пытались вынести приговор Иисусу, не выслушав его; а позднее с Иосифом Аримафейским открыто признался в своей вере и пришел за телом Иисуса, когда даже большинство учеников бежали от мест последних страданий и смерти своего Учителя.
\usection{7. Урок о семье}
\vs p142 7:1 После деятельного периода пасхальной недели в Иерусалиме, посященного учению и индивидуальной работе, Иисус провел следующую среду в Вифании, отдыхая со своими апостолами. В этот день после полудня Фома задал вопрос, который вызвал пространный и поучительный ответ. Фома сказал: «Учитель, в день, когда мы были выделены как посланники царства, ты многое рассказал нам, наставляя относительно нашего образа жизни, но чему нам учить массы? Как этим людям жить, когда царство придет в большей полноте? Будут ли твои последователи владеть рабами? Должны ли верующие в тебя стремиться к бедности и отказываться от собственности? Будет ли господствовать одно милосердие, так что не будет у нас больше закона и правосудия?» Иисус и двенадцать апостолов провели весь день и весь вечер после ужина, обсуждая вопросы Фомы. В целях этих записей мы приводим следующее краткое изложение наставлений Учителя:
\vs p142 7:2 Иисус прежде всего постарался объяснить своим апостолам, что сам он жил на земле особой, неповторимой жизнью во плоти и что они, двенадцать, были призваны участвовать в этом опыте пришествия Сына Человеческого; и как таковые соратники тоже должны принять на себя многие из особых ограничений и обязательств, связанных с опытом пришествия в целом. В речи Иисуса был скрытый намек на то, что Сын Человеческий был единственной личностью, когда\hyp{}либо жившей на земле, которая одновременно могла зреть и в самое сердце Бога, и в самые глубины человеческой души.
\vs p142 7:3 Иисус очень понятно объяснил, что царство небесное --- это эволюционный опыт, начинающийся здесь, на земле, и проходящий через сменяющие друг друга этапы жизни к Раю. В этот вечер он со всей определенностью заявил, что на некоторой будущей стадии развития царства вновь посетит этот мир в духовной силе и божественной славе.
\vs p142 7:4 Затем он объяснил, что «идея царства» --- отнюдь не лучший способ иллюстрации отношения человека к Богу; что он использовал данную метафору, потому что еврейский народ ожидал царства и потому что Иоанн проповедовал грядущее царство. Иисус сказал: «Люди другой эпохи будут лучше понимать евангелие царства, когда о нем будут говорить на языке, описывающем семейные отношения, --- когда человек будет понимать религию как учение об отцовстве Бога и братстве людей, как учение о сыновстве по отношению к Богу». Учитель довольно долго рассуждал о земной семье как наглядном примере семьи небесной и заново сформулировал два основополагающих закона жизни: первую заповедь о любви к отцу, главе семьи, и вторую заповедь о взаимной любви между детьми --- заповедь любить своего брата, как самого себя. Затем он объяснил, что это свойство братской привязанности будет неизменно являть себя в бескорыстном и преданном служении обществу.
\vs p142 7:5 Вслед за этим произошло памятное обсуждение главных особенностей семейной жизни и их применимости к отношениям, существующим между Богом и человеком. Иисус заявил, что истинная семья основана на следующих семи фактах:
\vs p142 7:6 \ublistelem{1.}\bibnobreakspace \bibemph{Факт существования.} Естественные отношения и явление наследственности у смертных осуществляются в семье: дети наследуют определенные черты своих родителей. Дети происходят от своих родителей; существование личности зависит от акта совершенного родителями. Отношения отца и ребенка присущи всей природе и пронизывают собой всякое живое бытие.
\vs p142 7:7 \P\ \ublistelem{2.}\bibnobreakspace \bibemph{Безопасность и удовольствие.} Истинные отцы получают огромное удовольствие, заботясь о нуждах своих детей. Многие отцы не довольствуются удовлетворением простейших потребностей своих детей, но с радостью обеспечивают также их удовольствия.
\vs p142 7:8 \P\ \ublistelem{3.}\bibnobreakspace \bibemph{Образование и воспитание.} Мудрые отцы тщательно планируют образование и соответствующее воспитание своих сыновей и дочерей. Уже в юном возрасте их готовят к более серьезным обязанностям в дальнейшей жизни.
\vs p142 7:9 \P\ \ublistelem{4.}\bibnobreakspace \bibemph{Дисциплина} \bibemph{и ограничения.} Дальновидные отцы также обеспечивают необходимую дисциплину, руководство и исправление и иногда ограничивают действия своего молодого и незрелго потомства.
\vs p142 7:10 \P\ \ublistelem{5.}\bibnobreakspace \bibemph{Товарищество и верность.} Любящий отец поддерживает со своими детьми близкие и полные любви отношения. Его ухо всегда чутко к их просьбам; он постоянно готов делить с ними лишения и помогать им в трудностях. Отец верховно заинтересован в возрастающем благосостоянии своего потомства.
\vs p142 7:11 \P\ \ublistelem{6.}\bibnobreakspace \bibemph{Любовь и милосердие.} Сострадающий отец легко прощает; отцы не бывают злопамятны к своим детям. Отцы не похожи на судей, врагов или заимодавцев. Настоящие семьи строятся на терпимости, терпении и прощении.
\vs p142 7:12 \P\ \ublistelem{7.}\bibnobreakspace \bibemph{Забота о будущем.} Земные отцы любят оставлять наследство своим сыновьям. Семья продолжается из поколения в поколение. Смерть отмечает лишь конец одного поколения и начало другого. Смерть заканчивает жизнь одного человека и вовсе не обязательно жизнь семьи.
\vs p142 7:13 \P\ Учитель несколько часов обсуждал, как эти особенности семейной жизни приложимы к отношениям человека, земного чада, к Богу, Райскому Отцу. Его вывод был таков: «Эти отношения сына с Отцом во всей полноте я знаю в совершенстве, ибо всего, что вам должно достигнуть в сыновстве в вечном будущем, я достиг уже сейчас. Сын Человеческий готов вознестись одесную Отца, так что во мне явлен путь, ныне шире открытый всем вам, идя по нему, вы сможете увидеть Бога еще до того, как закончите славное восхождение, и станете совершенны, как совершен Отец ваш Небесный».
\vs p142 7:14 Услышав эти необычайные слова, апостолы вспомнили высказывания Иоанна во время крещения Иисуса; они также живо вспоминали происшедшее тогда во время своей проповеднической деятельности после смерти и воскресения Учителя.
\vs p142 7:15 Иисус есть божественный Сын, обладающий всей полнотой доверия Отца Всего Сущего. Он пребывал с Отцом и понимал его во всей полноте. Теперь же, он жил своей земной жизнью в полном соответствии с замыслом Отца, и это воплощение во плоти позволило ему до конца понять человека. Иисус был совершенством человека; он достиг именно того совершенства, какого суждено достигнуть в нем и через него всем истинно верующим. Иисус открыл человеку совершенного Бога и явил Богу в себе совершенного сына миров.
\vs p142 7:16 Хотя Иисус вел беседу в течение нескольких часов, Фома все же не был удовлетворен, поскольку он сказал: «Но, Учитель, мы не находим, что Отец Небесный всегда добр и милосерден к нам. Мы на земле часто мучительно страдаем и не всегда получаем ответ на наши молитвы. В чем мы не понимаем смысла твоего учения?»
\vs p142 7:17 Иисус ответил: «Фома, Фома, когда же ты научишься слышать духовным ухом? Когда же ты поймешь, что сие царство есть царство духовное и что Отец мой --- тоже духовен? Разве ты не понимаешь, что я учу вас как духовных детей духовной семьи небесной, отец и глава которой --- бесконечный и вечный дух? Разве не позволительно мне использовать земную семью в качестве иллюстрации божественных отношений, не прилагая буквально мое учение к делам материальным? Неужели в умах своих вы не можете отделить духовные реальности царства от материальных, социальных, экономических и политических проблем этого времени? Почему, когда я говорю на языке духа, вы упорно переводите смысл моих слов на язык плоти, потому лишь что я позволяю себе в качестве примера использовать обыкновенные и реальные отношения? Дети мои, умоляю вас, перестаньте прилагать учение о царстве духа к недостойным делам рабства, бедности, домов и земель, к материальным проблемам человеческой справедливости и правосудия. Сии временные вопросы заботят людей этого мира, и хотя так или иначе они касаются всех людей, вы призваны представлять меня в мире так же, как я представляю Отца моего. Вы --- духовные посланцы духовного царства, особые представители духовного Отца. Мне уже давно пора наставлять вас как взрослых мужей духовного царства. Так неужели мне всегда придется обращаться к вам только как к детям? Неужели вы никогда не повзрослеете в восприятии духа? И все же я люблю вас и буду терпелив к вам вплоть до самого конца нашего общения во плоти. Но и после дух мой будет шествовать впереди вас по всему миру».
\usection{8. В южной Иудее}
\vs p142 8:1 К концу апреля враждебное отношение к Иисусу среди фарисеев и саддукеев усилилось настолько, что Учитель и его апостолы решили на время покинуть Иерусалим и отправились на юг трудиться в Вифлиеме и Хевроне. Весь месяц май прошел в индивидуальной работе в этих городах и среди народа окрестных селений. Во время этого путешествия не было публичных проповедей, а были лишь посещения отдельных домов. Часть этого времени, пока апостолы учили евангелию и помогали больным, Иисус и Авенир провели в Ен\hyp{}Геди, общаясь с колонией назореев. Иоанн Креститель вышел из этого места, и Авенир возглавил эту группу. Многие из назорейского братства уверовали в Иисуса, но большинство этих аскетичных и эксцентричных людей отказывались принимать его как учителя, посланного небом, поскольку он не учил посту и другим формам самоотречения.
\vs p142 8:2 Люди, жившие в этой местности, не знали, что Иисус родился в Вифлееме. Так же, как большинство его последователей, они всегда считали, что Учитель родился в Назарете, однако двенадцать апостолов знали правду.
\vs p142 8:3 Пребывание на юге Иудеи было спокойным и плодотворным периодом труда, и множество душ обратилось к царству. К первым дням июня агитация против Иисуса в Иерусалиме утихла настолько, что Учитель и апостолы вернулись туда, чтобы наставлять и утешать верующих.
\vs p142 8:4 Хотя Иисус и апостолы почти весь июнь провели в Иерусалиме или недалеко от него, они не учили публично в это время. Большей частью они жили в палатках, которые разбивали в тенистом парке, или в саду, называемом в те дни Гефсиманией. Этот парк был расположен на западном склоне Масличной горы неподалеку от потока Кедрон. Субботы они обычно проводили в Вифании с Лазарем и его сестрами. Иисус входил в стены Иерусалима всего несколько раз, однако множество людей, интересовавшихся его учением, приходило в Гефсиманию увидеться с ним. Однажды в пятницу вечером Никодим и некий Иосиф из Аримафеи осмелились пойти на встречу с Иисусом, но из страха повернули назад, хотя стояли уже перед входом в палатку Учителя. И конечно же они не понимали, что Иисус об их деяниях знает все.
\vs p142 8:5 Узнав, что Иисус вернулся в Иерусалим, правители евреев приготовились арестовать его; однако, увидев, что он не проповедует публично, пришли к заключению, что он напуган их прежними нападками, и решили позволить ему продолжать учить таким частным образом и в дальнейшем не преследовать. И так до последних дней июня все было тихо, пока некий Симон, член Синедриона, не поддержал публично учение Иисуса, объявив себя его последователем перед правителями евреев. Немедленно вспыхнула новая агитация за арест Иисуса, которая усилилась настолько, что Учитель решил удалиться в города Самарии и Десятиградия.
