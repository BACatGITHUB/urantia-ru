\upaper{76}{Второй Сад}
\author{Солония}
\vs p076 0:1 Когда Адам предпочел оставить первый сад нодитам без сопротивления, он и его последователи не пошли на запад, поскольку у эдемитов не было кораблей, способных выдержать морское путешествие. Не могли они идти и на север: северные нодиты уже шли к Эдему. Они боялись идти на юг: эта холмистая местность была наводнена враждебными племенами. Путь был открыт только на восток, и они отправились туда, к прекрасной в те времена области между реками Тигр и Евфрат. И многие из тех, кто оставался, позднее присоединились к адамитам в их новом убежище в долине.
\vs p076 0:2 \pc Каин и Санса оба родились до того, как адамический караван достиг своей цели в междуречье в Месопотамии. Лаотта, мать Сансы, умерла при рождении своей дочери; Ева много страдала, но выжила, так как была более сильной. Ева выкормила Сансу, дочь Лаотты, своею грудью и вырастила ее вместе с Каином. Санса выросла очень талантливой женщиной. Она стала женой Саргана, вождя северной голубой расы, и внесла большой вклад в развитие голубого народа.
\usection{1. Эдемиты приходят в Месопотамию}
\vs p076 1:1 Почти целый год потребовался каравану Адама, чтобы достичь реки Евфрат. Они вышли к реке в сезон разлива и разбили лагерь в долинах к западу от нее. Они простояли там шесть недель, после чего переправились на землю, расположенную между реками, это место и должно было стать их вторым садом.
\vs p076 1:2 Когда до жителей, населяющих эту местность, дошел слух, что на них движется царь и первосвященник Эдемского Сада, они поспешили скрыться в горах на востоке. Прибыв, Адам обнаружил, что вожделенная страна пуста. На новом месте Адам и его помощники занялись работой по строительству домов и созданию нового культурного и религиозного центра.
\vs p076 1:3 Этот район был известен Адаму как один из трех первоначальных вариантов, представленных комитетом, которому Ван и Амадон поручили выбрать подходящее местонахождение для Сада. В те времена две реки сами по себе служили хорошей естественной преградой, а недалеко от второго сада на севере Евфрат и Тигр подходили близко друг к другу, так что для защиты территории, лежащей между двумя реками с юга, надлежало построить только оборонительную стену длиной в пятьдесят шесть миль.
\vs p076 1:4 \pc После того, как они обосновались в новом Эдеме, нужно было привыкать к тяжелому образу жизни. То, что эта земля проклята, казалось абсолютной правдой. Природа вновь возвращалась на круги своя. Теперь адамиты были вынуждены сражаться за свое выживание с необработанной почвой и мириться с житейскими реалиями, сталкиваясь с враждебностью природы и превратностями смертного существования. Первый сад был почти готов, когда они пришли туда, а второй вынуждены были создавать своими руками, трудясь «в поте лица своего».
\usection{2. Каин и Авель}
\vs p076 2:1 Не прошло и двух лет после рождения Каина, как родился Авель, первый ребенок Адама и Евы, появившийся на свет во втором саду. Когда Авель подрос и ему исполнилось двенадцать лет, он решил стать пастухом; Каин предпочел заняться земледелием.
\vs p076 2:2 В те дни существовал обычай приносить в жертву плоды собственного труда. Пастухи приносили животных из своего стада, земледельцы --- часть урожая. По этому обычаю Каин и Авель тоже регулярно совершали жертвоприношения. Оба мальчика часто спорили о том, какое из призваний лучше, скотоводство или земледелие. Авель не преминул заметить, что именно его жертвенным животным было выказано предпочтение. Безуспешно Каин взывал к традициям первого Эдема, где предпочтение отдавалось плодам урожая. Но Авель не принимал это во внимание и насмехался над беспокойством своего старшего брата.
\vs p076 2:3 Конечно, во времена первого Эдема Адам пытался устранить из обряда приношения в жертву животных, так что Каин имел законное основание для своего утверждения. Однако организация религиозной жизни во втором Эдеме оказалась трудным делом. Адам был обременен тысячами забот, связанных с делами строительства, обороны и земледелия. Находясь в состоянии глубокой духовной депрессии, он перепоручил организацию богопочитания и образования потомкам тех нодитов, которые занимались этим в первом саду; и даже за такое короткое время деятельность жрецов\hyp{}нодитов привела к возврату к обычаям и правилам доадамических времен.
\vs p076 2:4 Два мальчика никогда не ладили друг с другом, и этот спор о жертвах еще больше усугублял ненависть между ними. Авель знал, что он --- сын и Адама, и Евы, и никогда не упускал случая напомнить Каину, что Адам не является его отцом. Каин не был чисто\hyp{}фиолетовым, так как его отец происходил из рода нодитов, смешавшихся позднее с голубыми и красными расами и с аборигенами племени андонитов. Все это, в соединении с природной драчливостью, вело к тому, что в Каине все больше и больше усиливалась ненависть к своему младшему брату.
\vs p076 2:5 Юношам было, соответственно, восемнадцать и двадцать лет, когда вражда между ними нашла, наконец, свой выход. Однажды насмешки Авеля привели его драчливого брата в такое исступление, что Каин в ярости набросился на него и убил.
\vs p076 2:6 \pc Поведение Авеля показывает, что окружающая среда и воспитание имеют большое значение как факторы формирования характера. У Авеля была идеальная наследственность, а наследственность лежит в основе формирования любого характера. Но влияние более низменного окружения фактически свело на нет эту великолепную наследственность. Авель, особенно в свои юные годы, испытал на себе большое влияние неблагоприятного окружения. Он был бы совершенно другим человеком, если бы дожил до двадцати пяти или тридцати лет: его благородная наследственность непременно сказалась бы. И если хорошее окружение не может в большой степени способствовать преодолению недостатков характера с дурной наследственностью, то плохое окружение может очень существенно испортить и превосходную наследственность, по крайней мере, в молодые годы. Хорошее социальное окружение и соответствующее образование, как воздух и вода, необходимы для того, чтобы хорошая наследственность проявила себя в полной мере.
\vs p076 2:7 \pc Родители узнали о смерти Авеля, когда собаки пригнали стадо домой без своего хозяина. Для Адама и Евы Каин стал бы постоянным жестоким напоминанием об их безумном поступке, поэтому они поддержали его решение покинуть сад.
\vs p076 2:8 Жизнь Каина в Месопотамии не была такой уж легкой, ибо он в каком\hyp{}то смысле был символом срыва. Не то, чтобы его товарищи плохо к нему относились, просто он не мог не знать, что они подсознательно испытывают к нему чувство неприязни. Каин знал, что, поскольку он не носил никакого племенного знака, он был бы убит первым же человеком из соседнего племени, если тому случилось бы с ним встретиться. Страх и угрызения совести заставили его раскаяться. Настройщик никогда не пребывал в Каине; раньше Каин всегда оказывал вызывающее неповиновение семейной дисциплине и презирал веру своего отца. Но теперь он отправился к Еве, к своей матери, и попросил у нее духовной помощи и наставлений. И когда он искренне захотел получить божественную помощь, Настройщик вошел в него. И этот Настройщик, пребывая в нем и помогая, давал Каину определенное преимущество, превосходство, которое ставило его в один ряд с людьми племени Адама, внушавшими великий страх.
\vs p076 2:9 Итак, Каин ушел в землю Нода к востоку от второго Эдема. Он стал великим вождем одной из групп народа своего отца и, до некоторой степени, реализовал предсказание Серапататии, ибо в течение всей своей жизни поддерживал мир между этой частью нодитов и адамитами. Каин женился на Ремоне, своей отдаленной родственнице, а их первенец, Енох, стал главой эламских нодитов. В течение столетий эламиты и адамиты жили в мире.
\usection{3. Жизнь в Месопотамии}
\vs p076 3:1 По мере того как шло время, во втором саду становились все более очевидными последствия срыва. Адам и Ева очень скучали по своему прежнему дому, по его красоте и покою, по своим детям, отправленным в Эдентию. Конечно, невыносимо тяжело было видеть эту изумительную пару низведенной до положения смертных мира сего, но они выносили свое униженное положение со стойкостью и спокойствием.
\vs p076 3:2 Адам благоразумно тратил большую часть времени на обучение своих детей и их товарищей основам общественного управления, методам образования и религиозным обрядам. Если бы не эта предусмотрительность, после его смерти разразилось бы столпотворение. А так смерть Адама почти не сказалась на ходе дел людских. Но еще задолго до того, как Адам и Ева покинули сей мир, они поняли, что и их дети, и их последователи постепенно сумели забыть время их славы в Эдеме. И это для большинства их последователей было даже лучше: забыв величие Эдема, они не склонны были испытывать ненужное неудовольствие из\hyp{}за своего менее благоприятного окружения.
\vs p076 3:3 \pc Светские правители адамитов происходили по прямой линии от сынов первого сада. Первый сын Адама, Адам\hyp{}сын (Адам бен Адам) основал второй центр фиолетовой расы на севере от второго Эдема. Второй сын Адама, Ева\hyp{}сын, стал искусным правителем и вождем; он был великим помощником своему отцу. Ева\hyp{}сын прожил не так долго, как Адам, и его старший сын, Янсад, стал преемником Адама --- руководителем племен адамитов.
\vs p076 3:4 \pc Религиозные правители (или священники) происходили от Сифа, второго по старшинству сына Адама и Евы, рожденного во втором саду. Он родился спустя сто двадцать девять лет после прибытия Адама на Урантию. Сиф, став главой священников во втором саду, был поглощен работой по улучшению духовного состояния народа своего отца. Его сын Енос основал образ богопочитания, а его внук Каинан организовал миссионерскую службу в ближних и дальних соседних племенах.
\vs p076 3:5 Духовенство сифитов выполняло тройную функцию, включающую религию, здравоохранение и образование. Священники этого ордена были обучены совершать богослужения на религиозных церемониях, работать врачами и санитарными инспекторами и преподавать в школах сада.
\vs p076 3:6 \pc Караван Адама взял с собой в междуречье семена и луковицы сотен растений и злаков из первого сада, они привели с собой большие стада и по несколько особей каждого домашнего животного. Вследствие этого у них было огромное преимущество перед соседними племенами. Они пользовались многими плодами предшествующей цивилизации первого Сада.
\vs p076 3:7 Вплоть до своего исхода из первого сада Адам и его семья всегда питались фруктами, злаками и орехами. По дороге в Месопотамию они в первый раз попробовали травы и овощи. Во втором саду скоро стали употреблять в пищу мясо, но Адам и Ева не прикасались к нему, оно никогда не входило в их ежедневный рацион. Не употребляли мясо в пищу ни Адам\hyp{}сын, ни Ева\hyp{}сын, ни остальные дети из первого поколения, родившиеся в первом саду.
\vs p076 3:8 \pc Адамиты значительно превосходили окружающие народы по своему культурному уровню и интеллектуальному развитию. Они разработали третий алфавит и так или иначе заложили основы всего того, что стало предвестником современного искусства, науки и литературы. Здесь в междуречье Тигра и Евфрата они развили искусство письма, обработки металлов, керамики, ткачества, создали архитектуру, которая оставалась непревзойденной в течение тысячелетий.
\vs p076 3:9 Домашняя жизнь фиолетовых народов, для их времени, была самим совершенством. Дети должны были проходить курсы обучения земледелию, ремесленничеству и животноводству. Или вместо этого они обучались выполнять тройные обязанности сифитов: быть священниками, врачами и учителями.
\vs p076 3:10 Говоря о сифитских священниках, не следует смешивать этих возвышенных и благородных учителей здоровья и веры, этих истинных воспитателей, с невежественным и корыстным духовенством позднейших племен и соседних народов. Их религиозные представления о Боге и вселенной были развитыми и довольно точными, их рецепты сохранения здоровья были для своего времени превосходными, а методы обучения никогда не были превзойдены.
\usection{4. Фиолетовая раса}
\vs p076 4:1 Адам и Ева были родоначальниками фиолетовой расы, девятой человеческой расы на Урантии. У Адама и его потомков были голубые глаза, и фиолетовые люди отличались превосходным телосложением и светлым цветом волос --- это были блондины, шатены или рыжие.
\vs p076 4:2 Ева, как и женщины первых эволюционирующих рас, не испытывала боли при родах. Только женщины смешанных рас, появившихся в результате союза эволюционирующего человека с нодитами, а позднее --- с адамитами, испытывают при рождении детей сильные приступы боли.
\vs p076 4:3 Адам и Ева, как и их собратья в Иерусеме, снабжались энергией двояким образом: употребляя пищу и воспринимая световую энергию, кроме того, они усваивали некие сверхфизические энергии, природа которых не известна на Урантии. Их потомки на Урантии не унаследовали способность родителей усваивать энергию и циркуляцию света. У них уже была единственная система циркуляции --- система кровообращения, подобная человеческой. Им предназначалось быть смертными, хотя жили они намного дольше, но продолжительность их жизни с каждым последующим поколением все уменьшалась, постепенно приближаясь к человеческому стандарту.
\vs p076 4:4 Адам и Ева, их дети в первом поколении не употребляли в пищу мясо животных. Они питались исключительно «плодами деревьев». Начиная со второго поколения, потомки Адама стали употреблять в пищу молочные продукты, но многие продолжали придерживаться рациона питания, исключающего мясо. Многие из южных племен, с которыми они позднее объединились, также не ели мясо. Большинство племен\hyp{}вегетарианцев мигрировали впоследствии на восток и дожили до наших дней, смешавшись с народами Индии.
\vs p076 4:5 Как по физическому, так и по духовному зрению Адам и Ева значительно превосходили современных людей. Их особые органы чувств были значительно тоньше, острее, они могли видеть срединников, ангелов, Мелхиседеков, падшего Принца Калигастию, который несколько раз приходил поговорить со своими благородными преемниками. После срыва, в течение более ста лет они сохраняли способность видеть небесные существа. Такие же особые органы чувств, хотя и не столь острые, были и у их детей, причем чувствительность органов уменьшалась с каждым последующим поколением.
\vs p076 4:6 Обычно в адамических детях всегда пребывал Настройщик, так как все они обладали несомненной способностью к выживанию. Их потомки, находящиеся на более высоком уровне развития, были подвержены чувству страха в значительно меньшей степени, чем дети эволюции. Страх потому так присущ нынешним расам Урантии, что ваши предки вследствие давней неудачи реализации планов расового физического подъема получили слишком мало жизненной плазмы Адама.
\vs p076 4:7 Клетки тела Материальных Сынов и их потомства обладают гораздо большей сопротивляемостью болезням, чем клетки эволюционирующих существ, изначально населявших планету. Клетки туземных рас подобны живым микроскопическим и ультрамикроскопическим организмам, вызывающим болезни. И именно этими фактами объясняется, почему народы Урантии должны на научной основе предпринимать непрерывные усилия, чтобы справиться с великим множеством физических болезней. Вы обладали бы гораздо большей сопротивляемостью к ним, если бы ваши расы несли в себе больше адамической плазмы жизни.
\vs p076 4:8 \pc После создания на Евфрате второго сада Адам решил оставить после себя как можно больше жизненной плазмы, чтобы мир мог воспользоваться этим после его смерти. Соответственно, Ева была назначена главой комиссии двенадцати по усовершенствованию рас. И еще при жизни Адама эта комиссия отобрала на Урантии 1682 женщины, отличающиеся наиболее высоким уровнем развития, и они были оплодотворены плазмой Адама. За исключением 112, все остальные их дети достигли зрелого возраста, так что мир несказанно выиграл в результате появления 1570 превосходных мужчин и женщин. Хотя кандидатуры для материнства отбирались во всех соседних племенах, а они представляли почти все расы земли, большинство женщин, выбранных для этой миссии, принадлежало к высокородным нодитам, и именно они заложили первоначальные основы могучей расы андитов. Их дети рождались и воспитывались в среде племен, к которым принадлежали матери.
\usection{5. Смерть Адама и Евы}
\vs p076 5:1 Вскоре после создания второго Эдема Адам и Ева были соответствующим образом оповещены, что их раскаянье принято и что, хотя они по\hyp{}прежнему обречены разделить судьбу смертных мира сего, но, тем не менее, безусловно заслуживают быть принятыми в число спящих в посмертии Урантии. Они полностью уверовали в эту благую весть о воскресении и восстановлении, которую столь трогательно им возвестили Мелхиседеки. Их прегрешение было заблуждением, а вовсе не грехом осознанного и намеренного мятежа.
\vs p076 5:2 Как граждане Иерусема Адам и Ева не имели Настройщиков Мысли, не было у них и Настройщика в то время, когда они работали на Урантии в первом саду. Но вскоре после того, как они были низведены до положения смертных, они ощутили в себе присутствие чего\hyp{}то нового и ясно осознали, что человеческое существование, соединенное с искренним раскаяньем, сделало возможным пребывание в них Настройщика. Именно сознание того, что в каждом из них пребывает Настройщик, значительно облегчало Адаму и Еве жизнь в течение всего оставшегося времени. Они знали, что, как Материальные Сыны Сатании, они потерпели поражение, но они знали также, что для них, как для возносящихся сынов вселенной, дорога к жизни в Раю все еще остается открытой.
\vs p076 5:3 \pc Адам знал о диспенсационном воскресении, которое произошло одновременно с его прибытием на планету, и он полагал, что он и его подруга, вероятно, реперсонализируются в новые облики в связи с приходом следующего чина сыновства. Он не знал, что Михаил, владыка этой вселенной, вот\hyp{}вот должен появиться на Урантии. Он предполагал, что следующий Сын, который прибудет на Урантию, будет из чина Авоналов. Но, тем не менее, для Адама и Евы всегда было утешением обдумывать одно\hyp{}единственное личное послание, полученное от Михаила, которое, кстати, понять им было отнюдь не просто. В этом послании, среди прочих выражений дружбы и утешения, говорилось: «Я рассмотрел обстоятельства вашего срыва, я знаю стремление ваших сердец всегда быть верными воле моего Отца, и вы будете вызволены из объятий смертного сна, когда я приду на Урантию, если только подчиненные мне Сыны моего царства не призовут вас раньше этого срока».
\vs p076 5:4 И это было великой загадкой для Адама и Евы. Они могли понять это послание как туманное обещание возможности особого воскресения, и это служило им огромным утешением, но оставался непонятным смысл высказывания, что они будут спать до момента воскресения, которое будет связано с личным появлением Михаила на Урантии. Эдемская пара всегда утверждала, что Сын Бога когда\hyp{}нибудь придет, и эту веру, по крайней мере страстную надежду, что мир их поражения и скорби может стать той сценой, на которой правитель этой вселенной решит действовать как Райский Сын Пришествия, они передали своим близким. Это казалось им слишком хорошим, чтобы быть правдой, но Адама временами все\hyp{}таки посещала мысль, что мучимая раздорами Урантия может, в конце концов, стать самой счастливой планетой в системе Сатании, предметом зависти всего Небадона.
\vs p076 5:5 \pc Адам прожил 530 лет. Он умер от того, что можно назвать старостью. Его организм просто физически износился: процесс распада постепенно взял верх над процессом возобновления, и наступил неизбежный конец. Ева умерла на девятнадцать лет раньше от сердечной недостаточности. Они были погребены в центре храма божественного почитания, который построили в соответствии с их планами вскоре после завершения строительства стены колонии. Так было положено начало обычаю хоронить знаменитых и благочестивых мужчин и женщин под плитами мест для Богопочитания.
\vs p076 5:6 \pc Сверхматериальное управление Урантией продолжалось под руководством Мелхиседеков, но непосредственный физический контакт с эволюционирующими расами был прерван. С давних дней, с момента прибытия облеченного в плоть штата Планетарного Принца, затем --- во времена Вана и Амадона и до прибытия Адама и Евы, на планете пребывали материальные представители правительства вселенной. Но адамический срыв положил конец такой форме правления, существовавшей более четырехсот пятидесяти тысяч лет. В духовных сферах ангельские помощники в союзе с Настройщиками Мысли продолжали сражаться, и те, и другие героически боролись за спасение отдельных людей, но вплоть до прибытия Мелхиседека Махивенты смертным людям земли не было объявлено глобального плана достижения мирового благосостояния. Только во времена Авраама Махивента со всей мощью, терпением и властью Сына Бога заложил основы дальнейшего подъема и духовного возрождения несчастной Урантии.
\vs p076 5:7 Однако несчастье не было единственным уделом Урантии; эта планета оказалась самой счастливой во всей локальной вселенной Небадона. Жители Урантии должны считать, что им крупно повезло, что грубые просчеты их предков и ошибки их первых правителей ввергли планету в состояние безнадежной смуты, к тому же усиленное злом и грехом, что беспросветная тьма этой смуты так воззвала к Михаилу из Небадона, что он выбрал этот мир для того, чтобы именно здесь раскрыть любящую сущность небесного Отца. Не то, чтобы Урантия нуждалась в Сыне\hyp{}Творце для приведения в порядок своих запутанных дел, все, скорее, заключалось в том, что зло и грех на Урантии стали прекрасной возможностью еще явственнее открыть несравненную любовь, милосердие и терпение Райского Отца.
\usection{6. Возвращение Адама и Евы к жизни}
\vs p076 6:1 Адам и Ева шли к своей смерти с твердой верой в обещания, которые были даны им Мелхиседеком, что когда\hyp{}нибудь они восстанут от смертного сна, чтобы вновь начать жить в мирах\hyp{}обителях, столь хорошо им знакомых по дням, предшествующим их миссии на Урантии в материальной плоти человека фиолетовой расы.
\vs p076 6:2 Они недолго пребывали в забытьи бессознательного сна смертных мира сего. На третий день после смерти Адама и на второй день после его благоговейного погребения Гавриилу были вручены распоряжения Ланафоржа, приказывающие провести специальную поверку выдающихся личностей, переживших адамический срыв. Распоряжения были подтверждены одним из Всевышних Эдентии, действовавшим в то время, и поддержаны Объединяющим Дней в Спасограде, действующим от имени Михаила. В соответствии с этим мандатом специального воскрешения номер двадцать шесть серии Урантии, Адам и Ева получили новый облик и были воссозданы в залах воскрешения обителей Сатании вместе с 1316 их товарищами по работе в первом саду. Множество других верных душ уже были возвращены к жизни к моменту прибытия Адама, которому сопутствовало решение о диспенсации как спящих в посмертии, так и живых, которым было предназначено восхождение.
\vs p076 6:3 \pc Адам и Ева быстро прошли через миры постепенного восхождения и обрели, наконец, гражданство Иерусема, чтобы снова стать жителями своей родной планеты, но на этот раз --- в качестве существ иного вселенского чина. Они покидали Иерусем будучи постоянными гражданами --- Сынами Бога, они возвратились, став возносящимися гражданами --- сынами человеческими. В столице системы они сразу же были назначены на службу по делам Урантии, а впоследствии членами совета двадцати четырех, который в настоящее время является консультативно\hyp{}контрольным органом Урантии.
\vs p076 6:4 \pc Так заканчивается рассказ о Планетарных Адаме и Еве Урантии, история злоключений, трагедии и триумфа, по крайней мере, личного триумфа ваших Материальных Сына и Дочери, которые заблуждались, хотя и имели добрые намерения; и, несомненно, это история будущего триумфа вашего мира и его обитателей, раздираемых мятежами и пораженных злом. Подводя итоги, можно сказать, что Адам и Ева внесли гигантский вклад в быстрое развитие цивилизации и ускорение биологического прогресса человеческого рода. Они оставили на земле великую культуру, но такая высокая цивилизация не могла выжить, так как адамическое наследие было рано ослаблено и в конечном итоге сошло почти на нет. Народ создает цивилизацию, а цивилизация не создает народ.
\vsetoff
\vs p076 6:5 [Представлено Солонией, серафическим «голосом в Саду».]
