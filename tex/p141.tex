\upaper{141}{Начало публичной деятельности}
\author{Комиссия срединников}
\vs p141 0:1 В первый день недели, 19 января 27 года н.э. Иисус и двенадцать апостолов приготовились покинуть свое пристанище в Вифсаиде. Апостолы ничего не знали о планах своего Учителя, кроме того, что они направляются в Иерусалим, чтобы присутствовать на апрельском праздновании Пасхи, и что предполагается следовать через долину Иордана. Они покинули дом Зеведея только около полудня, потому что семьи апостолов и других учеников пришли попрощаться и пожелать им удачи в новой деятельности, которую они были готовы начать.
\vs p141 0:2 Перед самым уходом апостолы обнаружили отсутствие Учителя, и Андрей отправился искать его. После недолгих поисков он нашел Иисуса сидящим в лодке у берега, и он плакал. Апостолы часто видели своего Учителя, казалось, весьма опечаленным, наблюдали они и краткие периоды, когда он был глубоко погружен в размышления, но никто из них никогда не видел его плачущим. Андрей был в некоторой степени поражен, увидев Учителя в таком состоянии накануне отбытия в Иерусалим, и решился приблизиться к Иисусу и спросить: «В этот великий день, Учитель, когда мы должны отправиться в Иерусалим, чтобы возвестить царство Отца, почему же ты плачешь? Кто из нас обидел тебя?» И Иисус, возвращаясь с Андреем, чтобы присоединиться к апостолам, ответил: «Ни один из вас не огорчил меня. Я опечален лишь тем, что никто из семьи моего отца Иосифа не вспомнил о том, чтобы прийти и пожелать нам счастливого пути». В это время Руфь была в гостях у своего брата Иосифа в Назарете. Других членов семьи удерживали гордость, разочарование, непонимание и мелочное возмущение, возникшие как результат оскорбленных чувств.
\usection{1. Исход из Галилеи}
\vs p141 1:1 Капернаум был недалеко от Тивериады, и слава Иисуса начала распространяться по всей Галилее и даже за ее пределами. Иисус знал, что Ирод скоро обратит внимание на его деятельность; так что он счел за лучшее двинуться со своими апостолами на юг и в Иудею. Более сотни верующих пожелали отправиться с ними, но Иисус обратился к ним и попросил не сопровождать апостольскую группу на пути вниз по Иордану. И хотя те согласились остаться, многие из них последовали за Учителем через несколько дней.
\vs p141 1:2 В первый день Иисус и апостолы добрались только до Тарихеи, где остановились на ночь. На следующий день они дошли до того места на Иордане возле Пеллы, где примерно за год до этого проповедовал Иоанн и где Иисус принял крещение. Здесь они пребывали более двух недель, во время которых учили и проповедовали. К концу первой недели несколько сот человек собрались в лагере возле того места, где жили Иисус и двенадцать апостолов, и пришли они из Галилеи, Финикии, Сирии, Десятиградья, Переи и Иудеи.
\vs p141 1:3 Иисус не произносил публичных проповедей. Андрей поделил все множество людей и назначил проповедников для предполуденных и послеполуденных собраний; после вечерней трапезы Иисус беседовал с двенадцатью апостолами. Он не учил их чему\hyp{}то новому, но повторял все, чему учил прежде, и отвечал на многочисленные вопросы. В один из таких вечеров он рассказал двенадцати апостолам кое\hyp{}что о сорока днях, которые провел в горах поблизости от этого места.
\vs p141 1:4 Многие из тех, кто пришли из Переи и Иудеи, были уже крещены Иоанном и хотели больше узнать об учении Иисуса. Апостолы сильно преуспели в обучении приверженцев Иоанна ввиду того, что никоим образом не умаляли проповеди Иоанна и даже не крестили в это время своих новых учеников. Но для последователей Иоанна всегда было камнем преткновения то, что Иисус, если он был тем, кого предрекал Иоанн, ничего не делал, чтобы освободить его из тюрьмы. Приверженцы Иоанна никогда не могли понять, почему Иисус не предотвратил жестокой смерти их любимого учителя.
\vs p141 1:5 Каждый вечер Андрей тщательно наставлял других апостолов в тонком и сложном деле поддержания хороших взаимоотношений с последователями Иоанна Крестителя. В этот первый год публичного служения Иисуса более трех четвертей его последователей были теми, кто ранее следовали за Иоанном и приняли от него крещение. В течение всего этого года, 27 г. н.э., была потихоньку взята на себя деятельность Иоанна в Перее и Иудее.
\usection{2. Божественный закон и воля Отца}
\vs p141 2:1 Ночью перед отбытием из Пеллы Иисус дал апостолам дальнейшие наставления, касающиеся нового царства. Учитель сказал: «Вас учили ожидать прихода царства Бога, и теперь я пришел, возвещая, что это долгожданное царство приблизилось, даже что оно уже здесь и среди нас. В каждом царстве должен быть царь, восседающий на троне и возвещающий его законы. И таким образом у вас сложилось представление о царстве небесном как о славном первенстве еврейского народа надо всеми народами земли во главе с Мессией, восседающим на троне Давида и с этого места чудодейственной власти, возглашающим законы всему миру. Но, дети мои, вы смотрите не глазами веры, и вы слушаете не духовным слухом. Я возвещаю, что царство небесное есть понимание и признание правления Бога в сердцах людей. Воистину, есть Царь в этом царстве, и Царь этот --- мой Отец и ваш Отец. Мы --- его истинно верные подданные, но намного выше этого факта преображающая истина, что мы --- его \bibemph{сыновья.} В моей жизни эта истина должна стать явленной всем. Наш Отец тоже восседает на троне, но на троне нерукотворном. Трон Бесконечного есть вечная обитель Отца на небе небес; он наполняет собой все сущее и возвещает свои законы вселенным над вселенными. И Отец также правит в сердцах своих детей на земле через дух, который он послал, чтобы жил он в душах смертных людей.
\vs p141 2:2 Если вы подданные этого царства, вы воистину способны услышать закон Правителя Вселенной; но когда, благодаря благой вести царства, которую я пришел возвестить, вы через веру откроете, что вы и есть сыновья, вы будете впредь видеть себя не подвластными закону подданными всесильного царя, а приближенными сыновьями любящего и божественного Отца. Истинно, истинно говорю вам, когда воля Отца --- это ваш \bibemph{закон,} едва ли вы пребываете в царстве. Но когда воля Отца становится воистину вашей \bibemph{волей,} тогда вы и вправду пребываете в царстве, потому что царство тем самым стало укрепившимся в вас опытом. Когда воля Бога --- ваш закон, вы --- благородные подданные\hyp{}рабы; но когда вы верите в это новое евангелие божественного сыновства, воля моего Отца становится вашей волей и вы возноситесь к высокому положению свободных детей Бога, свободных сынов царства».
\vs p141 2:3 Некоторые из апостолов восприняли что\hyp{}то из этого учения, но никто из них не постиг всю значимость этой необычайной вести, разве что Иаков Зеведей. Но эти слова проникли в их сердца и исходили оттуда в последующие годы служения, внося радость в их пастырство.
\usection{3. Пребывание в Амафе}
\vs p141 3:1 Учитель и его апостолы оставались поблизости от Амафы почти три недели. Апостолы продолжали проповедовать перед народом по два раза в день, а Иисус проповедовал каждую субботу после полудня. Стало невозможно, как прежде, отдыхать в среду; поэтому Андрей устроил так, чтобы в каждый из шести дней недели два апостола отдыхали, тогда как в субботней службе принимали участие все.
\vs p141 3:2 Петр, Иаков и Иоанн больше всех проповедовали перед народом. Филипп, Нафанаил, Фома и Симон много работали лично с людьми и вели занятия с теми, кто хотели получить ответы на свои конкретные вопросы; близнецы продолжали осуществлять общий надзор по поддержанию порядка, тогда как Андрей, Матфей и Иуда составили триумвират, занимавшийся общим руководством, хотя каждый из этих троих осуществлял также и значительную религиозную деятельность.
\vs p141 3:3 Андрей был в основном поглощен задачей урегулирования постоянно вспыхивавших взаимонепонимания и разногласий между приверженцами Иоанна и новыми приверженцами Иисуса. Серьезные ситуации возникали каждые несколько дней, но Андрею с помощью его апостольских сподвижников удавалось убеждать противоборствующие стороны прийти к согласию, по крайней мере временному. Иисус отказывался участвовать в каких\hyp{}либо из этих обсуждений; не давал он и никаких советов, как лучше улаживать эти трудности. Он ни разу не дал апостолам указаний, как следует разрешать эти сложные проблемы. Когда Андрей приходил к Иисусу с этими вопросами, он всегда говорил: «Неразумно хозяину участвовать в семейных неурядицах своих гостей; мудрый родитель никогда не принимает чью\hyp{}то сторону в мелочных ссорах своих собственных детей».
\vs p141 3:4 \P\ Во всех отношениях с апостолами и со всеми учениками Учитель проявлял огромную мудрость и при этом идеальную справедливость. Иисус был воистину Учителем людей; он оказывал огромное влияние на своих собратьев благодаря своему обаянию и силе своей личности. Его тяжелая, кочевая и бездомная жизнь оказывала необъяснимое влияние. В его внушительной манере учить, его ясной логике, силе его аргументации, его проницательном понимании, живости ума, его непревзойденной выдержке и его величественной терпимости была интеллектуальная привлекательность и духовная притягательность. Он был простым, мужественным, честным и бесстрашным. Помимо очевидного физического и интеллектуального воздействия личности Учителя, он обладал такими привлекательными качествами, которые стали ассоциироваться с его личностью, --- терпение, нежность, кротость, доброта и смирение.
\vs p141 3:5 Иисус из Назарета был воистину сильной и мощной личностью; он обладал интеллектуальной силой и духовной твердостью. Его личность привлекала не только духовно настроенных женщин из числа его последователей, но и образованного и интеллектуального Никодима, и закаленного римского воина --- командира, поставленного в караул у креста, который после того, как видел смерть Учителя, сказал: «Воистину, он был Сын Бога». И мужественные, крепкие галилейские рыбаки звали его Учителем.
\vs p141 3:6 Картины с изображением Иисуса были в высшей степени неудачными. Эти изображения Христа оказали вредное влияние на молодежь; торговцы в храме вряд ли бежали бы от Христа, если бы он был таким, каким его обычно изображали ваши художники. Его характеризовала величавая мужественность; он был добродетельным но естественным. Иисус не казался мягким, сентиментальным, кротким и добрым мистиком. Его учение было возбуждающе действенным. Он не только \bibemph{желал добра,} но шел по жизни, действительно \bibemph{творя добро.}
\vs p141 3:7 Учитель никогда не говорил: «Придите ко мне все вы, кто празден, и все кто являются мечтателями». Но много раз говорил он: «Придите ко мне все вы, кто \bibemph{трудится,} и я дам вам покой --- духовную силу». Бремя Учителя воистину легко, но, несмотря на это, он никогда не налагает его; каждый конкретный человек должен принять его бремя по собственной свободной воле.
\vs p141 3:8 Иисус показывал, как покорять посредством жертвы, жертвования гордостью и эгоизмом. Проявляя милосердие, он указывал путь духовного избавления от всякого недоброжелательства, недовольства, гнева и эгоистической жажды власти и мести. И когда он говорил: «Не противься злу», --- то впоследствии объяснял, что не имел в виду мириться с грехом или советовать сродниться с порочностью. Он намеревался более всего научить прощению, «не противиться дурному обращению с собственной личностью, злостному оскорблению своего чувства собственного достоинства».
\usection{4. Учение об Отце}
\vs p141 4:1 Пребывая в Амафе, Иисус проводил много времени с апостолами, наставляя их в новом понимании Бога; вновь и вновь внушал он им, что \bibemph{Бог есть Отец,} а не великий и верховный учетчик, чье основное занятие --- делать записи, порочащие своих заблудших детей на земле, отмечать грех и зло, чтобы использовать затем против них, когда он впоследствии будет судить их как справедливый Судья всего творения. Евреи издавна представляли себе Бога как царя всего сущего, даже как Отца своего народа, но никогда прежде какое\hyp{}либо большое количество смертных не воспринимали представления о Боге как любящем Отце \bibemph{каждого индивидуума.}
\vs p141 4:2 На вопрос Фомы: «Кто есть этот Бог царства?» Иисус ответил: «Бог есть \bibemph{твой} Отец, и религия --- моя благая весть --- это ничто иное как исполненное веры осознание истины о том, что ты его сын. И я здесь среди вас во плоти, чтобы своей жизнью и учением сделать явными оба эти представления».
\vs p141 4:3 Иисус также стремился освободить сознание своих апостолов от представления о том, что приносить в жертву животных есть религиозный долг. Но эти люди, воспитанные в обстановке культа ежедневных жертвоприношений, медленно постигали, что он имел в виду. Тем не менее, Учитель не уставал учить. Когда ему не удавалось повлиять на сознание всех апостолов, приводя один пример, он вновь формулировал свою идею и использовал для наглядности притчу другого рода.
\vs p141 4:4 \P\ В это же самое время Иисус начал более подробно учить двенадцать апостолов относительно их миссии «утешать страждущих и служить больным». Учитель многому учил их о человеке в целом --- о единстве тела, разума и духа, образующем конкретного мужчину или женщину. Иисус рассказал своим сподвижникам о трех формах недугов, с которыми они могут встретиться, и затем объяснил, как следует помогать всем, кто испытывает горесть человеческих болезней. Он учил их различать:
\vs p141 4:5 \ublistelem{1.}\bibnobreakspace Болезни плоти --- те недуги, которые обычно считают физическими заболеваниями.
\vs p141 4:6 \ublistelem{2.}\bibnobreakspace Умственные расстройства --- те нефизические недуги, которые впоследствии стали рассматриваться как эмоциональные и психические осложнения и расстройства.
\vs p141 4:7 \ublistelem{3.}\bibnobreakspace Одержимость злыми духами.
\vs p141 4:8 \P\ В нескольких случаях Иисус разъяснял своим апостолам природу и рассказывал им кое\hyp{}что о происхождении этих злых духов, которых в те дни часто называли также нечистыми духами. Учитель хорошо знал, в чем разница между одержимостью злыми духами и безумием, но апостолы этого не знали. Поскольку их знания о ранней истории Урантии были ограниченны, Иисус не мог всесторонне разъяснить им этот вопрос. Но он много раз говорил им, имея в виду этих злых духов: «Они не будут больше досаждать людям, когда я вознесусь к моему Отцу на небеса и пошлю на всякую плоть дух мой и когда царство придет в могуществе и духовном величии».
\vs p141 4:9 От недели к неделе и от месяца к месяцу на протяжении всего этого года апостолы уделяли все больше и больше внимания служению, направленному на исцеление больных.
\usection{5. Духовное единство}
\vs p141 5:1 Одно из самых богатых событиями вечерних совещаний в Амафе было посвящено обсуждению духовного единства. Иаков Зеведей спросил: «Учитель, как нам научиться единству взглядов и таким образом прийти к большему согласию друг с другом?» Услышав этот вопрос, Иисус был настолько взволнован, что ответил: «Иаков, Иаков, когда я учил вас, что вы должны все видеть одинаково? Я пришел в этот мир провозгласить такую духовную свободу, которая даст смертным возможность жить своей собственной самобытной и свободной жизнью перед Богом. Я не желаю, чтобы общественное согласие и братство покупались ценой принесения в жертву свободы личности и духовной самобытности. Чего я требую от вас, апостолы мои, так это \bibemph{духовного единства ---} и его вы можете испытать в радости соединенной воедино приверженности беззаветному исполнению воли моего Отца Небесного. Вы не должны видеть все одинаково или чувствовать одинаково или даже думать одинаково для того, чтобы духовно \bibemph{быть одинаковыми.} Духовное единство проистекает из сознания того, что в каждом из вас постоянно пребывает и все более преобладает духовный дар Отца Небесного. Ваша апостольская гармония должна произрастать из того, что духовные надежды каждого из вас идентичны по своим истокам, природе и предназначению.
\vs p141 5:2 Таким образом вы можете познать на собственном опыте совершенное единство духовной цели и духовного понимания, проистекающее из взаимного осознания пребывающих в каждом из вас Райских духов; и вы можете испытывать это глубочайшее духовное единство, несмотря на величайшее многообразие типов вашего индивидуального интеллектуального мышления, эмоциональности и общественного поведения. Ваши личности могут быть неповторимо разнообразными и явственно различными, тогда как ваши духовные качества и духовные плоды божественного почитания и братской любви могут достичь такого единения, что все, кто наблюдает за вашей жизнью, несомненно, заметят это духовное тождество и единство душ; они узнают, что вы были со мной и поэтому научились, и должным образом, исполнять волю Отца небесного. Вы можете достигнуть единства в служении Богу, даже если совершаете это служение в соответствии с вашими собственными изначальными дарованиями разума, тела и души.
\vs p141 5:3 Ваше духовное единство подразумевает две вещи, которые всегда будут пребывать в гармонии в жизни каждого отдельного верующего. Во\hyp{}первых, вы движимы общим мотивом своего жизненного служения; вы все стремитесь прежде всего исполнять волю Отца Небесного. Во\hyp{}вторых, у всех у вас есть общая цель жизни; вы все стремитесь найти Отца на небесах, тем самым доказывая вселенной, что стали подобны ему».
\vs p141 5:4 Много раз во время обучения двенадцати апостолов Иисус возвращался к этой теме. Не раз он говорил им, что не хочет, чтобы те, кто верят в него, становились догматиками или приходили к единому стандарту, соответствующему религиозным толкованиям пусть даже хороших людей. Снова и снова предостерегал он своих апостолов от формулировки вероучений и установления традиций как средства руководства и контроля над людьми, верующими в евангелие царства.
\usection{6. Последняя неделя в Амафе}
\vs p141 6:1 Незадолго до конца последней недели в Амафе Симон Зелот привел к Иисусу некоего Тегерму, перса, ведущего дела в Дамаске. Тегерма услышал об Иисусе и пришел в Капернаум повидать его, а там, узнав, что Иисус направился со своими апостолами вниз по Иордану в сторону Иерусалима, поспешил, чтобы встретиться с ним. Андрей представил Тегерму Симону, чтобы тот наставил его. Симон считал перса «огнепоклонником», хотя Тегерма прилагал огромные усилия, чтобы объяснить, что огонь лишь зримый символ Чистого и Святого. После беседы с Иисусом перс выказал желание остаться на несколько дней, чтобы услышать учение и послушать проповеди.
\vs p141 6:2 Когда Симон Зелот и Иисус были одни, Симон спросил Учителя: «Почему я не смог убедить его? Почему он так противился мне и с такой готовностью обратил свой слух к тебе?» Иисус отвечал: «Симон, Симон, сколько раз учил я тебя воздерживаться от любых попыток \bibemph{устранить} что\hyp{}либо из сердец тех, кто стремится к спасению? Как часто говорил я тебе, что трудиться надо только для того, чтобы \bibemph{вложить} что\hyp{}то в эти алкающие души? Веди людей в царство, и великие и живые истины царства сразу же изгонят все серьезные заблуждения. Если ты принес смертному человеку благую весть о том, что Бог его отец, тебе легче убедить его, что он воистину сын Бога. А сделав это, ты принес свет спасения тому, кто пребывает во тьме. Симон, когда Сын Человеческий впервые пришел к вам, пришел ли он, осуждая Моисея и пророков и провозглашая новый и лучший образ жизни? Нет. Я пришел не с тем, чтобы отнять то, что вы имели от своих предков, но чтобы указать вам в совершенном виде то, что ваши отцы видели лишь частично. Так иди же, Симон, учи и проповедуй царство, и когда человек будет благополучно и надежно введен в царство, тогда настанет пора, когда такой человек придет к тебе с вопросами и попросит наставлений, касающихся дальнейшего совершенствования души в божественном царстве».
\vs p141 6:3 Симон был удивлен этими словами, но он сделал так, как Иисус учил его, и перс Тегерма оказался в числе тех, кто вошел в царство.
\vs p141 6:4 \P\ В ту ночь Иисус рассказывал апостолам про новую жизнь в царстве. Он сказал, в частности: «Когда вы вступаете в царство, то заново рождаетесь. Нельзя научить глубоким духовным вещам тех, кто рождены только из плоти; сначала убедитесь, что человек рожден в духе, прежде чем поведете его по духовному пути. Не беритесь показывать людям красоты храма до тех пор, пока вначале не ввели их в храм. Представьте людей Богу \bibemph{в качестве} сыновей Бога прежде, чем начнете рассуждать по поводу учения об отцовстве Бога и сыновстве людей. Не боритесь с людьми --- всегда будьте терпеливы. Это не ваше царство; вы только посланцы. Просто идите вперед, провозглашая: это царствие небесное --- Бог есть ваш Отец, а вы его сыновья, и эта благая весть, если вы всем сердцем в нее поверите, \bibemph{есть} ваше вечное спасение».
\vs p141 6:5 Апостолы во многом преуспели за время пребывания в Амафе. Но они были очень сильно разочарованы, что Иисус не дал им никаких советов о том, как вести себя с приверженцами Иоанна. Даже в важном вопросе крещения единственное, что сказал Иисус, было: «Иоанн крестил водой, но когда вы войдете в царство небесное, вы будете крещены Духом «.
\usection{7. В Вифании по другую сторону Иордана}
\vs p141 7:1 6 февраля Иисус, его апостолы и большая группа последователей, следуя вниз по Иордану, добрались до брода возле Вифании в Перее, того места, где Иоанн впервые провозгласил грядущее царство. Иисус со своими апостолами пробыл здесь, уча и проповедуя, четыре недели прежде, чем они отправились дальше в Иерусалим.
\vs p141 7:2 Во вторую неделю пребывания в Вифании за Иорданом Иисус повел Петра, Иакова и Иоанна в горы за реку на юг от Иерихона, чтобы три дня отдохнуть. Учитель поведал этим троим ученикам много новых и более глубоких истин о царстве небесном. Для целей этих записей мы перегруппируем эти учения следующим образом:
\vs p141 7:3 \P\ Иисус старался объяснить, что желает, чтобы его ученики, вкусив благостной духовной сущности царства, жили в миру так, чтобы люди, которые \bibemph{видят} их жизнь, обретали осознание царства и стали бы спрашивать верующих о путях этого царства. Всем этим искренним искателям истины всегда радостно \bibemph{слышать} благую весть о даре веры, который позволяет войти в царство с его вечными и божественными духовными реалиями.
\vs p141 7:4 Учитель стремился внушить всем учителям евангелия царства, что единственное их дело --- открывать каждому отдельному человеку Бога как его отца --- вести каждого человека к осознанию своего сыновства; а затем представить этого человека богу как сына веры. Оба эти важнейшие откровения воплощены в Иисусе. Он воистину «путь, и истина, и жизнь». Религия Иисуса была целиком основана на его жизни в пришествии на землю. Когда Иисус покинул этот мир, он не оставил после себя книг, законов или каких\hyp{}то других форм организации людей, влияющих на их религиозную жизнь.
\vs p141 7:5 Иисус открыл всем, что приходил он, чтобы установить с людьми личные и вечные отношения, которые всегда будут выше любых других человеческих отношений. И он подчеркивал, что это тесное духовное братство должно распространиться на всех людей всех возрастов, всякого общественного положения и всех народов. Единственной наградой, которую он предложил своим детям, было: в этом мире --- духовная радость и общение с Богом; в грядущем мире --- вечная жизнь в постижении божественных духовных реалий Райского Отца.
\vs p141 7:6 Иисус придавал огромное значение тому, что он называл двумя истинами первостепенной важности в учении царства, и они таковы: спасение посредством веры и одной лишь веры, связанное с революционным учением о достижении человеческой свободы путем искреннего осознания истины: «И познаете истину, и истина сделает вас свободными». Иисус был истиной, воплощенной в плоть, и он обещал направить свой Дух Истины в сердца всех своих детей после своего возвращения к Отцу на небеса.
\vs p141 7:7 Иисус учил этих апостолов сути истины, предназначенной для целой эпохи на земле. Они часто слушали то, что в действительности было предназначено для одухотворения и просвещения других миров. Он показал пример нового и необычного образа жизни. С человеческой точки зрения, он действительно был евреем, но жизнь свою он прожил для всего мира как смертный этого мира.
\vs p141 7:8 Чтобы обеспечить признание своего Отца, Иисус, открывая замысел царства объяснял, что намеренно игнорировал «сильных мира сего». Он начал свою деятельность с бедными, тем самым классом, которым так пренебрегали большинство эволюционных религий предшествующих времен. Он не презирал ни одного человека; его замыслы предназначались всему миру, даже всей вселенной. Он был настолько смел и решителен в этих заявлениях, что даже Петр, Иаков и Иоанн невольно склонны были думать, что он, быть может, не в себе.
\vs p141 7:9 Он старался мягко внушить этим апостолам истину, что пришел со своей миссией пришествия не для того, чтобы послужить примером нескольким людям, но чтобы показать и установить образец человеческой жизни для всех народов во всех мирах по всей вселенной. И этот образец близок к высшему совершенству, даже к высшей добродетельности самого Вселенского Отца. Но апостолы не могли понять смысл его слов.
\vs p141 7:10 Он возвестил, что пришел исполнять роль учителя, учителя, посланного с небес, и представить духовную истину материальному разуму. И именно это он и делал; он был учителем, а не проповедником. С человеческой точки зрения, Петр был намного более действенным проповедником, чем Иисус. Действенность проповедей Иисуса проистекала из уникальности его личности и лишь в малой степени была связана с ораторской или эмоциональной их притягательностью. Иисус обращался прямо к душам людей. Он был учителем человеческого духа, но через разум. Он жил с людьми.
\vs p141 7:11 Именно в этот раз Иисус дал понять Петру, Иакову и Иоанну, что его деятельность на земле в некоторой степени ограничена предначертанием, указанным ему «небесным сподвижником», имея в виду совет, данный перед пришествием его Райским братом Иммануилом. Он сказал им, что пришел, чтобы исполнить волю своего Отца и только ее. Будучи, таким образом, движим обуявшей его сердце одной\hyp{}единственной целью, он не был озабочен и обеспокоен злом, царящим в мире.
\vs p141 7:12 Апостолы начинали осознавать искреннее и неподдельное дружелюбие Иисуса. Но хотя Учитель был доступен для людей, он всегда жил независимо от них и как бы над ними всеми. Ни на один миг он не поддавался никакому чисто человеческому влиянию, и не бывал во власти бренных человеческих суждений. Он не обращал никакого внимания на общественное мнение, не оказывали на него воздействия почести и хвала. Он редко отвлекался, чтобы исправить неправильное понимание или возмутиться искажению его слов. Он никогда ни у одного человека не спрашивал совета; он никогда не просил молиться за него.
\vs p141 7:13 Иаков был поражен тем, что Иисус, казалось, видел конец с самого начала. Учитель редко бывал чем\hyp{}либо удивлен. Он никогда не был взволнован, раздосадован или смущен. Он никогда не извинялся ни перед одним человеком. Он бывал временами опечален, но никогда не приходил в уныние.
\vs p141 7:14 Иоанн яснее других осознавал, что, несмотря на все свои божественные дарования, в конечном счете, Иисус был человеком. Он жил как человек среди людей, понимал и любил их, умел управлять людьми. В своей личной жизни он действительно был человеком, но при этом безупречным. И он всегда был бескорыстным.
\vs p141 7:15 Хотя Петр, Иаков и Иоанн не слишком много могли понять из того, что Иисус говорил в этот раз, его добрые слова сохранились в их сердцах, и после распятия и воскресения они вспоминались вновь и вновь, обогащая и вселяя бодрость в их последующее пастырское служение. Не удивительно, что эти апостолы не постигли полностью слова Учителя, ибо он раскрывал перед ними предначертания новой эры.
\usection{8. Деятельность в Иерихоне}
\vs p141 8:1 На протяжении всего четырехнедельного пребывания в Вифании за Иорданом Андрей по нескольку раз каждую неделю поручал апостолам по двое отправляться в Иерихон на день или на два. В Иерихоне было много людей, уверовавших в Иоанна, и большинство из них приветствовали более продвинутое, учение Иисуса и его апостолов. Во время этих посещений Иерихона апостолы непосредственно на практике начали выполнять наставления Иисуса о служении больным; они заходили в каждый дом в городе и старались дать утешение каждому страждущему человеку.
\vs p141 8:2 Апостолы занимались в Иерихоне и публичной деятельностью, но в основном их усилия носили более спокойный и личный характер. Теперь они обнаружили, что благая весть о царстве приносит большое облегчение больным; что их весть является целительной для страждущих. И именно в Иерихоне наказ Иисуса двенадцати апостолам проповедовать радостную весть о царстве и служить страждущим впервые стал в полной мере выполняться.
\vs p141 8:3 Они остановились в Иерихоне по пути в Иерусалим, и там их догнала депутация из Месопотамии, которая прибыла, чтобы поговорить с Иисусом. Апостолы собирались провести здесь всего лишь день, но, когда прибыли эти искатели истины с Востока, Иисус провел с ними три дня, и те вернулись в свои дома, находящиеся в разных местах в долине Евфрата, счастливыми от знания новых истин о небесном царстве.
\usection{9. Отбытие в Иерусалим}
\vs p141 9:1 В понедельник, последний день марта, Иисус и апостолы начали путь в горы в направлении Иерусалима. Лазарь из Вифании дважды приходил к Иордану повидаться с Иисусом, и все уже было приготовлено для того, чтобы Учитель и его апостолы могли остановиться у Лазаря и его сестер в Вифании во время их пребывания в Иерусалиме.
\vs p141 9:2 Последователи Иоанна, оставшись в Вифании за Иорданом, учили и крестили народ, так что Иисуса, когда он прибыл в дом Лазаря, сопровождали только двенадцать апостолов. Здесь Иисус и апостолы оставались пять дней, отдыхая перед тем, как следовать дальше в Иерусалим на Пасху. Великим событием в жизни Марфы и Марии было пребывание Учителя и его апостолов в доме их брата, где они могли служить им, когда они в этом нуждались.
\vs p141 9:3 В воскресенье утром, 6 апреля, Иисус и апостолы отправились в Иерусалим; впервые Учитель и все двенадцать апостолов были там вместе.
