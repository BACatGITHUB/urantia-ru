\upaper{126}{Два критических года}
\author{Комиссия срединников}
\vs p126 0:1 Из всей земной жизни Иисуса четырнадцатый и пятнадцатый годы были наиболее критическими. Эти годы, когда у него уже появилось первое осознание своей божественности и судьбы, но еще не было достаточной степени связи с его внутренним Настройщиком, были наиболее трудными во всей его богатой испытаниями жизни на Урантии. Эти два года можно назвать периодом истинного искушения, великого испытания. Ни один юный человек, проходя через смятения и трудности адаптации юношеского возраста, не переживал более критических испытаний, чем те, через которые прошел Иисус, переходя от детства к ранней зрелости.
\vs p126 0:2 Начался этот важный период юношеского развития Иисуса с завершения его визита в Иерусалим и возвращения в Назарет. Мария была вначале счастлива, что ее мальчик снова с ней --- вернулся домой, чтобы оставаться послушным сыном, каким был всегда. Она думала, что теперь он станет более восприимчив к ее планам относительно его будущего. Но недолго смогла она отдаваться материнским иллюзиям и сокровенной семейной гордости, очень скоро она еще в большей степени утратила свои иллюзии. Мальчик все больше и больше времени проводил с отцом, все реже шел к ней со своими проблемами, в то время как для обоих родителей все более непостижимыми были его частые переходы от занятий этого мира к периодам глубокого раздумья --- это были размышления об его отношении к делам Отца своего. Земные родители, в сущности, не понимали Иисуса, но они искренне его любили.
\vs p126 0:3 \pc Становясь старше, Иисус испытывал все большую любовь и жалость к еврейскому народу, и вместе с тем с годами в нем нарастало праведное возмущение тем, что в храме Отца его служат священники, назначение которых осуществлялось политическим путем. Иисус с большим уважением относился к искренним фарисеям и честным книжникам и с таким же большим презрением --- к лицемерным фарисеям и бесчестным теологам. Он с презрением относился ко всем религиозным деятелям, которые не были искренними. Думая о лидерах Израиля, он иногда чувствовал искушение склониться к возможности сделаться Мессией, которого ожидают евреи, но ни разу все же не поддался этому соблазну.
\vs p126 0:4 \pc История об его подвигах в общении с мудрецами в Иерусалимском храме доставила удовольствие всему Назарету, особенно его бывшим учителям в синагогальной школе. Некоторое время все превозносили его. Все вспоминали его мудрость и прекрасное поведение в детстве, пророчили ему будущность великого вождя Израиля: наконец\hyp{}то из Назарета в Галилее выйдет воистину великий учитель. И все предвкушали тот день, когда Иисусу исполнится пятнадцать лет и ему будет позволено регулярно читать Писание в синагоге по Субботам.
\usection{1. Его четырнадцатый год (8 г. н.э.)}
\vs p126 1:1 Это календарный год его четырнадцатого дня рождения. Иисус стал прекрасным мастером по производству ярем, и работал одинаково хорошо и с кожей, и с холстом. Он также быстро становился отличным плотником и специалистом по тонкой столярной работе. Этим летом Иисус часто поднимался на вершину горы к северо\hyp{}западу от Назарета для молитвы и размышлений. Постепенно он все больше осознавал природу своего пришествия на землю.
\vs p126 1:2 Всего лишь за сто лет до того на этой горе поклонялись Ваалу, теперь же на ней была гробница Симеона, считавшегося святым человеком Израиля. С вершины этой горы Симеона Иисусу открывалась панорама Назарета и прилегающих земель. Глядя на Мегиддо, он вспоминал историю одержанной там первой великой победы египетской армии в Азии, а также историю того, как позже другая подобная армия нанесла здесь поражение иудейскому царю Иосии. Невдалеке он мог видеть Таанах, где Дебора и Барак победили Сисару. Дальше вырисовывались холмы Дофана, где, как его учили, Иосиф был продан своими братьями в египетское рабство. Переводя взгляд на Ебал и Геризим, Иисус воссоздавал в памяти предания об Аврааме, Иакове и Авимелехе. Так он вспоминал и обдумывал исторические и легендарные события из жизни народа своего отца Иосифа.
\vs p126 1:3 Он продолжал повышать свое образование, читая под руководством синагогальных учителей, а также продолжал домашнее обучение своих братьев и сестер по мере того, как они достигали подобающего для этого возраста.
\vs p126 1:4 Иосиф устроил так, что с начала этого года он мог откладывать доход со своей собственности в Назарете и Капернауме для оплаты предстоящего длительного обучения Иисуса в Иерусалиме, куда он, как планировалось, должен был отправиться в августе следующего года, когда ему исполнится пятнадцать лет.
\vs p126 1:5 К началу этого года Иосифа и Марию нередко посещали сомнения в судьбе их первенца. Несомненно, он одаренный и обаятельный ребенок, но так трудно понять его, так тяжело постичь, и к тому же ничего чудесного или необычайного до сих пор не произошло. Много раз его исполненная гордости мать, затаив дыхание, ожидала, что вот\hyp{}вот ее сын совершит нечто сверхъестественное или чудесное, но неизменно бывала жестоко разочарована. Все это обескураживало, приводило в уныние. Верующие люди в те времена были искренне убеждены, что пророки и другие посланные Богом люди всегда творят чудеса и совершают необыкновенные вещи, чтобы продемонстрировать свое предназначение и утвердить свой божественный авторитет. Иисус не совершал ничего подобного, поэтому смятение родителей, размышляющих о его будущем, росло.
\vs p126 1:6 Многое в доме назаретской семьи показывало, что экономическое положение ее улучшилось, особенно увеличившееся число гладких белых досок, использовавшихся для письма углем. Кроме того, Иисусу было позволено возобновить уроки музыки: он очень любил играть на арфе.
\vs p126 1:7 \pc Воистину, можно сказать, что весь этот год Иисус «преуспевал в любви у Бога и человеков». Перспективы семьи казались хорошими, будущее --- светлым.
\usection{2. Смерть Иосифа}
\vs p126 2:1 Все шло хорошо до рокового дня вторника 25 сентября, когда гонец из Сефориса принес трагическое известие о том, что Иосиф разбился, упав с подъемника во время работы в резиденции губернатора. По пути к дому Иосифа гонец остановился в мастерской, рассказав Иисусу о несчастном случае с отцом, и они вместе пошли домой, чтобы сообщить печальное известие Марии. Иисус жаждал немедленно отправиться к отцу, но Мария не хотела об этом и слышать: она сама должна поспешить, чтобы быть подле мужа. Она решила, что десятилетний Иаков будет сопровождать ее в Сефорис, а Иисус останется дома с младшими детьми до ее возвращения, так как она не знала, насколько тяжело состояние Иосифа. Но Иосиф умер от травм еще до ее прибытия. Они привезли его в Назарет и на следующий день похоронили рядом с его предками.
\vs p126 2:2 \pc Именно в то время, когда перспективы были хороши, а будущее выглядело блестящим, жестокий удар обрушился на главу назаретской семьи, дела этого дома пришли в расстройство и рухнули все планы, связанные с Иисусом и его дальнейшим образованием. У юного плотника, которому едва исполнилось четырнадцать лет, возникло осознание того, что не только божественную природу в плотском образе должен он явить на земле, выполнив тем самым миссию, возложенную небесным Отцом, но и его юной человеческой природе необходимо взять на себя ответственность заботиться об овдовевшей матери и семерых братьях и сестрах, не считая младенца, который еще только должен был родиться. Этот подросток из Назарета остался единственной опорой и утешением для своей внезапно понесшей утрату семьи. Так дано было совершиться естественному порядку событий на Урантии, который вынудил судьбоносного молодого человека столь рано взять на себя эту тяжелую ответственность, которая и дала ему подлинное образование и дисциплину, ответственность, связанную с тем, что он стал главой человеческой семьи, стал отцом для своих собственных братьев и сестер, защитником и кормильцем своей матери, хранителем отцовского дома --- единственного, который ему суждено было знать в этом мире.
\vs p126 2:3 Иисус с готовностью принял эту ответственность, столь внезапно обрушившуюся на него, и добросовестно выполнил ее до самого конца. По крайней мере, одну большую проблему и предполагаемую трудность в его жизни она трагически разрешила: от него уже не ожидали, что он отправится учиться у раввинов в Иерусалиме. Таким образом, истинно во веки веков --- Иисус «не сидел у ног ни одного человека». Он был всегда готов учиться и у наискромнейшего из малых детей, но его авторитет как учителя истины не проистекал из человеческих источников.
\vs p126 2:4 По\hyp{}прежнему Иисус ничего не знал о том, что Гавриил посетил его мать накануне его рождения; в начале своего публичного служения, в день крещения он узнал об этом от Иоанна.
\vs p126 2:5 \pc С годами этот юный плотник из Назарета все более стал оценивать любые общественные институты и религиозные практики одной\hyp{}единственной мерой: что дает это человеческой душе? Приближает ли Бога к человеку? Приближает ли человека к Богу? Нельзя сказать, чтобы он полностью отвергал такие сферы жизни, как отдых и социальное общение, однако постепенно все больше и больше своего времени и энергии Иисус отдавал двум вещам: заботе о семье и подготовке к исполнению на земле небесной воли своего Отца.
\vs p126 2:6 \pc В этом году для соседей стало обыкновением заглядывать зимними вечерами к Иисусу, чтобы внимать его игре на арфе, послушать его истории (потому что юноша был прекрасным рассказчиком) и услышать его чтение из греческого Писания.
\vs p126 2:7 Экономические дела семьи продолжали идти сравнительно гладко благодаря тому, что к моменту смерти Иосифа в семье оставалась значительная сумма денег. Иисус рано продемонстрировал, что способен принимать верные решения и обладает финансовой прозорливостью. Он был великодушным и в то же время экономным, одновременно бережливым и щедрым. Он показал себя мудрым и умелым управляющим имуществом своего отца.
\vs p126 2:8 Несмотря на все старания Иисуса и назаретских соседей внести в дом оживление, Мария и даже дети оставались подавленными печалью. Иосифа не стало. Иосиф был необыкновенным мужем и отцом, и им всем не хватало его. И трагичнее всего казалась мысль, что они не смогли поговорить с ним перед смертью и получить его прощальное благословение.
\usection{3. Пятнадцатый год (9 г. н.э.)}
\vs p126 3:1 К середине этого пятнадцатого года --- мы считаем время по европейскому календарю двадцатого века, а не по еврейскому летоисчислению --- Иисус твердо взял в свои руки управление хозяйством семьи. Еще до конца этого года от их сбережений практически ничего не осталось, и они оказались перед необходимостью продать один из назаретских домов, которым Иосиф владел совместно с соседом Иаковом.
\vs p126 3:2 В среду 17 апреля 9 г. н.э., вечером, родилась Руфь, младшая в семье. Иисус сделал все, что было в его силах, чтобы заменить отца, поддерживая и помогая своей матери во время этих трудных и необыкновенно печальных родов. Ни один отец не смог бы любить и пестовать свою дочь с большей любовью и преданностью, чем на протяжении почти двух десятилетий (до начала своего публичного служения) Иисус заботился о маленькой Руфи. И столь же хорошим отцом он был для остальных членов своей семьи.
\vs p126 3:3 \pc В течение этого года Иисус впервые составил молитву, которой впоследствии научил своих апостолов и которая для многих стала известна как «Отче наш». Можно сказать, что это было развитие семейной религиозной практики, содержавшей прежде множество восхвалений и несколько формальных молитв. После смерти своего отца Иисус пытался научить старших детей индивидуально выражать себя в молитве, --- тому, чем так наслаждался сам, --- но они не смогли уловить его мысль и всякий раз возвращались к заученным формам молитв. Пытаясь вдохновить старших из своих братьев и сестер на индивидуальную молитву, Иисус пробовал привести их к этому при помощи наводящих фраз, и вскоре без всякого намерения с его стороны сложилось так, что все они стали использовать форму молитвы, составленную преимущественно из этих наводящих строк, которым он их научил.
\vs p126 3:4 В конце концов Иисус отказался от мысли обучить всех членов семьи спонтанной молитве, и однажды октябрьским вечером он сел за низкий каменный стол около маленькой приземистой лампы и на гладкой кедровой доске размером около восемнадцати квадратных дюймов куском угля записал молитву, которая с тех пор стала традиционным семейным молением.
\vs p126 3:5 \pc В этом году Иисуса сильно беспокоило смятение в мыслях. Семейные обязанности пока совершенно не позволяли ему подумать о том, чтобы немедленно взяться за осуществление какого\hyp{}либо плана, следуя указанию, полученному в Иерусалиме, о том, что ему «должно быть в том, что принадлежит Отцу». Иисус правильно считал, что забота о семье его земного отца должна быть первейшим его долгом, обеспечение семьи должно стать основной обязанностью.
\vs p126 3:6 \pc В течение этого года Иисус нашел место в так называемой книге Еноха, которое побудило его впоследствии принять имя «Сын Человеческий» в обозначение миссии своего пришествия на Урантию. Он всесторонне обдумывал идею еврейского Мессии и пришел к твердому убеждению, что ему не суждено быть этим Мессией. Он жаждал помочь народу своего отца, но он никогда не станет вести еврейские армии на разгром чужеземных властителей Палестины. Он знал, что никогда не будет сидеть на троне Давида в Иерусалиме. Он не верил также в то, что его миссия --- стать духовным освободителем или моральным учителем исключительно для еврейского народа. Поэтому его жизненная миссия абсолютно не соответствовала этим страстным упованиям и предполагаемым мессианским пророчествам Иудейского Писания, --- по крайней мере, в том смысле, в каком евреи понимали эти предсказания пророков. Он был убежден также, что не явится тем Сыном Человеческим, которого описал пророк Даниил.
\vs p126 3:7 Но как назовет он себя, когда придет его время выступить в качестве мирового учителя? Какими словами заявит о своей миссии? Каким именем назовут его люди, которые уверуют в его учение?
\vs p126 3:8 \pc Постоянно в мыслях своих возвращаясь к этим проблемам, он обнаружил в синагогальной библиотеке Назарета среди апокалиптических книг, которые изучал тогда, рукопись, озаглавленную «Книга Еноха», и хотя он был уверен, что ее автором не является древний Енох, однако очень заинтересовался ее содержанием, читал и перечитывал много раз. В ней было одно место, которое произвело на него особое впечатление, место, в котором появлялось это выражение «Сын Человеческий». Автор этой так называемой «Книги Еноха» снова и снова говорил о Сыне Человеческом, описывая деяния, которые должен он совершить на земле, объясняя, что до того, как сойти на землю ради спасения людей, этот Сын Человеческий прошел через чертоги небесной славы со своим Отцом, Отцом всех; и что он отказался от великолепия и славы, чтобы принести весть о спасении страждущим смертным. Читая эти строки (хорошо понимая, что в восточном мистицизме, значительно повлиявшем на это учение, таится много ложного), Иисус, однако, почувствовал в сердце и осознал умом, что ни одно из всех мессианских пророчеств Иудейского Писания и ни одна из всех теорий об освободителе евреев не были так близки к истине, как история о Сыне Человеческом, вставленная в эту лишь отчасти признанную «Книгу Еноха». Тогда и там он решил принять имя «Сын Человеческий». Это он и сделал впоследствии, вступая в свое публичное служение. Иисус обладал способностью безошибочно распознавать истину и никогда не колебался избрать ее, из какого бы источника истина ни исходила.
\vs p126 3:9 К этому времени он окончательно решил многие вопросы, связанные с предстоящей ему работой для мира, но ничего не говорил об этом матери, которая все еще упорно мечтала о нем как о будущем еврейском Мессии.
\vs p126 3:10 И здесь к Иисусу пришло великое смятение его юности. Он решил, что суть его миссии на земле --- «быть в том, что принадлежит Отцу», дать откровение любящей природы Отца всему человечеству, и начал переосмысливать многие утверждения Писания, касающиеся прихода национального освободителя, еврейского учителя или царя. К какому событию относятся эти пророчества? Сам он --- еврей или нет? Происходит ли он из дома Давида или нет? Его мать утверждала, что да; его отец считал, что нет. Сам он пришел к выводу, что нет. Но значит ли это, что пророки ошибались относительно природы и задач миссии Мессии?
\vs p126 3:11 В конце концов, быть может, его мать права? В большинстве случаев, когда в прошлом возникали разногласия, она оказывалась права. Если он --- новый учитель и не Мессия, как он распознает еврейского Мессию, если тот появится в Иерусалиме во время его земной миссии, и далее, каковы должны быть его отношения с этим еврейским Мессией? И после того, как он начнет исполнять свою жизненную миссию, какими должны быть его отношения с семьей? С еврейским обществом и религией? с Римской империей? с неевреями и их религиями? Каждую из этих глобальных проблем молодой галилеянин тщательно перебирал в уме и серьезно обдумывал, продолжая работать за плотницким верстаком, тяжелым трудом зарабатывая на хлеб для себя, своей матери и еще восьми голодных ртов.
\vs p126 3:12 \pc К концу этого года Мария увидела, что семейные запасы иссякают. Она поручила Иакову торговлю голубями. Тогда же они купили вторую корову, и с помощью Мириам начали продавать молоко своим назаретским соседям.
\vs p126 3:13 \pc Периоды глубокой задумчивости Иисуса, его частые уходы на вершину горы для молитвы и многие странные мысли, которые Иисус высказывал время от времени, сильно тревожили его мать. Иногда она думала, что юноша не в себе, но она умеряла страхи, вспоминая о том, что, в конце концов, Иисус --- обетованное дитя и должен быть не таким, как другие подростки.
\vs p126 3:14 А Иисус тем временем учился не высказывать свои идеи, не открывать все свои мысли миру, даже собственной матери. Начиная с этого года, он все меньше делился тем, что было у него на уме; иными словами, все меньше говорил о вещах, недоступных среднему человеку, из\hyp{}за которых его могли считать необычным или отличным от ординарных людей. Во всех проявлениях он стал совершенно обычным и обыкновенным, однако страстно мечтал о ком\hyp{}то, кто мог бы понять его проблемы. Он жаждал иметь надежного и достойного доверия друга, но его проблемы были слишком сложны для понимания окружавших его людей. Уникальность необычной ситуации вынуждала его нести эту ношу в одиночестве.
\usection{4. Первая проповедь в синагоге}
\vs p126 4:1 После своего пятнадцатого дня рождения Иисус мог официально занимать кафедру в синагоге по Субботам. И прежде его много раз просили читать Писание в те дни, когда не было проповедников, но теперь настал день, когда он по закону имел право проводить службу. Поэтому в первую же Субботу после пятнадцатого дня рождения Иисуса хазан поручил ему вести утреннюю службу в синагоге. И когда в этот день все верующие Назарета собрались в синагоге, молодой человек, подготовивший определенные места из Писания, встал и начал читать:
\vs p126 4:2 \pc «Дух Господа Бога на мне, ибо Господь помазал меня благовествовать кротким, послал меня исцелять сокрушенных сердцем, проповедовать пленным освобождение и духовным узникам открывать темницы, проповедовать лето господне благоприятное и судный день Бога нашего, утешить всех скорбящих, вместо пепла дать украшение, вместо плача елей радости, вместо унылого духа хвалебную песнь, чтобы они могли и назвать их древами праведности, насажденными Господом.
\vs p126 4:3 «Ищите добра, а не зла, чтобы вам остаться в живых, и тогда Господь Бог Саваоф будет с Вами. Возненавидьте зло и возлюбите добро, и восстановите у ворот правосудие; может быть, Господь Бог сил, будет милостивым к остатку Иосифова.
\vs p126 4:4 «Омойтесь, очиститесь; удалите злые деяния ваши от очей моих; перестаньте делать зло; научитесь делать добро; ищите правды; спасайте угнетенного; защищайте сироту; вступайтесь за вдову.
\vs p126 4:5 «С чем предстать мне пред Господом, преклониться пред Богом всей земли? Предстать ли пред ним со всесожжениями, с тельцами однолетними? Но можно ли угодить Господу десятью тысячами овец, реками елея? Разве дам ему первенца моего за преступление мое и плод чрева моего --- за грех души моей? О, человек! сказано тебе, что добро и чего требует от тебя Господь: действовать справедливо, любить дела милосердия и смиренно ходить пред Богом твоим.
\vs p126 4:6 «Кому же тогда вы уподобите Господа и восседающего над кругом земным? Поднимите глаза ваши и посмотрите на того, кто сотворил все миры эти? Кто выводит воинство их счетом, и всех их называет именами их. И он все творит могуществом великой силы своей, ибо он силен во власти своей и ничто у него не выбывает. Он дает могущество слабым и прибавляет силы усталому. Не бойся, ибо я с тобою; не смущайся, ибо я Бог твой; я укреплю тебя, и я помогу тебе, и поддержу тебя правою десницею праведности моей. Ибо я Господь, Бог твой; И я буду держать тебя за правую руку твою, говоря тебе: Не бойся, помни, что я помогаю тебе.
\vs p126 4:7 А ты мой свидетель, говорит Господь, ты и раб мой, которого я избрал, чтобы ты знал и верил мне, и разумел, что это Я: прежде меня не было Бога и после меня не будет. Я, Я, Господь, и нет Спасителя кроме меня».
\vs p126 4:8 \pc Закончив чтение, он сел на свое место. Люди расходились по домам, размышляя о словах, столь прекрасно им прочитанных. Никогда прежде жители его города не видели Иисуса таким величаво\hyp{}торжественным, никогда еще не звучал его голос для них так убежденно и искренне; никогда раньше не замечали они в нем такой мужественности и решительности, такой властности.
\vs p126 4:9 В тот субботний день после полудня Иисус вместе с Иаковом поднялся на вершину Назаретского холма. Когда они вернулись домой, он записал по\hyp{}гречески десять заповедей углем на двух гладких досках. Позже Марфа раскрасила и разрисовала доски; долго потом висели они на стене над маленьким верстаком Иакова.
\usection{5. Денежные проблемы}
\vs p126 5:1 Постепенно Иисус и его семья вернулись к скромной жизни прежних лет. Их одежда и даже пища стали более простыми. У них было вдоволь молока, масла и сыра. В сезон они жили тем, что давал их огород; но с каждым месяцем приходилось все больше экономить. Их завтраки были очень скромны: лучшую пищу оставляли для вечерней трапезы. Впрочем, среди евреев отсутствие богатства не приводило к социальной приниженности.
\vs p126 5:2 Этот юноша знал уже почти все о том, как живут люди его времени. И то, как хорошо он понимал жизнь людей дома, в поле и в мастерской, показывают его последующие учения, в которых столь часто раскрывалось, что он непосредственно знаком со всеми сторонами человеческого опыта.
\vs p126 5:3 Хазан назаретской синагоги продолжал твердо верить, что Иисусу суждено стать великим учителем, возможно, преемником знаменитого Гамалиила в Иерусалиме.
\vs p126 5:4 \pc Но по\hyp{}видимому, все планы Иисуса относительно карьеры были нарушены. Судя по состоянию дел, будущее не представлялось светлым. Но он не унывал, он не терял уверенности в себе. Он жил изо дня в день, добросовестно совершая будничные дела и выполняя \bibemph{насущные} обязанности человека его положения. Жизнь Иисуса --- поддержка на все времена для всех разочарованных идеалистов.
\vs p126 5:5 Дневной заработок простого плотника становился все меньше и меньше. К концу этого года Иисус, трудясь от зари до зари, мог заработать лишь ничтожную сумму, соответствующую примерно 25 центам в день. К следующему году семье стало трудно платить налоги, не говоря уже о взносах в синагогу и храмовой подати, равной половине шекеля. В течение этого года сборщик налогов, пытаясь получить с Иисуса еще хоть какие\hyp{}то деньги, даже угрожал забрать его арфу.
\vs p126 5:6 Опасаясь, что сборщики налогов обнаружат и конфискуют греческий экземпляр Писания, Иисус на свой пятнадцатый день рождения подарил его Назаретской синагогальной библиотеке как приношение Господу в честь достижения зрелости.
\vs p126 5:7 \pc Сильнейшее потрясение своего пятнадцатого года испытал Иисус в Сефорисе, куда он отправился, чтобы получить ответ Ирода на поданную ему апелляцию по поводу спора относительно денег, причитавшихся Иосифу к моменту его смерти от несчастного случая. Иисус и Мария надеялись, что эта сумма будет довольно значительной, но казначей в Сефорисе предложил им ничтожно мало. Поэтому братья Иосифа подали апелляцию самому Ироду, и вот теперь Иисус, стоя во дворце, выслушал решение Ирода, согласно которому отцу во время его смерти не причиталось ничего. Из\hyp{}за этого несправедливого решения Иисус с тех пор не доверял Ироду Антипе. И неудивительно, что однажды он сказал о нем «этот лис».
\vs p126 5:8 Неотрывная работа за верстаком в течение всего этого и последующих лет лишила Иисуса возможности встречаться с людьми из караванов. Принадлежавшую семье мастерскую на караванной стоянке взял теперь дядя, а Иисус постоянно работал в домашней мастерской, где он мог в любой момент помочь Марии с семейными делами. Примерно в это время он начал посылать Иакова на караванную стоянку, чтобы тот приносил вести о происходящем в мире и, таким образом самому быть в курсе текущих событий.
\vs p126 5:9 На пути к зрелости Иисус испытал все переживания и сомнения, которые испытывают и будут испытывать обычные молодые люди предшествующих и последующих веков. Однако суровый опыт жизни, полной забот о поддержании семьи, надежно защитил его от праздных размышлений и мистических увлечений.
\vs p126 5:10 \pc В этом году Иисус взял в аренду большой кусок земли к северу от дома, на котором был устроен семейный огород. Каждый из старших детей получил свой личный огород, и между ними развернулось азартное сельскохозяйственное соперничество. Их старший брат проводил некоторое время с ними в огороде каждый день в сезон посадки овощей и ухода за ними. Работая со своими младшими братьями и сестрами в огороде, Иисус много раз испытывал желание переселиться в сельскую местность, где можно было бы вести свободную и независимую жизнь на природе. Но им не суждено было вырасти в сельской местности; Иисус, как вполне практичный юноша, хотя и идеалист, разумно и энергично решал любую встающую перед ним проблему и делал все, что в его силах, чтобы он и его семья приспособились к реальной ситуации и могли жить в условиях, более всего соответствующих их индивидуальным и коллективным устремлениям.
\vs p126 5:11 В какой\hyp{}то момент у него возникла некоторая надежда собрать средства, достаточные для покупки маленькой фермы, --- при условии получения денег, заработанных отцом на строительстве дворца Ирода. Тогда Иисус серьезно обдумывал план переезда семьи в сельскую местность. Но после того, как Ирод отказался заплатить какие\hyp{}либо деньги, причитавшиеся Иосифу, с мыслью о сельском доме пришлось расстаться. Однако и сейчас они могли позволить себе многие радости сельской жизни, ведь у них теперь были три коровы, четыре овцы, выводок цыплят, осел и собака, не говоря уже о голубях. Даже у младших детей были свои постоянные обязанности в хорошо отлаженной системе ведения хозяйства и домашнего уклада этой назаретской семьи.
\vs p126 5:12 \pc К концу пятнадцатого года своей жизни Иисус завершил трудный и опасный период человеческой жизни, время перехода от полной безмятежности детства к осознанию приближающейся зрелости, которая несла ему повышенную ответственность и возможность обрести больший опыт, необходимый для формирования благородного характера. Период его умственного и физического развития завершился, начался подлинный жизненный путь этого юноши из Назарета.
