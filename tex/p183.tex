\upaper{183}{Предательство и арест Иисуса}
\author{Комиссия срединников}
\vs p183 0:1 Разбудив в конце концов Петра, Иакова и Иоанна, Иисус предложил им пойти в свои шатры и постараться уснуть, чтобы отдохнуть перед трудным завтрашним днем. Но апостолы к этому времени уже бодрствовали; короткий сон их освежил, а кроме того, их растревожило прибытие на место действия двух взволнованных вестников, которые спросили Давида Зеведеева и быстро пошли его искать, когда Петр объявил им, где тот несет караул.
\vs p183 0:2 Хотя восемь из апостолов крепко спали, греки, расположившиеся лагерем рядом с ними, были гораздо более опасливые, так что выставили часового, чтобы тот в случае возникновения опасности поднял тревогу. Когда эти два вестника вбежали в лагерь, греческий часовой стал будить всех своих соотечественников, которые выскочили из шатров совершенно одетыми и вооруженными. Теперь, кроме восьми апостолов, проснулся весь лагерь. Петр хотел позвать своих соратников, но Иисус недвусмысленно запретил ему это. Учитель мягко убеждал их идти в свои шатры, но они не хотели следовать его совету.
\vs p183 0:3 Не сумев убедить своих последователей разойтись, Учитель оставил их и пошел к масличному прессу, находившемуся рядом со входом в Гефсиманский сад. Хотя три апостола, греки и другие обитатели лагеря не решались сразу последовать за ним, Иоанн Марк побежал через оливковую рощу и спрятался в небольшом сарае рядом с масличным прессом. Иисус ушел из лагеря и от своих друзей, чтобы пришедшие его арестовать могли сделать это, не потревожив апостолов. Учитель опасался, что его апостолы, проснувшись, будут присутствовать во время его ареста, и предательство Иуды озлобит их настолько, что они окажут сопротивление солдатам и их возьмут под стражу вместе с ним. Он боялся, что если всех их арестуют, они вместе с ним и погибнут.
\vs p183 0:4 Хотя Иисус знал, что план предать его смерти возник в советах еврейских правителей, он также понимал, что все подобные низкие козни полностью одобряют Люцифер, Сатана и Калигастия. Знал Иисус и о том, что эти бунтовщики будут также довольны, увидев и его гибель, и гибель всех апостолов.
\vs p183 0:5 Иисус сидел на масличном прессе один и ожидал прихода предателя, и в это время его видели лишь Иоанн Марк и бесчисленное воинство небесных наблюдателей.
\usection{1. Воля Отца}
\vs p183 1:1 Существует большая опасность неверно толковать смысл многочисленных высказываний и многих событий, связанных с последними днями жизни Учителя во плоти. Жестокое обращение с Иисусом невежественных слуг и бессердечных солдат, бесчестное поведение его судей и бесчувственное отношение к нему так называемых религиозных лидеров не следует путать с тем обстоятельством, что, смиренно покорившись всем страданиям и унижениям, Иисус воистину исполнял волю Райского Отца. Воля же Отца действительно и поистине была такова, что его Сын должен полностью, от рождения до смерти, испить чашу смертного опыта, однако Отец Небесный ни коим образом не побуждал к варварскому поведению тех якобы цивилизованных людей, которые сначала так жестоко мучили Учителя, а затем подвергли столь же ужасным оскорблениям его несопротивляющуюся личность. Нечеловеческие и жестокие страдания, которые Иисус был вынужден претерпеть в последние часы своей смертной жизни, никоим образом не были частью божественной воли Отца, которую человеческая природа Иисуса обещала исполнить столь триумфально, окончательно подчинив человека Богу, о чем и свидетельствовала троекратная молитва, произнесенная им в саду, пока его физически утомленные апостолы спали крепким сном.
\vs p183 1:2 Отец Небесный желал, чтобы совершивший пришествие Сын закончил свой земной путь \bibemph{естественно,} так же, как все смертные должны кончить свои жизни на земле и во плоти. Обыкновенные мужчины и женщины не могут рассчитывать на то, что их последние часы на земле и наступающее за ними событие смерти облегчит особая диспенсация. Поэтому Иисус решил оставить свою жизнь во плоти так, чтобы это соответствовало естественному ходу событий, и упорно отказывался освободить себя из жестких тисков трагического стечения бесчеловечных событий, которые со страшной неотвратимостью продолжали нести его к неслыханному унижению и позорной смерти. Каждый элемент проявления этого поразительного сгустка ненависти и невиданной жестокости был делом рук злонамеренных людей и порочных смертных. Бог на небе не желал этого, не принуждали к этому и архивраги Иисуса, хотя они многое сделали для того, чтобы безрассудные и нечестивые смертные таким образом отвергли совершившего пришествие Сына. Даже отец греха, и тот отвратил лицо свое от мучительно\hyp{}ужасной картины распятия.
\usection{2. Иуда в городе}
\vs p183 2:1 После того, как Иуда так стремительно вышел из\hyp{}за стола во время вкушения Тайной Вечери, он тотчас отправился в дом своего двоюродного брата, и они вдвоем пошли прямо к командиру храмовых стражников. Иуда попросил командира собрать стражников, сообщив ему, что готов вести их к Иисусу. Поскольку Иуда появился у командира немного раньше, чем его ожидали, произошла небольшая заминка перед тем, как они отправились к дому Марка, где, как рассчитывал Иуда, Иисус еще беседует с апостолами. Но Учитель и одиннадцать апостолов покинули дом Илии Марка примерно за пятнадцать минут до прихода предателя и стражников. К моменту, когда идущие арестовать Иисуса пришли к дому Марка, Иисус и одиннадцать апостолов были уже далеко за стенами города и шли в лагерь на Масличной горе.
\vs p183 2:2 Иуда был сильно смущен тем, что ему не удалось застать Иисуса в доме Марка, в окружении одиннадцати человек, из которых лишь двое были вооружены и могли оказать сопротивление. Иуда знал, что днем, когда они покинули лагерь, только Симон Петр и Симон Зилот препоясались и были с мечами; Иуда надеялся схватить Иисуса, когда в городе было тихо и сопротивление маловероятно. Предатель боялся, что, если он станет ждать, когда Иисус и апостолы вернутся в лагерь, то столкнется уже с шестьюдесятью преданными учениками, к тому же он знал, что у Симона Зилота было достаточно оружия. Размышляя о том, как одиннадцать верных апостолов возненавидят его, Иуда все больше нервничал и боялся, что все они постараются его убить. Иуда в душе своей был не только предателем, но и настоящим трусом.
\vs p183 2:3 Когда в комнате наверху Иисуса найти не удалось, Иуда попросил капитана охраны вернуться в храм. К этому времени правители уже начали собираться в доме первосвященника, готовясь учинить расправу над Иисусом, поскольку согласно их сделке с предателем арест Иисуса должен был произойти до полуночи того дня. Иуда объяснил своим сообщникам, что они не застали Иисуса в доме Марка и, чтобы арестовать его, придется идти в Гефсиманию. Затем предатель сообщил, что в одном лагере с Иисусом находятся больше шестидесяти преданных ему последователей и что все они хорошо вооружены. Правители евреев напомнили Иуде, что Иисус всегда проповедовал непротивление, но Иуда возразил, что они не могут рассчитывать на то, что все последователи Иисуса следуют подобному учению. Он действительно боялся за себя и потому решился просить, чтобы его сопровождало сорок вооруженных солдат. Поскольку у еврейских властей не было в подчинении такого числа вооруженных людей, они сразу отправились в крепость Антонию и попросили римского командира выделить им такую охрану; однако тот, узнав, что они собираются арестовать Иисуса, тотчас отказался удовлетворить их просьбу и отослал их к высшему начальнику. Таким образом, более часа ушло на то, что они ходили от одного начальника к другому, пока наконец не были вынуждены пойти к самому Пилату, чтобы получить разрешение привлечь вооруженных римских стражников. Когда же они пришли в дом Пилата, было уже поздно, тот вместе с женой удалился в свои покои. Пилат не решался вмешаться в это дело, тем более что его жена просила не отвечать на просьбу согласием. Однако поскольку к нему пришел председатель еврейского синедриона и обратился с личной просьбой об этой услуге, правитель решил, что разумно будет его прошение удовлетворить, полагая, что позднее сумеет исправить любую ошибку, которую они задумали совершить.
\vs p183 2:4 Поэтому, когда Иуда около половины двенадцатого часа выходил из храма, его сопровождали более шестидесяти человек --- храмовые стражи, римские солдаты и любопытные слуги первосвященников и правителей.
\usection{3. Арест Учителя}
\vs p183 3:1 Когда отряд вооруженных солдат и стражников, с факелами и светильниками подошел к саду, Иуда обогнал группу, чтобы быстро указать на Иисуса, давая возможность пришедшим арестовать Учителя, легко его схватить прежде, чем товарищи Иисуса бросятся к нему на помощь. Была и другая причина, по которой Иуда решил быть впереди врагов Учителя. Он рассчитывал, что все будет выглядеть так, будто он прибыл к месту действия раньше солдат, так что у апостолов и людей, собравшихся вокруг Иисуса, он не будет ассоциироваться с вооруженными стражами, шедшими за ним следом. Иуда решил даже сделать вид, будто он спешит предупредить их о приближении идущих арестовать Иисуса, но этому плану помешало приветствие, разоблачившее предателя. Хотя Учитель говорил с Иудой доброжелательно, но он приветствовал его как предателя.
\vs p183 3:2 Как только Петр, Иаков, Иоанн и около тридцати человек, живших с ними в лагере, увидели вооруженный отряд с факелами, показавшийся на краю холма, они поняли, что эти солдаты идут арестовать Иисуса, и все вместе бросились к масличному прессу, где Учитель сидел один в свете луны. В то время, как отряд солдат приблизился к Иисусу с одной стороны, трое апостолов и их товарищи подбежали с другой. Когда же Иуда выступил вперед поприветствовать Учителя, обе группы замерли; между ними стоял Иисус, и Иуда уже был готов запечатлеть на его челе предательский поцелуй.
\vs p183 3:3 Предатель надеялся, что, приведя стражей в Гефсиманию, он сможет просто показать Иисуса солдатам, либо, в крайнем случае, исполнит обещание поприветствовать его поцелуем, а затем быстро удалится. Иуда очень боялся, что при этом будут присутствовать все апостолы, которые все вместе нападут на него, мстя за то, что он посмел предать их возлюбленного учителя. Когда же Учитель поприветствовал Иуду как предателя, тот настолько смутился, что даже не попытался бежать.
\vs p183 3:4 Иисус предпринял последнее усилие спасти Иуду от предательства, ибо прежде, чем изменник смог подойти к нему, он сделал шаг в сторону и, обращаясь к крайнему слева воину, командиру римлян, спросил: «Кого ищете?» Тот ответил: «Иисуса Назорея». Тогда Иисус тотчас встал перед офицером в спокойном величии Бога всего творения и сказал: «Это я». Многие из вооруженного отряда слышали, как Иисус проповедовал в храме, другие знали о его великих делах, и, увидев, как он столь смело назвал себя, стоявшие в первых рядах внезапно отступили назад. Изумление, вызванное спокойствием и величием, с которыми Иисус назвал себя, лишило их самообладания. Поэтому Иуде не требовалось продолжать исполнение своего плана предательства. Учитель открыто явил себя своим врагам, и те могли схватить его без помощи Иуды. Однако предатель должен был что\hyp{}нибудь сделать, чтобы оправдать свое присутствие среди этого вооруженного отряда, а кроме того, он хотел сыграть свою роль в спектакле предательской сделки с правителями евреев, дабы получить немалую награду и почести, которые, как он считал, посыплются на него как плата за его обещание предать Иисуса в их руки.
\vs p183 3:5 Когда стражники оправились от первоначального потрясения, вызванного видом Иисуса и звуком его необычного голоса, и когда их окружили апостолы и ученики, Иуда подошел к Иисусу и, запечатлев поцелуй на его челе, сказал: «Приветствую тебя, Господин и Учитель». И когда Иуда с этими словами обнял своего Учителя, Иисус сказал: «Не довольно ли, друг! Целованием ли предаешь Сына Человеческого?»
\vs p183 3:6 Апостолы и ученики были буквально ошеломлены увиденным. Какое\hyp{}то мгновение никто не двигался. Затем Иисус, освободившись из предательских объятий Иуды, подошел к стражникам и солдатам и опять спросил: «Кого ищете?» И вновь командир сказал: «Иисуса из Назорея». Иисус же снова ответил: «Уже сказал я вам, что это я. Стало быть, если меня ищете, отпусти сих остальных, пусть идут своим путем. Я готов идти с вами».
\vs p183 3:7 Иисус был готов вернуться в Иерусалим со стражниками, и командир в общем\hyp{}то хотел позволить трем апостолам и их товарищам уйти с миром. Однако прежде, чем они смогли удалиться, в то время, как Иисус стоял, ожидая распоряжений командира, некий Малх, сириец\hyp{}телохранитель первосвященника, подошел к Иисусу и приготовился связать ему руки за спиной, хотя римлянин и не приказывал этого делать. Увидев, что с Учителем непочтительно обращаются, Петр и его товарищи не могли больше себя сдерживать. Петр выхватил меч и вместе с другими бросился вперед, чтобы поразить Малха. Однако не успели солдаты прийти на помощь слуге первосвященника, как Иисус движением руки, запретил это Петру, и строгим голосом сказал: «Убери меч, Петр. Взявший меч мечом и погибнет. Неужели не понимаешь, что воля Отца такова, чтобы я пил чашу сию? И разве не знаешь, что я даже сейчас могу приказать более, нежели двенадцати легионам ангелов и их сподвижников, и те избавят меня от рук этих нескольких человек».
\vs p183 3:8 Хотя, таким образом, Иисус успешно пресек физическое сопротивление своих последователей, этого было достаточно, чтобы вызвать страх командира стражи, который теперь с помощью своих воинов крепко схватил Иисуса и быстро связал его. Когда же Иисусу связывали руки крепкими веревками, он сказал: «Почему с мечами и кольями вышли вы на меня, как будто на разбойника? Каждый день бывал я с вами в храме, публично уча народ, и вы не поднимали на меня рук».
\vs p183 3:9 Когда Иисуса связали, командир, опасаясь, что последователи Учителя могут попытаться его освободить, распорядился схватить и их; но солдаты замешкались, поскольку, услышав приказ командира арестовать их, последователи Иисуса поспешно бежали в лощину. Иоанн же Марк все это время оставался в находившемся рядом сарае один. Когда стражи вместе с Иисусом отправились обратно в Иерусалим, Иоанн Марк попытался незаметно выбраться из сарая, чтобы догнать убегающих апостолов и учеников; однако, как только он вышел наружу, один из последних возвращавшихся солдат, тех, что преследовали учеников, проходя рядом и увидев этого молодого человека в полотняной одежде, погнался за ним и чуть было его не схватил. Фактически солдат догнал Иоанна, даже смог схватить его за одежду, но юноша ее сбросил и убежал нагим, оставив солдата с одеждой в руках. Иоанн Марк поспешно бросился к Давиду Зеведееву, который находился у верхней тропы. Когда он рассказал о случившемся Давиду, оба поспешили вернуться в лагерь к шатрам спящих апостолов и сообщили всем восьмерым о предательстве и аресте Учителя.
\vs p183 3:10 Приблизительно в то же время, когда Иоанн будил восьмерых апостолов, стали возвращаться бежавшие в лощину, и вскоре они все вместе собрались около масличного пресса, чтобы обсудить, как им действовать. Тем временем Симон Петр и Иоанн Зеведеев, прятавшиеся среди оливковых деревьев, уже крались за толпой солдат, стражников и слуг, которые вели теперь Иисуса в Иерусалим, как вели бы отъявленного преступника. Иоанн шел за толпой на небольшом расстоянии, Петр же следовал издали. Вырвавшийся из рук солдата Иоанн Марк достал себе другую одежду, которую нашел в шатре Симона Петра и Иоанна Зеведеева. Догадавшись, что стражники собираются отвести Иисуса в дом бывшего первосвященника Анны, он пошел краем оливковых рощ и оказался на месте раньше толпы, спрятался неподалеку от ворот во дворец первосвященника.
\usection{4. Беседа у масличного пресса}
\vs p183 4:1 Иаков Зеведеев оказался разлучен с Симоном Петром и своим братом Иоанном, а потому у масличного пресса присоединился к другим апостолам и людям, жившим с ними в одном лагере, дабы обсудить, что им следует предпринять после ареста Учителя.
\vs p183 4:2 Андрей был освобожден от всякой ответственности по управлению группой своих собратьев\hyp{}апостолов; поэтому в этот величайший из всех кризисов в их жизнях он молчал. После краткого обмена мнениями Симон Зилот встал на каменную стенку масличного пресса и, обратившись со страстным призывом сохранить верность Учителю и делу царства, предложил своим собратьям\hyp{}апостолам и другим ученикам догнать толпу и освободить Иисуса. И большинство из собравшихся согласилось бы подчиниться его воинственному призыву, если бы не совет Нафанаила, который как только Симон кончил говорить, встал и обратил их внимание на неоднократно повторявшееся наставление Иисуса о непротивлении. Затем Нафанаил напомнил им, что в эту же ночь Иисус велел им сохранить свои жизни для того, чтобы пойти в мир, возвещая благую весть евангелия царства небесного. Доводы Нафанаила поддержал Иаков Зеведеев, который тут же рассказал, как Петр и другие извлекли мечи, чтобы спасти Учителя от ареста, но Иисус приказал Симону и взявшим меч вместе с ним вложить в ножны свои клинки. Матфей и Филипп также сказали свое слово, но эти разговоры не приводили ни к чему определенному до тех пор, пока Фома, обратив их внимание на то, что Иисус советовал Лазарю не подвергать себя смерти, не отметил, что они не смогут ничего сделать, дабы спасти своего Учителя, так как тот сам отказался позволить своим друзьям защищать его и упорно не хотел использовать свои божественные возможности, чтобы сокрушить своих земных врагов. Фома убедил их разойтись в разные стороны с условием, что Давид Зеведеев останется в лагере и станет для группы центром хранения и распространения информации. К половине третьего часа ночи лагерь опустел; в нем оставался только Давид, у которого в распоряжении были три или четыре вестника; остальные же были посланы добывать сведения о том, куда забрали Иисуса и что с ним собираются сделать.
\vs p183 4:3 Пятеро апостолов, Нафанаил, Матфей, Филипп и близнецы, ушли и спрятались в Беф\hyp{}Фаге и в Вифании. Фома, Андрей, Иаков и Симон Зилот скрывались в городе. Симон же Петр и Иоанн Зеведеев, следуя за толпой, пришли к дому Анны.
\vs p183 4:4 Вскоре после рассвета Симон Петр побрел назад в лагерь в Гефсимании, являя собой картину глубокого отчаяния. Давид отправил Петра в качестве вестника в дом Никодима в Иерусалиме, где в это время находился его брат Андрей.
\vs p183 4:5 До самого последнего момента распятия Иоанн Зеведеев, как и велел ему Иисус, оставался рядом с ним; он\hyp{}то и передавал час за часом вестникам Давида сведения, которые те приносили Давиду в лагерь в саду и которые затем доставлялись скрывавшимся апостолам и семье Иисуса.
\vs p183 4:6 Несомненно, пастырь поражен, и овцы рассеялись! Хотя все они смутно сознают, что Иисус предостерег их об этом, внезапное исчезновение Учителя потрясло их так сильно, что они были не в состоянии нормально мыслить.
\vs p183 4:7 Вскоре после рассвета, сразу после того, как Петр был послан к своему брату, в лагерь, почти задыхаясь и опередив остальных членов семьи Иисуса, прибыл Иуда, брат Иисуса во плоти, но узнал лишь, что Учитель уже арестован; с этими сведениями он поспешил вернуться по Иерихонской дороге к матери Иисуса и его братьям и сестрам. Через Иуду Давид Зеведеев послал родным Иисуса предложение собраться в доме Марфы и Марии в Вифании и там ожидать известий, которые его вестники будут регулярно им приносить.
\vs p183 4:8 В таком положении находились апостолы, первые ученики и земная семья Иисуса в ночь с четверга на пятницу. И все эти группы и люди поддерживали связь друг с другом с помощью службы вестников, которой продолжал руководить Давид Зеведеев из центра в гефсиманском лагере.
\usection{5. В пути к дворцу первосвященника}
\vs p183 5:1 Перед тем как схватившие Иисуса отправились вместе с ним из сада, между еврейским старшиной храмовых стражей и римским командиром отряда воинов возник спор о том, куда вести Иисуса. Старшина храмовых стражей распорядился доставить его в дом Каиафы, действующего первосвященника. Командир же римских солдат приказал отвести Иисуса во дворец Анны, бывшего первосвященника и тестя Каиафы. Так он поступил потому, что все вопросы, связанные с применением еврейских религиозных законов, римляне привыкли решать непосредственно с Анной. И приказ римлянина был исполнен; Иисуса отвели в дом Анны для предварительного допроса.
\vs p183 5:2 Иуда шел рядом с командиром и слышал весь разговор, но в споре участия не принимал, ибо ни еврейский, ни римский командир не стали бы слушать предателя --- так он был презираем.
\vs p183 5:3 Примерно в это время Иоанн Зеведеев, вспомнив наставление Учителя всегда оставаться неподалеку от него, поспешил приблизиться к Иисусу, шедшему между двух командиров. Увидев, что Иоанн подошел и идет рядом, старшина храмовых стражей приказал своему помощнику: «Схватите его и свяжите. Он один из последователей этого человека». Римский же командир, услышав это и посмотрев вокруг, увидел Иоанна и приказал ему перейти на его сторону и идти рядом, запретив кому бы то ни было прикасаться к нему. Затем римлянин сказал еврейскому старшине: «Этот человек не предатель и не трус. Я видел его в саду, он не брался за меч и не оказывал нам сопротивления. У него есть мужество выйти вперед, чтобы быть со своим Учителем, и никто не поднимет на него руку. Римский закон допускает, чтобы с любым заключенным перед судом стоял хотя бы один друг, и этому человеку никто не помешает стоять рядом со своим Учителем». Услышав это, Иуда был так постыжен и унижен, что замедлил шаг и, отстав от идущих, пришел к дворцу Анны один.
\vs p183 5:4 Этим и объясняется, почему в эту ночь и на следующий день Иоанну Зеведееву было позволено оставаться рядом с Иисусом на всем протяжении его страданий. Евреи боялись что\hyp{}либо сказать Иоанну, как\hyp{}либо досаждать ему, поскольку тот имел нечто вроде статуса римского советника, которому было поручено выступать в качестве наблюдателя за действиями еврейского религиозного суда. Привилегированное положение Иоанна упрочилось еще больше, когда, передавая Иисуса старшине храмовых стражей у ворот дворца Анны, римлянин, обращаясь к своему помощнику, приказал: «Иди вместе с этим заключенным и смотри, чтобы эти евреи не убили его без согласия Пилата. Смотри, чтобы они не расправились с ним и чтобы его другу, галилеянину, позволили стоять рядом и наблюдать за всем происходящим». Таким образом, Иоанн мог быть рядом с Иисусом до самой его смерти на кресте, а остальные десять апостолов были вынуждены продолжать скрываться. Иоанн был под защитой римлян, и евреи не смели досаждать ему до смерти Учителя.
\vs p183 5:5 Весь путь до дворца Анны Иисус молчал. С момента ареста до своего появления перед Анной Сын Человеческий не проронил ни слова.
