\upaper{125}{Иисус в Иерусалиме}
\author{Комиссия срединников}
\vs p125 0:1 Ничто в богатой событиями земной жизни Иисуса не было столь волнующим, таким по\hyp{}человечески трогательным, как это первое запомнившееся ему посещение Иерусалима. Иисуса особо вдохновил опыт личного участия в храмовых беседах, и то, что случилось, надолго запечатлелось в его памяти как великое событие отрочества и ранней юности. Иисусу впервые представилась возможность насладиться несколькими днями независимой жизни, свободой уходить и возвращаться без каких\hyp{}либо ограничений. Этот краткий период вольности в течение недели после Пасхи стал первым опытом такой полной свободы от обязанностей, какую он еще никогда не испытывал. Лишь много лет спустя Иисус пережил подобный же период свободы от обязательств, хотя и очень короткий.
\vs p125 0:2 \pc Женщины редко появлялись на Пасху в Иерусалиме, да от них и не требовалось присутствовать. Иисус, однако, фактически отказался идти без матери. Когда же Мария решилась отправится в Иерусалим, то, глядя на нее, многие другие женщины тоже отправились в путь, так что среди жителей Назарета, шедших на Пасху в Иерусалим, женщин оказалось намного больше, чем когда бы то ни было. По дороге путники пели сто тринадцатый псалом.
\vs p125 0:3 По пути от Назарета до вершины Масличной горы Иисус сильно волновался, предчувствуя то, что скоро откроется перед ним. Все счастливое детство он с благоговением слушал рассказы об Иерусалиме и его храме --- теперь он увидит их наяву. С вершины Масличной горы и снаружи при близком рассмотрении храм выглядел еще прекраснее, чем ожидал Иисус, но когда он вошел в священные врата, для него началось великое разочарование.
\vs p125 0:4 Вместе с родителями Иисус прошел через храм, чтобы присоединиться к группе вновь посвященных сыновей завета, которым вскоре предстояло стать гражданами Израиля. Иисуса несколько разочаровало поведение толпы в храме, но первое сильное потрясение в этот день он испытал, когда Мария, оставив их, отправилась в галерею для женщин. Иисус никогда не думал, что мать не имеет права быть рядом с ним во время обрядов посвящения, и был возмущен тем, что ей приходилось страдать от подобной несправедливости. Сильно расстроенный этим, он, однако, кроме нескольких слов протеста, ничего не сказал Иосифу. Но Иисус задумался и задумался глубоко, о чем свидетельствовали вопросы, заданные им книжникам и учителям неделю спустя.
\vs p125 0:5 Иисус прошел обряды посвящения, но был разочарован их поверхностной и рутинной сущностью. Ему недоставало личной вовлеченности, столь характерной для ритуалов синагоги в Назарете. Потом Иисус вернулся поприветствовать мать и приготовился вместе с отцом впервые обойти храм, его многочисленные дворы, галереи и коридоры. Постройки храма одновременно вмещали более двухсот тысяч молящихся, и хотя размеры этих зданий (по сравнению с другими, которые Иисусу приходилось видеть прежде) поражали воображение, ему интереснее было размышлять о духовном смысле храмовых обрядов и связанного с ними богослужения.
\vs p125 0:6 Хотя многие из храмовых обрядов глубоко тронули его своей красотой и символичностью, но Иисус был совершенно разочарован теми толкованиями подлинного смысла этих обрядов, которые предлагали родители в ответ на многочисленные пытливые расспросы сына. Иисус просто не мог принять объяснений смысла богослужений и религиозного поклонения, которые подразумевают веру в мстительность Бога и гнев Всемогущего. При дальнейшем обсуждении этих вопросов после посещения храма, когда Иосиф стал мягко настаивать на том, чтобы сын принял ортодоксальные еврейские верования, Иисус вдруг повернулся к родителям и, взволнованно глядя в глаза отца, сказал: «Мой отец, этого не может быть --- Отец на небесах не может так относиться к своим заблудшим детям на земле. Небесный Отец не может любить их меньше, чем ты любишь меня. Я же знаю, какой бы неразумный поступок я ни совершил, ты никогда не обрушишь свой гнев на меня и не дашь выход ярости против меня. Но если ты, мой земной отец, таким образом отражаешь Божественное, то насколько же полнее доброта и милосердие Небесного Отца. Я не могу поверить, что Небесный Отец любит меня меньше, чем отец земной».
\vs p125 0:7 Услышав такие слова своего первенца, Иосиф с Марией промолчали и более никогда не пытались изменить его понимания любви Бога и милосердия Небесного Отца.
\usection{1. Иисус осматривает храм}
\vs p125 1:1 На всех дворах храма, куда приходил Иисус, царил дух непочтения к Богу, и это сильно потрясло и огорчило его. Иисус счел, что поведение толпы в храме несовместимо с пребыванием «в доме Отца». Другое сильное потрясение юноша испытал, когда Иосиф отвел его на двор, где собирались неевреи и где шумный гомон, выкрики и ругательства, беспорядочно смешиваясь с блеянием овец и гулом голосов, выдавали присутствие менял, торговцев жертвенными животными и другим товаром.
\vs p125 1:2 Однако больше всего Иисуса оскорбил вид беспечных блудниц, которые без стеснения расхаживали по этой части храма; таких же накрашенных женщин он видел недавно, во время путешествия в Сефорис. Подобное осквернение святыни всколыхнуло в юном сердце Иисуса бурю негодования, о чем он без колебаний сказал Иосифу.
\vs p125 1:3 Иисус был восхищен настроением и службой в храме, но его поразило духовное уродство, которое отражалось на лицах многих людей бездумно участвовавших в богослужении.
\vs p125 1:4 Затем Иисус с Иосифом пошли на двор священников, расположенный под каменным выступом стены перед храмом, где находился алтарь, и наблюдали, как убивают множество животных, а священники, совершающие убийства, умывают руки в медном фонтане. Залитый кровью пол, обагренные кровью руки, вопли умирающих животных --- всего этого не мог вынести любящий природу мальчик. Ужасное зрелище вызвало у него отвращение. Он схватил отца за руку и стал умолять увести его. Вместе они снова пересекли двор для неевреев, но даже непристойные жесты и грубый смех, которые Иисус здесь наблюдал, принесли облегчение после только что увиденного.
\vs p125 1:5 Иосиф понял, что храмовые обряды плохо подействовали на сына, и поступил мудро, отведя мальчика посмотреть «красные ворота» --- изумительной красоты сооружение из коринфской бронзы. Но Иисусу уже было довольно впечатлений для первого посещения храма. Они вернулись на верхний двор, забрали Марию и, удалясь от толпы, вышли на свежий воздух, осматривая асмонейский дворец, величественный дом Ирода, и башню римских стражей. Во время прогулки Иосиф объяснил Иисусу, что только жителям Иерусалима позволено наблюдать за ежедневным жертвоприношением в храме, а галилеяне могут участвовать в богослужении всего три раза в год: во время Пасхи, в праздник Пятидесятницы (через семь недель после Пасхи) и в праздник кущей в октябре. Праздники эти были учреждены Моисеем. Они поговорили и о двух других введенных позднее праздниках --- посвящения и Пурим, а затем отправились в дом родственников Марии приготовиться к Пасхе.
\usection{2. Иисус и Пасха}
\vs p125 2:1 В дни празднования Пасхи в семье Симона из Вифании гостило пять семей из Назарета, и хозяин купил пасхального агнца. Убийство в огромном количестве точно таких ягнят сильно взволновало Иисуса во время посещения храма. Семья Иисуса собиралась вкушать пасху с родственниками Марии, но мальчик убедил родителей принять приглашение Симона и отправиться в Вифанию.
\vs p125 2:2 В ту ночь все собрались для исполнения пасхального обряда и ели жареное мясо с пресным хлебом и горькими травами. Иисуса как вновь посвященного сына завета попросили рассказать о происхождении праздника Пасхи. Иисус выполнил просьбу, но немного огорчил родителей, включив в свой рассказ немало замечаний, которые в той или иной степени отражали впечатления, произведенные на его молодой, но пытливый ум тем, что он совсем недавно видел и слышал. Так началось семидневное празднование Пасхи.
\vs p125 2:3 Уже тогда, ничего не сказав родителям, Иисус задумался о том, можно ли обойтись, празднуя Пасху, без убийства агнца. Он был уверен, что Небесного Отца совсем не радуют жертвоприношения и с годами все больше утверждался в решении учредить бескровную Пасху.
\vs p125 2:4 Ночью Иисус спал очень мало: сны об убийстве животных и их страшных мучениях не давали ему покоя. Его разум страдал, а сердце разрывалось от противоречий и абсурдности теологии всей системы еврейских обрядов. Родители Иисуса тоже почти не спали. События прошедшего дня привели их в замешательство. Иосиф и Мария были до глубины души расстроены непонятным и решительным отношением сына. Первую половину ночи Мария пребывала в смятении, Иосиф же оставался спокоен, но и он был в замешательстве. Оба боялись откровенно обсудить с Иисусом эти проблемы, хотя сын с радостью заговорил бы с родителями, решись они вызвать его на беседу.
\vs p125 2:5 Служба в храме на следующий день понравилась Иисусу намного больше и почти стерла неприятные впечатления прошлого дня. Рано утром молодой Лазарь взял с собой Иисуса, и они отправились исследовать Иерусалим и окрестности. К концу дня Иисус обнаружил в храме места, где собравшиеся обсуждали учение и вели диспут; кроме нескольких посещений святая святых, перед завесой которых Иисус останавливался, размышляя, что же в действительности скрывается за ней, почти все время он провел здесь, около беседующих.
\vs p125 2:6 Всю пасхальную неделю Иисус находился среди вновь посвященных сыновей завета, а это означало, что ему приходилось сидеть за перилами, отделявшими всех, кто не был полноправным гражданином Израиля. Сознавая свою молодость, Иисус не отваживался задавать вопросы, которые занимали его, и решил дождаться окончания праздника, когда ограничения, налагаемые на вновь посвященных юношей, будут сняты.
\vs p125 2:7 В среду пасхальной недели Иисусу разрешили пойти домой к Лазарю и провести ночь в Вифании. В этот вечер Лазарь, Марфа и Мария слушали, как Иисус рассуждал о преходящем и вечном, человеческом и божественном, и с той поры все трое полюбили юношу, как родного брата.
\vs p125 2:8 В конце недели Иисус встречался с Лазарем реже, ибо тому не разрешалось даже близко подходить к беседующим возле храма, хотя он и присутствовал, слушая некоторые из речей, которые произносили во внешних дворах. Лазарь и Иисус были ровесниками, однако до исполнения тринадцати лет юношей в Иерусалиме редко допускали к посвящению в сыны закона.
\vs p125 2:9 Во время пасхальной недели родители не раз наблюдали, как Иисус сидел в одиночестве, обхватив голову руками и глубоко задумавшись. Раньше они не видели сына в подобном состоянии и, не ведая, в каком смятении пребывает его ум и как обеспокоена переживаемым его душа, находились в растерянности. Не зная, что делать, Иосиф и Мария даже радовались приближению конца пасхальной недели, желая поскорее увезти Иисуса, которой вел себя так странно, в Назарет.
\vs p125 2:10 День за днем Иисус размышлял о своих проблемах. К концу недели он уже изменил свое отношение ко многому; однако, когда настало время возвращаться в Назарет, его молодой ум все еще был в растерянности от вопросов, которые оставались без ответа, и проблем, которые так и не были решены.
\vs p125 2:11 Перед тем как отправиться из Иерусалима, Иосиф и Мария договорились с назаретским учителем Иисуса, что, когда Иисусу исполнится пятнадцать лет, он вернется в Иерусалим и будет учиться в одной из лучших раввинских академий. Вместе с родителями и учителем Иисус несколько раз ходил в эту школу, но, ко всеобщему огорчению, остался безразличен ко всему, что там говорили и делали. Марию глубоко огорчила реакция сына на посещение Иерусалима, Иосиф тоже был сильно озадачен странными замечаниями мальчика и его необычным поведением.
\vs p125 2:12 В конечном счете, пасхальная неделя оказалась великим событием в жизни Иисуса. Он получил удовольствие от возможности познакомиться с многочисленными сверстниками, которым, подобно ему, предстояло принять посвящение, и воспользовался общением с ними, чтобы узнать, как живут люди в Месопотамии, Туркестане и Парфии, а также в западных провинциях Римской империи. Он уже хорошо знал, в каких условиях росло молодое поколение Египта и других областей близ Палестины. Ведь в это время в Иерусалиме были тысячи молодых людей, и юноша из Назарета познакомился лично и более или менее подробно расспросил более ста пятидесяти из них. Но особенно Иисуса интересовали прибывшие из дальневосточных и западных стран. Благодаря новым знакомствам у него возникло желание путешествовать, чтобы лучше узнать, каким трудом современники добывают свой хлеб.
\usection{3. Отъезд Иосифа и Марии}
\vs p125 3:1 Решено было, что путешественники из Назарета встретятся у храма поздним утром первого дня после окончания праздника Пасхи. В назначенный час все стали собираться в обратный путь. Иисус же тем временем отправился в храм, чтобы послушать обсуждения, пока родители ждали, когда соберутся попутчики. Через какое\hyp{}то время путники приготовились к отправлению. По обычаям посещения праздников в Иерусалиме и возвращения с них мужчины и женщины путешествовали отдельно. В Иерусалим Иисус пришел с матерью и другими женщинами. Теперь же, приняв посвящение, ему следовало возвращаться с отцом вместе с мужчинами. Однако, когда жители Назарета тронулись в путь и направились к Вифании, Иисус был настолько увлечен беседой об ангелах, что позабыл о времени отъезда родителей. Он так и не понял, что остался один, пока в храме не объявили полуденный перерыв.
\vs p125 3:2 Путешественники из Назарета не заметили отсутствия Иисуса, поскольку Мария решила: наверное, он с мужчинами, а Иосиф подумал, что сын идет с женщинами, как шел он в Иерусалим, ведя ослика Марии. Что Иисуса нет, выяснилось только в Иерихоне, где путешественники решили заночевать. Расспросив всех прибывших в город и узнав, что сына никто не видел, Иосиф и Мария провели бессонную ночь, беспокоясь о том, что же с ним случилось. Вспоминая многие из его необычных реакций на события пасхальной недели, они мягко укоряли друг друга за то, что не заметили, как во время отправления из Иерусалима сына не оказалось с ними.
\usection{4. Первый и второй день в храме}
\vs p125 4:1 Между тем Иисус оставался в храме, весь день слушая беседы и наслаждаясь атмосферой покоя и благопристойности --- ведь людей, толпившихся здесь во время пасхальной недели, теперь почти не было. После окончания дневных бесед, ни в одной из которых Иисус не участвовал, он отправился в Вифанию и пришел туда, когда семейство Симона уже готовилось к ужину. Дети обрадовались Иисусу, и он остался ночевать в доме Симона. Однако Иисус почти ни с кем не общался и провел большую часть времени один, размышляя в саду.
\vs p125 4:2 Рано утром Иисус снова направился в храм. На вершине Масличной горы он остановился и заплакал, глядя на картину, открывшуюся его глазам: он видел духовно нищих, скованных традициями людей, которые жили под надзором римских легионов. Начало дня застало Иисуса в храме: он уже решился участвовать в беседах. Тем временем на рассвете встали и Иосиф с Марией, решившие вернуться в Иерусалим. Вначале они поспешили к дому родственников, у которых гостили во время пасхальной недели, но после расспросов выяснилось: Иисуса никто не видел. Проведя в поисках весь день, но так и не найдя сына, Иосиф и Мария вернулись к родственникам переночевать.
\vs p125 4:3 Присутствуя при второй беседе, Иисус осмелился задавать вопросы. То, как он участвовал в беседах, было в высшей степени поразительным, но при этом его манера поведения соответствовала его юному возрасту. Некоторые наиболее острые вопросы Иисуса смущали учителей еврейского закона, но он проявлял дух такой искренней беспристрастности, сочетавшейся с очевидной жаждой знаний, что большинство учителей храма склонилось к тому, чтобы относиться к юноше со всей серьезностью. Однако, когда Иисус осмелился усомниться в справедливости смертного приговора пьяному нееврею, который, расхаживая за пределами двора язычников, нечаянно вошел в запретную и считавшуюся священной часть храма, один из наиболее нетерпимых учителей не выдержал и, сердито посмотрев на юношу, спросил, сколько тому лет. Иисус ответил: «Тринадцать без четырех месяцев». «Тогда, --- возразил учитель с уже нескрываемым раздражением, --- почему ты здесь, если не достиг еще возраста сына заповеди?» Иисус объяснил, что принял посвящение во время Пасхи и уже закончил школу в Назарете. На это учителя как один с насмешкой ответили: «Нам следовало знать --- он из Назарета». Однако старейший из них настоял: Иисуса нельзя обвинять в том, что начальство синагоги в Назарете сочло его формально закончившим учебу, тогда как ему было только двенадцать, а не тринадцать лет; и хотя некоторые из тех, кто пытался унизить Иисуса, встали и ушли, все\hyp{}таки было решено, что юноша может остаться и слушать беседы.
\vs p125 4:4 Когда завершился этот, уже второй день, проведенный в храме, Иисус отправился на ночлег в Вифанию. Он снова пошел в сад и стал размышлять и молиться. Ясно было, что ум его занят размышлениями над тяжелыми проблемами.
\usection{5. Третий день в храме}
\vs p125 5:1 В третий день, проведенный Иисусом с учителями и книжниками, в храме собралось множество наблюдателей, которые, узнав о юноше из Галилеи, пришли подивиться, как совсем еще молодой человек ставит в тупик умудренных в законе мужей. Симон тоже пришел из Вифании посмотреть, чем занимается Иисус. Весь этот день Иосиф и Мария продолжали искать Иисуса; несколько раз они заходили и в храм, но не догадались посмотреть внимательнее на ведущих беседу, хотя один раз были от сына так близко, что почти могли слышать его чарующий голос.
\vs p125 5:2 К концу дня все внимание основной группы беседующих сосредоточилось на вопросах, которые задавал Иисус. Среди них были такие:
\vs p125 5:3 \ublistelem{1.}\bibnobreakspace Что в действительности скрывается за завесой во святая святых?
\vs p125 5:4 \pc \ublistelem{2.}\bibnobreakspace Почему матерей Израиля отделяют от мужей, молящихся в храме?
\vs p125 5:5 \pc \ublistelem{3.}\bibnobreakspace Если Бог --- Отец, любящий своих детей, то нужно ли убивать животных для того, чтобы заслужить его милость? Или, быть может, люди неправильно понимают учение Моисея?
\vs p125 5:6 \pc \ublistelem{4.}\bibnobreakspace Если храм служит поклонению Небесному Отцу, совместимо ли с этим присутствие в нем менял и торговцев?
\vs p125 5:7 \pc \ublistelem{5.}\bibnobreakspace Должен ли ожидаемый Мессия явиться князем этого мира, восседающим на престоле Давида, или ему надлежит стать светом жизни в установлении духовного царства?
\vs p125 5:8 \pc Весь день слушатели Иисуса изумлялись его словам, но никого не поразили они так, как Симона. Более четырех часов юноша из Назарета засыпал еврейских учителей вопросами, заставляющими думать и проникающими в сердце. Он высказал несколько толкований на замечания старейшин, передавая свое учение через вопросы, которые задавал. Тонкой и искусной формулировкой вопросов Иисус бросал вызов их учению, одновременно выдвигая свое. В его словах чувствовалось подкупающее сочетание проницательности и юмора, которое внушило уважение к юноше даже тем, кому его молодость пришлась не по душе. Спрашивая собеседников, Иисус был в высшей степени честен и почтителен, и в этот богатый событиями день в храме показал такое же нежелание пользоваться преимуществом перед противником, какое стало потом характерной чертой всего его последующего служения людям. В юности, а затем и в зрелом возрасте, он был совершенно свободен от любых эгоистических желаний победить в споре только для того, чтобы насладиться триумфом своей логики, и в наивысшей степени стремился лишь к одному --- провозгласить непреходящую истину и тем самым дать людям более полное откровение о вечном Боге.
\vs p125 5:9 \pc Когда день подошел к концу, Симон и Иисус отправились назад в Вифанию. Большую часть пути мужчина и мальчик молчали. И вновь Иисус остановился на вершине Масличной горы, но, посмотрев на город и храм, не заплакал, а лишь склонил голову в безмолвной молитве.
\vs p125 5:10 После ужина в Вифании Иисус опять отказался присоединиться к веселой копании и удалился в сад, где пробыл до глубокой ночи. Но напрасно он старался составить какой\hyp{}нибудь определенный план, как приступить к исполнению цели всей своей жизни, и решить, как наилучшим образом открыть духовно слепым соотечественникам более прекрасное представление о Небесном Отце и таким образом вызволить их из страшного плена закона, обрядов, ритуалов и косных традиций. Однако озарение так и не снизошло на юношу, жаждущего истины.
\usection{6. Четвертый день в храме}
\vs p125 6:1 Как ни странно, Иисус словно забыл о земных родителях, и даже во время завтрака, когда мать Лазаря заметила, что Иосиф и Мария, наверное, уже почти дома, казалось, не понимал, что те, вероятно, волнуются, не зная, где пропадает сын.
\vs p125 6:2 Он снова отправился в храм, но на сей раз не предавался размышлениям на Масличной горе. Большая часть утренней беседы была посвящена закону и пророкам, и учителя поражались, сколь хорошо Иисус знает Писание не только на древнееврейском, но и на греческом языке. Однако их изумили не столько знания Иисуса, сколько его молодость.
\vs p125 6:3 Во время дневной беседы не успели собравшиеся начать обсуждение заданного Иисусом вопроса о назначении молитвы, как старший из учителей подозвал его и, усадив рядом с собой, предложил Иисусу изложить свою точку зрения на молитву и богопочитание.
\vs p125 6:4 \pc Предыдущим вечером родители Иисуса слышали о загадочном юноше, который столь искусно спорил с толкователями закона, но у них и в мыслях не было, что этот молодой человек мог оказаться их сыном. Они уже почти собрались отправиться к дому Захарии, предполагая, что Иисус мог отправиться туда, чтобы повидаться с Елизаветой и Иоанном. Однако, подумав, что, вероятно, в это время Захария в храме, решили свернуть с пути в город Иудин. Каково же было их удивление, когда, пройдя через дворы храма, они узнали голос сына и увидели его сидящим среди учителей.
\vs p125 6:5 Иосиф стоял безмолвный, но Мария дала волю накопившимся тревоге и страху и, подбежав к сыну, который встал поприветствовать изумленных родителей, сказала: «Чадо мое, почему поступаешь так с нами? Уже больше трех дней отец твой и я с великой скорбью ищем тебя. Что овладело тобой и почему ты оставил нас?» Наступило напряженное мгновение. Глаза собравшихся обратились к Иисусу. Все ждали, что он ответит. Иосиф посмотрел на сына с укоризной, но ничего не сказал.
\vs p125 6:6 \pc Необходимо напомнить, что Иисуса уже следовало считать молодым человеком. Ведь он закончил образование, которое давали всем детям, был признан сыном заповеди и принял посвящение как гражданин Израиля. И все\hyp{}таки мать не совсем мягко упрекнула его перед собравшимися людьми как раз тогда, когда он был занят самым серьезным и великим делом своей молодой жизни, и тем самым способствовала бесславному концу величайшей из когда\hyp{}либо выпадавших ему возможностей послужить учителем истины, проповедником праведности, открывателем сущности Отца Небесного, который есть Любовь.
\vs p125 6:7 Но юноша достойно вышел из положения. И если принять во внимание все причины, которые повлияли на возникшую ситуацию, легче понять глубину мудрости ответа мальчика на упрек матери, высказанный ненароком и сгоряча. Задумавшись на мгновение, Иисус ответил Марии: «Зачем было вам так долго искать меня? Разве не знали вы, что найдете меня в доме Отца, ибо пришло время, когда мне должно быть в том, что принадлежит Отцу моему?»
\vs p125 6:8 Все поразились такому ответу и, тихо удалясь, оставили сына наедине с родителями. Немного погодя Иисус вывел всех из затруднительного положения, спокойно сказав: «Давайте отправимся в путь. Каждый из нас поступал так, как считал правильным. Все произошло по воле нашего Отца небесного, пойдемте домой».
\vs p125 6:9 Молча все трое отправились в путь и к ночи прибыли в Иерихон. В пути они остановились только один раз --- на вершине Масличной горы, когда Иисус поднял посох и, дрожа от волнения всем телом, сказал: «О Иерусалим, Иерусалим и жители твои! Что за рабы вы все, подчиненные римскому игу, ставшие жертвами своих же традиций! Но я вернусь очистить этот храм и избавить народ мой от рабства!»
\vs p125 6:10 На протяжении всего пути в Назарет, который занял три дня, Иисус почти не говорил и родители мало говорили при нем. Понять поведение своего первенца им было действительно трудно, но они сохранили в сердцах слова Иисуса, хотя и не смогли до конца постичь их смысл.
\vs p125 6:11 По возвращении домой Иисус коротко объяснился с родителями. Он уверил Иосифа и Марию, что по\hyp{}прежнему их любит и пообещал впредь никогда не давать повода страдать из\hyp{}за себя. Этот важный разговор Иисус закончил словами: «Исполняя волю Небесного Отца, я буду послушен и отцу земному, ожидая, когда придет мой час».
\vs p125 6:12 \pc Хотя Иисус в уме не раз отказывался \bibemph{уступить} благонамеренным, но все\hyp{}таки ошибочным стремлениям родителей повлиять на развитие его мышления и составить план будущей жизни сына на земле, он, тем не менее, всеми способами, совместимыми с его призванием служить делу исполнения воли Райского Отца, со всей тонкостью \bibemph{подчинял} себя желаниям отца земного и обычаям своей земной семьи. Даже тогда, когда был не согласен, он все равно делал все возможное, чтобы подчиниться. Иисус чудесно совмещал преданность долгу с исполнением обязанностей по отношению к собственной семье и служением людям.
\vs p125 6:13 \pc Иосиф не понимал сына, но Мария, вспоминая о том, что произошло, постепенно успокоилась и в конце концов стала смотреть на слова, сказанные Иисусом на Масличной горе, как на пророчество: ее сыну предстоит стать Мессией и избавителем Израиля. Она с новыми силами начала настраивать его на патриотический и националистический лад, подключив к этому и своего брата, любимого дядю Иисуса, и делала все возможное, чтобы приготовить своего первенца к руководству над теми, кому предстояло восстановить престол Давида и навсегда сбросить ярмо политической зависимости от неевреев.
