\upaper{43}{Созвездия}
\author{Малаватия Мелхиседек}
\vs p043 0:1 Урантия обычно называется 606\hyp{}й Сатании в Норлатиадеке Небадона, что означает шестьсот шестой обитаемый мир в локальной системе Сатания, находящейся в созвездии Норлатиадека --- одном из ста созвездий локальной вселенной Небадона. Созвездия --- это первичные подразделения локальной вселенной, и их правители служат связующим звеном между локальными системами обитаемых миров и центральной администрацией локальной вселенной, находящейся в Спасограде, и --- посредством отражательности --- со сверхадминистрацией Древних Дней на Уверсе.
\vs p043 0:2 \pc Правительство вашего созвездия находится в скоплении из 771 архитектурной сферы, центральной и самой большой из которых является Эдентия, резиденция администрации Отцов Созвездия, Всевышних Норлатиадека. Сама Эдентия приблизительно в сто раз больше, чем ваш мир. Семьдесят окружающих Эдентию больших планет примерно в десять раз больше Урантии, а десять спутников, вращающихся вокруг каждого из этих семидесяти миров, имеют размер примерно с Урантию. Эти 771 архитектурная сфера вполне сравнимы по размерам с архитектурными сферами других созвездий.
\vs p043 0:3 \pc В Эдентии принято спасоградское исчисление времени и измерение расстояний, и, как и сферы столицы вселенной, центральные миры созвездия полностью обеспечены всеми чинами небесных разумных существ. В общем, эти личности не очень отличаются от чинов администрации вселенной, описанных выше.
\vs p043 0:4 На службу в созвездия назначаются серафимы\hyp{}руководители, третий род ангелов локальной вселенной. Они размещают свои центры на столичных планетах и широко служат в окружающих моронтийных учебных мирах. В Норлатиадеке все семьдесят больших сфер вместе с семьюстами малыми спутниками населены унивитациями --- постоянными гражданами созвездия. Все эти архитектурные миры административно полностью управляются различными группами местных уроженцев, которые, по большей части, не раскрыты, но к числу которых относятся эффективные спиронги и прекрасные спорнагии. Приходясь на среднюю точку процесса моронтийного обучения, моронтийная жизнь созвездий, как ты можешь предположить, является и типичной, и идеальной.
\usection{1. Центры созвездий}
\vs p043 1:1 Эдентия изобилует очаровательными нагорьями, обширными возвышенностями из физической материи, увенчанными моронтийной жизнью и покрытыми духовным великолепием, но там нет таких скалистых гор, какие имеются на Урантии. Есть десятки тысяч искрящихся озер и тысячи и тысячи связующих речек, но нет ни огромных океанов, ни стремительных рек. Только поверхности нагорий лишены этих текущих рек.
\vs p043 1:2 Вода Эдентии и аналогичных архитектурных сфер ничем не отличается от воды эволюционных планет. Водные системы таких сфер бывают как поверхностные, так и подземные, и влага постоянно циркулирует. Вокруг Эдентии можно проплыть по этим различным водным маршрутам, хотя главной средой перемещения является атмосфера. Духовные существа свободно перемещаются над поверхностью планеты, моронтийные же и материальные существа для движения через атмосферу используют материальные и полуматериальные средства.
\vs p043 1:3 Эдентия и связанные с ней миры имеют настоящую атмосферу, обычную смесь трех газов, которая характерна для таких архитектурных творений и которая включает два элемента атмосферы Урантии плюс моронтийный газ, пригодный для дыхания моронтийных созданий. Но, хотя эта атмосфера является и материальной, и моронтийной, бурь и ураганов не бывает; не бывает также ни лета, ни зимы. Это отсутствие атмосферных ненастий и сезонных изменений дает возможность украшать всю открытую поверхность в этих особым образом сотворенных мирах.
\vs p043 1:4 Нагорья Эдентии --- это великолепные физические детали рельефа, и их красота усиливается от бесконечного богатства жизни, изобилующей на всем их протяжении. За исключением нескольких довольно изолированных сооружений, эти нагорья не содержат ничего сотворенного руками созданий. Материальные и моронтийные украшения ограничены территориями проживания. Небольшие возвышенности являются местонахождением особых резиденций и отличаются красивыми как биологическими, так и моронтийными видами.
\vs p043 1:5 \pc На вершине седьмой гряды нагорья расположены залы воскресения Эдентии, в которых пробуждаются восходящие смертные вторичного модифицированного чина восхождения. Эти палаты, в которых воссоздаются создания, находятся под руководством Мелхиседеков. На первой из принимающих планет Эдентии (подобной планете Мелхиседек возле Спасограда) также имеются специальные залы воскресения, в которых воссоздаются смертные модифицированных чинов восхождения.
\vs p043 1:6 Мелхиседеки также содержат на Эдентии два специальных учебных заведения. Одно --- школа чрезвычайных ситуаций --- занимается изучением проблем, появившихся в результате бунта в Сатании. Другое --- школа пришествия --- предназначено для овладения новыми проблемами, связанными с последним пришествием Михаила именно в один из миров Норлатиадека. Это второе учебное заведение было основано почти четыре тысячи лет назад, непосредственно после того, как Михаил объявил, что Урантия избрана тем миром, куда будет совершено пришествие.
\vs p043 1:7 \pc Стеклянное море --- принимающая территория Эдентии --- находится недалеко от административного центра и окружена центральным амфитеатром. Вокруг этой территории расположены центры руководства семьюдесятью категориями дел созвездия. Половина Эдентии разделена на семьдесят треугольных отделений, границы которых сходятся в одной точке у центральных зданий их соответствующих секторов. Остальная часть этой планеты представляет собой один огромный природный парк --- сады Бога.
\vs p043 1:8 Во время периодических посещений Эдентии, хотя для осмотра тебе будет открыта вся планета, большую часть времени ты будешь проводить в том административном треугольнике, номер которого соответствует номеру мира, в котором ты в тот момент обитаешь. Ты всегда будешь желанным гостем в законодательных собраниях в качестве наблюдателя.
\vs p043 1:9 Моронтийная территория, отведенная для восходящих смертных, постоянно проживающих на Эдентии, находится в средней части тридцать пятого треугольника и примыкает к центру финалитов, расположенному в тридцать шестом треугольнике. Общий центр унивитаций занимает огромную территорию в средней части тридцать четвертого треугольника и непосредственно примыкает к месту, отведенному для проживания моронтийных граждан. Из этого устройства видно, что предусмотрено размещение, по крайней мере, семидесяти больших видов небесных живых существ и что каждая из этих семидесяти треугольных территорий соотносится с какой\hyp{}то одной из семидесяти крупных сфер моронтийного обучения.
\vs p043 1:10 Эдентийское стеклянное море --- это один громадный округлый кристалл, размером примерно сто миль в окружности и около тридцати миль в глубину. Этот великолепный кристалл служит приемным полем для всех серафимов перемещения и других существ, прибывающих из\hyp{}за пределов этой планеты; такое стеклянное море чрезвычайно облегчает приземление серафимов перемещения.
\vs p043 1:11 Кристаллическое поле подобного рода обнаруживается почти во всех архитектурных мирах; и, помимо своей декоративной ценности, оно служит для многих целей, оно используется для отображения сверхвселенской отражательности собранным группам и как фактор в технике преобразования энергии для модификации пространственных потоков и для адаптации других поступающих потоков физической энергии.
\usection{2. Правительство созвездия}
\vs p043 2:1 Созвездия --- это автономные единицы локальной вселенной, и каждое созвездие административно управляется в соответствии со своими собственными законодательными актами. Когда суды Небадона проводят судебные заседания по вселенским делам, решения по всем внутренним вопросам выносятся в соответствии с законами, принятыми в соответствующем созвездии. Эти судебные постановления Спасограда вместе с законодательными актами созвездий исполняются руководителями локальных систем.
\vs p043 2:2 Созвездия, таким образом, действуют как законодательные или законотворческие единицы, тогда как локальные системы выступают в качестве распорядительных единиц, проводящих законы в жизнь. Спасоградское правительство является верховной судебной и координирующей властью.
\vs p043 2:3 \pc Хотя верховная судебная функция принадлежит центральной администрации локальной вселенной, в центре каждого созвездия есть два вспомогательных, но важных суда: совет Мелхиседеков и суд Всевышнего.
\vs p043 2:4 Все судебные проблемы сначала рассматриваются советом Мелхиседеков. Двенадцать представителей этого чина, получившие определенный необходимый опыт на эволюционных планетах и в центральных мирах систем, уполномочены рассматривать свидетельства, систематизировать заявления и выносить предварительные вердикты, которые передаются дальше в суд Всевышнего, правящего Отца Созвездия. Смертное подразделение этого второго суда состоит из семи судей, и все они являются восходящими смертными. Чем выше восходишь во вселенной, тем более несомненно, что ты будешь судим себе подобными.
\vs p043 2:5 \pc Законодательный орган созвездия делится на три группы. Законодательная программа созвездия зарождается в нижней палате восходящих --- группе, возглавляемой финалитом и состоящей из тысячи представителей смертных. Каждая система выдвигает по десять членов для заседания в этом совещательном собрании. На Эдентии в настоящее время этот орган укомплектован не полностью.
\vs p043 2:6 Средняя палата законодателей составлена из сонмов серафимов и их сподвижников, других детей Духа\hyp{}Матери локальной вселенной. Эта группа насчитывает сто членов, назначенных руководящими личностями, которые возглавляют разные виды деятельности, осуществляемые этими существами в созвездии.
\vs p043 2:7 Консультативный, или высший орган законодателей созвездия состоит из палаты пэров --- палаты божественных Сынов. Этот отряд избирается Всевышними Отцами и насчитывает десять членов. В этой верхней палате могут служить только Сыны с особым опытом. Эта группа, расследующая факты и экономящая время, очень эффективно служит обоим низшим подразделениям законодательного собрания.
\vs p043 2:8 В объединенный совет законодателей входят по три члена от каждой из этих отдельных ветвей совещательного собрания созвездия, а возглавляет его младший правящий Всевышний. Эта группа утверждает в окончательной форме все законодательные акты и дает санкцию на их обнародование возвестниками. Одобрение этой верховной комиссии превращает законодательные акты в законы данной сферы; их постановления окончательны. Законодательные решения Эдентии составляют основной закон всего Норлатиадека.
\usection{3. Всевышние Норлатиадека}
\vs p043 3:1 Правители созвездий относится к чину Ворондадеков сыновства локальной вселенной. Когда эти Сыны облечены полномочиями нести активную службу во вселенной в качестве правителей созвездий или кого\hyp{}то еще, их называют \bibemph{Всевышними,} поскольку из всех чинов Сынов Бога Локальной Вселенной они воплощают в себе самую высокую управленческую мудрость в сочетании с самой дальновидной и мудрой верностью. Их личная честность и коллективная верность никогда не ставились под сомнение; в Небадоне Сыны\hyp{}Ворондадеки ни разу не вызывали какого\hyp{}либо недовольства.
\vs p043 3:2 \pc По меньшей мере, по три Сына\hyp{}Ворондадека облечены Гавриилом полномочиями Всевышних каждого из созвездий Небадона. Член этого трио, его возглавляющий, называется \bibemph{Отцом Созвездия,} а двое его помощников --- это \bibemph{старший Всевышний и младший Всевышний.} Отец Созвездия правит десять тысяч стандартных лет (примерно 50\,000 урантийских лет), прослужив до этого на протяжении таких же периодов младшим помощником и старшим помощником.
\vs p043 3:3 Псалмопевец знал, что Эдентия управляется тремя Отцами Созвездий и, соответственно, говорил об их обиталище во множественном числе: «Речные потоки веселят град Божий, святое жилище Всевышних».
\vs p043 3:4 \pc Со стародавних времен и до настоящего времени на Урантии существует величайшая путаница относительно различных правителей вселенной. Многие более поздние учители принимали своих непонятных и странных племенных божеств за Всевышних Отцов. Еще позже иудеи слили всех этих небесных правителей в одно составное Божество. Один учитель понимал, что Всевышние не были Верховными Правителями, ибо он сказал: «Живущий под кровом Всевышнего под сенью Всемогущего покоится». В урантийских записях временами очень трудно точно понять, кто обозначается термином «Всевышний». Но Даниил вполне понял эти вопросы. Он сказал: «Всевышний правит в царстве людей и дает его, кому хочет».
\vs p043 3:5 \pc Отцы Созвездия мало занимаются индивидуумами обитаемой планеты, но они тесно связаны с теми законодательными и законотворческими функциями созвездий, которые имеют самое прямое отношение к каждой человеческой \bibemph{расе} и национальной \bibemph{группе} обитаемых миров.
\vs p043 3:6 Хотя между вами и вселенской администрацией стоит власть созвездия, обычно вы как индивидуумы мало интересуетесь правительством созвездия. В нормальной ситуации вас больше всего интересует локальная система, Сатания; но временно из\hyp{}за определенных условий в системе и на планете, проистекающих из бунта Люцифера, Урантия тесно связана с правителями созвездия.
\vs p043 3:7 Всевышние Эдентии во время раскола Люцифера захватили в бунтарских мирах отдельные сферы планетарной власти. Они продолжили осуществлять эту власть, и Древние Дней давно уже утвердили такой принятый на себя контроль над этими заблудшими мирами. Они, без сомнения, будут продолжать осуществлять эти принятые на себя полномочия до тех пор, пока жив Люцифер. Обычно, в верной системе, значительной частью этой власти был бы наделен Владыка Системы.
\vs p043 3:8 Но есть еще и другая причина, по которой Урантия оказалась особым образом связанной со Всевышними. Когда Михаил, Сын\hyp{}Творец, выполнял свою заключительную миссию пришествия, то, поскольку преемник Люцифера не имел всей полноты власти в локальной системе, всеми урантийскими делами, касавшимися пришествия Михаила, руководили непосредственно Всевышние Норлатиадека.
\usection{4. Гора собраний --- Верные Дней}
\vs p043 4:1 Святейшая гора собраний --- это место обитания Верного Дней, действующего на Эдентии представителя Райской Троицы.
\vs p043 4:2 Верный Дней --- это Райский Сын Троицы, и он присутствовал на Эдентии в качестве личного представителя Иммануила со времени сотворения этого центрального мира. Верный Дней вечно стоит по правую руку от Отцов Созвездия и дает им советы, но он никогда не предлагает совет, если его об этом не просят. Высокие Сыны Рая никогда не участвуют в ведении дел локальной вселенной, кроме как по просьбе действующих правителей этих сфер. Но для Всевышних созвездия Верный Дней является всем тем, чем Объединяющий Дней является для Сына\hyp{}Творца.
\vs p043 4:3 Резиденцией Верного Дней Эдентии является центр Райской системы вневселенской связи и информации данного созвездия. Эти Сыны Троицы со своим штатом личностей Хавоны и Рая во взаимосвязи с руководящим Объединяющим Дней поддерживают прямую и постоянную связь с представителями своего чина повсюду во всех вселенных, даже в Хавоне и Раю.
\vs p043 4:4 Святейшая гора изысканно красива и чудесно обустроена, но сама резиденция Райского Сына скромна по сравнению с центральным обиталищем Всевышних и с окружающими семьюдесятью сооружениями, составляющими жилой комплекс Сынов\hyp{}Ворондадеков. Эти резиденции обустроены исключительно для проживания; они полностью отделены от обширных зданий административного центра, где занимаются делами созвездия.
\vs p043 4:5 Резиденция Верного Дней на Эдентии расположена к северу от этих резиденций Всевышних и называется «горой Райского собрания». На этом священном нагорье восходящие смертные периодически собираются, чтобы послушать рассказ этого Сына Рая о долгом и увлекательном пути восходящих смертных через миллиард миров совершенства Хавоны и дальше к неописуемым прелестям Рая. И именно на этих особых встречах на Горе Собраний моронтийные смертные более обстоятельно знакомятся с разными группами личностей, являющихся уроженцами центральной вселенной.
\vs p043 4:6 Вероломный Люцифер, некогда бывший владыкой Сатании, заявляя свои притязания на увеличение полномочий, стремился заместить все высшие чины сыновства в плане управления локальной вселенной. Он вынашивал замыслы в своем сердце и говорил: «Я вознесу свой трон выше Сынов Бога; я буду восседать на горе собраний на севере; я буду подобен Всевышним».
\vs p043 4:7 \pc Сто Владык Систем периодически собираются на конклавы Эдентии, на которых совещаются по вопросам благоденствия созвездия. После сатанийского бунта архибунтовщики из Иерусема, как ни в чем ни бывало, являлись на эти советы Эдентии точно так же, как делали это прежде. И не находилось способа прекратить эту бесцеремонную наглость до тех пор, пока не завершилось пришествие Михаила на Урантию и он не принял на себя после этого неограниченное владычество над всем Небадоном. И никогда с того дня не позволялось этим подстрекателям греха заседать в советах Эдентии верных Владык Систем.
\vs p043 4:8 То, что об этом знали учителя старых времен, ясно из записи: «И был день, когда Сыны Бога пришли и предстали перед Всевышними, и Сатана также пришел и предстал перед ними». И это констатация факта, независимо от того, в какой связи она появляется.
\vs p043 4:9 \pc Со времени триумфа Христа весь Норлатиадек очищается от греха и бунтовщиков. В какой\hyp{}то момент до смерти Михаила во плоти, Сатана, сподвижник падшего Люцифера, пытался посетить такой конклав Эдентии, но твердость в неприятии архибунтовщиков дошла до такой степени, что двери сочувствия закрылись почти повсеместно и была выбита почва из\hyp{}под ног противников Сатании. Когда нет двери, открытой для восприятия зла, не существует и возможности для поощрения греха. Во всей Эдентии двери сердец закрылись для Сатаны; он был единодушно отвергнут собравшимися Владыками Систем, и именно в это время Сын Человеческий «видел Сатану, падшего с неба, как молния».
\vs p043 4:10 После бунта Люцифера было создано новое сооружение возле резиденции Верного Дней. Это временное строение является штабом связного Всевышнего, который действует в тесном контакте с Райским Сыном в качестве советчика правительства созвездия по всем вопросам, касающимся политики и отношения чина Дней к греху и бунту.
\usection{5. Отцы Эдентии после бунта Люцифера}
\vs p043 5:1 Во время бунта Люцифера периодическая смена Всевышних на Эдентии была приостановлена. Сейчас у нас те же самые правители, которые исполняли обязанности в то время. Мы предполагаем, что в составе этих правителей не будет производиться изменений до тех пор, пока окончательно не покончат с Люцифером и его сподвижниками.
\vs p043 5:2 Однако состав нынешнего правительства созвездия был расширен, и в него были включены двенадцать Сынов чина Ворондадеков. Это:
\vs p043 5:3 \ublistelem{1.}\bibnobreakspace Отец Созвездия. Теперешний Всевышний правитель Норлатиадека --- номер 617\,318 в ряду Ворондадеков Небадона. Он нес службу во многих созвездиях повсюду в нашей локальной вселенной прежде, чем принять на себя свои обязанности в Эдентии.
\vs p043 5:4 \ublistelem{2.}\bibnobreakspace Старший Всевышний сподвижник.
\vs p043 5:5 \ublistelem{3.}\bibnobreakspace Младший Всевышний сподвижник.
\vs p043 5:6 \ublistelem{4.}\bibnobreakspace Всевышний советчик, являющийся личным представителем Михаила со времени достижения им статуса Сына\hyp{}Мастера.
\vs p043 5:7 \ublistelem{5.}\bibnobreakspace Всевышний распорядитель --- личный представитель Гавриила, постоянно пребывающий на Эдентии со времени бунта Люцифера.
\vs p043 5:8 \ublistelem{6.}\bibnobreakspace Всевышний глава планетарных наблюдателей, управляющий Ворондадеками\hyp{}наблюдателями, пребывающими в изолированных мирах Сатании.
\vs p043 5:9 \ublistelem{7.}\bibnobreakspace Всевышний арбитр --- Сын\hyp{}Ворондадек, которому вверена обязанность урегулировать в созвездии все сложности, проистекающие из бунта.
\vs p043 5:10 \ublistelem{8.}\bibnobreakspace Всевышний чрезвычайный руководитель --- Сын\hyp{}Ворондадек, которому поручена задача адаптировать постановления законодательных органов Норлатиадека, связанные с чрезвычайными мерами по отношению к изолированным в результате бунта мирам Сатании.
\vs p043 5:11 \ublistelem{9.}\bibnobreakspace Всевышний посредник --- Сын\hyp{}Ворондадек, назначенный согласовывать особые, связанные с пришествием, корректировки на Урантии с повседневным административным управлением созвездием. Необходимость в деятельности этого сына вызвана определенной деятельностью архангелов и других необычных служений на Урантии, а также особой деятельностью Блестящей Вечерней Звезды в Иерусеме.
\vs p043 5:12 \ublistelem{10.}\bibnobreakspace Всевышний судья\hyp{}адвокат --- глава чрезвычайного суда, занимающегося урегулированием особых проблем Норлатиадека, проистекающих из смятения, являющегося результатом бунта в Сатании.
\vs p043 5:13 \ublistelem{11.}\bibnobreakspace Всевышний связной --- Сын\hyp{}Ворондадек, прикрепленный к правителям Эдентии, но получивший полномочия особого советника при Верном Дней по вопросам касательно того, каким образом лучше всего решать проблемы, имеющие отношение к бунту и неверности созданий.
\vs p043 5:14 \ublistelem{12.}\bibnobreakspace Всевышний руководитель --- председатель чрезвычайного совета Эдентии. Все личности, назначенные в Норлатиадек из\hyp{}за переворота в Сатании, образуют чрезвычайный совет, а возглавляет его Сын\hyp{}Ворондадек, имеющий экстраординарный опыт.
\vs p043 5:15 И здесь еще не учитываются многочисленные Ворондадеки --- посланцы созвездий Небадона и прочие, также постоянно пребывающие в Эдентии.
\vs p043 5:16 \pc Со времени бунта Люцифера Отцы Эдентии всегда проявляли особую заботу об Урантии и других изолированных мирах Сатании. Пророк давно распознал направляющую руку Отцов Созвездия в делах наций. «Когда Всевышний давал уделы народам, когда он расселял сынов Адама, тогда он установил пределы народов».
\vs p043 5:17 В каждом находящемся в карантине или изолированном мире есть Сын\hyp{}Ворондадек, действующий как наблюдатель. Он не участвует в административном управлении планетой, кроме тех случаев, когда Отец Созвездия приказывает вмешаться в дела наций. В действительности, именно этот Всевышний наблюдатель «правит в царстве людей». Урантия --- один из изолированных миров Норлатиадека, и со времени предательства Калигастии на планете постоянно находится Ворондадек\hyp{}наблюдатель. Когда Махивента Мелхиседек в полуматериальной форме осуществлял служение на Урантии, он проявлял почтительное уважение к исполнявшему тогда обязанности Всевышнего наблюдателя, как об этом написано: «И Мелхиседек, царь Салима, был священником Всевышнего». Мелхиседек раскрыл отношение этого Всевышнего наблюдателя к Аврааму, когда сказал: «И благословен Всевышний, который предал врагов твоих в руки твои».
\usection{6. Сады Бога}
\vs p043 6:1 Столицы систем особенно красивы своими материальными и минеральными сооружениями, тогда как центр вселенной больше отражает духовное великолепие, а столицы созвездий --- это кульминация моронтийной деятельности и живых украшений. В центральных мирах созвездий более широко используются живые украшения, и именно это преобладание жизни --- ботанического искусства является причиной того, что эти миры называют «садами Бога».
\vs p043 6:2 \pc Около половины Эдентии отведено изысканным садам Всевышних, и эти сады относятся к числу самых очаровательных моронтийных творений в локальной вселенной. Этим объясняется, почему необычайно красивые места в обитаемых мирах Норлатиадека так часто называются «Эдемским садом».
\vs p043 6:3 Центральное местоположение в этом великолепном саду занимает храм поклонения Всевышних. Псалмопевец, должно быть, знал что\hyp{}то об этом, ибо писал: «Кто взойдет на гору Всевышних? Кто станет на этом святом месте? Тот, у которого руки чисты и сердце непорочно, кто не предавался своею душою суете и не клялся лживо». В этом храме в каждый десятый день отдыха вся Эдентия во главе со Всевышними предается исполненным поклонения размышлениям о Боге Верховном.
\vs p043 6:4 \pc В архитектурных мирах есть десять форм жизни материального чина. На Урантии имеется растительная и животная жизнь, но в таких мирах, как Эдентия, существует десять типов материальных чинов жизни. Будь вам дано увидеть эти десять типов эдентийской жизни, вы быстро классифицировали бы первые три как растительные, а последние три --- как животные, но были бы совершенно не в состоянии понять природу занимающих промежуточное положение четырех групп обильных и чарующих форм жизни.
\vs p043 6:5 Даже явно животная жизнь очень отличается от той, что имеется в эволюционных мирах, --- настолько, что совершенно невозможно описать человеческому разуму уникальный характер и нежную натуру этих не говорящих существ. Есть тысячи и тысячи живых существ, которых вы вряд ли могли бы себе даже вообразить. В целом, животные творения относятся к совершенно другому чину, нежели грубые виды животных эволюционных планет. Но эти животные существа чрезвычайно умны и необыкновенно услужливы, и все их разнообразные виды удивительно кротки и трогательно общительны. В таких архитектурных мирах нет плотоядных существ; на всей Эдентии нет ничего такого, что заставляло бы какое\hyp{}либо живое существо бояться.
\vs p043 6:6 Растительная жизнь тоже сильно отличается от урантийской и состоит как из материальных, так и из моронтийных разновидностей. Материальная растительность имеет характерную зеленую окраску, моронтийные же эквиваленты растительной жизни имеют фиолетовый или лиловый тон с разными оттенками и отливами. Такая моронтийная растительность является чисто энергетической; при еде она усваивается без остатка.
\vs p043 6:7 Наделенные десятью типами физической жизни, не считая моронтийных вариантов, эти архитектурные миры предоставляют огромные возможности для биологического украшения ландшафта и материальных и моронтийных сооружений. Небесные ремесленники направляют местных спорнагий на эту обширную работу по ботаническому декорированию и биологическому украшению. Если ваши художники для воплощения своих замыслов вынуждены прибегать к инертной краске и безжизненному мрамору, то небесные ремесленники и унивитации чаще используют живые материалы для того, чтобы представить свои идеи и запечатлеть свои идеалы.
\vs p043 6:8 Если вы получаете удовольствие от цветов, кустарников и деревьев на Урантии, тогда ваш взгляд усладит ботаническая красота и великолепие флоры божественных садов Эдентии. Но не хватит моих способностей для того, чтобы описать и попытаться дать человеческому разуму адекватное представление об этих красотах небесных миров. Поистине, глаз не видел такого великолепия, какое ожидает вас по прибытии в эти миры на пути человеческого восхождения.
\usection{7. Унивитации}
\vs p043 7:1 Унивитации являются постоянными гражданами Эдентии и связанных с ней миров, и все семьсот семьдесят миров, окружающих центр созвездия, находятся под их руководством. Эти дети Сына\hyp{}Творца и Творческого Духа занимают промежуточное положение между материальным и духовным уровнями существования, но они не являются моронтийными созданиями. Уроженцы каждой из семидесяти больших планет Эдентии обладают различными видимыми формами, и, чтобы соответствовать восходящей шкале унивитаций, формы моронтийных смертных подстраиваются каждый раз, когда их местопребывание перемещается с одной планеты Эдентии на другую при последовательном переходе из мира в мир --- от мира номер один до мира номер семьдесят.
\vs p043 7:2 Духовно унивитации одинаковы; по интеллекту они различаются, как и люди; по форме они сильно напоминают моронтийное состояние существования, и сотворено семьдесят различных функционирующих чинов этих личностей. В каждом из этих чинов по интеллектуальной деятельности выделяется десять больших разновидностей унивитаций, и каждый из этих различающихся интеллектуальных типов возглавляет особые учебные заведения и школы культуры прогрессирующей профессиональной или практической социализации на каком\hyp{}то одном из десяти спутников, вращающихся вокруг каждого из больших миров Эдентии.
\vs p043 7:3 Эти семьсот малых миров являются техническими сферами практического образования в области работы всей локальной вселенной и открыты для всех классов разумных существ. Эти школы, обучающие особому мастерству и техническим знаниям, устроены не только для восходящих смертных, хотя моронтийные учащиеся и составляют гораздо большую группу, чем все остальные посещающие эти учебные курсы. Когда тебя принимают в какой\hyp{}либо один из этих семидесяти больших миров социальной культуры, то сразу же дают допуск на каждый из десяти окружающих спутников.
\vs p043 7:4 В различных гостящих колониях восходящие моронтийные смертные преобладают среди руководителей восстановления, унивитации же составляют самую большую группу, связанную с небадонским отрядом небесных ремесленников. Во всем Орвонтоне никакие внехавонские существа, кроме абандонтеров Уверсы, не могут сравниться с унивитациями по художественному мастерству, социальной адаптируемости и способности к взаимодействию.
\vs p043 7:5 Эти граждане созвездия в действительности не являются членами отряда ремесленников, но они добровольно трудятся вместе со всеми группами и вносят большой вклад в превращение миров созвездия в главные сферы реализации великолепных художественных возможностей переходной культуры. За пределами центральных миров созвездия они не действуют.
\usection{8. Учебные миры Эдентии}
\vs p043 8:1 Физическое дарование Эдентии и окружающих ее планет почти совершенно; они едва ли могли бы сравниться с духовным великолепием сфер Спасограда, но далеко превосходят великолепие учебных миров Иерусема. Все эти планеты Эдентии питаются энергией непосредственно от вселенских пространственных потоков, и эти огромные энергетические системы, как материальные, так и моронтийные, квалифицированно управляются и распределяются центрами созвездия, которым помогает компетентный отряд Мастеров Физических Контролеров и Руководителей Моронтийной Мощи.
\vs p043 8:2 Время, проводимое в семидесяти учебных мирах переходной моронтийной культуры, связанное с эдентийским периодом восхождения смертных, --- это самый спокойный период на пути восходящего смертного вплоть до достижения им статуса финалита; это действительно типичная моронтийная жизнь. Хотя ты перенастраиваешься каждый раз, когда переходишь из одного большого мира культуры в другой, но сохраняешь то же самое моронтийное тело, и нет периодов бессознательного состояния личности.
\vs p043 8:3 Во время пребывания на Эдентии и связанных с ней планетах ты будешь занят, главным образом, овладением групповой этикой, секретами приятных и полезных взаимоотношений между различными вселенскими и сверхвселенскими чинами разумных существ.
\vs p043 8:4 В мирах\hyp{}обителях ты окончательно достиг цельности развивающейся человеческой личности; в столице системы добился иерусемского гражданства и достиг готовности подчинить свое «я» дисциплине групповой деятельности и согласованных мероприятий; но теперь в учебных мирах созвездия ты должен достичь настоящей социализации своей развивающейся моронтийной личности. Приобретение этого высшего культурного навыка состоит в том, чтобы научиться:
\vs p043 8:5 \ublistelem{1.}\bibnobreakspace Счастливо жить и эффективно трудиться вместе с десятью различными собратьями\hyp{}моронтийцами, когда десять таких групп объединяются в компании по сто, а затем в отряды по тысяче существ.
\vs p043 8:6 \ublistelem{2.}\bibnobreakspace Радостно жить и сердечно сотрудничать с десятью унивитациями, которые, хотя и сходны в интеллектуальном плане с моронтийными существами, сильно отличаются во всех прочих отношениях. И затем ты должен действовать вместе с этой группой из десяти существ, когда она сообразуется с десятью другими семьями, которые, в свою очередь, объединяются в отряд из тысячи унивитаций.
\vs p043 8:7 \ublistelem{3.}\bibnobreakspace Достичь адаптации одновременно и к собратьям\hyp{}моронтийцам, и к этим унивитациям\hyp{}хозяевам. Приобрести способность добровольно и эффективно сотрудничать с существами своего собственного чина в тесном рабочем взаимодействии с несколько непохожей группой разумных существ.
\vs p043 8:8 \ublistelem{4.}\bibnobreakspace Социально функционируя таким образом вместе с существами похожими и непохожими на тебя, достичь интеллектуальной гармонии с обеими группами сподвижников и профессионально адаптироваться к ним.
\vs p043 8:9 \ublistelem{5.}\bibnobreakspace Достигнув удовлетворительной социализации личности на интеллектуальном и профессиональном уровнях, далее совершенствовать способность жить в тесном контакте с похожими и относительно непохожими существами так, чтобы постоянно уменьшалась раздражительность и постепенно исчезали отрицательные чувства. Руководители восстановления вносят большой вклад в достижение этого с помощью проведения групповой игровой деятельности.
\vs p043 8:10 \ublistelem{6.}\bibnobreakspace Нацелить всю эту технику социализации на усиление прогрессивной координации пути Райского восхождения; увеличить вселенскую проницательность, усиливая способность понимать вечные целевые значения, скрытые в этих, казалось бы, незначительных видах пространственно\hyp{}временной деятельности.
\vs p043 8:11 \ublistelem{7.}\bibnobreakspace И затем, с одновременным ростом духовной проницательности, довести до совершенства все эти процедуры многообразной социализации, имеющие отношение к усилению всех фаз личностных способностей, --- посредством группового духовного объединения и моронтийного согласования. В интеллектуальном, социальном и духовном плане два нравственных создания посредством партнерства не просто удваивают свой личностный потенциал вселенских достижений; они, скорее, почти учетверяют свои возможности достижений и свершений.
\vs p043 8:12 \pc Мы обрисовали эдентийскую социализацию как объединение моронтийного смертного с группой существ семьи унивитаций, состоящей из десяти несхожих в интеллектуальном отношении индивидуумов, и сопутствующее аналогичное объединение с десятью собратьями\hyp{}моронтийцами. Но в первых семи больших мирах с десятью унивитациями живет только один восходящий смертный. Во второй группе из семи больших миров с каждой группой из десяти исконных жителей живут по два смертных --- и так далее до последней группы из семи больших планет, где десять моронтийных существ поселяются с десятью унивитациями. Учась лучше социализироваться с унивитациями, вы будете применять такую усовершенствованную этику на практике в своих отношениях с собратьями --- моронтийными прогрессорами.
\vs p043 8:13 Как восходящие смертные вы получите удовольствие от своего пребывания в эдентийских мирах восхождения, но не испытаете того личностного захватывающего чувства удовлетворения, которое характерно для начального контакта со вселенскими делами в центре системы или для прощального соприкосновения с этими реальностями в завершающих мирах столицы вселенной.
\usection{9. Граждане Эдентии}
\vs p043 9:1 После окончания учебы в мире номер семьдесят местом пребывания восходящих смертных становится Эдентия. Теперь восходящие впервые посещают «Райские собрания» и слышат рассказ о своем долгом пути в изложении Верного Дней, первой из встреченных ими Верховных Личностей, происходящих от Троицы.
\vs p043 9:2 \pc Все это пребывание в учебных мирах созвездия, кульминацией которого становится получение эдентийского гражданства, является для моронтийных прогрессоров периодом истинного и небесного блаженства. На протяжении всего пребывания в мирах системы ты развивался из почти животного в моронтийное создание; ты был более материален, чем духовен. На сферах Спасограда ты будешь развиваться из моронтийного существа до статуса истинного духа; ты будешь более духовным, нежели материальным. Но на Эдентии восходящие находятся посередине между своим прошлым и будущим статусами, в середине своего перехода от эволюционного животного к восходящему духу. На протяжении всего пребывания на Эдентии и в ее мирах ты «подобен ангелам»; ты постоянно движешься вперед, но все это время сохраняешь общий и типичный моронтийный статус.
\vs p043 9:3 Это пребывание восходящего смертного в созвездии --- самая единообразная и стабильная эпоха на всем пути моронтийного продвижения. Содержанием этого опыта является обучение восходящих преддуховной социализации. Он аналогичен предфиналитному духовному опыту в Хавоне и предабсонитному обучению в Раю.
\vs p043 9:4 \pc На Эдентии восходящие смертные заняты, главным образом, назначениями в семьдесят последовательных миров унивитаций. Они служат также в разных качествах на самой Эдентии, главным образом, в связи с программой созвездия, имеющей отношение к благоденствию групп, рас, наций и планет. Всевышние не слишком много занимаются содействием продвижению индивидуумов в обитаемых мирах; они правят, скорее, в царствах людей, чем в сердцах индивидуумов.
\vs p043 9:5 И в тот день, когда ты будешь готов отправиться из Эдентии в спасоградский путь, ты остановишься и оглянешься на одну из самых прекрасных и живительных эпох своего обучения по эту сторону Рая. Но великолепие всего этого увеличивается, когда восходишь внутрь и достигаешь большей способности шире понимать божественные значения и духовные ценности.
\vsetoff
\vs p043 9:6 [Под покровительством Малаватии Мелхиседека.]
