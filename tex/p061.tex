\upaper{61}{Эра млекопитающих на Урантии}
\vs p061 0:1 Эра млекопитающих простирается от появления плацентарных млекопитающих до конца ледникового периода, занимая немногим менее пятидесяти миллионов лет.
\vs p061 0:2 Во время Кайнозоя мировой рельеф был привлекателен --- покатые холмы, обширные долины, широкие реки и огромные леса. Дважды за этот отрезок времени поднимался и опускался Панамский перешеек; трижды то же происходило с сухопутным мостом через Берингов пролив. Было много разнообразных видов животных. Деревья были усыпаны птицами и, несмотря на продолжающуюся борьбу развивающихся видов животных за превосходство, весь мир был раем животных.
\vs p061 0:3 В накопившихся за пять периодов этой пятидесятимиллионной эры отложениях содержатся окаменелые записи последовательных династий млекопитающих, ведущие через века к появлению самого человека.
\usection{1.\bibnobreakspace Новая стадия континентальной суши\\Эпоха ранних млекопитающих}
\vs p061 1:1 50\,000\,000 лет назад массы суши по всему свету были в целом выше уровня воды или лишь слегка затоплены. Существуют как наземные, так и морские формации и отложения этого периода, но главным образом наземные. В течение длительного времени суша постепенно поднималась, но одновременно с этим и смывалась в более низкие места и в море.
\vs p061 1:2 В начале этого периода в Северной Америке \bibemph{внезапно} появился плацентарный тип млекопитающих, и они стали самым важным эволюционным достижением за все предшествующее время. Существовали отряды неплацентарных млекопитающих, но эта новая группа появилась непосредственно и \bibemph{внезапно} от существовавшего ранее рептильного предка, потомки которого выжили во времена вымирания динозавров. Прародителем плацентарных млекопитающих был небольшой, высоко активный, плотоядный, прыгающий вид динозавра.
\vs p061 1:3 Основные инстинкты млекопитающих начали проявляться в этих примитивных типах млекопитающих. Млекопитающие обладают огромным преимуществом в выживании по сравнению со всеми остальными формами животной жизни, поскольку:
\vs p061 1:4 \ublistelem{1.}\bibnobreakspace Приносят сравнительно взрослое и хорошо развитое потомство.
\vs p061 1:5 \ublistelem{2.}\bibnobreakspace Любовно кормят, заботятся и оберегают свое потомство.
\vs p061 1:6 \P\ \ublistelem{3.}\bibnobreakspace Прекрасно развитый мозг обусловливает их способность к самосохранению.
\vs p061 1:7 \ublistelem{4.}\bibnobreakspace Используют возросшую подвижность, чтобы убегать от врагов.
\vs p061 1:8 \ublistelem{5.}\bibnobreakspace Благодаря превосходному интеллекту хорошо адаптируются и приспособливаются к окружающей среде.
\vs p061 1:9 \P\ 45\,000\,000 лет назад были подняты континентальные хребты одновременно с общим затоплением берегов. Быстро развивались млекопитающие. Процветал небольшой рептильный яйцекладущий вид млекопитающих, и предки более поздних кенгуру странствовали по Австралии. Вскоре появились небольшие лошади, быстроногие носороги, тапиры с хоботом, примитивные свиньи, белки, лемуры, опоссумы и несколько племен обезьяноподобных животных. Все они были небольшими, примитивными и наиболее приспособленными к жизни в лесах горных районов. Появилась крупная страусоподобная нелетающая птица, достигающая десяти футов в высоту и откладывающая яйца размером девять на тринадцать дюймов. Это были предки более поздних гигантских пассажирских птиц, которые были настолько умны, что когда\hyp{}то переносили человеческие существа по воздуху.
\vs p061 1:10 Млекопитающие раннего Кайнозоя жили на суше, под водой, в воздухе и в кронах деревьев. У них было от одной до одиннадцати пар млечных желез, и все они были покрыты развитой шерстью. Так же, как у позднее появившихся отрядов, у них возникли два сменяющих друг друга комплекта зубов и они обладали крупным по сравнению с размерами тела мозгом. Но среди них не было современных форм.
\vs p061 1:11 \P\ 40\,000\,000 лет назад массы суши северного полушария начали подниматься, и это сопровождалось новыми значительными отложениями и очередной наземной активностью, включая разливы лавы, деформации, образования озер и эрозию.
\vs p061 1:12 В последний период этой эпохи большая часть Европы была затоплена. Вслед за небольшим подъемом континент покрылся озерами и заливами. Арктический океан через Уральскую впадину потек к югу и соединился со Средиземным морем, которое в то время распространялось к северу; горные хребты Альп, Карпат, Аппенин и Пиренеев оказались над водой, как острова в море. Панамский перешеек был над водой; Атлантический и Тихий океаны были разделены. Северная Америка была связана с Азией сухопутным мостом через Берингов пролив, а с Европой --- через Гренландию и Исландию. Сухопутный пояс в северных широтах прерывался только Уральским проливом, который соединял арктические моря с увеличившимся Средиземным морем.
\vs p061 1:13 Существенные пласты фораминиферовых известняков были отложены в европейских водах. Сегодня эти породы подняты на высоту 10\,000 футов в Альпах, 16\,000 футов в Гималаях и 20\,000 футов на Тибете. Меловые отложения этого периода находят вдоль берегов Африки и Австралии, на западном побережье Южной Америки и в Вест\hyp{}Индии.
\vs p061 1:14 \P\ В течение этого так называемого \bibemph{Эоценового} периода эволюция млекопитающих и других родственных форм жизни продолжалась почти без помех и перебоев. Северная Америка была соединена сушей со всеми континентами за исключением Австралии, и мир постепенно заселился примитивной фауной --- млекопитающими различных видов.
\usection{2.\bibnobreakspace Стадия недавнего потопа\\Эпоха продвинутых млекопитающих}
\vs p061 2:1 Этот период характеризовался дальнейшей быстрой эволюцией плацентарных млекопитающих --- более прогрессивных форм млекопитающих, развившихся в эти времена.
\vs p061 2:2 Хотя ранние плацентарные млекопитающие появились от плотоядных предков, очень скоро возникли растительноядные ветви и наконец распространились всеядные семейства млекопитающих. Покрытосеменные растения, современная сухопутная флора, которая включает большинство ныне существующих растений и деревьев, появившихся в более ранние периоды, были основной пищей быстро развивающихся млекопитающих.
\vs p061 2:3 \P\ 35\,000\,000 лет назад началась эпоха доминирования плацентарных млекопитающих. Обширный южный сухопутный мост воссоединял необъятный Антарктический континент с Южной Америкой, южной Африкой и Австралией. Несмотря на огромные массы суши в высоких широтах, мировой климат оставался сравнительно мягким благодаря очень большому размеру тропических морей и достаточно низко расположенной суше, что не позволяло образовываться ледникам. Обширные лавовые выбросы происходили в Гренландии и Исландии; какое\hyp{}то количество угля было отложено между этими слоями.
\vs p061 2:4 Заметные изменения происходили в фауне планеты. Морская жизнь подвергалась огромной модификации; большинство современных отрядов морской жизни уже существовало, и важную роль продолжали играть фораминиферы. Насекомые были очень похожи на таковых из предыдущей эры. Ископаемые слои Флориссан из Колорадо принадлежат к последним годам этих отдаленных времен. Большая часть семейств современных насекомых восходит к этому периоду; но многие из существовавших тогда, к настоящему времени вымерли, хотя сохранились их окаменелости.
\vs p061 2:5 На суше это была выдающаяся эпоха реноваций и распространения млекопитающих. Из ранних и более примитивных млекопитающих больше сотни видов вымерло до окончания этого периода. Даже крупные млекопитающие с маленьким мозгом вскоре исчезли. Мозг и подвижность заменили панцирь и размеры в борьбе за выживание животных. И по мере того, как динозавры приходили в упадок, млекопитающие медленно добивались доминирования на суше, быстро и полностью уничтожая оставшихся рептильных предков.
\vs p061 2:6 Вместе с исчезновением динозавров происходили и другие значительные изменения в различных ветвях группы ящер. Выжившими представителями ранних семейств рептилий являются черепахи, змеи и крокодилы, вместе с древней почтенной лягушкой; это единственная группа, представляющая ранних предков человека.
\vs p061 2:7 Различные группы млекопитающих происходят от уникального, вымершего в настоящее время животного. Это плотоядное создание было нечто средним между кошкой и тюленем; оно могло жить и на суше, и в воде, было высоко развитым и очень активным. В Европе появился предок семейства собачьих, от которого вскоре произошло много видов маленьких собак. Примерно в это же время появились и быстро стали заметной формой жизни постоянно грызущие грызуны: бобры, белки, землеройки, мыши и кролики; с тех пор в этом семействе произошли незначительные изменения. Более поздние отложения этого периода содержат окаменелые остатки предковых форм собак, котов, енотов, и ласок.
\vs p061 2:8 \P\ 30\,000\,000 лет назад начали появляться современные типы млекопитающих. Первоначально млекопитающие, будучи горными типами, жили главным образом на холмах; \bibemph{неожиданно} началась эволюцию равнинного, или копытного типа, пасущихся видов, которые дифференцировались от плотоядных с когтями. Эти пасущиеся пошли от единого предка, имеющего пять пальцев и сорок четыре зуба, который вымер до окончания этой эпохи. Во время этого периода эволюция пальцев не пошла далее трехпалой стадии.
\vs p061 2:9 Лошадь, выдающийся пример эволюции, в эти времена обитала как в Северной Америке, так и в Европе, хотя ее эволюция не была полностью завершена вплоть до позднего ледникового периода. Хотя семейство носорогов появилось к окончанию этого периода, его наибольшее распространение произошло впоследствии. Также сформировалось небольшое свиноподобное существо, которое стало предком многих видов свиней, пекари и гиппопотамов. В Северной Америке примерно в середине этого периода возникли и заселили западные равнины верблюды и ламы. Позднее ламы мигрировали в Южную Америку, верблюды в Европу, а в Северной Америке они вскоре вымерли, хотя некоторые верблюды обитали там вплоть до ледникового периода.
\vs p061 2:10 Примерно в это же время произошло знаменательное событие в западной части Северной Америки: впервые появились ранние предки древних лемуров. Хотя это семейство и не может считаться настоящими лемурами, их появление обозначило возникновение ветви, от которой позднее образовались настоящие лемуры.
\vs p061 2:11 Как и сухопутные змеи предыдущего периода, которые ушли в моря, теперь целое племя плацентарных млекопитающих покинуло сушу и нашло свое место в океанах. И с тех пор они оставались в море и дали начало современным китам, дельфинам, бутылконосам, тюленям и морским львам.
\vs p061 2:12 Продолжали развиваться птицы планеты, но с некоторыми важными эволюционными изменениями. Существовало большинство современных птиц, включая чаек, цапель, фламинго, канюков, соколов, орлов, перепелов и страусов.
\vs p061 2:13 \P\ К окончанию \bibemph{Олигоцена,} занимающего десять миллионов лет, растительная жизнь, наряду с морской жизнью, и сухопутными животными, чрезвычайно эволюционировала и была представлена на земле почти так же, как в наши дни. Позднее произошло значительное разделение, но предковые формы большинства живых существ и растительности тогда еще существовали.
\usection{3.\bibnobreakspace Стадия современных гор\\Эпоха слона и лошади}
\vs p061 3:1 Поднятие суши и разделение морей медленно изменяли погоду по всему свету, постепенно охлаждая его, но климат все еще был мягким. Секвойи и магнолии росли в Гренландии, но субтропические растения начали мигрировать к югу. К концу этого периода растения и деревья теплого климата в основном исчезли из северных широт, их место было занято более выносливыми растениями и листопадными деревьями.
\vs p061 3:2 Значительно увеличилось разнообразие трав, и зубы многих видов млекопитающих постепенно изменились, приспосабливаясь к современному типу пастбищ.
\vs p061 3:3 \P\ 25\,000\,000 лет назад произошло небольшое затопление суши, которое последовало за долгой эпохой ее поднятия. Регион Скалистых гор оставался высоко поднятым, так что отложение эрозионного вещества продолжалось на всех предгорьях к востоку. Сьерры были значительно подняты вторично; по сути с того времени они продолжают подниматься. Огромный четырехмильный вертикальный провал в Калифорнийском регионе сформировался в это время.
\vs p061 3:4 \P\ 20\,000\,000 лет назад был по\hyp{}настоящему золотой век млекопитающих. Сухопутный перешеек через Берингов пролив был над водой, и многие группы животных мигрировали в Северную Америку из Азии, включая мастодонтов с четырьмя бивнями, коротконогих носорогов и многие разновидности семейства кошачьих.
\vs p061 3:5 Появился первый олень, и вскоре Северную Америку заполонили жвачные --- олени, быки, верблюды, бизоны и несколько видов носорогов --- но гигантские свиньи, имевшие более шести футов в высоту, вымерли.
\vs p061 3:6 Колоссальные слоны этого и последующего периодов имели крупный мозг, большие тела, и вскоре они заселили весь мир, за исключением Австралии. Наконец в мире стали доминировать огромные животные с мозгом, достаточно крупным, чтобы уметь приспосабливаться. Конкурируя с высокоразвитой жизнью этих времен, ни одно животное размером со слона не смогло бы выжить, если бы не имело мозга крупного размера и высочайшего качества. По интеллекту и способности к адаптации только лошадь может сравниться со слоном, и только человек превзошел его. Но даже несмотря на это, из существовавших к началу этого периода пятидесяти видов слонов выжило только два.
\vs p061 3:7 \P\ 15\,000\,000 лет назад горные регионы Евразии поднимались и в этих районах происходила незначительная вулканическая активность, не сравнимая с потоками лавы в западном полушарии. Такие нестабильные условия наблюдались по всему миру.
\vs p061 3:8 Закрылся Гибралтарский пролив, и Перинеи были соединены с Африкой древним сухопутным мостом, но Средиземное море соединялось с Атлантикой узким каналом, проходившим через Францию; горные вершины и нагорья возвышались, как острова, над этим древним морем. Затем эти европейские моря стали отступать. Немногим позже Средиземное море соединилось с Индийским океаном, а когда к окончанию этого периода Суэцкий регион поднялся, Средиземное море на некоторое время стало внутренним соленым морем.
\vs p061 3:9 Сухопутный мост Исландии погрузился, и арктические воды смешались с водами Атлантического океана. Атлантическое побережье Северной Америки быстро остывало, но Тихоокеанское побережье оставалось более теплым, чем в настоящее время. Великие океанские течения воздействовали на климат так же, как и сейчас.
\vs p061 3:10 Продолжали развиваться млекопитающие. Неисчислимые табуны лошадей присоединились к верблюдам на западных равнинах Северной Америки; это действительно была эпоха лошадей, равно как и слонов. Мозг лошади второй по совершенству после мозга слона, только в одном отношении он безусловно уступает: лошадь никогда не смогла полностью преодолеть глубинный инстинкт, заставляющий ее ускакать при угрозе. В отличие от слонов лошадь лишена эмоционального контроля, но возможности слонов сильно ограничены размерами и малоподвижностью. В этот период возникло животное, которое в чем\hyp{}то было и слоном, и лошадью, но оно вскоре было уничтожено быстро развивающимся семейством кошачьих.
\vs p061 3:11 \P\ Сейчас, когда Урантия вступает в так называемую <<безлошадную эпоху>>, остановитесь и задумайтесь, что это животное значило для ваших предков. Люди вначале использовали лошадей в пищу, потом для передвижения, а позднее в сельском хозяйстве и на войне. Лошадь долгое время служила человечеству и играла важную роль в развитии человеческой цивилизации.
\vs p061 3:12 \P\ Биологические реалии этого периода много значили для подготовки последующего появления человека. В центральной Азии развились современные виды примитивных обезьян и горилл, имевшие общего предка, вымершего к настоящему времени. Но ни один из этих видов не относится к ветви живых существ, которые позднее стали предками человеческой расы.
\vs p061 3:13 Семейство собачьих было представлено несколькими группами, в частности волками и лисами; племя кошачьих --- пантерами и саблезубыми тиграми, последние впервые появились в Северной Америке. Численность современных видов семейства кошачьих и собачьих увеличивалась по всему свету. В северных широтах процветали и развивались ласки, куницы, выдры и еноты.
\vs p061 3:14 Продолжали эволюционировать птицы, хотя произошло всего несколько существенных изменений. Рептилии были близки к современным типам --- змеям, крокодилам и черепахам.
\vs p061 3:15 \P\ Так подошел к концу интересный и очень насыщенный событиями период мировой истории. Эта эпоха слона и лошади известна как \bibemph{Миоцен.}
\usection{4.\bibnobreakspace Недавняя стадия подъема континентов\\Последняя великая миграция млекопитающих}
\vs p061 4:1 Это период доледникового поднятия суши в Северной Америке, Европе и Азии. Сильно изменилась топография суши. Возникали горные цепи, потоки изменяли свои русла, и по всему миру извергались одиночные вулканы.
\vs p061 4:2 \P\ 10\,000\,000 лет назад началась эпоха широко распространенных локальных наземных отложений в низинах континентов, но значительная часть этих отложений позднее была уничтожена Большая часть Европы, включая часть Англии, Бельгию и Францию, в это время по\hyp{}прежнему была под водой, а Средиземное море покрыло большую часть северной Африки. В Северной Америке обширные отложения образовывались у оснований гор, в озерах и в огромных водоемах на суше. Эти отложения в среднем примерно около двухсот футов толщиной, они более или менее окрашены и в них редки окаменелости. На западе Северной Америки находились два огромных пресноводных озера. Поднимались Сьерры, Шаста, Худ и Рейниер начинали формировать свою горную систему. Но до последующего ледникового периода Северная Америка не начала своего сползания в сторону Атлантического понижения.
\vs p061 4:3 В течение короткого времени вся суша мира, исключая Австралию, опять соединилась, и произошла последняя великая всемирная миграция животных. Северная Америка была соединена и с Южной Америкой, и с Азией, и между ними происходил свободный взаимообмен формами животной жизни. Азиатские ленивцы, броненосцы, антилопы и медведи проникли в Северную Америку, тогда как североамериканские верблюды перешли в Китай. Носороги мигрировали по всему свету, за исключением Австралии и Южной Америки, но в западном полушарии к концу этого периода они вымерли.
\vs p061 4:4 В целом, жизнь предшествующего периода продолжала развиваться и распространяться. Семейство кошачьих доминировало в животной жизни, а морская жизнь находилась практически в застое. Многие из лошадей были еще трехпалыми, но появлялись современные типы; ламы и жирафоподобные верблюды сосуществовали с лошадьми на равнинных пастбищах. В Африке появился жираф, который имел такую же длинную шею, как и сейчас. В Южной Америке развились ленивцы, броненосцы, муравьеды и южноамериканский тип примитивных обезьян. До того как континенты были окончательно изолированы, мастодонты, эти массивные животные, мигрировали повсюду, за исключением Австралии.
\vs p061 4:5 \P\ 5\,000\,000 лет назад лошадь стала такой же, как сейчас, и из Северной Америки мигрировала по всему миру. Но на континенте, где она зародилась задолго до появления краснокожего человека, лошадь вымерла.
\vs p061 4:6 Климат постепенно становился холоднее; наземные растения медленно смещались к югу. Усиливающийся холод на севере в первую очередь остановил миграцию животных по северным перешейкам (по --- если миграция шла через перешеек, на --- если миграция происходила на самом перешейке); впоследствии эти североамериканские сухопутные мосты ушли под воду. Вскоре сухопутные перешейки между Африкой и Южной Америкой в конце концов были затоплены, и материки западного полушария оказались изолированными почти так же, как и сейчас. Начиная с этого времени в восточном и западном полушариях начали развиваться особые типы жизни.
\vs p061 4:7 \P\ Так подошел к концу этот период протяженностью почти в десять миллионов лет, но предок человека пока еще не появился. Это время обычно обозначается как \bibemph{Плиоцен.}
\usection{5.\bibnobreakspace Ранний ледниковый период}
\vs p061 5:1 К окончанию предшествующего периода земли северо\hyp{}восточной части Северной Америки и северной Европы на огромном протяжении были высоко подняты, в Северной Америке обширные области вздымались на 30\,000 футов и более. Над северными регионами ранее преобладал мягкий климат, и арктические воды были полностью открыты для испарения и оставались свободными ото льда почти до конца ледникового периода.
\vs p061 5:2 Одновременно с этими поднятиями суши переместились океанские течения и изменили свое направление сезонные ветры. Такие условия в конечном итоге привели к практически постоянному выпадению осадков над северными нагорьями из сильно насыщенной влагой, находящейся в движении, атмосферы. На эти приподнятые, и потому холодные, участки начал падать снег, и он продолжал падать, пока его покров не достиг в толщину 20\,000 футов. Высоко расположенные области с наибольшей толщиной снега определили центры районов последующего растекания ледников. И ледниковый период продолжался ровно столько, сколько эти обильные осадки продолжали покрывать северные нагорья огромной мантией снега, который вскоре превратился в твердый, но подвижный лед.
\vs p061 5:3 Огромные ледяные щиты этого периода все были расположены на возвышенных нагорьях, а не в гористых местах, где их находят сегодня. Половина ледников была в Северной Америке, одна четверть в Евразии и одна четверть --- во всех остальных регионах, в основном в Антарктике. Африка лишь слегка подверглась воздействию льда, но Австралия была почти полностью покрыта антарктическим ледовым одеялом.
\vs p061 5:4 Северные регионы мира испытали шесть отдельных и отчетливых наступлений льда, хотя было множество локальных наступлений и отступлений, связанных с активностью каждого отдельного ледяного щита. Лед в Северной Америке сконцентрировался в двух, а позднее в трех центрах. Гренландия была покрыта, а Исландия была полностью погребена под потоками льда. В Европе лед в разное время покрывал Британские острова, за исключением побережья южной Англии, и распространился на западную Европу до Франции.
\vs p061 5:5 \P\ 2\,000\,000 лет назад первый североамериканский ледник начал свое южное наступление. Ледниковый период находился в процессе становления, и этому леднику потребовался почти миллион лет для наступления и отступления назад, к северным центрам образования. Центральный ледяной щит распространился к югу до Канзаса; восточный и западный ледовые центры не были тогда такими обширными.
\vs p061 5:6 \P\ 1\,500\,000 лет назад первые гигантские ледники отступали к северу. В то же время огромное количество снега падало на Гренландию и северо\hyp{}восточную часть Северной Америки, и вскоре эта восточная ледовая масса начала дрейфовать к югу. Это было второе наступление льда.
\vs p061 5:7 В Евразии эти два первых вторжения льда не были обширными. Во время ранних эпох ледникового периода в Северной Америке обитали мастодонты, шерстистые мамонты, лошади, верблюды, олени, мускусные быки, бизоны, наземные ленивцы, гигантские бобры, саблезубые тигры, ленивцы, огромные как слоны, и многие группы семейств кошачьих и псовых. Но, начиная с того времени, их число быстро сокращалось из\hyp{}за усиливающихся холодов. К концу ледникового периода большинство видов этих животных в Северной Америке вымерло.
\vs p061 5:8 Вдали от льда сухопутная и водная жизнь в мире изменилась мало. Между наступлениями льда климат был почти таким же мягким, что и сейчас, возможно немного теплее. Ледники, в конце концов, были локальными явлениями, хотя они и распространились и покрыли необъятные просторы. Прибрежный климат сильно менялся между временами ледникового спокойствия и теми временами, когда чудовищные айсберги соскальзывали с берегов Мена в Атлантику, проникали через пролив Пюже в Тихий океан и с грохотом вламывались в Норвежские фьорды в Северном море.
\usection{6.\bibnobreakspace Примитивный человек в ледниковый период}
\vs p061 6:1 Великим событием в ледниковый период было развитие примитивного человека. Немного к западу от Индии, на суше, которая сейчас находится под водой, среди потомков азиатских мигрантов более древних североамериканских типов лемуров \bibemph{неожиданно} появились ранние млекопитающие. Эти небольшие животные в основном ходили на задних ногах и у них относительно их размеров и в сравнении с мозгом других животных был крупный мозг. В семнадцатом поколении этого отряда жизни \bibemph{неожиданно} выделилась новая и более высоко организованная группа животных. Эти новые срединные млекопитающие, почти вдвое превосходившие по размеру и росту своих предков и обладающие пропорционально возросшими умственными способностями, только прижились, когда \bibemph{неожиданно} появились Приматы --- третья жизненно важная мутация. (В то же время, ретроградное развитие ветви срединных млекопитающих положило начало предкам обезьян; и с этого момента ветвь людей, стремительно эволюционируя, двигалась вперед, тогда как племена обезьян оставались неизменными или в конечном счете регрессировали.)
\vs p061 6:2 \P\ 1\,000\,000 лет назад Урантия была зарегистрирована как \bibemph{населенный мир.} В результате мутации среди ветви прогрессирующих приматов \bibemph{неожиданно} появились два типа примитивных человеческих существ --- истинных предков человечества.
\vs p061 6:3 Это событие произошло приблизительно во время третьего ледникового наступления; таким образом, ваши ранние предки были рождены и выращены в стимулирующей, бодрящей и тяжелой окружающей среде. И единственные сохранившиеся потомки этих аборигенов Урантии, эскимосы, даже сейчас предпочитают жить в холодных северных краях.
\vs p061 6:4 \P\ В западном полушарии почти до окончания ледникового периода человеческих существ не было. Но во время межледниковых эпох они двинулись к западу вокруг Средиземноморья и вскоре заняли континент Европы. В пещерах западной Европы можно найти человеческие кости, перемешанные с останками как тропических, так и арктических животных, что свидетельствует о том, что человек жил в этих регионах в более поздние эпохи наступающих и отступающих ледников.
\usection{7.\bibnobreakspace Продолжающийся ледниковый период}
\vs p061 7:1 Во время ледникового периода происходили и другие явления, но в северных широтах самым активным и раиболее мощным было воздействие льда. Никакие другие явления на суше не оставили такого характерного отпечатка на топографии. Одиночные валуны и поверхностные неоднородности, такие как рытвины, озера, сдвинутые камни и скальная крошка не связаны ни с какими другими явлениями в природе. Лед также является причиной таких небольших возвышений, или поверхностных изгибов, известных как друмлины. И ледник, когда он наступает, передвигает реки и изменяет весь облик земли. Только ледники оставляют за собой эти характерные ледниковые наносы --- поддонные, боковые и конечные морены. Эти ледниковые наносы, в частности поддонные морены, простираются от восточного побережья к северу и западу в Северной Америке, и их находят в Европе и Сибири.
\vs p061 7:2 \P\ 750\,000 лет назад четвертый ледяной щит, объединение североамериканского центрального и восточного ледяного поля, проник далеко на юг; в наибольшем развитии он достиг южного Иллинойса, передвинув реку Миссисипи на пятьдесят миль к западу, а в восточной части он простирался на юг до реки Огайо и центральной Пенсильвании.
\vs p061 7:3 В Азии сибирский ледяной щит продвинулся на юг дальше, чем когда\hyp{}либо, тогда как в Европе наступающий лед остановился вблизи горного барьера Альп.
\vs p061 7:4 \P\ 500\,000 лет назад, во время пятого наступления льда, новое событие ускорило ход развития человеческой эволюции. \bibemph{Внезапно} и в течение одного поколения от исходной человеческой ветви мутировало шесть цветных рас. Это вдвойне важная дата, поскольку она соответствует дате прибытия Планетарного Принца.
\vs p061 7:5 В Северной Америке наступление пятого ледника представляло собой объединенное вторжение всех трех центров оледенения. Однако восточный рукав распространился только немного ниже долины Святого Лаврентия, а западный ледяной щит лишь незначительно продвинулся на юг. Но центральный поток устремился к югу, покрыв большую часть штата Айова. В Европе это вторжение льда не было таким обширным, как предшествующее.
\vs p061 7:6 \P\ 250\,000 лет назад началось шестое и последнее оледенение. И несмотря на то, что северные нагорья стали слегка погружаться, это был период наибольшего отложения снега на северных ледяных полях.
\vs p061 7:7 Во время этого вторжения три огромных ледяных щита срослись в одну обширную ледяную массу, и в этой ледниковой активности принимали участие все западные горы. Это было самым большим ледяным вторжением в Северную Америку; лед продвинулся к югу на тысячу пятьсот миль от своих центров сосредоточения, и Северная Америка подверглась воздействию самой низкой для себя температуры.
\vs p061 7:8 \P\ 200\,000 лет назад, во время наступления последнего ледника, произошел эпизод, который много значил в ряду событий на Урантии, --- бунт Люцифера.
\vs p061 7:9 \P\ 150\,000 лет назад шестое, и последнее, оледенение достигло крайней точки в своем распространении на юг, западный ледяной щит даже пересек канадскую границу; центральный продвинулся вниз до Канзаса, Миссури и Иллинойса; восточный щит наступал на юг и покрыл большую часть Пенсильвании и Огайо.
\vs p061 7:10 Из этого ледника выделилось множество языков, или ледяных рукавов, которые прорезали ложа современных озер, больших и малых. Во время его отступления возникла североамериканская система Великих озер. И геологи Урантии очень тщательно исследовали различные стадии этого формирования и правильно определили, что эти водоемы в различное время сбрасывали воды вначале в долину Миссисипи, затем на восток в долину Гудзона и, наконец, в северном направлении в залив Святого Лаврентия. Тридцать семь тысяч лет назад система Великих озер начала сбрасывать воду по современному руслу Ниагары.
\vs p061 7:11 \P\ 100\,000 лет назад, во время отступления последнего ледника, начали образовываться обширные полярные ледовые щиты, и центры сосредоточения льда существенно сместились к северу. И до тех пор, пока полярные районы остаются покрытыми льдом, едва ли возможен другой ледниковый период, независимо от будущих поднятий суши или изменений параметров океанских течений.
\vs p061 7:12 Последний ледник наступал в течение ста тысяч лет, и ему понадобилось примерно такое же время для завершения отступления к северу. Регионы умеренного климата остаются свободными ото льда немногим более пятидесяти тысяч лет.
\vs p061 7:13 Суровый ледниковый период уничтожил многие виды животных и радикально изменил множество других. Многие были жестко отсеяны миграциями туда и обратно, которые были обусловлены наступающими и отступающими ледниками. Животные, которые следовали за ледниками туда и обратно, --- это медведь, бизон, северный олень, мускусный бык, мамонт и мастодонт.
\vs p061 7:14 Мамонты искали открытые прерии, а мастодонты предпочитали защищенные опушки лесистых регионов. Мамонты, вплоть до своего вымирания, были распространены от Мексики до Канады; сибирская разновидность покрылась шерстью. Мастодонт сохранился в Северной Америке до тех пор, пока не был уничтожен краснокожим человеком, так же как белокожий человек позднее истребил бизона.
\vs p061 7:15 В Северной Америке во время последнего оледенения вымерли лошадь, тапир, лама и саблезубый тигр. На их место из Южной Америки пришли ленивцы, броненосцы и водосвинки.
\vs p061 7:16 Вынужденная миграция жизни перед наступающим льдом привела к необычному смешению растений и животных, и с отступлением льда последнего оледенения многие арктические виды как растений, так и животных, остались обитать высоко в горах, куда они перебрались, чтобы избежать уничтожения ледником. Таким образом, и сейчас эти перемещенные растения и животные встречаются высоко в Альпах в Европе и даже на Аппалачских горах в Северной Америке.
\vs p061 7:17 \P\ Ледниковый период --- это последний завершившийся геологический период, так называемый \bibemph{Плейстоцен,} его продолжительность более двух миллионов лет.
\vs p061 7:18 \P\ 35\,000 лет назад великий ледниковый период окончился всюду, за исключением полярных регионов планеты. Эта дата также важна, поскольку примерно в это же время прибыли Материальный Сын и Дочь и началась Адамическая диспенсация, что по времени примерно соответствует началу Голоцена, или постледникового периода.
\vs p061 7:19 \P\ Это повествование, от начала жизни млекопитающих до отступления льда и до наступления исторических времен, охватывает период почти в пятьдесят миллионов лет. Этот последний, современный, геологический период известен вашим исследователям как \bibemph{Кайнозой,} или новейшая эра.
\vs p061 7:20 [Представлено Постоянно пребывающим Носителем Жизни.]
