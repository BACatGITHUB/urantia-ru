\upaper{161}{Дальнейшие дискуссии с Роданом}
\author{Комиссия срединников}
\vs p161 0:1 В воскресенье 25 сентября 29 года н.э. апостолы и евангелисты собрались в Магадане. В тот вечер после долгой беседы со своими соратниками Иисус удивил всех, объявив, что на следующий день утром он и двенадцать апостолов отправятся в Иерусалим на праздник кущей. Иисус велел евангелистам навестить верующих в Галилее, а женскому отряду --- на время вернуться в Вифсаиду.
\vs p161 0:2 Когда пришло время отправляться в Иерусалим, Нафанаил и Фома еще продолжали дискуссии с Роданом Александрийским и поэтому Учитель разрешил им задержаться в Магадане на несколько дней. Пока Иисус и десять апостолов шли в Иерусалим, Фома и Нафанаил продолжали вести жаркие споры с Роданом. На предыдущей неделе, когда Родан излагал свою философию, Фома и Нафанаил, сменяя друг друга, рассказывали греческому философу о евангелии царства. И Родан понял, что один из бывших апостолов Иоанна Крестителя, который был его учителем в Александрии, достаточно хорошо преподал ему учение Иисуса.
\usection{1. Личность Бога}
\vs p161 1:1 Был один вопрос, который Родан и оба апостола толковали по\hyp{}разному: вопрос этот касался личности Бога. Родан охотно согласился со всем, что касалось атрибутов Бога, но утверждал, что Отец Небесный не является, не может быть личностью в том смысле, в каком ее понимает человек. Хотя апостолы затруднялись доказать, что Бог --- личность, Родану доказать, что он --- не личность, было еще труднее.
\vs p161 1:2 Родан утверждал, что сущность личности состоит в совокупном факте полного и взаимного общения равных существ, которые способны к сочувственному пониманию. Родан сказал: «Чтобы быть личностью, Бог должен располагать символами духовного общения, которые позволили бы ему быть полностью понятым теми, кто с ним общается. Однако поскольку Бог бесконечен и вечен, так как он --- Творец всех остальных существ, это означает, что с точки зрения равенства существ, Бог во вселенной --- один. Нет никого, равного ему, и никого, с кем он мог бы общаться как равный. Бог действительно может быть источником личности в целом, однако как таковой он превосходит личность, так же как Творец --- выше и больше создания».
\vs p161 1:3 Это утверждение сильно озадачило Фому и Нафанаила, и они попросили Иисуса помочь им, но Учитель отказался участвовать в их дискуссиях. Иисус сказал Фоме: «Не очень важно, какую \bibemph{идею} об Отце вы разделяете, важно, чтобы вы были духовно знакомы с \bibemph{идеалом} его бесконечной и вечной природы».
\vs p161 1:4 Фома заявил, что Бог общается с человеком и, следовательно, Отец --- это личность даже в рамках определения Родана. Грек же отверг это утверждение на том основании, что Бог лично сам себя не открывает; что он по\hyp{}прежнему тайна. Тогда Нафанаил сослался на свой собственный опыт общения с Богом, и с ним Родан согласился, подтвердив, что подобные переживания он недавно испытывал сам, однако, заявил он, эти переживания доказали только \bibemph{реальность} Бога, а не существование его \bibemph{личности.}
\vs p161 1:5 В понедельник к ночи Фома сдался. Однако к ночи во вторник Нафанаил убедил Родана, заставив его поверить в личность Отца, и добился такой перемены взглядов грека следующими рассуждениями:
\vs p161 1:6 \ublistelem{1.}\bibnobreakspace Отец в Раю наслаждается равноправным общением по крайней мере с двумя существами, полностью равными ему самому и совершенно ему подобными, а именно: с Вечным Сыном и Бесконечным Духом. В свете учения о Троице, грек был вынужден признать возможность обладания личностью Отцом Всего Сущего. (Последующие обсуждения этих дискуссий расширили представления двенадцати апостолов о Троице. Разумеется, все верили, что Иисус --- Вечный Сын).
\vs p161 1:7 \pc \ublistelem{2.}\bibnobreakspace Поскольку Иисус был равен Отцу и так как сей Сын для своих земных детей свершил проявление личности, подобное явление служило доказательством факта и подтверждало возможность обладания личностью всеми тремя лицами Бога, а также навсегда решило вопрос в отношение способности Бога общаться с человеком и возможности человека общаться с Богом.
\vs p161 1:8 \pc \ublistelem{3.}\bibnobreakspace Что Иисус был взаимосвязан и ничем не ограничен в общении с человеком; что Иисус был Сыном Бога. Что родство Сына и Отца предполагает равенство в общении и взаимность сочувственного понимания; что Иисус и Отец суть одно. Что Иисус одновременно поддерживал полное понимания общение и с Богом, и с человеком и что поскольку и Бог и человек воспринимали значения символов, которыми Иисус пользовался при общении, то и Бог, и человек обладали атрибутами личности, необходимыми для взаимного общения. Что личность Иисуса наглядно показывала личность Бога и вместе с тем убедительно доказывала присутствие Бога в человеке. Что два факта, связанные с одним и тем же третьим фактом, связаны и друг с другом.
\vs p161 1:9 \pc \ublistelem{4.}\bibnobreakspace Что личность для человека есть высшее представление человеческой сущности и божественных ценностей; что Бог для человека есть высшее представление божественной сущности и бесконечных ценностей; что, следовательно, Бог должен быть божественной и бесконечной личностью, личностью реальной, личностью, хотя и бесконечно и вечно превосходящей человеческое представление и определение личности, тем не менее всегда и всемирно личностью.
\vs p161 1:10 \pc \ublistelem{5.}\bibnobreakspace Что Бог должен быть личностью, поскольку он --- Творец всякой личности и предназначения. На Родана огромное влияние оказало поучение Иисуса: «Будьте совершенны, как совершен Отец ваш небесный».
\vs p161 1:11 \pc Выслушав эти доводы, Родан сказал: «Вы меня убедили. Я открыто признаю Бога личностью, если вы позволите в моем определении такой веры расширить понятие личности и отнести к ней такие ценности, как сверхчеловеческое, трансцендентное, верховное, бесконечное, вечное, окончательное и всемирное. Теперь я убежден: хотя Бог должен быть бесконечно больше личности, он не может быть в чем\hyp{}то меньшим ее. Я рад завершить спор и признать Иисуса личным откровением Отца и решением всех нерешенных проблем логики, разума и философии».
\usection{2. Божественная природа Иисуса}
\vs p161 2:1 После того, как Нафанаил и Фома однозначно одобрили воззрения Родана на евангелие царства, осталось рассмотреть еще только один вопрос, а именно: учение о божественной сущности Иисуса, доктрину, обнародованную сравнительно недавно. Нафанаил и Фома совместно представили свою точку зрения на божественную природу Учителя, и нижеследующее повествование представляет собой обобщенное, систематизированное и вновь представленное изложение их учения:
\vs p161 2:2 \ublistelem{1.}\bibnobreakspace Иисус признал свою божественность, и мы верим ему. Его служение сопровождалось многими удивительными явлениями, которые мы можем понять, лишь веря, что он --- Сын Божий и Сын Человеческий.
\vs p161 2:3 \pc \ublistelem{2.}\bibnobreakspace Его живое общение с нами являет собой идеал человеческой дружбы; только божественное существо может быть таким другом человеку. Он не только самая истинно бескорыстная личность, которую мы когда\hyp{}либо знали. Он друг даже грешникам и не боится любить своих врагов. Он очень предан нам. И даже когда он без колебаний порицает нас, всем ясно, что в действительности он нас любит. Чем больше его знаешь, тем больше любишь его. Вас очаровывает его непоколебимая преданность. На протяжении всех лет, пока нам не удавалось осознать его миссию, он был верным другом. Он не льстит, он одинаково добр со всеми нами; он неизменно нежен и сострадателен. Свою жизнь и все остальное он делит с нами. Мы --- счастливая община; у нас все общее. Мы не верим, что простой смертный смог бы жить такой безупречной жизнью в столь сложных обстоятельствах.
\vs p161 2:4 \pc \ublistelem{3.}\bibnobreakspace Мы считаем Иисуса божественным, потому что он никогда не поступает неправильно и никогда не ошибается. Его мудрость необычайна, а благочестие --- превосходно. День за днем он живет в точном соответствие с волей Отца. Он никогда не кается в проступках, потому что не нарушает ни одного из законов Отца. Он молится о нас и вместе с нами, но никогда не просит молиться о нем. Мы верим, что он совершенно безгрешен. Мы не думаем, что тот, кто есть лишь человек, когда\hyp{}либо мог бы жить такой жизнью. Иисус заявляет, что живет совершенной жизнью, и мы признаем, что так оно и есть. Наше благочестие происходит из покаяния, его же --- из праведности. Он даже открыто утверждает, что прощает грехи, и, действительно, исцеляет болезни. Ни один обычный человек в здравом уме не будет претендовать на прощение грехов; это --- божественная прерогатива. И таким совершенным в своей праведности он казался уже с момента нашей первой встречи с ним. Мы возрастаем в благодати и познании истины; наш же Учитель проявляет совершенную праведность с самого начала. Все люди, и добрые и злые, признают эти стороны добродетели в Иисусе. И все же его благочестие никогда не бывает навязчивым или показным. Он и кроток и бесстрашен. Кажется, он одобряет нашу веру в его божественность. Он --- либо то, чем открыто признает себя, либо величайший лицемер и мошенник, какого когда\hyp{}либо знал мир. Мы же убеждены: он --- именно то, чем себя называет.
\vs p161 2:5 \pc \ublistelem{4.}\bibnobreakspace Уникальность его характера и способность в полной мере владеть своими чувствами убеждают нас, что он являет собой сочетание человеческого и божественного. Он неизменно отзывается на человеческую нужду, когда видит ее, и страдание никогда не оставляет его равнодушным. Он одинаково сострадает и физической боли, и душевной муке, и духовной печали. Он быстро замечает и великодушно признает присутствие веры или же любой другой добродетели у своих собратьев\hyp{}людей. Он так справедлив и честен и одновременно так милосерден и внимателен. Он огорчается от духовного упрямства людей и радуется, когда они соглашаются увидеть свет истины.
\vs p161 2:6 \pc \ublistelem{5.}\bibnobreakspace Кажется, что он знает помышления человеческих умов и понимает желания человеческих сердец. И всегда сочувствует нашему обеспокоенному духу. Кажется, он обладает всеми чувствами, только они у него величественно прекрасны. Он чрезвычайно любит добродетель и в равной степени ненавидит грех. Он обладает сверхчеловеческим сознанием присутствия Божества. Он молится, как человек, но ведет себя, как Бог. Кажется, он все предвидит, и даже сейчас не боится говорить о своей смерти, загадочно намекая о своем грядущем прославлении. Будучи добр, он в то же время смел и отважен. Исполняя свой долг, он никогда не колеблется.
\vs p161 2:7 \pc \ublistelem{6.}\bibnobreakspace Нас постоянно поражает феномен его сверхчеловеческой осведомленности. И дня не проходит без того, чтобы не выяснилось, что Учитель знает, что происходит вне его непосредственного присутствия. Кажется, он также знает о мыслях своих сподвижников. У него, бесспорно, есть общение с небесными личностями; несомненно, его духовный уровень намного более высок, нежели у каждого из нас. Все кажется открыто его уникальному пониманию. Задавая нам вопросы, он стремится вовлечь нас в беседу, а не получить сведения.
\vs p161 2:8 \pc \ublistelem{7.}\bibnobreakspace Последнее время Учитель не боится заявлять о своей сверхчеловеческой сущности. С момента посвящения нас в апостолы и до самого последнего времени он ни разу не отрицал, что пришел от Отца свыше. Он говорит с властностью божественного учителя. Учитель не боится опровергать современные религиозные учения и провозглашать евангелие как власть имеющий. Он всегда настойчив, позитивен и властен. Даже Иоанн Креститель, и тот, услышав, как говорит Иисус, объявил его Сыном Бога. Он кажется самодостаточным. Не ищет поддержки масс и безразличен к мнению людей. Он смел и вместе с тем лишен гордыни.
\vs p161 2:9 \pc \ublistelem{8.}\bibnobreakspace Он постоянно говорит о Боге как о вечно присутствующем союзнике во всем, что он делает. Он идет, творя добро, ибо Бог, кажется, пребывает в нем. Он делает поразительнейшие заявления о самом себе и своей миссии на земле --- высказывания, которые были бы абсурдными, если бы он не был божественен. Однажды он заявил: «Прежде нежели был Авраам, я есть». Он определенно заявил о своей божественности и открыто объявляет о своем сотрудничестве с Богом. Многократно повторяя о своей тесной связи с Отцом Небесным, он практически исчерпывает возможности языка. Он не боится даже утверждать, что он и Отец --- одно. Он говорит, что всякий, кто видел его, видел и Отца. Причем говорит и делает все эти потрясающие вещи с детской непосредственностью. О своей связи с Отцом он упоминает точно так же, как говорит о своей связи с нами. Он производит впечатление что прекрасно знает Бога и говорит об этих отношениях, как как о само собой разумеющемся.
\vs p161 2:10 \pc \ublistelem{9.}\bibnobreakspace Похоже, что в своих молитвах он обращается непосредственно к Отцу. Мы слышали несколько его молитв, но и они указывают на то, что он говорит с Богом словно лицом к лицу. Кажется, что он знает будущее так же, как прошлое. Он просто не мог бы быть всем этим и делать эти удивительные дела, если бы не был кем\hyp{}то большим, нежели человек. Мы знаем, что он человек, и в этом уверены, но мы почти так же уверены в том, что он и божественен. Мы верим, что он божественен. Мы убеждены, что он Сын Человеческий и Сын Божий.
\vs p161 2:11 \pc Завершив свои беседы с Роданом, Нафанаил и Фома поспешили в Иерусалим, чтобы присоединиться к своим собратьям\hyp{}апостолам, и прибыли в город в пятницу той же недели. Эти трое верующих обрели огромный жизненный опыт, и остальные апостолы многому научились, слушая воспоминания Нафанаила и Фомы об этих событиях.
\vs p161 2:12 Родан вернулся в Александрию, где долго преподавал свою философию в школе Меганты. В дальнейших делах царства небесного он стал могущественным человеком и до конца своих земных дней был истинным верующим и в числе других завершил свою жизнь в Греции во времена, когда преследования достигли своего пика.
\usection{3. Человеческий и божественный разум Иисуса}
\vs p161 3:1 Сознание собственной божественности постепенно возрастало в Иисусе вплоть до дня его крещения. Видимо, осознав свою божественную сущность, предчеловеческое существование и вселенские исключительные права, он обрел способность различным образом ограничивать свое человеческое сознание собственной божественности. Нам представляется, что от крещения до распятия Иисус мог исключительно по своему выбору либо полагаться только на человеческий разум, либо пользоваться знаниями и человеческого и божественного разума. Порою казалось, что он использовал только ту информацию, что была доступна человеческому интеллекту. В других же случаях казалось, что он действует, опираясь на всю полноту знания и мудрости, какой может обладать только сверхчеловеческая сущность его божественного сознания.
\vs p161 3:2 Его уникальные свершения мы можем понять, лишь согласившись с теорией, согласно которой он при желании мог сам ограничивать сознание своей божественности. Мы четко сознаем, что он часто скрывал свое предвидение событий от своих соратников и что он знал, о чем они думают и что планируют. Мы понимаем, что он не хотел, чтобы его последователи в полной мере сознавали, что он способен распознавать их мысли и проникать в их замыслы. У него не было желания слишком сильно превосходить представление о человеческом, каким оно было в умах его апостолов и учеников.
\vs p161 3:3 Мы совершенно неспособны установить различие между его способом самоограничения своего божественного сознания и его методом сокрытия от своих человеческих соратников своей способности предвидеть и проникать в чужие мысли. Мы убеждены, что он прибегал и к тому, и к другому, но не всегда в состоянии определить, каким из них он воспользовался в том или ином конкретном случае. Мы часто наблюдали, как он действует, обращаясь лишь к человеческой составляющей своего сознания; а потом наблюдали его беседующим с предводителями небесных воинств вселенной и замечали бесспорное действие ума божественного. Мы также бессчетное число раз были свидетелями деятельности этой объединенной личности человека и Бога, казалось бы, движимой совершенным союзом человеческого и божественного разума. На этом наши знания о данном явлении исчерпываются; мы, действительно, по\hyp{}настоящему не знаем суть этой тайны.
