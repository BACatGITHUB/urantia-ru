\upaper{80}{Продвижение Андитов на Запад}
\vs p080 0:1 Хотя европейский голубой человек сам по себе не создал великой культурной цивилизации, он все\hyp{}таки обладал биологической основой, которая, когда его Адамизованные племена смешались с более поздними Андитами\hyp{}завоевателями, произвела самый сильный и жизнестойкий из всех родов, когда\hyp{}либо появлявшихся на Урантии со времен фиолетовой расы и ее Андических наследников, который смог создать агрессивную цивилизацию.
\vs p080 0:2 Современные белые народы сочетают в себе уцелевшие наследственные свойства Адамического рода, которые смешались с расами Сангика, с некоторой частью красной и желтой и, особенно, с голубой расами. У всех белых рас присутствует значительный процент Андонической крови, но процент наследственных черт древних Нодитов у них все\hyp{}таки больше.
\usection{1.\bibnobreakspace Адамиты вступают в Европу}
\vs p080 1:1 До того, как Андиты были окончательно вытеснены из долины Евфрата, многие их собратья уже вступили в Европу --- это были искатели приключений, учителя, торговцы и воины. В первые дни фиолетовой расы Средиземноморская впадина была ограничена Гибралтарским перешейком и разделена Сицилийской перемычкой. По этим внутренним озерам человек проложил одни из самых древних морских торговых путей, где голубые люди с севера и обитатели Сахары с юга встречались с Нодитами и Адамитами с востока.
\vs p080 1:2 Вокруг восточной впадины Средиземноморья Нодиты основали одни из наиболее обширных своих культурных центров, и из них они проникли отчасти в южную Европу, но, главным образом, в северную Африку. Широколицые Нодито\hyp{}Андонические сирийцы очень рано начали заниматься гончарным производством и земледелием в поселениях, построенных в медленно поднимающейся дельте Нила. Они ввезли овец, коз, крупный рогатый скот и других домашних животных, ввели в употребление значительно усовершенствованные методы обработки металлов, причем центром такого производства тогда была Сирия.
\vs p080 1:3 В течение более тридцати тысяч лет в Египет шел постоянный приток жителей Месопотамии, которые несли с собой свое искусство и культуру, обогащая искусство и культуру долины Нила. Но большой приток представителей народов Сахары оказал пагубное влияние на цивилизацию на Ниле, так что приблизительно пятнадцать тысяч лет назад культура Египта находилась на самом низком уровне.
\vs p080 1:4 Однако в более ранние времена мало что препятствовало миграции Адамитов на запад. Сахара представляла собой открытые пастбища, заселенные пастухами и земледельцами. Эти обитатели Сахары никогда не занимались ремеслами и не строили города. Они принадлежали к сине\hyp{}черной группе, которая несла в себе множество наследственных черт вымерших зеленой и оранжевой рас. Но они получили ничтожную долю фиолетовой наследственности до того, как произошел подъем суши и изменилось направление влажных ветров --- муссонов, что окончательно уничтожило остатки этой процветающей и мирной цивилизации.
\vs p080 1:5 Кровь Адама была поделена между большинством человеческих рас, но одни получили большую долю, чем другие. Смешанные народы Индии и более темнокожее население Африки не привлекали Адамитов. Они легко могли бы смешаться с красной расой, если бы те не находились так далеко в Америках; они были расположены к желтой расе, но и к ней в Дальний Восток, доступ был также затруднен. Поэтому и в том случае, когда ими двигала страсть к странствиям и альтруизм, и когда их вытесняли из долины Евфрата, они, вполне естественно, выбирали союз с голубыми расами Европы.
\vs p080 1:6 У голубых людей, господствовавших тогда в Европе, не было религиозных обычаев, которые бы отталкивали прибывших ранее Адамитов, к тому же фиолетовая и голубая расы испытывали друг к другу сильное сексуальное влечение. Лучшие из голубых людей почитали за высокую честь получить разрешение на брак с Адамитами. Каждый мужчина голубой расы горел желанием стать особенно искусным и артистичным, чтобы стать достойным любви женщины из рода Адамитов, и самым сильным стремлением лучших женщин голубой расы было обратить на себя внимание Адамита.
\vs p080 1:7 Постепенно эти мигрирующие сыны Эдема соединялись с представителями высших типов голубой расы, укрепляя их культурные традиции и одновременно безжалостно истребляя оставшиеся племена неандертальцев. Этот путь смешения рас, соединенный с уничтожением носителей низших наследственных черт, привел к появлению больше дюжины групп мужественных и развитых лучших голубых людей, одну из которых вы назвали кроманьонцами.
\vs p080 1:8 Вследствие и этих, и других причин, среди которых не последнюю роль играли более благоприятные пути миграции, волны древней культуры Месопотамии устремились почти исключительно в Европу. И именно эти факторы и определили предпосылки современной европейской цивилизации.
\usection{2.\bibnobreakspace Климатические и геологические изменения}
\vs p080 2:1 Раннее продвижение фиолетовой расы в Европу было внезапно остановлено довольно неожиданными климатическими и геологическими изменениями. С отступлением северных ледников влажные ветры сместились с запада на север, постепенно превратив обширные области открытых пастбищ Сахары в бесплодную пустыню. Эта засуха привела к рассеянию малорослых брюнетов, темноглазых, но узколицых обитателей великого плоскогорья Сахары.
\vs p080 2:2 Относительно чистокровные представители синей расы двинулись на юг, в леса центральной Африки, там они с тех пор и остались. Менее чистокровные группы распространялись по трем направлениям: развитые западные племена двинулись в Испанию, а оттуда --- в прилегающие части Европы, образуя ядро будущих средиземноморских рас узколицых брюнетов. Наименее развитая группа с востока Сахары мигрировала в Аравию, а оттуда --- через северную Месопотамию и Индию --- на далекий Цейлон. Центральная группа отправилась на север и на восток --- в долину Нила и в Палестину.
\vs p080 2:3 Именно влияние этого вторичного сангикского исхода и возволяет говорить об определенной степени сходства между современными народами, разбросанными от Деканского плоскогорья по Ирану и Месопотамии и вдоль обоих побережий Средиземного моря.
\vs p080 2:4 \P\ Приблизительно во время этих климатических изменений в Африке Британские острова отделилась от континента, побережье Дании поднялось из моря, а перешеек Гибралтара, закрывающий западный бассейн Средиземноморья, в результате землетрясения опустился, и уровень этого внутреннего озера поднялся до уровня Атлантического океана. Вскоре Сицилийская перемычка ушла под воду и образовалось единое Средиземное море, соединенное с Атлантическим океаном. Этот природный катаклизм затопил многочисленные поселения и вызвал самые большие во всей мировой истории жертвы из\hyp{}за наводнений.
\vs p080 2:5 Это затопление Средиземноморского бассейна сразу же ограничило передвижение Адамитов на запад, в то же время большой приток обитателей Сахары заставил все большее число Адамитов уходить на север и на восток от Эдема. Так как потомки Адама шли из долин Тигра и Евфрата на север, то путь им преграждали горы и Каспийское море, в те времена более обширное. В течение многих поколений Адамиты пасли скот, охотились и возделывали землю вокруг своих поселений, разбросанных по всему Туркестану. Мало\hyp{}помалу этот прекрасный народ расширил свою территорию, вступив в Европу. Но теперь Адамиты вошли в Европу с востока и обнаружили, что культура голубого человека на тысячи лет отстает от азиатской, поскольку эта область не имела почти никаких связей с Месопотамией.
\usection{3.\bibnobreakspace Кроманьонский голубой человек}
\vs p080 3:1 Древние культурные центры голубого человека были расположены вдоль всех рек Европы, но сегодня только Сомма течет по тому же руслу, что и в доледниковый период.
\vs p080 3:2 Хотя мы говорим о голубом человеке как о народе, заполонившем европейский континент, расовых типов существовало множество. Даже тридцать пять тысяч лет назад Европейские голубые расы уже представляли собой смешанные народы, которые несли в себе наследственные черты красных и желтых людей, на атлантическом побережье и в областях современной России они впитали значительную долю Андонической крови, а на юге они соединялись с народами Сахары. Но было бы бесполезно даже пытаться перечислить это множество расовых групп.
\vs p080 3:3 Европейская цивилизация раннего пост\hyp{}Адамического периода представляла собой уникальную смесь пассионарности и искусства голубого человека с творческим воображением Адамитов. Голубые люди были расой чрезвычайно энергичной, но они значительно снизили культурный и духовный уровень Адамитов. Последним было очень трудно насаждать свою религию среди кроманьонцев из\hyp{}за того, что многие из Адамитов были склонны обманывать девушек и развратничать. В течение десяти тысяч лет религия в Европе находилась в состоянии упадка по сравнению с ее развитием в Индии и Египте.
\vs p080 3:4 Во всех своих делах голубые люди были абсолютно честны и полностью лишены сексуальных пороков смешанных Адамитов. Они уважали девственность и практиковали многоженство только тогда, когда в результате войн сокращалась численность мужского населения.
\vs p080 3:5 Эти кроманьонские народы были отважной и дальновидной расой. Они придерживались эффективной системы воспитания детей. В этом деле принимали участие оба родителя, и во всем помогали старшие дети. Каждый ребенок заботливо обучался ухаживать за пещерой, обучался искусству изготовления кремневых орудий. Уже в раннем возрасте девушки были весьма сведущими в домашнем хозяйстве и примитивном сельском хозяйстве, а юноши --- искусными охотниками и отважными воинами.
\vs p080 3:6 Голубые люди были охотниками, рыбаками и собирателями съестных припасов; они превосходно строили лодки. Они делали каменные топоры, срубали деревья, сооружали бревенчатые хижины, частично уходящие под землю и покрытые крышами из шкур. Существуют народы, которые все еще строят подобные хижины в Сибири. Южные кроманьонцы жили, главным образом, в пещерах и гротах.
\vs p080 3:7 В жестокую зиму нередко случалось, что сторожа, охраняющие ночью вход в пещеру, замерзали до смерти. Они были мужественными людьми, но, прежде всего, все они были мастерами своего дела; смешение с Адамитами резко ускорило развитие творческого воображения. Вершина искусства голубого человека была достигнута приблизительно пятнадцать тысяч лет назад, еще до прихода более темнокожих рас из Африки через Испанию на север.
\vs p080 3:8 \P\ Около пятнадцати тысяч лет назад широко распространились альпийские леса. Европейские охотники вытеснялись в долины рек и на морское побережье под воздействием тех же климатических изменений, которые превратили лучшие охотничьи угодья мира в безводные и бесплодные пустыни. Так как влажные ветры сместились на север, обширные открытые пространства пастбищ Европы стали покрываться лесами. Именно эти значительные и сравнительно резкие климатические изменения заставили расы Европы превратиться из охотников, промышляющих на открытых пространствах, в пастухов и отчасти --- в рыбаков и земледельцев.
\vs p080 3:9 Эти изменения, хотя и привели к успехам в культуре, вызвали определенные биологические ухудшения. Во время предыдущей охотничьей эры высшие племена вступали в браки с высокоразвитыми пленниками, захваченными во время войны, и неизменно убивали тех, кого они считали ниже себя. Но поскольку они начали основывать поселения и заниматься земледелием и торговлей, они стали сохранять жизнь посредственностям из числа пленников и использовать их как рабов. И впоследствии потомки именно этих рабов и привели к столь заметному вырождению всего кроманьонского типа. Этот упадок культуры продолжался до тех пор, пока не был получен новый импульс с востока, когда заключительное и массовое вторжение обитателей Месопотамии прокатилось по Европе, стремительно поглотив кроманьонский тип и культуру и положив начало цивилизации белых рас.
\usection{4.\bibnobreakspace Вторжение Андитов в Европу}
\vs p080 4:1 Хотя Андиты вливались в Европу непрерывным потоком, но было семь мощных вторжений, причем последние орды прибыли уже на конях тремя большими волнами. Одни вошли в Европу по пути, идущему через острова Эгейского моря и далее вверх по долине Дуная, но большинство более древних и более чистокровных племен устремились в северо\hyp{}западную Европу по северному пути --- через степи Волги и Дона.
\vs p080 4:2 Между третьим и четвертым вторжениями орды Андонитов проникли в Европу с севера, из Сибири путем, идущим по русским рекам и по Балтийскому морю. Они сразу же были ассимилированы северными племенами Андитов.
\vs p080 4:3 Предыдущие миграции более чистокровных родов фиолетовой расы были гораздо миролюбивее, чем вторжения их более поздних военизированных стремящихся к завоеваниям Андических потомков. Адамиты были мирными людьми, Нодиты --- воинственными. Союз этих родов, позднее смешавшись с расами Сангика, породил талантливых, агрессивных Андитов, которые и осуществили подлинные военные завоевания.
\vs p080 4:4 \P\ Фактором эволюции, который определил господство Андитов на Западе, была лошадь. Лошадь предоставила мигрирующим Андитам не существовавшее до той поры преимущество в мобильности, дав возможность последним группам всадников\hyp{}Андитов стремительно обойти вокруг Каспийского моря и распространиться по всей Европе. Все предыдущие волны Андитов двигались так медленно, что они рассеивались на относительно незначительном расстоянии от Месопотамии. Но более поздние волны двигались так быстро, что входили в Европу сплоченными массами, все еще сохраняя некоторые черты более высокой культуры.
\vs p080 4:5 В течение десяти тысяч лет перед появлением в шестом и седьмом тысячелетии до Христа стремительной конницы Андонитов весь обитаемый мир за пределами Китая и района Евфрата достиг весьма ограниченного прогресса в области культуры. По мере продвижения на восток через Русскую равнину, ассимилируя лучших людей голубой расы и уничтожая худших, смешавшись, они в конце концов стали одним народом. Это были предки так называемых Нордических рас, праотцы скандинавских, германских и англо\hyp{}саксонских народов.
\vs p080 4:6 \P\ Прошло немного времени, и по всей северной Европе высшие голубые роды были полностью поглощены Андитами. Только в Лапландии ( и отчасти --- в Бретани) более древние Андониты смогли сохранить некое подобие индивидуальности.
\usection{5.\bibnobreakspace Завоевание Северной Европы Андитами}
\vs p080 5:1 Племена северной Европы непрерывно усиливались и улучшались в результате постоянного притока мигрантов из Месопотамии через южнорусские области, прилегающие к Туркестану, и когда последние волны конницы Андитов пронеслись по Европе, в этом регионе людей с Андической наследственностью было больше, чем во всем остальном мире.
\vs p080 5:2 В течение трех тысяч лет военным центром северных Андитов была Дания. Именно отсюда распространялись дальше последовательные волны завоеваний, но, по мере того, как уходящие столетия становились свидетелями окончательного смешения завоевателей из Месопотамии с завоеванными народами, первые все больше утрачивали Андические свойства и все больше приобретали черты белой расы.
\vs p080 5:3 \P\ В то время как голубой человек был ассимилирован на севере и в, конце концов, сломлен белыми всадниками, совершавшими набеги, продвигающиеся племена смешанной белой расы, которые проникли на юг, встретили упорное и длительное сопротивление кроманьонцев, но превосходство умственных способностей и постоянно увеличивающиеся биологические резервы дали им возможность уничтожить эту более древнюю расу.
\vs p080 5:4 Решающая борьба между белым и голубым человеком завершилась битвами в долине Соммы. Здесь цвет голубой расы упорно сопротивлялся идущим на юг Андитам, но прежде чем окончательно проиграть превосходящей военной стратегии белых захватчиков, эти кроманьонцы больше пятисот лет успешно защищали свои территории. Тор, победоносный командующий армиями севера в последней битве на Сомме, стал героем северных белых племен и впоследствии некоторые из этих племен почитали его как бога.
\vs p080 5:5 \P\ Укрепления голубого человека, которые продержались дольше всего, находились на юге Франции, но последнее мощное военное сопротивление было сломлено на Сомме. Дальнейшее завоевание имело успех благодаря проникновению торговли, перенаселенности районов, расположенных вдоль рек, бракам с высшими представителями голубой расы и безжалостному уничтожению низших.
\vs p080 5:6 Когда племенной совет старейшин Андитов признавал пленника из низших негодным, тот согласно хитроумному ритуалу передавался священникам\hyp{}шаманам, которые отводили его к реке и совершали над ним обряды посвящения <<угодьям удачной охоты>>, то есть топили. Так белые завоеватели Европы уничтожали всех попавшихся на пути людей, которые не смогли быстро им уподобиться, и так --- очень скоро --- голубой человек прекратил свое существование.
\vs p080 5:7 \P\ Кроманьонский голубой человек заложил биологическую основу современных европейских рас, но кроманьонцы не сохранились, а были ассимилированы более поздними и более сильными завоевателями их родной земли. Голубой род передал белым расам Европы много положительных качеств и большую физическую силу, но чувство юмора и воображение смешанным европейским народам досталось от Андитов. Союз Андитов с голубыми людьми, в результате которого образовались северные белые расы, сразу же привел к упадку Андической цивилизации, регрессу, который носил временный характер. Со временем обнаружилось скрытое превосходство этих северных варваров и достигло своей кульминации в нынешней европейской цивилизации.
\vs p080 5:8 \P\ К 5000 году до н.э. эволюционирующие белые расы господствовали во всей северной Европе, включая северную Германию, северную Францию и Британские острова. Центральная Европа в течение еще какого\hyp{}то времени контролировалась голубыми людьми и круглоголовыми Андонитами. Последние обитали, главным образом, в долине Дуная и никогда не были полностью вытеснены оттуда Андитами.
\usection{6.\bibnobreakspace Андиты на берегах Нила}
\vs p080 6:1 После окончательной миграции Андитов культура в долине Евфрата постепенно пришла в упадок, и очередной центр цивилизации находился теперь в долине Нила. Египет стал преемником Месопотамии в качестве центра самой прогрессивной на земле группы.
\vs p080 6:2 Долина Нила начала страдать от наводнений несколько раньше долин Месопотамии, но справлялась с ними много лучше. Ранняя задержка в развитии с лихвой компенсировалась продолжающимся притоком иммигрантов\hyp{}Андитов, так что культура Египта, в действительности обязанная своим происхождением региону Евфрата, казалось, двигалась вперед. Но в 5000 году до н.э., во время периода наводнений в Месопотамии, в Египте существовало семь четко выраженных народностей все они, за исключением одной, пришли из Месопотамии.
\vs p080 6:3 \P\ Когда произошел последний исход из долины Евфрата, Египту повезло, ибо он получил очень много искусных художников и ремесленников. Эти ремесленники\hyp{}Андиты обнаружили, что находятся в абсолютно привычной для себя обстановке, поскольку они детально знали жизнь реки, ее разливы, ирригацию и засушливые сезоны. Им нравилось защищенное положение долины Нила; там они меньше подвергались враждебным набегам и нападениям, чем на берегах Евфрата. И они значительно обогатили искусство египтян в обработке металлов. Они работали с железной рудой, поступающей с горы Синай, заменив ею руды из районов Черного моря.
\vs p080 6:4 \P\ Египтяне очень давно создали сложную национальную систему богов, собрав воедино божества отдельных городов. Они разработали обширную теологию и имели столь же многочисленное, хотя и обременительное духовенство. Отдельные вожди пытались оживить останки древних религиозных учений Сифитов, но эти попытки оказывались недолговечными. Андиты построили первые каменные сооружения в Египте. Первая и наиболее изошренно сконструированная из каменных пирамид была воздвигнута Имхотепом, гениальным архитектором\hyp{}Андитом в то время, когда он был премьер\hyp{}министром. Предыдущие сооружения были сделаны из кирпича, и хотя в разных частях света уже были воздвигнуты многие каменные сооружения, в Египте это было первое. Но после этого великого архитектора строительное искусство Египта приходило в упадок.
\vs p080 6:5 Блестящая эпоха культуры была прервана междоусобной войной на берегах Нила, и вскоре страна была, как раньше Месопотамия, наводнена низшими племенами из негостеприимной Аравии, и черными с юга. В результате, в течение более пятисот лет уровень социального развития все более понижался.
\usection{7.\bibnobreakspace Андиты на островах Средиземноморья}
\vs p080 7:1 В период упадка культуры в Месопотамии на островах восточного Средиземноморья в течение некоторого времени продолжала существовать высшая цивилизация.
\vs p080 7:2 Около 12\,000 года до н.э. одно блестящее племя Андитов переселилось на Крит. Это был единственный остров, на котором так рано обосновалась столь развитая группа, и только почти через две тысячи лет потомки этих моряков колонизировали соседние острова. Эту группу составляли узколицые малорослые Андиты, которые породнились с Ванитами, выходцами из северных Нодитов. Все они были ниже шести футов ростом и их буквально вытеснили с материка более рослые менее развитые собратья. Эти переселенцы на Крит были весьма искусны в ткачестве, обработке металлов, гончарном деле, устройстве водопроводов и в использовании камня как строительного материала. У них была письменность, и они занимались скотоводством и земледелием.
\vs p080 7:3 Почти через две тысячи лет после колонизации Крита группа высокорослых потомков Адама\hyp{}сына, отправившись из своего родного нагорья почти прямо на север от Месопотамии, достигла, пройдя по северным островам, Греции. Этих предковгреков вел на запад Сато, прямой потомок Адама\hyp{}сына и Ратты.
\vs p080 7:4 Эта группа, которая, в конце концов, обосновалась в Греции, состояла из трехсот семидесяти пяти наиболее развитых людей, последних представителей второй цивилизации потомков Адама\hyp{}сына. Эти более поздние сыны Адама\hyp{}сына несли в себе самые ценные на тот момент наследственные свойства нарождающихся белых рас. Они отличались высоким интеллектуальным уровнем и с точки зрения физического развития, это были самые красивые люди со времен первого Эдема.
\vs p080 7:5 \P\ Вскоре Греция и регион островов Эгейского моря заняли место Месопотамии и Египта в качестве западного центра торговли, искусства и культуры. Но точно так же, как это было и в Египте, так и здесь вся культура и искусство Эгейского мира уходили корнями в Месопотамию, за исключением культуры тех потомков Адама\hyp{}сына, которые стали прародителями греков. Все искусство, весь гений этого более позднего народа есть прямое наследство потомков Адама\hyp{}сына, первого сына Адама и Евы, и его замечательной второй жены, которая происходила по прямой линии из чистокровных Нодитов штата Калигастии. Не удивительно, что греки, согласно их мифологическим преданиям, считали себя прямыми потомками богов и сверхчеловеческих существ.
\vs p080 7:6 Регион Эгейского моря прошел через пять четких культурных этапов, каждый из которых отличался меньшей духовностью, чем предыдущий, и вскоре последняя великолепная эра искусства погибла под бременем быстро возрастающего числа потомков заурядных дунайских рабов, ввезенных поздними поколениями греков.
\vs p080 7:7 Именно в это же время на Крите особенно широко распространился \bibemph{культ матери,} которого придерживались потомки Каина. Этот культ прославлял Еву в богослужениях <<великой матери>>. Изображения Евы были повсюду. По всему Криту и Малой Азии были воздвигнуты тысячи алтарей. И культ матери просуществовал вплоть до эпохи Христа, причем позднее он был включен в раннехристианскую религию в виде прославления и поклонения Марии, земной матери Иисуса.
\vs p080 7:8 \P\ Примерно в 6500 году до н.э. духовное наследие Андитов пришло в глубокий упадок. Потомки Адама, рассеянные по всему свету, были, в конце концов, поглощены более древними и более многочисленными человеческими расами. И этот упадок Андической цивилизации вместе с исчезновением их религиозных канонов привел духовно обедненные расы мира в плачевное состояние.
\vs p080 7:9 \P\ К 5000 году до н.э. три самых чистокровных рода потомков Адама обитали в Шумере, северной Европе и Греции. По всей Месопотамии медленно шел процесс вырождения в результате наплыва смешанных и более темнокожих людей, которые проникали туда из Аравии. И приход низших народов способствовал дальнейшему распылению биологического и культурного наследия Андитов за пределами Месопотамии. Со всей этой плодородной местности более предприимчивые народы двинулись на запад, к островам. Эти мигранты выращивали зерно и овощи и вели с собой домашних животных.
\vs p080 7:10 Около 5000 года до н.э. громадная масса развитых обитателей Месопотамии двинулась из долины Евфрата и обосновалась на Кипре; приблизительно через две тысячи лет эта цивилизация была уничтожена ордами северных варваров.
\vs p080 7:11 \P\ Другая крупная колония обосновалась на берегу Средиземного моря, неподалеку от будущего Карфагена. И из северной Африки большое число Андитов проникло в Испанию, и они впоследствии смешались в Швейцарии со своими собратьями, которые еще раньше пришли в Италию с островов Эгейского моря.
\vs p080 7:12 \P\ Когда культура Египта вслед за культурой Месопотамии стала приходить в упадок, многие из более талантливых и развитых родов бежали на Крит, где внесли значительный вклад в эту уже в то время передовую, цивилизацию. А когда позднее приход из Египта низших групп стал угрожать цивилизации Крита, более развитые в культурном отношении роды перебрались на запад, в Грецию.
\vs p080 7:13 \P\ Греки были не только великими учителями и художниками, но и величайшими в мире торговцами и колонизаторами. Прежде, чем исчезнуть в потоке невежества и неполноценности, который, в конце концов, поглотил их искусство и торговлю, им удалось создать на западе так много очагов культуры, что великое множество достижений древнегреческой цивилизации продолжало существовать в среде более поздних народов южной Европы, а многие смешанные потомки этих потомков Адама\hyp{}сына вошли в состав племен, обитающих на близлежащих больших материках.
\usection{8.\bibnobreakspace Дунайские Андониты}
\vs p080 8:1 Андические народы долины Евфрата мигрировали не только на север, в Европу, где смешивались с голубыми людьми, но и на запад, в районы Средиземноморья, где скрещивались с остатками смешанных обитателей Сахары и южными голубыми людьми. И эти две ветви белой расы были (и остаются в настоящее время) отдалены друг от друга на значительное расстояние ареалом широколицых горцев, уцелевших остатков древних Андонических племен, которые долгое время населяли центральные области Европы.
\vs p080 8:2 Эти потомки Андона были рассеяны в большинстве горных областей центральной и юго\hyp{}восточной Европы. Их часто усиливали прибывающие мигранты из Малой Азии, области, которую они захватили и где были их многочисленные поселения. Древние хетты --- непосредственные потомки Андонического рода; их бледная кожа и широкие лица типичны для этой расы. Этой наследственной чертой обладали предки Авраама, и она в значительной степени определила характерный тип лица их более поздних еврейских потомков, которые, несмотря на то, что их культура и религия происходили от Андитов, говорили совсем на другом языке. Их язык был, определенно, Андоническим.
\vs p080 8:3 Племена, которые жили в домах, построенных на сваях или бревенчатых причалах на озерах Италии, Швейцарии и южной Европы, составляли расширяющуюся периферию африканских, эгейских и, особенно, дунайских миграций.
\vs p080 8:4 Дунайские племена были Андонитами --- земледельцами и пастухами, они проникли в Европу через Балканский полуостров, медленно продвигаясь на север по долине Дуная. Они занимались гончарным ремеслом и земледелием, предпочитая жить в долинах. Самые северные поселения дунайских племен находились в Бельгии, в районе Льежа. Эти племена быстро вырождались по мере удаления от центра и источника своей культуры. Самые лучшие образцы гончарного искусства являются продуктом более древних поселений.
\vs p080 8:5 Дунайские племена в результате деятельности миссионеров с Крита стали поклонниками культа матери. Эти племена позднее слились с группами моряков\hyp{}Андонитов, почитателей культа метери, приплывших на судах с побережья Малой Азии. Таким образом, большое пространство центральной Европы рано заселили представители этих смешанных типов широколицых белых рас, исповедовавших культ матери и следовавших религиозному ритуалу кремации умерших, поскольку в обычае поклонников культа матери было сжигать своих мертвых в каменных сооружениях.
\usection{9.\bibnobreakspace Три белые расы}
\vs p080 9:1 К концу миграций Андитов расовые смешения в Европе начинают оформляться в следующие три белые расы:
\vs p080 9:2 \ublistelem{1.}\bibnobreakspace \bibemph{Северная белая раса.} Эта так называемая Нордическая раса состояла преимущественно из голубых людей и Андитов, но содержала также значительное количество Андонической крови и в меньшей степени крови красных и желтых народов Сангика. Таким образом, северная белая раса наследовала свойства этих четырех самых лучших человеческих родов. Но больше всего в ней свойств от голубой расы. Типичный древний Нордический человек был узколицым, высоким и светловолосым. Но уже давным\hyp{}давно эта раса полностью смешалась со всеми другими ветвями рода белых людей.
\vs p080 9:3 Первобытная культура, с которой столкнулись Нордические захватчики, была культурой вырождающихся дунайских племен, смешавшихся с голубыми людьми. Нордическо\hyp{}Датская и Дунайско\hyp{}Андоническая культура встретились на Рейне и смешались друг с другом, что подтверждается существованием в современной Германии двух расовых групп.
\vs p080 9:4 Нордические люди продолжали торговать янтарем с побережья Балтийского моря, установив через перевал Бреннера (в Альпах) интенсивные торговые связи с широколицыми из долины Дуная. Этот длительный контакт с дунайскими племенами привел северян к почитанию культа матери, и в течение нескольких тысяч лет кремация мертвых была почти повсеместно распространена в Скандинавии. Это объясняет, почему не обнаружены останки более древних белых рас, несмотря на то, что они были захоронены по всей Европе, --- находят только пепел в каменных и глиняных урнах. Строили эти белые люди и дома: они никогда не жили в пещерах. И опять это объясняет, почему существует так мало свидетельств древней культуры белого человека, тогда как предшествующий кроманьонский тип хорошо сохранился, будучи надежно замурован в пещерах и гротах. Такое впечатления, что в северной Европе однажды возникла первобытная культура вымирающих дунайских племен и голубого человека, а сразу вслед за ней --- культура внезапно появившегося и значительно превосходящего их белого человека.
\vs p080 9:5 \P\ \ublistelem{2.}\bibnobreakspace \bibemph{Центральная белая раса.} Эта группа является преимущественно Андонической, хотя включает кровь и голубых, и желтых, и Андических родов. Это широколицые, смуглые и коренастые люди. Их поселения вклинились между Нордическими и Средиземноморскими народами, причем широкое основание клина находится в Азии, а вершина упирается в восточную Францию.
\vs p080 9:6 В течение более двадцати тысяч лет Андониты вытеснялись Андитами все дальше и дальше на север от центральной Азии. К 3000 году до н.э. увеличивающаяся засуха начинает гнать Андонитов обратно в Туркестан. Это движение Андонитов на юг продолжается более тысячи лет, и, разделившись при обходе Черного и Каспийского морей, они проникают в Европу двумя путями --- через Украину и через Балканы. В этом вторжении участвовали оставшиеся группы потомков Адама\hyp{}сына, а во время второй половины периода вторжения --- и значительное число иранских Андитов, а также многих потомков священников Сифитов.
\vs p080 9:7 К 2500 году до н.э. Андониты в своем продвижении на запад достигли Европы. Варвары туркестанских нагорий заполонили всю Месопотамию, Малую Азию и бассейн Дуная, что и вызвало наиболее серьезный и продолжительный регресс в развитии культуры вплоть до сегодняшнего времени. Эти захватчики, определенно, андонизировали наследственность центральноевропейских рас, сформировав так называемый альпийский тип.
\vs p080 9:8 \P\ \ublistelem{3.}\bibnobreakspace \bibemph{Южная белая раса.} Эта темноволосая средиземноморская раса --- результат смешения Андического и голубого человека, с меньшей долей Андонической крови, чем на севере. Через обитателей Сахары эта группа приобрела также значительную долю крови вторичных народов Сангика. Позднее в эту южную ветвь белой расы влилась кровь сильных Андических элементов от жителей восточного Средиземноморья.
\vs p080 9:9 Однако Андиты редко встречались на побережье Средиземного моря, пока в 2500 году до н.э не настало время великого вторжения кочевников. В течение столетий, когда кочевники вторгались в области восточного Средиземноморья, передвижение и торговля по суше были практически прекращены. Это вызвало бурное развитие морских перевозок и морской торговли; около четырех с половиной тысяч лет назад средиземноморская морская торговля процветала. Следствием расширения морских перевозок стало стремительное распространение потомков Андитов по всей территории бассейна Средиземноморья.
\vs p080 9:10 Эти смешения рас положили начало южноевропейской расы, самой пестрой из всех европейских рас. И с той поры эта раса продолжала подвергаться дальнейшему смешению, особенно с желто\hyp{}голубыми Андическими народами Аравии. В действительности, эта средиземноморская раса столь свободно смешивалась с окружающими народами, что, в сущности, ее невозможно выделить в отдельный тип, но, в общем, ее представители --- это низкорослые, длиннолицые, темноволосые люди.
\vs p080 9:11 На севере Андиты посредством браков и в ходе войн уничтожили голубых людей, но на юге те уцелели в большом количестве. Баски и берберы представляют собой две выживших ветви этой расы, но даже эти народы полностью смешались с обитателями Сахары.
\vs p080 9:12 \P\ Такова была мозаика рас в центральной Европе около 3000 года до н.э. Несмотря на частичный срыв Адама, высшие типы все\hyp{}таки смешались.
\vs p080 9:13 \P\ Это было время нового каменного века, уже переходящего в бронзовый век. В Скандинавии это был бронзовый век, связанный с поклонением культу матери. В южной Франции и Испании это был новый каменный век, связанный с поклонением солнцу. Это было время постройки круглых, не имеющих крыши храмов солнца. Европейские белые расы были энергичными строителями, которым доставляло наслаждение устанавливать большие камни как символы солнца --- то же самое делали их более поздние потомки в Стоунхендже. Популярность культа солнца в южной Европепоказывает, что это был период расцвета земледелия.
\vs p080 9:14 Суеверия этой сравнительно недавней эры поклонения солнцу существуют в народных обычаях Бретани даже в настоящее время. Эти бретонцы, несмотря на то, что были обращены в христианство свыше полутора тысяч лет назад, все еще берегут амулеты нового каменного века, охраняющие от сглаза. Они все еще держат в камине так называемые <<камни грома>> в качестве защиты от молнии. Бретонцы никогда не смешивались со скандинавскими Нордическими людьми. Они --- уцелевшие потомки жителей западной Европы, произошедших от Андонитов, смешавшихся с средиземноморским родом.
\vs p080 9:15 \P\ Но ошибочно полагать, что белые народы можно подразделить на нордические, альпийские и средиземноморские. Вообще говоря, произошло слишком сильное смешение рас, чтобы их можно было сгруппировать подобным образом. Одно время существовало достаточно четкое разделение белой расы на такие типы, но с тех пор произошло всеобъемлющее смешение этих типов между собой, и сейчас уже не представляется возможным установить их различия с какой\hyp{}либо определенностью. Даже в 3000 году до н.э. древние общественные группы представляли собой одну расу не в большей степени, чем в настоящее время таковой являются жители Северной Америки.
\vs p080 9:16 Эта европейская культура в течение пяти тысяч лет продолжала развиваться и --- до некоторой степени --- смешиваться. Но языковой барьер препятствовал полному взаимному обмену между различными западными нациями. В прошлом столетии эта культура опробовала самую благоприятную возможность гармоничного смешения на космополитическом населении Северной Америки; и будущее этого континента определят характер расовых факторов, которым будет позволено проникнуть в жизнь сегодняшнего и будущего поколений и существующий уровень социальной культуры.
\vs p080 9:17 [Представлено Архангелом Небадона.]
