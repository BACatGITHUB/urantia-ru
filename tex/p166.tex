\upaper{166}{Последнее посещение северной Переи}
\author{Комиссия срединников}
\vs p166 0:1 С 11 по 20 февраля Иисус и двенадцать апостолов совершали путешествие по тем городам и селениям северной Переи, где трудились сподвижники Авенира и члены женского отряда. Они убедились, что вестникам евангелия сопутствует успех, и Иисус многократно обращал внимание своих апостолов на то, что евангелие царства может распространяться и не сопровождаемое чудесами и необычными явлениями.
\vs p166 0:2 Миссия в Перее, продолжавшаяся три месяца, в целом была успешно выполнена с определенной помощью со стороны двенадцати апостолов, и отныне евангелие стало отражать не столько личность Иисуса, сколько суть его учения. Однако последователи Иисуса не долго следовали его наставлениям, ибо вскоре после его смерти и воскресения отошли от его учений и начали строить раннюю церковь, опираясь на предания о чудотворных деяниях Учителя и о его богочеловеческой личности.
\usection{1. Фарисеи в Рагаве}
\vs p166 1:1 В субботу 18 февраля Иисус был в Рагаве, где жил некий богатый фарисей по имени Нафанаил; а поскольку довольно много его товарищей\hyp{}фарисеев следовало по стране за Иисусом и двенадцатью апостолами, в это субботнее утро он устроил завтрак примерно для двадцати из них, и пригласил Иисуса в качестве почетного гостя.
\vs p166 1:2 К моменту прихода Иисуса на этот завтрак большинство фарисеев и два или три законника уже были там и сидели за столом. Не подходя к чашам с водой для омовения рук, Учитель сразу занял свое место за столом слева от Нафанаила. Многие из фарисеев, особенно те, что благосклонно относились к учениям Иисуса, знали, что он умывал руки только из гигиенических соображений и питал отвращение к омовениям рук как к ритуальному действию, а потому не удивились, когда он, не умыв рук, пошел прямо к столу. Нафанаила же такое несоблюдение Иисусом строгих требований фарисейского обычая потрясло. В отличие от фарисеев Иисус к тому же не умывал руки ни после каждой перемены блюд, ни по завершению трапезы.
\vs p166 1:3 После продолжительных перешептываний между Нафанаилом и недружелюбно настроенным фарисеем, соседом справа, после многозначительных взглядов и кривых усмешек сидевших напротив Учителя, Иисус в конце концов сказал: «Я полагал, что вы пригласили меня в этот дом преломить с вами хлеб и, возможно, узнать от меня о провозглашении нового евангелия царства Бога; однако вижу, что вы привели меня сюда полюбоваться на то, как вы соблюдаете свои обряды, выставляя напоказ свою непоколебимую уверенность в собственной правоте. Эту услугу вы мне уже оказали; чем еще удостоите меня как вашего гостя на встрече сей?»
\vs p166 1:4 Когда Учитель произнес это, все опустили глаза и промолчали. И так как никто ничего не говорил, Иисус продолжал: «Многие из вас, фарисеев, присутствуют здесь как мои друзья, некоторые даже считаются моими учениками, однако большинство фарисеев упорствует в своем нежелании увидеть свет и признать истину даже тогда, когда дело евангелия явлено перед ними в своей великой силе. Как тщательно вы чаши и блюда очищаете, в то время как сосуды для духовной пищи грязны и нечисты! Вы заботитесь о том, чтобы явить людям благочестивую и святую внешность, но души ваши исполнены самодовольства, жадности, вымогательства и всяческих духовных пороков. Ваши вожди смеют замышлять и даже планировать убийство Сына Человеческого. Разве не понимаете вы, неразумные, что Бог небес видит и ваше внутреннее побуждение души, и ваше внешнее притворство, и ваши якобы благочестивые заявления? Не думайте, что раздача милостыни или уплата десятины очистит вас от неправды и позволит вам предстать незапятнанными пред Судьей всех людей. Горе вам, фарисеям, упорствующим в отвержении света жизни! Вы аккуратны в уплате десятины и хвастливы в раздаче милостыни, но сознательно с презрением не признаете пришествие Бога и отвергаете откровение любви его. Хоть и хорошо, что вы уделяете внимание этим второстепенным обязанностям, вы не должны забывать об исполнении более важных требований. Горе всем, кто избегает справедливости, отказывается от милосердия и отвергает истину! Горе всем, кто презирает откровение Отца и вместе с тем ищет главные места в синагогах и жаждет льстивых приветствий на рыночных площадях!»
\vs p166 1:5 \pc Когда Иисус хотел подняться и уйти, некто из законников, сидевший за столом, обратился к нему и сказал: «Однако, Учитель, говоря это, ты обижаешь и нас. Неужели в фарисеях, книжниках или законниках нет ничего хорошего?» И Иисус, вставая, ответил законнику: «Вы, подобно фарисеям, любите возлежать на первых местах на пиршествах и ходить в длинных одеждах, а на людские плечи налагаете бремена тяжелые, неудобоносимые. А когда души людей спотыкаются под этими тяжкими бременами, вы и пальцем не пошевелите. Горе вам, что получит великое наслаждение, строя гробницы пророкам, которых убили отцы ваши! Согласие же ваше с тем, что отцы ваши сделали, явлено ныне, когда вы замышляете убить тех, кто сегодня пришли и делают то, что в свои дни пророки делали --- возвещают праведность Бога и открывают милосердие Отца Небесного. Но за все прошлые поколения кровь пророков и апостолов взыщется от рода сего, порочного и самонадеянного. Горе всем вам, законникам, что вы забрали ключи знания у простого народа! Вы сами отказываетесь встать на путь истины и одновременно препятствуете всем, кто пытается вступить на него. Но этим вам не затворить двери царства небесного; их отворили мы для всех имеющих веру войти, и сии врата милосердия не закроют предубеждение и высокомерие лжеучителей и пастырей неистинных, что подобны белоснежным склепам, которые снаружи кажутся красивыми, а внутри полны костей мертвых и всякой духовной нечистоты».
\vs p166 1:6 И закончив речь за столом Нафанаила, Иисус вышел из дома, так и не вкусив пищи. И из фарисеев, слышавших эти слова, некоторые уверовали в его учение и вошли в царство, но большинство остались на пути тьмы, еще больше укрепились в решении выждать момент и использовать что\hyp{}нибудь из сказанного им, чтобы привлечь его к суду и приговору синедриона в Иерусалиме.
\vs p166 1:7 \pc Существовало всего три вещи, которым фарисеи уделяли особое внимание.
\vs p166 1:8 \ublistelem{1.}\bibnobreakspace Строгая уплата десятины.
\vs p166 1:9 \ublistelem{2.}\bibnobreakspace Скрупулезное соблюдение законов очищения.
\vs p166 1:10 \ublistelem{3.}\bibnobreakspace Уклонение от общения со всеми нефарисеями.
\vs p166 1:11 \pc На сей раз Иисус постарался показать духовное бесплодие только первых двух обычаев, а свои замечания в упрек фарисеям за их несогласие участвовать в общественных отношениях с нефарисеями отложил до другого, последующего случая, когда он снова окажется за одним столом со многими из этих же людей.
\usection{2. Десять прокаженных}
\vs p166 2:1 На следующий день Иисус и двенадцать апостолов пошли в город Амафу, расположенный недалеко от границы Самарии, и на подходе к нему встретили десять прокаженных, которые жили неподалеку. Девять человек из этой группы были евреи, а один --- самарянин. В обычной ситуации эти евреи воздержались бы от какого\hyp{}либо общения или соприкосновения с этим самарянином, однако общая болезнь легко преодолела все религиозные предубеждения. Они много слышали об Иисусе и чудесах исцеления, совершенных им ранее, а поскольку, когда Учитель вместе с двенадцатью апостолами совершал эти путешествия, семьдесят вестников, как правило, объявляли о времени ожидаемого прихода Иисуса, десять прокаженных узнали, когда приблизительно он должен появиться вблизи этого места, поэтому они расположились здесь на окраине города, надеясь привлечь его внимание и попросить об исцелении. Увидев приближавшегося Иисуса, прокаженные, не смея подойти к нему, встали вдали и кричали ему: «Учитель, помилуй нас; очисти нас от нашей болезни. Исцели нас, как исцелял ты других».
\vs p166 2:2 Иисус как раз объяснял двенадцати апостолам, почему неевреи Переи наряду с менее ортодоксальными евреями в большей степени готовы были уверовать в евангелие, которое проповедовали семьдесят вестников, нежели более ортодоксальные и скованные традициями евреи Иудеи. Он обратил их внимание на то, что подобным же образом с большей готовностью их послание было принято галилеянами и даже самарянами. Но двенадцать апостолов еще вряд ли были готовы питать добрые чувства к издавна презираемым самарянам.
\vs p166 2:3 Поэтому, завидев среди прокаженных самарянина, Симон Зилот попытался убедить Учителя продолжить путь и не задерживаться даже затем, чтобы обменяться с ними приветствиями. Иисус же сказал Симону: «А что если самарянин любит Бога так же, как и евреи? Нам ли судить наших собратьев? Кто знает, если мы вернем этим людям здоровье, возможно, самарянин окажется даже благодарнее евреев. Уверен ли ты в своем мнении, Симон?» И Симон быстро ответил: «Если очистишь их, скоро узнаешь». И Иисус ответил: «Да будет так, Симон, и ты вскоре узнаешь истину о благодарности людей и полном любви милосердии Бога».
\vs p166 2:4 Подойдя к прокаженным, Иисус сказал: «Если хотите исцелиться, пойдите, покажитесь священникам, как требует закон Моисеев». И когда они шли, очистились. Самарянин же, видя, что он исцелен, возвратился и, в поисках Иисуса, начал громким голосом прославлять Бога. И найдя Учителя, пал ниц к ногам его и благодарил за свое очищение. Девять же остальных, евреи, также увидели свое исцеление, но хотя тоже были благодарны за свое очищение, все же пошли дальше показаться священникам.
\vs p166 2:5 Когда самарянин стоял на коленях у ног Иисуса, Учитель, глядя на двенадцать апостолов, и особенно на Симона Зилота, сказал: «Не десять ли очистились? Где же тогда остальные девять, евреи? Лишь один, сей иноплеменник, возвратился воздать славу Богу». Затем Иисус сказал самарянину: «Встань и иди своим путем; вера твоя спасла тебя».
\vs p166 2:6 Когда незнакомец ушел, Иисус снова посмотрел на своих апостолов. И все апостолы, посмотрели на Иисуса, кроме Симона Зилота, опустившего глаза. Двенадцать апостолов не проронили ни слова. Ничего не сказал и Иисус; этого и не требовалось.
\vs p166 2:7 \pc Хотя все десять этих людей считали себя действительно больными проказой, только четверо страдали от этой болезни. Остальные же шесть исцелились от кожной болезни, которую принимали за проказу по ошибке. Но самарянин действительно был прокаженным.
\vs p166 2:8 \pc Иисус велел двенадцати апостолам ничего не говорить об очищении прокаженных и, когда они вошли в Амафу, заметил: «Как видите, получается так, что дети дома даже тогда, когда они непослушны воле Отца, принимают благодеяние как должное. Пренебрегая воздать благодарение, когда Отец дарует им исцеление, они не считают это большим проступком; посторонние же, получая дары от главы дома, исполняются удивления и вынуждены воздать благодарение в знак признания дарованных им благ». И по\hyp{}прежнему апостолы ничего не сказали в ответ на слова Учителя.
\usection{3. Проповедь в Герасе}
\vs p166 3:1 Когда Иисус и двенадцать апостолов встречались с вестниками царства в Герасе, один из фарисеев, веривший в него, задал ему такой вопрос: «Господи, много или мало будет спасенных?» И Иисус, отвечая, сказал:
\vs p166 3:2 \pc «Вас учили, что спасены будут лишь дети Авраамовы; что лишь усыновленные неевреи могут надеяться на спасение. Некоторые из вас рассудили, что поскольку из всего множества, вышедшего из Египта, только Халев и Иисус Навин дожили до того, чтобы войти в землю обетованную, лишь сравнительно немногие из ищущих царства небесного, найдут в него вход.
\vs p166 3:3 Есть среди вас и другое суждение, и в суждении том много истины. Оно утверждает, что путь, ведущий в жизнь вечную, прям и тесен, что дверь, ведущая в нее, тесна настолько, что из ищущих спасения лишь немногие сумеют пройти через эту дверь. Есть у вас также учение, согласно которому путь, ведущий к погибели, пространен, что врата, ведущие к ней, широки и что много избравших идти этим путем. Притча эта не лишена смысла. Но я заявляю вам, что спасение в первую очередь зависит от вашего личного выбора. Даже если дверь, за которой открывается жизнь вечная, тесна, она все равно достаточно просторна, чтобы через нее прошли все, искренне ищущие входа, ибо я эта дверь. И Сын никогда не откажет во входе ни одному из детей вселенной, верой стремящихся найти Отца через Сына.
\vs p166 3:4 Однако опасность грозит всем, кто желает отложить свое вхождение в царство и продолжает преследовать удовольствия, к которым стремятся незрелые люди, и предаваться своекорыстным утехам: отказавшись войти в царство духовным путем, они впоследствии могут искать вход в него, когда в грядущей эре откроется слава лучшего пути. А потому, когда те, кто отвергал царство, когда я приходил в подобии человеческом, попытаются найти вход, когда он откроется в подобии божьем, тогда я всем подобным эгоистам скажу: не знаю, откуда вы. У вас была возможность подготовиться к сему небесному бытию, но вы отказались от всех подобных предложений милосердия; вы отвергли все приглашения войти, пока дверь открыта; теперь же для вас, отвергших спасение, дверь заперта. Эта дверь не открыта для тех, кто желает войти в царство ради себялюбивой славы. Спасение --- не для тех, кто не желает заплатить за него искренним посвящением исполнению воли Отца моего. Если вы в духе и в душе отвернулись от царства небесного, то бесполезно в помыслах и теле стоять пред этой дверью и стучать, говоря: „Господи, отвори нам; мы тоже хотим быть великими в царстве“. Тогда я объявлю, что вы не моего стада. Я не приму вас и не позволю вам быть среди тех, кто сражается в благом сражении веры и заслужили награду за бескорыстное служение в царстве на земле. Когда же скажете: „Разве мы не ели и не пили с тобой и разве не учил ты на улицах наших?“, тогда снова объявлю вам, что вы духовные чужестранцы; что мы не сотоварищи в милосердном служении Отца на земле; что я не знаю вас; и тогда Судья всей земли скажет вам: „Отойдите от нас, все вы, кто наслаждался, творя беззаконие“.
\vs p166 3:5 Однако не бойтесь; всякий, искренне желающий найти жизнь вечную, войдя в царство Божье, обязательно найдет это вечное спасение. Вы же, отвергающие это спасение, однажды увидите, как пророки семени Авраамова садятся с верующими нееврейских наций в сем прославленном царстве, дабы вкусить хлеба жизни и освежить себя водой его. И берущие царство духовной силой и настойчивым приступом веры живой придут с севера и юга, и с востока и запада. И вот многие первые станут последними, и многие последние первыми».
\vs p166 3:6 Это была действительно новая и необычная трактовка старой и известной притчи о прямом и тесном пути.
\vs p166 3:7 Апостолы и многие из учеников с трудом постигали смысл прежнего заявления Иисуса: «Пока не родитесь заново, не родитесь от духа, не можете войти в царство Божье». Тем не менее, для всех, кто честен сердцем и искренен в вере, навсегда истинны слова: «Се, стою у дверей сердец человеческих и стучу, и если кто откроет мне, войду и буду вечерять с ним и накормлю его хлебом жизни; мы будем едины в духе и цели, а потому всегда будем братьями в долгом и плодотворном служении поиска Райского Отца». Итак, много или мало будет спасенных, зависит от того, много или мало будет внявших приглашению: «Я\hyp{}дверь, я\hyp{}новый и живой путь, всякий желающий может войти и приступить к бесконечным поискам вечной жизни в истине».
\vs p166 3:8 Даже апостолы, и те не могли до конца понять его учения о необходимости использования духовной силы, чтобы прорваться через все материальные преграды и преодолеть земные препятствия, которые по воле случая могут оказаться на пути постижения наиважнейших духовных ценностей новой жизни в духе как освобожденные сыновья Бога.
\usection{4. Учение о несчастных случаях}
\vs p166 4:1 Хотя большинство жителей Палестины ели только два раза в день, Иисус и апостолы во время путешествий имели обыкновение в середине дня останавливаться, чтобы передохнуть и подкрепиться. Во время такой полуденной остановки на пути к Филадельфии Фома и спросил Иисуса: «Учитель, я слушал твои рассуждения, пока мы шли этим утром, и хотел бы узнать, участвуют ли духовные существа в осуществлении некоторых необычных событий в материальном мире, а также спросить, способны ли ангелы или иные духовные существа предотвращать несчастные случаи?»
\vs p166 4:2 \pc В ответ на вопрос Фомы Иисус сказал: «Как давно я с вами, а вы все\hyp{}таки продолжаете задавать мне подобные вопросы? Разве не заметили вы, что Сын Человеческий живет с вами как один из вас и последовательно отказывается использовать силы небесные для своего личного поддержания? Разве не все мы живем так же, как живут все люди? Разве, не считая откровения Отца и иногда исцеления его больных детей, вы видите силу духовного мира, явленную в материальной жизни мира сего?
\vs p166 4:3 Слишком долго отцы ваши верили, что процветание есть признак божественного одобрения; что несчастья служат доказательством неудовольствия Божьего. Заявляю вам: подобные верования --- предрассудки. Разве не замечаете вы, что гораздо большее число бедных с радостью принимает евангелие и немедленно входит в царство? Если богатство свидетельствует о божественном благорасположении, то почему богатые столь часто отказываются верить этой благой вести с небес?
\vs p166 4:4 Отец посылает дождь на справедливых и несправедливых; подобно тому, солнце светит на праведных и неправедных. Вы знаете о тех галилеянах, чью кровь Пилат смешал с жертвами, но я говорю вам: эти галилеяне никоим образом не были грешнее всех своих собратьев, потому лишь только что это случилось именно с ними. Вы также знаете, о восемнадцати человеках, на которых упала Силоамская башня, убив их. Не думайте, что эти погибшие люди были виновнее всех своих братьев в Иерусалиме. Эти люди были просто невинными жертвами одного из несчастных случаев, происходящих во времени.
\vs p166 4:5 Существуют три группы событий, которые могут произойти в ваших жизнях:
\vs p166 4:6 . Вы можете участвовать в обычных событиях, являющихся частью жизни, которой вы и ваши собратья живете на земле.
\vs p166 4:7 . Вы можете случайно стать жертвой какой\hyp{}либо природной катастрофы, какого\hyp{}либо человеческого несчастия, полностью сознавая, что подобные происшествия никоим образом не являются заранее подготовленными или каким\hyp{}то иным способом осуществленными духовными силами.
\vs p166 4:8 . Вы можете пожинать плоды ваших же непосредственных усилий подчиниться естественным законам, управляющим миром.
\vs p166 4:9 \pc Один человек посадил у себя на дворе смоковницу; он много раз искал на ней плодов и, не найдя ни одного, призвал к себе виноградарей и сказал: „Три года я приходил сюда, искал плодов на этой смоковнице и не нашел. Сруби это бесплодное дерево; для чего оно занимает землю?“ Но главный садовник ответил своему господину: „Оставь ее еще на один год, чтобы я мог ее окопать и удобрить, и тогда, если не принесет плодов, на следующий год срубишь ее“. И, исполнив таким образом законы земледелия, они получили в награду обильный урожай, так как дерево было живым и добрым.
\vs p166 4:10 О болезни и здоровье вы должны знать, что эти телесные состояния являются следствием материальных причин; здоровье --- это не улыбка небес, а болезнь --- не хмурый взгляд Бога.
\vs p166 4:11 Человеческие дети Отца имеют равные способности для принятия материальных даров; поэтому детям человеческим он дарует физические блага без пристрастия. Что же касается дарования духовных даров, здесь Отец ограничен способностью человека принять эти божественные дарования. Хотя Отец не взирает на лица, в даровании духовных даров он ограничен верой человека и его готовностью всегда подчиняться воле Отца».
\vs p166 4:12 \pc Пока они шли дальше к Филадельфии, Иисус продолжал учить их и отвечать на их вопросы о несчастных случаях, болезнях и чудесах, но они не смогли до конца понять это наставление. Один час учения не может в корне изменить верования всей жизни, а потому Иисус счел необходимым неоднократно повторить это послание, снова и снова говорить то, что он хотел, чтобы они поняли; но даже при всем этом они сумели осознать значение его земной миссии только после его смерти и воскресения.
\usection{5. Собрание в Филадельфии}
\vs p166 5:1 Иисус и двенадцать апостолов шли навестить Авенира и его сподвижников, которые проповедовали и учили в Филадельфии. Из всех городов Переи в Филадельфии больше всего народу, евреев и неевреев, богатых и бедных, ученых и неученых, приняло учение семидесяти вестников и, таким образом, вошло в царство небесное. Синагога Филадельфии никогда не подвергалась надзору синедриона в Иерусалиме, а потому никогда не была закрытой для учения Иисуса и его сподвижников. В это самое время Авенир три раза в день учил в филадельфийской синагоге.
\vs p166 5:2 Эта синагога позднее стала христианской церковью и была центром миссионеров, откуда евангелие распространялось в восточные области. Она долго служила оплотом учений Учителя и единственная в этой местности веками являлась центром христианского учения.
\vs p166 5:3 У евреев в Иерусалиме всегда были трения с евреями Филадельфии. И после смерти и воскресения Иисуса иерусалимская церковь, главой которой был брат Господа Иаков, в отношениях с филадельфийским собранием верующих начала испытывать серьезные трудности. Главой филадельфийской церкви стал Авенир и оставался им до своей смерти. Эта напряженность в отношениях с Иерусалимом и объясняет, почему в записях Евангелия в Новом Завете об Авенире ничего не сказано. Вражда между Иерусалимом и Филадельфией продолжалась на протяжении всей жизни Иакова и Авенира и сохранялась еще какое\hyp{}то время после разрушения Иерусалима. Филадельфия действительно была оплотом ранней церкви на юге и востоке так же, как Антиохия --- на севере и на западе.
\vs p166 5:4 \pc Расхождение во мнениях со всеми лидерами ранней христианской церкви было очевидным несчастьем Авенира. Он расходился с Петром и Иаковом (братом Иисуса) в вопросах управления и полномочий иерусалимской церкви; он разошелся с Павлом в вопросах философии и теологии. В своей философии Авенир был больше вавилонянином, нежели эллином и упрямо сопротивлялся всем попыткам Павла изменить учения Иисуса так, чтобы в них осталось меньше того, что могло вызвать возражения сначала евреев, а потом и греко\hyp{}римских верующих в мистерии.
\vs p166 5:5 Таким образом, Авенир был вынужден жить в изоляции. Он был главой церкви, не признанной в Иерусалиме. Он осмеливался не повиноваться Иакову, брату Господа, которого впоследствии поддержал Петр. Такое поведение фактически отделило Авенира от всех бывших сподвижников. Затем Авенир осмелился противоречить Павлу. Хотя Авенир полностью сочувствовал Павлу в его миссионерской деятельности среди неевреев и поддерживал Павла в его спорах с церковью в Иерусалиме, он решительно возражал против трактовки учений Иисуса, которую решил проповедовать Павел. В свои последние годы Авенир осуждал Павла как «умного исказителя учений жизни Иисуса из Назарета, Сына Бога живого».
\vs p166 5:6 На протяжении последних лет жизни Авенира и какое\hyp{}то время после его смерти верующие Филадельфии строже чем какая\hyp{}либо другая группа на земле, придерживались религии Иисуса, как он жил и учил.
\vs p166 5:7 Авенир дожил до 89 лет и умер в Филадельфии в 21\hyp{}й день ноября 74 года н.э. До самого конца он оставался истинным верующим в евангелие царства небесного и его учителем.
