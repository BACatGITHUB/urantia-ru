\upaper{148}{Подготовка евангелистов в Вифсаиде}
\author{Комиссия срединников}
\vs p148 0:1 С 3 мая по 3 октября 28 года н.э. Иисус и апостолы жили в доме Зеведея в Вифсаиде. В течение всех пяти месяцев сухого сезона недалеко от дома Зеведея на берегу моря располагался огромный лагерь, который был значительно расширен, чтобы разместить в нем растущую семью Иисуса. В этом приморском лагере постоянно находилось от пятисот до полутора тысяч человек, хотя состав обитателей непрерывно менялся, среди них были и искатели истины, желающие получить исцеление, и просто любопытные. Этим палаточным городом руководил Давид Зеведеев, которому помогали близнецы Алфеевы. Место, занимаемое лагерем, было образцом порядка, чистоты и вообще управления. Больные с заболеваниями различного характера содержались отдельно и находились под наблюдением верующего врача\hyp{}сирийца по имени Елман.
\vs p148 0:2 На протяжении этого времени, по крайней мере, один раз в неделю апостолы ловили рыбу, продавая улов Давиду для пропитания его приморского лагеря. Полученные таким образом средства передавались в общую казну. Раз в месяц апостолам позволялось провести неделю со своими семьями или друзьями.
\vs p148 0:3 В то время как Андрей продолжал осуществлять общее руководство деятельностью апостолов, Петр полностью отвечал за школу евангелистов. Все апостолы вносили свою лепту в обучение групп евангелистов, занятия с которыми проходили каждый день с утра до полудня; после полудня же учителя и ученики учили народ. После ужина пять вечеров в неделю апостолы проводили занятия, на которых евангелисты задавали вопросы. Один раз в неделю на этом часе вопросов председательствовал Иисус, отвечая на вопросы, не получившие ответа на предыдущих собраниях.
\vs p148 0:4 За пять месяцев в этом лагере побывало несколько тысяч человек. Его часто посещали интересующиеся со всех концов Римской империи и из земель к востоку от Евфрата. Это был самый продолжительный и хорошо организованный процесс обучения, проводимый Учителем на одном и том же месте. Родные же Иисуса большую часть этого времени провели или в Назарете или в Кане.
\vs p148 0:5 Руководство лагерем осуществлялось иным образом, нежели группой людей с общими интересами, как было в случае семьи апостолов. Давид Зеведеев управлял этим огромным палаточным городом так, что тот стал самодостаточным предприятием, несмотря на то, что в нем никому и никогда не отказывали в месте. Этот постоянно изменявшийся лагерь был неотъемлемой частью школы подготовки евангелистов, которой заведовал Петр.
\usection{1. Новая школа пророков}
\vs p148 1:1 Иисус назначил комиссию в составе Петра, Иакова и Андрея, в задачи которой входило отбирать кандидатов на прием в школу евангелистов. В этой новой школе пророков среди учащихся были представлены все расы и национальности Римской империи и Востока вплоть до Индии. Занятия в школе предусматривали учебу и практику. Чему учащиеся учились до полудня, тому они учили собравшихся у берега моря после полудня. После ужина в непринужденной обстановке они обсуждали и то, как учились до полудня, и то, как учили после полудня.
\vs p148 1:2 Каждый из учителей\hyp{}апостолов учил своему собственному пониманию евангелия царства. Они не пытались учить одинаково; стандартных или догматических формулировок теологических доктрин не было. Хотя все они учили \bibemph{одной и той же истине,} каждый апостол предлагал свое собственное личное толкование учения Учителя. И Иисус одобрял такое проявление разнообразия личного опыта в делах царства, неизменно согласовывая и координируя эти многочисленные и отличавшиеся друг от друга понимания евангелия во время проводимого им еженедельно часа ответов на вопросы. Несмотря на эту огромную степень личной свободы в вопросах преподавания, Симон Петр как теолог постепенно выдвинулся на первое место в школе евангелистов. После Петра же наибольшим личным влиянием пользовался Иаков Зеведеев.
\vs p148 1:3 Более ста евангелистов, воспитанных за эти пять месяцев у берега моря, стали той группой, из которой (не считая Авенира и апостолов Иоанна) позднее были отобраны семьдесят учителей и проповедников евангелия. В школе евангелистов все не было общим до такой степени, как у двенадцати.
\vs p148 1:4 Хотя эти евангелисты учили евангелию и проповедовали его, они не крестили верующих до тех пор, пока позднее не были посвящены и уполномочены Иисусом в качестве семидесяти вестников царства. Лишь семеро из огромного числа тех, кто был исцелен при заходе солнца на этом месте, оказались среди этих учеников\hyp{}евангелистов. Сын же знатного человека из Капернаума был одним из тех, кто воспитывался для служения евангелию в школе Петра.
\usection{2. Больница в Вифсаиде}
\vs p148 2:1 С помощью отряда, состоявшего из двадцати пяти молодых женщин и двенадцати мужчин, сирийский врач Елман в находившемся у берега моря лагере организовал и в течение четырех месяцев руководил тем, что следует считать первой больницей царства. В этом расположенном на небольшом расстоянии к югу от главного палаточного города лазарете лечили больных всеми известными материальными методами, а также используя духовные методы молитвы и укрепляя веру. Иисус посещал больных этого лагеря не реже трех раз в неделю и лично общался с каждым страдальцем. Насколько нам известно, среди тысячи пораженных недугами и больных, вышедших из этого лазарета поправившимися или здоровыми, не было так называемых чудес сверхъестественного исцеления. Однако подавляющее большинство из этих получивших помощь людей не прекращало провозглашать, что исцелил их Иисус.
\vs p148 2:2 Многие исцеления, совершенные Иисусом в процессе служения на благо пациентов Елмана, действительно казались похожими на чудотворство, однако нам сообщили, что они были всего лишь такими преображениями ума и духа, какие могут происходить в жизни исполненных надеждой и проникнувшихся верой людей, которые находятся под прямым и вдохновляющим влиянием сильной, уверенной и благодетельной личности, чье служение изгоняет страх и устраняет тревогу.
\vs p148 2:3 Елман и его помощники пытались научить этих больных истине об «одержимости злыми духами», но, увы, с малым успехом. Вера, что болезнь тела и психическое расстройство могут быть вызваны пребыванием так называемого нечистого духа в уме или теле больного человека, была почти универсальной.
\vs p148 2:4 Когда дело касалось способа лечения или раскрытия, до сих пор, неизвестных причин болезни, Иисус в своем общении с больными и страдающими не игнорировал наставления своего Райского брата Иммануила, данными еще перед тем, как Иисус вступил на путь воплощения на Урантии. Несмотря на это, те, кто служил больным, усвоили много полезных уроков, наблюдая за тем, как Иисус вселял веру и уверенность в больных и страдальцев.
\vs p148 2:5 Лагерь расформировался незадолго до приближения сезона простуд и гриппа.
\usection{3. Дело Отца}
\vs p148 3:1 На протяжении этого периода Иисус проводил публичные службы в лагере менее двенадцати раз и лишь однажды произнес речь в капернаумской синагоге --- в предпоследнюю субботу перед тем, как они второй раз отправились с публичными проповедями по Галилее вместе с только что подготовленными евангелистами.
\vs p148 3:2 После своего крещения Учитель никогда не проводил столько времени в одиночестве, как в этот период подготовки евангелистов в лагере в Вифсаиде. Когда бы любой из апостолов ни решался спросить Иисуса, почему он так редко проводит с ними время, он неизменно отвечал, что «занят тем, что принадлежит Отцу моему».
\vs p148 3:3 Во время его отсутствия Иисуса сопровождали только двое из апостолов. Он временно освободил Петра, Иакова и Иоанна от их обязанности быть его личными помощниками, чтобы они также могли участвовать в подготовке новых кандидатов в евангелисты, которых насчитывалось уже более ста. Когда Учитель желал уйти в горы, чтобы быть в том, что принадлежит Отцу, он брал с собой тех двух апостолов, которые в это время были не заняты. Таким образом, каждый из двенадцати насладился возможностью близкого общения и личного контакта с Иисусом.
\vs p148 3:4 Хотя это не предназначалось для данной записи, однако нам дано было прийти к заключению, что Учитель в течение многих из этих периодов уединения в горах прямо общался со многими из его главных руководителей делами вселенной. Приблизительно со времени своего крещения сей воплотившийся Владыка нашей вселенной все более сознательно и активно вовлекался в определенные аспекты управления вселенной. И мы всегда придерживались мнения, что на протяжении этих недель, когда сократилось его участие в делах земли, он каким\hyp{}то неоткрытым его ближайшим соратникам образом был занят руководством теми высоко духовными разумными существами, которым было поручено управлять обширной вселенной, и что человек Иисус решил называть подобные действия со своей стороны участием в «том, что принадлежит Отцу моему».
\vs p148 3:5 Множество раз, когда Иисус часами пребывал в одиночестве, но двое из его апостолов были рядом с ним, они наблюдали, как его облик мгновенно и разнообразно преображался, хотя они и не слышали, чтобы он произносил какие\hyp{}либо слова. Не наблюдали они и каких\hyp{}либо видимых проявлений небесных существ, которые могли находиться в общении с их Учителем, таких, подобных которым они увидели во время события, происшедшего впоследствии.
\usection{4. Зло, грех и порочность}
\vs p148 4:1 По своему обыкновению каждую неделю Иисус два вечера посвящал особой беседе с отдельными желавшими поговорить с ним людьми, которая происходила в неком уединенном и укрытом уголке сада Зеведея. Во время одной из этих вечерних бесед наедине Фома спросил Учителя: «Почему, чтобы войти в царство, людям необходимо родиться от духа? Разве для того, чтобы уйти из\hyp{}под власти нечистого, необходимо перерождение? Учитель, что такое зло?» Услышав эти вопросы, Иисус сказал Фоме:
\vs p148 4:2 \P\ «Не делай ошибку, путая \bibemph{зло с нечистым,} а вернее с \bibemph{порочным.} Тот, кого вы называете нечистым, есть сын себялюбия, высокий управитель, который сознательно и преднамеренно предпринял восстание против правления моего Отца и верных ему Сыновей. Однако я уже победил этих грешных мятежников. Хорошо пойми эти различные подходы к Отцу и его вселенной. Никогда не забывай законы, объясняющие отношения к воле Отца.
\vs p148 4:3 Зло --- это несознательное и ненамеренное нарушение божественного закона, воли Отца. Зло является также мерой несовершенства подчинения воле Отца.
\vs p148 4:4 Грех --- это сознательное, преднамеренное и умышленное нарушение божественного закона, воли Отца. Грех является мерой нежелания подчиниться божественному водительству и духовному руководству.
\vs p148 4:5 Порочность --- это своевольное, непреклонное и упорное нарушение божественного закона, воли Отца. Порочность является мерой непрекращающегося отвержения полного любви замысла Отца, касающегося перехода личности в жизнь вечную, и отвержения милосердного и спасительного служения Сыновей.
\vs p148 4:6 По природе своей, смертный человек, до рождения от духа, подвержен влиянию присущих ему порочных наклонностей, однако подобные естественные несовершенства в поведении не являются ни грехом, ни порочностью. Смертный человек еще только вступает на долгий путь восхождения к совершенству Отца в Раю. Быть несовершенным или обладать недостаточным естественным дарованием не грешно. Человек, действительно, подвержен злу, однако никоим образом не является сыном нечистого до тех пор, пока сам сознательно и умышленно не изберет пути греха и порочной жизни. Зло присуще естественному порядку вещей в этом мире; грех же есть состояние сознательного мятежа, привнесенного в этот мир теми, кто отпал от духовного света и погрузился в кромешную тьму.
\vs p148 4:7 Ты, Фома, смущен доктринами греков и ошибками персов. Ты не понимаешь отношений зла и греха, потому что считаешь что человечество на земле ведет свое начало от совершенного Адама и затем быстро деградировало через грех до современного плачевного состояния человека. Однако почему ты отказываешься понимать значение записи, где говорится о том, как сын Адама Каин перешел в землю Нода и там взял себе жену? И почему отказываешься толковать значение записи, где показано, как сыновья Бога нашли себе жен среди дочерей человеческих?
\vs p148 4:8 Люди по природе своей, действительно, злы, но не обязательно грешны. Новое рождение --- крещение духа --- важно для избавления от зла и необходимо для вхождения в царство небесное, однако ни то, ни другое не умаляет значимости того, что человек является сыном Бога. И это присущее человеку потенциальное зло не означает, что человек неким непостижимым образом отстранен от Отца Небесного, так что, подобно чужому, иноземцу или пасынку, должен тем или иным способом искать законного усыновления Отцом. Все подобные представления происходят, во\hyp{}первых, от неправильного понимания Отца и, во\hyp{}вторых, от незнания происхождения, природы и предназначения человека.
\vs p148 4:9 Греки и другие учили, что человек непрерывно опускается от божественного совершенства к забвению и гибели; я же пришел показать, что человек, входя в царство, уверенно и несомненно вступает на путь восхождения к Богу и к божественному совершенству. Всякое существо, тем или иным образом не достигшее божественных и духовных идеалов воли вечного Отца, потенциально зло, но подобные существа никоим образом не грешны и тем более не порочны.
\vs p148 4:10 Фома, разве ты не читал об этом в Писании, где написано: „Вы --- дети Господа Бога вашего“. «„Я буду ему Отцом, и он будет мне сыном“. „Я избрал его в сына; я буду его Отцом“. „Веди сыновей моих издалека и дочерей моих от концов земли, каждого, кто называется моим именем, кого я сотворил для славы моей“. „Вы сыны Бога живого“. „Имеющие духа Бога --- воистину сыновья Бога“. Хотя в земном сыне присутствует материальная часть его земного отца, в каждом сыне царства, вошедшем в него благодаря своей вере, присутствует духовная часть Отца Небесного».
\vs p148 4:11 \P\ Все это и более того Иисус сказал Фоме, и апостол понял многое из сказанного, хотя Иисус наказал ему «ничего не говорить другим об этих вопросах, пока я не вернусь к Отцу». И Фома не упоминал об этой беседе до тех пор, пока Учитель не покинул этот мир.
\usection{5. Назначение страдания}
\vs p148 5:1 Во время еще одной из таких личных бесед в саду Нафанаил спросил Иисуса: «Учитель, хоть я и начинаю понимать, почему ты отказываешься давать исцеление всем подряд, я по\hyp{}прежнему затрудняюсь понять, почему любящий Отец Небесный допускает, чтобы так много из детей его на земле страдало от стольких болезней». Учитель ответил Нафанаилу, сказав:
\vs p148 5:2 \P\ «Ты, Нафанаил, и многие другие пребываете в подобном недоумении, поскольку не сознаете того, как естественный порядок в этом мире столь многократно нарушался грешными предприятиями определенных мятежных изменников воли Отца. И я пришел, чтобы положить начало восстановлению порядка. Однако для того, чтобы вернуть эту часть вселенной на изначальные пути и, таким образом, освободить детей человеческих от дополнительного бремени греха и мятежа, потребуется много веков. Только зла вполне достаточно, чтобы испытать человека на путях его восхождения; борьба же с грехом вовсе не необходима для продолжения существования в посмертии.
\vs p148 5:3 Однако, сын мой, ты должен знать, что Отец не причиняет страдания своим детям намеренно. Упорно отказываясь идти по лучшим путям божественной воли, человек сам навлекает на себя ненужное страдание. Страдание потенциально содержится во зле, однако в основном порождено грехом и порочностью. В этом мире произошло много необычных событий, и неудивительно, что все мыслящие люди озадачены картинами страдания и болезней, которые они наблюдают. Однако в одном можете быть уверены: Отец не посылает страдание как беспричинное наказание за проступки. Несовершенства и недостатки, присущие злу, суть врожденные свойства; наказания за грех неизбежны; разрушительные последствия порочности неумолимы. Человек не должен винить Бога в тех страданиях, что являются естественным результатом жизни, которую он избрал. Не должен человек жаловаться и на те переживания, что являются частью жизни, которой живут в этом мире. Воля Отца состоит в том, чтобы смертный человек упорно и последовательно стремился к улучшению своего положения на земле. Разумное усердие) позволит человеку преодолеть большую часть своего земного страдания.
\vs p148 5:4 Наша миссия, Нафанаил, заключается в том, чтобы помогать людям решать их духовные проблемы и таким образом оживлять их умы, чтобы они могли лучше подготовиться и вдохновиться к решению своих разнообразных материальных проблем. Я знаю, что то, что ты читал в Писании, вызывает в тебе недоумение. Слишком часто в нем преобладало стремление приписывать Богу ответственность за все, что невежественному человеку не удается понять. Отец не несет личной ответственности за все, что вы, быть может, не сумеете осознать. Не сомневайтесь в любви Отца лишь потому, что определенный справедливый и мудрый закон, предписанный им, иногда причиняет вам страдание, поскольку вы неумышленно или намеренно преступили подобное божественное указание.
\vs p148 5:5 Однако, Нафанаил, в Писании есть много такого, что научило бы тебя, если бы ты читал вдумчиво. Разве не помнишь ты, что написано: „Наказания Господа, сын мой, не презирай, и не утомляйся обличением его; ибо кого любит Господь, того наказывает, как и отец наказывает сына кем радуется“. „Не по изволению своему наказывает Господь“. „Прежде страдания моего я заблуждал, а ныне закон твой храню“. „Страдание было благом мне, дабы я мог научиться божественным уставам“. „Знаю я скорби твои. Прибежище твое Бог вечный, и ты под дланями вечными“. „Господь есть также прибежище угнетенному, прибежище во времена страданий“. „Господь укрепит его на одре болезни его; Господь не забудет больного“. „Как отец проявляет сострадание к детям своим, так и Господь сострадает боящимся его. Он знает тело твое, помнит, что ты прах“. „Он исцеляет сокрушенных сердцем и перевязывает раны их“. „Он надежда бедного, сила нуждающегося в горе его, убежище от бури, тень от опустошительного зноя“. „Он дает утомленному силу и изнемогшему дарует крепость“. „Трости надломленной не переломит, и льна курящегося не угасит“. „Когда будешь переходить через воды страдания, я буду с тобою, и когда реки несчастья затопят тебя, я не покину тебя“. „Он послал меня перевязывать раны сокрушенным сердцем, возвещать пленным освобождение и утешать всех скорбящих“. „В страдании есть исправление; не из праха возникает горе“».
\usection{6. Ошибочное понимание страдания --- беседа об Иове}
\vs p148 6:1 В тот же самый вечер в Вифсаиде Иоанн тоже спросил Иисуса, почему так много явно невинных людей страдают от стольких болезней и испытывают столько бед. В ответ на вопросы Иоанна, помимо всего прочего, Учитель сказал:
\vs p148 6:2 \P\ «Сын мой, ты не понимаешь значения несчастья или назначения страдания. Разве ты не читал шедевра семитской литературы --- историю в Писании о страданиях Иова? Разве ты не помнишь, как эта чудесная притча начинается с рассказа о материальном процветании слуги Господа? Ты прекрасно помнишь, что Иов был благословлен детьми, богатством, знатностью, положением, здоровьем и всем остальным, что ценят люди в этой временной жизни. Согласно освященным веками учениям детей Авраамовых, такое материальное процветание было вседостаточным свидетельством божественного благоволения. Однако подобное материальное благосостояние и такое временное процветание отнюдь не служат признаком благоволения Бога. Отец мой небесный любит бедных так же, как и богатых, и не взирает на лица.
\vs p148 6:3 Хотя за нарушением божественного закона рано или поздно следует неотвратимое возмездие, хотя в конце концов люди, несомненно, пожинают то, что посеяли, вы все же должны знать, что человеческое страдание не всегда является наказанием за грех, совершенный в прошлом. И Иов, и его друзья не сумели найти правильный ответ на мучившие их вопросы. Ты же в просветлении, которым теперь обладаешь, едва ли станешь приписывать Сатане или Богу те роли, которые они играют в этой уникальной притче. Хотя Иов через страдание не нашел разрешения своих интеллектуальных тревог или решения своих философских затруднений, он одержал великие победы и даже вопреки крушению своих теологических воззрений поднялся до тех духовных высот, где смог искренне сказать: „Я отрекаюсь“; тогда ему было даровано спасение \bibemph{видения Бога.} Так даже через непонятое им страдание Иов возвысился до сверхчеловеческого уровня нравственного понимания и духовного осознания. Когда страждущий слуга обретает видение Бога, тогда в душе наступает мир, превосходящий всякое человеческое понимание.
\vs p148 6:4 Первый из друзей Иова Елифаз увещевал страдальца проявлять в своих несчастьях такую же силу духа, к какой Иов сам призывал других во дни своего процветания. Этот ложный утешитель сказал: „Доверься своей религии, Иов; помни, что страдают порочные, а не праведные. Видимо, ты заслужил это наказание, иначе ты бы не страдал. Ты хорошо знаешь, что ни один человек не может быть праведным в глазах Божиих. Тебе известно, что порочные никогда по\hyp{}настоящему не процветают. Так или иначе человеку предопределено страдание, и, возможно, Господь карает тебя ради твоего же блага“. Неудивительно, что бедного Иова совсем не утешило подобное толкование проблем человеческого страдания.
\vs p148 6:5 Однако совет его второго друга Вилдада был еще более угнетающим, несмотря на его правильность с точки зрения принятой тогда теологии. Вилдад сказал: „Бог не может быть несправедливым. Твои дети, видимо, были грешниками, раз они погибли; должно быть, ты заблуждаешься, иначе ты бы так не страдал. Если же ты действительно праведен, Бог, несомненно, избавит тебя от страданий твоих. Из истории общения Бога с человеком ты должен был узнать, что Всемогущий сокрушает только порочных“.
\vs p148 6:6 И далее, ты помнишь, как Иов, отвечая своим друзьям, сказал: „Я хорошо знаю, что Бог не слышит моего крика о помощи. Как может Бог быть справедливым и одновременно не обращать внимания на мою безвинность? Я вижу, что напрасно взываю к Всемогущему. Неужели вы не понимаете, что Бог терпит притеснение благих нечестивыми? А поскольку человек столь слаб, какова же вероятность того, что он заслужит внимания в руках всемогущего Бога? Бог создал меня таким, каков я есть, и когда он так обращается со мной, я беззащитен. И зачем вообще Бог сотворил меня, чтобы я так жестоко страдал?“
\vs p148 6:7 И кто может оспорить позицию Иова, принимая во внимание совет его друзей и ошибочные представления о Боге, существующие в его собственном разуме? Разве ты не видишь, что Иов стремился к \bibemph{человеческому} Богу, что он жаждал общения с божественным Существом, которое знает положение смертного человека и понимает, что праведный, будучи невинен, часто должен страдать в какой\hyp{}то части этой первой жизни на долгом пути восхождения к Раю? Именно по этой причине Сын Человеческий и пришел от Отца, чтобы жить такой жизнью во плоти, когда он смог бы утешать и помогать всем тем, кто в дальнейшем должен быть призван переносить страдания Иова.
\vs p148 6:8 Затем третий друг Иова Софар произнес еще менее утешительные слова, сказав: „Ты глуп, претендуя на праведность, ибо видишь, каким подвергаешься страданиям. Однако я согласен с тем, что пути Божии неисповедимы. Возможно, во всех твоих несчастьях есть какая\hyp{}то скрытая цель“. И, выслушав всех троих своих друзей, Иов воззвал о помощи непосредственно к Богу, ссылаясь на то, что „человек, рожденный женою, краткодневен и пресыщен печалями“.
\vs p148 6:9 Затем произошла вторая беседа с его друзьями. Елифаз сделался еще более суровым, обличительным и саркастичным. Вилдад вознегодовал на презрение Иова к своим друзьям. Софар еще раз повторил свой пессимистический совет. К этому времени друзья стали Иову отвратительны, и он снова воззвал к Богу; воззвал же теперь к справедливому Богу против Бога несправедливости, воплощенной в философии его друзей и содержащейся даже в его собственных религиозных взглядах. Затем Иов нашел убежище в утешении будущей жизни, где несправедливости смертного существования могут быть исправлены с большей справедливостью. Неудачная попытка получить помощь от человека приводит Иова к Богу. Далее в его сердце происходит великая борьба между верой и сомнением. Наконец, человеческий страдалец начинает видеть свет жизни; его измученная душа восходит к новым высотам надежды и мужества; он может страдать еще и может даже умереть, но его просвещенная душа теперь издает победный крик: „Искупитель мой жив!“
\vs p148 6:10 Иов был совершенно прав, когда противостоял учению, согласно которому Бог причиняет страдание детям, дабы наказать их родителей. Иов всегда был готов признать, что Бог праведен, однако он жаждал какого\hyp{}нибудь успокаивающего душу откровения о личной природе Вечного. В этом и заключается наша миссия на земле. Отныне страждущим смертным не может быть отказано в утешении познания любви Бога и понимания милосердия Отца Небесного. Хотя словоизъявление Бога, прогремевшее из туч, было величественной идеей для того времени, ты уже узнал, что Отец не открывает себя так, но говорит в сердце человека тихим, спокойным голосом: „Вот путь; иди по нему“. Разве ты не сознаешь, что Бог пребывает внутри тебя, что он стал тем, что ты есть, дабы сделать тебя тем, что он есть?»
\vs p148 6:11 Затем Иисус заключил: «Отец Небесный совершенно не хочет причинять страдания детям человеческим. Человек страдает в первую очередь из\hyp{}за случайностей, происходящих во времени, и порочности зла, присущих незрелому физическому существованию. Затем он страдает от неумолимых последствий греха --- нарушения закона жизни и света. И, наконец, человек собирает урожай своего собственного порочного упорства, восставая против справедливого правления неба на земле. Однако несчастья человека не являются \bibemph{личной} карой божественного осуждения. Человек может многое сделать и сделает для облегчения своих временных страданий. Однако раз и навсегда надо избавиться от предрассудка, будто Бог наказывает человека по повелению нечистого. Внимательно изучи Книгу Иова, и ты обнаружишь, как часто даже добропорядочные люди могут искренне разделять неверные представления о Боге; затем заметь, как даже жестоко страдавший Иов нашел Бога утешения и спасения, несмотря на такие ошибочные учения. По крайней мере, вера его пронзила облака страдания, чтобы увидеть свет жизни, изливающийся от Отца целительным милосердием и вечной праведностью».
\vs p148 6:12 Много дней обдумывал Иоанн в своем сердце эти высказывания. Благодаря этой беседе с Учителем в саду, заметно изменилась вся его дальнейшая жизнь, и позднее он многое сделал, чтобы побудить других апостолов изменить свою точку зрения на источник, природу и назначение обыкновенных человеческих страданий. Однако до кончины Учителя Иоанн никогда не рассказывал об этой беседе.
\usection{7. Человек с иссушенной рукой}
\vs p148 7:1 Во вторую субботу перед тем, как апостолы и новый отряд евангелистов во второй раз отправились с проповедями по Галилее, Иисус произнес в капернаумской синагоге речь «Радости Праведной Жизни». Когда Иисус кончил говорить, вокруг него собралась большая группа ищущих исцеления калек, хромых, больных и страдающих недугами. В этой группе были также апостолы, многие из новых евангелистов и фарисеи\hyp{}шпионы из Иерусалима. Куда бы ни шел Иисус (кроме тех случаев, когда он в горах был в том, что принадлежит Отцу) шесть иерусалимских шпионов следовали за ним.
\vs p148 7:2 Когда Иисус стоял, беседуя с народом, главный из фарисеев\hyp{}шпионов подговорил человека с иссушенной рукой подойти к Учителю и спросить его, законно ли быть исцеленным в субботу или ему следует искать помощи в другой день. Увидев этого человека, услышав его слова и поняв, что он подослан фарисеями, Иисус сказал: «Выйди вперед, пока я буду спрашивать тебя. Если бы у тебя была овца и она бы упала в яму в субботу, разве не полез бы ты в яму, не взял бы ее и не вытащил бы? Законно ли делать подобное в субботу?» И человек ответил: «Да, Учитель, такое доброе дело законно сделать в день субботы». Тогда, обращаясь ко всем, Иисус сказал: «Я знаю, зачем вы подослали этого человека ко мне. Если бы вы смогли искусить меня проявить милосердие в субботу, вы бы нашли повод обвинить меня. Вы все молча согласились с тем, что законно вытащить из ямы несчастную овцу даже в субботу, и я призываю вас свидетельствовать, что проявлять милосердную доброту в субботу законно не только по отношению к животным, но и к людям. Ведь человек гораздо ценнее овцы! Я объявляю: законно делать добро людям в субботу». И когда они все молча стояли перед ним, Иисус, обращаясь к человеку с иссушенной рукой, сказал: «Встань здесь рядом со мной, чтобы все тебя видели. А теперь, чтобы ты знал, что воля Отца моего состоит в том, чтобы ты делал добро в субботу, я приказываю: если веруешь в свое исцеление, протяни руку твою».
\vs p148 7:3 И когда этот человек протянул свою руку, она сделалась здоровой. Народ был готов броситься на фарисеев, но Иисус приказал им успокоиться и сказал: «Я только что сказал вам, что в субботу законно делать добро, спасать жизнь, но я не учил вас делать зло и давать волю желанию убивать». Обозленные фарисеи ушли и, несмотря на то, что была суббота, поспешили в Тиверию и держали совет с Иродом, делая все, что было в их власти, дабы возбудить в нем предубеждение, чтобы иродиане стали их союзниками против Иисуса. Однако Ирод отказался действовать против Иисуса, посоветовав им отнести свои жалобы в Иерусалим.
\vs p148 7:4 Это был первый случай, когда Иисус сотворил чудо в ответ на вызов своих врагов. И Учитель совершил это так называемое чудо не для того, чтобы показать свою силу целителя, а для того, чтобы активно выступить против превращения субботнего покоя, предписываемого религией, в настоящую кабалу бессмысленных запретов для всего человечества. Этот человек вернулся к своей профессии каменщика и оказался одним из тех, за чьим исцелением последовала жизнь, полная благодарения и праведности.
\usection{8. Последняя неделя в Вифсаиде}
\vs p148 8:1 Последнюю неделю пребывания в Вифсаиде иерусалимские шпионы разделились в своем отношении к Иисусу и его учениям. На троих из этих фарисеев огромное впечатление произвело то, что они увидели и услышали. Тем временем в Иерусалиме некий Авраам, молодой и влиятельный член синедриона, публично поддержал учения Иисуса и был крещен Авениром в Силоамской купели. Это событие взволновало весь Иерусалим, и в Вифсаиду немедленно были отправлены посланцы отозвать шестерых фарисеев\hyp{}шпионов.
\vs p148 8:2 \P\ Греческий философ, обращенный к царству во время предыдущего путешествия по Галилее, вернулся с неким богатым евреем из Александрии, и они еще раз пригласили Иисуса прийти в их город, чтобы основать совместную школу философии и религии и лазарет для больных. Но Иисус вежливо отклонил приглашение.
\vs p148 8:3 \P\ Приблизительно в это время в вифсаидский лагерь прибыл некий Кирмет, экстатический пророк из Багдада. У этого мнимого пророка в состоянии транса были необычные видения, и когда в этом состоянии его тревожили, он прорицал необычные сновидения. Он произвел в лагере значительное беспокойство, и Симон Зилот хотел довольно грубо обойтись с этим самообольщающимся притворщиком, но в дело вмешался Иисус, на несколько дней предоставив Кирмету полную свободу действий. Все, кто слышал его проповедь, вскоре поняли, что его учение несостоятельно, если судить о нем с позиции евангелия царства. Вскоре он вернулся в Багдад, взяв с собой лишь полдюжины неустойчивых и колеблющихся душ. Однако до того, как Иисус вступился за багдадского пророка, Давид Зеведеев с несколькими добровольными помощниками взял Кирмета на озеро и, несколько раз окунув его в воду, посоветовал ему убраться восвояси --- организовать и построить свой собственный лагерь.
\vs p148 8:4 \P\ В тот же самый день с финикийкой Бет\hyp{}Марион случился такой припадок фанатизма, что она потеряла рассудок и чуть не утонула, пытаясь ходить по воде, после чего ее отослали ее друзья.
\vs p148 8:5 \P\ Новообращенный фарисей Авраам из Иерусалима передал все свои земные владения в апостольскую казну, и во многом благодаря этому вкладу стало возможным немедленно отправить сто только что подготовленных евангелистов. Андрей уже объявил о закрытии лагеря, и все готовились либо идти домой, либо следовать за евангелистами в Галилею.
\usection{9. Исцеление паралитика}
\vs p148 9:1 Первого октября в пятницу после полудня, когда Иисус проводил свое последнее совещание с апостолами, евангелистами и другими лидерами расформировавшегося лагеря, в присутствии шестерых фарисеев из Иерусалима, сидевших в первом ряду этого собрания в просторном и расширенном зале в доме Зеведея, произошел один из самых странных и уникальных случаев за всю земную жизнь Иисуса. На этот раз Учитель говорил, стоя в этом большом зале, построенном для того, чтобы проводить такие собрания в сезон дождей. Дом со всех сторон окружала огромная толпа народа, напрягавших слух, пытаясь услышать что\hyp{}нибудь из рассуждений Иисуса.
\vs p148 9:2 Во время, когда дом был таким образом заполнен людьми и со всех сторон окружен внимательными слушателями, из Капернаума на небольшой лежанке друзья принесли человека, давно страдавшего параличом. Этот паралитик услышал, что Иисус готовится покинуть Вифсаиду, и, переговорив с каменщиком Аароном, который столь недавно был исцелен, попросил, чтобы его отнесли к Иисусу, где бы он смог искать исцеления. Друзья больного пытались войти в дом Зеведея и через передние и через задние двери, но здесь столпилось слишком много народа. Однако паралитик отказывался смириться с неудачей; он велел своим друзьям достать лестницы, с помощью которых они забрались на крышу зала, где говорил Иисус, и, разобрав черепицу, начали на веревках спускать больного с постелью. Таким образом страдалец оказался на полу прямо перед Учителем. Увидев, что они сделали, Иисус перестал говорить, тогда как бывшие с ним в зале изумлялись упорству больного и его друзей. Паралитик сказал: «Учитель, я не хочу прерывать твою проповедь, но я полон решимости стать здоровым. Я не из тех, кто, получив исцеление, тут же забывает твое учение. Я хочу быть здоровым, чтобы служить в царстве небесном». И вот, несмотря на то, что болезнь этого человека нашла на него из\hyp{}за того, что он сам неразумно прожил свою жизнь, Иисус, видя веру его, сказал паралитику: «Не бойся, сын; прощены грехи твои. Твоя вера спасет тебя».
\vs p148 9:3 Услышав, что изрек Иисус, фарисеи из Иерусалима вместе с другими книжниками и законниками, сидевшими с ними, стали говорить себе: «Как смеет этот человек такое произносить? Неужели он не понимает, что подобные слова --- богохульство? Кто, кроме Бога, может прощать грехи?» Иисус же, чувствуя в духе своем, что они так рассудили в умах их и между собой, обратился к ним, говоря: «Почему вы так рассуждаете в сердцах ваших? Кто вы такие, чтобы судить меня? Какая разница, скажу ли я этому паралитику: прощены грехи твои, или: встань, возьми свою постель и ходи? Но чтобы вы, видящие все это, могли узнать окончательно, что Сын Человеческий имеет на земле силу и власть прощать грехи, я скажу этому больному человеку: встань, возьми свою постель и иди в дом свой». И когда Иисус сказал это, паралитик встал; люди расступились и он вышел из дома. И те, кто видел это, изумились. Петр распустил собравшихся, и многие молились и славили Бога, признавая, что прежде никогда не видели таких необыкновенных происшествий.
\vs p148 9:4 \P\ И приблизительно в это время прибыли посланцы синедриона и приказали шестерым шпионам вернуться в Иерусалим. Услышав это известие, они стали горячо спорить между собой; когда же они закончили свои прения, главный шпион и двое его помощников вернулись вместе с посланцами в Иерусалим, тогда как трое из фарисеев\hyp{}шпионов признались в вере в Иисуса и немедленно отправились к озеру, были крещены Петром и приняты апостолами в царство.
