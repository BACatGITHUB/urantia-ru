\upaper{49}{Обитаемые миры}
\author{Мелхиседек}
\vs p049 0:1 Все обитаемые смертными миры по своему происхождению и природе эволюционные. Эти сферы --- место рождения, эволюционная колыбель смертных рас времени и пространства. Каждое мгновение жизни идущего по пути восхождения является настоящей подготовительной школой к следующей стадии бытия; причем это верно для каждой стадии прогрессивного восхождения человека к Раю; это так же верно для начального существования смертных на эволюционной планете, как верно и для последней школы Мелхиседеков в центре вселенной, которую восходящие смертные посещают непосредственно перед своим перемещением в систему сверхвселенной и достижения ими бытия духа первой стадии.
\vs p049 0:2 \pc В основном все обитаемые миры сгруппированы для небесного управления в локальные системы, причем каждая из этих локальных систем включает приблизительно одну тысячу эволюционных миров. Это ограничение установлено указом Древних Дней и относится к фактически существующим эволюционным планетам, где живут смертные, которые обладают потенциальной возможностью продолжения существования в посмертии. К этой группе не принадлежат ни миры, установленные в свете и жизни, ни планеты, находящиеся на дочеловеческой стадии развития.
\vs p049 0:3 \pc Сама же Сатания является незавершенной системой, содержащей всего 619 обитаемых миров. Таким планетам присвоены порядковые номера в соответствии с их регистрацией как обитаемых миров, как миров, населенных созданиями, обладающими волей. Так, Урантии присвоен номер \bibemph{606 Сатании;} это означает, что в данной локальной вселенной это --- 606\hyp{}й мир, где продолжительный процесс эволюции увенчался появлением людей. В Сатании 36 необитаемых планет, приближающихся к стадии дарования жизни, и несколько подготавливается для Носителей Жизни. Существует почти двести сфер, развивающихся таким образом, что в ближайшие несколько миллионов лет они будут готовы к насаждению жизни.
\vs p049 0:4 Не все планеты приспособлены к тому, чтобы приютить смертную жизнь. Малые планеты, имеющие высокую скорость вращения вокруг оси, совершенно не приспособлены для живых обитателей. В нескольких физических системах Сатании планеты, вращающиеся вокруг находящегося в центре солнца, слишком велики для обитания, так как их огромная масса создает подавляющую гравитацию. Многие из этих огромных сфер имеют спутники, иногда до полдюжины и больше, причем эти луны часто имеют размеры, очень близкие к размерам Урантии, так что для обитания они почти идеальны.
\vs p049 0:5 Старейший обитаемый мир Сатании, мир номер один, --- это Анова, один из сорока четырех спутников, вращающихся вокруг огромной темной планеты, но он различным образом освещается светом трех расположенных рядом с ним солнц. Анова находится на продвинутой стадии прогрессивной цивилизации.
\usection{1. Планетарная жизнь}
\vs p049 1:1 Пространственно\hyp{}временные вселенные развиваются постепенно; совершенствование жизни --- земной или небесной --- не является ни случайным, ни магическим. Космическая эволюция не всегда может быть понятной (предсказуемой), но она совершенно не случайна.
\vs p049 1:2 Биологической единицей материальной жизни является протоплазменная клетка --- совокупное объединение химической, электрической и других основных энергий. Химические формулы могут отличаться в каждой системе, и метод размножения живой клетки в каждой локальной вселенной может быть несколько иной, однако Носители Жизни всегда являются живыми катализаторами, инициирующими первоначальные реакции материальной жизни; они --- зачинатели энергетических контуров живой материи.
\vs p049 1:3 Всем мирам локальной системы присуще несомненное физическое родство; тем не менее, у каждой планеты свое собственное устройство жизни, и не существует даже пары миров, совершенно одинаковых по имеющимся там видам растений и животных. Эти планетарные отличия в типах жизни системы --- результат решений Носителей Жизни. Но эти существа не капризны и не эксцентричны; управление вселенными они осуществляют согласно закону и порядку. Законы Небадона являются божественными указами Спасограда, и эволюционный порядок жизни в Сатании соответствует эволюционному паттерну Небадона.
\vs p049 1:4 Эволюция есть принцип человеческого развития, но сам процесс сильно отличается в разных мирах. Иногда жизнь зарождается в одном центре, а иногда, как это было на Урантии, в трех. В мирах с атмосферой она обычно (но не всегда) имеет морское происхождение; многое зависит от физического статуса планеты. Носители Жизни в своей функции зарождения жизни обладают огромной широтой выбора.
\vs p049 1:5 В развитии планетарной жизни растительная форма всегда предшествует животной и до дифференцирования животных паттернов бывает уже полностью развитой. Все типы животных произошли от основных паттернов предшествовавшего им растительного царства живых вещей; раздельно они не возникают.
\vs p049 1:6 Ранние стадии эволюции жизни не вполне соответствуют вашим современным представлениям. \bibemph{Смертный человек вовсе не эволюционная случайность.} Существует точный метод, всемирный закон, определяющий развертывание системы планетарной жизни на сферах пространства. Время и создание огромного числа видов при этом не являются определяющими. Мыши размножаются намного быстрее слонов, тем не менее, слоны развиваются быстрее, чем мыши.
\vs p049 1:7 Процесс планетарной эволюции упорядочен и управляем. Развитие высших организмов из низших видов жизни происходит не случайно. Иногда эволюционный процесс временно задерживается разрушением определенных благоприятных линий живой плазмы, содержащихся в избранных видах. Часто для восполнения ущерба, нанесенного утратой всего одной высшей черты человеческой наследственности, требуется не один век. Эти избранные или высшие наследственные черты живой протоплазмы следует ревностно и разумно оберегать с момента их появления. Причем в большинстве обитаемых миров эти высшие потенциалы жизни ценятся намного больше, чем на Урантии.
\usection{2. Планетарные физические типы}
\vs p049 2:1 В каждой системе существует основной стандартный паттерн растительной и животной жизни. Однако Носители Жизни зачастую сталкиваются с необходимостью изменения этих основных паттернов, чтобы обеспечить их соответствие различным физическим условиям, с которыми Носители Жизни встречаются в многочисленных мирах пространства. Они способствуют развитию обобщенного системного типа смертного создания, однако существует семь отчетливых типов, а также тысячи тысяч небольших отклонений от этих семи преимущественных видоизменений:
\vs p049 2:2 \ublistelem{1.}\bibnobreakspace Атмосферные типы.
\vs p049 2:3 \ublistelem{2.}\bibnobreakspace Типы стихий.
\vs p049 2:4 \ublistelem{3.}\bibnobreakspace Гравитационные типы.
\vs p049 2:5 \ublistelem{4.}\bibnobreakspace Температурные типы.
\vs p049 2:6 \ublistelem{5.}\bibnobreakspace Электрические типы.
\vs p049 2:7 \ublistelem{6.}\bibnobreakspace Типы получения энергии.
\vs p049 2:8 \ublistelem{7.}\bibnobreakspace Не названные типы.
\vs p049 2:9 \pc В систему Сатания входят все эти типы и многочисленные промежуточные группы, хотя некоторые из них представлены весьма скудно.
\vs p049 2:10 \pc \ublistelem{1.}\bibnobreakspace \bibemph{Атмосферные типы.} Физические отличия миров, на которых обитают смертные, главным образом определяются природой атмосферы; влияние остальных факторов сравнительно мало сказывается на планетарной дифференциации жизни.
\vs p049 2:11 Существующее состояние атмосферы Урантии почти идеально для дышащего типа человека, однако человеческий тип может быть модифицирован таким образом, что он сможет жить как на сверхатмосферных, так и на субатмосферных планетах. Подобные модификации распространяются и на животную жизнь, сильно отличающуюся на различных обитаемых сферах. Как в суб\hyp{}так и в сверхатмосферных мирах существуют чрезвычайно многообразные модификации типов животных.
\vs p049 2:12 Из атмосферных типов Сатании около двух с половиной процентов являются слабодышащими, около пяти процентов --- сильнодышащими и более девяносто одного процента среднедышащими, что в общей сложности составляет девяносто восемь с половиной процентов миров Сатании.
\vs p049 2:13 Существа, подобные расам Урантии, относятся к среднедышащим; вы представляете собой средний или типичный дышащий чин смертного бытия. Если же разумным существам пришлось бы существовать на планете с атмосферой, похожей на атмосферу вашей ближайшей соседки Венеры, то они бы относились к группе сильнодышащих, тогда как существа, населяющие планету с атмосферой, столь же разреженной, как атмосфера вашего внешнего соседа Марса, назывались бы слабодыщащими.
\vs p049 2:14 Если же смертные обитали бы на планете, лишенной воздуха, такой, как ваша луна, то они принадлежали бы к особому чину недышащих. Этот тип представляет собой радикальное или экстремальное приспособление к планетарной среде и рассматривается отдельно. Недышащие и составляют оставшиеся полтора процента миров Сатании.
\vs p049 2:15 \pc \ublistelem{2.}\bibnobreakspace \bibemph{Типы стихий.} Эти видоизменения связаны с отношением смертных к воде, воздуху и земле, причем существует четыре отчетливых вида разумной жизни в плане их связи с этими средами. Урантийские расы относятся к наземному чину.
\vs p049 2:16 Для вас представить себе окружающую среду, которая преобладала в ранние периоды некоторых миров, практически невозможно. Эти необычные условия вынуждают развивающуюся животную жизнь оставаться в морской питательной среде на протяжении периодов более продолжительных, чем на планетах, очень рано предоставляющих гостеприимную наземно\hyp{}атмосферную среду. Напротив, в некоторых мирах, населенных сильнодышащими, если планета не слишком велика, иной раз целесообразно предусмотреть тип смертного, который мог бы без труда перемещаться в атмосфере. Эти воздухоплаватели иногда занимают промежуточное положение между водными и наземными группами, причем, как правило, они в основном живут на земле и в конечном итоге превращаются в обитателей суши. Однако есть миры, где они на протяжении многих веков продолжают летать даже после того, как становятся существами наземного типа.
\vs p049 2:17 И поразительно и любопытно наблюдать раннюю цивилизацию первобытной расы людей, формирующихся в одном случае в воздухе и на верхушках деревьев, а в другом --- на мелководье тенистых укрытых тропических бухт, а также на дне, на краях и на берегах этих морских садов первобытных рас таких необычайных сфер. Даже на Урантии был продолжительный период, во время которого первобытный человек сохранялся и развивал свою примитивную цивилизацию, живя большей частью на вершинах деревьях, как делали это его обитавшие на деревьях предки. У вас на Урантии до сих пор есть группа крохотных млекопитающих (семейство летучих мышей), которые являются воздухоплавателями, причем ваши тюлени и киты, обитающие в морской среде, также относятся к чину млекопитающих.
\vs p049 2:18 В Сатании из типов стихий семь процентов являются водными, десять воздушными, семьдесят --- наземными и тринадцать комбинированными наземно\hyp{}воздушными типами. Однако эти модификации первых разумных существ не являются ни человеко\hyp{}рыбами, ни человеко\hyp{}птицами. Они являются человеческими и до\hyp{}человеческими типами --- ни суперрыбами, ни чудо\hyp{}птицами, а именно --- смертными.
\vs p049 2:19 \pc \ublistelem{3.}\bibnobreakspace \bibemph{Гравитационные типы.} Согласно модификации творческого замысла, разумные существа бывают устроены так, что могут свободно функционировать в сферах как больших, так и меньших Урантии и, таким образом, являются достаточно приспособленными к гравитации тех планет, которые по своим размерам и плотности не идеальны для жизни.
\vs p049 2:20 Различные планетарные типы смертных отличаются ростом, причем средний рост в Небадоне чуть меньше семи футов. Некоторые из более крупных миров населены существами, чей рост составляет всего два с половиной фута. Рост смертного колеблется от этой величины до среднего роста на планетах средних размеров и далее приблизительно до десяти футов в малых обитаемых сферах. В Сатании существует только одна раса с ростом меньше четырех футов. Двадцать процентов обитаемых миров Сатании населено смертными модифицированных гравитационных типов, занимающих наиболее крупные и малые планеты.
\vs p049 2:21 \pc \ublistelem{4.}\bibnobreakspace \bibemph{Температурные типы.} Возможно создать живые существа, способные переносить температуры как намного выше, так и намного ниже температурного диапазона жизни урантийских рас. Существует пять отчетливых чинов существ, классифицируемых по механизмам терморегуляции. На этой шкале урантийские расы имеют номер три. Тридцать процентов миров Сатании населено расами модифицированных температурных типов. Двенадцать процентов относятся к верхним диапазонам температур, восемнадцать --- к нижним, если сравнивать с жителями Урантии, которые функционируют в группе средних температур.
\vs p049 2:22 \pc \ublistelem{5.}\bibnobreakspace \bibemph{Электрические типы.} Электрические, магнитные и электронные свойства миров сильно отличаются. Существует десять видов устройства смертной жизни, по\hyp{}разному приспособленных к тому, чтобы переносить различную энергию сфер. Эти десять разновидностей также несколько по\hyp{}разному реагируют на химические лучи обычного солнечного света. Однако такие незначительные отличия никоим образом не влияют на интеллектуальную или духовную жизнь.
\vs p049 2:23 Из электрических групп смертной жизни почти двадцать три процента принадлежат к классу номер четыре, то есть к урантийскому типу бытия. Эти типы распределены следующим образом: номер 1 --- один процент; номер 2 --- два процента; номер 3 --- пять процентов; номер 4 --- двадцать три процента; номер 5 --- двадцать семь процентов; номер 6 --- двадцать четыре процента; номер 7 --- восемь процентов; номер 8 --- пять процентов; номер 9 --- три процента; номер 10 --- два процента --- округленно.
\vs p049 2:24 \pc \ublistelem{6.}\bibnobreakspace \bibemph{Типы получения энергии.} По способу получения энергии не все миры одинаковы. Не все обитаемые миры имеют океан атмосферы, подходящей для дыхательного газообмена, такой, как на современной Урантии. На наиболее ранних и наиболее поздних стадиях развития многих планет существа вашего теперешнего типа жить не могли бы; а когда респираторные факторы планеты очень высоки либо очень низки, а все остальные предварительные условия являются подходящими для разумной жизни, тогда в таких мирах Носители Жизни часто устанавливают модифицированную форму смертного существования --- это существа, способные осуществлять жизненно важные процессы обмена непосредственно, при помощи энергии света и прямых преобразований мощи Мастерами\hyp{}Физическими Контролерами.
\vs p049 2:25 Существует шесть различных типов питания животных и смертных: слабодышащие используют первый тип питания; морские обитатели --- второй; среднедышащие (так же, как на Урантии) --- третий. Сильнодышащие используют четвертый тип поглощения энергии, а недышащие --- пятый тип питания и энергии. Шестой способ получения энергии используют только срединные создания.
\vs p049 2:26 \pc \ublistelem{7.}\bibnobreakspace \bibemph{Неназванные типы.} В планетарной жизни существуют и другие многочисленные физические разновидности, однако все их различия полностью определяются анатомической модификацией, физиологической дифференциацией и электрохимической приспособленностью. Подобные отличия не затрагивают интеллектуальную и духовную жизнь.
\usection{3. Миры недышащих}
\vs p049 3:1 Большинство обитаемых планет заселено разумными существами дышащего типа. Но существуют и чины смертных, способные жить в мирах с разреженной атмосферой или вообще без нее. Среди обитаемых миров Орвонтона число сфер, где могут жить существа данного типа, менее семи процентов. В Небадоне это меньше трех процентов. Во всей Сатании только девять таких миров.
\vs p049 3:2 В Сатании так мало обитаемых миров, населенных недышащими существами, потому, что эта позднее формированная часть Норлатиадека до сих пор изобилует метеоритными пространственными телами, и миры, лишенные защитного слоя атмосферы, подвергаются непрекращающейся бомбардировке этими скитальцами. Из метеорных сгустков состоят даже ядра некоторых комет; но, как правило, это разъединенные на части материальные тела малых размеров.
\vs p049 3:3 Ежедневно в атмосферу Урантии попадают миллионы и миллионы метеоритов, движущихся со скоростью почти двести миль в секунду. В мирах, обитаемых недышащими существами, развитым расам приходится многое предпринимать, чтобы защитить себя от ущерба, наносимого метеорами; для этого используются электрические установки, действующие таким образом, что метеориты ими или поглощаются, или отводятся. Огромная опасность подстерегает недышащих, когда они решаются выйти за пределы этих защищенных зон. Эти миры также подвержены разрушительным электрическим бурям, природа которых на Урантии неизвестна. Во время таких гигантских флуктуаций энергии обитатели должны спасаться в специальных сооружениях с защитной изоляцией.
\vs p049 3:4 Жизнь в мирах недышащих существ радикально отличается от жизни на Урантии. Недышащие существа не едят пищу и не пьют воду, как это делают урантийские расы. Реакции нервной системы, механизм теплорегулирования и метаболизм этих особых обитателей радикально отличается от подобных же функций у смертных Урантии. За исключением размножения почти все акты жизнедеятельности у них другие, и несколько отличаются даже способы деторождения.
\vs p049 3:5 В мирах недышащих существ животные виды совершенно непохожи на те, что встречаются на планетах с атмосферой. Образ жизни недышащих существ отличается от способа существования в мире с атмосферой; отличается даже продолжение существования в посмертии обитателей этих миров, являющихся кандидатами на слияние с Духом. Тем не менее, эти существа наслаждаются жизнью и развивают различные виды деятельности в своем мире практически с такими же злоключениями и радостями, что переживают и смертные, живущие в мирах с атмосферой. Умом и характером недышащие существа не отличаются от других типов смертных.
\vs p049 3:6 Вас, конечно, более чем заинтересует планетарное поведение этого типа смертных, поскольку подобная раса существ населяет сферу, расположенную совсем рядом с Урантией.
\usection{4. Эволюционные создания, обладающие волей}
\vs p049 4:1 Между смертными различных миров, даже среди тех, что принадлежат к одним и тем же интеллектуальным и физическим типам, существуют большие различия, однако все смертные, обладающие волей, являются прямоходящими животными, двуногими.
\vs p049 4:2 Существует шесть основных эволюционных рас: три первичных --- красная, желтая и голубая и три вторичных --- оранжевая, зеленая и синяя; однако на многих планетах, населенных существами с мозгом, состоящим из трех частей, живут только три первичных типа. Некоторые локальные системы тоже имеют только эти три расы.
\vs p049 4:3 В среднем люди наделены двенадцатью особыми физическими чувствами, хотя особые чувства смертных с мозгом, состоящим из трех частей, несколько более развитые, чем у типов с мозгом, состоящим из одной и двух частей; они видят и слышат значительно лучше, чем расы Урантии.
\vs p049 4:4 Дети, как правило, рождаются по одному, рождение нескольких детей является исключением, и семейная жизнь достаточно единообразна на всех типах планет. Во всех развитых мирах преобладает равенство полов; мужчины и женщины равны в даровании разума и духовном статусе. Мы не считаем планету вышедшей из состояния варварства до тех пор, пока одн пол стремятся угнетать другой. Эта особенность опыта создания всегда в значительной степени изменяется к лучшему после прибытия Материального Сына и Дочери.
\vs p049 4:5 \pc На всех освещаемых и обогреваемых солнцем планетах происходят смена времен года и температурные изменения. Сельское хозяйство существует повсеместно во всех мирах с атмосферой; возделывание земли --- это единственное занятие, общее для развивающихся рас всех подобных планет.
\vs p049 4:6 Все смертные в свои первые дни ведут одинаковую борьбу со своими микроскопическими врагами, подобную которой вы переживаете сейчас на Урантии, хотя, быть может, и не такую обширную. Продолжительность жизни на различных планетах изменяется от двадцати пяти лет в примитивных мирах до почти пятисот в более развитых и более старых сферах.
\vs p049 4:7 Все люди, и в племени, и в расе --- существа общественные. Эти групповые расслоения свойственны их природе и строению. Подобные тенденции могут быть изменены лишь совершенствующейся цивилизацией и постепенным одухотворением. Социальные и экономические проблемы, проблемы управления обитаемыми мирами изменяются в соответствии с возрастом планет и степенью влияния пребывающих в этих мирах, сменяя друг друга, божественных Сыновей.
\vs p049 4:8 \pc Разум --- это дарование Бесконечного Духа; в различных средах обитания разум действует одинаково. Разумы смертных схожи друг с другом, несмотря на определенные структурные и химические отличия, характеризующие природные свойства обладающих волей творений локальных систем. Несмотря на личностные или физические планетарные отличия, умственная жизнь всех этих различных чинов смертных очень похожа, а деятельность, начинающаяся сразу после смерти, почти одинакова.
\vs p049 4:9 Но без бессмертного духа разум смертного в посмертии существовать не может. Разум человека смертен; бессмертен только дарованный человеку дух. Продолжение существования в посмертии зависит от одухотворения, происходящего в результате служения Настройщика, --- от рождения и развития бессмертной души; по крайней мере, не должен развиваться антагонизм по отношению к выполняемой Настройщиком миссии духовного преобразования материального разума.
\usection{5. Планетарные выпуски смертных}
\vs p049 5:1 Дать точное описание планетарных выпусков смертных будет довольно сложно, потому что вы о них знаете так мало, а их разновидностей существует так много. Смертные создания можно, однако, изучать с различных точек зрения, в частности следующих:
\vs p049 5:2 \ublistelem{1.}\bibnobreakspace Приспособление к планетарной среде.
\vs p049 5:3 \ublistelem{2.}\bibnobreakspace Выпуски, определяемые типом мозга.
\vs p049 5:4 \ublistelem{3.}\bibnobreakspace Выпуски, определяемые восприятием духа.
\vs p049 5:5 \ublistelem{4.}\bibnobreakspace Планетарные эпохи смертных.
\vs p049 5:6 \ublistelem{5.}\bibnobreakspace Серии родственных созданий.
\vs p049 5:7 \ublistelem{6.}\bibnobreakspace Выпуски, определяемые слиянием с Настройщиком.
\vs p049 5:8 \ublistelem{7.}\bibnobreakspace Методы ухода с земли.
\vs p049 5:9 \pc Обитаемые миры семи сверхвселенных населены смертными, которые одновременно относятся и к какой\hyp{}нибудь одной и к нескольким категориям каждого из этих семи обобщенных классов эволюционной жизни созданий. Но даже эти общие классификации не учитывают наличия ни таких существ, как мидсонитеры, ни некоторых других форм разумной жизни. Обитаемые миры, которые представлены в этих повествованиях, населены эволюционными смертными созданиями, однако существуют и иные формы жизни.
\vs p049 5:10 \pc \ublistelem{1.}\bibnobreakspace \bibemph{Приспособление к планетарной среде.} С точки зрения приспособления жизни творения к планетарной среде существуют три главные группы обитаемых миров: группа нормального приспособления; группа радикального приспособления и экспериментальная группа.
\vs p049 5:11 Виды нормального приспособления к планетарным условиям следуют общим физическим паттернам, рассмотренным ранее. Миры недышащих являются типичными представителями миров радикального, или экстремального, приспособления, однако в эту группу входят и другие типы. Экспериментальные миры обычно идеально адаптированы к типичным формам жизни, и на этих десятичных планетах Носители Жизни пытаются произвести полезные изменения в стандартных укладах жизни. Так как ваш мир является экспериментальной планетой, он заметно отличается от родственных ему сфер в Сатании; на Урантии возникло множество форм жизни, не встречающихся больше нигде; в то же время на вашей планете отсутствуют многие повсеместно распространенные виды.
\vs p049 5:12 Во вселенной Небадона все миры с видоизмененной жизнью последовательно связаны и образуют особую область вселенских дел, за которой наблюдают специально назначенные руководители; причем все эти экспериментальные миры периодически инспектируются отрядом вселенских управителей, возглавляемым ветераном\hyp{}финалитом, известным в Сатании как Табамантия.
\vs p049 5:13 \pc \ublistelem{2.}\bibnobreakspace \bibemph{Выпуски, определяемые типом мозга.} Одной из сторон физического единообразия смертных является наличие у них мозга и нервной системы; тем не менее, существуют три основные структуры строения мозга: мозг, состоящий из одной, двух и трех частей. Жители Урантии обладают мозгом, состоящим из двух частей, у них несколько большее воображение, они больше любят приключения и являются лучшими философами, чем смертные с мозгом, состоящим из одной части, но несколько менее духовны, этичны и менее склонны к богопочитанию, нежели чины с мозгом, состоящим из трех частей. Эти различия в строении мозга характерны даже для дочеловеческого, животного существования.
\vs p049 5:14 Определенное представление о мозге, состоящем из одной части, вы можете получить исходя из сведений о двухполушарном типе головного мозга жителей Урантии. Третью часть мозга, если мозг состоит из трех частей, можно лучше всего представить как развитие вашей низшей или рудиментарной зоны мозга, достигшее такой степени, что начав функционировать, эта часть стала контролировать, главным образом, физические функции тела, что освободило две сильнее развитые части для выполнения более возвышенных задач: одну --- для интеллектуальных, а вторую --- для осуществления соответствующей духовной деятельности Настройщика Мыслей.
\vs p049 5:15 В то время как земные достижения рас с мозгом, состоящим из одной части, по сравнению с созданиями, у которых мозг состоит из двух частей, несколько ограничены, более древние планеты, населенные группой, у которой мозг состоит из трех частей, демонстрируют такой уровень цивилизации, который удивил бы жителей Урантии и в каком\hyp{}то смысле затмил бы вашу, если сравнивать эти цивилизации. В области технического развития и материальной цивилизации и даже в области интеллектуальных достижений миры, в которых обитают смертные с мозгом, состоящим из двух частей, могут сравниться со сферами, населенными существами с мозгом, состоящим из трех частей. Но в высшем контроле над разумом и в развитии интеллектуально\hyp{}духовного обмена вы отчасти им уступаете.
\vs p049 5:16 При сравнении все такие оценки, касающиеся интеллектуального развития или духовных достижений любого мира или группы миров, должны соответствующим образом учитывать возраст планеты; многое, очень многое зависит от возраста, помощи биологических реализаторов подъема и последующих миссий различных чинов божественных Сынов.
\vs p049 5:17 Хотя люди, наделенные мозгом, состоящим из трех частей, способны к чуть более высокой планетарной эволюции, чем чины с мозгом, состоящим из одной или двух частей, все они имеют один и тот же тип жизненной плазмы и осуществляют планетарную деятельность весьма похожими способами --- во многом так же, как это делают люди на Урантии. Эти три типа смертных распределены по всем мирам локальных систем. В большинстве случаев планетарные условия почти никак не сказывались на решениях Носителей Жизни проектировать эти различные чины для разных миров; составлять и исполнять такие планы --- исключительное право Носителей Жизни.
\vs p049 5:18 С точки зрения пути восхождения эти три чина занимают равное положение. Каждый из них должен подняться по одной и той же интеллектуальной лестнице развития, и каждый должен справиться с одними и теми же духовными испытаниями в своем продвижении вперед. В этих различных мирах управление со стороны систем и сверхконтроль со стороны созвездий одинаково свободны от дискриминации; идентично даже правление Планетарных Принцев.
\vs p049 5:19 \pc \ublistelem{3.}\bibnobreakspace \bibemph{Выпуски, определяемые восприятием духа.} Существуют три группы устройства разума, связанных с духовными делами. Эта классификация не имеет отношения к чинам смертных обладающих мозгом, состоящим из одной, двух или трех частей; она относится, главным образом, к химии желез, а точнее, к структуре некоторых желез, сравнимых с гипофизом. Расы в каких\hyp{}то мирах имеют одну железу, в других --- две, как у жителей Урантии, а еще есть сферы, где расы имеют три таких уникальных органа. Природное воображение и духовная восприимчивость определенно зависят от влияния этого особенного химического дара.
\vs p049 5:20 Среди типов, отличающихся по восприятию духа, шестьдесят пять процентов относятся ко второй группе, как и расы Урантии. Двенадцать процентов относятся к первому типу, и по природе своей они менее восприимчивы, тогда как двадцать три процента более тяготеют к духовному во время земной жизни. Однако эти отличия не сохраняются после естественной смерти; все эти расовые различия свойственны лишь жизни во плоти.
\vs p049 5:21 \pc \ublistelem{4.}\bibnobreakspace \bibemph{Планетарные эпохи смертных.} Эта классификация учитывает последовательность временных диспенсаций, то, как они влияют на земной статус человека и на его восприятие небесного служения.
\vs p049 5:22 Жизнь на планетах инициируется Носителями Жизни, которые наблюдают за ее развитием до какого\hyp{}то момента после эволюционного появления смертного человека. Прежде чем покинуть планету, Носители Жизни надлежащим образом ставят во главе ее в качестве правителя мира Планетарного Принца. Вместе с этим правителем прибывают и все его подчиненные и служители\hyp{}помощники, причем одновременно с этим прибытием выносится первое судебное решение в отношении живых и мертвых.
\vs p049 5:23 С появлением человеческих групп этот Планетарный Принц прибывает, дабы ознаменовать наступление человеческой цивилизации и направлять человеческое общество. Ваш полный смятения мир не может служить показателем первых дней правления Планетарных Принцев, ибо в начале подобного управления Урантией ваш Планетарный Принц Калигастия примкнул к восстанию Владыки Системы --- Люцифера. С тех пор ваша планета идет по пути бурь.
\vs p049 5:24 В нормальном эволюционном мире развитие рас достигает своего естественного биологического пика при правлении Планетарного Принца, и вскоре после этого Владыка Системы посылает на такую планету материальных Сына и Дочь. Эти присланные существа служат биологическими реализаторами подъема; невыполнение ими своих обязанностей на Урантии еще больше усложнило вашу планетарную историю.
\vs p049 5:25 Когда интеллектуальное и этическое развитие человечества почти доходит до эволюционных пределов, настает время миссии повелителя --- Райского Сына\hyp{}Авонала; позднее, когда духовный статус такого мира достигает предела своего естественного развития, планету посещает Райский Сын пришествия. Главная миссия Райского Сына пришествия заключается в том, чтобы установить планетарный статус, пустить Дух Истины действовать в масштабе планеты и, таким образом, осуществить всеобщее пришествие Настройщиков Мысли.
\vs p049 5:26 И в этом отношении Урантия отличается от остальных планет: в вашем мире никогда не было миссии повелителя, как не принадлежал к числу Авоналов и ваш Сын пришествия; вашей планете выпала особая честь стать планетой --- смертным домом Сына\hyp{}Владыки, Михаила из Небадона.
\vs p049 5:27 Благодаря служению всех следующих один за другим чинов божественного сыновства, обитаемые миры и их развивающиеся расы начинают приближаться к вершине планетарной эволюции. Такие миры готовы к кульминационной миссии --- прибытию Сынов Троицы\hyp{}Учителей. Эта эпоха Сынов\hyp{}Учителей является преддверием к завершающему планетарному периоду --- эволюционной утопии --- периоду света и жизни.
\vs p049 5:28 Особое внимание этой классификации людей будет уделено в следующей главе.
\vs p049 5:29 \pc \ublistelem{5.}\bibnobreakspace \bibemph{Серии родственных созданий.} Планеты структурно организованы не только по вертикали --- в системы, созвездия и так далее, управление вселенной обеспечивает также и объединение по горизонтали --- в соответствие с типом, выпуском и другими признаками родства. Точнее говоря, это горизонтальное управление вселенной имеет отношение к координации различной деятельности родственной природы, независимо взращенной в различных сферах. Эти родственные классы вселенских созданий периодически инспектируются определенными смешанными отрядами высоких личностей, во главе которых стоят многоопытные финалиты.
\vs p049 5:30 Эти факторы родства явлены на всех уровнях, ибо родственные серии существуют среди не\hyp{}человеческих личностей так же, как и среди смертных созданий, --- они есть даже между человеческими и надчеловеческими чинами. Разумные существа разделены по вертикали на двенадцать больших групп, каждая из которых состоит из семи основных подразделений. Согласование этих уникально связанных между собой групп живых существ, вероятно, осуществляется Верховным Существом каким\hyp{}то не вполне понятным методом.
\vs p049 5:31 \pc \ublistelem{6.}\bibnobreakspace \bibemph{Выпуски, определяемые слиянием с Настройщиком.} Духовная классификация или объединение в группы всех смертных во время их жизненного опыта, предшествующего слиянию, полностью определяется отношением статуса личности к пребывающему в ней Таинственному Помощнику. В отличие от ближайшей вселенной, где лишь чуть больше половины миров служат пристанищем для существ, которые являются кандидатами на вечное слияние с пребывающим в них Настройщиком, почти девяносто процентов обитаемых миров Небадона заселено смертными, сливающимися с Настройщиками.
\vs p049 5:32 \pc \ublistelem{7.}\bibnobreakspace \bibemph{Методы ухода с земли.} В принципе существует только один способ, посредством которого может начинаться человеческая жизнь в обитаемых мирах, --- путем размножения созданий и естественного рождения; однако существует множество способов, с помощью которых человек освобождается от своего земного состояния и получает доступ к движущемуся внутрь потоку восходящих к Раю.
\usection{6. Уход с земли}
\vs p049 6:1 Все отличающиеся друг от друга физические типы и планетарные выпуски смертных одинаково пользуются служением Настройщиков Мысли, ангелов\hyp{}хранительниц и различных чинов сонма вестников Бесконечного Духа. Все они одинаково освобождаются от оков плоти естественной смертью и все одинаково отправляются в моронтийные миры духовной эволюции и совершенствования разума.
\vs p049 6:2 Время от времени по знаку планетарных властей или правителей системы осуществляются особые воскрешения спящих в посмертии. Такие воскрешения происходят, по крайней мере, раз в тысячу лет планетарного времени, когда не все, но «многие из спящих во прахе пробуждаются». В связи с этими особыми воскрешениями мобилизуются особые группы восходящих созданий на особенную службу локально\hyp{}вселенского плана вознесения смертных. С этими особыми воскрешениями связаны как причины практического свойства, так и сентиментальные ассоциации.
\vs p049 6:3 На всем протяжении ранних периодов существования обитаемого мира многие призываются в сферы\hyp{}обители на особых и происходящих раз в тысячу лет воскрешениях, однако большинство продолжающих существование в посмертии реперсонализируется при объявлении новой диспенсации, связанной с пришествием божественного Сына, призванного к планетарному служению.
\vs p049 6:4 \pc \ublistelem{1.}\bibnobreakspace \bibemph{Смертные диспенсационного или группового чина продолжения существования в посмертии.} С прибытием в обитаемый мир первого Настройщика появляются и серафимы\hyp{}хранительницы; они необходимы для ухода с земли. На всем протяжении сна в посмертии духовные ценности и вечные реальности начавших развиваться и бессмертных душ охраняются как святыня личными или групповыми серафимами\hyp{}хранительницами.
\vs p049 6:5 Групповые хранительницы, приданные спящим в посмертии, всегда действуют вместе с Сынами\hyp{}судьями во время их пришествия в мир. «Он пошлет ангелов своих, и они соберут избранных его от четырех ветров». Вместе с каждым серафимом, получившим задание реперсонализировать спящего смертного, действует возвратившийся Настройщик, тот самый фрагмент бессмертного Отца, который пребывал в смертном во дни его жизни во плоти, и, таким образом, восстанавливается идентичность и воскрешается личность. Во время сна своих подопечных ожидающие Настройщики служат в Божеграде; в этот промежуток времени они никогда не пребывают в разуме другого смертного.
\vs p049 6:6 Хотя наиболее старые миры, где существуют смертные, служат пристанищем для высокоразвитых и утонченно духовных типов людей, фактически освобожденных от моронтийной жизни, для ранних периодов развития рас животного происхождения характерны примитивные смертные, которые столь незрелы, что слияние их с Настройщиками невозможно. Пробуждение этих смертных осуществляется серафимом\hyp{}хранительницей вместе с индивидуализированной частью бессмертного духа Третьего Источника и Центра.
\vs p049 6:7 Итак, спящие в посмертии реперсонализируются во время диспенсационных поверок. Что же касается личностей мира, спасти которых невозможно, то в этом случае бессмертный дух не действует вместе с групповой хранительницей предназначения, а это и есть прекращение существования творения. Хотя некоторые ваши записи и изображают эти события как имеющие место на планетах, где умирают смертные, в действительности все это происходит в мирах\hyp{}обителях.
\vs p049 6:8 \pc \ublistelem{2.}\bibnobreakspace \bibemph{Смертные индивидуальных чинов восхождения.} Индивидуальное развитие людей измеряется их последовательным достижением и прохождением (овладением) семи космических кругов. Эти круги продвижения смертных являются уровнями связанных между собой интеллектуальных, социальных, духовных ценностей и ценностей космической проницательности. Начиная с седьмого круга, смертные устремляются к первому, и все, достигшие третьего, немедленно получают прикрепленных к ним личных хранительниц предназначения. Эти смертные могут быть реперсонализированы в моронтийной жизни независимо от диспенсационного или иных судебных решений.
\vs p049 6:9 Но протяжении ранних периодов эволюционного мира лишь немногие смертные предстают перед судом на третий день. Но по мере того, как проходят эпохи, все больше и больше личных хранительниц предназначения приписывается к продвигающимся смертным, и, таким образом, все возрастающее число этих развивающихся творений реперсонализируется в первом мире\hyp{}обители на третий день после естественной смерти. Во время подобных событий возвращение Настройщика свидетельствует о пробуждении человеческой души, а это и есть реперсонализация умершего, столь же действенная, как при массовой поверке, объявляемой в эволюционных мирах в конце диспенсации.
\vs p049 6:10 Существует три группы индивидуально идущих по пути восхождения: менее развитые прибывают в начальный или первый мир\hyp{}обитель. Группа более развитых может начать моронтийный путь в любом из промежуточных миров\hyp{}обителей в соответствии с предшествующим планетарным продвижением. Наиболее развитые из этих чинов начинают свое моронтийное существование уже в седьмом мире\hyp{}обители.
\vs p049 6:11 \pc \ublistelem{3.}\bibnobreakspace \bibemph{Смертные чинов восхождения с испытательным сроком.} С точки зрения вселенной прибытие Настройщика определяет идентичность, и все существа, в которых пребывает Настройщик, находятся в списке потенциально продолжающих существование в посмертии. Однако временная жизнь в эволюционных мирах полна неопределенности, и многие умирают в юности, так и не избрав путь, ведущий к Раю. Такие дети и юноши, в которых пребывает Настройщик, следуют за родителем с наиболее продвинутым духовным статусом и, таким образом, на третий день, при особом воскрешении или при регулярных, объявляемых раз в тысячу лет и диспенсационных поверках отправляются в испытательные ясли, находящиеся в мире финалитов системы.
\vs p049 6:12 Дети, умершие в возрасте, слишком юном для того, чтобы иметь Настройщиков Мысли, реперсонализируются в мире финалитов локальных систем одновременно с прибытием любого из родителей в миры\hyp{}обители. Физическое существование ребенок обретает при смертном рождении, но в плане продолжения существования в посмертии все дети, лишенные Настройщиков, все еще считаются связанными со своими родителями.
\vs p049 6:13 В должное время Настройщики Мысли приходят, чтобы пребывать в этих малышах, тогда как служение серафимов продолжающим существование в посмертии, которые входят в обе группы чинов с испытательным сроком, в общем похоже на служение наиболее продвинутого из родителей или эквивалентно служению родителя в случае, если в посмертии продолжает существовать только один из них. Достигшим третьего круга независимо от статуса их родителей предоставляются личные хранители.
\vs p049 6:14 Такие же испытательные ясли есть и в сферах финалитов созвездия, и в центре вселенной; эти ясли предназначены для лишенных Настройщиков детей первичных и вторичных модифицированных чинов восходящих.
\vs p049 6:15 \pc \ublistelem{4.}\bibnobreakspace \bibemph{Смертные вторичных модифицированных чинов восхождения.} Это продвинувшиеся люди промежуточных эволюционных миров. Как правило, они не защищены от естественной смерти, но освобождены от прохождения через семь миров\hyp{}обителей.
\vs p049 6:16 Группа менее совершенных пробуждается в центре своей локальной системы, минуя только миры\hyp{}обители. Промежуточная группа отправляется в учебные миры созвездия, минуя весь моронтийный строй локальной системы. В более позднии планетарные периоды духовных устремлений, многие продолжающие существование в посмертии пробуждаются в центре созвездия и там начинают свое восхождение к Раю.
\vs p049 6:17 Но прежде чем любая из этих групп сможет продолжить свой путь, они должны в качестве инструкторов отправиться в обратное путешествие к мирам, мимо которых они прошли, и обрести большой опыт в качестве учителей в тех мирах, которые они пропустили, будучи учениками. Впоследствии все они продолжают идти к Раю по предопределенным путям продвижения смертных.
\vs p049 6:18 \pc \ublistelem{5.}\bibnobreakspace \bibemph{Смертные первичного модифицированного чина восхождения.} Эти смертные принадлежат к слившемуся с Настройщиком типу эволюционной жизни, однако чаще всего они представляют завершающие фазы человеческого развития в развивающемся мире. Эти чудесные существа освобождены от прохождения через врата смерти; они охватываются Сыном; их переносят от живущих, и они немедленно появляются в присутствии Сына\hyp{}Владыки в центре локальной вселенной.
\vs p049 6:19 Это --- смертные, сливающиеся со своим Настройщиком во время смертной жизни, и такие слившиеся с Настройщиком личности свободно пересекают пространство прежде, чем их облекут в моронтийные формы. Эти слившиеся души Настройщик переносит прямо в залы воскрешения высших моронтийных сфер, где они получают свое начальное моронтийное облачение так же, как и остальные смертные, прибывающие из эволюционных миров.
\vs p049 6:20 Этот первичный модифицированный порядок восхождения смертных может применяться к индивидуумам в любом из планетарных выпусков в мирах с разной стадией слияния с Настройщиками, от самых низших до самых высших, --- однако чаще он действует в более старых из этих сфер после того, как эти сферы испытали благотворное влияние многочисленных пребываний божественных Сынов.
\vs p049 6:21 С установлением планетарной эры света и жизни многие отправляются в моронтийные миры вселенной, используя первичный модифицированный порядок перемещения. Далее, на продвинутых стадиях установленного существования, когда большинство смертных, покидающих мир, включаются в этот класс, планета считается принадлежащей к этому выпуску. На этих прочно установленных в свете и жизни сферах естественная смерть становится все менее и менее частой.
\vsetoff
\vs p049 6:22 [Представлено Мелхиседеком Школы Планетарного Управления Иерусема.]
