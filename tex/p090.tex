\upaper{90}{Шаманизм --- знахари и жрецы}
\vs p090 0:1 Эволюция религиозных ритуалов двигалась от умиротворения, уклонения, экзорцизма, принуждения, примирения и умилостивления к жертвоприношению, искуплению и спасению. Религиозный ритуал развивался от форм примитивного культа к фетишам и далее к магии и чудесам; и по мере того, как ритуал становился все сложнее вследствие усложнения представлений человека о сверхчеловеческих сферах, в нем неизбежно стали играть главенствующую роль знахари, шаманы и жрецы.
\vs p090 0:2 Согласно развивающимся представлениям примитивного человека, постепенно сформировалось убеждение, что духовный мир не обращает внимание на простого смертного. Только молитвам избранных смертных внемлют боги; только необыкновенные мужчина или женщина могут быть услышаны духами. Религия, таким образом, входит в новую фазу --- она постепенно начинает восприниматься «из вторых рук»; между адептом религии и объектом поклонения всегда как посредник выступает знахарь, шаман или священник. И в настоящее время большинство существующих систем религиозных верований проходят через эту ступень эволюционного развития.
\vs p090 0:3 Эволюционная религия родилась из обыкновенного и всепоглощающего страха, который охватывает человеческий разум, когда тот сталкивается с неизвестным, необъяснимым и непостижимым. Со временем религия достигает чрезвычайно ясного осознания всеобъемлющей любви, любви, которая непреодолимо охватывает человеческую душу, когда та пробуждается к пониманию безграничной любви Отца Всего Сущего к сыновьям вселенной. Но между началом и завершением религиозной эволюции лежит долгая эпоха шаманов, которые позволяют себе стоять между человеком и Богом в качестве посредников, толкователей и ходатаев.
\usection{1. Первые шаманы --- знахари}
\vs p090 1:1 Шаман был колдуном, занимающим высокое положение, ритуальным человеком\hyp{}фетишем и центральной фигурой всех обрядов эволюционной религии. У многих племен шаман стоял выше военного вождя, что знаменовало начало господства церкви над государством. Иногда шаман исполнял роль жреца и даже жреца\hyp{}царя. Впоследствии у некоторых племен были как шаманы\hyp{}колдуны (провидцы), так и появившиеся позже шаманы\hyp{}жрецы. И во многих случаях звание шамана передавалось по наследству.
\vs p090 1:2 Поскольку в древние времена все ненормальное приписывалось одержимости духами, любое заметное отклонение от психической или физической нормы квалифицировалось как признак способности быть знахарем. Многие из таких мужчин страдали эпилепсией, многие из женщин --- истерией, и этими двумя типами объяснялась большая часть древних озарений, равно как и случаев одержимости духами и дьяволами. Немалое число этих первых жрецов относились к такому типу, который позже стали именовать параноиками.
\vs p090 1:3 Хотя в малозначительных случаях они, возможно, и прибегали к обману, но огромное большинство шаманов верили в факт своей одержимости духами. Женщины, которые могли вводить себя в состояние транса или каталепсии, становились могущественными шаманами; позже такие женщины считались пророчицами и спиритическими медиумами. Обычно их каталептический транс сопровождался якобы общением с духами умерших. Многие из женщин\hyp{}шаманов были также профессиональными танцовщицами.
\vs p090 1:4 Но не все шаманы пребывали в самозаблуждении; многие были ловкими и искусными обманщиками. С развитием этого ремесла, чтобы стать знахарем, новичку уже было необходимо пройти десятилетний период ученичества, полный тягот и самоотречения. Шаманы выработали профессиональную манеру одеваться и нарочито таинственный стиль поведения. Они часто пользовались наркотическими средствами, чтобы вызвать определенные физические состояния, которые могли произвести впечатление и заинтриговать соплеменников. Ловкость рук воспринималась простым народом как сверхъестественное, и хитроумные жрецы первыми стали пользоваться чревовещанием. Многие из древних шаманов случайно открыли для себя гипноз; другие занимались самогипнозом, долго и пристально глядя на свой пупок.
\vs p090 1:5 Хотя многие и прибегали к таким трюкам и обманам, их репутация как касты, по большому счету, держалась на несомненных достижениях. Когда шаман терпел неудачу и не мог правдоподобно оправдаться, то его или прогоняли с должности, или убивали. Таким образом, честные шаманы быстро погибали; выживали только искусные актеры.
\vs p090 1:6 Именно шаманизм забрал полновластное руководство делами племени из рук старых и сильных, передав его в руки ловких, умных и дальновидных.
\usection{2. Практика шаманизма}
\vs p090 2:1 Заклинание духов было очень точной и чрезвычайно сложной процедурой, сравнимой с современными церковными обрядами, проводимыми на древнем языке. Человеческий род очень рано стал искать сверхчеловеческой помощи, \bibemph{откровения;} и люди верили, что шаманы действительно получают такие откровения. Хотя шаманы в своей деятельности использовали огромную силу внушения, это внушение почти неизменно носило негативный характер; только в относительно недавние времена стали прибегать к технике позитивного внушения. На раннем этапе становления своего ремесла шаманы специализировались в таких занятиях, как вызывание дождя, исцеление болезней и расследование преступлений. Однако главное для знахаря было не исцелять болезни, а предугадывать опасности жизни и предотвращать их.
\vs p090 2:2 Древняя черная магия, как религиозная, так и светская, называлась белой магией, когда ею занимались жрецы, провидцы, шаманы или знахари. Занимающихся черной магией называли кудесниками, волшебниками, магами, ведьмами, чародеями, колдунами, заклинателями и ворожеями. С течением времени всякое такое якобы соприкосновение со сверхъестественным стало квалифицироваться или как ведьмовство, или как шаманство.
\vs p090 2:3 Колдовство включало в себя \bibemph{маги} \bibemph{ю,} творимую с помощью древних, неправильных и непризнанных духов; шаманство подразумевало \bibemph{чудеса,} творимые принятыми духами и признанными богами племени. В более поздние времена ведьма стала ассоциироваться с дьяволом, и, таким образом, сформировались предпосылки для многочисленных относительно недавних проявлений религиозной нетерпимости. Колдовство было религией у многих примитивных племен.
\vs p090 2:4 Шаманы безоговорочно верили в то, что случайности открывают волю духов; они часто бросали жребий, чтобы принять решение. В настоящее время эта страсть бросать жребий находит выражение не только в многочисленных азартных играх, но и во всем известных считалочках. Некогда человек, на котором останавливался счет, должен был умереть; теперь он только \bibemph{водит} в какой\hyp{}нибудь детской игре. То, что было серьезным делом для древнего человека, сохранилось как забава у современных детей.
\vs p090 2:5 Знахарь чрезвычайно доверял знакам и знамениям, таким как: «Когда вы услышите шелест верхушек тутовых деревьев, тогда вам следует энергично браться за дело». На заре человеческой истории шаманы обратили внимание на звезды. Во всем мире верили в примитивную астрологию и занимались ею; повсеместно было распространено и толкование снов. Вслед за всем этим вскоре появились импульсивные женщины\hyp{}шаманы, претендующие на умение общаться с духами умерших.
\vs p090 2:6 Появившись в древние времена, вызыватели дождя, или шаманы погоды, существовали на протяжении всех эпох. Для древних земледельцев жестокая засуха означала смерть; одной из главных задач древней магии было управление погодой. Даже у цивилизованного человека погода все еще обычная тема для разговоров. Все древние народы верили в способность шамана вызывать дождь, но когда он терпел неудачу и не мог убедительно оправдаться, объясняя ее, то его, как правило, убивали.
\vs p090 2:7 Кесари вновь и вновь изгоняли астрологов, но те неизменно возвращались из\hyp{}за массовой веры в их силу. Их невозможно было изгнать, и даже в шестнадцатом веке после Рождества Христа властители Западной церкви и государств покровительствовали астрологии. Тысячи, казалось бы, умных людей до сих пор верят, что человек может родиться под счастливой или несчастливой звездой; что расположение небесных светил определяет исход разных земных событий. Легковерные люди до сих пор покровительствуют предсказателям.
\vs p090 2:8 Греки верили в действенность совета оракула, китайцы пользовались магией для защиты от демонов, в Индии процветал шаманизм, который до сих пор распространен в Центральной Азии. Во многих местах на земле лишь недавно отказались от практики шаманизма.
\vs p090 2:9 Время от времени появлялись истинные пророки и учителя, осуждавшие и разоблачавшие шаманизм. Даже у исчезающих красных людей в последние сто лет был такой пророк, Шони Тенскватава, предсказавший затмение солнца в 1808 году и осуждавший пороки белых. Много истинных учителей появлялось у разных племен и народов на протяжении долгих веков истории эволюции. И они вечно будут продолжать появляться, бросая вызов шаманам или священникам любых эпох, противодействующим всеобщему образованию и пытающимся мешать научному прогрессу.
\vs p090 2:10 Многими путями и хитроумными методами древние шаманы завоевали себе репутацию гласа Божия и стражей провидения. Они окропляли новорожденных водой и даровали им имена; они совершали обрезание мальчиков. Они руководили всеми похоронными церемониями и соответствующим образом сообщали о благополучном прибытии умерших в страну духов.
\vs p090 2:11 Шаманы\hyp{}жрецы и знахари часто становились очень богатыми, потому что накапливали получаемые различные вознаграждения, предназначенные якобы на пожертвования духам. Нередко у шамана сосредотачивалось практически все материальное богатство его племени. После смерти богатого человека было принято делить его собственность, отдавая треть шаману и треть на общественные или благотворительные нужды. Этот принцип до сих пор соблюдается в некоторых районах Тибета, где половина всего мужского населения принадлежит к этому непроизводительному классу.
\vs p090 2:12 Шаманы хорошо одевались и обычно имели по несколько жен; они были первыми аристократами, и на них не распространялись никакие племенные ограничения. Очень часто они не отличались высоким интеллектом и нравственностью. Они устраняли соперников, объявляя их или ведьмами, или чародеями, и сплошь и рядом обретали такое влияние и могущество, что могли управлять вождями или царями.
\vs p090 2:13 Примитивный человек относился к шаману, как к неизбежному злу; он боялся его, но не любил. Древний человек уважал знание; он чтил мудрость и воздавал ей должное. Шаман был, в принципе, обманщиком, но поклонение шаманизму хорошо подтверждает, сколь ценной считалась мудрость в процессе эволюции человечества.
\usection{3. Шаманистическая теория болезни и смерти}
\vs p090 3:1 Поскольку древний человек считал, что причуды призраков и капризы духов непосредственно воздействуют на него самого и его материальное окружение, не удивительно, что его религию так сильно интересовали материальные явления. Современный человек борется непосредственно с материальными проблемами; он сознает, что разумные действия оказывают влияние на материю. Точно так же и первобытный человек хотел изменить жизнь и даже управлять ею и энергиями физического мира; а поскольку ограниченное понимание космоса привело его к вере в то, что призраки, духи и боги лично и непосредственно во всех деталях управляют жизнью и материей, вполне логично, что он направлял свои усилия на то, чтобы обрести расположение и поддержку этих сверхчеловеческих сущностей.
\vs p090 3:2 В свете сказанного многое из необъяснимого и иррационального в древних культах становится понятным. Культовые обряды были попыткой древнего человека управлять материальным миром, в котором он находился. И многие из его усилий были направлены на то, чтобы продлить жизнь и обеспечить себе здоровье. Поскольку все болезни и сама смерть первоначально считались явлениями, связанными с духами, шаманы неизбежно должны были быть не только знахарями и жрецами, но и врачами и хирургами.
\vs p090 3:3 Примитивный разум человека страдал от нехватки практических знаний, но оставался логичным. Когда вдумчивые люди видели болезнь и смерть, они пытались определить причину этих кар, и, в соответствии со своим пониманием, шаманы и мудрецы выдвигали следующие причины таких бед:
\vs p090 3:4 \ublistelem{1.}\bibnobreakspace \bibemph{Призраки --- непосредственное воздействие духов.} Согласно самой ранней гипотезе, выдвинутой для объяснения болезней и смерти, духи вызывали болезнь, выманивая душу из тела; если ей не удавалось вернуться, то наступала смерть. Древние так боялись злых действий духов, вызывающих болезни, что заболевших часто оставляли одних даже без пищи и воды. Несмотря на ошибочную основу таких верований, они были эффективны, ибо больных изолировали и тем самым предотвращали распространение инфекционных заболеваний.
\vs p090 3:5 \P\ \ublistelem{2.}\bibnobreakspace \bibemph{Насилие --- очевидные причины.} Причины некоторых травм и смертей было настолько легко установить, что их с древних времен не соотносили с действиями духов. Смерть и ранения на войне, при столкновении с животными и в других легко устанавливаемых обстоятельствах, рассматривались как естественные явления. Но долгое время верили, что духи, тем не менее, несут ответственность за медленное исцеление и за инфицирование ран, даже имеющих «естественные» причины. Если не удавалось обнаружить никакой видимой естественной причины, то ответственность за болезнь и смерть возлагалась на духов\hyp{}призраков.
\vs p090 3:6 В наши дни в Африке и других местах можно найти примитивные народы, которые убивают кого\hyp{}нибудь каждый раз, когда происходит ненасильственная смерть. Их знахари указывают виновных. Если мать умрет при родах, ребенка тут же задушат --- жизнь за жизнь.
\vs p090 3:7 \P\ \ublistelem{3.}\bibnobreakspace \bibemph{Магия --- воздействие врагов.} Считалось, что многие болезни вызваны колдовством, воздействием дурного глаза и магии. Некогда действительно было опасно указывать на кого\hyp{}нибудь пальцем; и до сих пор это считается дурной манерой. В случае непонятной болезни и смерти древние проводили официальное следствие, вскрывали тело и принимали что\hyp{}нибудь найденное за причину болезни; иначе последовало бы объяснение, что смерть наступила по причине колдовства, что вызвало бы необходимость казнить ответственного за это колдуна. Эти древние вскрытия трупов спасли жизнь многим предполагаемым колдунам. Нередко приходили к выводу, что соплеменник мог умереть в результате своего же собственного колдовства, и тогда никого не обвиняли.
\vs p090 3:8 \P\ \ublistelem{4.}\bibnobreakspace \bibemph{Грех --- наказание за нарушение табу.} Еще в относительно недавние времена верили, что болезнь --- это наказание за грех, личный или расовый. У людей на этой стадии эволюции, существует теория, что болезнь не может поразить человека, если он не нарушал табу. Для таких верований типично было считать болезнь и страдание «стрелами Всемогущего». Китайцы и месопотамцы долгое время полагали, что болезнь --- результат действий злых демонов, а халдеи верили, что причиной страданий могут быть и звезды. Эта теория болезни как следствия божественного гнева до сих пор бытует среди многих, по общему мнению, цивилизованных групп урантийцев.
\vs p090 3:9 \P\ \ublistelem{5.}\bibnobreakspace \bibemph{Естественные причины.} Человечество очень медленно познавало материальные секреты взаимосвязи причины и следствия в физических сферах энергии, материи и жизни. Древние греки, сохранившие традиции учений Адама\hyp{}сына, одними из первых осознали, что всякая болезнь является результатом естественных причин. Наступление эры науки медленно, но верно рушит вековечные воззрения человека на болезнь и смерть. Лихорадка была одним из заболеваний, в числе первых исключенным из категории сверхъестественных явлений, и развитие науки неуклонно разбивало цепи невежества, так долго сковывающие человеческий разум. Понимание процессов старения и воздействия инфекции позволяет человеку постепенно изжить страх перед призраками, духами и богами как непосредственными виновниками человеческих невзгод и страданий.
\vs p090 3:10 \P\ Эволюция безошибочно достигает своей цели: она вселяет в человека суеверный страх перед неизвестным и ужас перед невидимым, являющиеся теми строительными лесами, на которые опираются представления о Боге. А имея доказательства того, что, благодаря скоординированным действиям откровения, возникло развитое понимание Божества, эволюция затем точно так же безошибочно приводит в движение те интеллектуальные силы, которые неумолимо уничтожат эти леса, уже отслужившие свое.
\usection{4. Медицина при шаманах}
\vs p090 4:1 Вся жизнь древних людей представляла собой целую систему профилактических мер; их религия, в немалой степени, содержала методики предотвращения болезней. И несмотря на ошибочность теорий, последние с энтузиазмом применялись на практике; древние безгранично верили в свои методы лечения, а это уже само по себе --- сильное лекарство.
\vs p090 4:2 \P\ Вера, необходимая для выздоровления в процессе невежественного лечения одного из этих древних шаманов, в общем\hyp{}то, не существенно отличалась от веры, которая необходима, чтобы обрести исцеление у какого\hyp{}нибудь его более позднего преемника, занимающегося ненаучным лечением болезней.
\vs p090 4:3 \P\ Наиболее примитивные племена очень боялись больных, и на протяжении многих веков их старательно избегали, постыдно игнорировали. Когда эволюция шаманизма привела к возникновению касты жрецов и знахарей, согласных лечить болезни, это стало огромным гуманистическим прогрессом. Затем вошло в обычай, чтобы весь клан собирался возле больного, помогая шаману криками изгонять духов болезни. Нередко шаманом, проводящим диагностику, была женщина, тогда как само лечение осуществлял мужчина. Как правило, метод диагностики болезни сводился к обследованию внутренностей какого\hyp{}нибудь животного.
\vs p090 4:4 Болезнь исцеляли пением, воплями, наложением рук, воздействуя на пациента дыханием и многими другими способами. В более поздние времена широко практиковали сон в храме, во время которого якобы происходило исцеление. Впоследствии во время сна в храме знахари стали пытаться проводить настоящие хирургические операции; к числу первых операций относилась трепанация черепа с целью дать выход духу, вызывающему головную боль. Шаманы научились лечить переломы и вывихи, вскрывать фурункулы и гнойники; женщины\hyp{}шаманы стали весьма сведущи в акушерстве.
\vs p090 4:5 Был распространен метод лечения, заключавшийся в том, чтобы потереть зараженное или больное место на теле чем\hyp{}нибудь магическим, выбросить этот амулет и якобы обрести исцеление. Верили, что если кто\hyp{}нибудь случайно поднимет выброшенный амулет, то немедленно приобретет ту же самую инфекцию или болезнь. Прошло немало времени прежде, чем стали применяться травы и другие настоящие лекарства. Из магического обряда натирания тела с целью изгнания духов развился массаж, предшественником которого были усилия, направленные на втирание лекарств, подобно тому, как современные люди втирают в кожу мази. Считалось, что для избавления от духа, вызывающего болезнь, полезно ставить банки, сосать больные места и делать кровопускания.
\vs p090 4:6 Поскольку вода была сильным фетишем, ее использовали для лечения многих болезней. Издавна считалось, что духа, вызывающего болезнь, можно изгонять с помощью потения. Высоко ценились паровые бани; природные горячие источники вскоре превратились в примитивные курорты. Древний человек обнаружил, что тепло успокаивает боль; он использовал солнечное тепло, свежие органы животных, горячую глину и горячие камни, и многие подобные методы лечения применяются до сих пор. Для воздействия на духов пользовались ударными инструментами; тамтамы существовали повсеместно.
\vs p090 4:7 Некоторые люди считали, что причиной болезни является злой сговор между духами и животными. Это привело к вере в то, что от каждой болезни, вызванной животными, есть действенное растительное лекарство. Самыми большими приверженцами теории универсальной целебности растений были красные люди; они всегда вливали немного крови в ямку, остающуюся после выдергивания корня растения.
\vs p090 4:8 В качестве лечебных мер часто использовали пост, диету и ревульсивные средства. Человеческие выделения, несомненно, обладали магической силой и поэтому высоко ценились; таким образом, в числе первых лекарств были кровь и моча, а вскоре к ним добавились корни и различные соли. Шаманы верили, что духов болезни можно изгнать из тела дурно пахнущими и неприятными на вкус лекарствами. С ранних времен стандартным лечением стало очищение кишечника, а целебные свойства сырого какао и хинина были одним из самых ранних фармацевтических открытий.
\vs p090 4:9 Греки первыми выработали действительно рациональные методы лечения больных. И греки, и египтяне получили свои медицинские знания из долины Евфрата. Масло и вино были очень древним средством для лечения ран; шумеры использовали касторовое масло и опиум. Многие из этих древних и эффективных тайных лекарств теряли свою силу, когда становились известными; секретность всегда была непременным условием успешности обмана и суеверий. Только факты и истина добиваются полного понимания и радуются освещению и просвещению, которое несут научные исследования.
\usection{5. Жрецы и ритуалы}
\vs p090 5:1 Главное в ритуале --- это безупречность его проведения; у дикарей он должен был совершаться с величайшей тщательностью. Обряд обладает силой воздействия на духов только тогда, когда ритуал исполнен правильно. Если в ритуале есть изъян, то он лишь вызывает гнев и негодование богов. Поэтому, когда постепенно развивающийся разум человека понял, что \bibemph{техника ритуала} является решающим фактором его действенности, древние шаманы неизбежно должны были рано или поздно превратиться в жрецов, обученных руководить тщательным проведением ритуалов. Итак, на протяжении десятков тысяч лет бесконечные ритуалы обременяли общество и наносили вред цивилизации, ложась невыносимым бременем на каждый поступок в течение жизни, на каждое совместное человеческое предприятие.
\vs p090 5:2 Ритуал --- это техника освящения обычаев; ритуал создает и увековечивает мифы, а также способствует сохранению общественных и религиозных обычаев. С другой стороны, сам ритуал порождается мифами. Часто вначале ритуал бывает общественным, затем становится экономическим и, наконец, обретает священность и статус религиозного обряда. По способу проведения ритуалы могут быть индивидуальными или групповыми --- или же и теми, и другими --- как это видно на примере молитвы, танца и драмы.
\vs p090 5:3 Слова становятся частью ритуала, как, например, слова аминь и села. Привычка проклинать, богохульствовать представляет собой недостойное использование прежних ритуальных повторений священных имен. Паломничества к святыням --- очень древний ритуал. Впоследствии из этого ритуала выросли сложные обряды очищения и освящения. Обряды посвящения в тайные общества в действительности, были у древних племен грубым религиозным ритуалом. В старых мистериальных культах техника поклонения заключалась в долгом исполнении лишь одного разросшегося религиозного ритуала. В конце концов, ритуал развился в современные типы общественных церемоний и религиозного поклонения --- такие, как молитва, песнопения, проповеди и другие индивидуальные и групповые религиозные обряды.
\vs p090 5:4 \P\ Жрецы произошли от шаманов, которые, становясь в ходе эволюции оракулами, прорицателями, певцами, танцорами, создателями погоды, хранителями религиозных реликвий, настоятелями храмов и предсказателями событий, обрели, наконец, статус подлинных вершителей религиозного почитания. Со временем это звание стало наследственным; возникла постоянная каста жрецов.
\vs p090 5:5 С развитием религии у жрецов стала складываться специализация в соответствии с их врожденными талантами или особыми склонностями. Одни занялись песнопениями, другие --- молитвами, третьи --- жертвоприношениями; позже появились ораторы --- проповедники. А когда религия сделалась узаконенной, эти жрецы провозгласили, что «держат ключи от неба».
\vs p090 5:6 Жрецы всегда стремились произвести впечатление на простых людей и внушить им благоговение, проводя религиозные обряды на древнем языке или используя разные магические пассы, чтобы озадачивать верующих и увеличивать почтение к себе и свою власть. Огромная опасность всего этого заключается в том, что ритуал имеет тенденцию подменять собой религию.
\vs p090 5:7 Жречество многое сделало для того, чтобы задержать развитие науки и воспрепятствовать духовному прогрессу, но оно внесло вклад в стабилизацию цивилизации и в развитие некоторых областей культуры. Но многие современные священники перестали быть вершителями ритуала почитания Бога, обратив свои взоры на теологию --- попытку дать определение Бога.
\vs p090 5:8 Нельзя отрицать, что жрецы были камнем на шее у народов, но истинные религиозные лидеры играли неоценимую роль, указывая путь к высшим и лучшим реальностям.
\vs p090 5:9 [Представлено Мелхиседеком Небадона.]
