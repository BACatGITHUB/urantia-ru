\upaper{130}{На пути в Рим}
\author{Комиссия срединников}
\vs p130 0:1 Большую часть двадцать восьмого и весь двадцать девятый год жизни на земле Иисус путешествовал по римскому миру. Иисус вместе с двумя индусами --- Гонодом и его сыном Ганидом --- покинул Иерусалим в воскресенье утром, 26 апреля 22 г. н.э. Они совершили свое путешествие в соответствии с планом, и Иисус попрощался с отцом и сыном в городе Чаракс на берегу Персидского залива десятого декабря следующего 23 года н.э.
\vs p130 0:2 \pc Из Иерусалима они через Иоппию направились в Кесарию. В Кесарии они наняли лодку до Александрии. Из Александрии они поплыли на Крит в Ласею. С Крита, по пути к Карфагену, они зашли в Киренею. Из Карфагена на нанятой лодке, с остановками на Мальте, в Сиракузах и Мессине доплыли до Неаполя. Оттуда они направились в Капую, а дальше по Аппиевой дороге совершили путешествие в Рим.
\vs p130 0:3 После пребывания в Риме они сушей дошли до Тарентума, откуда отправились морем к Афинам в Грецию, останавливаясь в Никополе и в Коринфе. Из Афин они через Троаду пошли в Ефес. Оттуда они уплыли на Кипр, зайдя по дороге в порт на Родосе. На Кипре они пробыли довольно долго, осматривая остров и отдыхая, и затем уплыли в Антиохию в Сирии. Из Антиохии они направились на юг в Сидон и затем в Дамаск. Оттуда они с караваном пришли в Месопотамию, миновав по пути Типсах и Лариссу. Какое\hyp{}то время они прожили в Вавилоне, посетили Ур и другие места, а потом направились в Сузу. Из Сузы пошли в Чаракс, где Гонод и Ганид сели на корабль и отплыли в Индию.
\vs p130 0:4 \pc Азы того языка, на котором разговаривали Гонод и Ганид, Иисус усвоил еще тогда, когда четыре месяца работал в Дамаске. Находясь там, он много занимался переводами с греческого на один из языков Индии, в чем ему помогал земляк Гонода.
\vs p130 0:5 \pc Иисус в этом средиземноморском путешествии ежедневно около половины дня проводил, обучая Ганида или работая переводчиком на деловых совещаниях и общественных дружеских встречах Гонода. Остаток дня был в его распоряжении и посвящался установлению тех тесных личных отношений со своими собратьями, тех близких связей со смертными этого мира, которые так характерны для его деятельности в годы, предшествовавшие его публичному служению.
\vs p130 0:6 Путем непосредственного наблюдения и тесного общения Иисус знакомился с более высокой материальной и интеллектуальной цивилизацией Запада и Леванта; от Гонода и его незаурядного сына он узнал очень многое о цивилизации и культуре Индии и Китая, так как Гонод, хотя и был гражданином Индии, совершил три длинных путешествия по империи желтой расы.
\vs p130 0:7 Во время этого длительного и близкого общения юный Ганид многому научился у Иисуса. Они сильно привязались друг к другу, и отец молодого человека много раз пытался убедить Иисуса поехать вместе с ними в Индию, но Иисус всегда отказывался, ссылаясь на то, что ему необходимо вернуться к своей семье в Палестину.
\usection{1. В Иоппии --- беседа об Ионе}
\vs p130 1:1 Будучи в Иоппии Иисус, встретил Гадиаха, переводчика\hyp{}филистимлянина, который работал у некоего Симона, кожевника. Представители Гонода в Месопотамии совершили множество сделок с этим Симоном; поэтому Гонод и его сын хотели встретиться с ним по пути в Кесарию. Пока они жили в Иоппии, Иисус и Гадиах стали близкими друзьями. Юный филистимлянин искал истину. Иисус же даровал истину. Он \bibemph{был} истиной для этого поколения на Урантии. Когда встречаются тот, кто алчет найти истину, и тот, кто жаждет даровать истину, результатом является великое и освобождающее просвещение, рожденное опытом новой истины.
\vs p130 1:2 Однажды после ужина Иисус и юный филистимлянин прогуливались по берегу моря, и Гадиах, не зная, что этот «книжник из Дамаска» чрезвычайно сведущ в иудейских преданиях, указал Иисусу на пристань, с которой, как считалось, Иона отправился в свое злополучное плавание в Фарсис. После этого он задал Иисусу такой вопрос: «Ты и правда считаешь, что большая рыба действительно проглотила Иону?» Иисус понял, что это предание оказало огромное влияние на жизнь молодого человека и что размышление над ним внушило ему мысль, что безрассудно пытаться убежать от долга; поэтому Иисус не сказал ничего, что могло бы ненароком разрушить те основы, на которых строится вся сегодняшняя практическая жизнь Гадиаха. Отвечая на его вопрос, Иисус сказал: «Друг мой, все мы --- Ионы, и наши жизни должны быть прожиты в соответствии с волей Бога, и каждый раз, когда мы в сегодняшней жизни забываем об исполнении нашего долга, увлекшись далекими соблазнами, мы тем самым немедленно попадаем под воздействие тех влияний, которые неподвластны добру и не управляются силами праведности. Нарушая долг, мы жертвуем истиной. Бегство от служения свету и жизни в конце концов может привести только к отчаянной борьбе с ужасными китами эгоизма, ведущими в конечном счете к тьме и смерти, если только эти отвернувшиеся от Бога Ионы не обратят свои сердца, пусть даже из самых глубин отчаяния, к поискам Бога и его благости. И когда такие отчаявшиеся души искренне ищут Бога --- алчут истины и жаждут справедливости --- нет ничего, что могло бы удержать их далее в плену. Когда они ищут света всем сердцем, неважно, как низко они могли пасть перед этим, --- дух Господа Бога Небесного спасет их от плена; жизненные бедствия извергнут их на твердую землю новых возможностей обновленного служения и более мудрого жития».
\vs p130 1:3 Гадиах был необыкновенно тронут учением Христа, и они долго разговаривали ночью на берегу моря и, прежде чем разойтись по домам, вместе молились друг за друга. Это был тот самый Гадиах, который впоследствии слушал проповеди Петра, глубоко уверовал в Иисуса из Назарета и однажды вечером вступил в достопамятный спор с Петром в доме Доркаса. И Гадиах непосредственно повлиял на окончательное решение Симона, богатого купца, торговавшего кожами, принять христианство.
\vs p130 1:4 \pc (В этом рассказе о личном общении Иисуса со смертными собратьями во время его путешествия по Средиземноморью мы, в соответствии с имеющимся у нас разрешением, свободно переводим его слова на современный язык, общепринятый на Урантии во время его написания.)
\vs p130 1:5 \pc На последней встрече Иисуса с Гадиахом возникла дискуссия о добре и зле. Юный филистимлянин был очень обеспокоен тем, что ему казалось несправедливым присутствие в мире зла наряду с добром. Он спросил: «Как может Бог, если он бесконечно добр, позволять нам испытывать горечь зла? И кто, в конце концов, творит зло?» В те дни многие еще верили, что бог творит и добро, и зло, но Иисус никогда не учил этому заблуждению. В ответ на этот вопрос Иисус сказал: «Брат мой, Бог есть любовь; следовательно, он должен быть благ, и его благость так велика и истинна, что в ней не может быть ничтожности и нереальности зла. Бог настолько совершенно благ, что в нем нет места для отрицающего зла. Зло есть незрелое решение и пустое заблуждение тех, кто сопротивляется добру, отвергает красоту и предает истину. Зло --- это только неверно приспособившаяся незрелость или разрушающее и извращенное влияние невежества. Зло есть неизбежная тьма, следующая по пятам за неразумным отвержением света. Зло --- это тьма и ложь. Это то, что, будучи принято сознательно и добровольно одобрено, становится грехом.
\vs p130 1:6 Твой Отец Небесный, одаривший тебя властью выбирать между истиной и заблуждением, создал возможность отрицания положительного пути света и жизни; но этих заблуждений зла на самом деле не существует до тех пор, пока разумное создание, неверно выбрав жизненный путь, не призовет их к жизни. А потом это зло превращается в грех с помощью знания и преднамеренного выбора такого своевольного и мятежного создания. Вот почему наш Отец Небесный позволяет добру и злу идти рядом до конца жизни, так же, как природа разрешает пшенице и плевелам расти рядом до жатвы». После того как в ходе последующей беседы Гадиаху стало совершенно ясно истинное значение этих знаменательных высказываний, он был полностью удовлетворен ответом Иисуса на свой вопрос.
\usection{2. В Кесарии}
\vs p130 2:1 Иисус и его друзья задержались в Кесарии позже предполагавшегося срока, так как обнаружилось, что одно из огромных рулевых весел судна, на котором они собирались отплыть, треснуло. Капитан решил остаться в порту, пока не будет сделано новое весло. Для его изготовления не хватало искусных мастеров по дереву, так что Иисус вызвался помочь. По вечерам Иисус и его друзья прогуливались по превосходной дамбе, окружавшей порт. Ганиду очень нравилось слушать объяснения Иисуса о том, как устроена система водоснабжения города, и каким образом использовались морские приливы и отливы для мытья улиц и очистки сточных труб. На юного индуса сильное впечатление произвел храм Августа, расположенный на возвышении и увенчанный колоссальной статуей римского императора. На второй день их пребывания они втроем побывали на представлении в огромном амфитеатре, вмещавшем двадцать тысяч человек, а вечером отправились в театр на греческую пьесу. Подобных зрелищ Ганид еще никогда не видел, и он задавал Иисусу много вопросов. Утром третьего дня они посетили с официальным визитом губернаторский дворец, так как Кесария была столицей Палестины и резиденцией римского прокуратора.
\vs p130 2:2 \pc В их гостинице остановился также купец из Монголии, и так как он, человек с Дальнего Востока, замечательно говорил по\hyp{}гречески, Иисус несколько раз подолгу беседовал с ним. На этого человека сильное впечатление произвела жизненная философия Иисуса, и он никогда не забывал его мудрых слов о том, чтобы «жить на земле небесной жизнью, каждодневно подчиняясь воле Небесного Отца». Купец был таоистом, поэтому он абсолютно уверовал в учение о всемирном Божестве. Вернувшись в Монголию, он начал наставлять этим новым истинам своих соседей и деловых партнеров, и именно вследствие этой деятельности его сын решил стать таоистским священником. Посвятив проповеди этой новой истины всю свою жизнь, молодой человек приобрел огромное влияние, и его сын и внук пошли по его стопам и так же предано хранили верность учению о Едином Боге --- Высшем Правителе Небес.
\vs p130 2:3 Хотя восточная ветвь ранней христианской церкви, с центром в Филадельфии, более строго придерживалась учения Иисуса, чем иерусалимские братья, к сожалению, не нашлось человека, подобного Петру, который смог бы отправиться в Китай, и никого, похожего на Павла, чтобы пойти в Индию, где была тогда столь благоприятная духовная почва, чтобы бросить в нее семена нового евангелия царства. Само учение Иисуса в том виде, как его придерживались филадельфийцы, могло бы совершить такой же быстрый и действенный переворот в умах духовно изголодавшихся азиатских народов, какой произвели проповеди Петра и Павла на Западе.
\vs p130 2:4 \pc Однажды один из молодых людей, мастеривший вместе с Иисусом рулевое весло, очень заинтересовался словами, которые тот ронял время от времени, пока они усердно трудились на верфи. Когда Иисус вскользь заметил, что Отец Небесный заботится о благоденствии своих детей на земле, молодой грек, Анаксанд, сказал: «Если боги так обеспокоены моей судьбой, почему они не устранят жестокого и несправедливого мастера этой мастерской?» Он был поражен, когда Иисус ответил ему: «Так как тебе ведомы пути добра и ты ценишь справедливость, может быть, боги приблизили к тебе этого заблудшего человека, чтобы ты мог направить его на лучший путь. Возможно, ты есть та соль, которая должна сделать этого брата более приятным для остальных людей; если ты, конечно, не потерял своего вкуса. Сейчас этот человек --- твой хозяин, и творимое им зло неблагоприятно влияет на тебя. Почему бы тебе не утвердить своего господства над злом силой добра и таким образом не стать хозяином взаимоотношений между вами двоими? Я предсказываю, что хорошее в тебе может пересилить плохое в нем, если только ты дашь хорошему действительно благоприятную возможность для этого. В жизни смертного нет события более замечательного, чем наслаждаться радостью от того, что материальная жизнь соединилась с духовной энергией и божественной правдой в одной из триумфальных битв с заблуждением и злом. Это чудесный и преобразующий опыт --- стать живым каналом духовного света для смертного, находящегося в духовной тьме. Если ты больше осиян правдой, чем этот человек, его нужда должна отозваться в тебе. Ты, конечно, не трус, который может стоять на берегу моря и наблюдать, как твой собрат, не умеющий плавать, погибает! Насколько же душа этого человека, барахтающаяся во тьме, ценнее его тела, тонущего в воде!»
\vs p130 2:5 Слова Иисуса чрезвычайно тронули Анаксанда. Он сразу же передал своему начальнику свой разговор с Иисусом, и той же ночью они оба попросили Иисуса посоветовать им, как им достичь душевного благоденствия. И позже, после основания в Кесарии христианской миссии, оба этих человека, один --- грек, а другой --- римлянин, поверили проповеди Филиппа и стали видными деятелями основанной им церкви. Позже этот молодой грек был назначен управляющим у римского центуриона Корнилия, который уверовал благодаря проповеди Петра. Анаксанд продолжал нести свет тем, кто пребывал во тьме, вплоть до дней тюремного заключения Павла в Кесарии, когда он случайно погиб, помогая страдающим и умирающим во время великого избиения двадцати тысяч евреев.
\vs p130 2:6 \pc К этому времени Ганид начал понимать, что его домашний учитель посвящает свой досуг необычайному личному служению своим собратьям, и это побудило молодого индуса попытаться понять причины этого неустанного труда. Он спросил: «Почему ты постоянно посещаешь чужих людей и беседуешь с ними?» И Иисус ответил ему: «Ганид, для человека, знающего Бога, нет чужих людей. Пройдя через опыт обретения Отца Небесного, ты обнаруживаешь, что все люди --- твои братья, и может ли показаться странным, что человек испытывает радость при встрече с только что найденным братом? Познакомиться со своими братьями и сестрами, узнать их заботы и научиться любить их --- это верховный жизненный опыт».
\vs p130 2:7 Однажды во время беседы, которая затянулась далеко за полночь, молодой человек попросил Иисуса объяснить ему разницу между волей Бога и тем актом выбора, совершаемым человеческим разумом, который тоже называют волей. Смысл ответа Иисуса заключался в следующем: Воля Бога есть путь Бога, участие в выборе Бога вопреки любым другим возможностям. Следовательно, исполнение воли Бога --- это развивающийся опыт обретения все большего сходства с Богом, а Бог есть источник и цель всего, что есть добро, красота и истина. Воля человека есть путь человека, итог и сущность того, чем смертный выбирает быть и что он решает делать. Воля есть осознанный выбор осознающего себя существа, ведущий к решению\hyp{}поведению, основанному на разумном размышлении.
\vs p130 2:8 В тот день Иисус и Ганид забавлялись игрой с очень умной собакой пастуха, и Ганид захотел узнать, есть ли у собаки душа, есть ли у нее воля, в ответ на его вопросы Иисус сказал: «У собаки есть разум, который может знать материального человека, ее хозяина, но не может знать Бога, который есть дух; следовательно, собака не имеет духовной природы и не может обладать духовным опытом. Собака может иметь волю, дарованную ей природой и развитую дрессировкой, но подобная сила ума не есть духовная сила, и она не сравнима с человеческой волей, так как не \bibemph{является мыслящей ---} она не есть результат различения высших нравственных значений или выбора духовных и вечных ценностей. Именно обладание этой способностью духовного различения и выбора истины и делает смертного человека нравственным существом, одаренным характерным свойством духовной ответственности и возможностью вечной жизни». Далее Иисус объяснил, что именно отсутствие таких умственных способностей у животного и делает для животного мира совершенно невозможным развитие во времени языка или переживание чего\hyp{}либо, равного продолжению существования личности в вечности. Благодаря знаниям, полученным в этот день, Ганид никогда больше не разделял веры в переселение душ людей в тела животных.
\vs p130 2:9 \pc На следующий день Ганид обсудил происшедшее со своим отцом, и именно в ответ на вопрос Гонода Иисус объяснил, что «человеческие воли, занятые принятием только временных решений, связанных с материальными проблемами животного существования, обречены на исчезновение во времени. Те же, кто принимает искренние нравственные решения и делает непреложный духовный выбор, тем самым последовательно отождествляются с присутствующим в них божественным духом, и поэтому они все больше преобразуются в ценности вечной жизни --- в бесконечную череду божественного служения».
\vs p130 2:10 \pc В этот же день мы впервые услышали великую правду, которая, если ее выразить на современном языке, будет звучать так: «Воля --- это такое проявление человеческого разума, которое дает возможность субъективному сознанию выразить себя объективно и пережить в личном опыте феномен стремления стать Богоподобным». И в этом смысле всякое размышляющее и движимое духом существо может стать \bibemph{творческим.}
\usection{3. В Александрии}
\vs p130 3:1 Посещение Кесарии было насыщено событиями, когда же лодка была готова, Иисус и двое его друзей отплыли в полдень в Египет, в Александрию.
\vs p130 3:2 Плавание до Александрии было весьма приятым для всех троих. Ганид был в восторге от путешествия и постоянно забрасывал Иисуса вопросами. Когда они приблизились к порту, молодой человек был потрясен огромным маяком Фароса, расположенным на острове, который Александр присоединил молом к основному материку, создав таким образом два величественных порта и тем самым сделав Александрию морским торговым перекрестком Африки, Азии и Европы. Этот огромный маяк был одним из семи чудес света и предшественником всех последующих маяков. Они поднялись рано утром, чтобы осмотреть это великолепное сооружение, предназначенное спасать человеческие жизни, и среди возгласов восторга Ганида Иисус сказал: «И ты, сын мой, будешь подобен этому маяку, когда вернешься в Индию, даже после кончины твоего отца; ты уподобишься свету жизни для тех, кто пребывает вокруг тебя во мраке, и будешь указывать тем, кто этого пожелает, путь благополучного достижения гавани спасения». И Ганид, сжав руку Иисуса, сказал: «Я буду».
\vs p130 3:3 \pc И вновь мы отмечаем, что первые учителя христианской религии совершали большую ошибку, когда обращали свое внимание исключительно на западную цивилизацию римского мира. Учение Христа в том виде, как его придерживались верующие в Месопотамии в первом столетии, было бы с готовностью воспринято различными группами религиозных людей в Азии.
\vs p130 3:4 \pc Через четыре часа после прибытия они поселились недалеко от восточного конца длинной и широкой улицы, имевшей сто футов в ширину и пять миль в длину, которая тянулась до западной границы этого города с населением в один миллион жителей. После первого осмотра главных достопримечательностей --- университета (музея), библиотеки, мавзолея императора Александра, дворца, храма Нептуна, театра и гимнастического зала --- Гонод занялся делами, а Иисус и Ганид пошли в библиотеку, величайшую в мире. Здесь было собрано около миллиона рукописей со всего цивилизованного мира --- Греции, Рима, Палестины, Парфянского царства, Индии, Китая и даже Японии. В этой библиотеке Ганид увидел величайшее в мире собрание индийской литературы; и они, пока находились в Александрии, ежедневно проводили здесь какое\hyp{}то время. Иисус рассказал Ганиду о том, что в этой библиотеке был сделан перевод иудейского писания на греческий. Они вновь и вновь обсуждали все религии мира, и Иисус старался показать молодому уму правду каждой из них, всегда добавляя при этом: «Но Ягве есть Бог, представление о котором развивалось от откровений Мелхиседека и завета Авраама. Евреи были потомками Авраама и впоследствии заняли ту самую землю, где жил и учил Мелхиседек и откуда он посылал учителей по всему миру. И в конце концов в их религии признание Господа Бога Израиля как Отца Всего Сущего отражено более ясно, чем в любой другой мировой религии».
\vs p130 3:5 \pc Под руководством Иисуса Ганид составил собрание учений всех тех религий мира, которые признавали Всемирное Божество, включая даже и те, которые в той или иной степени признавали существование подчиненных ему божеств. После длительного обсуждения Иисус и Ганид решили, что в религии римлян не было настоящего Бога, что, в сущности, их религия была не больше, чем поклонение императору. Они пришли к выводу, что у греков была скорее философия, чем религия, в которой присутствовала личность Бога. Они решили не рассматривать культы мистерий из\hyp{}за путаницы, происходившей вследствие их многочисленности, и из\hyp{}за их разнообразных представлений о Божестве, возникших по\hyp{}видимому, из других более древних религий.
\vs p130 3:6 Хотя сами переводы были сделаны в Александрии, Ганид окончательно привел их в порядок и добавил к ним свои собственные замечания только к концу пребывания в Риме. Он с большим удивлением обнаружил, что все лучшие авторы мировой духовной литературы более или менее ясно осознавали существование вечного Бога и во многом сходились во мнении о его природе и его отношениях со смертными людьми.
\vs p130 3:7 \pc Во время пребывания в Александрии Иисус и Ганид много времени проводили в музее. Этот музей был не столько собранием редких предметов, сколько университетом изящных искусств, науки и литературы. В те времена здесь находился интеллектуальный центр Западного мира, ученые профессора читали здесь свои лекции. Изо дня в день Иисус объяснял лекции для Ганида; однажды на второй неделе молодой человек воскликнул: «Учитель Иешуа, ты знаешь больше, чем эти профессора; ты должен встать и сказать им те великие вещи, которые ты сказал мне; их разум затуманен, потому что они слишком много думали. Я поговорю с отцом, и он устроит это». Иисус улыбнулся и сказал: «Ты восхищенный ученик, но эти учителя не расположены к тому, чтобы ты или я учили их. Тщеславие бездуховной учености --- опасно для человеческого опыта. Истинный учитель поддерживает свою интеллектуальную состоятельность тем, что всегда остается учеником».
\vs p130 3:8 Александрия была городом, в котором смешались разные культуры западного мира, вторым после Рима по величине и великолепию городом мира. Здесь находилась самая большая в мире еврейская синагога, местопребывание высшего органа Александрийского Синедриона, управляемого семидесятью старейшинами.
\vs p130 3:9 Среди многих людей, с которыми Гонод вел дела, был некий еврейский банкир, Александр, чей брат Филон был знаменитым религиозным философом того времени. Филон был занят достойной похвалы, но чрезвычайно трудной задачей --- он пытался увязать греческую философию и иудейскую теологию. Ганид и Иисус много говорили об учении Филона и предполагали посетить некоторые из его лекций, но во время их пребывания в Александрии этот знаменитый эллинский еврей был прикован болезнью к постели.
\vs p130 3:10 Иисус хвалил Ганиду многое в греческой философии и в доктринах стоиков, но он помог юноше осознать истину, что эти системы верований, так же как и туманные учения, распространенные среди какой\hyp{}то части его народа, религиозны лишь в том смысле, что они вели людей к обретению Бога и радости живого опыта познания Вечного.
\usection{4. Беседа о реальности}
\vs p130 4:1 Накануне отъезда из Александрии Ганид и Иисус долго беседовали с одним из ведущих профессоров университета, который читал лекции об учении Платона. Иисус переводил ученого греческого педагога, но не делал никаких собственных замечаний в опровержение греческой философии. Гонод в этот вечер отлучился по делам, так что, когда профессор ушел, учитель и ученик долго и задушевно беседовали о доктрине Платона. Хотя Иисус сдержанно одобрил отдельные положения учения этого грека, относящиеся к теории, согласно которой предметы материального мира суть призрачные отражения невидимых, но более существенных духовных реальностей, он стремился создать более прочную основу для мышления юноши; итак, он начал пространно рассуждать о природе реальности во вселенной. По существу, то, что Иисус сказал Ганиду, в современном изложении сводится к следующему:
\vs p130 4:2 \pc Источником вселенской реальности является Бесконечный. Материальные предметы конечного творения суть пространственно\hyp{}временное отражение Райского Паттерна (Образа) и Вселенского Разума вечного Бога. Причинность в физическом мире, самосознание в интеллектуальном мире и возрастающая индивидуальность в духовном мире --- эти реальности, представленные во вселенском масштабе, соединенные в вечной взаимосвязи и переживаемые в совершенном качестве и божественной ценности, --- составляют \bibemph{реальность Верховного.} Но в вечно меняющейся вселенной Изначальная Личность причинности, интеллекта и духовного опыта неизменна, абсолютна. Все предметы, даже в вечной вселенной неограниченных ценностей и божественных качеств, могут и часто действительно меняются, кроме Абсолютов и тех, что достигли абсолютного физического состояния, интеллектуального развития или духовной идентичности.
\vs p130 4:3 Высочайший уровень, которого может достичь конечное существо, --- это распознание Отца Всего Сущего и знание Верховного. Но даже после этого эти существа финального предназначения продолжают ощущать изменения в движениях физического мира и его материальных явлений. Подобным же образом они продолжают постигать продвижение собственного «Я» по мере своего постоянного восхождения в духовном мире и растущего осознания своей способности к углубленному пониманию разумного космоса и отклика на него. Только в совершенстве, гармонии и единстве воли создание может достичь общности с Творцом; и такое состояние божественности обретается и поддерживается лишь тем, что создание, продолжая жить и во времени, и в вечности, постоянно подчиняет свою конечную личную волю бесконечной воле Творца. Желание исполнить волю Отца всегда должно быть верховным в душе и должно превалировать над разумом идущего по пути восхождения сына Бога.
\vs p130 4:4 Одноглазый человек никогда не сможет надеяться отчетливо представить себе глубину перспективы. Точно так же не могут ни одноглазые ученые материалисты, ни одноглазые духовные мистики и аллегористы правильно представить себе и адекватно воспринять истинные глубины вселенской реальности. Все истинные ценности опыта создания скрыты в глубине познания.
\vs p130 4:5 Причинность, лишенная мысли, не может развить утонченное и сложное из грубого и простого, так же и опыт, лишенный духовности, не может развить божественные черты вечной жизни из материальных разумов смертных, живущих во времени. Свойством вселенной являющимся отличительным признаком бесконечного Божества, является нескончаемое творческое дарование личности, которая может продолжать свое существование в непрестанном стремлении к Божеству.
\vs p130 4:6 Личность --- это такое проявление космоса, такая фаза вселенской реальности, которая, постоянно изменяясь, в то же время может сохранять свою идентичность и в самый момент таких изменений и навсегда после них.
\vs p130 4:7 Жизнь есть адаптация первоначальной космической причинности к требованиям и возможностям вселенских ситуаций, и она возникает в результате действия Вселенского Разума и активации духовной искры Бога, который есть дух. Значение жизни --- в ее способности приспосабливаться; ценность жизни --- в ее способности совершенствоваться, достигая даже высот Божественного сознания.
\vs p130 4:8 Неспособность самосознающей жизни к адаптации во вселенной приводит к космической дисгармонии. Если личная воля окончательно отклоняется от общей тенденции развития вселенных, это приводит к интеллектуальной изоляции, обособлению личности. Потеря постоянно пребывающего в ней духовного наставника приводит к прекращению духовного существования. Разумная и развивающаяся жизнь становится тогда сама по себе и сама для себя неоспоримым доказательством исполненного смысла вселенского проявления воли божественного Творца. И эта жизнь в совокупности стремится к высшим ценностям, имея своей конечной целью Отца Всего Сущего.
\vs p130 4:9 Без высшей и квази\hyp{}духовной помощи интеллекту разум человека только в определенной степени превосходит животный уровень. Поэтому животные (не обладающие верой и мудростью) не могут иметь опыта сверхсознания, осознания собственного сознания. Ум животного осознает только объективную вселенную.
\vs p130 4:10 Знание --- сфера материального или распознающего факты разума. Истина --- сфера одухотворенного интеллекта, который осознает свое знание Бога. Знание доказуемо; истина постигается опытом. Знание составляет достояние разума; истина является опытом души, развивающегося „Я“. Знание есть функция недуховного уровня; истина есть фаза разумно\hyp{}духовного уровня вселенной. Материальный разум воспринимает мир фактического знания; одухотворенный интеллект различает мир истинных ценностей. Эти две точки зрения, объединенные во времени и приведенные в гармонию, обнаруживают мир реальности, в котором мудрость переводит явления вселенной на язык развивающегося личного опыта.
\vs p130 4:11 Заблуждение (зло) --- следствие несовершенства. Качество несовершенства или факты неправильной адаптации обнаруживаются на материальном уровне серьезным изучением и научным анализом; на нравственном уровне они постигаются человеческим опытом. Существование зла является доказательством заблуждений разума и незрелости развивающегося „Я“. Во вселенской интерпретации зло является, таким образом, и мерой несовершенства. Возможность совершать ошибки свойственна обретению мудрости, пути продвижения от частного и временного к общему и вечному, от относительного и несовершенного к конечному и совершенному. Заблуждение есть признак относительного несовершенства, с ним неизбежно столкнется человек на пути восхождения к Райскому совершенству. Заблуждение (зло) не есть реальное свойство вселенной; оно есть просто выражение относительности взаимосвязи несовершенства неполного конечного с восходящими уровнями Верховного и Предельного.
\vs p130 4:12 \pc Хотя Иисус говорил все это наиболее понятным юноше языком, к концу обсуждения веки у Ганида отяжелели, и вскоре он задремал. На следующее утро они поднялись рано, чтобы погрузиться в лодку, готовую отплыть в Ласею на острове Крит. Но прежде чем они сели в лодку, юноша продолжал задавать вопросы о зле, на которые Иисус ответил:
\vs p130 4:13 \pc Зло --- понятие относительное. Оно возникает из наблюдения проявление несовершенства, которое появляется в тени, отбрасываемой конечным миром вещей и тварей, так как такой космос затемняет живительный свет вселенского проявления реальностей Бесконечного Существа.
\vs p130 4:14 Потенциальное зло присуще неизбежной неполноте откровения Бога как выражению бесконечности и вечности, ограниченных временем и пространством. Наличие частичного в законченном составляет относительность реальности, приводит к необходимости интеллектуального выбора и устанавливает ценностные уровни духовного познания и ответа на него. Неполное и конечное представление о Бесконечном, которым обладает ограниченный временем и пространством разум создания, есть само в себе и само по себе \bibemph{потенциальное зло.} Если зло, изначально присущее интеллектцуальной дисгармонии и духовной скудости усугублено тем, что намеренно ничего не делается для его духовного исправления, то это равносильно \bibemph{реальному злу.}
\vs p130 4:15 Все застывшие, мертвые понятия есть потенциальное зло. Конечная тень относительной и живой правды находится в постоянном движении. Статичные понятия неизменно задерживают развитие науки, политики, общества и религии. Статичные понятия могут представлять собой определенное знание, но в них недостаточно мудрости, и они лишены истины. Но не следует позволять понятию относительности ввести тебя в заблуждение настолько, чтобы ты не смог распознать упорядочения вселенной, совершающегося под руководством космического разума, и стабилизирующего управления ею с помощью энергии и духа Верховного.
\usection{5. На острове Крит}
\vs p130 5:1 Путешественники прибыли на Крит с единственной целью --- развлечься, погулять по острову, побродить по горам. Критяне в то время пользовались среди окружающих их народов не слишком завидной репутацией. Несмотря на это, Иисус и Ганид приобщили множество душ к более высоким уровням мышления и жизни и тем самым заложили основание для быстрого восприятия евангельских учений в дальнейшем, когда на остров прибыли первые проповедники из Иерусалима. Иисус любил критян, несмотря на резкие слова, сказанные о них Павлом, когда тот впоследствии послал на остров Тита, чтобы тот реорганизовал их церкви.
\vs p130 5:2 На гористой стороне Крита Иисус впервые долго беседовал с Гонодом о религии. На отца это произвело большое впечатление, и он сказал: «Ничего удивительного, что мальчик верит всему, что ты говоришь ему, но я никогда не знал, что в Иерусалиме, а тем более в Дамаске, была такая религия». Именно во время их пребывания на острове Гонод впервые предложил Иисусу отправиться с ними в Индию, и Ганид пришел в восторг при мысли, что Иисус, может быть, согласится на такое предложение.
\vs p130 5:3 Однажды, когда Ганид спросил Иисуса, почему он не посвятил себя работе публичного учителя, тот ответил: «Сын мой, все должно ожидать наступления своего часа. Ты родился в этом мире, но никакое страстное желание и никакое проявление нетерпения не помогут тебе вырасти. Во всех подобных делах ты должен ждать, когда придет время. Только время поможет созреть зеленому плоду на дереве. Одно время года сменяется другим и рассвет следует за закатом только с течением времени. Сейчас я с тобой и твоим отцом нахожусь на пути в Рим, и этого достаточно на сегодня. Мой завтрашний день всецело в руках моего Отца на небесах». И затем он поведал Ганиду историю о Моисее и о сорока годах неустанного ожидания и непрерывных приготовлений.
\vs p130 5:4 Во время посещения Светлой Гавани произошло одно событие, о котором Ганид никогда не забывал; вспоминая об этом эпизоде, он всегда желал обладать возможностью изменить что\hyp{}нибудь в кастовой системе своей родной Индии. На общественной дороге пьяный дегенерат напал на девушку\hyp{}рабыню. Когда Иисус увидел, в каком положении оказалась девушка, он бросился вперед и уберег ее от нападения сумасшедшего. В то время как испуганное дитя пряталось за него, он одной мощной правой рукой удерживал разъяренного человека на безопасном расстоянии до тех пор, пока бедняга не довел себя до изнеможения, осыпая воздух яростными ударами. Ганид очень хотел помочь Иисусу уладить дело, но его отец запретил вмешиваться. Хотя они не знали языка девушки, она оценила их милосердный поступок и знаками выразила свои чувства, когда они втроем провожали ее до дома. Возможно, это было практически единственное личное столкновение Иисуса с другим человеком за все время его жизни во плоти. Но тем вечером перед ним встала трудная задача объяснить Ганиду, почему он не ударил того пьяного. Ганид считал, что этого человека следовало ударить по меньшей мере столько же раз, сколько раз он ударил девушку.
\usection{6. Молодой человек, который боялся}
\vs p130 6:1 Однажды, когда они бродили по горам, у Иисуса произошел долгий разговор с одним молодым человеком, испуганным и подавленным. Этому юноше не удалось получить утешения и обрести твердость духа при общении со своими приятелями, и он искал в горах одиночества; он вырос с ощущением беспомощности и неполноценности. Эти природные качества усугубились многочисленными сложными обстоятельствами, с которыми молодому человеку пришлось столкнуться в детстве, и в особенности из\hyp{}за смерти отца, который умер, когда мальчику было двенадцать лет. Когда они встретились, Иисус сказал: «Приветствую тебя, друг мой! Почему ты так подавлен в такой прекрасный день? Если с тобой случилось что\hyp{}нибудь, что причиняет тебе страдание, может быть, я смогу каким\hyp{}нибудь образом помочь тебе. Во всяком случае, мне доставляет истинное удовольствие предложить тебе свою помощь».
\vs p130 6:2 Молодой человек не был расположен беседовать, так что Иисус попробовал во второй раз подступиться к его душе и сказал: «Я понимаю, ты поднялся сюда наверх, в горы, чтобы избавиться от людей; поэтому, конечно, ты не расположен говорить со мной, но я хотел бы знать, хорошо ли тебе знакомы эти места, знаешь ли ты, куда ведут эти тропинки? Может быть, ты можешь указать мне, как лучше добраться до Феникса?» Юноша хорошо знал эти горы и настолько увлекся, рассказывая Иисусу, как пройти в Феникс, что начертил все тропинки на земле и объяснил все подробности пути. Но он был очень удивлен и заинтересован, когда Иисус, попрощавшись и сделав вид, будто собирается уходить, неожиданно повернулся и сказал: «Я хорошо знаю, что ты хотел бы остаться наедине со своей печалью; но с моей стороны было бы неблагородно и неучтиво, после того, как ты столь великодушно объяснил мне, как лучше добраться до Феникса, равнодушно покинуть тебя, не сделав ни малейшей попытки ответить на твое стремление найти помощь и руководство в поисках лучшего пути к цели жизни, который ты ищешь в своем сердце, скитаясь здесь в горах. Так же, как ты хорошо знаешь все дороги в Феникс, так как ходил по ним много раз, так и я хорошо знаю дорогу к городу твоих обманутых надежд и неосуществившихся желаний. И так как ты попросил меня о помощи, я не обману твоих надежд». Юноша был крайне смущен, но все же ему удалось кое\hyp{}как пробормотать: «Но\ldots я не просил тебя ни о чем\ldots » Иисус, мягко положив ему руку на плечо, сказал: «Нет, сын мой, не словами, а этим горящим взглядом ты взывал к моему сердцу. Мальчик мой, для того, кто любит своих собратьев, красноречивый призыв о помощи виден по лицу, обескураженному и отчаявшемуся. Присядь же со мною рядом, пока я буду говорить тебе о тропинках служения и дорогах счастья, ведущих от печалей отдельного человеческого „Я“ к исполненной любви деятельности в братстве людей и в служении Богу небес».
\vs p130 6:3 К тому моменту молодой человек уже страстно желал поговорить с Иисусом и склонился к его ногам, умоляя помочь и указать ему дорогу, ведущую из его мира личной скорби и поражения. Иисус сказал: «Поднимись, друг мой! Встань как человек! Ты можешь быть окружен маленькими врагами, и многие препятствия могут задерживать тебя, но величие и реальность этого мира и вселенной --- на твоей стороне. Каждое утро солнце встает, чтобы поприветствовать тебя так же, как оно приветствует сильных и богатых людей на земле. Смотри --- у тебя сильное тело и крепкие мышцы --- ты физически очень хорошо развит. Конечно, это почти бесполезно, пока ты сидишь здесь на склоне горы и горюешь о своих несчастьях, реальных и выдуманных. Но твое тело может помочь тебе совершить великие дела, если ты поспешишь туда, где великие дела ждут своего осуществления. Ты стараешься убежать от своего несчастного „Я“, но это невозможно. Ты сам и проблемы твоей жизни реальны; ты не можешь убежать от них, пока ты жив. Но посмотри: опять же, твой ум чист и деятелен. У твоего сильного тела есть разум, чтобы управлять им. Задай своему разуму работу, чтобы он разрешил свои проблемы; учи свой интеллект работать на тебя; не соглашайся, чтобы страх и дальше господствовал над тобой, как над лишенным разума животным. Твой ум должен быть твоим стойким союзником в разрешении жизненных проблем, а не ты должен быть, как это было раньше, его малодушным и испуганным рабом и пленником уныния и поражения. Но самое ценное в тебе, твоя возможность достижения реальных результатов --- это пребывающий в тебе дух, который будет ободрять и вдохновлять твой ум, чтобы он управлял собой и активизировал тело, если ты освободишь его от пут страха и тем самым дашь возможность своей духовной природе начать освобождаться от зла бездеятельности с помощью исполненной силы живой веры. И вера эта сразу победит страх перед людьми неотразимой силой новой и всепобеждающей любви к \bibemph{твоим собратьям,} которая очень скоро переполнит твою душу благодаря родившемуся в твоем сердце сознанию, что ты дитя Бога.
\vs p130 6:4 В этот день, сын мой, тебе предстоит родиться заново, ты должен измениться и стать человеком веры, твердости и преданного служения людям во славу Бога. И когда ты изменишь свое отношение к жизни в себе самом, ты станешь по\hyp{}другому относиться ко вселенной; ты родился заново --- родился в духе --- и с этих пор вся твоя жизнь исполнится победных свершений. Несчастья будут подбадривать тебя; разочарования будут побуждать тебя; трудности будут бросать тебе вызов, а препятствия будут ободрять тебя. Поднимись, юноша! Простись с жизнью раба страха и спасающегося бегством труса. Поторопись вернуться к исполнению своего долга и веди свою жизнь во плоти, как сын Бога, как смертный, посвященный одухотворяющему служению человеку на земле и предназначенный для превосходного и вечного служения Богу в вечности».
\vs p130 6:5 И этот юноша, Фортунат, стал впоследствии предводителем христиан на Крите и ближайшим союзником Тита в его труде, посвященном духовному совершенствованию верующих Крита.
\vs p130 6:6 \pc Путешественники чувствовали себя по\hyp{}настоящему отдохнувшими и посвежевшими, когда в один прекрасный день приготовились отплыть в Карфаген, в северную Африку, с двухдневной остановкой в Киринее. Это именно здесь Иисус и Ганид оказали первую помощь юноше по имени Руф, который был ранен, когда поломалась нагруженная телега, запряженная волами. Они отнесли его домой к матери, и отцу юноши, Симону, которому и в голову никогда не приходило, что человек, чей крест он впоследствии нес по приказу римских солдат, был тем чужестранцем, который однажды помог его сыну.
\usection{7. В Карфагене --- беседа о времени и пространстве}
\vs p130 7:1 Большую часть времени по пути в Карфаген Иисус беседовал со своими спутниками на социальные, политические и торговые темы; о религии почти не говорили. Гонод и Ганид впервые обнаружили, что Иисус был хорошим рассказчиком, и постоянно просили рассказывать истории о ранних годах его жизни в Галилее. Они узнали и то, что он воспитывался в Галилее, а не в Иерусалиме или Дамаске.
\vs p130 7:2 Когда Ганид, видя, как влекло к Иисусу почти всех людей, которые им встречались, спросил, что должен делать человек, чтобы подружиться с другим человеком, Иисус ответил: «Начни интересоваться своими собратьями; учись, как любить их, и ищи возможность сделать для них то, что, по твоему мнению, они хотят, чтобы для них сделали», и затем он процитировал старую еврейскую пословицу --- «Кто хочет иметь друзей, тот и сам должен быть дружелюбным».
\vs p130 7:3 В Карфагене у Иисуса была длинная и незабываемая беседа со служителем культа Митры о бессмертии, времени и вечности. Этот перс получил образование в Александрии и действительно хотел учиться у Иисуса. Выраженная современным языком, сущность сказанного Иисусом в ответ на его многочисленные вопросы звучит так:
\vs p130 7:4 \pc Время есть поток текущих преходящих событий, воспринимаемых сознанием живого создания. Время --- это название, данное непрерывной последовательности событий, посредством которого они распознаются и вычленяются. Пространственная вселенная есть феномен, сопряженный со временем, если смотреть на нее из любой внутренней точки, находящейся вне неизменного местонахождения Рая. Течение времени обнаруживается только в связи с чем\hyp{}то, что не движется в пространстве как временное явление. Во вселенной вселенных Рай и его Божества все трансцендентны времени и пространству. В обитаемых мирах человеческая личность (в которой постоянно пребывает направляющий ее дух Райского Отца) есть единственная физическая реальность, которая может выйти за пределы материальной последовательности временных событий.
\vs p130 7:5 Животные не чувствуют времени так, как люди, и даже человеку, из\hyp{}за его фрагментарного и ограниченного восприятия, время предстает как последовательность событий; но по мере того, как человек идет по пути восхождения и внутренне совершенствуется, он обретает более широкое видение этой череды событий, которая все больше и больше воспринимается им в своей целостности. То, что раньше казалось чередой событий, позже будет рассматриваться как цельный и полностью связанный цикл; таким образом в сознании круговая одновременность событий будет все больше вытеснять одномоментное осознание их линейной последовательности.
\vs p130 7:6 Существует семь различных концепций пространства, обусловленного временем. Пространство измеряется временем, а не время --- пространством. У ученых путаница возникает из\hyp{}за неумения распознать реальность пространства. Пространство есть не просто интеллектуальное понятие, сводящееся к изменению взаимосвязей объектов во вселенной. Пространство не пусто, и единственная известная человеку сущность, которая может отчасти даже выходить за пределы пространства, есть разум. Разум может функционировать независимо от пространственной взаимосвязи материальных объектов. Пространство относительно и сравнительно конечно для всех сотворенных существ. Чем ближе сознание подходит к осознанию наличия семи космических измерений, тем больше понятие потенциального пространства приближается к предельности. Но потенциал пространства действительно пределен только на абсолютном уровне.
\vs p130 7:7 Несомненно, что вселенская реальность имеет расширяющееся и всегда относительное значение на восходящих и совершенствующихся уровнях космоса. В конечном счете, продолжающие существование смертные достигают идентичности только в семимерной вселенной.
\vs p130 7:8 \pc Пространственно\hyp{}временному представлению разума, имеющего материальную природу, предназначено претерпевать последовательные расширения по мере того, как сознательная и рассуждающая личность поднимается вверх по уровням вселенной. Когда человек поднимается на такую ступень развития разума, которая находится между материальным и духовным уровнями существования, его идеи времени\hyp{}пространства необыкновенно расширяются как в плане качества восприятия, так и в отношении количества лично обретенного опыта. Расширение космических представлений продвигающейся по пути восхождения духовной личности происходит благодаря увеличению как глубины осознания, так и диапазона сознания. И по мере того как личность переходит, двигаясь вверх и вглубь, к трансцендентным уровням Богоподобия, пространственно\hyp{}временные представления все больше приближаются к вневременным и внепространственным понятиям Абсолютов. По мере достижения трансцендентного уровня эти понятия абсолютного уровня становятся относительно доступны видению детей предельного предназначения.
\usection{8. На пути в Неаполь и Рим}
\vs p130 8:1 Первая остановка на пути в Италию была сделана на острове Мальта. Здесь Иисус долго беседовал с разочарованным и отчаявшимся молодым человеком по имени Клавдий. Этот молодой человек раздумывал, не покончить ли ему с жизнью, но, поговорив с книжником из Дамаска, сказал: «Я буду смело смотреть в лицо жизни, как подобает человеку. Я больше не буду трусить. Я вернусь к своим близким и начну все сначала». Вскоре он стал горячим проповедником кинизма, позже объединился с Петром, чтобы проповедовать христианство в Риме и Неаполе, а после смерти Петра отправился проповедовать евангелие в Испанию. Но он так никогда и не узнал, что человек, вдохновивший его на Мальте, был Иисус, которого он впоследствии провозгласил Спасителем мира.
\vs p130 8:2 \pc В Сиракузах они провели целую неделю. Самым выдающимся событием за время их пребывания там было обращение Езры, отпавшего от веры еврея, который держал таверну, где остановились Иисус и его спутники. Езра был пленен обхождением Иисуса и попросил помочь ему вернуться к вере Израиля. О мучавшем его чувстве безнадежности он сказал: «Я хочу быть верным сыном Авраама, но я не могу найти Бога». Иисус ответил ему: «Если ты действительно хочешь найти Бога, то это желание само по себе есть свидетельство того, что ты уже нашел его. Беда не в том, что ты не можешь найти Бога, ибо Отец уже нашел тебя; твоя беда просто в том, что ты не знаешь Бога. Разве ты не читал у пророка Иеремии: „И взыщете Меня и найдете, если взыщете Меня всем сердцем вашим“? И еще, разве этот же пророк не говорит „И дам им сердце, чтобы знать Меня, что я --- Господь, и они будут Моим народом, а я буду их Богом?“ И не читал ли ты также Писания, где говорится: „Он будет смотреть на людей и если кто скажет: Грешил я и извращал правду, и не воздано мне, тогда освободит Бог душу его от тьмы и увидит он свет„«? И Езра нашел Бога к удовлетворению своей души. Позже этот еврей совместно с одним состоятельным новообращенным греком построил первую христианскую церковь в Сиракузах.
\vs p130 8:3 \pc В Мессине они остановились всего на один день, но и этого было вполне достаточно, чтобы изменить жизнь маленького мальчика, продавца фруктов, у которого Иисуса купил фрукты и которого в свою очередь накормил хлебом жизни. Мальчик никогда не забывал слов Иисус и его доброго взгляда, когда тот сказал, положив руку ему на плечо: «Прощай, мой мальчик, и будь по\hyp{}настоящему мужественным, когда вырастешь, и после того, как ты насытишь тело, учись насыщать и душу. И мой Отец на небесах будет с тобой и будет идти впереди тебя». Мальчик стал почитателем культа Митры, а позже обратился в христианскую веру.
\vs p130 8:4 \pc Наконец они достигли Неаполя и поняли, что находятся недалеко от конечной цели своего путешествия, Рима. У Гонода было много дел в Неаполе, и ему нужен был Иисус в качестве переводчика, а остальное время Иисус и Ганид на досуге осматривали достопримечательности и бродили по городу. Ганид научился безошибочно узнавать людей, оказавшихся в нужде. Они обнаружили множество нищих в этом городе и раздали много милостыни. Но Ганид никогда не смог понять смысла слов Иисуса, когда тот отказался остановиться и сказать слово утешения уличному бродяге, которому подал монету. Иисус сказал: «К чему впустую тратить слова на того, кто не может воспринять смысла сказанного тобой? Дух Отца не может учить и спасать того, кто не способен быть его сыном». Иисус имел в виду, что у этого человека не было нормального разума и он был не способен отзываться на духовное водительство.
\vs p130 8:5 В Неаполе не случилось никакого из ряда вон выходящего события. Иисус и молодой человек тщательно изучили город и одарили хорошим настроением и множеством улыбок сотни мужчин, женщин и детей.
\vs p130 8:6 Оттуда они отправились в Рим по дороге, идущей через Капую, где и остановились на три дня. Рядом со своими вьючными животными они шли по Аппиевой дороге в Рим, и все трое горели желанием увидеть эту владычицу империи и величайший город в мире.
