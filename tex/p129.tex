\upaper{129}{Дальнейшие годы взрослой жизни Иисуса}
\vs p129 0:1 Иисус полностью и окончательно устранился от управления домашними делами Назаретского семейства и от непосредственного руководства каждым его членом. Он продолжал, вплоть до принятия крещения, вносить деньги в бюджет семьи и проявлять острую заинтересованность в духовном благополучии всех своих братьев и сестер. И он всегда был готов сделать все, что было в человеческих силах, для благоденствия и счастья своей вдовствующей матери.
\vs p129 0:2 Теперь Сын Человеческий сделал все необходимые приготовления, чтобы навсегда покинуть Назаретский дом; и ему было нелегко сделать это. Иисус, конечно, любил своих близких; он любил свою семью, и эта естественная привязанность усугублялась его необычайной преданностью им. Чем больше мы преданы своим ближним, тем сильнее мы начинаем любить их; и так как Иисус столь самоотверженно посвятил себя семье, то он пылко и глубоко их любил.
\vs p129 0:3 Мало\hyp{}помалу вся семья начала осознавать, что Иисус готовится покинуть их. Грусть предстоящей разлуки лишь в некоторой степени смягчалась тем, что их постепенно подготавливали к объявлению об этом предполагаемом отъезде. Более четырех лет они наблюдали, как он собирается окончательно отделиться о них.
\usection{1. Двадцать седьмой год (21 г. н.э.)}
\vs p129 1:1 В январе этого года, 21 г. н.э., в дождливое воскресное утро Иисус без особенных церемоний покинул свою семью, сообщив им только то, что он направляется в Тивериаду, и намеревается далее посетить другие города на Галилейском море. Вот так он и покинул их, и больше никогда он уже не был постоянным членом этой семьи.
\vs p129 1:2 Он провел неделю в Тивериаде, новом городе, который должен был вскоре стать столицей Галилеи вместо Сефориса; не найдя там ничего интересного для себя, он, пройдя через Магдалу и Вифсаиду, прибыл в Капернаум, где и остановился, чтобы навестить Зеведея, друга своего отца. Сыновья Зеведея были рыбаками; сам он был корабельщиком. Иисус из Назарета был опытен и в конструировании, и в строительстве; он был искусным столяром; и Зеведей давно знал о мастерстве назаретского ремесленника. Долгое время Зеведей размышлял над созданием лодок усовершенствованной конструкции; теперь он изложил свой план Иисусу и пригласил гостившего у него плотника присоединиться к своему предприятию, и Иисус с готовностью согласился.
\vs p129 1:3 Иисус проработал с Зеведеем чуть больше года, но за это время он создал лодку новой конструкции и разработал совершенно новый метод изготовления лодок. С помощью более совершенной техники и значительно улучшенной технологии обработки досок паром Иисус и Зеведей начали строить лодки самой лучшей конструкции, суда, которые были гораздо более безопасны для плавания по озеру, чем лодки старого типа. В течение нескольких лет у Зеведея было больше заказов на эти новые лодки, чем могло справиться его маленькое предприятие; меньше чем за пять лет практически все суда на озере были построены в мастерской Зеведея в Капернауме. Иисус стал хорошо известен рыбакам как создатель этих новых лодок.
\vs p129 1:4 Зеведей был достаточно состоятельным человеком; его судостроительные мастерские находились на озере к югу от Капернаума, а дом --- ниже по побережью недалеко от рыбацких кварталов Вифсаиды. Год с небольшим, который Иисус провел в Капернауме, он жил в доме Зеведея. Он долго работал в мире в одиночку, то есть без отца, и ему очень понравился этот период работы с партнером, заменившим отца.
\vs p129 1:5 Жена Зеведея, Саломея, была родственницей Анны, бывшего в свое время первосвященником в Иерусалиме и все еще самого влиятельного из саддукеев, так как он был смещен всего за восемь лет до того. Саломея прониклась большим почтением к Иисусу. Она любила его так же, как своих собственных сыновей, Иакова, Иоанна и Давида, в то время как ее четыре дочери смотрели на Иисуса как на своего старшего брата. Иисус часто выходил на рыбную ловлю с Иаковом, Иоанном и Давидом, и они узнали, что он такой же опытный рыбак, как и судостроитель.
\vs p129 1:6 \P\ Весь этот год Иисус каждый месяц посылал деньги Иакову. Он вернулся в Назарет в октябре, чтобы присутствовать на свадьбе Марфы, и после этого снова не был в Назарете больше двух лет, когда он ненадолго вернулся перед двойной свадьбой Симона и Иуды.
\vs p129 1:7 \P\ В течение этого года Иисус строил лодки и продолжал наблюдать, как живут люди на земле. Часто он ходил в Капернаум, чтобы побывать на месте стоянки караванов, так как город находился на прямой проезжей дороге из Дамаска на юг. Капернаум был укрепленным римским военным постом, и офицер, командовавший гарнизоном, был неевреем, веровавшим в Ягве, «благочестивым человеком», как обычно называли таких новообращенных евреи. Этот офицер принадлежал к богатой римской семье, и он взялся построить в Капернауме прекрасную синагогу, которая была подарена евреям незадолго до того, как Иисус начал жить у Зеведея. В этом году Иисус провел больше половины всех служб в этой новой синагоге, и некоторые из людей, путешествовавших с караванами, которым довелось присутствовать на них, запомнили его как плотника из Назарета.
\vs p129 1:8 Когда дело дошло до уплаты налогов, Иисус зарегистрировался как «квалифицированный ремесленник из Капернаума». Начиная с этого дня и до конца его земной жизни он был известен как житель Капернаума. Он никогда не заявлял никакого другого официального места жительства, хотя по разным причинам позволял другим считать местом его жительства Дамаск, Вифанию, Назарет и даже Александрию.
\vs p129 1:9 В синагоге Капернаума он нашел много новых книг в сундуках библиотеки и не менее пяти вечеров в неделю проводил за их усиленным изучением. Один вечер он посвящал общению со старшими и один проводил с молодежью. В личности Иисуса была некая благожелательность и притягательность, что неизменно привлекало молодых людей. Он всегда умел сделать так, что они чувствовали себя свободно в его присутствии. Возможно, самый главный секрет его успеха у них заключался в том, что, с одной стороны, ему всегда было интересно то, чем они занимались, но, с другой стороны, он редко давал им советы, если только они сами его об этом не просили.
\vs p129 1:10 Семья Зеведея почти боготворила Иисуса, и они никогда не упускали случая бывать на собраниях, где присутствующие обсуждали с ним разные вопросы, эти встречи он проводил каждый вечер после ужина, прежде чем отправиться в синагогу для занятий. Часто приходила и соседская молодежь, чтобы принять участие в этих вечерних собраниях. На этих встречах в узком кругу Иисус проводил беседы на разные сложны темы, которые, однако, всегда были доступны пониманию слушателей. Он говорил с ними, совершенно откровенно выражая свое отношение и взгляды на политику, социологию, науку и философию, но никогда не осмеливался говорить с позиций непререкаемого авторитета, за исключением тех случаев, когда обсуждали вопросы религии --- отношение человека к Богу.
\vs p129 1:11 Раз в неделю Иисус собирал вместе всех членов семьи и всех людей, работавших в мастерской, и на берегу, так как у Зеведея было много работников. И именно среди этих рабочих Иисус впервые был назван Учителем. Все они любили его. И ему нравилось работать с Зеведеем в Капернауме, но он скучал по детям, игравшим на улице возле плотницкой мастерской в Назарете.
\vs p129 1:12 Из всех сыновей Зеведея Иакова больше других интересовал Иисус как учитель, как философ. Иоанна больше волновали его религиозные взгляды и учение. Давид уважал его как мастера, но на его религиозные воззрения и философское учение обращал мало внимания.
\vs p129 1:13 По субботам часто приезжал Иуда, чтобы послушать беседу Иисуса в синагоге, после которой оставался, чтобы повидаться с ним. И чем больше Иуда виделся со своим старшим братом, тем больше он убеждался, что Иисус поистине великий человек.
\vs p129 1:14 \P\ В этом году Иисус сделал большие успехи в овладении своим человеческим разумом и поднялся на новый высокий уровень сознательного общения с Настройщиком Мысли, пребывающим в нем.
\vs p129 1:15 Это был последний год его спокойной жизни. Больше никогда Иисус не проведет целый год на одном месте или за одним занятием. Дни его земного паломничества стремительно приближались. До времени его интенсивной деятельности оставалось совсем немного, но его простую, но исключительно деятельную жизнь в прошлом отделяли от его еще более интенсивго и напряженного публичного служения несколько лет длительных путешествий и чрезвычайно разнообразной личной деятельности. Его образование как человека мира сего должно было завершиться, прежде чем он смог бы приступить к своей миссии учительства и проповедования как совершенный Богочеловек божественного и послечеловеческого этапа своего пришествия на Урантию.
\usection{2. Двадцать восьмой год (22 г. н.э.)}
\vs p129 2:1 В марте 22 г. н.э. Иисус покинул Зеведея и Капернаум. Он попросил небольшую сумму денег, чтобы оплатить расходы на дорогу в Иерусалим. За работу у Зеведея, он забирал лишь небольшие суммы денег, которые он каждый месяц посылал своей семье в Назарет. Один месяц Иосиф приходил в Капернаум за деньгами, в следующий месяц Иуда заходил в Капернаум, чтобы взять деньги у Иисуса и отнести их в Назарет. Место, где рыбачил Иуда, находилось всего в нескольких милях к югу от Капернаума.
\vs p129 2:2 Когда Иисус прощался с семьей Зеведея, он согласился остаться в Иерусалиме до наступления Пасхи, и все они пообещали присутствовать там во время этого события. Они даже договорились совершить праздничную Пасхальную трапезу вместе. Все они очень горевали, когда Иисус покинул их, особенно дочери Зеведея.
\vs p129 2:3 \P\ Прежде чем покинуть Капернаум, Иисус долго разговаривал со своим новообретенным другом и близким товарищем, Иоанном, сыном Зеведея. Он сказал Иоанну, что предполагает проводить время в длительных путешествиях до тех пор, «пока не настанет мой час», и попросил Иоанна вместо него ежемесячно пересылать немного денег в Назарет его семье, до тех пор, пока причитающаяся ему сумма денег не исчерпается. И Иоанн дал ему следующее обещание: «Учитель, иди и занимайся своим делом, делай свою работу в мире; я заменю тебя в этом или в любом другом деле, и я буду охранять твою семью так же, как лелеял бы мою собственную мать и заботился бы о моих собственных братьях и сестрах. Я буду расходовать твои деньги, находящиеся у моего отца, так, как ты распорядился, и по мере того, как они будут нужны; а когда твои деньги будут израсходованы, если я не получу от тебя еще денег, а твоя мать будет нуждаться, тогда я разделю с ней мой собственный заработок. Иди с миром своей дорогой. Я заменю тебя во всех этих делах».
\vs p129 2:4 Поэтому после того, как Иисус отправился в Иерусалим, Иоанн справился у своего отца, Зеведея, о деньгах, причитавшихся Иисусу, и был удивлен, узнав, что это такая большая сумма. Поскольку Иисус полностью передал это дело в их руки, они пришли к выводу, что лучше всего было бы вложить эти деньги в недвижимость и использовать доход от нее для помощи семье в Назарете; и так как Зеведей знал об одном небольшом доме в Капернауме, который был заложен и должен был быть продан, он посоветовал Иоанну купить этот дом на деньги Иисуса и закрепить право владения им за своим другом. И Иоанн поступил так, как посоветовал ему отец. В течение двух лет сумма ренты предназначалась для выплаты денег по закладной, и эта сумма вместе с другой крупной суммой, которую Иисус вскоре прислал Иоанну, чтобы он использовал ее на нужды семьи, оказалась почти равна величине заклада; Зеведей покрыл разницу, так что, когда пришло время, Иоанн заплатил остаток заклада, тем самым обеспечив свободное право владения этим двухкомнатным домом. Таким образом Иисус стал владельцем дома в Капернауме, но ему не было сообщено об этом.
\vs p129 2:5 \P\ Когда семья в Назарете услышала, что Иисус отбыл из Капернаума, они, не зная о его финансовом соглашении с Иоанном, решили, что для них пришло время обходиться в дальнейшем без какой бы то ни было помощи со стороны Иисуса. Иаков помнил о своем договоре с Иисусом и с помощью своих братьев немедленно принял на себя полную ответственность за обеспечение семьи.
\vs p129 2:6 \P\ Но давайте вернемся назад, чтобы посмотреть, что делает Иисус в Иерусалиме. В течение почти двух месяцев большую часть времени он проводил в храме, слушая споры, и время от времени посещал различные школы раввинов. Субботние дни он, в основном, проводил в Вифании.
\vs p129 2:7 Иисус привез с собой в Иерусалим письмо от Саломеи, жены Зеведея, рекомендующее его бывшему первосвященнику, Анне, как того, «кто для меня то же, что и мой собственный сын». Анна провел с ним много времени, лично сопровождал его при посещении многих академий религиозных учителей в Иерусалиме. Несмотря на то, что Иисус внимательно осматривал эти школы и тщательно изучал их методы обучения, он ни разу не задал открыто ни одного вопроса. Хотя Анна считал Иисуса великим человеком, он пребывал в замешательстве, что ему посоветовать. Он понимал, что было бы глупо предлагать ему поступать в одну из школ Иерусалима в качестве ученика, и в то же время он хорошо знал, что Иисус никогда не будет удостоен звания профессионального учителя, поскольку он никогда не учился в этих школах.
\vs p129 2:8 Вскоре приблизилось время Пасхи, и вместе с толпами людей, приехавшими отовсюду, в Иерусалим из Капернаума прибыл и Зеведей со всей своей семьей. Все они остановились в просторном доме Анны, где отпраздновали Пасху как единая счастливая семья.
\vs p129 2:9 \P\ Еще до конца Пасхальной недели Иисус, по\hyp{}видимому случайно, встретил одного богатого путешественника с сыном, молодым человеком примерно семнадцати лет. Эти путешественники были родом из Индии; находясь на пути в Рим и в разные другие города Средиземноморья, они устроили так, чтобы прибыть в Иерусалим во время Пасхи, в надежде найти кого\hyp{}нибудь, кого можно было бы нанять как переводчика для них обоих и наставника для сына. Отец настаивал на том, чтобы Иисус согласился путешествовать с ними. Иисус рассказал ему о своей семье и о том, что с его стороны едва ли было бы правильно уехать почти на два года, в течение которых они могут оказаться в нужде. После этого этот путешественник с Востока предложил выдать Иисусу вперед годичное жалованье с тем, чтобы он мог доверить эту сумму своим друзьям, которые, таким образом, могли бы избавить его семью от нужды. И Иисус согласился совершить путешествие.
\vs p129 2:10 Иисус передал большую сумму денег Иоанну, сыну Зеведея. И вы уже слышали рассказ о том, как Иоанн употребил эти деньги на покрытие закладных обязательств по собственности в Капернауме. Перед этой поездкой по Средиземноморью Иисус совершенно доверился Зеведею, но велел ему не говорить об этом никому, даже своей собственной плоти и крови, и в течение этого двухлетнего периода Зеведей никому не рассказывал о местонахождении Иисуса. Незадолго до возвращения Иисуса из этого путешествия семья в Назарете почти не сомневалась, что его уже нет в живых. И только заверения Зеведея, который вместе со своим сыном Иоанном несколько раз по разным поводам приходил в Назарет, не давали надежде в сердце Марии умереть.
\vs p129 2:11 \P\ В течение этого времени дела семьи в Назарете шли очень хорошо; Иуда значительно увеличил размер своей доли и продолжал вносить некоторую сумму сверх обычного взноса вплоть до своей женитьбы. Несмотря на то, что они не очень нуждались в помощи, Иоанн Зеведеев, выполняя распоряжение Иисуса, каждый месяц привозил подарки для Марии и Руфи.
\usection{3. Двадцать девятый год (23 г. н.э.)}
\vs p129 3:1 Весь двадцать девятый год жизни Иисус провел, завершая путешествия по Средиземноморью. В той мере, в какой нам разрешено открыть этот опыт, основные события представлены в рассказах, которые следуют непосредственно за этим текстом.
\vs p129 3:2 \P\ По многим причинам во время всего путешествия по римскому миру Иисус был известен как \bibemph{книжник из Дамаска.} В Коринфе и в других городах, в которых они останавливались на обратном пути, он, однако, был известен как \bibemph{еврейский наставник.}
\vs p129 3:3 Этот период жизни Иисуса был богат событиями. Во время этого путешествия он много общался с людьми, но этот опыт --- та сторона его жизни, которую он никогда не открывал никому из членов своей семьи и ни одному из апостолов. Иисус прожил свою жизнь во плоти и покинул этот мир так, что никто (за исключением Зеведея из Вифсаиды) не знал, что он совершил это длинное путешествие. Некоторые из его друзей думали, что он вернулся в Дамаск; другие думали, что он уехал в Индию. Его собственная семья склонялась к мысли, что он в Александрии, так как они знали, что некогда он был приглашен туда на должность помощника хазана.
\vs p129 3:4 Когда Иисус вернулся в Палестину, он не сделал ничего, чтобы изменить мнение своей семьи, что из Иерусалима он отбыл в Александрию. Он позволил им и дальше пребывать в уверенности, что все то время, что его не было в Палестине, он провел в этом городе образования и культуры. Только Зеведей, корабельщик из Вифсаиды, знал правду об этом деле, но никому не сказал.
\vs p129 3:5 \P\ Пытаясь понять значение жизни Иисуса на Урантии, вы должны помнить о причинах пришествия Михаила. Чтобы понять значение многих его на первый взгляд странных поступков, вы должны, во\hyp{}первых, постичь цель его пребывания в вашем мире. Он постоянно заботился о том, чтобы не сделать выдающейся и приковывающей внимание личной карьеры. Он не хотел оказывать на своих собратьев какого бы то ни было необычного или подавляющего влияния. Его работа состояла в том, чтобы открыть Отца Небесного своим смертным собратьям, и в то же время посвятить себя величественной задаче прожить свою смертную жизнь, все время подчиняясь воле того же самого Райского Отца.
\vs p129 3:6 \P\ Для понимания жизни Иисуса на земле будет также полезно, если все смертные ученики его божественного пришествия будут помнить, что пока он жил этой воплощенной жизнью на Урантии, он жил \bibemph{для} всей своей вселенной. Для каждой обитаемой сферы всей вселенной Небадона нечто чрезвычайное и вдохновляющее было связано с жизнью, которую он проживал во плоти, имевшей смертную природу. То же самое верно и для всех тех миров, которые стали обитаемыми после насыщенного событиями времени его пребывания на Урантии. И это будет равным образом так же верно для всех миров, которые могут стать обитаемы существами, наделенными волей, во всей дальнейшей истории этой локальной вселенной.
\vs p129 3:7 \P\ Благодаря опыту, приобретенному за время этого путешествия по римскому миру, Сын Человеческий практически закончил свое направленное на получение знаний общение с самыми разными людьми своего времени и поколения. Ко времени своего возвращения в Назарет благодаря этому путешествию\hyp{}обучению он узнал уже почти все о том, как человек живет и обеспечивает свое существование на Урантии.
\vs p129 3:8 Истинной целью этого путешествия по Средземноморью было узнать людей. Во время этого путешествия он тесно сходился с сотнями представителей рода человеческого. Он встречал и проникался любовью к самым разным людям, богатым и бедным, высоким и низким, черным и белым, образованным и необразованным, культурным и некультурным, приземленным и духовным, религиозным и нерелигиозным, нравственным и безнравственным.
\vs p129 3:9 Во время своего средиземноморского путешествия Иисус как человек сделал большие успехи в совершенствовании своего материального и смертного разума, и постоянно пребывавший в нем Настройщик Мысли тоже значительно преуспел в возвышении и духовном овладении того же самого человеческого интеллекта. К концу этого путешествия Иисус, по существу, знал --- со всей возможной для человека уверенностью, --- что он был Сыном Божиим, Сыном\hyp{}Творцом Вселенского Отца. Настройщику все легче удавалось пробуждать в уме Сына Человеческого туманные воспоминания о его Райском опыте, связанном с его божественным Отцом, предшествовавшем его приходу в эту локальную вселенную Небадона с целью ее устройства и управления ею. Таким образом Настройщик мало помалу привносил в человеческое сознание Иисуса эти необходимые воспоминания о его прошлом божественном существовании в разнообразных эпохах почти вечного прошлого. Последним эпизодом предшествующего его воплощению опыта, который Настройщику предстояло возродить, было его прощальное совещание с Иммануилом из Спасограда как раз перед тем, как он отказался от сознательной личности, чтобы целиком отдаться воплощению на Урантии. И это последнее воспоминание из существования, предшествовавшего воплощению, прояснилось в сознании Иисуса в тот самый день, когда Иоанн крестил его в Иордане.
\usection{4. Иисус\hyp{}человек}
\vs p129 4:1 Для наблюдающих небесных духов локальной вселенной это средиземноморское путешествие было самым увлекательным из всех событий земной жизни Иисуса вплоть до его распятия и плотской смерти. Это был чудесный период его \bibemph{личного служения,} в отличие от эпохи публичного служения, которая должна была наступить уже очень скоро. Это удивительное событие поглощало все внимание наблюдателей, потому что в это время он все еще был плотником из Назарета, строителем лодок из Капернаума, книжником из Дамаска; он все еще был Сыном Человеческим. Он все еще не достиг полного господства над своим человеческим разумом; Настройщик еще не до конца овладел смертной личностью и не закончил создание двойника его смертной идентичности. Иисус все еще был человеком среди людей.
\vs p129 4:2 В течение этого двадцать девятого года в чисто человеческом религиозном опыте --- личном духовном росте --- Сын Человеческий почти достиг вершины возможного. Этот опыт духовного развития представлял собой постоянный постепенный рост, начавшийся в момент прибытия его Настройщика Мысли и окончившийся в день завершения и подтверждения этого естественного и нормального для человека взаимоотношения между материальным разумом человека и умственной одаренностью духа --- феномен превращения этих двух разумов в один, опыт, которого Сын Человеческий достиг в совершенстве и завершенности как воплощенный смертный мира сего в день его крещения в Иордане.
\vs p129 4:3 Хотя в течение этих лет периоды формального общения с его Отцом Небесным не были такими уж многочисленными, он все более совершенствовал эффективные способы личного общения с постоянно пребывающим в нем духовным присутствием Райского Отца. Он жил настоящей жизнью, полной жизнью, действительно нормальной, естественной и обычной жизнью во плоти. Он на личном опыте познал эквивалент реальности всей полноты жизни человеческих существ в материальных мирах со временем и пространством.
\vs p129 4:4 Сын Человеческий испытал ту резкую перемену человеческого настроения, при которой острая радость сменяется глубокой печалью. Он был радостным ребенком и на редкость жизнелюбивым человеком, и вместе с тем «человеком, претерпевшим скорбь и познавшим печаль». В духовном смысле он действительно прожил смертную жизнь целиком, от основания до вершины, от начала и до конца. С материальной точки зрения может показаться, что он избежал обеих крайностей социального существования человека, но интеллектуально в совершенстве познал опыт человечества во всей его полноте и завершенности.
\vs p129 4:5 Иисус знает о мыслях и чувствах, порывах и побуждениях эволюционных и восходящих смертных этого мира от рождения до смерти. Он прожил человеческую жизнь от начала физической, интеллектуальной и духовной индивидуальности и до опыта самой смерти, пройдя через младенчество, детство, юность и зрелость. Он не только прошел эти обычные и знакомые людям периоды интеллектуального и духовного развития, но он также по собственному опыту знал те более высокие и развитые стадии примирения человека и Настройщика, которых лишь немногие смертные когда\hyp{}либо достигали на Урантии. Тем самым он прожил полную жизнь смертного человека не только так, как ее проживают в вашем мире, но и так, как ее проживают во всех остальных эволюционирующих мирах со временем и пространством, даже в наиболее развитом и самом высоком из всех миров, установленных в свете и жизни.
\vs p129 4:6 Хотя эта совершенная жизнь, которую он прожил в облике смертной плоти, могла не получить безоговорочного и всеобщего одобрения со стороны его смертных собратьев, случайно оказавшихся его современниками на Земле, тем не менее жизнь, которую Иисус из Назарета прожил во плоти на Урантии, получила действительное одобрение Отца Всего Сущего как жизнь, которая в одно и то же время и в одной и той же личности осуществила полноту откровения вечного Бога смертным людям и явила безупречную человеческую личность, удовлетворившую Бесконечного Творца.
\vs p129 4:7 И это было его истинной и верховной целью. Он пришел жить на Урантии не для того, чтобы служить совершенным и наглядным примером для каждого ребенка и взрослого, мужчины и женщины той эпохи или любой другой. Правда, на самом деле в его полной, богатой, красивой и возвышенной жизни мы все можем найти много исключительно образцового и божественно вдохновляющего, но это потому, что он жил истинной и подлинно человеческой жизнью. Иисус прожил свою жизнь на земле не для того, чтобы дать остальным человеческим существам пример для подражания. Он прожил эту жизнь во плоти в том же самом милосердном служении, в котором и вы можете прожить ваши жизни на земле; и он прожил свою смертную жизнь в свое время и в соответствии с тем, \bibemph{каким он был,} и этим он действительно подал нам пример прожить наши жизни в наше время и в соответствии с тем, \bibemph{каковы мы есть.} Вы не можете стремиться прожить его жизнь, но вы можете решить \bibemph{прожить свою жизнь} так же и тем же способом, как он прожил свою. Иисус не может служить конкретным и подробным примером на все случаи жизни для всех смертных всех возрастов во всех владениях этой локальной вселенной, но он служит вечным источником вдохновения и проводником всех Райских пилигримов из миров начального восхождения вверх через вселенную вселенных и дальше через Хавону к Раю. Иисус есть \bibemph{новый живой путь} от человека к Богу, от неполноты к совершенству, от земного к небесному, от времени к вечности.
\vs p129 4:8 \P\ К концу двадцать девятого года Иисус из Назарета фактически прожил жизнь так, как это требуется от смертных как от существ, облеченных в плоть. Он пришел на землю, чтобы полнота Бога была явлена человеку; теперь он стал почти совершенным человеком, ожидающим возможности быть явленным Богу. И он сделал все это до того, как ему исполнилось тридцать лет.
