\upaper{190}{Моронтийные явления Иисуса}
\author{Комиссия срединников}
\vs p190 0:1 Теперь воскресший Иисус готовится провести на Урантии какое\hyp{}то время, чтобы самому познать моронтийный путь восхождения, который проходят смертные миров. Хотя этот период моронтийной жизни должен пройти в мире его смертного воплощения, он, однако, во всех отношениях будет соответствовать опыту смертных Сатании, совершающих последовательный переход от моронтийной жизни в одном из семи миров\hyp{}обителей Иерусема к жизни в другом.
\vs p190 0:2 Эта власть --- власть даровать жизнь, присущая Иисусу и позволившая ему воскреснуть из мертвых, и есть способность жить вечно, которой он наделяет верующих царства и которая даже сейчас делает непреложным их воскрешение из оков естественной смерти.
\vs p190 0:3 Смертные миров воскреснут в утро воскресения в переходном или таком же моронтийном теле, какое имел и Иисус, восстав из гробницы в это воскресное утро. Эти тела не имеют кровообращения, и подобные существа не вкушают обычной материальной пищи; тем не менее эти моронтийные тела \bibemph{реальны.} Когда разные верующие видели Иисуса после воскресения, они его действительно видели и не были самообольщающимися жертвами видений или галлюцинаций.
\vs p190 0:4 Твердая вера в воскресение Иисуса была краеугольным положением веры во всех ветвях раннего евангельского учения. В Иерусалиме, Александрии, Антиохии и Филадельфии все учителя евангелия были едины в этой безусловной вере в воскресение Учителя.
\vs p190 0:5 \pc Что же касается того, какое знаменательное участие приняла Мария Магдалина в провозглашении воскресения Учителя, то следует отметить, что Мария была главным представителем отряда женщин, таким же, как Петр --- отряда апостолов. Мария не была главой женщин\hyp{}служительниц, но была их главным учителем и общественным представителем. Мария стала женщиной весьма осмотрительной, поэтому смелость, проявленная ею в разговоре с мужчиной, которого она приняла за садовника Иосифа, свидетельствует лишь о том, в какой ужас она пришла, найдя гробницу пустой. Это ее глубокая и отчаянная любовь и безграничная преданность заставили ее на минуту забыть о традиционной сдержанности, с которой еврейская женщина должна была подходить к незнакомому мужчине.
\usection{1. Вестники воскресения}
\vs p190 1:1 Апостолы не хотели, чтобы Иисус покидал их, а потому пренебрегли всеми его словами о смерти и его обещаниями воскреснуть. Они не ожидали, что воскресение будет таким, каким оно было, и отказывались верить в него, пока на своем собственном опыте не столкнулись с неопровержимыми свидетельствами и безусловными доказательствами.
\vs p190 1:2 Когда апостолы отказались поверить рассказу пяти женщин, которые утверждали, что видели Иисуса и говорили с ним, Мария Магдалина вернулась к гробнице, а остальные пошли обратно в дом Иосифа, где и поведали его дочери и другим женщинам о пережитом ими. И женщины их рассказу поверили. В начале седьмого дочь Иосифа Аримафейского и четыре женщины, видевшие Иисуса, отправились в дом Никодима, где рассказали обо всем случившемся Иосифу, Никодиму, Давиду Зеведееву и другим людям, которые там собрались. Никодим и остальные сомневались в правдивости их рассказа, сомневались в том, что Иисус воскрес из мертвых и предположили, что евреи спрятали его тело. Иосиф же и Давид были склонны поверить рассказу, причем настолько, что поспешили отправиться осмотреть гробницу, где и нашли все в точности таким, как описывали женщины. И они были последними, кто лицезрел гроб в таком виде, ибо в половине восьмого первосвященник послал командира храмовых стражников к гробнице убрать погребальные одежды. Командир завернул их в полотняную простыню и сбросил с ближайшей скалы.
\vs p190 1:3 От гробницы Давид и Иосиф сразу пошли к дому Илии Марка, где в комнате наверху держали совет с десятью апостолами. Только Иоанн Зеведеев был готов поверить, хотя и не совсем, что Иисус воскрес из мертвых. Петр же сначала поверил, но, не найдя Учителя, впал в глубокое сомнение. Все они склонялись к мысли, что евреи спрятали тело. Давид не хотел с ними спорить, но уходя сказал: «Вы апостолы и должны понимать эти вещи. Я с вами спорить не буду; тем не менее я сейчас пойду к дому Никодима, где этим утром я назначил вестникам сбор, и когда они соберутся, пошлю их с последней миссией как вестников воскресения Учителя. Я слышал, как Учитель сказал, что он умрет и на третий день воскреснет, и я ему верю». Сказав это удрученным и отчаявшимся посланцам царства, этот начальник связи и разведки, сам назначивший себя на эту должность, покинул апостолов. Выходя из комнаты на верхнем этаже, он бросил суму Иуды, в которой была вся апостольская казна, на колени Матфею Левию.
\vs p190 1:4 Было около половины десятого, когда последний из двадцати шести вестников Давида прибыл в дом Никодима. Давид быстро собрал их на просторном дворе и обратился к ним:
\vs p190 1:5 \pc «Люди и братья, все это время вы служили мне согласно клятве, которую вы дали мне и друг другу, и призываю вас засвидетельствовать, что я еще никогда не посылал вас с ложными сведениями. Я собираюсь направить вас с последней миссией как добровольных вестников царства и, делая это, освобождаю вас от ваших клятв и, таким образом, распускаю отряд вестников. Люди, я объявляю вам, что мы закончили нашу работу. Учитель более не нуждается в смертных вестниках; он воскрес из мертвых. До того, как его арестовали, он говорил нам, что умрет и на третий день воскреснет. Я видел гробницу --- она пуста. Я беседовал с Марией Магдалиной и четырьмя другими женщинами, которые разговаривали с Иисусом. Теперь же я вас распускаю, прощаюсь с вами и направляю вас исполнять, что каждому назначено; послание же, которое вы понесете верующим, таково: „Иисус воскрес из мертвых; гробница пуста“».
\vs p190 1:6 \pc Большинство из присутствовавших попытались убедить Давида не делать этого. Но не смогли повлиять на него. Тогда они попытались отговорить вестников, но те не внимали словам сомневающихся. И так незадолго до десяти часов в это воскресное утро двадцать шесть гонцов отправились в путь как первые вестники великого свершившегося события --- воскресения Иисуса. Отправились же с этой миссией они так же, как делали это много раз прежде, исполняя клятву, данную Давиду Зеведееву и друг другу. Эти люди в высшей степени доверяли Давиду. И отправились с этим поручением, даже не задержавшись, чтобы поговорить с теми, кто видел Иисуса, потому что верили Давиду на слово. Большинство из них верили тому, что рассказал Давид, и даже те, кто в чем\hyp{}то и сомневался, так же быстро понесли послание.
\vs p190 1:7 \pc Апостолы, духовный отряд царства, в этот день собрались в комнате наверху, где и сидели в страхе и сомнениях, в то время как эти миряне, являя собой первую попытку распространения евангелия Учителя о братстве людей, по приказу своего бесстрашного и деятельного главы идут возвещать воскресение Спасителя мира и вселенной. И занимаются этим важным служением еще до того, как его избранные представители готовы поверить его слову и принять свидетельства очевидцев.
\vs p190 1:8 \pc Эти двадцать шесть вестников были посланы в дом Лазаря в Вифании и во все центры верующих от Вирсавии на юге до Дамаска и Сидона на севере; от Филадельфии на востоке до Александрии на западе.
\vs p190 1:9 Простившись со своими собратьями, Давид пошел за своей матерью в дом Иосифа, после чего они оправились в Вифанию, чтобы присоединиться к ожидающему их семейству Иисуса. Там, в Вифании Давид жил у Марфы и Марии, пока те не продали свои земные владения, и сопровождал их в путешествии в Филадельфию, предпринятом, дабы присоединиться к их брату Лазарю.
\vs p190 1:10 Спустя приблизительно неделю после этих событий Иоанн Зеведеев забрал мать Иисуса в свой дом в Вифсаиде. Иаков же, старший брат Иисуса, остался с его семьей в Иерусалиме. Руфь жила в Вифании в семье Лазаря. Остальные члены семейства Иисуса вернулись в Галилею. В начале июня на следующий день после женитьбы на Руфи, самой младшей сестре Иисуса, Давид Зеведеев оставил Вифанию и вместе с Марфой и Марией отправился в Филадельфию.
\usection{2. Явление Иисуса в Вифании}
\vs p190 2:1 С момента моронтийного воскресения и до часа своего духовного вознесения на небо Иисус девятнадцать раз в зримом облике являлся верующим в него на земле. Он не являлся ни своим врагам, ни тем, кто не мог извлечь духовной пользы из его явления в зримом облике. Его первое явление было пяти женщинам возле гробницы; второе --- Марии Магдалине и тоже возле гробницы.
\vs p190 2:2 Третье явление произошло в это воскресенье около полудня в Вифании. Вскоре после полудня самый старший из братьев Иисуса Иаков стоял в саду Лазаря перед пустой гробницей воскрешенного брата Марфы и Марии, обдумывая известия, которые около часа назад принес им вестник Давида. Иаков был всегда склонен верить в миссию своего старшего брата на земле, но уже давно утратил связи с делом Иисуса и стал глубоко сомневаться в более поздних утверждениях апостолов о том, что Иисус --- Мессия. Новости, принесенные вестником, испугали и поразили все семейство. Пока же Иаков стоял перед пустой гробницей Лазаря, к месту событий пришла Мария Магдалина и стала взволнованно рассказывать родным Иисуса о том, что случилось с ней в ранние утренние часы у гробницы Иосифа. Не успела она закончить свой рассказ, пришел Давид Зеведеев со своей матерью. Руфь конечно же поверила рассказу; переговорив с Давидом и Саломеей, поверил и Иуда.
\vs p190 2:3 Между тем, пока они искали Иакова, он, прежде, чем его нашли, встал в саду рядом с гробницей и почувствовал чье\hyp{}то близкое присутствие, словно кто\hyp{}то коснулся его плеча; оглянувшись посмотреть, кто это, он увидел, что перед ним постепенно стала вырисовываться странная фигура. Иаков был чрезмерно изумлен, чтобы говорить, и слишком напуган, чтобы бежать. Тогда незнакомец, заговорив, произнес: «Иаков, я пришел призвать тебя к служению царства. Вместе со своими собратьями крепко возьмитесь за руки и следуйте за мной». Услышав свое имя, Иаков понял, что к нему обращается его старший брат Иисус. Все они с большим или меньшим трудом узнавали Учителя в моронтийном облике, но, когда Иисус начинал общаться с ними, мало кому из них было сложно узнать его голос или каким\hyp{}либо иным образом определить его очаровательную личность.
\vs p190 2:4 Когда Иаков понял, что к нему обращается Иисус, то стал опускаться на колени, восклицая: «Отец мой и брат мой», но Иисус велел ему встать и заговорил с ним. И они почти три минуты гуляли по саду и беседовали; беседовали о пережитом в прежние дни и предвозвещали события ближайшего будущего. Когда же они приблизились к дому, Иисус сказал: «Прощай, Иаков, прощай до времени, когда я буду приветствовать вас всех вместе».
\vs p190 2:5 Иаков вбежал в дом, когда его еще искали в Беф\hyp{}Фаге, и воскликнул: «Я только что видел Иисуса и говорил с ним, я с ним общался. Он не умер; он воскрес! Он исчез пред моими глазами, говоря: „Прощай до времени, когда я буду приветствовать вас всех вместе“». Он еще не кончил говорить, как вернулся Иуда, и Иаков специально для него снова рассказал о том, как встретился с Иисусом в саду. И все они поверили в воскресение Иисуса. Тогда Иаков объявил, что в Галилею не вернется, и Давид воскликнул: «Его видят не только взволнованные женщины; даже мужчины сильные духом, и те начали видеть его. И я тоже надеюсь увидеть его сам».
\vs p190 2:6 \pc И Давиду не пришлось долго ждать, ибо в четвертый раз Иисус зримо явился своим земным родным и их друзьям около двух часов в том же доме Марфы и Марии. В этот раз его увидели двадцать человек. Учитель появился в открытой задней двери и сказал: «Мир вам. Приветствую всех, кто некогда был родствен мне по плоти, и сердечный привет братьям моим и сестрам по братству в царстве небесном. Как могли вы сомневаться? Почему так долго медлили, не решаясь всем сердцем последовать за светом истины? Посему войдите все в братство Духа Истины в царстве Отца». Когда же они немного пришли в себя от первого потрясения и изумления и сделали движение, словно намереваясь обнять его, он стал невидим для них.
\vs p190 2:7 \pc Все они хотели бежать в город, чтобы рассказать сомневающимся апостолам о том, что случилось, но Иаков удержал их. Лишь Марии Магдалине было позволено вернуться в дом Иосифа. Иаков запретил всем рассказывать о факте этого моронтийного посещения из\hyp{}за неких слов, сказанных ему Иисусом во время беседы в саду. Но Иаков никогда больше ничего не рассказывал о своей встрече в этот день в доме Лазаря в Вифании с воскресшим Учителем.
\usection{3. В доме Иосифа}
\vs p190 3:1 Пятое моронтийное явление Иисуса, увиденное глазами смертных, произошло примерно в пятнадцать минут пятого после полудня в то же самое воскресенье в присутствии приблизительно двадцати пяти верующих женщин, которые собрались в доме Иосифа Аримафейского. Мария Магдалина вернулась в дом Иосифа всего за несколько минут до этого явления. Брат Иисуса Иаков заклинал ничего не рассказывать апостолам об явлении Учителя в Вифании. Однако он не просил Марию воздерживаться от рассказов о случившемся ее верующим сестрам. Поэтому, взяв со всех женщин обещание сохранить тайну, она поведала им о том, что произошло совсем недавно, пока она была с родными Иисуса в Вифании. Когда же она была уже на середине этого захватывающего повествования, внезапная тишина опустилась на них, и они рядом с собой узрели явственно видимую фигуру воскресшего Иисуса. Он поприветствовал их и сказал: «Мир вам. В братстве царства не будет ни евреев, ни неевреев; ни богатых, ни бедных; ни свободных, ни узников; ни мужчин, ни женщин. Вы тоже призваны распространять благую весть о свободе человечества, которую дает евангелие сыновства по отношению к Богу в царстве небесном. Идите и всему миру возвещайте сие евангелие и укрепляйте дух уверовавших в него. Делая же это, не забывайте служить больным и укреплять нерешительных и порабощенных страхом. Я же буду с вами всегда, даже до самого конца». И, сказав это, стал невидим; женщины же пали ниц и молча молились.
\vs p190 3:2 \pc Из пяти моронтийных явлений Иисуса, состоявшихся до этого времени, Мария Магдалина была свидетелем четырех.
\vs p190 3:3 \pc Из\hyp{}за того, что вестники отправились за несколько часов до полудня, и из\hyp{}за нечаянного разглашения некоторых сведений, касавшихся явления Иисуса в доме Иосифа, с наступлением вечера к правителям евреев стали поступать известия о том, что в городе ходят слухи, что Иисус воскрес, и многие утверждают, что его видели. Эти слухи чрезвычайно встревожили членов синедриона. После поспешного совещания с Анной Каиафа назначил собрание синедриона на восемь часов вечера. На этом собрании были приняты меры, направленные на изгнание из синагог любого, кто упоминал о воскресении Иисуса. Было предложено даже казнить всякого, кто будет утверждать, что видел его; впрочем, это предложение даже не дошло до голосования, поскольку все разошлись в смятении, скорее напоминающем настоящую панику. Они смели думать, что с Иисусом покончено. На самом деле им предстояло осознать, что в действительности их беды от Назорея только начались.
\usection{4. Явление грекам}
\vs p190 4:1 Около половины пятого в доме некого Флавия Учитель совершил свое шестое моронтийное явление приблизительно сорока собравшимся там верующим грекам. В то время, когда те обсуждали слухи о воскресении Учителя, он явил им себя, несмотря на то, что двери были надежно заперты, и, обращаясь к ним, сказал: «Мир вам! Хоть Сын Человеческий и явился на земле среди евреев, он пришел служить всем людям. В царстве Отца моего не будет ни евреев, ни неевреев; все вы будете братьями --- сынами Бога. Поэтому идите и всему миру возвещайте сие евангелие спасения, каким вы приняли его от посланцев царства, я же буду с вами в братстве Отчих сынов веры и истины». И, дав им сей наказ, покинул их, и они его больше не видели. Весь вечер они не выходили из дома; благоговение и страх обуяли их настолько, что они не решались покинуть его. Ни один из этих греков не спал ночью; они бодрствовали, обсуждая случившееся и надеясь, что Учитель, возможно, снова их посетит. Многие из этой группы были в Гефсимании, когда солдаты арестовали Иисуса и Иуда предал его своим поцелуем.
\vs p190 4:2 \pc Слухи о воскресении Иисуса и молва о его многочисленных явлениях своим последователям быстро распространяются, и приводят весь город в состояние крайнего возбуждения. Учитель уже явился своему семейству, женщинам и грекам и вскоре явит себя апостолам. Синедрион вскоре должен начать обсуждать новые проблемы, столь внезапно обрушившиеся на еврейских правителей. Иисус же много думает о своих апостолах, но желает, чтобы те прежде, чем он посетит их, несколько часов оставались в одиночестве, посвятив это время серьезным размышлениям и глубоким раздумьям.
\usection{5. Прогулка с двумя братьями}
\vs p190 5:1 В Эммаусе, расположенном милях в семи к северу от Иерусалима, жили два брата\hyp{}пастуха, которые провели пасхальную неделю в Иерусалиме, принося жертвы и участвуя в обрядах и празднествах. Старший из них, Клеопа, в какой\hyp{}то мере уверовал в Иисуса, за что был изгнан из синагоги. Брат же его Иаков верующим не был, хотя и очень интересовался тем, что слышал об учениях и трудах Учителя.
\vs p190 5:2 В это воскресение около пяти часов пополудни в трех милях от Иерусалима два брата устало брели по дороге на Эммаус, весьма увлеченно беседуя об Иисусе, его учениях, труде и особенно о слухах о том, что его гробница пуста и что есть женщины, которые говорили с ним. Клеопа, в основном, был готов поверить этим рассказам, но Иаков настаивал на том, что все это, возможно, обман. Пока же они по дороге к дому так спорили и рассуждали, Иисус в моронтийном состоянии явился им (это было его седьмое явление) и пошел рядом, в то время как они продолжали свой путь. Клеопа часто слушал, как учит Иисус, и несколько раз вместе с ним присутствовал на трапезах в домах иерусалимских верующих. Однако он не узнал Учителя даже тогда, когда тот прямо заговорил с ними.
\vs p190 5:3 Пройдя с ними небольшое расстояние, Иисус спросил: «О чем вы так горячо спорили, когда я подошел к вам?» Когда же Иисус сказал это, они остановились и с печальным удивлением посмотрели на него. Клеопа сказал: «Неужели ты, один из пришедших в Иерусалим, не знаешь о происшедшем в нем в эти дни?» Тогда Учитель спросил: «О чем?» Клеопа ответил: «Если ты не знаешь об этом, значит, ты единственный в Иерусалиме, кто не слышал слухов об Иисусе из Назарета, который был пророк, сильный в слове и деле пред Богом и всем народом. Первосвященники и наши правители предали его римлянам и потребовали, чтобы те распяли его. Вот, многие из нас надеялись, что он есть тот, кто избавит Израиль от нееврейского ига. Но это не все. Вот уже третий день, как он был распят, и сегодня некоторые женщины изумили нас, объявив, что рано утром пошли к его гробнице и нашли ее пустой. Эти же женщины настаивают на том, что они говорили с этим человеком, и утверждают, будто он воскрес из мертвых. Когда же они рассказали об этом мужчинам, двое из апостолов его побежали к гробнице и тоже нашли ее пустой\ldots » --- здесь Иаков перебил брата и сказал: «Но они не видели Иисуса».
\vs p190 5:4 Когда же они пошли дальше, Иисус сказал им: «Как же медлительны вы в понимании истины! Поскольку вы говорите мне, что рассуждаете об учениях и делах этого человека, то позвольте мне просветить вас, ибо я более чем знаком с этими учениями. Разве не помните вы, что этот Иисус всегда учил, что царство его не от мира сего и что все люди, будучи сынами Бога, должны найти свободу и волю в духовной радости, которую дает принадлежность к братству, основанному на служении любви в этом новом царстве истины о любви Отца Небесного? Разве не помните, как этот Сын Человеческий провозглашал спасение Божье всем людям, служа больным и страждущим и освобождая тех, кто был скован страхом и порабощен злом? Разве не знаете, что сей назарянин говорил своим ученикам, что он должен идти в Иерусалим, быть предан врагам своим, которые его убьют, и что на третий день он воскреснет? Разве не говорили вам все это? И разве не читали вы в Писании об этом дне спасения для еврея и нееврея, где сказано, что в нем все племена земли благословятся; что он услышит вопль нуждающихся и спасет души бедных, ищущих его; и что все народы назовут его блаженным? Что такой Избавитель будет как тень от высокой скалы в земле жаждущей. Что как пастырь истинный он упасет стадо свое, собрав агнцев на руках своих и нежно прижав их к груди своей? Что откроет глаза духовно слепым и выведет узников отчаяния в совершенную свободу и свет; что все, сидящие во тьме, увидят великий свет вечного спасения. Что он исцелит сокрушенных сердцем, возвестит освобождение узникам греха и откроет темницы тем, кто порабощен страхом и закабален злом. Что утешит скорбящих и вместо печали и духа тягостного дарует им радость спасения. Что он станет желаем всеми народами и будет вечной радостью тех, кто ищет праведности. Что этот Сын истины и правды восстанет над миром с исцеляющим светом и спасительной силой; что спасет даже людей своих от грехов их; что действительно будет искать и спасать заблудших. Что он не погубит слабых, но даст спасение всем, кто алчет и жаждет праведности. Что те, кто веруют в него, будут иметь жизнь вечную. Что он изольет дух свой на всякую плоть и что сей Дух Истины в каждом верующем будет колодезем воды, текущей в жизнь вечную. Разве не поняли вы, сколь велико евангелие царства, которое дал вам этот человек? Разве не видите, как велико спасение, посетившее вас?»
\vs p190 5:5 К этому времени приблизились они к селению, где жили те братья. С тех пор, как Иисус начал учить их, пока они шли по дороге, эти два человека не произнесли ни слова. Вскоре они подошли к своему скромному жилищу, и Иисус уже намеревался проститься с ними и идти дальше, но они удержали его, попросив войти в дом и побыть с ними. Они настаивали, что уже близится ночь и он должен остаться с ними. В конце концов Иисус согласился, и вскоре по приходу они сели за стол. Они дали Иисусу хлеб для благословения, и когда он, преломив, подал его им, глаза их открылись и Клеопа понял, что их гостем был сам Учитель. Когда же он сказал: «Се Учитель», Иисус в моронтийном состоянии стал невидим для них.
\vs p190 5:6 Тогда они сказали друг другу: «Неудивительно, что сердца наши горели в нас, когда он говорил с нами, пока мы шли по дороге, и когда изъяснял нам Писание!»
\vs p190 5:7 Есть они не стали. Они видели Учителя в моронтийном состоянии и, бросившись вон из дома, поспешили вернуться в Иерусалим, чтобы разнести благую весть о воскресении Спасителя.
\vs p190 5:8 В тот вечер около девяти часов и как раз перед тем, как Учитель явился десяти апостолам, оба взволнованных брата вбежали наверх в комнату к апостолам и объявили, что они видели Иисуса и говорили с ним. И рассказали обо всем, что сказал им Иисус, и как они до момента преломления хлеба не понимали, кто он.
