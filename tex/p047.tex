\upaper{47}{Семь миров\hyp{}обителей}
\author{Блестящая Вечерняя Звезда}
\vs p047 0:1 Когда Сын\hyp{}Творец был на Урантии, то говорил о «многих обителях во вселенной Отца». В определенном смысле все пятьдесят шесть миров, окружающих Иерусем, предназначены для переходной культуры восходящих смертных, однако семь спутников мира номер один более известны как миры\hyp{}обители.
\vs p047 0:2 Сам переходный мир номер один полностью посвящен исключительно деятельности, связанной со следованием по пути восхождения, и является центром отряда финалитов, приписанного к Сатании. Сейчас этот мир служит центром для более чем ста тысяч рот финалитов, причем в каждой из этих групп тысяча прославленных существ.
\vs p047 0:3 Когда система устанавливается в свете и жизни и по мере того, как миры\hyp{}обители один за другим перестают служить местом воспитания смертных, они переходят в руки неуклонно возрастающего населения финалитов, которое накапливается в этих более старых и более усовершенствованных системах.
\vs p047 0:4 Семь миров\hyp{}обителей находятся в ведении моронтийных руководителей и Мелхиседеков. В каждом мире есть действующий губернатор, который подчиняется непосредственно иерусемским правителям. Примирители Уверсы содержат центр в каждом из миров\hyp{}обителей, рядом с которым расположено место встреч Технических Советчиков в локальной вселенной. Руководители восстановления и небесные ремесленники содержат совместные центры в каждом из этих миров. Спиронги действуют, начиная с мира номер два и далее, тогда как все семь миров, вместе с другими планетами переходной культуры и миром\hyp{}центром, щедро наделены спорнагиями стандартного творения.
\usection{1. Мир финалитов}
\vs p047 1:1 Хотя в переходном мире номер один постоянно находятся лишь финалиты и отдельные группы спасенных детей и их опекунов, тем не менее, там предусмотрены меры для размещения всех классов духовных существ, переходных смертных и приезжих учащихся. Спорнагии, действующие во всех этих мирах, --- гостеприимные хозяева для всех существ, которых они могут распознать. Они смутно чувствуют финалитов, но представить их себе не могут. Они должны относиться к ним во многом так же, как и вы в вашем теперешнем физическом состоянии относитесь к ангелам.
\vs p047 1:2 Хотя мир финалитов --- это сфера изысканной физической красоты и необычайных моронтийных красот, тем не менее, великая духовная обитель, расположенная в центре действий, --- храм финалитов для обычного материального или начального моронтийного зрения невидим. Но многие из этих реальностей преобразователи энергии могут делать видимыми для восходящих смертных, и время от времени, например, по случаю классных собраний учеников мира\hyp{}обители в этой культурной сфере, именно так и действуют.
\vs p047 1:3 На всем протяжении пребывания в мирах\hyp{}обителях ты некоторым образом духовно сознаешь присутствие твоих прославленных братьев, достигших Рая, но когда время от времени действительно видишь, как они трудятся в своих центральных обителях, это придает новые силы. До тех пор, пока ты не обретешь истинное духовное зрение, видеть финалитов непосредственно ты не будешь.
\vs p047 1:4 \pc В первом мире\hyp{}обители все продолжающие существование в посмертии должны удовлетворять требованиям родительской комиссии той планеты, где они родились. Теперешняя комиссия Урантии состоит из недавно прибывших двенадцати пар родителей, во плоти имевших опыт воспитания трех и более детей до достижения ими возраста половой зрелости. Служба в этой комиссии ротационная и, как правило, продолжается всего десять лет. Все, кто не может удовлетворить требованиям членов этой комиссии в плане своего родительского опыта, должны пройти дальнейшую подготовку, служа в домах Материальных Сынов на Иерусеме, либо, пройти часть этой подготовки в испытательных яслях в мире финалитов.
\vs p047 1:5 Однако независимо от родительского опыта, родителям, пребывающим в мирах\hyp{}обителях и имеющим подрастающих детей в испытательных яслях, предоставлены все возможности сотрудничать с моронтийными хранителями этих детей в вопросах их обучения и воспитания. Этим родителям позволяется наносить туда визиты до четырех раз в год. Одну из самых трогательно\hyp{}прекрасных картин на всем пути восхождения можно наблюдать в момент, когда родители из миров\hyp{}обителей заключают в объятия своих материальных потомков во время периодических путешествий в мир финалитов. Хотя один или оба родителя могут покидать мир\hyp{}обитель раньше своего ребенка, довольно часто они проводят часть времени вместе.
\vs p047 1:6 Ни один из восходящих смертных не может избежать опыта воспитания детей --- своих собственных или чужих --- либо в материальных мирах, либо, впоследствии, в мире финалитов или на Иерусеме. Отцы должны пройти через этот важнейший опыт так же обязательно, как и матери. Представление современных народов Урантии, будто воспитание ребенка --- это в основном дело матери, ошибочно и достойно сожаления. Детям нужны отцы так же, как матери, и так же, как и матери, отцу необходим этот родительский опыт.
\usection{2. Испытательные ясли}
\vs p047 2:1 Школы Сатании, принимающие детей, расположены в мире финалитов, в первой из сфер переходной культуры Иерусема. Эти принимающие детей школы предназначены для воспитания и подготовки детей, живущих во времени, в том числе и тех, кто умер в эволюционных пространственных мирах до обретения во вселенских документах статуса индивидуума. В случае продолжения существования в посмертии одного или обоих родителей такого ребенка, хранительница предназначения назначает связанного с ней херувима хранителем потенциальной идентичности ребенка, обязав херувима доставить эту неразвившуюся душу в руки Учителей Мира\hyp{}Обители в испытательных яслях моронтийных миров.
\vs p047 2:2 Эти же оставленные херувимы, в качестве Учителей Мира\hyp{}Обители под Руководством Мелхиседеков содержат огромные учебные заведения для воспитания подопечных финалитов. Эти подопечные финалитов, эти дети восходящих смертных, всегда персонализируются с обретением своего точного физического состояния в момент смерти, за исключением лишь репродуктивных возможностей. Это пробуждение происходит строго во время прибытия родителей в первый мир\hyp{}обитель. Причем в дальнейшем этим детям, таким, каковы они есть, предоставляются все возможности для избрания небесного пути так же, как если бы они делали подобный выбор в мирах, где смерть столь безвременно оборвала их жизненный путь.
\vs p047 2:3 В мире\hyp{}яслях испытуемые творения объединяют в группы в соответствии с тем, имеют ли они Настройщиков или нет, ибо Настройщики приходят, чтобы пребывать в этих материальных детях так же, как и во временных мирах. О детях, не достигших возраста, когда в них поселяется Настройщик, заботятся семьи, имеющие пять детей в возрасте от одного года (и меньше) до приблизительно пятилетнего возраста либо до момента, когда в детях поселяется Настройщик.
\vs p047 2:4 В развивающихся мирах все дети, имеющие Настройщиков Мысли, но до смерти не сделавшие выбор Райского пути, тоже реперсонализируются в мире финалитов системы, где они точно так же растут в семьях Материальных Сынов и их сподвижников, как растут и те малыши, что прибыли без Настройщиков, но затем, по достижении возраста, необходимого для принятия морального выбора, получают Таинственных Помощников.
\vs p047 2:5 В мире финалитов дети и молодые люди, в которых пребывает Настройщик, также воспитываются в семьях с пятью детьми в возрасте от шести до четырнадцати лет. Эти семьи состоят из детей в возрасте приблизительно шесть, восемь, десять, двенадцать и четырнадцать лет. После шестнадцати лет, если окончательный выбор сделан, молодые люди переносятся в первый мир\hyp{}обитель и начинают свое восхождение к Раю. Некоторые из детей делают выбор до вступления в этот возраст и отправляются в миры восхождения, однако в мирах\hyp{}обителях очень редко встречаются дети моложе шестнадцати (если судить по урантийским нормам) лет.
\vs p047 2:6 В испытательных яслях мира финалитов этим молодым людям служат серафимы\hyp{}хранительницы так же, как они духовно служат смертным на эволюционных планетах; физические же потребности молодых людей удовлетворяют верные спорнагии. И так эти дети растут в переходном мире до тех пор, пока не сделают свой окончательный выбор.
\vs p047 2:7 Когда материальная жизнь истекла, то, если выбор в пользу жизни восхождения не сделан или если эти дети, живущие во времени, определенно решают отказаться от восхождения в Хавоне, смерть автоматически обрывает их испытательные пути. Судебных решений по таким делам не бывает; воскрешения от такой второй смерти нет. Они просто исчезают, как если бы их вообще никогда не было.
\vs p047 2:8 Если же дети выбирают Райский путь совершенства, то их немедленно готовят к перемещению в первый мир\hyp{}обитель, куда многие из них успевают прибыть вовремя, чтобы присоединиться к своим родителям в восхождении в Хавоне. После прохождения через Хавону и достижения Божеств, эти спасенные души смертного происхождения входят в число восходящих постоянных граждан Рая. Детей, которые были лишены ценного и необходимого опыта эволюции в мирах, где рождаются смертные, в Отряды Финалитов не зачисляют.
\usection{3. Первый мир\hyp{}обитель}
\vs p047 3:1 В мирах\hyp{}обителях воскрешенные смертные продолжают свои жизни с того же уровня, на котором их застигла смерть. Уходя с Урантии в первый мир\hyp{}обитель, ты заметишь значительную перемену, но если бы ты пришел из более устойчивой и развитой временной сферы, то едва ли заметил бы разницу, за исключением того, что ты теперь обладаешь другим телом; вместилище плоти и крови осталось в мире, где ты родился.
\vs p047 3:2 Самым центром всей деятельности первого мира\hyp{}обители является зал воскрешения --- огромный храм созидания личности. Эта гигантская структура состоит из центрального места встреч серафимов\hyp{}хранительниц предназначения, Настройщиков Мысли и архангелов воскресения. При воскрешении умерших вместе с этими небесными существами действуют и Носители Жизни.
\vs p047 3:3 Копии разума смертных и паттерны активной памяти творения, трансформированные из материальных уровней в духовные, являются индивидуальной собственностью отделившихся Настройщиков Мысли; эти одухотворенные факторы разума, памяти и личности творения навсегда остаются частью таких Настройщиков. Матрица разума творения и пассивные потенциалы идентичности присутствуют в моронтийной душе, доверенной серафимам\hyp{}хранительницам предназначения на хранение. Именно воссоединение доверенной серафимам моронтийной души с духовным разумом, доверенным Настройщику, и воссоздает личность создания, являясь воскрешением спящего в посмертии.
\vs p047 3:4 Если переходная личность смертного происхождения не будет воссоздана таким образом, то духовные элементы не продолжившего существование в посмертии смертного творения навсегда останутся составной частью индивидуального опытного дарования Настройщика, когда\hyp{}то пребывавшего в человеке.
\vs p047 3:5 Из Храма Новой Жизни простираются семь радиальных крыльев --- залов воскрешения смертных рас. Каждая из этих структур посвящена воссозданию одной из семи рас, живущих во времени. В каждом из семи крыльев есть сто тысяч личных комнат воскрешения, заканчивающихся круглыми залами классных собраний, которые служат комнатами пробуждения почти миллиона индивидуумов. Эти залы окружены комнатами воссоздания личностей смешанных рас устойчивых постадамических миров. Независимо от метода, который может использоваться в отдельных мирах времени в связи с особыми или диспенсационными воскрешениями, реальное и сознательное воссоздание действительной и полноценной личности происходит в залах воскрешения обители номер один. На протяжении всей вечности ты будешь вспоминать глубокие, оставшиеся в памяти впечатления от твоего первого наблюдения этих начальных моментов воскрешения.
\vs p047 3:6 \pc Из залов воскрешения ты проследуешь в сектор Мелхиседеков, где тебе будет предоставлено постоянное местожительство. Затем для тебя наступят десять дней личной свободы. И ты будешь волен исследовать ближайшие окрестности твоего нового дома и знакомиться с программой, которая тебя ждет впереди. У тебя будет время и удовлетворить желание навести справки в журнале записей, и навестить твоих родственников и других земных друзей, которые, возможно, прибыли в эти миры раньше тебя. В конце десятидневного периода досуга ты приступишь ко второму шагу в твоем путешествии к Раю, ибо миры\hyp{}обители --- это настоящие сферы подготовки, а не просто планеты остановки.
\vs p047 3:7 \pc В мире\hyp{}обители номер один (или в другом мире в случае более высокого статуса) ты возобновишь твою интеллектуальную подготовку и духовное развитие точно с того же уровня, на котором их оборвала смерть. В период времени между планетарной смертью или перенесением и воскрешением в мире\hyp{}обители смертный человек, помимо переживания факта продолжения существования в посмертии, не приобретает абсолютно ничего. Ты начинаешь там точно в том же месте, где закончил здесь.
\vs p047 3:8 Почти весь опыт пребывания в мире\hyp{}обители номер один связан со служением устранения недостатков. Продолжающим существование в посмертии смертным, прибывающим на эту первую из сфер остановки, свойственно такое количество различных изъянов в характере и недостатков в опыте, что основная деятельность мира связана с исправлением и лечением этих многочисленных последствий жизни во плоти в материально\hyp{}эволюционных мирах пространства и времени.
\vs p047 3:9 Пребывание в мире\hyp{}обители номер один служит для развития продолжающих существование в посмертии смертных, по крайней мере, до статуса постадамической диспенсации в нормальных эволюционных мирах. Духовно же ученики мира\hyp{}обители, разумеется, намного опережают подобное состояние простого человеческого развития.
\vs p047 3:10 Если тебя не задерживают в мире\hyp{}обители номер один, то, когда десять дней подойдут к концу, ты войдешь в состояние сна перенесения и перейдешь в мир номер два, после чего каждые десять дней будешь продвигаться вперед до тех пор, пока не прибудешь в мир своего назначения.
\vs p047 3:11 \pc Центр семи основных кругов администрации первого мира\hyp{}обители занимает храм Моронтийных Компаньонов --- личных проводников, приписанных к восходящим смертным. Эти компаньоны являются потомками Духа\hyp{}Матери локальной вселенной, и в моронтийных мирах Сатании их насчитывается несколько миллионов. Помимо тех, кто получил назначение в качестве групповых компаньонов, тебе придется много общаться также с интерпретаторами и переводчиками, хранителями строений и руководителями экскурсий. Причем все эти компаньоны в высшей степени готовы к сотрудничеству с теми, кто занимается развитием твоих личностных факторов разума и духа в моронтийном теле.
\vs p047 3:12 В начале жизни в первом мире\hyp{}обители к каждому отряду из тысячи смертных, идущих по пути восхождения, приписывается один Моронтийный Компаньон, но проходя через семь сфер\hyp{}обителей, ты встретишь их гораздо больше. Эти прекрасные и разносторонние существа --- общительные товарищи и очаровательные гиды. Они вольны сопровождать индивидуумов или избранные группы в любую из сфер переходной культуры, в том числе и в их миры\hyp{}спутники. Для всех идущих по пути восхождения смертных они --- экскурсоводы и товарищи по досугу. Они часто сопровождают группы продолжающих существование во время периодических посещений Иерусема, и ты в любой день, находясь там, можешь отправиться в регистрационный сектор столицы системы и встретиться с восходящими смертными из всех семи миров\hyp{}обителей, так как они свободно путешествуют туда и обратно между местами их постоянного пребывания и центром системы.
\usection{4. Второй мир\hyp{}обитель}
\vs p047 4:1 Именно на этой сфере ты еще подробнее знакомишься с жизнью в обители. Распределения по слоям моронтийной жизни становятся более отчетливыми; рабочие группы и общественные организации начинают действовать, общины приобретают надлежащие пропорции и продвигающиеся смертные вводят новые общественные порядки и исполняют правительственные решения.
\vs p047 4:2 Продолжающие существование в посмертии существа, слившиеся с Духом, живут в мирах\hyp{}обителях вместе с идущими по пути восхождения смертными, слившимися с Настройщиками. Хотя различные чины небесной жизни отличаются друг от друга, тем не менее, все они дружественные и братские. Во всех мирах восхождения ты не найдешь ничего, сравнимого с человеческой нетерпимостью и дискриминацией замкнутых кастовых систем.
\vs p047 4:3 По мере твоего восхождения от одного мира\hyp{}обители к другому, они все более наполняются моронтийной деятельностью продвигающихся смертных. Идя вперед, ты будешь замечать все больше и больше иерусемских особенностей, сообщенных мирам\hyp{}обителям. Стеклянная гладь появляется на втором мире\hyp{}обители.
\vs p047 4:4 Новое развившееся и должным образом приспособленное моронтийное тело обретается во время каждого перехода из одного мира\hyp{}обители в другой. Ты ложишься спать у серафимов перемещения и просыпаешься с новым, но неразвитым телом в залах воскрешения --- почти так же, как тогда, когда ты впервые прибыл в мир\hyp{}обитель номер один, с той лишь разницей, что во время этих снов при перемещении между мирами\hyp{}обителями Настройщик не покидает тебя. После перехода из эволюционных миров в начальный мир\hyp{}обитель твоя личность остается цельной.
\vs p047 4:5 При восхождении по пути моронтийной жизни память твоего Настройщика остается совершенно неповрежденной. Чисто животные и низменные материальные умственные ассоциации естественным образом разрушаются вместе с физическими мозгом, однако все то в твоей умственной жизни, что было достойным и представляло собой ценность, дающую продолжение существования в посмертии, копируется Настройщиком и удерживается им как часть личной памяти на всем протяжении пути восхождения. Переходя из одного мира\hyp{}обители в другой и из одной части вселенной в другую, вплоть до самого Рая, ты будешь помнить все твои достойные переживания.
\vs p047 4:6 Хотя у тебя будут моронтийные тела, во всех семи мирах\hyp{}обителях ты будешь продолжать есть, пить и отдыхать. Ты будешь вкушать пищу моронтийного чина, царство живой энергии, неизвестную в материальных мирах. Как пища, так и вода, моронтийным телом используются полностью; остаточных отходов нет. Остановись и задумайся: обитель номер один --- вполне материальная сфера, представляющая лишь первые начатки моронтийной системы. Ты еще почти человек и недалеко ушел от ограниченных точек зрения жизни смертного, но каждый мир представляет собой явный новый прогресс. От сферы к сфере ты становишься все менее материальным, все более интеллектуальным и чуть более духовным. Самое значительное духовное развитие происходит в последних трех из этих семи следующих друг за другом мирах.
\vs p047 4:7 Биологические недостатки во многом были устранены в первом мире\hyp{}обители. Там же были скорректированы (или внесены в план для устранения в будущем в семьях Материальных Сынов на Иерусеме) дефекты в планетарных переживаниях, связанных с половой жизнью, семейными отношениями и обязанностями родителей.
\vs p047 4:8 Обитель номер два в большей степени обеспечивает устранение всех интеллектуальных недостатков и лечение всех разновидностей душевной дисгармонии. Попытка осознать значение моронтийной моты, начатая в первом мире\hyp{}обители, здесь продолжается более основательно. Развитие на обители номер два сравнимо с интеллектуальным состоянием культуры, следующей за культурой Сынов\hyp{}Повелителей идеальных эволюционных миров.
\usection{5. Третий мир\hyp{}обитель}
\vs p047 5:1 Третья обитель является центром Учителей Миров\hyp{}Обителей. Хотя действуют они на всех семи сферах\hyp{}обителях, свой общий центр они содержат в центре школьных кругов мира номер три. В обителях и в высших моронтийных мирах этих наставников миллионы. Эти продвинутые и прославленные херувимы служат в качестве моронтийных учителей на всем пути от миров\hyp{}обителей до последней сферы воспитания восходящих локальной вселенной. Они же будут среди последних, кто сердечно попрощается с тобой, когда приблизится время расставания, время, когда ты скажешь до свидания --- по крайней мере на несколько эпох --- вселенной, в которой ты родился, время, когда серафимы примут тебя в свои объятья, чтобы перенести в принимающие миры малого сектора сверхвселенной.
\vs p047 5:2 Во время твоего пребывания в первом мире\hyp{}обители у тебя есть разрешение посещать первый из переходных миров, центр финалитов и испытательные ясли системы для воспитания неразвившихся эволюционных детей. Прибывая в обитель номер два, ты получаешь разрешение периодически посещать переходный мир номер два, где расположен центр моронтийных руководителей всей Сатании и подготовительные школы для различных моронтийных чинов. Когда ты достигнешь мира\hyp{}обители номер три, тебе немедленно дадут разрешение посещать третью переходную сферу --- центр ангельских чинов и дом их различных подготовительных школ системы. Визиты в Иерусем из этого мира становятся все более полезными и представляют для развивающихся смертных все больший интерес.
\vs p047 5:3 Третья обитель является миром великих личных и общественных достижений для всех, кто не постиг этих кругов культуры до освобождения из плоти в мирах, где рождаются смертные. На этой сфере начинается более позитивная учебная работа. Воспитание в первых двух мирах\hyp{}обителях в основном направлено на исправление недостатков --- она является негативной --- ибо связана с восполнением опыта жизни во плоти. В этом третьем мире\hyp{}обители продолжающие существовать в посмертии действительно начинают развивать свою прогрессивную моронтийную культуру. Главной целью этой подготовки является углубление понимания взаимосвязи моронтийной моты и логики смертного, согласование моронтийной моты и человеческой философии. Смертные, продолжающие существование, теперь обретают практическое понимание истинной метафизики. А это и есть реальное знакомство с разумным пониманием космических значений и вселенских взаимоотношений. Культура третьего мира\hyp{}обители имеет черты, присущие природе периода, наступающего на устойчивой обитаемой планете после пришествия Сына.
\usection{6. Четвертый мир\hyp{}обитель}
\vs p047 6:1 Когда ты прибываешь в четвертый мир\hyp{}обитель, ты уже прочно вступил в моронтийную жизнь и далеко ушел от исходного материального бытия. Теперь тебе дано разрешение совершать визиты в переходный мир номер четыре, чтобы познакомиться с центром и подготовительными школами сверхангелов, в том числе и с Блестящими Вечерними Звездами. Во время периодических посещений Иерусема, благодаря благому служению этих сверхангелов четвертого переходного мира, моронтийные посетители получают возможность вплотную приблизиться к различным чинам Сынов Бога, ибо новые секторы столицы системы постепенно открываются для идущих вперед смертных по мере того, как они наносят эти повторяющиеся визиты в мир\hyp{}центр. Новое величие все больше открывается развивающимся умам этих восходящих.
\vs p047 6:2 В четвертой обители восходящий индивидуум еще более подобающим образом находит свое место в деятельности группы и в классовых функциях моронтийной жизни. Здесь восходящие развивают в себе более высокую оценку возвещений и иных аспектов культуры и прогресса локальной вселенной.
\vs p047 6:3 Во время этого периода подготовки в мире номер четыре восходящие смертные на самом деле впервые знакомятся с потребностями действительно общественной жизни моронтийных созданий и наслаждаются ею. Причем участие в общественной деятельности, направленной не на личное возвышение и не на своекорыстные интересы, --- действительно новый опыт для эволюционных созданий. Здесь их ожидает новый общественный порядок --- порядок, основанный на понимающем сочувствии, взаимной признательности, бескорыстной любви взаимного служения и всеобъемлющей мотивации осознания общего и верховного предназначения --- Райской цели богопочитания и божественного совершенства. Все идущие по пути восхождения становятся сознающими Богопознание, Богооткровение, Богоискательство и Богообретение.
\vs p047 6:4 Интеллектуально социальную культуру этого четвертого мира\hyp{}обители можно сравнить с интеллектуальной и общественной жизнью эпохи, наступающей на нормально эволюционирующих планетах вслед за эпохой Сына\hyp{}Учителя. Духовный же статус на четвертом мире\hyp{}обители намного опережает такую смертную диспенсацию.
\usection{7. Пятый мир\hyp{}обитель}
\vs p047 7:1 Перемещение в пятый мир\hyp{}обитель представляет собой огромный шаг вперед в жизни моронтийного прогрессора. Опыт, приобретаемый в этом мире, настоящее предвкушение жизни на Иерусеме. Здесь ты начинаешь сознавать высокое предназначение верных эволюционных миров, так как они могут нормально развиваться до этой стадии в процессе своей естественной планетарной эволюции. Культура этого мира\hyp{}обители в общем соответствует культуре ранней эры света и жизни на планетах нормального эволюционного развития. И это позволяет тебе понять, почему так устроено, что высоко культурные и развитые типы существ, иногда населяющие эти продвинутые эволюционные миры, освобождены от прохождения через один, два или даже через все миры\hyp{}обители.
\vs p047 7:2 Овладев языком локальной вселенной перед тем, как покинуть четвертый мир\hyp{}обитель, ты теперь больше времени посвящаешь совершенствованию языка Уверсы с тем, чтобы еще до прибытия на Иерусем со статусом постоянного жителя в совершенстве владеть обоими языками. От центра системы до Хавоны все восходящие смертные двуязычные. Далее необходимо только пополнять сверхвселенский словарь, однако для постоянного жительства в Раю следует его еще дополнительно расширить.
\vs p047 7:3 По прибытии в обитель номер пять путешественник получает разрешение посещать переходный мир с соответствующим номером --- центр Сынов. Здесь идущий по пути восхождения смертный лично знакомится с различными группами божественного сыновства. Он уже слышал об этих благородных существах и уже встречался с ними на Иерусеме, но теперь начинает узнавать их по\hyp{}настоящему.
\vs p047 7:4 В пятой обители ты начинаешь знакомиться с учебными мирами созвездия. Здесь ты встречаешься с первыми воспитателями, которые начинают готовить тебя к последующему пребыванию в созвездии. Большая часть этой подготовки продолжается в мирах номер шесть и семь, а завершающие штрихи ей придаются на Иерусеме в секторе восходящих смертных.
\vs p047 7:5 В обители номер пять происходит реальное рождение космического сознания. Ты становишься существом, постигающим перспективы вселенной. Это --- поистине время раздвигающихся горизонтов. Развивающимся умам восходящих смертных начинает открываться, что некое колоссально\hyp{}величественное, некое небесно\hyp{}божественное предназначение ожидает всех, завершивших прогрессивное восхождение к Раю, которое началось так трудно, но одновременно и так радостно, и так благоприятно. Приблизительно в этой срединной точке восходящий смертный начинает проявлять основанный на пережитом опыте подлинный энтузиазм по отношению к восхождению к Хавоне. Учеба становится добровольной, бескорыстное служение --- естественным, а богопочитание --- самопроизвольным. Настоящий моронтийный характер расцветает, а настоящее моронтийное творение развивается.
\usection{8. Шестой мир\hyp{}обитель}
\vs p047 8:1 Прибывающим на эту сферу позволено посещать переходный мир номер шесть, где они еще больше узнают о высоких духах сверхвселенной, хотя многих из этих небесных существ они и не способны зримо представить. Кроме того, здесь они получат первые уроки, касающиеся будущей духовной жизни, которая начинается сразу после моронтийной подготовки в локальной вселенной.
\vs p047 8:2 Этот мир часто посещает помощник Владыки Системы, и здесь начинается первоначальная подготовка в методах управления вселенной. Теперь тебе преподают первые уроки, затрагивающие дела всей вселенной.
\vs p047 8:3 \pc Для восходящих смертных это --- блестящая эпоха, во время которой, как правило, происходит полное слияние человеческого разума с божественным Настройщиком. Потенциально это слияние могло бы произойти и раньше, однако подлинно действующая идентичность не часто достигается до момента пребывания в пятом или даже в шестом мире\hyp{}обители.
\vs p047 8:4 \pc О союзе развивающейся бессмертной души с вечным и божественным Настройщиком свидетельствует то, что серафим призывает руководящего сверхангела для воскресших смертных и архангела, назначенного идущим на суд третьего дня; затем в присутствии моронтийных товарищей такого продолжающего существовать в посмертии эти вестники подтверждения говорят: «Сей есть возлюбленный сын, в котором мое благоволение». Эта простая церемония отмечает вступление восходящего смертного на вечный путь Райского служения.
\vs p047 8:5 Сразу после подтверждения слияния с Настройщиком новое моронтийное существо впервые представляют его собратьям, называя его новое имя, и даруют ему сорок дней духовного отдыха от всех текущих дел, в течение которых оно наедине с собой будет выбирать один из возможных путей в Хавону и один из многих способов достижения Рая.
\vs p047 8:6 \pc Однако эти блестящие существа все еще относительно материальны и далеки от того, чтобы быть истинными духами; с духовной точки зрения, они, скорее, похожи на сверхсмертных и пока еще несколько ниже ангелов. Но они воистину становятся чудесными существами.
\vs p047 8:7 Во время пребывания в мире номер шесть ученики мира\hyp{}обители достигают статуса, сопоставимого с высоким развитием, характеризующим те эволюционные миры, которые в своем нормальном развитии вышли за пределы исходной стадии света и жизни. Организация общества в этой обители очень хорошо упорядочена. По мере восхождения из одного из этих миров в другой тень смертного творения становится все меньше и меньше. Оставляя грубые следы планетарного животного происхождения позади, вы становитесь заслуживающими все большего и большего восхищения. «Прохождение через великие страдания» делает прославленных смертных очень добрыми и понимающими, преисполненными высокого сострадания и терпимыми.
\usection{9. Седьмой мир\hyp{}обитель}
\vs p047 9:1 Опыт, приобретаемый в этом мире, является вершиной достижений начального посмертного пути. Во время пребывания здесь ты получишь наставления множества учителей, все они будут участвовать в подготовке тебя к постоянному жительству на Иерусеме. Во время пребывания в седьмом мире\hyp{}обители любые заметные различия между смертными, прибывающими из изолированных и отсталых миров, и продолжающими существование в посмертии из более развитых и просвещенных сфер фактически стираются. Здесь тебя очистят от всех пережитков дурной наследственности, сомнительного окружения и недуховных планетарных наклонностей. Последние следы «печати зверя» здесь окончательно стираются.
\vs p047 9:2 Во время пребывания в обители номер семь дается разрешение посещать переходный мир номер семь, мир Отца Всего Сущего. Здесь ты начинаешь новое и более духовное богопочитание невидимого Отца, которому будешь иметь обыкновение следовать все больше и больше на всем протяжении твоего долгого пути восхождения. В этом мире переходной культуры ты найдешь храм Отца, но самого Отца не увидишь.
\vs p047 9:3 \pc Теперь начинается формирование классов для завершения образования на мирах\hyp{}обителях и отправления в Иерусем. Раньше из мира в мир вы переходили как индивидуумы; теперь же вы готовитесь отправиться на Иерусем группами, хотя восходящий, при определенных условиях, может решить и задержаться в седьмом мире\hyp{}обители, чтобы дать возможность отставшему члену его земной рабочей группы или рабочей группы обители отправиться вместе с ним.
\vs p047 9:4 Наблюдать за вашим отправлением на Иерусем со статусом постоянного жителя штат седьмой обители собирается на стеклянной глади. Возможно, вы уже посещали Иерусем сотни или даже тысячи раз, но при этом всегда были гостем; раньше вы никогда не доходили до столицы системы вместе с группой ваших собратьев, которые, как восходящие смертные, навсегда и окончательно распростились со своей стезей в обителях. Вскоре вас будут приветствовать на принимающем поле мира\hyp{}центра как иерусемских граждан.
\vs p047 9:5 \pc От вашего продвижения через семь миров дематериализации вы будете получать огромное наслаждение; эти миры действительно делают бессмертными. В первом мире\hyp{}обители вы в основном человек, всего лишь смертное существо без материального тела, человеческий разум, помещенный в моронтийную форму --- материальное тело моронтийного мира, а не смертное прибежище из плоти и крови. Во время слияния с Настройщиками вы действительно переходите от смертного состояния к бессмертному статусу, и ко времени завершения вами иерусемской стези вы будете полноценными моронтийцами.
\usection{10. Иерусемское гражданство}
\vs p047 10:1 Прием нового класса выпускников мира\hyp{}обители служит для всего Иерусема сигналом собраться в качестве встречающего комитета. Даже спорнагии, и те радуются прибытию этих торжествующих восходящих смертных эволюционного происхождения, тех, кто закончил планетарную гонку и завершил совершенствование в мирах\hyp{}обителях. На этих праздничных мероприятиях отсутствуют только физические контролеры и Руководители Моронтийной Мощи.
\vs p047 10:2 \pc Иоанну Богослову было видение прибытия класса развивающихся смертных из седьмого мира\hyp{}обители на их первое небо, к красотам Иерусема. Он записал: «И видел я как бы стеклянную гладь, смешанную с огнем; и тех, кто одержал победу над зверем, который изначально был в них, и победу над образом, остававшимся во всех мирах\hyp{}обителях, и, наконец, над последней печатью и следом, стоящими на стеклянной глади, держащими арфу Бога и поющими песнь избавления от смертного страха и от смерти». (Усовершенствованная космическая связь будет во всех этих мирах; повсеместный прием вами таких сообщений будет возможен благодаря наличию у каждого из вас «арфы Бога», --- моронтийного приспособления, компенсирующего неспособность незрелого моронтийного чувствительного механизма непосредственно настраиваться на прием космических сообщений.)
\vs p047 10:3 У Павла также было видение отряда восходящих граждан, состоящего из совершенствующихся смертных, на Иерусеме, ибо он писал: «Но вы приступили к горе Сиону и ко граду Бога живого, к небесному Иерусалиму и неисчислимому воинству ангелов, к великому собору Михаила и к духам праведников, достигших совершенства».
\vs p047 10:4 \pc По достижении смертными постоянного пребывания в центре системы, действительных воскрешений больше не будет. Моронтийная форма, даруемая вам при оставлении стези в мире\hyp{}обители, такая же, какая будет сопровождать вас до конца вашего опыта пребывания в локальной вселенной. Время от времени будут происходить перемены, но вы сохраните ту же самую форму до тех пор, пока не расстанетесь с ней, когда явитесь как духи первой стадии, готовящиеся к переходу в сверхвселенские миры восходящей культуры и духовного воспитания.
\vs p047 10:5 Смертные, проходящие свой путь продвижения в обителях, семь раз переживают сон приспособления и пробуждение воскрешения. Но вот последний зал воскрешения, последняя комната пробуждения осталась позади в седьмом мире\hyp{}обители. Перемена формы больше не повлечет за собой пробелы в сознании или нарушения целостности личной памяти.
\vs p047 10:6 \pc Смертная личность, рожденная в эволюционных мирах и помещенная в ковчег плоти, --- объятая Таинственными Помощниками и облаченная Духом Истины, --- не является всецело мобилизованной, реализованной и объединенной до тех пор, пока такой иерусемский гражданин не получит разрешение на отправление в Эдентию и не будет провозглашен истинным членом моронтийного отряда Небадона --- бессмертным, продолжающим существование в посмертии существом, связанным с Настройщиком, существом, восходящим к Раю, личностью с моронтийным статусом, истинным чадом Всевышних.
\vs p047 10:7 \pc Смерть --- это способ ухода от материальной жизни во плоти; и опыт продвижения по жизни в обителях, в семи мирах исправляющего обучения и культурного образования представляет собой введение продолжающих существование в посмертии в моронтийный путь --- переходную жизнь, лежащую между эволюционно\hyp{}материальным существованием и высшим духовным достижением восходящих существ, живущих во времени, которым суждено достигнуть врат вечности.
\vsetoff
\vs p047 10:8 [Под покровительством Блестящей Вечерней Звезды.]
