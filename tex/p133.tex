\upaper{133}{Возвращение из Рима}
\author{Комиссия срединников}
\vs p133 0:1 Покидая Рим, Иисус не попрощался ни с кем из своих друзей. Книжник из Дамаска появился в Риме без объявления и точно так же исчез. Прошел целый год, прежде чем те, кто знали и любили его, оставили надежду увидеть его снова. К концу второго года маленькие группы тех, кто знали его, начали собираться вместе; их объединяли общий интерес к учению Иисуса и воспоминания о том прекрасном времени, которое они провели с ним. Эти небольшие группы стоиков, киников и почитателей мистериальных культов продолжали время от времени устраивать эти неформальные встречи вплоть до появления в Риме первых проповедников христианской религии.
\vs p133 0:2 \P\ Гонод и Ганид приобрели в Александрии и в Риме столько вещей, что решили отправить свое имущество с караваном, шедшим в Тарентум, в то время как трое путешественников налегке пустились в путь пешком по великой Аппиевой дороге через всю Италию. В пути им встречались самые разные люди. Вдоль дороги жили многие благородные римские граждане и греческие колонисты, но среди них уже начали поселяться потомки многочисленных представителей низшего сословия рабов.
\vs p133 0:3 Однажды, когда путешественники были уже на полпути к Тарентуму, во время отдыха за дневной трапезой Ганид прямо спросил Иисуса, что тот думает о кастовой системе Индии. Иисус ответил ему: «Хотя люди во многом отличаются друг от друга, перед Богом и в духовном мире все смертные равны и в глазах Бога делятся только на две группы, а именно: на тех, кто желает исполнять его волю, и тех, кто этого делать не хочет. Точно так же и вселенная среди жителей обитаемого мира различает два основных класса: тех, кто знает Бога, и тех, кто не знает его. Неспособных узнать Бога относят к животному миру данной планеты. Людей же можно разделить на множество классов в соответствии с различными свойствами, такими как физическое и умственное развитие, общественное положение, профессия, моральные качества, однако к какому бы классу ни относились смертные, перед судом Бога они равны, ибо Бог поистине не взирает на лица. И хотя нельзя не признать, что люди различаются в интеллектуальном, социальном и нравственном плане, вы не должны делать подобных различий в духовном братстве людей, собравшихся для почитания Бога в его присутствии».
\usection{1. Милосердие и правосудие}
\vs p133 1:1 Весьма интересный случай произошел однажды в послеполуденное время у дороги, недалеко от Тарентума. Путешественники увидели, как грубый расхулиганившийся юнец жестоко избивает маленького мальчика. Иисус сразу же бросился на помощь подвергшемуся нападению мальчику, крепко схватил его обидчика и держал до тех пор, пока мальчуган не убежал. Как только Иисус отпустил малолетнего хулигана, на него накинулся Ганид и стал отчаянно колотить, но Иисус, к удивлению Ганида, сразу вмешался и, удержав его, позволил испуганному мальчишке удрать. Переведя дыхание, юноша взволнованно воскликнул: «Не понимаю тебя, Учитель. Если милосердие приказывает спасти меньшего мальчика, то не требует ли справедливость наказать большего обидчика?» Отвечая, Иисус сказал:
\vs p133 1:2 «Ганид, ты, очевидно не понимаешь. Милосердное служение --- всегда служение индивидуальное, справедливое же наказание --- функция общественных, правительственных или вселенских административных групп. Как индивидуум я обязан проявить милосердие. Я должен прийти на помощь мальчику, на которого напали, и с полным основанием могу применить силу, дабы сдержать обидчика. Именно так я и поступил. Я добился освобождения пострадавшего, на чем мое милосердное служение и закончилось. Затем я силой удерживал обидчика столько времени, сколько нужно было, чтобы слабейший мог убежать, а потом прекратил вмешательство. Я не стал судить обидчика и, таким образом, разбираться в причинах его поступка --- выяснять, почему он напал на своего товарища --- а затем осуществлять наказание, которое, по моему разумению, было бы справедливым воздаянием за преступление, которое он совершил. Милосердие, Ганид, может быть неограниченным, правосудие же --- должно быть точным. Неужели ты не понимаешь, что на свете нет и двух людей, которые сошлись бы в том, какое наказание удовлетворило бы требованиям правосудия? Один счел бы необходимым наказать преступника сорока ударами плетью, другой --- двадцатью, а третий в качестве справедливого наказания предложил бы заключение в одиночной камере. Разве тебе не ясно, что в этом мире подобную ответственность следует возлагать на группу людей или же отправлять правосудие должны избранные представители этой группы? Во вселенной вершить правосудие доверяют тем, кто в полной мере знает предысторию преступления и его мотивы. В цивилизованном обществе и организованной вселенной отправление правосудия предполагает вынесение справедливого приговора, основанного на решении беспристрастного суда, и подобное исключительное право доверяется судебным группам миров и всезнающим управителям высших вселенных всего творения».
\vs p133 1:3 Иисус и юноша несколько дней обсуждали проблему проявления милосердия и отправления правосудия, и Ганид наконец до определенной степени понял, почему Иисус сам лично не стал драться. И все же он задал Иисусу еще один вопрос, на который никак не мог найти ответа, удовлетворившего бы его полностью. Вопрос был таков: «Но, Учитель, что бы ты сделал, если бы на тебя напал недобрый человек, оказавшийся сильнее тебя, который стал бы угрожать твоей жизни? Разве не стал бы ты себя защищать?» Иисус не мог дать совершенно исчерпывающий ответ на вопрос юноши, ибо не хотел открывать ему, что он, Иисус, живет на земле, дабы явить всей наблюдавшей за ним вселенной любовь Райского Отца; тем не менее, Иисус сказал Ганиду:
\vs p133 1:4 «Ганид, я очень хорошо понимаю, что многие из этих проблем смущают тебя, и постараюсь ответить на твой вопрос. Прежде всего, каким бы нападениям я ни подвергся, я бы постарался определить, является ли напавший на меня сыном Бога --- моим братом по плоти, --- и если бы я понял, что данное существо не обладает ни нравственным суждением, ни духовным разумом, то без колебаний, невзирая на последствия для обидчика, сделал бы все, что в моих силах, дабы себя защитить. Однако давать подобный отпор моему собрату, который является сыном Бга, даже с целью самообороны я бы не стал. То есть не стал бы наказывать его заранее и без суда за нападение на меня. Я всеми возможными способами постарался бы его остановить и отговорить от подобного шага, и если бы мои попытки избежать столкновения оказались тщетными, я постарался бы смягчить конфликт. Ганид, я полностью доверяю заботе Отца Небесного обо мне; я посвящаю себя исполнению воли моего Отца на небе. Я не верю, что мне может грозить \bibemph{настоящее зло;} не верю я и тому, что делу моей жизни действительно могут помешать происки врагов, и мы, конечно, не должны опасаться насилия со стороны друзей. Я абсолютно уверен в дружественном отношении ко мне всей вселенной, и в эту всепобеждающую правду я буду верить, уповая на нее всем сердцем, невзирая на любую видимость обратного».
\vs p133 1:5 Однако ответ Иисуса так и не удовлетворил Ганида до конца. Они еще много раз говорили об этом, и Иисус рассказал ему кое\hyp{}что о своем детстве, а также об Иакове, сыне каменщика. Узнав о том, как Иаков вызвался защищать Иисуса, Ганид сказал: «О, теперь я начинаю понимать! Во\hyp{}первых, едва ли любой нормальный человек захочет напасть на такого доброго человека, как ты, а если кто\hyp{}нибудь, не подумав, и сделает нечто подобное, то рядом обязательно окажется другой смертный, который бросится тебе на помощь, как и ты помогаешь всякому, когда видишь, что он попал в беду. Учитель, сердцем я с тобой соглашаюсь, но ум все\hyp{}таки говорит мне: если бы Иаковом был я, то с радостью наказал бы тех злых людей, которые посмели на тебя напасть, зная, что ты не будешь защищаться. Я думаю, что тебе нечего опасаться на жизненном пути, ведь ты отдаешь столько времени, помогая другим и служа ближним в несчастье. Да, скорее всего рядом с тобой всегда будет кто\hyp{}нибудь, кто за тебя заступится». Иисус ответил: «Такое испытание еще не пришло, Ганид, а когда оно настанет, мы будем подчиняться воле Отца». Вот почти все, что сказал Учитель мальчику о сложном вопросе самозащиты и непротивления. В другой раз ему удалось выпытать мнение Иисуса о том, что организованное общество имеет полное право применять силу для исполнения своего справедливого приговора.
\usection{2. Погрузка на корабль в Тарентуме}
\vs p133 2:1 Ожидая на причале разгрузки корабля, путешественники увидели, что некий человек дурно обращается со своей женой. По своему обыкновению Иисус вступился за сторону, подвергшуюся нападению. Он подошел к разгневанному мужу сзади и, тихонько похлопав его по плечу, сказал: «Друг мой, могу я немного поговорить с тобой с глазу на глаз?» Такое обращение привело рассерженного мужчину в замешательство, и после минутного колебания он запинаясь проговорил: «Зачем? Чего ты от меня хочешь?» Иисус отвел его в сторону и сказал: «Друг мой, я понимаю, с тобой должно быть случилось что\hyp{}то ужасное; я бы очень желал, чтобы ты рассказал мне, что же могло произойти с таким сильным мужчиной, отчего он набросился на свою жену, мать своих детей прямо здесь, у всех на глазах. Я уверен --- ты чувствуешь, что у тебя есть на то веская причина. Так что же сделала эта женщина такого, чтобы заслужить подобное обращение с собой мужа? Глядя на тебя, я, кажется, замечаю на твоем лице не только любовь к справедливости, но и желание проявить милосердие. Я не побоюсь сказать: случись тебе, проходя мимо, увидеть, что на меня напали грабители, ты без колебания бросился бы меня спасать. Мне кажется, за свою жизнь ты совершил немало таких смелых поступков. А теперь, дружище, расскажи, что же все\hyp{}таки произошло? Эта женщина что\hyp{}нибудь не так сделала или все дело в том, что ты потерял голову и бездумно набросился на нее?» Сердце мужчины тронули не столько эти слова, сколько добрый взгляд и сочувствующая улыбка, которой Иисус наградил его, заканчивая свои замечания. Мужчина сказал: «Я вижу --- ты священник киников, и благодарен тебе за то, что ты меня остановил. Моя жена ничего особенного не сделала, она хорошая женщина, но она раздражает меня своей манерой критиковать меня на людях; это просто выводит меня из себя. Я сожалею о том, что недостаточно умею владеть собой, и обещаю постараться сдержать свое прежнее обещание, данное одному из твоих собратьев, который много лет назад учил меня искать лучший путь. Обещаю тебе».
\vs p133 2:2 Тогда, прощаясь с ним, Иисус сказал: «Брат мой, всегда помни, мужчина не имеет законной власти над женщиной до тех пор, пока женщина сознательно и добровольно не даст ему такой власти. Твоя жена взяла на себя обязательство вместе с тобой пройти жизненный путь, помогать тебе бороться с его тяготами и принять на себя большую часть бремени в рождении и воспитании ваших детей; поэтому справедливость требует, чтобы в ответ за это особое служение она получила от тебя ту особую защиту, какую мужчина может дать женщине как партнеру, который должен вынашивать, рожать и воспитывать детей. Нежная забота и внимание, которые мужчина готов дарить жене и детям, означает, что этот мужчина достиг более высоких уровней творческого и духовного самосознания. Разве ты не знаешь, что мужчина и женщина --- соратники Бога, ведь они тоже сотрудничают с ним в создании существ, которые, вырастая, обретают потенциал бессмертной души? Отец Небесный обращается с Духом\hyp{}Матерью детей вселенной как с равной себе. Идя вместе по жизни и поровну деля ее радости и невзгоды с матерью своих детей, которая во всей полноте разделяет с тобой этот божественный опыт воспроизведения самих себя в собственных детях, ты уподобляешься Богу. О если бы ты только мог любить своих детей, как Бог любит тебя. Тогда бы ты любил и лелеял свою жену, как Отец Небесный славит и превозносит Бесконечный Дух, матерь всех духовных детей необъятной вселенной».
\vs p133 2:3 Взойдя на корабль, путешественники увидели, как супруги со слезами на глазах молча обнимают друг друга. Услышав вторую половину обращения Иисуса к мужчине, Гонод весь день провел в размышлениях и решил, вернувшись в Индию, перестроить свою домашнюю жизнь.
\vs p133 2:4 Путешествие до Никополя было приятным, но и продолжительным из\hyp{}за неблагоприятного ветра. Все трое провели много часов, говоря о том, как они жили в Риме, и вспоминая все, что случилось с ними с тех пор, как они впервые встретились в Иерусалиме. Ганид все больше проникался духом личного служения. Он начал учить стюарда на корабле, однако на второй день, когда дело дошло до обсуждения серьезных религиозных вопросов, обратился за помощью к Иешуа.
\vs p133 2:5 Они провели несколько дней в Никополе, в городе, который основал Август около пятидесяти лет назад как «город победы» в память о битве при Актиуме --- на этом месте он расположился лагерем со своей армией перед сражением. Они остановились в доме Иерамии, греческого прозелита иудейской веры, с которым познакомились на корабле. В этом же доме с сыном Иерамии провел всю зиму и апостол Павел во время своего третьего миссионерского путешествия. Из Никополя на том же корабле они поплыли в Коринф, столицу римской провинции Ахаии.
\usection{3. В Коринфе}
\vs p133 3:1 Ко времени прибытия в Коринф Ганид весьма серьезно заинтересовался еврейской религией, а потому не было ничего удивительного в том, что однажды, когда они проходили мимо синагоги и увидели людей, входящих в нее, он попросил Иисуса взять его на службу. В тот день они слушали рассуждения ученого раввина о «Судьбе Израиля», а после службы познакомились с неким Криспом, главным управителем этой синагоги. На синагогальные службы они ходили еще много раз, однако главным образом затем, чтобы встретиться с Криспом. Ганид очень полюбил Криспа, его жену и их пятерых детей. Ему доставляло большое удовольствие наблюдать за тем, как организована семейная жизнь этого еврея.
\vs p133 3:2 Пока Ганид изучал семейную жизнь, Иисус учил Криспа лучшим путям религиозной жизни. Иисус провел более двадцати бесед с этим прогрессивным евреем; неудивительно, что спустя годы, когда Павел проповедовал в той же самой синагоге, а евреи отвергли его послание, проголосовав за то, чтобы в дальнейшим запретить ему выступать в синагоге, и апостол пошел к неевреям, этот самый Крисп со всей своей семьей принял новую религию, став одним из главных столпов христианской церкви, которую Павел впоследствии основал в Коринфе.
\vs p133 3:3 За восемнадцать месяцев, в течение которых Павел проповедовал в Коринфе (позднее к нему присоединились Тимофей и Сила), он встретился со многими другими людьми, которых учил «еврейский наставник сына индийского купца».
\vs p133 3:4 В Коринфе путешественники встречались с представителями всех рас, которые прибыли сюда с трех континентов. После Александрии и Рима это был самый космополитический город Средиземноморской империи. В этом городе многое привлекало внимание, и Ганид без устали бродил по крепости, расположенной на высоте две тысячи футов над уровнем моря. Значительную часть своего свободного времени он также провел возле синагоги и в доме Криспа. Вначале он был потрясен, а потом очарован положением женщины в еврейском доме; оно явилось для молодого индуса настоящим откровением.
\vs p133 3:5 Иисус и Ганид были частыми гостями и в другом еврейском доме --- в доме Иуста, благочестивого купца, который жил рядом с синагогой. Впоследствии множество раз, когда апостол Павел останавливался в этом доме, он слушал рассказы о том, как приходил сюда индийский мальчик со своим наставником\hyp{}евреем; и Павлу и Иусту очень хотелось узнать, что же стало со столь мудрым и блестящим учителем\hyp{}иудеем.
\vs p133 3:6 Еще в Риме, Ганид заметил, что Иисус отказывается ходить вместе с ними в публичные бани. Позднее молодой человек несколько раз пытался заставить Иисуса подробнее высказать свою точку зрения на отношения между полами. Хотя Иисус и отвечал на подобные вопросы мальчика, он, тем не менее, казалось, никогда не испытывал желания долго рассуждать на эти темы. Однажды вечером, когда они прогуливались по Коринфу, там, где стена крепости спускалась к самому морю, к ним пристали две публичные женщины. Ганид усвоил для себя (и совершенно правильно), что Иисус --- человек высоких идеалов, что он с отвращением относится ко всему нечистому и порочному; поэтому он ответил женщинам резко и грубым жестом их отогнал. Увидев это, Иисус сказал Ганиду: «У тебя хорошие намерения, но ты не должен позволять себе так говорить с детьми Бога, даже если окажется, что они заблудшие дети. Кто мы такие, чтобы судить этих женщин? Неужели тебе известны все обстоятельства, которые вынудили их добывать хлеб таким способом? Побудь здесь со мной, пока мы будем говорить об этом». Блудницы удивились тому, что он сказал, еще больше, чем Ганид.
\vs p133 3:7 Пока они стояли там в лунном свете, Иисус продолжил: «В сознании каждого человека живет божественный дух, дар Отца Небесного. Этот благой дух всегда старается вести нас к Богу, помочь нам найти Бога и узнать Бога; однако в смертном человеке также заложено множество естественных физических наклонностей, которые Творец вложил в него, дабы они служили благополучию индивидуума и его расы. В наше время люди нередко теряют твердость духа, когда в мире, где царят грех и эгоизм, пытаются понять себя и преодолеть многочисленные жизненные трудности. Я вижу, Ганид, что ни одна из этих женщин не стала порочной по собственной воле. Глядя на их лица, я могу сказать, что им пришлось испытать немало горя; они много пострадали от судьбы, которая была к ним явно жестока; они не выбрали такой образ жизни сознательно; поддавшись унынию, граничащему с отчаянием, они не устояли перед временными трудностями и решили, что такой противный способ добывания хлеба --- лучший выход из положения, которое казалось им безнадежным. Ганид, у некоторых людей, действительно, порочное сердце; они сознательно совершают мерзости, но скажи мне, посмотрев на эти лица, теперь залитые слезами, видишь ли ты в них что\hyp{}нибудь дурное или порочное?» Здесь Иисус остановился, ожидая ответа Ганида, и тот, задыхаясь и запинаясь, проговорил: «Нет, Учитель, не вижу. Я приношу извинения за свою грубость и прошу их простить меня». Тогда Иисус сказал: «И я от их имени говорю тебе: они простили тебя, как говорю и от имени Отца моего Небесного: он их тоже простил. Теперь же все вместе пойдемте со мной в дом друга, где мы попросим дать нам подкрепиться и подумаем о новой и лучшей жизни в будущем». До этих пор изумленные женщины не произнесли ни слова; посмотрев друг на друга, они молча пошли за мужчинами.
\vs p133 3:8 Вообразите же удивление жены Иуста, когда в сей поздний час в ее доме появились Иисус с Ганидом и этими двумя незнакомками и Иисус сказал: «Ты, конечно, простишь нас за то, что мы пришли к тебе в этот час, но Ганид и я хотели бы слегка перекусить, и мы разделим наш ужин с нашими новыми приятельницами, которые тоже нуждаются в пище; а кроме того, мы пришли к тебе с мыслью, что тебе будет интересно обсудить с нами, как лучше помочь этим женщинам начать новую жизнь. Свою историю они смогут рассказать тебе сами, но, как я догадываюсь, им пришлось немало пережить, и само присутствие их здесь, в твоем доме, свидетельствует о том, как страстно они желают узнать добрых людей и сколь охотно они воспользуются возможностью показать всему миру --- и даже ангелам небесным --- какими отважными и благородными женщинами они могут стать».
\vs p133 3:9 Когда Марта, жена Иуста, разложила еду на столе, Иисус неожиданно стал со всеми прощаться и сказал: «Уже поздно, и отец молодого человека давно ожидает нас, поэтому мы просим нас извинить за то, что оставляем вас --- трех женщин --- возлюбленных детей Всевышнего одних. Я буду молиться о том, чтобы он духовно наставлял вас, пока вы будете строить планы о новой и лучшей жизни на земле и вечной жизни после смерти».
\vs p133 3:10 Таким образом Иисус и Ганид оставили женщин одних. До сих пор обе блудницы не произнесли ни слова; ничего не говорил и Ганид. Марта какое\hyp{}то время тоже молчала, но вскоре оказалась на высоте положения и сделала для незнакомок все, на что надеялся Иисус. Старшая из этих двух женщин вскоре умерла со светлыми надеждами на вечное спасение, младшая же работала у Иуста и впоследствии стала и оставалась до самой смерти членом первой христианской церкви в Коринфе.
\vs p133 3:11 Несколько раз в доме Криспа Иисус и Ганид встречались с неким Гаием, который потом стал преданным сторонником Павла. За эти два месяца, проведенные в Коринфе, они откровенно побеседовали со множеством достойных внимания людей, более половины из которых благодаря этим, по\hyp{}видимости, случайным встречам стали членами христианской общины, возникшей впоследствии.
\vs p133 3:12 Когда Павел впервые отправился в Коринф, он не намеревался там долго задерживаться. Но он не знал, насколько хорошо еврейский наставник приготовил путь для его трудов. Более того, он обнаружил огромный интерес, уже пробужденный Акилой и Прискиллой. Акила был одним из киников, с которым Иисус общался в Риме. Эти супруги были еврейскими беженцами из Рима и быстро восприняли учение Павла. Вместе с ними апостол жил и работал, поскольку Акила и Прискилла тоже занимались изготовлением шатров. Благодаря этим обстоятельствам Павел и продолжил свое пребывание в Коринфе.
\usection{4. Личные деяния в Коринфе}
\vs p133 4:1 В Коринфе с Иисусом и Ганидом случилось еще много интересного. Они непосредственно побеседовали с огромным числом людей, извлекших большую пользу из наставлений, полученных от Иисуса.
\vs p133 4:2 \P\ Мельника он учил, как размалывать зерна истины жерновами жизненного опыта, чтобы самое трудное в божественной жизни было доступно даже слабым и немощным смертным. Иисус говорил: «Питай молоком истины тех, кто в познании духа еще младенец. В живом и любовном служении подавай людям духовную пищу в привлекательном для них виде, приготовив ее так, чтобы она была доступна всем, кто у тебя ее просит».
\vs p133 4:3 \P\ Римскому центуриону он сказал: «Отдай кесарю кесарево, а Богу божье. Преданное служение Богу и верная служба кесарю не противоречат друг другу, если кесарь не осмеливается самонадеянно претендовать на такое почитание себя, какое пристало лишь Божеству. Преданность Богу, если ты сумеешь познать его, сделает тебя еще более преданным и верным в твоем служении достойному императору».
\vs p133 4:4 \P\ Ревностному главе почитателей культа Митры он сказал: «Ты правильно поступаешь, ибо ищешь религию вечного спасения, но ты ошибаешься в том, что хочешь найти столь славную истину в искусственных мистериях и в человеческих философиях. Разве не известно тебе, что мистерия вечного спасения --- в твоей собственной душе. Разве не знаешь ты, что Бог небес послал свой дух, чтобы он жил в тебе, и что сей дух всех смертных, возлюбивших истину и любящих Бога, выведет из этой жизни и проведет через врата смерти к вечным высотам света, где Бог ожидает своих детей, дабы принять их к себе? Не забывай никогда: вы, знающие Бога, --- сыновья Бога, если истинно стремитесь уподобиться ему».
\vs p133 4:5 \P\ Учителю\hyp{}эпикурейцу он сказал: «Ты поступаешь правильно, выбирая лучшее и ценя добро, но разве ты мудр, не умея увидеть то еще более важное в жизни смертного, что заключено в духовных сферах, исходящих из осознания присутствия Бога в человеческом сердце? Самое великое в опыте всей жизни человека --- это достижение осознания того, что человек знает Бога, дух которого живет в тебе и стремится вести тебя по долгому и почти бесконечному пути достижения личного присутствия нашего всеобщего Отца, Бога всего творения, Господа вселенных».
\vs p133 4:6 \P\ Греческому подрядчику и строителю он сказал: «Друг мой, возводя материальные сооружения для людей, созидай в себе духовную личность по образу и подобию божественного духа в твоей душе. Не позволяй своим достижениям строителя временных вещей превзойти твои достижения как духовного сына царства небесного. Возводя временные дома для других, позаботься о том, чтобы запечатлеть свое имя на вечных чертогах, предназначенных для тебя. Всегда помни: существует град, основание которого --- праведность и истина; его строитель и творец --- Бог».
\vs p133 4:7 \P\ Римскому судье он сказал: «Верша суд над людьми, помни, что однажды сам предстанешь перед судом Правителей вселенной. Суди не только справедливо, но и милосердно, ибо и сам когда\hyp{}нибудь окажешься в руках Верховного Судии и будешь жаждать милосердного разбирательства. Суди других так, как при сходных обстоятельствах судили бы тебя самого, и, таким образом, руководствуйся духом закона так же, как и его буквой. Как ты вершишь основанное на справедливости правосудие, помня о нужде тех, кто предстал перед тобой, так и у тебя будет право ожидать, что милосердие смягчит правосудие, когда в свое время перед Судьей всей земли предстанешь ты».
\vs p133 4:8 \P\ Хозяйке греческого постоялого двора он сказал: «Будь гостеприимна, как тот, кто принимает у себя детей Всевышнего. Возвысь свой ежедневный изнурительный труд до уровня высочайшего искусства, все более сознавая, что, служа людям, ты служишь Богу, который пребывает в них своим духом, нисшедшим, дабы поселиться в сердцах людей и тем самым преобразовать их ум, а их души привести к познанию Райского Отца всех ниспосланных даров божественного духа».
\vs p133 4:9 \P\ Иисус многократно встречался с купцом\hyp{}китайцем. Прощаясь с ним, он посоветовал ему: «Поклоняйся одному Богу, своему истинному духовному прародителю. Помни, что дух Отца всегда живет в тебе, постоянно указывая твоей душе путь к небу. Следуя не сознаваемым тобой указаниям этого бессмертного духа, ты, несомненно, пойдешь дальше по высокому пути отыскания Бога. Когда же ты достигнешь Отца Небесного, то произойдет это потому, что, ища его, ты все больше и больше уподоблялся ему. Итак, прощай, Чанг, но прощай ненадолго, ибо мы снова встретимся в мирах света, где у Отца душ много чудесных мест, где могут остановиться те, кто направляется к Раю».
\vs p133 4:10 \P\ Путешественнику из Британии он сказал: «Брат мой, я вижу, ты ищешь истину, и я говорю тебе, что дух Отца всякой истины может пребывать в тебе. Пытался ли ты когда\hyp{}нибудь искренне говорить с духом своей собственной души? Сделать подобное, действительно, трудно, добиться осознания успеха удается редко; однако всякая честная попытка общения материального разума с духом, пребывающим в нем, приводит к определенному успеху, невзирая на то, что большая часть всех подобных чудесных переживаний человека должна надолго оставаться в виде сверхсознательных следов в душах таких знающих Бога смертных».
\vs p133 4:11 \P\ Мальчику, убежавшему из дома, Иисус сказал: «Запомни, существуют две вещи, от которых убежать нельзя: тебе не уйти от Бога и от самого себя. Куда бы ты ни шел, ты несешь с собой самого себя и дух Отца Небесного, живущий в твоем сердце. Сын мой, перестань себя обманывать; наберись смелости и смотри жизни прямо в глаза; будь твердо убежден в том, что с Богом тебя связывают отношения отца и сына, и в несомненности вечной жизни, --- как я учил тебя об этом. Прямо сегодня поставь себе целью стать настоящим мужчиной, мужчиной, который решил смотреть жизни в лицо смело и сознательно».
\vs p133 4:12 \P\ Приговоренному к смерти преступнику в его последний час он сказал: «Брат мой, для тебя настали недобрые времена. Ты сбился с пути; ты запутался в сетях преступления. Из разговора с тобой я знаю: ты не собирался делать того, что будет стоить тебе твоей временной жизни. Однако ты совершил это зло, и твои сограждане вынесли тебе приговор --- виновен; они решили, что ты должен умереть. Ни ты, ни я не можем отказать государству в праве защищать себя так, как оно считает нужным. Видимо, не в человеческих силах избежать наказания за преступление, совершенное тобой. Твои сограждане обязаны судить тебя по твоим поступкам, однако есть Судья, к которому ты можешь взывать о прощении и который будет судить тебя по твоим настоящим побуждениям и лучшим из твоих намерений. Если твое раскаяние подлинно, а вера искренна, ты можешь не бояться предстать перед судом Бога. То, что совершенная тобой ошибка влечет за собой наказание смертью, налагаемое людьми, отнюдь не противоречит возможности, что твоя душа добьется справедливости и заслужит милосердия в небесных судах».
\vs p133 4:13 \P\ Иисус долго и откровенно беседовал с множеством страждущих душ, и подобных бесед было столько, что описать их все здесь невозможно. Трое путешественников наслаждались жизнью в Коринфе. Если не считать Афин, еще более, чем Коринф, знаменитого центра образования, этот город был наиважнейшим в Греции времен Римской империи, и два месяца, проведенные ими в этом процветающем центре торговли, всем троим предоставили возможность обрести весьма ценный опыт. Пребывание в этом городе оказалось одной из самых интересных остановок, что случились на обратном пути из Рима.
\vs p133 4:14 У Гонода было много дел в Коринфе, но в конце концов все дела были сделаны, и они приготовились отплыть в Афины. Свое путешествие они совершили на небольшой лодке, которую от одной гавани Коринфа к другой, отстоявшей от нее на десять миль, можно было перевозить сушей по дороге.
\usection{5. В Афинах --- беседа о науке}
\vs p133 5:1 Вскоре они прибыли в древнейший центр греческой науки и образования, и Ганид радовался мысли о том, что он в Афинах, в Греции, что он в культурном центре бывшей империи Александра, границы которой когда\hyp{}то простирались до его родной Индии. Дел у Гонода почти не было; поэтому большую часть своего времени Гонод провел с Иисусом и Ганидом, посещая многочисленные достопримечательности и слушая интересные беседы мальчика с его многосторонним учителем.
\vs p133 5:2 В Афинах по\hyp{}прежнему процветал великий университет, и трое путешественников часто ходили в его учебные залы. Посещая лекции в Александрийском музее, Ганид и Иисус подробно обсуждали учение Платона. Они все наслаждались произведениями искусства Греции, которые еще кое\hyp{}где встречались в городе.
\vs p133 5:3 И отец и сын получили огромное удовольствие от посвященной науке беседы Иисуса с греческим философом, которая произошла однажды вечером в гостинице, где они остановились. Педант говорил почти три часа; когда же он закончил свою речь, Иисус, в изложении языком современной мысли, сказал следующее:
\vs p133 5:4 \P\ Возможно, когда\hyp{}нибудь ученые научатся измерять энергию или силовое действие гравитации, света и электричества, однако те же ученые никогда не смогут (научно) объяснить, что представляют собой эти явления вселенной. Наука занимается изучением физического действия энергии, а религия --- вечными ценностями. Истинная философия происходит от мудрости, которая наилучшим образом соотносит эти качественные и количественные наблюдения. Для чистого же ученого\hyp{}физика всегда существует опасность заболеть математической гордыней и статистическим эгоизмом, не говоря уже о духовной слепоте.
\vs p133 5:5 В материальном мире действуют законы логики; на математику можно положиться, если применение ее ограничить изучением физических явлений; однако ни логику, ни математику нельзя считать абсолютно надежными и непогрешимыми средствами при решении жизненных проблем. В жизни бывают явления, которые нельзя отнести к чисто материальным. Арифметика утверждает: если один человек может остричь овцу за десять минут, то десять человек остригут ее за одну минуту. Расчет точный, но все же неправильный, ибо десять человек этого сделать не смогут; они будут так друг другу мешать, что работа их растянется очень надолго.
\vs p133 5:6 Математика говорит: если один человек обладает некоторой единицей интеллектуальной и нравственной ценности, то у десяти человек эта величина в десять раз больше. Однако, рассуждая о человеческой личности, гораздо правильнее сказать, что такое соединение личностей дает не просто арифметическую сумму, а сумму, равную квадрату числа личностей, участвующих в уравнении. Социальная группа человеческих существ, действующих слаженно и гармонично, обладает силой, намного превосходящей простую сумму сил ее отдельных членов.
\vs p133 5:7 Количество может рассматриваться как определенный \bibemph{факт} и, таким образом, приобретает некоторый характер научной однородности. Качество же, являясь категорией мыслительного толкования, представляет собой оценку \bibemph{ценностей} и должно, следовательно, оставаться опытом индивидуума. Когда и наука, и религия станут менее догматичными и более терпимыми к критике, тогда и философия начнет достигать \bibemph{единства} в разумном понимании вселенной.
\vs p133 5:8 Если только ты поймешь, что в действительности происходит в космической вселенной, то увидишь: в ней есть единство. Реальная вселенная дружественна к каждому из детей вечного Бога. В действительности проблема в том, как же конечному разуму человека достичь логического, истинного и соответствующего единства мышления? К этому состоянию познания вселенной человеческий разум может прийти лишь путем постижения того, что количественный факт и качественная ценность имеют общую первопричину, и первопричина эта --- Райский Отец. Подобное понимание реальности не только позволяет шире осознать значимость единства вселенских явлений, но и указывает духовную цель, личности --- достижение все высших уровней развития. И именно такое представление о единстве способно отразить неизменную основу существующей вселенной непрерывно изменяющихся безличных связей и совершенствующихся личных отношений.
\vs p133 5:9 Материя, дух и состояние, лежащее между ними, --- вот три взаимосвязанных и взаимозависимых уровня истинного единства реальной вселенной. Какими бы отличными друг от друга ни казались нам такие вселенские явления, как факт и ценность, они, тем не менее, объединяются в Верховном.
\vs p133 5:10 Реальность вещественного бытия присуща непознанной энергии так же, как видимой материи. Когда энергии вселенной замедляются до такой степени, что обретают состояние движения, тогда при благоприятных условиях эти же самые энергии становятся массой. Не забудь и о том, что разум, который лишь один способен воспринимать видимые реальности, сам по себе тоже реален. Основополагающая же причина этой вселенской энергии --- массы, разума и духа --- вечна; она существует и заключается в природе и проявлениях Отца Всего Сущего и его абсолютных равноправных сосуществующих.
\vs p133 5:11 \P\ Всех привели в восторг слова Иисуса, и грек, прощаясь с ними, сказал: «Наконец\hyp{}то мои глаза увидели еврея, который думает не только о своем расовом превосходстве и говорит не только о религии». Затем все отправились спать.
\vs p133 5:12 Пребывание в Афинах оказалось приятным и полезным, но не особенно плодотворным с точки зрения общения с людьми. Слишком многие жители Афин того времени страдали либо интеллектуальным высокомерием, происходившим из их былой репутации, либо умственной тупостью и невежеством; последние были потомками низшего сословия рабов тех древних времен, когда Греция была овеяна славой, а в умах ее народа царила мудрость. Но даже тогда среди граждан Афин все еще можно было найти немало блестящих умов.
\usection{6. В Ефесе --- беседа о душе}
\vs p133 6:1 Покинув Афины, путешественники отправились через Троаду в Ефес, столицу азиатской провинции Рима. Они много раз ходили к знаменитому храму Артемиды Ефесской, расположенному приблизительно в двух милях от города. Во всей Малой Азии Артемида была самой почитаемой богиней, на которую были перенесены черты еще более ранней богини\hyp{}матери древнейших анатолийских времен. Считалось, что грубый идол, установленный в огромном храме, посвященном культу Артемиды, упал с неба. Еще не все черты прежнего воспитания Ганида, которые предписывали почитать изображения как символы божества, были искоренены, и он решил, что не повредит купить маленькую серебряную модель храма в честь этой малоазиатской богини плодородия. В эту ночь они очень долго говорили о поклонении вещам, сделанным человеческими руками.
\vs p133 6:2 На третий день пребывания в Ефесе путешественники пешком оправились вниз по реке, чтобы понаблюдать, как очищают дно у входа в гавань. В полдень они разговорились с молодым финикийцем, который тосковал по родному дому и был чем\hyp{}то сильно расстроен; однако более всего он завидовал некому молодому человеку, который обошел его по службе. Иисус сказал ему утешительные слова и процитировал старинную иудейскую пословицу: «Талант человека определяет его место и приводит его к великим людям».
\vs p133 6:3 Из всех больших городов, которые путешественники посетили во время странствия по Средиземноморью, в этом им удалось сделать меньше всего на благо последующей работы христианских миссионеров. Христианство возникло в Ефесе главным образом благодаря усилиям Павла, который прожил здесь более двух лет, зарабатывая на хлеб тем, что делал палатки, и каждый вечер выступая с лекциями по религии и философии в главной аудитории школы Тиранна.
\vs p133 6:4 В Ефесе жил некий прогрессивный мыслитель, связанный с этой местной философской школой, и Иисус несколько раз с пользой беседовал с ним. В ходе этих бесед Иисус неоднократно употреблял слово «душа». Ученый грек в конце концов спросил у него, что он подразумевает под этим словом, и Иисус ответил:
\vs p133 6:5 \P\ «Душа --- это сознающая себя, распознающая истину и воспринимающая дух часть человека, навсегда возвысившая человеческое существо над уровнем животного мира. Самосознание само по себе и само в себе не есть душа. Нравственное самосознание --- это истинная самореализация человека, оно составляет основу человеческой души; душа же является той частью человека, которая представляет собой потенциал дарующих спасение ценностей, обретенных человеческим опытом. Нравственный выбор и духовное достижение, способность познавать Бога и желание уподобляться ему --- вот характерные особенности души. Душа человека не может существовать вне нравственного мышления и духовной деятельности. Бездеятельная душа --- это душа умирающая. Но душа человека отличается от божественного духа, живущего в сознании. Божественный дух вселяется в человека одновременно с первым нравственным усилием человеческого ума; в этот же момент происходит рождение души.
\vs p133 6:6 Спасение души либо ее утрата зависит от того, достигло ли нравственное сознание человека состояния продолжения существования в посмертии через вечный союз с вступившим с ним в общение даром бессмертного духа. Спасение есть одухотворение самореализации нравственного сознания, которое благодаря этому становится обладателем ценностей, определяющих спасение. Все формы душевных противоречий заключаются в отсутствии гармонии между нравственным, или духовным, самосознанием и самосознанием чисто интеллектуальным.
\vs p133 6:7 Человеческая душа, достигшая зрелости, величия и одухотворенности, приближается к небесному состоянию в том, что она становится ближе к сущности, находящейся между материальным и духовным, материальным «я» и божественным духом. Развивающуюся душу человеческого существа трудно описать и еще труднее показать, ибо ее невозможно обнаружить ни методом материальных исследований, ни путем духовных доказательств. Материальная наука не может продемонстрировать существование души, как не могут этого и чисто духовные искания. Однако, невзирая на неспособность и материальной науки, и духовных исканий обнаружить существование человеческой души, каждый смертный, обладающий нравственным сознанием, \bibemph{знает} о существовании \bibemph{своей} души как \bibemph{реального} и действительного личного опыта».
\usection{7. Пребывание на Кипре --- беседа о разуме}
\vs p133 7:1 Вскоре путешественники отплыли на Кипр с остановкой в Родосе. Долгий путь по воде доставил им немало удовольствия, и на остров, цель своего путешествия, они прибыли, отдохнув телом и воспрянув духом.
\vs p133 7:2 В их планы входило во время этого посещения Кипра по\hyp{}настоящему отдохнуть и развлечься, ибо их путешествие по Средиземноморью подходило к концу. Высадившись в Пафосе, они сразу начали собирать провизию, чтобы провести несколько недель в окрестных горах. На третий день после прибытия, как следует нагрузив вьючных животных, они двинулись в путь.
\vs p133 7:3 Две недели все трое наслаждались походом, но затем неожиданно и внезапно юный Ганид тяжело заболел. Две недели он страдал от жестокой лихорадки, часто бредил, и Иисус с Гонодом не отходили от больного мальчика. Иисус умело и нежно ухаживал за пареньком, и отца поразили те доброта и опытность, которые он проявил, помогая страдающему юноше. До человеческого жилья было далеко, а мальчик заболел настолько серьезно, что его нельзя было перевозить; поэтому они как могли приготовились лечить его прямо здесь, в горах.
\vs p133 7:4 За три недели, пока Ганид выздоравливал, Иисус рассказал ему много интересного о природе и разных ее состояниях. Какое же удовольствие получали все трое, гуляя в горах. Мальчик задавал вопросы, Иисус на них отвечал, а отец изумлялся, глядя на все это.
\vs p133 7:5 Во время последней недели, проведенной в горах, у Иисуса с Ганидом была долгая беседа о деятельности человеческого разума. После нескольких часов разговора мальчик задал такой вопрос: «Но, Учитель, что ты имеешь в виду, когда говоришь, что человек обладает более высокой по сравнению с высшими животными формой самосознания?» Если переложить ответ Иисуса на современный язык, то он был таким:
\vs p133 7:6 \P\ Сын мой, я уже многое рассказал тебе о разуме человека и божественном духе, живущем в нем; теперь же позволь мне подчеркнуть: самосознание --- это \bibemph{реальность.} Когда любое животное начинает само себя сознавать, оно становится примитивным человеком. Подобное происходит в результате координации действий безличной энергии и разума, постигающего дух; именно это явление и служит основанием для дарования человеческой личности абсолютной основы всего --- духа Отца Небесного.
\vs p133 7:7 Мысли --- это не просто регистрация ощущений; мысли --- это ощущения плюс осмысленное толкование их личностью или собственным «я», собственное же «я» больше суммы личных ощущений. В развивающейся индивидуальности начинает возникать нечто, приближающееся к единству; это единство происходит от присутствия пребывающей в человеке частицы абсолютного единства, которое духовно активирует подобный сознающий себя разум, изначально бывший животным.
\vs p133 7:8 Ни одно обыкновенное животное не может обладать самосознанием во времени. Животные обладают способностью физиологически координировать взаимосвязанные распознавания\hyp{}ощущения и соответствующей памятью, но ни одно из них не в состоянии осмысленно распознавать ощущения и осмысленно ассоциировать этот совокупный физический опыт, так как это проявляется в результате разумных и осмысленных человеческих толкований. И этот факт сознательного существования в сочетании с реальностью последующего духовного опыта делает человека потенциальным сыном вселенной и предопределяет достижение им со временем ее Верховного Единства.
\vs p133 7:9 Человеческая личность также не есть просто сумма последовательных состояний сознания. Без эффективной работы сортировщика и ассоциатора сознания не было бы и достаточного единства, гарантирующего возникновение собственного «я». Подобный разобщенный ум едва ли способен достигнуть уровней сознания, характерных для человека. Если бы ассоциации сознания были простой случайностью, то разум всех людей проявлял бы неконтролируемые и произвольные ассоциации, характерные для определенных форм умственного безумия.
\vs p133 7:10 Человеческий разум, сформированный исключительно на осознании физических ощущений, не может достичь духовных уровней; подобного рода материальный разум в высшей степени неспособен воспринимать нравственные ценности, у него нет направляющего чувства духовного влияния, которое столь необходимо для достижения личностью гармоничного единства во времени и неотделимо от существования личности в вечности.
\vs p133 7:11 Человеческий разум начинает рано демонстрировать качества, которые являются сверхматериальными; истинно мыслящий человеческий интеллект отнюдь не связан границами времени. То, что люди настолько по\hyp{}разному ведут свою жизнь, свидетельствует не только о различных наследственных качествах и разном влиянии окружения, но также о степени единения с пребывающим в них духом Отца, достигнутого собственным «я», или мерой отождествления его с ним.
\vs p133 7:12 Человеческому разуму трудно устоять в конфликте двойной зависимости. Пытаясь одновременно служить и добру и злу, душа испытывает страшное напряжение. Только тот ум достигает наивысшего счастья и эффективного единства, который полностью посвящен исполнению воли Отца Небесного. Неразрешенные конфликты уничтожают единство и могут привести к разрушению разума. Однако способности души к продолжению существования отнюдь не благоприятствуют такие попытки добиться спокойствия ума любой ценой, как отказ от благородных устремлений или измена духовным идеалам; подобного спокойствия скорее добиваются, мужественно отстаивая все истинное, и победа эта достигается преодолением зла могучей силой добра.
\vs p133 7:13 \P\ На следующий день путешественники отправились в Саламин, где они погрузились на корабль и отплыли в Антиохию, расположенную на побережье Сирии.
\usection{8. В Антиохии}
\vs p133 8:1 Антиохия была столицей Сирии, одной из провинций Рима; и здесь у имперского губернатора была своя резиденция. В Антиохии проживало полмиллиона человек; это был третий по величине и первый по развращенности и вопиющей аморальности жителей город империи. Гоноду предстояло заключить здесь немало сделок, поэтому Иисус с Ганидом в основном были предоставлены сами себе. В этом многонациональном городе они побывали почти всюду, кроме рощи Дафны. Гонод и Ганид посетили этот известный храм развращенности, но Иисус отказался идти вместе с ними. Подобные сцены не особенно шокировали индусов, но на еврея\hyp{}идеалиста действовали отталкивающе.
\vs p133 8:2 В конце путешествия с приближением к Палестине Иисус становился все более серьезным и задумчивым. В Антиохии он почти ни с кем не общался и редко выходил в город. После многочисленных расспросов, почему учитель проявляет так мало интереса к Антиохии, Ганид в конце концов вынудил Иисуса сказать: «От этого города до Палестины не так далеко. Возможно, когда\hyp{}нибудь я вернусь сюда».
\vs p133 8:3 \P\ В Антиохии с Ганидом произошел очень интересный случай. Этот молодой человек доказал, что он способный ученик и уже начал на практике применять некоторые из поучений Иисуса. В Антиохии жил некий занятый в деле его отца индус, который стал настолько неприятен и так всех раздражал, что уже начали подумывать о его увольнении. Когда Ганид услышал об этом, то отправился в контору отца и провел долгую беседу со своим соотечественником. Оказалось, что у того было чувство, будто он занимается не своим делом. Ганид рассказал ему об Отце Небесном и расширил его понимание религии. Но из всего сказанного Ганидом наиболее помогла иудейская пословица. Вот какими были эти мудрые слова: «За какую работу ни брались бы твои руки, делай ее как можно лучше».
\vs p133 8:4 Подготовив багаж к погрузке на караван верблюдов, путешественники направились в Сидон, а оттуда в Дамаск, в котором пробыли три дня, и приготовились к долгому переходу через пески пустыни.
\usection{9. В Месопотамии}
\vs p133 9:1 Для опытных путешественников в переходе с караваном через пустыню не было ничего необычного. Понаблюдав за тем, как учитель помогал навьючивать их двадцать верблюдов, и услышав, что он вызвался вести их животное сам, Ганид воскликнул: «Учитель, неужели есть хоть что\hyp{}нибудь, чего ты не умеешь делать?» Иисус в ответ только улыбнулся и сказал: «Учитель определенно пользуется уважением у прилежного ученика». Итак, они отправились в древний город Ур.
\vs p133 9:2 Иисуса очень интересовала древняя история Ура, места, где родился Авраам, и увлекли развалины и традиции Сузы настолько, что Гонод и Ганид остались в этих местах целых три недели, чтобы у Иисуса было достаточно времени для исследований, а также затем, чтобы воспользоваться дополнительной возможностью уговорить его поехать с ними в Индию.
\vs p133 9:3 В Уре у Ганида была продолжительная беседа с Иисусом о том, какая разница между знанием, мудростью и истиной. Его очаровало высказывание иудейского мудреца: «Главное --- мудрость: поэтому постигай мудрость. Всем своим стремлением к знанию обретай понимание. Высоко цени мудрость, и она возвысит тебя. Она прославит тебя, если ты прилепишься к ней».
\vs p133 9:4 \P\ Наконец настал день расставания. Все старались не унывать, особенно юноша, но это было совсем не просто. Из глаз текли слезы, но сердца были полны отваги. Прощаясь с учителем, Ганид сказал: «Прощай, Учитель, но прощай не навсегда. Когда я снова вернусь в Дамаск, я тебя обязательно найду. Я люблю тебя, ибо думаю, что Отец Небесный чем\hyp{}то на тебя похож; по крайней мере, я знаю: ты очень похож на то, что рассказывал мне о нем. Я буду помнить все, чему ты меня учил, но самое главное --- я никогда не забуду тебя». Затем сказал отец: «Прощай, великий учитель, ты сделал нас лучше и помог нам узнать Бога». Иисус ответил: «Мир с вами, и да пребудет с вами благословение Отца Небесного». Иисус стоял на берегу и смотрел, как небольшая лодка уносит Ганида и Гонода к их кораблю, стоящему на якоре. Итак, в Харране Учитель расстался с друзьями из Индии, расстался, чтобы в этом мире не увидеть их больше никогда; не суждено было и им узнать в этом мире, что человек, позднее явившийся миру как Иисус из Назарета, когда\hyp{}то был тем самым другом и учителем Иешуа, с которым они только что расстались.
\vs p133 9:5 В Индии Ганид стал влиятельным человеком, достойным последователем своего выдающегося отца, и, путешествуя по чужим странам, распространял великие истины, которые узнал от Иисуса, своего возлюбленного учителя. Позднее Ганид услышал о странном учителе в Палестине, закончившем свой путь на кресте, и увидел сходство евангелия сего Сына Человеческого с учением своего еврейского наставника, но так и не догадался, что тот и другой были одним лицом.
\vs p133 9:6 \P\ Так закончилась глава в жизни Сына Человеческого, которую можно назвать так --- \bibemph{миссия учителя Иешуа.}
