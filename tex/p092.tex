\upaper{92}{Последующая эволюция религии}
\author{Мелхиседек}
\vs p092 0:1 Человек имел религию естественного происхождения как часть своего эволюционного опыта задолго до того, как на Урантии были получены какие\hyp{}либо систематические откровения. Но эта религия \bibemph{естественного} происхождения уже сама по себе была результатом способностей человека, превосходящих способности животных. Эволюционная религия постепенно возникала на протяжении тысячелетий человеческого опыта при помощи следующих сил, действующих изнутри и наложивших свой отпечаток на дикарей, варваров и цивилизованных людей:
\vs p092 0:2 \ublistelem{1.}\bibnobreakspace \bibemph{Помощник почитания ---} появление в животном сознании сверхживотных потенциалов восприятия реальности. Это можно назвать первоначальным человеческим инстинктивным ощущением Божества.
\vs p092 0:3 \P\ \ublistelem{2.}\bibnobreakspace \bibemph{Помощник мудрости ---} проявление в почитающем Бога разуме тенденции направлять поклонение по высшим каналам выражения и к постоянно расширяющимся представлениям о божественной реальности.
\vs p092 0:4 \P\ \ublistelem{3.}\bibnobreakspace \bibemph{Святой Дух ---} это изначальное дарование сверхразума, и он неизменно является в каждой настоящей человеческой личности. Это служение разуму, стремящемуся к поклонению и жаждущему мудрости, дает возможность актуализовать постулат о продолжении существования в посмертии человека как теологическую концепцию и как реальный и фактический опыт личности.
\vs p092 0:5 \P\ Согласованного действия этих трех божественных помощников вполне достаточно для того, чтобы началось и продолжилось развитие эволюционной религии. К этим действующим силам позже добавляются Настройщики Мысли, серафимы и Дух Истины, и все они ускоряют процесс религиозного развития. Эти силы давно действуют на Урантии и продолжат действовать здесь до тех пор, пока эта планета остается обитаемой сферой. Значительная часть потенциала этих божественных сил никогда еще не имела возможности выразиться; значительная часть будет раскрыта в грядущие века, когда религия смертных ступень за ступенью будет подниматься к небесным высотам ценностей моронтии и духовной истины.
\usection{1. Эволюционная природа религии}
\vs p092 1:1 Эволюция религии восходит к древним страхам и боязни призраков и далее проходит через многие последовательные этапы развития, включая попытки сначала принуждать духов, а затем склонять их к содействию. Из племенных фетишей развились тотемы и племенные боги; магические формулы превратились в современные молитвы. Обрезание, бывшее вначале жертвой, стало гигиенической процедурой.
\vs p092 1:2 На протяжении дикого детства рас религия развивалась от почитания природы к почитанию духов\hyp{}призраков и далее к фетишизму. На заре цивилизации человечество отдавалось во власть более мистических и символических верований, тогда как сейчас, с приближением зрелости, оно становится способным воспринимать настоящую религию, даже начала откровения самой истины.
\vs p092 1:3 Религия возникает как биологическая реакция разума на духовные верования и на окружающую среду; это самое устойчивое и неизменное, что есть у народа. Во все века религия --- это адаптация общества к тому, что окутано тайной. Как социальный институт она включает в себя обряды, символы, культы, священные тексты, алтари, святыни и храмы. Обычными для всех религий являются святая вода, реликвии, фетиши, талисманы, церковные облачения, колокола, барабаны и духовенство. И полностью эволюционную религию невозможно совершенно отделить от магии или колдовства.
\vs p092 1:4 Таинственность и сила всегда возбуждали религиозные чувства и страхи, а эмоции всегда действовали как мощный определяющий фактор в их развитии. Страх всегда был основным религиозным стимулом. Страх создает богов эволюционной религии и порождает религиозные обряды первобытных верующих. С прогрессом цивилизации страх смягчается благоговением, восхищением, уважением и состраданием, а затем еще раскаянием и покаянием.
\vs p092 1:5 Согласно учению одного азиатского народа, «Бог есть великий страх»; это порождение чисто эволюционной религии. Иисус, откровение высочайшего типа религиозной жизни, возвестил, что «Бог --- это любовь».
\usection{2. Религия и нравы}
\vs p092 2:1 Религия --- самый стойкий и непоколебимый из всех человеческих институтов, но она все\hyp{}таки медленно адаптируется к переменам в обществе. Рано или поздно эволюционная религия начинает отражать измененные нравы, на которые, в свою очередь, могла влиять религия откровения. Медленно, неуклонно, но неохотно, религия (богопоклонение) следует по стопам мудрости --- знания, управляемого опирающимся на опыт разумом и озаряемого божественным откровением.
\vs p092 2:2 Религия прочно придерживается нравов; что \bibemph{было ---} то древнее и считается священным, священное. По этой и никакой другой причине каменные орудия долго еще сохранялись в эпоху бронзы и железа. Писание гласит: «И когда будете возводить мне алтарь из камня, не стройте из тесаного камня, ибо если, сооружая его, используете свои орудия, то тем оскверните его». Даже в наши дни индуисты разжигают огонь алтаря с помощью примитивного бура для добывания огня. В русле эволюционной религии новшества считались кощунственными. Святые дары должны состоять не из новых и произведенных продуктов, а из самой примитивной пищи: «Мясо, жаренное на огне, и пресный хлеб с горькими травами». Общественные обычаи всех типов и даже судубные процедуры упорно придерживаются старых форм.
\vs p092 2:3 Когда современный человек изумляется обилию в священных текстах разных религий того, что можно счесть непристойным, ему следует на минуту задуматься о том, что ушедшие поколения боялись отбрасывать все, что их предки считали священным и божественным. Многое из того, что одним поколением может рассматриваться как непристойное, предшествующие поколения считали частью принятых у них нравов и даже одобряемых религиозных ритуалов. Многочисленные религиозные противоречия были вызваны нескончаемыми попытками примирить старые, но достойные порицания традиции с выдвинутыми новыми понятиями, найти правдоподобные теории, оправдывающие увековечение в вероисповедании древних и устаревших обычаев.
\vs p092 2:4 Но глупо пытаться слишком резко ускорять религиозный рост. Раса или нация может усвоить из какой\hyp{}либо развитой религии только то, что более или менее сообразно и совместимо с ее эволюционным развитием на данный момент и ее способностью к адаптации. К числу факторов, определяющих направление и ход религиозной эволюции, относятся социальные, климатические, политические и экономические условия. Религия, то есть эволюционная религия, не определяет морали общества; наоборот, формы религии продиктованы расовой моралью.
\vs p092 2:5 Человеческие расы лишь поверхностно принимают непривычную и новую религию; фактически, они приспосабливают ее к своим нравам и старым верованиям. Это прекрасно подтверждает пример некоего новозеландского племени, священнослужители которого, номинально приняв христианство, заявили, что получили непосредственно от Гавриила откровение, гласящее, что это самое племя стало избранным Богом народом, и повелевающее, чтобы им было разрешено свободно позволять себе распутные половые связи и следовать своим прочим многочисленным старым и достойным порицания обычаям. И все эти новообращенные христиане немедленно приняли этот новый и менее строгий вариант христианства.
\vs p092 2:6 Религия в то или иное время разрешала всяческие противоречащие и несовместимые с ней поступки, одобряла практически все то, что сейчас считается аморальным или греховным. Совесть, не умудренная опытом и не подкрепленная разумом, никогда не была и никогда не может быть надежным и безошибочным ориентиром для человеческого поведения. Совесть не является божественным голосом, обращенным к человеческой душе. Это просто общая сумма моральных и этических нравов, присущих какой\hyp{}либо данной стадии существования; она просто отражает то, что представляется людям идеальной реакцией на каждую данную совокупность обстоятельств.
\usection{3. Природа эволюционной религии}
\vs p092 3:1 Изучение человеческой религии --- это исследование реликтовых элементов обществ прошлых эпох. Нравы антропоморфных богов являются точным отражением морали тех людей, которые создали этих богов. Древние религии и мифология верно воссоздают представления и традиции давно исчезнувших народов. Эти старые культовые обычаи продолжают существовать параллельно с новыми экономическими отношениями и общественными эволюционными изменениями и, конечно, вступают в серьезные противоречия с ними. Отдельные пережитки культа представляют точную картину прошлых религий различных рас. Всегда помните, что культы создаются не для открытия истины, но для распространения лежащих в их основе религиозных вероучений.
\vs p092 3:2 Религия всегда в значительной мере --- это обряды, ритуалы, соблюдения правил, церемоний и догм. Обычно она попадала под пагубное влияние вечно приносящего вред заблуждения --- ложной идеи об избранном народе. Все основные религиозные понятия --- инкарнация, вдохновение, откровение, искупительная жертва, покаяние, воздаяние за грехи, заступничество, жертвоприношение, молитва, исповедь, богопочитание, продолжение существования в посмертии, таинство, обряд, индульгенция, спасение, искупление, завет, нечистота, очищение, пророчество, изначальный грех --- восходят к древнейшим временам первобытного страха перед призраками.
\vs p092 3:3 \P\ Примитивная религия --- это не что иное, как борьба за физическое существование, расширившаяся и включившая также и борьбу за существование загробное. Правила, предписываемые такой верой, отражали распространение борьбы за выживание и на сферу воображаемого мира духов\hyp{}призраков. Но будьте осторожны, когда испытываете искушение подвергнуть критике эволюционную религию. Помните, это --- то, что \bibemph{уже свершилось;} это исторический факт. И, кроме того, помните, что сила любой идеи определяется не ее достоверностью или истинностью, а притягательностью ее для людей.
\vs p092 3:4 \P\ Эволюционная религия не предусматривает изменений или реформ; в отличие от науки она не обеспечивает собственной прогрессивной самокоррекции. Эволюционная религия внушает уважение потому, что ее приверженцы верят, что это и есть \bibemph{Истина;} «вера, некогда данная святым», по идее, должна быть окончательной и непогрешимой. Культ противостоит всякому развитию потому, что реальный прогресс, наверняка, изменит или уничтожит сам культ; поэтому реформы всегда должны навязываться ему силой.
\vs p092 3:5 Лишь два фактора могут изменять или поднимать на более высокий уровень догматы естественной религии: давление со стороны постепенно прогрессирующих нравов и периодические озарения, вызываемые эпохальными откровениями. И нет ничего удивительного в том, что прогресс шел медленно; в древние времена быть прогрессивным или изобретательным значило быть убитым как колдун. Культ медленно развивается на протяжении жизни поколений и целых эпох. Но все же движение вперед происходит. Эволюционная вера в призраков заложила основу для философии религии откровения, которая со временем уничтожит суеверия, лежавшие в ее основе.
\vs p092 3:6 Религия во многих отношениях препятствовала общественному развитию, но без религии не было бы ни твердой морали, ни этики, ни достойной цивилизации. Религия породила многочисленные направления внерелигиозной культуры: скульптура началась с изготовления идолов, архитектура --- со строительства храмов, поэзия --- с заклинаний, музыка --- с религиозных песнопений, разыгрываемые, чтобы попросить водительства духов, действа положили начало драме, а ежегодные религиозные праздники --- танцам.
\vs p092 3:7 Но, обращая внимание на тот факт, что религия играла основную роль в развитии и сохранении цивилизации, следует отметить и то, что естественная религия в значительной степени наносила вред и являлась помехой для той самой цивилизации, которую она, с другой стороны, взращивала и сохраняла. Религия мешала производственной деятельности и экономическому развитию; она непроизводительно расходовала массу труда и средств; не всегда играла положительную роль в сфере семейных отношений; не способствовала должным образом миру и доброй воле; иногда с пренебрежением относилась к образованию и сдерживала развитие науки; чрезмерно обедняла жизнь ради мнимого обогащения смерти. На эволюционной религии, человеческой религии, действительно лежит вина за все эти и многие другие ошибки, заблуждения и просчеты; тем не менее, она поддерживала этику в культуре, цивилизованную нравственность и сплоченность общества и дала возможность последующим религиям откровения возместить все эти многочисленные недостатки эволюционной религии.
\vs p092 3:8 \P\ Эволюционная религия была самым дорогостоящим, но ни с чем не сравнимым по эффективности человеческим институтом. Существование человеческой религии может быть оправдано только в свете эволюционной цивилизации. Если бы человек не явился высшим результатом эволюции животных, то такой путь религиозного развития не имел бы оправдания.
\vs p092 3:9 \P\ Религия способствовала накоплению капитала; она поощряла определенные виды трудовой деятельности; наличие свободного времени у жрецов способствовало развитию искусства и знаний; в конечном счете, человечество много приобрело в результате этих прошлых ошибок в этических методах. Шаманы, как честные, так и нечестные, стоили ужасно дорого, но все эти затраты себя полностью оправдали. Ученые профессии и сама наука возникли из недр паразитического жречества. Религия взращивала цивилизацию и обеспечивала непрерывность существования общества; во все времена она выполняла функцию полиции морали. Религия обеспечила ту человеческую дисциплину и самоконтроль, которые сделали возможной \bibemph{мудрость.} Религия --- это мощный эволюционный кнут, который безжалостно гонит ленивое и страждущее человечество от его естественного состояния интеллектуальной инертности вперед и вверх к высшим ступеням разума и мудрости.
\vs p092 3:10 И это священное наследие периода восхождения от животного --- эволюционная религия должна вечно продолжать очищаться и облагораживаться постоянной цензурой религии откровения и огненным горнилом подлинной науки.
\usection{4. Дар откровения}
\vs p092 4:1 Откровение --- эволюционно, но всегда прогрессирует. На протяжении веков мировой истории религиозные откровения становятся все более глубокими, и каждое последующее несет еще больше света. Откровения призваны классифицировать и подвергать цензуре последовательно возникавшие эволюционные религии. Но если откровение должно возвышать и поднимать эволюционные религии на более высокий уровень, тогда эти божественные послания должны являть учения, не слишком далекие от образа мыслей и воззрений той эпохи, когда они даны. Таким образом, откровение всегда связано с уровнем эволюции, как это и должно быть. Религия откровения всегда должна быть ограничена способностью человека к восприятию.
\vs p092 4:2 Независимо от того, каковы ее явные свойства и происхождение, религии откровения всегда характеризуются верой в некое Божество, являющееся конечной ценностью, и в некую концепцию продолжения существования личности в посмертии.
\vs p092 4:3 Эволюционная религия основана на чувствах, а не на логике. Это реакция человека на веру в гипотетический мир духов\hyp{}призраков --- человеческая вера\hyp{}рефлекс, вызванная осознанием существования неизвестного и страхом перед этим неизвестным. Религия же откровения дается реальным духовным миром; это реакция превосходящего интеллект космоса на жажду смертных верить во вселенские Божества и полагаться на них. Эволюционная религия представляет собой картину движения человечества ощупью и окольными путями в поисках истины; религия же откровения и \bibemph{является} этой самой истиной.
\vs p092 4:4 \P\ Было много случаев религиозного откровения, но только пять из них имели эпохальное значение. А именно:
\vs p092 4:5 \ublistelem{1.}\bibnobreakspace \bibemph{Даламатские учения.} Истинное представление о Первоисточнике и Центре впервые раскрыли на Урантии сто облеченных в плоть членов штата Принца Калигастии. Это расширяющееся откровение Божества продержалось более трехсот тысяч лет, пока не было внезапно прервано расколом на планете и крахом системы обучения. Если не считать деятельности Вана, то от воздействия даламатского откровения во всем мире практически не осталось и следа. Даже Нодиты забыли эту истину ко времени прихода Адама. Из всех, кто воспринял учения ста, дольше всего их сохраняли люди красной расы, но представление о Великом Духе было в американо\hyp{}индейской религии весьма смутным, пока соприкосновение с христианством существеннейшим образом не прояснило и не усилило его.
\vs p092 4:6 \P\ \ublistelem{2.}\bibnobreakspace \bibemph{Эдемические учения.} Адам и Ева снова изложили эволюционным народам представление об Отце Всего Сущего. Крушение первого Эдема оборвало ход Адамова откровения даже раньше, чем оно в полной мере началось. Но прерванные учения Адама были продолжены сифитскими священниками, и некоторые из этих истин не прошли для мира даром. Учения Сифитов сыграли решающую роль в определении направления всей религиозной эволюции Леванта. Но к 2500 году до н. э. человечество, в основном, забыло откровение, дарованное во времена Эдема.
\vs p092 4:7 \P\ \ublistelem{3.}\bibnobreakspace \bibemph{Мелхиседек из Салима.} Этот Сын\hyp{}Спаситель из Небадона положил начало третьему откровению истины на Урантии. Главными заповедями его учения были \bibemph{доверие} и \bibemph{вера.} Он учил доверять всемогущему благодеянию Бога и объяснял, что человек заслуживает благосклонность Бога своей верой в него. Его учения постепенно смешались с верованиями и обычаями разных эволюционных религий и, в конце концов, из них развились теологические системы, которые существовали на Урантии к началу первого тысячелетия нашей эры.
\vs p092 4:8 \P\ \ublistelem{4.}\bibnobreakspace \bibemph{Иисус из Назарета.} Христос\hyp{}Михаил в четвертый раз дал Урантии понятие о Боге как об Отце Всего Сущего, и с тех пор это учение в целом сохранилось. Главной сутью его учений были \bibemph{любовь} и \bibemph{служение,} исполненное любви богопочитание, которое сотворенный сын добровольно совершает в знак признательности и в ответ на исполненное любви пастырство Бога, своего Отца; добровольное служение своим собратьям, которому посвящают себя эти сотворенные сыновья, радостно сознавая, что, делая это, они служат одновременно и Богу Отцу.
\vs p092 4:9 \P\ \ublistelem{5.}\bibnobreakspace \bibemph{Тексты книги Урантии.} Тексты, к числу которых относится и данный текст, есть самое новое возвещение истины смертным Урантии. Эти тексты отличаются от всех предыдущих откровений тем, что они не плод трудов одной вселенской личности, но собрание изложенного многими существами. Но ни одно откровение, не достигающее уровня знаний Отца Всего Сущего, не может быть полным. Все прочие небесные служения --- неокончательные, временные и практически адаптированы к локальным условиям данного времени и пространства. Хотя такого рода признания, возможно, могут уменьшить силу и авторитет всех откровений, но на Урантии наступило время, когда такие прямые утверждения становятся желательными, даже несмотря на риск ослабить будущее влияние и авторитет этого, самого нового из всех откровений истины, данных смертным расам Урантии.
\usection{5. Великие религиозные лидеры}
\vs p092 5:1 Эволюционная религия представляет богов в человеческом образе; религия же откровения учит людей, что они сыновья Бога и даже созданы по образу и подобию Божества; в верованиях, представляющих собой синтез учений, данных через откровение, и продуктов эволюции, представление о Боге складывается из:
\vs p092 5:2 \ublistelem{1.}\bibnobreakspace Ранее существовавших представлений эволюционных культов.
\vs p092 5:3 \ublistelem{2.}\bibnobreakspace Высоких идеалов религии откровения.
\vs p092 5:4 \ublistelem{3.}\bibnobreakspace Личных взглядов великих религиозных лидеров --- пророков и учителей человечества.
\vs p092 5:5 \P\ В большинстве случаев наступление великой религиозной эпохи ознаменовывалось жизнью и учениями какой\hyp{}то выдающейся личности; на протяжении всей истории именно появление лидера приводило к возникновению большинства заслуживающих внимания моральных движений в истории. И люди всегда были склонны преклоняться перед учителем, даже в ущерб его учениям; чтить его личность, даже забывая об истинах, которые он возвещал. И это не лишено оснований; сердце эволюционного человека инстинктивно жаждет помощи свыше и из мира иного. Эта жажда призвана предвосхитить появление на земле Планетарного Принца и последующих Материальных Сынов. На Урантии человек был лишен этих сверхчеловеческих лидеров и правителей, и поэтому он стремится восполнить эту потерю, облекая своих человеческих лидеров легендами об их сверхчеловеческом происхождении и сверхъестественной жизни.
\vs p092 5:6 Многие народы считали, что их лидеры родились от девственниц; описание их жизни щедро украшено сверхъестественными эпизодами, и соответствующие группы людей всегда ожидают их возвращения. В Центральной Азии соплеменники до сих пор ожидают возвращения Чингис\hyp{}хана; в Тибете, Китае и Индии --- Будды; в исламе --- Магомета; среди американских индейцев --- Хесунанина Онамоналонтона; евреи, в общем\hyp{}то, ожидали возвращения Адама в качестве мирского правителя. В Вавилоне легенда об Адаме, представление о сыне Бога, связующем звене между человеком и Богом, воплотилась в боге Мардуке. После появления на земле Адама представление о так называемых сынах Бога широко распространилось среди рас мира.
\vs p092 5:7 Но независимо от того суеверного трепета, который они часто вызывали, очевидно, что личности этих учителей были временными точками опоры, необходимыми для рычагов истины откровения, чтобы развивать мораль, философию и религию человечества.
\vs p092 5:8 На протяжении длившейся миллионы лет истории Урантии существовали сотни и сотни религиозных лидеров, от Онагара до Гуру Нанака. За это время было много подъемов и спадов религиозной истины и духовной веры, и каждый раз расцвет урантийской религии был изначально связан с жизнью и учениями какого\hyp{}нибудь религиозного лидера. Говоря об учителях недавних времен, может быть, полезно сгруппировать их в соответствии с семью основными религиозными эпохами на Урантии после Адама:
\vs p092 5:9 \ublistelem{1.}\bibnobreakspace \bibemph{Сифитский период.} Сифитские священники, возрожденные под руководством Амосада, стали великими учителями эпохи после Адама. Они осуществляли свою деятельность во всех землях Андитов, и дольше всего их влияние сохранялось у греков, шумеров и индусов. У последних они продолжают существовать до настоящего времени как брахманы индуистской религии. Сифиты и их последователи никогда до конца не забывали раскрытого Адамом понятия о Троице.
\vs p092 5:10 \P\ \ublistelem{2.}\bibnobreakspace \bibemph{Эра миссионеров\hyp{}Мелхиседеков.} Религия Урантии была возрождена, в огромной степени, усилиями тех учителей, которым эту миссию поручил Махивента Мелхиседек, когда он жил и учил в Салеме почти за две тысячи лет до нашей эры. Эти миссионеры провозглашали, что благосклонность Бога завоевывается верой, и их учения, хотя и не привели непосредственно к появлению каких\hyp{}либо религий, тем не менее, заложили фундамент, на котором последующим учителям истины предстояло создавать религии Урантии.
\vs p092 5:11 \P\ \ublistelem{3.}\bibnobreakspace \bibemph{Эпоха после Мелхиседека.} Хотя в этот период учили и Аменемоп, и Эхнатон, но выдающимся религиозным гением эпохи после Мелхиседека был предводитель группы бедуинов Леванта и основатель еврейской религии --- Моисей. Моисей учил монотеизму. Он говорил: «Услышь, о Израиль, Господь наш Бог един». «Господь есть Бог. Нет другого, кроме него». Он упорно стремился искоренить остатки культа призраков у своего народа, введя даже смертную казнь для совершающих его обряды. Монотеизм Моисея был искажен его последователями, но в более поздние времена они вернулись ко многим его учениям. Величие Моисея --- в его мудрости и дальновидности. У других людей были и более великие концепции Бога, но ни один человек не смог настолько успешно склонить большое число людей принять такие передовые верования.
\vs p092 5:12 \P\ \ublistelem{4.}\bibnobreakspace \bibemph{Шестой век до н. э.} Многие стали возвещать истину в этом столетии, одном из самых великих веков религиозного пробуждения, которые только видела Урантия. Из их числа следует отметить Гаутаму, Конфуция, Лао Цзы, Зороастра и учителей джайнизма. Учения Гаутамы широко распространились в Азии, и миллионы людей почитают его как Будду. Конфуций был для китайской морали тем же, чем Платон --- для греческой философии, и хотя учения и того, и другого оказали влияние на религию, ни один из них, строго говоря, не был религиозным учителем; в Дао у Лао Цзы больше Бога, чем в гуманизме Конфуция или в идеализме Платона. Зороастр, хотя он и испытал сильное влияние широко распространенной концепции двойственного спиритизма, хорошего и плохого, в то же время, несомненно, возвысил идею одного вечного Божества и конечной победы света над тьмой.
\vs p092 5:13 \P\ \ublistelem{5.}\bibnobreakspace \bibemph{Первый век н. э.} Как религиозный лидер Иисус из Назарета начал с культа, основанного Иоанном Крестителем, и развил его, уводя, насколько мог, от существующих запретов и форм. Помимо Иисуса, величайшими учителями этой эпохи были Павел Тарсянин и Филон Александрийский. Их религиозные воззрения сыграли определяющую роль в эволюции веры, носящей имя Христа.
\vs p092 5:14 \P\ \ublistelem{6.}\bibnobreakspace \bibemph{Шестой век н. э.} Магомет основал религию, которая превосходила многие вероисповедания того времени. Она явилась выражением протеста против социальных требований религий иноземцев и против неупорядоченности религиозной жизни его собственного народа.
\vs p092 5:15 \P\ \ublistelem{7.}\bibnobreakspace \bibemph{Пятнадцатый век н. э.} Этот период стал свидетелем двух религиозных движений: подрыва единства христианства на Западе и синтеза новой религии на Востоке. В Европе узаконенное христианство достигло такой степени негибкости, что дальнейшее его развитие стало несовместимо с единством. На Востоке Нанак и его последователи осуществили синтез учений ислама, индуизма и буддизма, создав сикхизм, одну из наиболее развитых религий Азии.
\vs p092 5:16 \P\ Для будущего Урантии, несомненно, будет характерно появление учителей религиозной истины --- Отцовства Бога и братства всех творений. Но надо надеяться, что страстные и искренние усилия этих будущих пророков будут направлены не столько на укрепление межрелигиозных барьеров, сколько на упрочение основанного на духовной вере религиозного братства многочисленных приверженцев различных интеллектуальных теологий, которые так характерны для Урантии, принадлежащей к системе Сатании.
\usection{6. Составные религии}
\vs p092 6:1 Религии Урантии двадцатого века дают интересный материал для исследования социальной эволюции тяги человека к богопочитанию. Многие вероисповедания очень мало изменились со времен культа призраков. Африканские пигмеи не имеют единых религиозных ритуалов, хотя некоторые из них, отчасти, верят в существование мира духов. Они до сих пор остаются на том же уровне, на каком находился первобытный человек в начале эволюции религии. В примитивной религии основной была вера в продолжение существования в посмертии. Идея поклонения личностному Богу --- признак высокого уровня эволюционного развития, даже первой стадии откровения. У даяков возникли лишь самые примитивные религиозные ритуалы. У эскимосов и американских индейцев в обозримом прошлом были очень скудные представления о Боге; они верили в призраков и имели нечеткое понятие о каком\hyp{}то продолжении существования в посмертии. У современных австралийских аборигенов есть только страх перед призраками, боязнь темноты и довольно примитивное почитание предков. У зулусов еще только развивается религия, основанная на страхе перед призраками и принесении жертв. Многие африканские племена не прошли еще стадию фетишей в своем религиозном развитии, если не считать плодов миссионерской деятельности христиан и мусульман. Но некоторые этнические группы издавна придерживались идеи монотеизма, например, древние фракийцы, которые верили также и в бессмертие.
\vs p092 6:2 \P\ На Урантии эволюционная религия и религия откровения развиваются параллельно, взаимодействуя и образуя, таким образом, разнообразные теологические системы, существующие в мире ко времени написания этих текстов. Эти религии, религии Урантии двадцатого века, можно перечислить в следующем порядке:
\vs p092 6:3 \ublistelem{1.}\bibnobreakspace Индуизм --- самый древний.
\vs p092 6:4 \ublistelem{2.}\bibnobreakspace Иудаизм.
\vs p092 6:5 \ublistelem{3.}\bibnobreakspace Буддизм.
\vs p092 6:6 \ublistelem{4.}\bibnobreakspace Конфуцианство.
\vs p092 6:7 \ublistelem{5.}\bibnobreakspace Даосизм.
\vs p092 6:8 \ublistelem{6.}\bibnobreakspace Зороастризм.
\vs p092 6:9 \ublistelem{7.}\bibnobreakspace Шинтоизм.
\vs p092 6:10 \ublistelem{8.}\bibnobreakspace Джайнизм.
\vs p092 6:11 \ublistelem{9.}\bibnobreakspace Христианство.
\vs p092 6:12 \ublistelem{10.}\bibnobreakspace Ислам.
\vs p092 6:13 \ublistelem{11.}\bibnobreakspace Сикхизм --- самый поздний.
\vs p092 6:14 \P\ Самыми развитыми религиями древних времен были иудаизм и индуизм, и каждая из них оказала огромное влияние на весь ход религиозного развития, соответственно, на Западе и на Востоке. Как индусы, так и евреи, верили, что их религия ниспослана свыше и является религией откровения, а все остальные религии представляют собой искажения единственно истинной веры.
\vs p092 6:15 Население Индии состоит из индуистов, сикхов, мусульман и джайнов, и приверженцы каждой из этих религий имеют свои, отличные от других, представления о Боге, человеке и вселенной. В Китае придерживаются даосских и конфуцианских учений; шинтоизм почитают в Японии.
\vs p092 6:16 К числу великих мировых, межрасовых религий относятся иудаизм, буддизм, христианство и ислам. Буддизм распространен от Цейлона и Бирмы до Тибета, Китая и Японии. Он обнаружил способность адаптироваться к национальным нравам многих народов, и в этом отношении с ним может сравниться только христианство.
\vs p092 6:17 Иудаизм содержит философский переход от политеизма к монотеизму; он являет собой эволюционное звено между эволюционными религиями и религиями откровения. Иудеи оказались единственным западным народом, сменившим своих прежних эволюционных богов на религию откровения. Но эта истина не была широко принята вплоть до времен Исаии, который снова учил смешанному представлению о национальном божестве, соединенном с Творцом Всего Сущного: «О Господь сил небесных, Бог Израиля, ты --- Бог, ты один; ты сотворил небеса и землю». Некогда надежда на продолжение существования западной цивилизации опиралась на возвышенные иудейские представления о добре и развитые греческие представления о красоте.
\vs p092 6:18 Христианская религия --- это религия, созданная вокруг жизни и учений Христа, основанная на теологии иудаизма, позже видоизменившаяся в результате усвоения некоторых элементов зороастрийских учений и греческой философии, первоначально сформулированная тремя личностями: Филоном, Петром и Павлом. Со времени Павла она прошла много этапов эволюции и стала настолько западной, что многие неевропейские народы, вполне естественно, рассматривают христианство как чужую религию с чужим Богом и для чуждых им народов.
\vs p092 6:19 Ислам является культурно\hyp{}религиозным связующим звеном между Северной Африкой, Левантом и Юго\hyp{}Восточной Азией. Еврейская теология в сочетании с более поздними христианскими учениями обусловили монотеизм ислама. Учение о Троице оказалось камнем преткновения для последователей Магомета; они не смогли понять доктрину об одном Божестве в трех божественных личностях. Всегда трудно добиться, чтобы эволюционные умы \bibemph{вдруг} приняли высокие истины откровения. Человек --- эволюционное творение и должен обретать свою религию, в основном, эволюционным путем.
\vs p092 6:20 \P\ Почитание предков некогда знаменовало собой несомненный прогресс в религиозной эволюции, но поразительно и достойно сожаления то, что эта примитивная идея до сих пор сохраняется в Китае, Японии и Индии, при том, что там существует множество относительно более прогрессивных явлений, в частности, буддизм и индуизм. На Западе из почитания предков развилось поклонение национальным богам и уважение к национальным героям. В двадцатом веке эта националистическая религия, связанная с почитанием героев, проявляется в виде различного радикального и националистического секуляризма, характерного для многих западных народов и наций. Таких же воззрений, в значительной степени, придерживаются и в крупных университетах, и больших индустриальных городах англоязычных стран. Не слишком отличается от этих концепций и идея, что религия --- это просто «коллективные поиски благой жизни». «Национальные религии» есть не что иное, как возврат к древнему римскому поклонению императору и к шинтоизму --- поклонению императорскому роду.
\usection{7. Дальнейшая эволюция религии}
\vs p092 7:1 Религия никогда не может стать научным фактом. Философия, действительно, может опираться на научный фундамент, но религия всегда будет результатом или эволюции, или откровения, или же различным сочетанием того и другого, как это и есть сегодня в мире.
\vs p092 7:2 Нельзя изобрести новые религии; они возникают в результате или эволюции, или \bibemph{внезапного откровения.} Все новые эволюционные религии --- это просто более развитые формы старых верований, результат их адаптации и приспособления к новым условиям. Старое не перестает существовать; оно сливается с новым, как в случае со сикхизмом, который появился и расцвел на почве и из форм индуизма, буддизма, ислама и других современных культов. Примитивная религия была очень демократичной; дикарь легко заимствовал и давал. Только с возникновением религии откровения появляется автократичный и нетерпимый теологический эгоцентризм.
\vs p092 7:3 Все многочисленные религии Урантии хороши в той степени, в какой они приводят человека к Богу и к осознанию Отца. Любая группа верующих впадает в заблуждение, когда считает, что ее вера --- это \bibemph{Истина;} такие взгляды свидетельствуют скорее о теологическом высокомерии, чем о твердой вере. На Урантии нет такой религии, которая не могла бы с пользой для себя изучать и воспринимать лучшие из истин, содержащихся в любом другом вероисповедании, ибо истина содержится во всех. Верующие поступали бы лучше, если бы заимствовали самое лучшее из живой духовной веры своих соседей вместо того, чтобы осуждать худшие из сохранившихся у них суеверий и отживших ритуалов.
\vs p092 7:4 Все эти религии возникли вследствие различия интеллектуальных реакций человека на одно и то же духовное водительство. Нет надежды, что они когда\hyp{}либо достигнут единообразия вероучений, догматов и ритуалов --- это интеллектуальная сфера; но они могут и когда\hyp{}нибудь достигнут единения в истинном почитании Отца всех, ибо это сфера духовная, и вовеки истинно, что в духе все люди равны.
\vs p092 7:5 \P\ Примитивная религия была, главным образом, осознанием материальных ценностей, но цивилизация возвышает религиозные ценности, ибо истинная религия --- это посвящение себя служению самым значительным и верховным ценностям. С развитием религии этика становится философией морали, а мораль становится самодисциплиной в соответствии с нормами высочайших значений и верховных ценностей --- божественных и духовных идеалов. Религия становится спонтанным и совершенным посвящением себя, становится живым опытом верности в любви.
\vs p092 7:6 Показателями качества религии являются:
\vs p092 7:7 \ublistelem{1.}\bibnobreakspace Уровень ценностей --- чему предан верующий.
\vs p092 7:8 \ublistelem{2.}\bibnobreakspace Глубина значений --- способность человека идеалистично воспринимать эти высшие ценности.
\vs p092 7:9 \ublistelem{3.}\bibnobreakspace Сила посвящения --- степень преданности этим божественным ценностям.
\vs p092 7:10 \ublistelem{4.}\bibnobreakspace Беспрепятственное движение личности вперед по этому космическому пути идеалистической духовной жизни, осуществления своего сыновства по отношению к Богу и вечного, и бесконечного поступательного обретения гражданства во вселенной.
\vs p092 7:11 \P\ Религиозные значения развиваются в сознании ребенка, когда он переносит свои представления о всемогуществе с родителей на Бога. И весь религиозный опыт такого ребенка во многом зависит от того, преобладал ли в его отношениях с родителями страх или же любовь. Рабам всегда было особенно трудно трансформировать свой страх перед хозяином в идею божественной любви. Цивилизация, наука и развитая религия должны освободить человечество от этих страхов, родившихся из ужаса перед явлениями природы. И так же большая просвещенность должна освободить образованных смертных от всякой зависимости от посредников при общении с Божеством.
\vs p092 7:12 Неизбежно существование этих промежуточных этапов идолопоклоннических колебаний при переходе от поклонения человеческому и видимому к поклонению божественному и невидимому, но эти этапы могут быть сокращены, если осознать, что облегчению этого перехода служит пребывающий внутри человека божественный дух. Тем не менее, глубокое влияние на человека оказывает не только его представление о Божестве, но и образ избранных и почитаемых им героев. Очень жаль, что почитатели божественного и воскресшего Христа не замечают человека --- храброго и мужественного героя --- Иешуа бен Иосифа.
\vs p092 7:13 \P\ Современный человек в достаточной мере обладает религиозным сознанием, но его религиозные обычаи подрываются и ставятся под сомнение ускоренными преобразованиями в обществе и беспрецедентным развитием науки. Думающие мужчины и женщины хотят пересмотра религии, и эта потребность заставит религию произвести переоценку собственных ценностей
\vs p092 7:14 Перед современным человеком встает задача в течение одного поколения внести больше изменений в систему человеческих ценностей, чем было внесено за двести лет. И все это оказывает влияние на отношение в обществе к религии, поскольку религия --- это образ жизни, равно как и способ мышления.
\vs p092 7:15 \P\ Истинная религия всегда должна быть одновременно незыблемой основой и путеводной звездой для всех устойчивых цивилизаций.
\vs p092 7:16 [Представлено Мелхиседеком из Небадона.]
