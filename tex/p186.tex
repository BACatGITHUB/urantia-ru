\upaper{186}{Перед распятием}
\author{Комиссия срединников}
\vs p186 0:1 Когда Иисуса его обвинители вели к Ироду, Учитель повернулся к апостолу Иоанну и сказал: «Иоанн, ты больше ничего не можешь сделать для меня. Пойди к моей матери и приведи ее, чтобы мы увиделись с ней прежде, чем я умру». Услышав просьбу Учителя, Иоанн, хотя и не желал оставлять его одного среди врагов, поспешил в Вифанию, где в доме у Марфы и Марии, сестер Лазаря, которого Иисус воскресил из мертвых, собралась и ждала вестей вся семья Иисуса.
\vs p186 0:2 Несколько раз в это утро вестники приносили Марфе и Марии известия о ходе суда над Иисусом. Но семья Иисуса добралась до Вифании всего за несколько минут до прихода Иоанна, передавшего пожелание Иисуса увидеться со своей матерью прежде, чем его предадут смерти. После того, как Иоанн Зеведеев рассказал им обо всем, что произошло с момента ареста Иисуса в полночь, Мария, его мать, в сопровождении Иоанна тут же отправилась к своему старшему сыну. К тому времени, когда Мария и Иоанн добрались до города, Иисус под конвоем римских солдат, которые должны были его распять, уже дошел до Голгофы\ldots
\vs p186 0:3 Когда Мария, мать Иисуса, собралась идти с Иоанном к своему сыну, его сестра Руфь не захотела остаться в Вифании с остальными родственниками. Увидев, что она решила сопровождать мать, с ней пошел ее брат Иуда. Остальные члены семьи Учителя под опекой Иакова остались в Вифании, и вестники Давида Зеведеева почти ежечасно приносили им известия о ходе этого ужасного события --- предания смерти их старшего брата, Иисуса из Назарета.
\usection{1. Конец Иуды Искариота}
\vs p186 1:1 В эту пятницу около половины девятого утра Пилат закончил слушание дела Иисуса, и Учитель был передан под конвой римским солдатам, которые должны были распять его. Как только римляне забрали Иисуса, начальник еврейской стражи и его люди отправились обратно в храм, в свой штаб. Первосвященник и остальные члены синедриона сразу вслед за стражниками пошли к своему обычному месту встреч, в расположенный в храме зал из тесаного камня. Здесь уже собрались и другие члены синедриона, ожидающие известий о том, как обошлись с Иисусом. Пока Каиафа рассказывал членам синедриона о суде над Иисусом и вынесенном ему приговоре, к ним пришел Иуда, чтобы получить свою награду за то участие, которое он принял в аресте Учителя и судилище над ним.
\vs p186 1:2 Все эти евреи испытывали отвращение к Иуде; они смотрели на предателя с крайним презрением. Во время суда Каиафы над Иисусом и когда Иисус предстал перед Пилатом, совесть Иуды терзало сознание своего предательского поведения. И в то же время он все больше разочаровывался относительно награды, которую ему предстояло получить за предательство Иисуса. Ему не понравилась холодность и отчужденность еврейских властей предержащих; тем не менее, он ожидал щедрой награды за свое подлое поведение. Он мечтал, как его призовут на общее собрание всего синедриона, где он будет выслушивать похвалы и получит подобающие награды в знак признательности за великую услугу, как ему хотелось верить, оказанную им своей нации. Поэтому вообразите величайшее изумление этого себялюбивого предателя, когда слуга первосвященника, похлопав его по плечу, отозвал из зала и сказал: «Иуда, мне поручили заплатить тебе за предательство Иисуса. Вот твоя награда». И сказав это, слуга Каиафы передал Иуде мешочек с тридцатью серебряными монетами --- в то время цена хорошего здорового раба.
\vs p186 1:3 Иуда был потрясен, ошеломлен. Он было бросился обратно в зал, но его не впустил привратник. Он хотел обратиться к синедриону, но его не пожелали даже видеть. Иуда не мог поверить, что правители евреев могли позволить ему предать своих друзей и Учителя, а взамен предложить награду --- тридцать серебряных монет. Он был унижен, совершенно разочарован и абсолютно раздавлен. Он шел прочь от храма, словно в трансе. Машинально опустил мешочек с деньгами в свой глубокий карман, тот самый карман, в котором так долго носил апостольскую казну. И побрел из города вслед за толпами, направлявшимися смотреть на распятие.
\vs p186 1:4 Издали Иуда увидел, как поднимают крест, к которому гвоздями был прибит Иисус, и, увидев это, он бросился обратно к храму и, прорвавшись мимо привратника, оказался перед синедрионом, который все еще продолжал свое заседание. Изменник еле дышал, совершенно обезумел и с трудом выдавил слова: «Я согрешил, предав кровь невинную. Вы оскорбили меня. В награду за мою службу вы дали мне деньги --- цену раба. Я раскаиваюсь в содеянном; вот ваши деньги. Я хочу избавиться от вины за содеянное».
\vs p186 1:5 Выслушав Иуду, правители евреев стали насмехаться над ним. Один из них, сидевший поблизости от стоящего Иуды, жестом велел ему удалиться из зала и сказал: «Твой Учитель уже предан смерти римлянами, а что касается твоей вины --- что нам за дело до нее? Это твоя забота --- пошел прочь!»
\vs p186 1:6 Выйдя из зала синедриона, Иуда достал из мешочка тридцать серебреников, швырнул их, и они разлетелись по полу храма. Когда изменник вышел из храма, он был вне себя. Только теперь Иуда стал постигать истинную сущность греха. Исчезли все очарование, вся прелесть и упоительность греха. Теперь грешник остался один на один с обвинительным приговором своей утратившей иллюзии и разочарованной души. Грех был чарующ и манящ в момент совершения, но теперь надлежало встретиться с лишенными всякой романтики, неприкрытыми последствиями.
\vs p186 1:7 Этот бывший посланец царства небесного на земле шел теперь по улицам Иерусалима покинутый и одинокий. Его отчаяние было ужасным и почти безысходным. Он брел все дальше по городу, вышел за его стены, спустился в страшно уединенное место долины Геенна, вскарабкался на крутой склон скалы, взял пояс от своего плаща, привязал один его конец к небольшому деревцу, другим концом обвязал свою шею и прыгнул в пропасть, повиснув над ней. Но прежде, чем наступила смерть, завязанный его дрожащими пальцами узел развязался, и предатель, упав вниз на острые камни, разбился вдребезги\ldots
\usection{2. Отношение Учителя к происходящему}
\vs p186 2:1 Когда Иисуса схватили, он знал, что на земле деятельность его в обличии смертного во плоти закончена. Он прекрасно понимал, какой смертью умрет, и его мало заботили подробности так называемого суда над ним.
\vs p186 2:2 Представ перед судом синедриона, Иисус не стал оспаривать показания лжесвидетелей. Лишь на один вопрос он всегда отвечал, задавал ли его друг или враг, а именно, на вопрос о его природе и о божественности его миссии на земле. Когда его спрашивали, действительно ли он Сын Бога, он неизменно давал ответ. Он упорно отказывался говорить в присутствии любопытного и безнравственного Ирода. Перед Пилатом он говорил только тогда, когда полагал, что его слова могут помочь Пилату или любому другому искреннему человеку лучше узнать истину. Иисус учил своих апостолов, что бесполезно метать жемчуг перед свиньями, и теперь сам руководствовался тем, чему учил. Его поведение в этот момент являло собой пример терпеливой смиренности человеческой природы вкупе с величественным и безмолвным достоинством божественной природы. Он был готов обсудить с Пилатом любой вопрос, связанный с выдвинутыми против него политическими обвинениями, --- любой вопрос, который, по его мнению, входил в компетенцию правителя.
\vs p186 2:3 Иисус был убежден, что воля Отца такова, что ему следует подчиниться естественному и нормальному ходу земных событий, как должны подчиняться им все прочие смертные, и поэтому он отказался использовать даже свои чисто человеческие способности, например, убедительное красноречие, которое могло бы повлиять на исход интриг социально близоруких и духовно слепых своих смертных собратьев. Хотя Иисус жил и умер на Урантии, весь его человеческий жизненный путь от начала и до конца был наглядным примером, имеющим цель повлиять на всю созданную и непрерывно поддерживаемую им вселенную и наставить ее.
\vs p186 2:4 \pc Близорукие евреи неподобающим образом шумно требовали смерти Учителя, в то время как он стоял там, в величественным безмолвии взирая на картину смерти нации --- кровь от крови его земного отца.
\vs p186 2:5 \pc Иисус воспитал в себе такие качества человеческого характера, которые позволяли сохранять самообладание и достоинство перед лицом непрерывных и беспричинных нападок. Его невозможно было запугать. Когда слуга Анны первым обрушился на него, он лишь заметил, что будет уместнее выслушать свидетелей, которые могут давать показания против него должным образом.
\vs p186 2:6 Небесные силы, от начала и до конца взиравшие свыше на так называемый суд Пилата, не могли удержаться от того, чтобы не распространить по вселенной описание сцены «Пилат перед судом Иисуса».
\vs p186 2:7 На суде Каиафы, когда все лжесвидетельства рухнули, Иисус без колебаний ответил на вопрос первосвященника и, таким образом, сам предоставил им свидетельство, которое те желали получить для обвинения его в богохульстве.
\vs p186 2:8 Учитель ни разу не проявил ни малейшего интереса к добронамеренным, но вялым попыткам Пилата освободить его. В действительности, ему было жаль Пилата, и он пытался просветить его пребывающий во тьме разум. Он с полным безразличием отнесся ко всем призывам римского правителя к евреям снять свои обвинения, выдвинутые против него. На протяжении всего скорбного испытания он вел себя со скромным достоинством и непоказным величием. Он даже отказался подозревать своих будущих убийц в неискренности, когда те спрашивали его, действительно ли он «царь евреев». С небольшими оговорками, он не возражал против этого титула, зная, что, хотя они решили отвергнуть его, он --- единственный, кто мог бы дать нации подлинное руководство, даже в духовном смысле.
\vs p186 2:9 На судах Иисус говорил мало, но достаточно, чтобы показать всем смертным, какой характер может воспитать в себе человек в единстве с Богом, и открыть всей вселенной, как Бог может являть себя в жизни своего создания, когда это создание воистину стремится исполнять волю Отца, становясь, таким образом, деятельным сыном живого Бога.
\vs p186 2:10 Его любовь к невежественным смертным в полной мере проявилась в его терпении и величайшем самообладании несмотря на глумление, побои и понукания грубых солдат и бездумных слуг. Он не разгневался даже тогда, когда они завязали ему глаза и, издевательски ударяя его по лицу, восклицали: «Прореки нам, кто ударил тебя».
\vs p186 2:11 Пилат был более прав, нежели сам это понимал, когда после бичевания Иисуса указал на него толпе, воскликнув: «Смотрите, се человек!» Едва ли мог себе представить охваченный страхом римский правитель, что вселенная замерла в тот самый момент, напряженно вглядываясь в невероятное зрелище того, как их возлюбленный Повелитель подвергается унизительным насмешкам и побоям своих темных и опустившихся смертных подданных. И слова Пилата отдались эхом по всему Небадону: «Смотрите, се Бог и человек!» С того дня бесчисленные миллионы по всей вселенной вечно продолжали взирать на этого человека, а Бог Хавоны, верховный правитель вселенной вселенных, признает этого человека из Назарета воплощением идеала смертных созданий этой локальной пространственно\hyp{}временной вселенной. Всю свою несравненную жизнь он постоянно открывал человеку Бога. Теперь, во время завершающих событий его земного жизненного пути и последующей смерти, он по\hyp{}новому и трогательно открыл человека Богу.
\usection{3. Надежный Давид Зеведеев}
\vs p186 3:1 Вскоре после того, как Иисус по завершении рассмотрения его дела Пилатом был передан римским солдатам, к Гефсимании спешно направился отряд храмовых стражников, чтобы разогнать или схватить последователей Иисуса. Но эти последователи рассеялись еще задолго до их прихода. Апостолы скрылись в условленных тайных местах; греки разделились и отправились в Иерусалим по разным домам; остальные ученики также скрылись. Зеведеев был уверен, что враги Иисуса вернутся; поэтому он заранее перенес пять или шесть шатров вверх по лощине, к тому месту, куда Учитель так часто удалялся для молитвы и поклонения. Здесь он предполагал спрятаться и, в то же самое время, устроить центр, или место встречи для отряда вестников. Едва Давид покинул лагерь, как явились храмовые стражники. Никого там не обнаружив, они удовольствовались тем, что подожгли лагерь, а затем поспешили обратно в храм. Выслушав их доклад, синедрион был удовлетворен тем, что последователи Иисуса настолько сильно напуганы и подавлены, что не стоит опасаться мятежа или попыток освободить Иисуса из рук палачей. Наконец\hyp{}то они могли вздохнуть свободно, и поэтому все разошлись и отправились, каждый своим путем, готовиться к Пасхе.
\vs p186 3:2 Как только Пилат передал Иисуса римским солдатам для распятия, к Гефсимании спешно направился вестник, чтобы сообщить обо всем Давиду, и через пять минут гонцы уже были на пути к Вифсаиде, Пелле, Филадельфии, Сидону, Сихему, Хеврону, Дамаску и Александрии. И эти вестники несли сообщение, что Иисус вот\hyp{}вот будет распят римлянами по настоятельному требованию правителей евреев.
\vs p186 3:3 На протяжении всего этого трагического дня вплоть до известия о том, что Учителя положили в гробницу, Давид каждые полчаса посылал вестников с новостями к апостолам, к грекам и к земной семье Иисуса, собравшейся в доме у Лазаря в Вифании. Когда вестники отправились с сообщением, что Иисуса похоронили, Давид распустил свой отряд местных гонцов на празднование Пасхи и последующий субботний отдых, велев им в воскресенье утром незаметно прийти к нему в дом Никодима, где он намеревался переждать несколько дней вместе с Андреем и Симоном Петром.
\vs p186 3:4 Давид Зеведеев, с его своеобразным складом ума, единственный из близких учеников Иисуса был настроен буквально и прямо понимать утверждение Учителя, что он умрет и «снова воскреснет на третий день». Давид некогда слышал от него это пророчество и, имея рационалистичный склад ума, намеревался теперь собрать своих вестников рано утром в воскресенье в доме у Никодима, чтобы они были под рукой и могли бы быстро разнести весть, если Иисус восстанет из мертвых. Вскоре Давид обнаружил, что никто из последователей Иисуса не надеется на такое скорое воскресение его из мертвых; поэтому он особенно и не говорил ни о своем убеждении, и ни о сборе рано утром в воскресенье всего отряда вестников никому, кроме гонцов, посланных в пятницу утром в отдаленные города и места, где собрались верующие.
\vs p186 3:5 Итак, последователи Иисуса, рассеянные по всему Иерусалиму и его окрестностям, в ту ночь праздновали Пасху, а на следующий день не покидали убежище.
\usection{4. Приготовление к распятию}
\vs p186 4:1 После того, как Пилат умыл руки перед толпой, стремясь таким образом снять с себя вину за предание невинного человека распятию только потому, что боялся противостоять шумным требованиям правителей евреев, он приказал передать Учителя римским солдатам и дал указание их командиру распять его немедленно. Забрав Иисуса, солдаты отвели его во двор претории, сняли багряницу, надетую на него Иродом, и одели его в его собственные одежды. Солдаты поддразнивали и осмеивали его, но больше не причиняли ему физическую боль. Теперь Иисус был один с этими римскими солдатами. Его друзья скрывались; его враги разошлись; даже Иоанна Зеведеева больше не было подле него.
\vs p186 4:2 Пилат передал Иисуса солдатам в начале девятого, а около девяти они уже шли к месту распятия. В эти более чем полчаса Иисус не проронил ни слова. Управление огромной вселенной практически остановилось. Гавриил и главные правители Небадона или собрались на Урантии, или же их внимание было приковано к космическим сообщениям архангелов, и они старались постоянно быть в курсе того, что происходит с Сыном Человеческим на Урантии.
\vs p186 4:3 К тому времени, когда солдаты готовы были отправиться с Иисусом к Голгофе, на них уже оказало влияние его необыкновенное самообладание и удивительное чувство собственного достоинства его безропотное молчание.
\vs p186 4:4 Они задержались и не ушли вместе с Иисусом к месту распятия, главным образом, из\hyp{}за того, что командир в последний момент принял решение взять с собой двух приговоренных к смерти воров; поскольку Иисус должен был быть распят в то утро, римский командир решил, что и еще двоих вполне можно было бы казнить вместе с ним, не дожидаясь конца празднования Пасхи.
\vs p186 4:5 Как только воры были готовы, их привели во двор, где они увидели Иисуса, причем один из них --- впервые, другой же часто слышал его речи, и в храме, и много месяцев назад в лагере в Пелле.
\usection{5. Смерть Иисуса и ее связь с Пасхой}
\vs p186 5:1 Между смертью Иисуса и еврейской Пасхой нет прямой связи. Правда, Учитель закончил свою жизнь во плоти именно в этот день, в день приготовлений к еврейской Пасхе, примерно в то время, когда в храме приносили в жертву пасхальных ягнят. Но совпадение этих событий никоим образом не означает, что смерть Сына Человеческого на земле имеет какую\hyp{}то связь с еврейским жертвенным ритуалом. Иисус был евреем, но как Сын Человеческий он был смертным из сфер. События, о которых уже было рассказано и которые привели к близящемуся распятию Учителя, достаточно ясно показывают, что его смерть почти в это время была событием естественным и обусловленным исключительно человеческими действиями.
\vs p186 5:2 Человек, а не Бог замыслил и привел в исполнение казнь Иисуса на кресте. Правда, Отец не стал вмешиваться в ход человеческих событий на Урантии, но ведь Райский Отец не санкционировал, не требовал и не желал той смерти своего Сына, которой он был предан на земле. Бесспорно, так или иначе, рано или поздно Иисусу пришлось бы лишиться своего смертного тела, обретенного при воплощении во плоти, но он мог сделать это множеством способов без того, чтобы умирать на кресте рядом с двумя ворами. Все это было деянием людей, а не Бога.
\vs p186 5:3 К моменту своего крещения Учитель уже обрел необходимый опыт во плоти на земле, который нужен был для завершения его седьмого и последнего пришествия во вселенной. В это время Иисус уже исполнил свой долг на земле. Его жизнь, прожитая после этого, и даже то, как он умер, все это было его личным служением ради блага и духовного подъема его смертных созданий и в этом мире, и в других мирах.
\vs p186 5:4 Евангелие благой вести о том, что смертный человек может через веру духовно осознать, что он сын Бога, не зависит от смерти Иисуса. Правда, смерть Учителя поистине осветила это евангелие царства, но в еще большей степени это совершила его жизнь.
\vs p186 5:5 Все, что Сын Человеческий говорил и делал на земле, только приумножало невероятную притягательность учения о сыновстве по отношению к Богу и о братстве людей, но эти необходимые отношения между Богом и людьми органически вытекают из вселенского факта любви Бога к своим созданиям и внутренне присущего божественным Сынам милосердия. Трогательные и божественно прекрасные отношения между человеком и его Творцом и в этом мире, и во всех других мирах по всей вселенной вселенных существовали извечно; и они никоим образом не зависят от периодических пришествий Сынов\hyp{}Творцов Бога, обретающих природу и образ созданных ими разумных творений, что является частью цены, которую они должны заплатить за окончательное обретение безграничного владычества над своими локальными вселенными.
\vs p186 5:6 Отец Небесный так же сильно любил земного смертного человека до жизни и смерти Иисуса на Урантии, как и после этого необыкновенного проявления взаимного сотрудничества человека и Бога. Это великое событие --- воплощение Бога Небадона в человека на Урантии --- не могло прибавить новых свойств вечному, бесконечному и всемирному Отцу, но оно обогатило и просветило всех других управителей и все создания во вселенной Небадон. Хотя благодаря этому пришествию Михаила Отец Небесный не стал любить нас сильнее, но все остальные небесные разумные существа --- стали. И это потому, что Иисус не только открыл Бога человеку, но еще и по\hyp{}новому открыл человека Богам и небесным разумным существам вселенной вселенных.
\vs p186 5:7 Иисусу предстоит пожертвовать собой не за грехи. Он не собирается искупать первородную моральную вину человеческой расы. У человечества нет никакой общечеловеческой вины перед Богом. Вина кроется лишь в личных грехах конкретных людей и в их осознанном и умышленном неповиновении воле Отца и правлению его Сынов.
\vs p186 5:8 Грех и неповиновение никак не связаны с предначертанием о пришествии Райских Сынов Бога, хотя нам кажется, что план спасения является составной частью пришествия.
\vs p186 5:9 Спасение, даруемое Богом смертным Урантии, было бы таким же действенным и совершенно непреложным и в том случае, если бы Иисус не был предан смерти жестокими руками невежественных смертных. Если бы Учитель был благосклонно принят смертными земли и покинул бы Урантию, добровольно отказавшись от своей жизни во плоти, это никоим образом не умалило бы факт любви Бога и милосердия Сына --- факт сыновства по отношению к Богу. Вы, смертные --- сыновья Бога, и требуется лишь одно, чтобы эта истина реально проявилась в вашем личном опыте, а именно --- как ваша порожденная духом вера.
