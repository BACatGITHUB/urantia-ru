\upaper{134}{Переходные годы}
\author{Комиссия срединников}
\vs p134 0:1 Во время средиземноморского путешествия Иисус внимательно изучал людей, с которыми встречался на пути, и страны, которые проезжал, и окончательное решение о том, как и где провести остаток своей земной жизни, принял приблизительно в это время. Он всесторонне взвесил и к тому времени полностью одобрил замысел, согласно которому был рожден от еврейских родителей в Палестине, и поэтому сознательно возвратился именно в Галилею в ожидании начала своего жизненного предназначения --- публичной деятельности в качестве учителя истины; Иисус начал обдумывать план своей публичной деятельности на земле народа отца своего Иосифа, и делал это по своей собственной свободной воле.
\vs p134 0:2 На личном и чисто человеческом опыте Иисус убедился в том, что из всего Римского мира именно Палестина была наилучшим местом для того, чтобы вписать последние главы и разыграть финальные сцены его жизни на земле. Впервые он ощутил полное удовлетворение планом --- открыто проявить свою истинную природу и раскрыть собственную божественную личность именно среди евреев и неевреев родной Палестины. Он окончательно решил, что закончит свою земную жизнь и завершит свой смертный путь на той же самой земле, куда он пришел беспомощным младенцем, чтобы постичь свой человеческий опыт. Его жизнь на Урантии началась среди евреев в Палестине и он по собственному выбору закончил ее в Палестине, среди евреев.
\usection{1. Тридцатый год (24 г. н.э.)}
\vs p134 1:1 Расставшись с Гонодом и Ганидом в Чараксе (в декабре 23 г. н.э.), Иисус возвратился дорогой через Ур в Вавилон, где присоединился к каравану, держащему путь через пустыню в Дамаск. Из Дамаска он отправился в Назарет, проведя лишь несколько часов в Капернауме, где задержался, чтобы навестить семью Зеведея. Там он встретил своего брата Иакова, который незадолго до этого приехал к Зеведею, чтобы работать в лодочной мастерской. Поговорив с Иаковом и Иудой (случайно как раз в ту пору также оказавшимся в Капернауме) и передав брату своему Иакову тот маленький домик, который удалось купить Иоанну Зеведееву, Иисус отправился в Назарет.
\vs p134 1:2 В конце своего средиземноморского путешествия Иисус получил достаточно денег для того, чтобы не думать о заработках почти до начала своего публичного служения. Но кроме Зеведея из Капернаума и тех людей, что встречались ему в этом удивительном путешествии, никто в мире не ведал о том, что эта поездка состоялась. Семья его всегда считала, что все это время Иисус учился в Александрии. Иисус никогда не подтверждал этих предположений, но и не делал ничего, чтобы их опровергнуть.
\vs p134 1:3 Проведя в Назарете несколько недель, он общался со своей семьей и друзьями, немного поработал в мастерской со своим братом Иосифом, но больше всего внимания уделял он Марии и Руфи. Руфи в то время было почти пятнадцать лет, она уже стала молодой женщиной, и впервые за эти годы Иисус смог несколько раз подробно побеседовать с ней.
\vs p134 1:4 И Симон, и Иуда уже собирались жениться, но не хотели делать этого без одобрения Иисуса, поэтому откладывали эти события, ожидая возвращения старшего брата. Несмотря на то, что в большинстве случаев все безоговорочно признавали Иакова главой семьи, когда дело касалось женитьбы, ждали благословения Иисуса. Итак, Симон и Иуда сыграли двойную свадьбу в начале марта того же года, 24 г. н.э. Таким образом, все старшие дети были теперь женаты. Только Руфь, младшая, оставалась дома с Марией.
\vs p134 1:5 Иисус вполне естественно и непринужденно общался с каждым из членов своей семьи по отдельности, но когда они собирались все вместе, ему почти что нечего было им сказать, так что они даже обсуждали это между собой. Мария была в особенности обескуражена столь странным поведением своего первенца.
\vs p134 1:6 Примерно в то время, когда Иисус собрался покидать Назарет, оказалось, что проводник большого каравана, проходившего через город, тяжко занемог, и Иисус, владеющий языками, вызвался занять его место. Поскольку это путешествие требовало его почти что годичного отсутствия и поскольку все его братья были уже женаты, а мать жила дома с Руфью, Иисус созвал семейный совет и предложил матери и Руфи переехать в Капернаум и поселиться в том доме, что он недавно передал Иакову. Так что спустя несколько дней после того как Иисус отправился с караваном, Мария и Руфь переехали в Капернаум, где и жили в доме, предоставленном Иисусом, вплоть до смерти Марии. Иосиф и его семья переехали в старый дом в Назарете.
\vs p134 1:7 Это был, пожалуй, один из самых необычных годов с точки зрения внутреннего опыта Сына Человеческого; за этот год был сделан большой шаг вперед в достижении действенной гармонии между его человеческим разумом и внутренним Настройщиком. Настройщик был активно занят перестройкой его мышления и подготовкой разума к великим событиями уже недалекого будущего. Личность Иисуса готовилась к великой перемене в его отношении к миру. Это были переходные годы, этап преобразования того существа, что начало жизнь Богом, явившимся в мир как человек, а сейчас готовилось завершить свой земной путь человеком, явленным как Бог.
\usection{2. Путешествие с караваном к Каспию}
\vs p134 2:1 Иисус покинул Назарет и отправился с караваном к Каспийскому морю первого апреля 24 года н.э. Караван, к которому Иисус присоединился в качестве проводника, держал путь из Иерусалима, мимо Дамаска и озера Урмия, через Ассирию, Мидию и Парфию к юго\hyp{}восточному побережью Каспийского моря. Прошел целый год, пока он вернулся из этого путешествия.
\vs p134 2:2 Для Иисуса это путешествие с караваном было еще одним событием в исследовании мира и личном служении. Он обрел интересный опыт в общении со своей караванной семьей --- пассажирами, погонщиками верблюдов, охраной. Сотни мужчин, женщин и детей, встретившиеся ему на караванном пути, обогатили свои жизни, общаясь с Иисусом --- таким необычным для них проводником заурядного каравана. Не всем, кому выпала радость личного общения с Иисусом, это принесло пользу, но подавляющее большинство тех людей, что встретили его и говорили с ним, изменились к лучшему на весь остаток своей земной жизни.
\vs p134 2:3 Из всех скитаний по миру это путешествие к Каспийскому морю привело Иисуса ближе всего к Востоку, дав ему таким образом возможность лучше понять народы Дальнего Востока. Он вступил в личный непосредственный контакт с каждой из сохранившихся на Урантии рас, кроме красной. Его одинаково радовало личное служение каждому представителю из этих разнообразных рас и смешанных народов, и все они были восприимчивы к живой истине, которую он принес им. Европейцы с далекого Запада и азиаты с Дальнего Востока с одинаковым вниманием относились к его словам надежды и вечной жизни, и на них в равной степени произвела впечатление жизнь, исполненная любви, служения и духовного пастырства, которую он так тонко прожил среди них.
\vs p134 2:4 \pc Путешествие с караваном было удачным со всех точек зрения. Оно было самым интересным эпизодом во всей человеческой жизни Иисуса, поскольку в течение этого года он осуществлял руководство, отвечая за ценности, вверенные его попечительству и безопасность жизни людей, путешествовавших в составе каравана. И он выполнял свои многочисленные обязанности с огромной ответственностью, эффективностью и мудростью.
\vs p134 2:5 На обратном пути из прикаспийских областей Иисус покинул караван на озере Урмия, где прожил немного более двух недель. Он возвратился в Дамаск уже как пассажир со следующим караваном, где владельцы верблюдов просили его остаться у них на службе. Отклонив это предложение, Иисус вернулся с караваном в Капернаум, куда прибыл первого апреля 25 г. н.э. Он больше не считал Назарет своим домом. Домом Иисуса, Иакова, Марии и Руфи стал Капернаум. Но Иисус больше никогда уже не жил со своей семьей; в Капернауме он останавливался у Зеведея.
\usection{3. Лекции в Урмии}
\vs p134 3:1 На пути к Каспийскому морю Иисус останавливался на несколько дней отдохнуть и восстановить силы в старом персидском городе Урмия на западном побережье озера Урмия. На самом большом из группы островов, расположенных на незначительном расстоянии от берега Урмии, стояло величественное строение --- амфитеатр для чтения лекций, --- посвященное «духу религии». И в самом деле, это воистину был храм философии религий.
\vs p134 3:2 Этот храм религии был построен богатым купцом, гражданином Урмии, и его тремя сыновьями. Человека этого звали Симбойтон, и предками его были представители многих разных народов.
\vs p134 3:3 Лекции и дискуссии в этой религиозной школе начинались в 10 каждое утро. Дневные занятия продолжались в 3, а вечерние дебаты открывались в 8. Сам Симбойтон или один из его трех сыновей всегда председательствовали на этих учебных занятиях, дискуссиях и дебатах. Основатель этой уникальной школы религий жил и умер, так и не открыв никому своих собственных религиозных убеждений.
\vs p134 3:4 Несколько раз Иисус участвовал в дискуссиях, и перед тем, как он собрался покинуть Урмию, Симбойтон предложил ему на обратном пути пожить с ними две недели и прочитать двадцать четыре лекции о «Братстве людей», а также провести двенадцать вечерних занятий, отвечая на вопросы по лекциям и участвуя в дискуссиях и дебатах по его лекциям в частности и на тему о братстве людей в общем.
\vs p134 3:5 Согласно этой договоренности, Иисус остановился на обратном пути в Урмии и прочел лекции. Эти занятия были самыми систематизированными и структурированными из всего, что Учитель проповедовал на Урантии. Никогда до этого и никогда впоследствии не говорил он столь пространно на одну и ту же тему, как сделал это в лекциях и дискуссиях о человеческом братстве. По сути, лекции эти были посвящены «Царству Бога» и «Царствам Человеческим».
\vs p134 3:6 Более тридцати религий и религиозных культов были представлены на факультете этого храма философии религии. Учителя, читавшие лекции, выбирались и получали все полномочия и финансовую поддержку соответствующих религиозных групп. В то время на факультете было приблизительно семьдесят пять учителей, и они жили в домиках по двенадцать человек в каждом. По прошествии лунного месяца эти группы менялись, причем все решала жеребьевка. Проявление нетерпимости, недоброжелательности или чего\hyp{}либо другого, способного расстроить плавное течение общинной жизни, приводило к немедленному увольнению нарушителя. Без лишних церемоний такой учитель смещался с должности, и его заменял коллега, ожидавший своего часа.
\vs p134 3:7 Эти учителя разных религий прилагали огромные усилия, чтобы показать, насколько сходны их религии в отношении к основным понятиям этой и последующей жизни. Существовала лишь одна доктрина, обязательная для любого, кто допускался на этот факультет, --- каждый из учителей должен был представлять религию, которая признавала Бога, --- ту или иную форму верховного Божества. На факультете было пять независимых учителей, которые не являлись представителями какой\hyp{}либо из официально признанных религий, и Иисус предстал перед слушателями как такой независимый учитель.
\vs p134 3:8 \pc [Когда мы, срединники, впервые готовили обзор учения Иисуса в Урмии, между серафимами церквей и серафимами прогресса возникли разногласия по поводу целесообразности включения этих учений в Откровение Урантии. Условия жизни в двадцатом веке, преобладающие как в религиозной сфере, так и в сфере правления, настолько отличаются от господствовавших в дни Иисуса, что было сложно адаптировать учения Учителя в Урмии к проблемам Царства Бога и царств людей в том виде, в котором эти мировые явления существуют в двадцатом столетии. Нам так никогда и не удалось сформулировать ни одного положения из лекций Учителя, которое было бы приемлемо для обеих групп серафимов из планетарного правительства. В конце концов Мелхиседек\hyp{}председатель комиссии по откровению назначил комиссию, состоящую из трех серафимов, чтобы подготовить изложение наших взглядов на урмийские лекции Учителя, адаптированные к религиозным и политическим условиям двадцатого века. Соответственно, мы, три срединника второго рода, завершили эту адаптацию учений Иисуса, переформулировав его утверждения согласно условиям современного мира, а теперь мы представляем вашему вниманию эти записи, отредактированные Мелхиседеком\hyp{}председателем комиссии по откровению.]
\usection{4. Владычество --- божественное и человеческое}
\vs p134 4:1 Братство людей основано на отцовстве Бога. Семья Бога происходит от любви Бога --- Бог есть любовь. Бог Отец любит своих детей божественной любовью, всех без исключения.
\vs p134 4:2 Царство небесное, божественное правление основано на факте божественного владычества --- ибо Бог есть дух. Поскольку Бог есть дух, это царство --- духовно. Царство небесное не является ни материальным, ни исключительно интеллектуальным; это духовные отношения между Богом и человеком.
\vs p134 4:3 Если различные религии признают духовное владычество Бога Отца, тогда все эти религии будут жить в мире. Только когда одна из религий предполагает, что каким\hyp{}либо образом превосходит все остальные и имеет исключительное значение по сравнению с другими религиями, такая религия допускает, что может быть нетерпимой по отношению к другим религиям или даже осмеливается преследовать верующих других религий.
\vs p134 4:4 Религиозный мир --- братство --- никогда не сможет существовать, пока все религии не изъявят желание всецело освободиться от всякой церковной власти и полностью оставить всякие притязания на духовное владычество. Лишь Бог есть духовный владыка.
\vs p134 4:5 Вы не добьетесь равенства между религиями (религиозной свободы) без религиозных войн, если все религии не согласятся передать всякое религиозное владычество на более высокий, чем человеческий уровень, то есть самому Богу.
\vs p134 4:6 Царство Небесное в сердцах людей создаст религиозное единство (не обязательно единообразие), потому что любые и все религиозные группы, состоящие из таких верующих, будут свободны от каких\hyp{}либо претензий на церковные прерогативы --- от религиозного владычества.
\vs p134 4:7 Бог есть дух, и Бог наделяет частицей своего духовного «я» сердце человека. Духовно все люди равны. В Царстве Небесном нет каст, классов, социальных уровней и экономических групп. Вы все братья.
\vs p134 4:8 Но в тот момент, когда вы забудете о духовном владычестве Бога Отца, какая\hyp{}нибудь одна религия начнет утверждать свое превосходство над другими; и тогда вместо мира на земле и доброй воли меж людьми начнутся разногласия, взаимные обвинения, религиозные войны; по крайней мере, войны между верующими.
\vs p134 4:9 Существа, обладающие свободной волей, считающие себя равными, если они взаимно не признают себя подчиненными некоему сверхвладычеству, некоему авторитету над собой и превыше себя, рано или поздно поддаются искушению попробовать завоевать власть и добиться превосходства над другими людьми и группами. Идея равенства приносит мир лишь в том случае, когда существует взаимное признание некоего вышестоящего руководящего влияния сверхвладычества.
\vs p134 4:10 Религиозные люди Урмии жили вместе в относительном мире и спокойствии, потому что они совершенно отказались от всех своих претензий на религиозное владычество. Духовно все они верили во владычество Бога; полная и неоспоримая власть над обществом принадлежала верховному правителю, председательствующему главе --- Симбойтону. Они хорошо знали, что произошло бы с любым учителем, который решил бы главенствовать над своими товарищами\hyp{}учителями. Никакой длительный религиозный мир на Урантии невозможен, пока все религиозные группы добровольно не оставят свои претензии на особое божественное расположение, избранность и религиозную власть. Только когда Бог Отец становится верховным, люди превращаются в братьев по религии и живут вместе на земле в религиозном мире.
\usection{5. Политическое владычество}
\vs p134 5:1 [В то время, как слова Учителя о владычестве Бога остаются истиной --- пусть и несколько осложненной последующим появлением религии о нем самом в числе прочих мировых религий, --- невероятно трудно, вследствие эволюции политической жизни наций за последние девятнадцать с лишним столетий изложить его положения о политическом владычестве так, чтобы они были доступны пониманию. Во времена Иисуса существовали только две великие мировые силы --- Римская империя на западе и Ханская империя на востоке --- и они были разделены Парфянским царством и другими лежащими между ними землями Каспийского и Туркестанского регионов. Таким образом, в нижеследующем изложении мы несколько более отклонились от того, что составляло сущность лекции Учителя в Урмии о политическом владычестве, и пытались в то же время показать значение этих положений применительно к особенно критической стадии эволюции политического владычества в двадцатом столетии после Христа.]
\vs p134 5:2 \pc Война на Урантии не прекратится никогда, до тех пор пока нации будут держаться за иллюзорные представления о безграничном национальном владычестве. Существует лишь два уровня относительного владычества в населенном мире: духовная свободная воля отдельного индивидуума и коллективное владычество человечества как целого. Между уровнем индивидуального человеческого существа и уровнем человечества в целом все группировки и сообщества людей относительны и преходящи и обладают ценностью лишь постольку, поскольку ведут к благополучию, благосостоянию и прогрессу --- индивидуума и планетарного целого, человека и человечества.
\vs p134 5:3 Религиозные учителя всегда должны помнить, что духовное владычество Бога превосходит все посредствующие промежуточные духовные приверженности. Когда\hyp{}нибудь гражданские правители поймут, что Всевышние правят в царствах человеческих.
\vs p134 5:4 Это правление Всевышних в царствах человеческих не дает особых преимуществ какой\hyp{}либо специально выделенной группе смертных. Не существует «избранного народа». Правление Всевышних, руководящих сверху политической эволюцией, есть правление, направленное на то, чтобы обеспечить наибольшее благо для наибольшего количества людей среди \bibemph{всех} и на наибольший период времени.
\vs p134 5:5 Владычество --- это власть, и она развивается путем организации. Подобный рост организации политической власти хорош и благ, поскольку стремится охватить все более широкие сегменты всего человечества. Но тот же самый рост политической организации создает проблемы на каждой промежуточной стадии между первичной и естественной организацией политической власти --- семьей --- и конечной стадией политического роста --- правительством всего человечества, для всего человечества и во имя всего человечества.
\vs p134 5:6 Начиная с родительской власти в семейной группе, политическое владычество усиливается путем организации по мере того, как семьи объединяются в родовые кланы, которые в свою очередь соединяются в силу различных причин в племенные союзы --- уже сверхродовые политические блоки. А затем в результате торговли, коммерции, завоеваний племена объединяются в нации, тогда как сами нации иногда объединяются империей.
\vs p134 5:7 По мере того, как владычество переходит от меньших групп к большим, число войн уменьшается. Это означает, что количество небольших войн между меньшими нациями сокращается, но потенциальная возможность возникновения великих войн растет, поскольку нации, обладающие владычеством, становятся все крупнее и крупнее. Сейчас, когда весь мир исследован и завоеван, когда существует лишь несколько сильных и могущественных наций, когда эти великие и, по общему мнению обладающие владычеством нации начинают соприкасаться границами, когда только океаны разделяют их, это означает, что сцена для представления великих войн и общемировых конфликтов подготовлена. Так называемые суверенные нации не могут жить, касаясь друг друга бок о бок и при этом не провоцировать конфликты и не вызывать войн.
\vs p134 5:8 Трудность в эволюционном развитии политического владычества от уровня семьи к уровню всего человечества заключается в инерции\hyp{}сопротивлении, проявляющемся на всех промежуточных уровнях. Семьи иногда идут против интересов клана, тогда как кланы и интересы племени зачастую противостоят властным интересам государства как территории. Каждый новый шаг на пути эволюции политического владычества оказывается (и всегда оказывался) стеснен и затруднен «лесами» предшествующей политической формации. И это воистину так, поскольку человеческую приверженность, однажды вызванную к жизни, уже трудно изменить. Та преданность, которая позволяла эволюционировать племени, будет затруднять эволюцию сверхплемени --- территориального государства. А та самая преданность (патриотизм), что делает возможной эволюцию территориального государства, невероятно осложняет эволюционное развитие правительства для всего человечества.
\vs p134 5:9 Политическое владычество создается путем подавления независимости сначала индивидуумами внутри семьи, а затем семьями и кланами по отношению к племени и большим группировкам. Подобное развитие, при котором возможность действовать по своему усмотрению переходило от мелких ко все более большим политическим организациям в целом происходил непрерывно на Востоке, с периода воцарения династий Минг и Могол. На Западе подобная ситуация продолжалась более тысячи лет вплоть до конца Мировой войны, когда, к сожалению, ретроградное движение временно обратило вспять нормальную тенденцию, восстановив политическую независимость многочисленных небольших государств в Европе.
\vs p134 5:10 Урантия не сможет наслаждаться длительным миром до той поры, пока так называемые суверенные нации разумно и всецело не вверят свою власть в руки человеческого братства --- всечеловеческого правительства. Интернационализм --- Лиги Наций --- никогда не принесет постоянного мира человечеству. Всемирные конфедерации различных наций могут эффективно предотвращать маленькие войны и неплохо контролировать малые народы, но они не смогут предотвратить мировых войн или контролировать три, четыре или пять наиболее могущественных правительств. Перед лицом реальных конфликтов одна из этих мировых сил выйдет из Лиги и объявит войну. Вы не удержите нации от ведения войн, пока они заражены иллюзорным вирусом национальной независимости. Интернационализм --- шаг в верном направлении. Интернациональные силы правопорядка предотвратят множество мелких войн, но не смогут эффективно предотвратить крупные войны и конфликты между великими военными державами на земле.
\vs p134 5:11 По мере того, как число подлинно суверенных наций (великих сил) уменьшается, увеличивается как возможность, так и необходимость создания всечеловеческого правительства. Когда существует только несколько действительно суверенных (великих) сил, они должны или включиться в борьбу не на жизнь, а на смерть за национальное (имперское) превосходство, или же, добровольно отказавшись от некоторых прерогатив суверенитета, создать основное ядро наднациональной власти, которое будет служить началом действительного суверенитета для всего человечества.
\vs p134 5:12 \pc Мир не придет на Урантию до тех пор, пока каждый из так называемых суверенных народов не передаст свои полномочия на ведение войны в руки представительного правительства всего человечества. Политический суверенитет органически присущ народам мира. Когда все народы Урантии создадут всемирное правительство, они будут обладать правом и способностью наделить такое правительство ВЛАДЫЧЕСТВОМ; когда такая представительская или демократическая мировая власть будет повсюду контролировать сушу, воздух, морские силы, именно тогда восторжествует мир на земле и добрая воля среди людей, но не ранее того.
\vs p134 5:13 Приведем важный пример из жизни девятнадцатого и двадцатого веков. Сорок восемь штатов Американского Федеративного Союза давно уже наслаждаются миром. Они больше не воюют друг с другом. Они передали свой суверенитет федеральному правительству и, пройдя сквозь ужасы войны, отказались от всех претензий на иллюзию самоопределения. Каждый отдельный штат сам занимается своими внутренними делами, тогда как международные отношения, установление тарифов, иммиграция, военные операции, торговые отношения между штатами не входят в его компетенцию. Отдельные штаты не занимаются и вопросами гражданства. Все сорок восемь штатов испытывают разрушительное действие войны только в том случае, когда суверенной власти федерального правительства угрожает какая\hyp{}то опасность.
\vs p134 5:14 \pc Таким образом, эти сорок восемь штатов, найдя в себе силы отказаться от двоякой заманчивости суверенной власти и самоопределения, наслаждаются царящими меж ними миром и спокойствием. Так же и все народы Урантии начнут радоваться миру, когда добровольно подчинят свой относительный суверенитет глобальному правительству --- суверенитету братства человеческого. В этом мировом государстве малые народы будут наделены такой же властью, как и великие. Так, например, маленький штат Род\hyp{}Айленд посылает в американский конгресс двух своих сенаторов так же, как и самый населенный штат Нью\hyp{}Йорк или огромный штат Техас.
\vs p134 5:15 Модель ограниченного (в рамках штата) суверенитета была создана людьми и для людей. Национальный суверенитет Американского Федеративного Союза изначально был создан тринадцатью штатами для их собственного блага и для блага людей. Когда\hyp{}нибудь сверхнациональное владычество планетарного всечеловеческого правительства будет подобным же образом создано народами для их собственного блага и во имя блага всего человечества.
\vs p134 5:16 Граждане не рождаются для блага правительств; правительства --- это организации, созданные и предназначенные для блага людей. Эволюция политического суверенитета завершится только с появлением правительства, суверенитет которого будет распространяться на все человечество. Все остальные институты власти обладают относительной ценностью, промежуточным значением и подчиненным статусом.
\vs p134 5:17 С прогрессом науки войны будут становиться все более и более разрушительными, пока не превратятся в почти что расовое самоубийство. Сколько же мировых войн должно отгреметь и сколько лиг наций должно пасть, прежде чем люди захотят создать общечеловеческое правительство и начнут наслаждаться благословенным прочным миром и вкушать плоды покоя и доброй воли меж людьми всего мира.
\usection{6. Закон, свобода и владычество}
\vs p134 6:1 Если один человек жаждет свободы --- освобождения, --- то он должен помнить, что \bibemph{все} другие люди стремятся к такой же свободе. Подобные группы людей, стремящихся к свободе, не смогут жить в мире, не подчиняясь таким законам, правилам и предписаниям, которые обеспечили бы каждому определенную степень свободы, в то же время сохраняя ту же степень свободы для всех его собратьев. Если один человек станет абсолютно свободным, другому придется стать абсолютным рабом. И относительная природа свободы оправдана социально, экономически и политически. Свобода --- это дар цивилизации, ставший возможным благодаря силе ЗАКОНА.
\vs p134 6:2 Религия обеспечивает духовную возможность создания братства людей, но тем не менее необходимо всечеловеческое правительство для регулирования социальных, экономических и политических проблем, связанных с достижением подобной цели --- счастья и полноты человеческой жизни.
\vs p134 6:3 Войны и толки о них будут продолжаться, народ будет восставать против народа --- до тех пор, пока высшее политическое владычество в мире поделено и несправедливо удерживается группой наций\hyp{}государств. Англия, Шотландия и Уэльс постоянно сражались друг с другом до тех пор, пока не отказались от своей независимости, вверив власть Объединенному Королевству.
\vs p134 6:4 Следующая мировая война научит так называемые суверенные нации тому, что необходимо сформировать некий род федерации, создавая таким образом механизм предотвращения малых войн, войн между меньшими народами. Но мировые войны будут продолжаться до той поры, пока не будет создано всечеловеческое правительство. Мировое владычество предотвратит мировые войны --- ибо ничто больше не может их остановить.
\vs p134 6:5 Сорок восемь американских свободных штатов живут вместе в мире. Граждане этих сорока восьми штатов --- это люди разнообразных национальностей и рас, представители которых принадлежат в том числе и к постоянно враждующим между собой европейским народам. Эти американцы представляют почти все религии и религиозные секты и культы, какие только есть на свете, и здесь, в Северной Америке, они мирно уживаются друг с другом. И все это стало возможным благодаря тому, что эти сорок восемь штатов отказались от своего суверенитета и расстались с идеями о пресловутом праве на самоопределение.
\vs p134 6:6 И это не вопрос вооружения или разоружения. Точно так же и вопрос о воинской повинности или добровольной службе в армии никогда не будет связан с проблемой поддержания мира во всем мире. Если отобрать у сильных народов современное механическое вооружение и взрывчатые вещества всех типов, они станут драться кулаками, камнями и дубинками до тех пор, пока не оставят иллюзии о божественном праве на национальный суверенитет.
\vs p134 6:7 Война --- не самая большая и страшная болезнь человека; война --- это симптом, результат. Настоящая болезнь --- это вирус национальной самостоятельности.
\vs p134 6:8 Народы Урантии еще не обладали настоящим владычеством; никогда еще не было у них могущества, достаточного для того, чтобы защитить их от опустошений и разрушений, сопутствующих мировым войнам. В создании глобального общечеловеческого правительства народы не столько лишаются владычества, сколько в действительности создают настоящее и долговечное мировое владычество, которое отныне и вовеки будет способно полностью защитить их от любой войны. Местными делами будут заниматься местные правительства; национальными делами --- правительства национальные; международные дела будут в ведении общемирового правительства.
\vs p134 6:9 Мир во всем мире не поддержишь договорами, дипломатией, внешней политикой, альянсами, балансировкой власти или какими бы то ни было другими временными приемами и заигрыванием с национальными правительствами суверенитетом. Должен возникнуть мировой закон, и его способно поддержать лишь мировое правительство --- владычество всего человечества.
\vs p134 6:10 Каждый отдельный человек будет наслаждаться гораздо большей свободой под властью мирового правительства. Сегодня граждане великих держав испытывают гнет налогов и законов, и ими почти насильственно управляют. Когда национальные правительства по собственному желанию передадут свою власть (в том, что касается международных дел) в руки общемирового правительства, многое из того, что сейчас вторгается в область личной свободы человека, исчезнет.
\vs p134 6:11 Под руководством общемирового правительства национальным группам будет предоставлена реальная возможность осознать и прочувствовать свободы личности существующие при истинной демократии. С заблуждением о самоопределении будет покончено. Глобальное регулирование денежных и торговых потоков принесет новую эру мира во всем мире. Вскоре может возникнуть и общемировой язык, и по крайней мере, появится надежда на возможность существования общемировой религии --- религии с мировосприятием, охватывающим всех людей.
\vs p134 6:12 Устремления к общей безопасности не принесут мира до тех пор, пока само общество не включит в себя все человечество.
\vs p134 6:13 Политическое владычество представительского общечеловеческого правительства принесет на землю прочный мир, а духовное братство людей навсегда обеспечит добрую волю меж людьми. И другого пути для установления мира на земле и доброй воли меж людьми нет.
\separatorline
\vs p134 6:14 После смерти Симбойтона его сыновья столкнулись с большими проблемами, пытаясь поддерживать мир в школе. Влияние учений Иисуса было бы гораздо сильнее, если бы позднехристианские учителя, присоединившиеся к школе в Урмии, проявили бы больше мудрости и терпимости.
\vs p134 6:15 Старший сын Симбойтона обратился к Авениру в Филадельфию за помощью, но тот на редкость неудачно выбрал учителей, поскольку все они проявили себя как крайне негибкие и бескомпромиссные люди. Эти учителя стремились к тому, чтобы их религия доминировала над другими верованиями. Они так и не догадались, что столь часто упоминаемые наставления проводника каравана были сделаны самим Иисусом.
\vs p134 6:16 Поскольку разногласия в школе все увеличивались, три брата прекратили оказывать ей финансовую поддержку, и через пять лет школа и вовсе закрылась. Позднее она вновь открылась --- уже как митраистский храм, который сгорел в результате пожара, возникшего во время одного из оргиастических празднеств.
\usection{7. Тридцать первый год (25~г.\,н.э.)}
\vs p134 7:1 Когда Иисус вернулся из путешествия к Каспийскому морю, он уже знал, что его странствия по свету подходят к концу. Он только однажды с той поры выехал за пределы Палестины --- это была поездка в Сирию. Пробыв недолго в Капернауме, он отправился в Назарет и остановился там на несколько дней. В середине апреля он отбыл из Назарета в Тир. Оттуда он пошел на север, задержался ненадолго в Сидоне, но его целью была Антиохия.
\vs p134 7:2 В этот год Иисус в одиночестве странствовал по Палестине и Сирии. Во время этих путешествий его узнали под разными именами в различных частях страны: как плотника из Назарета, как корабельщика из Капернаума, книжника из Дамаска, учителя из Александрии.
\vs p134 7:3 В Антиохии Сын Человеческий провел больше двух месяцев --- работал, наблюдал, изучал, общался и совершал служение --- и все это время постигал, как живет человек, как думает, чувствует и реагирует на условия человеческого существования. В течение трех недель он работал в Антиохии, изготавливая шатры; здесь он пробыл дольше, чем во всех других местах, которые он посетил во время этого путешествия. Десять лет спустя, когда апостол Павел проповедовал в Антиохии и слышал, как его последователи упоминают об учении \bibemph{книжника из Дамаска,} он и понятия не имел о том, что его ученики слышали голос и внимали проповедям самого Учителя.
\vs p134 7:4 Из Антиохии Иисус направился на юг вдоль побережья к Кесарии, где и пробыл несколько недель, а затем продолжил путешествие вниз по побережью до Иоппии. Из Иоппии он повернул в глубь материка к Ямнии, Ашдоду и Газе. Покидая Газу, он продолжил путь дальше, в Вирсавию, где пробыл неделю.
\vs p134 7:5 Оттуда Иисус и начал свое последнее путешествие как обычный человек через сердце Палестины: из Вирсавии на юге к Дану на севере. За время этого путешествия на север он останавливался в Хевроне, Вифлееме (где увидел место своего рождения), Иерусалиме (в Вифанию он не заезжал), Веерофе, Левоне, Сихаре, Сихеме, Самарии, Гиве, Эн\hyp{}Ганниме, Ендоре, Мадоне; пройдя через Магдалу и Капернаум, он двинулся к северу; а пройдя восточнее ручья Мером, он направился через Карахту к Дану, или Кесарии Филипповой.
\vs p134 7:6 Пребывающий в нем Настройщик вел Иисуса к тому, чтобы покинуть места обитания людей и направиться к Горе Ермон, где он мог довершить овладение своим человеческим разумом и полностью посвятить себя делу своей жизни на земле.
\vs p134 7:7 Это был один из самых необычных и непостижимых периодов в жизни Учителя на Урантии. Другой, очень похожий период, --- это опыт, полученный им в уединении в горах недалеко от Пеллы сразу после крещения. Это время уединения на горе Ермон означало завершение его чисто человеческого пути, т.е. просто техническое окончание его пришествия в облике смертного, в то время как последнее уединение свидетельствовало о начале божественной фазы пришествия. И Иисус пробыл наедине с Богом шесть недель на склонах горы Ермон.
\usection{8. Пребывание на горе Ермон}
\vs p134 8:1 Проведя некоторое время неподалеку от Кесарии Филипповой, Иисус собрал провизию, нашел вьючное животное, договорившись с парнишкой по имени Тиглат, и направился по Дамасской дороге к деревушке, когда\hyp{}то известной под названием Бейт Женн у подножия горы Ермон. Здесь около середины августа в 25 г. н.э. он устроил стоянку, и, оставив вещи на попечении Тиглата, поднялся на пустынные склоны горы. Тиглат сопровождал Иисуса в первый день на гору до условленного места (около 6000 футов над уровнем моря), где они устроили хранилище из камней, куда Тиглат дважды в неделю должен был приносить еду.
\vs p134 8:2 В первый день, покинув Тиглата, Иисус поднялся невысоко в горы и остановился для молитвы. Среди прочего он попросил Отца своего отослать назад серафима\hyp{}хранительницу, чтобы «быть с Тиглатом». Он просил, чтобы ему было позволено выдержать свой последний бой со всем тем, что составляет земную жизнь, один на один. И его просьба была удовлетворена. Он отправился на великое испытание, поддерживаемый и направляемый лишь пребывающим в нем Настройщиком.
\vs p134 8:3 \pc На горе Иисус ел мало; полностью он воздерживался от пищи лишь день или два подряд. Сверхчеловеческие существа, с которыми он столкнулся на этой горе и с которыми он боролся в духе и которых победил в могуществе, были \bibemph{реальны;} это были его главнейшие враги в системе Сатании; они не были призраками и фантомами, как не были они и порождениями рассудка ослабевшего и голодающего смертного, который не в силах отличить реальность от видений расстроенного разума.
\vs p134 8:4 Последние три недели августа и первые три недели сентября Иисус провел на горе Ермон. В течение этих недель он завершил свое земное предназначение достигнуть кругов понимания разума и управления личности. В течение этого периода общения с небесным Отцом Настройщик также выполнил все то, что от него требовалось. Таким образом, цель земного пути этого смертного была достигнута здесь. Осталось лишь завершить финальную фазу согласования разума с Настройщиком.
\vs p134 8:5 После пяти с лишним недель непрерывного общения с Райским Отцом Иисус окончательно уверовал в свою природу и неизбежность своей победы над материальными уровнями личностного воплощения в пространстве и времени. Он полностью верил и без тени сомнений утверждал, что божественная сущность его природы преобладает над его человеческой сущностью.
\vs p134 8:6 \pc Ближе к концу пребывания на горе Иисус попросил Отца своего о позволении встретиться с его врагами в Сатании в образе Сына Человеческого, Иешуа бен Иосифа. И эта просьба была удовлетворена. И в последнюю неделю его пребывания на горе Ермон произошло великое искушение, вселенское испытание. Сатана (представляющий Люцифера) и мятежный Планетарный Принц Калигастия предстали пред Иисусом и были совершенно видимы ему. И это «искушение», это последнее испытание человеческой верности перед лицом коварства бунтовщиков, не было связано ни с пищей, ни с маковками храмов, ни с дерзкими действиями. Оно было связано не с царствами мира сего, но имело отношение к владычеству в огромной и великолепной вселенной. Символика ваших записей представляет те далекие века, когда мир мыслил по\hyp{}детски. Последующие же поколения должны понимать, какую великую борьбу выдержал Сын Человеческий в тот знаменательный день на горе Ермон.
\vs p134 8:7 На многочисленные и самые разнообразные предложения посланников Люцифера Иисус лишь отвечал: «Да исполнится воля Райского Отца моего, а тебя, мой мятежный сын, тебя пусть судят Древние Дней судом божественным. Я твой Отец\hyp{}Творец; едва ли могу я судить тебя справедливо, а милостью моею ты уже пренебрег. Вверяю тебя решению Судей всей вселенной».
\vs p134 8:8 На все предложенные Люцифером соблазны и компромиссы, на все лицемерные предложения, затрагивающие пришествие, Иисус лишь отвечал: «Да свершится воля Отца Моего». И когда суровое испытание было завершено, отосланный серафим\hyp{}хранительница вернулся к Иисусу и служил ему.
\vs p134 8:9 \pc В конце лета, пополудни, среди деревьев и в тиши природы Михаил из Небадона завоевал неоспоримое владычество над своей вселенной. В этот день он исполнил миссию, поставленную перед Сынами\hyp{}Творцами, --- прожить полностью жизнь в воплощении, в образе смертного, в эволюционирующих мирах со временем и пространством. Всей вселенной об этом важном достижении не было объявлено до дня его крещения, которое произошло много месяцев спустя, но в действительности все решилось в тот день на горе. И когда Иисус спустился с горы Ермон, то бунт Люцифера в Сатании и раскол, внесенный Калигастией, на Урантии были по сути дела разрешены. Иисус заплатил последнюю цену, которая требовалась от него, чтобы обладать владычеством в его вселенной, что само по себе установило статус всех мятежников и предопределило, что в будущем с подобными переворотами (если они когда\hyp{}нибудь произойдут) можно будет сладить быстро и эффективно. Итак, мы видим, что так называемое «великое искушение» Иисуса произошло до его крещения, а вовсе не непосредственно после этого события.
\vs p134 8:10 К концу его пребывания на горе, когда Иисус уже спускался, он встретил Тиглата, идущего навстречу с едой. Отправляя мальчика назад, он произнес лишь: «Время отдыха закончено. Я должен вернуться к делу Отца моего». По дороге назад в Дан он был молчалив и казался сильно преобразившимся. В Дане он расстался с мальчиком, подарив тому осла. Затем он направился дальше на юг в Капернаум той же дорогой, по которой пришел.
\usection{9. Время ожидания}
\vs p134 9:1 Лето заканчивалось, приближался день искупления и праздника кущей. В субботу Иисус собрал в Капернауме семейный совет, а на следующий день отправился в Иерусалим с Иоанном сыном Зеведеевым, он держал путь к востоку от озера, мимо Геразы и дальше к Иорданской долине. Хотя в пути он все же общался со своим спутником, Иоанн заметил большую перемену в Иисусе.
\vs p134 9:2 Иисус и Иоанн остановились на ночь в Вифании у Лазаря и его сестер, а на следующее утро спозаранку отправились в Иерусалим. Почти три недели они провели в городе и его окрестностях, по крайней мере Иоанн. Не раз Иоанн отправлялся в Иерусалим один, в то время как Иисус, в прогулках по окрестным горам, вновь и вновь погружался в духовное общение со своим Небесным Отцом.
\vs p134 9:3 Оба они присутствовали на торжественных церемониях по случаю дня искупления. Обряды, творимые в этот день, произвели на Иоанна большое впечатление --- большее, чем все остальные еврейские ритуалы, но Иисус оставался задумчивым и безмолвным зрителем. Сыну Человеческому это действо показалось жалким и патетическим. Он видел во всем этом искаженные представления о сути и чертах своего Небесного Отца. Он смотрел на происходящее в этот день как на пародию на божественную справедливость, на истину безграничного милосердия. Он хотел дать волю словам истинной правды о подлинной любви своего Отца и его исполненном милосердия правлении во вселенной, но верный Наблюдатель, увещевая, напоминал ему, что его час еще не пришел. Но той ночью в Вифании Иисус обронил множество замечаний, весьма озаботивших Иоанна; Иоанн так никогда до конца и не понял подлинного смысла сказанных Иисусом в тот вечер слов.
\vs p134 9:4 Иисус предполагал провести с Иоанном всю неделю праздника кущей. Этот праздник ежегодно отмечался по всей Палестине; это было время отдыха для евреев. Хотя Иисус не участвовал в праздничном веселье по этому случаю, было очевидно, что он с удовольствием и удовлетворением смотрел на беззаботное и счастливое ликование молодежи и стариков.
\vs p134 9:5 В середине праздничной недели и прежде чем торжества закончились, Иисус покинул Иоанна, сказав, что хочет вернуться в горы, где сможет лучше общаться со своим Райским Отцом. Иоанн хотел было сопровождать его, но Иисус настоял на том, чтобы он оставался на празднестве, говоря: «Тебе нет надобности нести бремя Сына Человеческого; только дозорный должен бдеть, пока город спокойно спит». Иисус не вернулся в Иерусалим. Проведя почти неделю в одиночестве в горах неподалеку от Вифании, он отправился в Капернаум. На пути домой он провел один день и одну ночь в уединении на склонах Гелвуй, неподалеку от мест, где лишил себя жизни царь Саул; когда он приехал в Капернаум, то выглядел более бодрым, чем когда покидал Иоанна в Иерусалиме.
\vs p134 9:6 На следующее утро Иисус пошел за сундучком, где хранились его личные вещи, оставленные в мастерской Зеведея, надел свой фартук и принялся за работу, говоря: «В ожидании своего часа мне надлежит трудиться». И он работал несколько месяцев --- до января следующего года --- в лодочной мастерской бок о бок со своим братом Иаковом. После этого периода совместной работы с Иисусом Иаков, какие бы сомнения ни мешали ему понимать дело жизни Сына Человеческого, никогда уже по\hyp{}настоящему и полностью не отступал от веры в его миссию.
\vs p134 9:7 В течение этого завершающего периода трудов в лодочной мастерской Иисус большую часть времени посвящал работам по внутренней отделке большого судна. Всю ручную работу он исполнял с большим усердием и, казалось, испытывал большое удовлетворение от своих человеческих успехов каждый раз, когда заканчивал работу, достойную похвалы. Хотя он не тратил времени на мелочи, он был старательным тружеником во всем, что касалось сути взятого на себя дела.
\vs p134 9:8 \pc Со временем до Капернаума докатились слухи о некоем Иоанне, проповедующем и крестящем кающихся в реке Иордан. И Иоанн проповедовал: «Царство Божье приблизилось; покайтесь и креститесь». Иисус прислушивался к этим известиям, в то время как Иоанн медленно продвигался от ближайшего к Иерусалиму брода через реку, вверх по Иорданской долине. Но Иисус работал, строя лодки, вплоть до той поры, когда Иоанн, двигаясь вдоль реки, в январе следующего, 26 г. н.э., не дошел до окрестностей Пеллы. Тогда Иисус сложил свои инструменты, заявив: «Час мой настал», --- и предстал пред Иоанном для крещения.
\vs p134 9:9 Иисус сильно переменился. Лишь немногие из тех людей, кому выпало насладиться общением с ним и служением его в то время, когда он странствовал по всей стране, потом узнавали во всенародном учителе того самого человека, которого они знали и любили как простого человека в прошлые годы. И, действительно, были причины для того, чтобы облагодетельствованные им ранее люди не могли позднее узнать его в роли всенародного и пользующегося авторитетом учителя. В течение долгих лет его разум и дух преобразились, и это случилось в тот исполненный событиями период пребывания на горе Ермон.
