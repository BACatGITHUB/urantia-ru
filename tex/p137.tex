\upaper{137}{Остановка в Галилее}
\author{Комиссия срединников}
\vs p137 0:1 Рано утром в субботу 23 февраля, 26 года н.э., Иисус спустился с гор, чтобы присоединиться к последователям Иоанна, расположившимся лагерем в Пелле. Весь тот день Иисус провел, смешавшись с толпой. Он оказал помощь пареньку, травмированному при падении, и отвез его домой в деревню рядом с Пеллой, передав на руки родителям.
\usection{1. Избрание первых четырех апостолов}
\vs p137 1:1 В эту субботу два лучших ученика Иоанна провели много времени с Иисусом. На одного из них по имени Андрей, Иисус произвел особенно большое впечатление; он сопровождал Иисуса, когда тот шел в Пеллу с травмированным мальчиком. На обратном пути в лагерь Иоанна он задал Иисусу много вопросов, и, дойдя почти до места своего назначения, они остановились для краткой беседы, во время которой Андрей сказал: «Я все время наблюдал за тобой с тех пор, как ты пришел в Капернаум, и я верю, что ты --- новый Учитель, и хотя не все в твоем учении я понимаю, я окончательно решил следовать за тобой; я хотел бы сидеть у твоих ног и узнать всю истину о новом царстве». И Иисус сердечно и с уверенностью приветствовал Андрея как первого из своих апостолов, из тех двенадцати, кому предстояло трудиться вместе с ним над установлением нового царства Божьего в сердцах человеческих.
\vs p137 1:2 \pc Андрей был молчаливым наблюдателем и искренним приверженцем деятельности Иоанна, и у него был очень одаренный и преисполненный энтузиазма брат по имени Симон, который был одним из главных учеников Иоанна. Не будет ошибкой утверждать, что он был одним из самых последовательных сторонников Иоанна.
\vs p137 1:3 Вскоре после того, как Иисус и Андрей возвратились в лагерь, Андрей разыскал своего брата Симона и, отведя его в сторону, сообщил о своей убежденности в том, что Иисус --- великий Учитель и что он как ученик дал ему обет. Затем он сказал, что Иисус принял его предложение служения, и посоветовал ему (Симону) тоже подойти к Иисусу и выразить готовность вступить в братство служителей нового царства. Сказал Симон: «С тех самых пор, как этот человек пришел работать в мастерскую Зеведея, я уверен, что он послан Богом, но как же быть с Иоанном? Неужели мы оставим его? Хорошо ли поступать так?» И они решили немедленно посоветоваться с самим Иоанном. Иоанн был опечален мыслью о потере двух способных советчиков и самых многообещающих учеников, но мужественно ответил на их вопросы, сказав: «Это только начало; в скором времени моя работа завершится, и мы все станем его учениками». Тогда Андрей знаком отозвал Иисуса в сторону и сообщил, что его брат желает присоединиться к нему, чтобы служить новому царству. Приветствуя Симона как своего второго апостола, Иисус сказал: «Симон, твой энтузиазм заслуживает похвалы, но и опасен для деятельности царства. Предупреждаю, что ты должен стать более осмотрительным в своих речах. Я хотел бы наречь тебя Петром».
\vs p137 1:4 \pc Родители травмированного мальчика, которые жили в Пелле, упрашивали Иисуса провести с ними ночь, чтобы их дом стал его домом, и он обещал. Прежде чем расстаться с Андреем и его братом, Иисус сказал: «Рано утром мы идем в Галилею».
\vs p137 1:5 \pc После того как Иисус возвратился в Пеллу на ночлег и в то время как Андрей и Симон еще продолжали обсуждать свое служение ради установления грядущего царства, в лагере появились сыновья Зеведея Иаков и Иоанн, вернувшиеся только что после долгих и тщетных поисков Иисуса в горах. Услышав от Симона Петра о том, как он и его брат Андрей стали первыми принятыми советниками нового царства и вместе со своим новым Учителем завтра утром уходят в Галилею, Иаков и Иоанн оба опечалились. Они знали Иисуса долгое время и любили его. Много дней они искали его в горах, а вернувшись, узнали, что других выбрали прежде них. Они справились, куда ушел Иисус, и поспешили найти его.
\vs p137 1:6 Когда они пришли, Иисус спал, но они разбудили его со словами: «Как же так, --- пока мы, которые так долго жили с тобой, ищем тебя в горах, ты ставишь других перед нами и выбираешь Андрея с Симоном твоими первыми сподвижниками в новом царстве?» Иисус отвечал им: «Успокойтесь в своих сердцах и спросите себя, кто послал вас искать Сына Человеческого, когда он был занят делом Отца своего?». После того, как они подробно рассказали о своем долгом поиске в горах, он дал им наставление: «Вы должны научиться искать тайну нового царства в своих сердцах, а не в горах. То, что искали вы, уже было в ваших душах. Воистину вы мои братья --- вы не нуждаетесь в том, чтобы быть принятыми мною, --- вы уже от этого царства, и теперь ободритесь и приготовьтесь идти с нами завтра в Галилею». «Но, Учитель, --- осмелился спросить Иоанн, --- будем ли Иаков и я твоими сподвижниками в новом царстве, так же, как Андрей и Симон?» И положив руку на плечо каждому из них, Иисус сказал: «Мои братья, вы уже пребывали со мною духовно в новом царстве, еще до того, как другие попросили о том, чтобы быть принятыми. Вам, братьям моим, нет необходимости просить о том, чтобы войти в царство: с самого начала вы со мною в нем. Перед людьми другие могут опередить вас, но в сердце своем я причислял вас к совету нового царства еще до того, как вы помыслили просить об этом. Да вы могли быть первыми и перед людьми, если бы не отсутствовали, занятые из лучших побуждений, но самовольно поиском того, кто не был потерян. В грядущем царстве не обращайте внимания на то, что питает вашу тревогу; всегда заботьтесь лишь о выполнении воли Отца, который на небесах».
\vs p137 1:7 Иаков и Иоанн с готовностью приняли упрек; никогда более не завидовали они Андрею с Симоном. И они приготовились идти в Галилею на следующее утро вместе с двумя своими товарищами\hyp{}апостолами. С того самого дня словом «апостол» стали называть избранных советников Иисуса, чтобы отличить их от огромного множества верующих учеников, которые впоследствии следовали за ним.
\vs p137 1:8 \pc Позже в тот вечер у Иакова, Иоанна, Андрея и Симона был разговор с Иоанном Крестителем; со слезами на глазах, но твердым голосом мужественный иудейский пророк отказался от двух своих ближайших учеников чтобы они стали апостолами Галилейского Принца грядущего царства.
\usection{2. Избрание Филиппа и Нафанаила}
\vs p137 2:1 В воскресенье утром 24 февраля 26 года н.э. Иисус простился с Иоанном Крестителем у реки рядом с Пеллой, и он никогда больше не видел его в плотском облике.
\vs p137 2:2 В тот день, когда Иисус с четырьмя учениками\hyp{}апостолами ушел в Галилею, в лагере последователей Иоанна было большое смятение. Назревал первый крупный раскол. Днем раньше Иоанн совершенно однозначно объявил Андрею и Эзре, что Иисус является Спасителем. Андрей решил следовать за Иисусом, но Эзра отверг мягкого в обращении назаретского плотника, заявив товарищам: «Пророк Даниил возвещает, что Сын Человеческий придет в облаках небесных, во власти и великой славе. Этот галилейский плотник, этот лодочный мастер из Капернаума не может быть Спасителем. Мыслимо ли, чтобы такой дар Божий явился из Назарета? Этот Иисус --- родственник Иоанна, и через сердечную свою доброту наш учитель обманут. Давайте держаться подальше от этого фальшивого Мессии». Когда Иоанн стал укорять Эзру за эти речи, тот ушел от него со многими учениками и поспешил на юг. Впоследствии они продолжали крестить от имени Иоанна и в конце концов основали секту, объединившую приверженцев Иоанна, которые не приняли Христа. Остатки ее существуют в Месопотамии по сей день.
\vs p137 2:3 \pc Пока в среде последователей Иоанна происходили эти неприятные события, Иисус и его четыре ученика\hyp{}апостола прошли немалую часть пути до Галилеи. Прежде чем они пересекли Иордан, на дороге, соединяющей Наин и Назарет, Иисус увидел идущего навстречу им некоего Филиппа из Вифсаиды с товарищем. Иисус прежде знал Филиппа, и четверо новых апостолов также были с ним хорошо знакомы. Вместе со своим другом Нафанаилом он шел к Иоанну в Пеллу, желая больше узнать о возвещенном наступлении царства Божьего, и был рад встретить Иисуса. Он был поклонником Иисуса с тех пор, как он впервые пришел в Капернаум. Но Нафанаил, который жил в Кане Галилейской, не знал Иисуса. Филипп пошел вперед приветствовать друзей, в то время как Нафанаил ожидал в тени дерева на обочине дороги.
\vs p137 2:4 Петр отвел Филиппа в сторону и стал объяснять ему, что они --- имея в виду себя, Андрея, Иакова и Иоанна, --- стали сподвижниками Иисуса в новом царстве, и горячо убеждал его предложить свое служение. Филипп был в затруднении. Как поступить? Здесь, внезапно, на обочине дороги возле Иордана, он должен был принять безотлагательное решение по самому важному вопросу всей своей жизни. У него завязался серьезный разговор с Петром, Андреем и Иоанном, в то время как Иисус в общих чертах описывал Иакову предстоящий им путь через Галилею и далее до Капернаума. В конце концов Андрей предложил Филиппу: «Почему бы не спросить Учителя?»
\vs p137 2:5 И вдруг Филиппу стало ясно, что Иисус --- на самом деле великий человек, возможно, Мессия, и решил поступить как он скажет; и он направился прямо к нему и спросил: «Учитель, идти мне к Иоанну или же присоединиться к моим друзьям, которые следуют за тобой?» И ответил Иисус: «Следуй за мной». И Филиппа охватил трепет уверенности, что он нашел Спасителя.
\vs p137 2:6 \pc Затем Филипп сделал знак всей группе подождать его, а сам между тем поспешил сообщить о принятом решении своему другу Нафанаилу, стоявшему в тени шелковицы, размышляя о многих вещах, которые он слышал об Иоанне Крестителе, грядущем царстве и ожидаемом Мессии. Филипп прервал его раздумья, воскликнув: «Я нашел Спасителя, того, о ком писали Моисей и пророки и о ком возвещал Иоанн». Нафанаил, подняв глаза, осведомился: «Откуда происходит этот учитель?» И Филипп ответил: «Это Иисус из Назарета, недавно поселившийся в Капернауме, сын плотника Иосифа». И, несколько пораженный, Нафанаил спросил: «Из Назарета может ли быть что доброе?» Но Филипп, взяв его за руку, сказал: «Иди и смотри».
\vs p137 2:7 Филипп подвел Нафанаила к Иисусу, который, ласково взглянув в лицо искренне сомневающегося, произнес: «Вот истинный израильтянин, в ком нет обмана. Следуй за мной». И Нафанаил, обернувшись к Филиппу, сказал: «Ты прав. Он действительно учитель людей. Я тоже последую за ним, если я достоин». Иисус кивнул Нафанаилу, повторив: «Следуй за мной».
\vs p137 2:8 \pc Так Иисус собрал уже половину будущего отряда ближайших своих сподвижников, пятеро из которых знали его какое\hyp{}то время и один, Нафанаил, был незнакомец. Без дальнейших промедлений они пересекли Иордан и, миновав деревню Наин, поздним вечером достигли Назарета.
\vs p137 2:9 Там они переночевали у Иосифа в доме, где провел детство Иисус. Сподвижники Иисуса не поняли, почему их новообретенный учитель так заботился о том, чтобы полностью уничтожить оставшиеся в доме в форме Десяти Заповедей, а также других правил и принципов, малейшие следы своих писаний. Однако его действия, в соединении с тем, что впоследствии они никогда не видели его пишущим на чем\hyp{}либо, кроме песка или пыли, глубоко запечатлелись в их сознании.
\usection{3. Посещение Капернаума}
\vs p137 3:1 На следующий день Иисус послал своих апостолов в Кану, поскольку все они были приглашены на свадьбу знатной молодой женщины этого города, а сам он собирался ненадолго навестить свою мать в Капернауме, перед тем сделав остановку в Магдале, чтобы увидеться со своим братом Иудой.
\vs p137 3:2 Прежде чем покинуть Назарет, новые сподвижники Иисуса поведали Иосифу и другим членам семьи Иисуса об удивительных событиях недавнего прошлого и открыто высказали свою веру в то, что Иисус является давно ожидаемым спасителем. И эти члены семьи Иисуса обсудили услышанное, а Иосиф сказал: «Может быть, в конце концов мать была права --- наш странный брат --- грядущий царь».
\vs p137 3:3 Иуда присутствовал на крещении Иисуса и, вместе с братом Иаковом, твердо уверовал в миссию Иисуса на земле. Правда, Иаков и Иуда оба оставались в сомнениях относительно характера миссии их брата, но у их матери возродились все ее прежние надежды на Иисуса как на Мессию, сына Давидова, и она поощряла веру сыновей в то, что их брат --- спаситель Израиля.
\vs p137 3:4 \pc Иисус прибыл в Капернаум в понедельник вечером, но не пошел в свой собственный дом, где жили Иаков и мать; он направился прямо в дом Зеведея. Все друзья в Капернауме заметили в нем большую и приятную перемену. Он снова выглядел вполне жизнерадостным и больше похожим на себя, каким был в прежние времена в Назарете. В годы, предшествующие его крещению и периодам одиночества непосредственно до и после него, он становился все более серьезным и погруженным в себя. Теперь он казался всем в точности таким, как прежде. В его облике чувствовались величественная значительность и возвышенность, но он вновь стал беспечным и радостным.
\vs p137 3:5 Мария была сильно взволнована ожиданием. Она предчувствовала, что обещание Гавриила скоро исполнится. Она ждала, что вся Палестина скоро будет поражена и ошеломлена чудесным откровением ее сына как сверхъестественного царя евреев. Но на многочисленные вопросы матери, Иакова, Иуды и Зеведея Иисус только отвечал, улыбаясь: «Будет лучше, если я останусь здесь на некоторое время; я должен выполнять волю моего Отца, который на небесах».
\vs p137 3:6 \pc На следующий день, во вторник, они все отправились в Кану на свадьбу Наоми, которая должна была состояться на другой день. Несмотря на многократные предупреждения Иисуса не говорить о нем никому, «пока не придет назначенный Отцом час», ученики упорно потихоньку широко распространяли весть, что они нашли Спасителя. Каждый из них уверенно ожидал, что на предстоящей свадьбе в Кане Иисус торжественно вступит в свои права на власть как Мессия и проявит при этом грандиозные мощь, силу и величие. Они помнили то, что рассказывали им о событиях, сопровождавших его крещение, и верили, что его дальнейший путь на земле будет отмечен все большими проявлениями сверхъестественных чудес и удивительных событий. Соответственно, вся округа собиралась присутствовать в Кане на свадебном пире Наоми и Иоава, сына Натана.
\vs p137 3:7 Мария уже много лет не была такой радостной. На пути в Кану она чувствовала себя словно королева\hyp{}мать, следующая, чтобы присутствовать на коронации сына. Впервые, с тех пор как Иисусу минуло тринадцать лет, семья и друзья видели его таким беззаботным и счастливым, таким внимательным и чутким к желаниям своих спутников, таким трогательно отзывчивым. И они шептались между собой в маленьких группках о том, что должно произойти. Как поведет себя дальше этот непостижимый человек? Как возвестит он славу грядущего царства? И все они были взволнованы мыслью о том, что станут свидетелями откровения мощи и силы Бога Израиля.
\usection{4. Свадьба в Кане}
\vs p137 4:1 К полудню среды почти тысяча гостей прибыли в Кану --- в четыре с лишним раза больше, чем было приглашено на свадебный банкет. По еврейскому обычаю свадьбы праздновались в среду, и приглашения на свадьбу были разосланы за месяц. В первой половине дня происходящее походило больше на публичный прием Иисуса, чем на свадьбу. Каждый хотел приветствовать этого почти знаменитого галилеянина, а он был чрезвычайно радушен со всеми, старыми и молодыми, евреями и неевреями. И все радовались, когда он согласился возглавить предшествующую свадьбе процессию.
\vs p137 4:2 Иисус к этому времени полностью осознавал свое человеческое существование, свое предшествующее божественное существование и состояние соединенности, или слияния, в себе божественной и человеческой природы. Он мог в любой момент, полностью сохраняя внутреннее равновесие, войти в свою человеческую роль или же личностно воплотиться в божественной природе.
\vs p137 4:3 С течением дня Иисус все больше отдавал себе отчет, что люди ожидают от него совершения чуда; он видел, что в особенности члены его семьи и шестеро учеников\hyp{}апостолов ожидали, что он, как подобает, возвестит свое грядущее царство каким\hyp{}нибудь ошеломляющим и сверхъестественным явлением.
\vs p137 4:4 Вскоре после полудня Мария подозвала Иакова, и они вместе отважились подойти к Иисусу с вопросом, доверяет ли он им настолько, чтобы сообщить, когда и в какой момент свадьбы он собирается проявить себя как «сверхъестественное существо». Не успев даже договорить до конца, они увидели, что вызвали характерное негодование Иисуса. Он сказал только: «Если любите меня, будьте готовы ожидать вместе со мной, как и я жду, воли моего Отца, который на небесах». Но на его лице красноречиво отразился его упрек.
\vs p137 4:5 Этот поступок матери очень разочаровал человека Иисуса; его собственная реакция на косвенно высказанное ею предложение, чтобы он позволил себе какую\hyp{}либо внешнюю демонстрацию своей божественности, оказала на него отрезвляющее действие. Это была одна из тех вещей, которые он во время своего еще недавнего уединения в горах принял решение не делать. Мария в течение нескольких часов была крайне подавлена. Она сказала Иакову: «Я не в силах его понять; что может означать все это? Неужели не будет конца этому странному поведению?» Иаков и Иуда пытались успокоить свою мать, в то время как Иисус удалился на час, чтобы побыть наедине с собой. Но он вернулся в общество вновь веселым и радостным.
\vs p137 4:6 \pc Свадьба проходила в выжидательной тишине, но вот церемония закончена, и ни слова, ни жеста от почетного гостя. Потом начал распространяться шепот, что плотник и лодочный мастер, которого Иоанн объявил «Спасителем», покажет себя во время вечерних торжеств, возможно за свадебным ужином. Но в мыслях шести учеников\hyp{}апостолов не осталось и намека на подобные ожидания после того, как он собрал их вместе перед свадебным ужином и очень серьезно сказал: «Не думайте, что я пришел сюда, чтобы сделать какое\hyp{}нибудь чудо для удовлетворения любопытных или убеждения сомневающихся. Мы здесь находимся в ожидании воли нашего Отца, который на небесах». Однако Мария и остальные, увидев его беседующим со сподвижниками, уже не сомневались, что должно случиться нечто необычайное. И они все уселись, чтобы насладиться свадебным ужином и праздничным вечером в хорошей компании.
\vs p137 4:7 \pc Отец жениха приготовил достаточно вина для всех гостей, приглашенных на брачное торжество, но мог ли он знать, что женитьба его сына станет событием, столь тесно связанным с ожидаемым проявлением Иисуса в качестве Мессии\hyp{}Спасителя? Он был в восторге от чести видеть знаменитого галилеянина среди своих гостей, но еще до того, как свадебный ужин завершился, слуги принесли огорчившее его известие, что вино подходит к концу. К тому времени, когда официальный ужин был закончен и гости разбрелись по саду, мать жениха по секрету сообщила Марии, что запасы вина истощились. И Мария уверенно ответила: «Не беспокойся --- я поговорю с моим сыном. Он поможет нам». Она позволила себе говорить так, несмотря на упрек, полученный лишь несколько часов назад.
\vs p137 4:8 В течение многих лет Мария неизменно искала помощи Иисуса во все критические моменты жизни их семьи в Назарете, поэтому для нее было совершенно естественно и на этот раз подумать о нем. Но в данном случае эта честолюбивая мать имела и другие мотивы позвать своего старшего сына на помощь. Когда Иисус стоял один в углу сада, мать подошла к нему со словами: «Сын мой, у них нет вина». Иисус отвечал: «Добрая женщина, что мне до этого?» Сказала Мария: «Но я уверена, что твой час наступил; разве ты не можешь помочь нам?» Последовал ответ Иисуса: «Вновь говорю, что пришел не для того, чтобы действовать таким образом. Почему ты опять докучаешь мне такими вещами?» И тогда, разрыдавшись, Мария взмолилась: «Но, сын мой, я им обещала, что ты поможешь нам; неужели ты не можешь сделать для меня что\hyp{}то?» И тогда говорил Иисус: «Женщина, как ты можешь давать такие обещания? Смотри, не делай этого больше. Во всем мы должны ожидать воли небесного Отца».
\vs p137 4:9 Мария, мать Иисуса, была глубоко подавлена; она была потрясена! И когда она стояла перед ним неподвижно, со слезами, струящимися по лицу, человеческое сердце Иисуса преисполнилось сочувствия к женщине, которая выносила его во плоти; и наклонившись вперед, он нежно положил руку на ее голову, говоря: «Ну же, матушка моя Мария, не огорчайся так от моих слов, которые кажутся жестокими; разве не говорил я тебе и прежде много раз, что пришел только для того, чтобы выполнять волю моего небесного Отца? С великой радостью сделал бы я то, о чем ты меня просишь, если бы это было по воле Отца\ldots » --- и вдруг Иисус умолк, он колебался. Мария, казалось, заметила: что\hyp{}то происходит. Приподнявшись на цыпочках, она обвила руками шею Иисуса, поцеловала его и кинулась туда, где находились слуги, со словами: «Что скажет мой сын, то делайте». Но Иисус молчал. Он понял, что уже и без того сказал --- или, вернее, мысленно пожелал, --- слишком много.
\vs p137 4:10 Мария танцевала от радости. Она не знала, как будет сотворено вино, но была уверена, что убедила наконец своего первенца утвердить свою власть, решиться выступить вперед, заявить о своем статусе и продемонстрировать свою силу Мессии. И благодаря присутствию и содействию определенных вселенских сил и личностей, о которых никто из бывших там не подозревал, ее не ожидало разочарование. Вино, о котором настоятельно просила Мария и появления которого Богочеловек Иисус, человечно и из сочувствия пожелал, было уже на подходе.
\vs p137 4:11 Поблизости стояли шесть каменных сосудов, наполненных водой, каждый из которых вмещал около 20 галлонов. Вода предназначалась для заключительных очистительных церемоний свадебного торжества. Суета слуг вокруг этих огромных каменных сосудов под деловитым руководством его матери привлекла внимание Иисуса, и, подойдя, он увидел, что они черпают оттуда вино кувшинами.
\vs p137 4:12 Постепенно Иисус начал понимать, что же произошло. Из всех присутствовавших на празднике бракосочетания в Кане Иисус был изумлен больше всех. Остальные ожидали, что он совершит чудо, но это было как раз то, чего он не собирался делать. И тут Сын Человеческий вспомнил предупреждение своего Персонализированного Настройщика, полученное им в горах. Он напомнил себе, как Настройщик предупредил его, что никакая сила или личность не может лишить его привилегии творца, заключающейся в независимости от времени. В данном случае преобразователи мощи, срединники и остальные нужные личности собрались около воды и других требующихся элементов, и ввиду выраженного желания Творца и Владыки Вселенной неизбежно должно было мгновенно появиться \bibemph{вино.} И тем более это должно было произойти, потому что Персонализированный Настройщик дал понять, что исполнение желания Сына ни в коей мере не противоречит воле Отца.
\vs p137 4:13 Но ни в каком смысле это не было чудом. Ни один закон природы не был изменен, исключен или хотя бы нарушен. Не произошло ничего, кроме того, что было аннулировано \bibemph{время} в связи с небесным соединением химических элементов, нужных для выработки вина. В этом случае в Кане помощники Творца изготовили вино точно так же, как они делают это в естественных природных процессах, \bibemph{за исключением} того, что они сделали это независимо от времени и с участием сверхчеловеческих помощников в соединении в пространстве необходимых составляющих химических элементов.
\vs p137 4:14 Далее, несомненно, что осуществление этого так называемого чуда не противоречило воле Райского Отца: иначе оно бы не произошло, так как Иисус уже предал себя всецело воле Отца.
\vs p137 4:15 \pc Когда слуги зачерпнули этого нового вина и принесли его шаферу, «правителю праздника» и когда он попробовал его, то подозвал жениха и сказал: «По обычаю, сначала дают хорошее вино, а потом, когда гости уже как следует выпили, --- приносят худшие плоды лозы; а ты лучшее вино приберег до самого конца пиршества».
\vs p137 4:16 Мария и ученики Иисуса очень обрадовались воображаемому чуду, которое, как они думали, Иисус намеренно совершил, но Иисус удалился в укромный уголок сада и на краткие моменты погрузился в серьезное размышление. В конце концов он пришел к выводу, что в данных обстоятельствах случившееся находилось вне его личного контроля и, поскольку не противоречило воле Отца, было неизбежно. Когда он вернулся к людям, они взирали на него с благоговением; все они поверили в него как в Мессию. Но Иисус был чрезвычайно обеспокоен, зная, что они поверили в него только из\hyp{}за необычного события, которому только что невольно были свидетелями. Снова Иисус уединился на некоторое время, уйдя на крышу дома, чтобы как следует все обдумать.
\vs p137 4:17 Иисус теперь полностью осознал, что должен всегда контролировать себя, иначе симпатия и жалость вновь будут приводить к подобным эпизодам. Тем не менее, еще много подобных событий произошло, прежде чем Сын Человеческий окончательно завершил свою смертную жизнь в плотском облике.
\usection{5. Снова в Капернауме}
\vs p137 5:1 Хотя многие гости оставались еще на целую неделю свадебных празднеств, Иисус со своими недавно избранными учениками\hyp{}апостолами --- Иаковом, Иоанном, Андреем, Петром, Филиппом и Нафанаилом --- рано утром следующего дня ушли в Капернаум, не простившись ни с кем. Семья Иисуса и все его друзья в Кане были очень расстроены тем, что он так внезапно покинул их, и Иуда, самый младший брат Иисуса, отправился на поиски его. Иисус и его апостолы пошли прямо в дом Зеведея в Вифсаиде. По дороге туда он говорил со своими недавно избранными сподвижниками о многих вещах, важных для грядущего царства, и особо предупредил их, чтобы они не упоминали о превращении воды в вино. Он также посоветовал им в их будущей работе избегать городов Сефориса и Тивериады.
\vs p137 5:2 После ужина в тот вечер, в доме Зеведея и Саломеи, произошло одно из самых важных собраний всего земного пути Иисуса. Только шестеро апостолов присутствовали на этой встрече; брат Иуда прибыл, когда они уже были готовы разойтись. Эти шестеро избранных мужчин путь из Каны в Вифсаиду проделали словно на крыльях. Они были полны ожиданий и трепетали от мысли о том, что избраны в качестве близких сподвижников Сына Человеческого. Но когда Иисус старался дать им понять, кто он, какова его миссия на земле и как может она завершиться, они были ошеломлены. Они не могли осознать того, что он говорил им. Они не в силах были вымолвить ни слова; даже Петр был невыразимо подавлен. Лишь вдумчивый Андрей отважился что\hyp{}то сказать в ответ на слова, произнесенные Иисусом. Когда Иисус понял, что они не воспринимают его вести, когда он увидел, что их представления об Еврейском Мессии являются совершенно устоявшимися, то он отправил их отдохнуть, сам же вышел пройтись и поговорить со своим братом Иудой. Иуда, прежде чем проститься с Иисусом, сказал с большим чувством: «Мой отец\hyp{}брат, я никогда не мог понять тебя. Я не уверен, что ты именно тот, о ком учила нас думать мать, и я не вполне постигаю грядущее царство, но я знаю, что ты могущественный человек Бога. Я слышал голос на Иордане и я верую в тебя, кто бы ты ни был». И сказав так, он ушел и отправился в свой дом в Магдалу.
\vs p137 5:3 Этой ночью Иисус не спал. Одев вечернюю накидку, он сидел на берегу озера, и думал, думал до рассвета следующего дня. В долгие часы тех ночных размышлений он ясно осознал, что ему никогда не удастся заставить своих последователей воспринимать его иначе, чем давно ожидавшегося Мессию. Наконец он осознал, что нет иного пути открыть дорогу своему посланию о царстве, чем представив его как осуществление пророчества Иоанна, а самого себя тем, кого так долго ожидали евреи. В конце концов, хотя он не является Мессией Давидова типа, он действительно исполняет собой пророчества тех из провидцев древности, которые обладали наиболее духовным видением. Никогда с тех пор он не отрицал полностью, что является Мессией. Он решил предоставить окончательное разрешение этой сложной ситуации на волю Отца.
\vs p137 5:4 На следующее утро Иисус присоединился к своим друзьям за завтраком, но это была унылая компания. Он провел с ними некоторое время, а по завершении трапезы собрал их вокруг себя и сказал: «Воля моего Отца, чтобы мы задержались здесь на некоторое время. Вы слышали слова Иоанна, что он пришел приготовить путь для царства; поэтому надлежит нам ожидать завершения его проповеди. Когда предтеча Сына Человеческого исполнит свою работу, мы начнем возвещать благую весть о царстве». Он направил апостолов вернуться к рыбной ловле, в то время как сам собрался идти с Зеведеем на лодочную верфь, пообещав им встретиться на следующий день в синагоге, где он должен был проповедовать, и назначив собрание в тот Субботний день после полудня.
\usection{6. События Субботнего дня}
\vs p137 6:1 Первое публичное выступление Иисуса после его крещения состоялось в синагоге Капернаума в Субботу, 2 марта 26 г. н.э. Синагога была переполнена. К истории крещения на Иордане прибавились теперь свежие новости из Каны о воде и вине. Иисус усадил на почетные места своих шестерых апостолов, рядом с ними разместились его кровные братья Иаков и Иуда. Его мать, вернувшаяся в Капернаум с Иаковом накануне вечером, сидела в женской части синагоги. Все собравшиеся были в нетерпении; они ожидали стать очевидцами новых необычайных проявлений сверхъестественной силы, которые были бы надлежащим свидетельством сущности и власти того, кто должен был сегодня говорить перед ними. Но им суждено было испытать разочарование.
\vs p137 6:2 Когда Иисус встал, управитель синагоги протянул ему свиток Писания, и он стал читать из пророка Исайи: «Так говорит Господь: „Небо --- престол Мой, а земля --- подножие ног Моих; где же построите вы дом для Меня, и где место покоя Моего? Ибо все это соделала рука Моя, и все сие было“, --- говорит Господь. --- „А вот на кого я призрю: на смиренного и сокрушенного духом и на трепещущего пред словом Моим“. Выслушайте слово Господа, трепещущие пред словом Его. „Ваши братья, ненавидели вас и изгоняли вас за имя Мое“. Но славен Господь. Он явится перед вами в радости, и все прочие будут постыжены. Вот, голос из города, голос из храма, голос Господа говорит: „Еще не мучилась родами, а родила; прежде нежели наступили боли ее, разрешилась сыном“. Кто слыхал таковое? Кто видал подобное этому? Возникала ли страна в один день? Рождался ли народ в один раз? Ибо так говорит Господь: „Вот, я направляю мир, как реку, и даже богатство язычников --- как разливающийся поток. Как утешает кого\hyp{}либо мать его, так утешу Я вас, и вы будете утешены в Иерусалиме. И вы увидите это, и возрадуется сердце ваше“».
\vs p137 6:3 Закончив чтение, Иисус вернул свиток его хранителю. Прежде чем сесть, он сказал просто: «Будьте терпеливы, и вы увидите славу Господа; это будет со всеми, кто ожидает вместе со мной и так учится выполнять волю моего Отца, который на небесах». И люди ушли по домам, размышляя о значении всего этого.
\vs p137 6:4 \pc После полудня в тот день Иисус и его апостолы, вместе с Иаковом и Иудой, вошли в лодку и немного отплыли от берега, где стали на якорь, и он говорил с ними о грядущем царстве. И они поняли больше, чем в четверг вечером.
\vs p137 6:5 Иисус наказал им выполнять свои повседневные обязанности, пока «не наступит время царства». Чтобы подбодрить их, он привел собственный пример, сказав, что возвращается к обычной работе в лодочной мастерской. Объясняя, что им следует проводить три часа каждый вечер за учением и подготовкой к будущей деятельности, Иисус далее сказал: «Мы все будем держаться рядом, пока Отец не повелит мне призвать вас. Каждый из вас должен теперь вернуться к своей обычной работе, как если бы ничего не произошло. Не говорите обо мне никому и помните, что мое царство должно прийти не благодаря шумным толкам и волшебным чарам, а через великое изменение, которое Отец мой сотворит в ваших сердцах и в сердцах тех, кто будет призван вместе с вами в совет царства. Вы теперь мои друзья; я доверяю вам и люблю вас; скоро вы станете моими личными сподвижниками. Будьте терпеливы, будьте кротки. Всегда будьте послушны воле Отца. Приготовьтесь к зову царства. Хотя вы испытаете великую радость в служении моему Отцу, но вы должны быть готовы и к горестям, ибо предупреждаю вас, что только через большие страдания многие войдут в царство. Но для тех, кто обретет царство, радость будет совершенной, и они будут названы благословленными земли. Но не питайте ложной надежды; мир усомнится в моих словах. Даже вы, мои друзья, не вполне понимаете то, что я раскрываю перед вашими смущенными умами. Не совершайте ошибки; нам предстоит трудиться для поколения, ищущего знамений. Они будут требовать совершения чудес в доказательство того, что я послан Отцом, и много времени пройдет, прежде чем они признают откровение любви моего Отца как доказательство моей миссии».
\vs p137 6:6 В тот вечер, когда они вернулись на берег, прежде чем разойтись, Иисус, стоя у самой воды, молился: «Мой Отец, благодарю тебя за малых сих, которые, несмотря на их сомнения, даже сейчас веруют. Ради них отрешился я от всего, чтобы выполнять твою волю. Пусть научатся они быть одним, как мы с тобой есть одно».
\usection{7. Четыре месяца подготовки}
\vs p137 7:1 Четыре долгих месяца --- март, апрель, май, июнь --- продолжалось это время ожидания. Иисус провел больше сотни долгих и серьезных, однако светлых и радостных бесед со своими шестью сподвижниками и братом Иаковом. Из\hyp{}за болезни в семье Иуда редко мог посещать занятия. Иаков, брат Иисуса, не терял веру в него, но Мария за эти месяцы промедления и бездействия почти отчаялась в своем сыне. Ее вера, поднявшаяся так высоко в Кане, теперь снова была почти потеряна. Все, что она смогла, это вернуться к своим так часто повторявшимся сетованиям: «Я не могу понять его. Я не могу представить себе, что все это значит». Но жена Иакова очень старалась укрепить мужество Марии.
\vs p137 7:2 В течение четырех месяцев эти семеро верующих, один из них --- собственный брат во плоти --- знакомились с Иисусом; они свыкались с мыслью о том, что живут бок о бок с этим Богочеловеком. Хотя они называли его Рабби, они учились не бояться его. Иисус обладал тем несравненным очарованием личности, которое позволяло ему жить среди них, не угнетая их своей божественностью. Для них оказалось действительно легко быть «друзьями Бога», Бога, воплощенного в облике смертного человека. Это время ожидания явилось суровой проверкой для всей группы уверовавших. Не происходило ничего, абсолютно ничего чудесного. День за днем они выполняли свою обычную работу, вечер за вечером сидели у ног Иисуса. И их собирала вместе его несравненная личность и благодатные слова, которыми он говорил с ними вечер за вечером.
\vs p137 7:3 Это время ожидания и обучения было особенно тяжко для Симона Петра. Неоднократно пытался он убедить Иисуса начать проповедь царства в Галилее, пока Иоанн продолжал проповедовать в Иудее. Но Иисус всегда отвечал Петру: «Будь терпелив, Симон. Продвигайся вперед. Ничего из этого не будет лишним, когда Отец призовет нас». И Андрей то и дело успокаивал Петра своими более выдержанными и философскими советами. На Андрея производила огромное впечатление человеческая естественность Иисуса. Он никогда не уставал размышлять о том, как тот, кто может жить так близко к Богу, может относиться с таким дружелюбием и вниманием к людям.
\vs p137 7:4 За все это время Иисус лишь дважды говорил в синагоге. К концу этих многих недель ожидания молва о его крещении и о вине в Кане стала затихать. И Иисус следил за тем, чтобы никаких явных чудес больше не происходило за это время. Но хотя они жили тихо в Вифсаиде, известия о необычных деяниях Иисуса были принесены Ироду Антипе, который послал своих шпионов разведать, каковы его намерения. Впрочем, Ирода больше беспокоили проповеди Иоанна. Он решил не преследовать Иисуса, который продолжал столь спокойно работать в Капернауме.
\vs p137 7:5 В это время ожидания Иисус ставил себе целью научить своих сподвижников тому, как относиться к различным религиозным группам и политическим партиям Палестины. Он неизменно говорил: «Мы стремимся привлечь всех их, но \bibemph{не принадлежим} ни к одной из них.»
\vs p137 7:6 \pc Книжников и раввинов, вместе взятых, называли фарисеями. Они именовали себя «сообщники». Во многих отношениях они были прогрессивной группой среди евреев, восприняв учения, не раскрытые ясно в еврейском Писании, такие, как вера в воскресение мертвых, --- доктрину, лишь упомянутую поздним пророком Даниилом.
\vs p137 7:7 \pc К саддукеям относились священство и некоторые богатые евреи. Они не так рьяно настаивали на точном соблюдении закона. Фарисеи и саддукеи в действительности были скорее религиозными партиями, нежели сектами.
\vs p137 7:8 \pc Ессеи были настоящей религиозной сектой, зародившейся во время восстания Маккавеев; их требования в некоторых отношениях были более строгими, чем у фарисеев. Они восприняли многие персидские верования и обряды, жили монастырскими братствами, воздерживались от брака, все имущество у них было общим. Они особо занимались учением об ангелах.
\vs p137 7:9 \pc Зилоты были группой горячих еврейских патриотов. Они утверждали, что любые методы оправданы в борьбе за избавление от римского ига.
\vs p137 7:10 \pc Иродиане были чисто политической партией, выступавшей за выход из\hyp{}под прямого римского правления путем восстановления династии Ирода.
\vs p137 7:11 \pc В самом сердце Палестины жили самаряне, с которыми «евреи не имели дела», несмотря на то, что многие их воззрения были сходны с еврейскими учениями.
\vs p137 7:12 \pc Все эти партии и секты, включая маленькое назорейское братство, верили в будущее пришествие Мессии. Все они ждали национального спасителя. Однако Иисус вполне определенно дал понять, что он и его ученики не должны вступать в союз ни с одной из этих философских или религиозных школ. Сын Человеческий не должен был быть ни назореем, ни ессеем.
\vs p137 7:13 Когда впоследствии Иисус посылал своих апостолов, подобно Иоанну, проповедовать учение и наставлять верующих, он придавал особое значение провозглашению «благих вестей о царствие небесном». Он неизменно внушал своим сподвижникам, что они должны «проявлять любовь, сострадание и симпатию». Он с самого начала учил своих последователей тому, что царство небесное --- это духовный опыт, связанный с воцарением Бога в сердцах людей.
\vs p137 7:14 Пока они так ожидали, прежде чем начать активную публичную проповедь, Иисус и семеро учеников проводили каждую неделю по два вечера в синагоге, изучая иудейское Писание. В последующие годы, после периодов активного публичного служения, апостолы вспоминали эти четыре месяца как самое ценное и плодотворное время во всем их общении с Учителем. Иисус научил этих людей всему, что они могли усвоить. Он не перегружал их учением, не давал то, что они были неспособны воспринять. Он не провоцировал хаос в их умах, сообщая им истины, выходящие далеко за пределы их понимания.
\usection{8. Проповедь о Царстве}
\vs p137 8:1 В Субботу, 22 июня, незадолго до того, как они отправились в свое первое проповедническое путешествие, и примерно через 10 дней после заключения в тюрьму Иоанна, Иисус вступил на синагогальную кафедру второй раз с тех пор, как привел своих апостолов в Капернаум.
\vs p137 8:2 За несколько дней до этой проповеди на тему «О Царстве», когда Иисус занимался своей работой на лодочной верфи, Петр принес ему известие об аресте Иоанна. Иисус вновь отложил свои инструменты, снял свой рабочий фартук и сказал Петру: «Час Отца наступил. Приготовимся проповедовать евангелие царства».
\vs p137 8:3 Последнюю свою работу за плотницким верстаком Иисус выполнил в предыдущий вторник 18 июня, 26 года н.э. Петр стремглав бросился из мастерской, к середине послеполуденного времени он обежал всех своих сотоварищей и, оставив их в роще на берегу, отправился на поиски Иисуса. Но он не мог найти его, потому что Учитель пошел в другую рощу для молитвы. И они не видели его до позднего вечера этого дня, когда он вернулся в дом Зеведея и попросил поесть. На следующий день он послал брата Иакова с просьбой предоставить ему возможность выступить в синагоге в ближайшую Субботу. Управитель синагоги был очень доволен тем, что Иисус вновь готов провести службу.
\vs p137 8:4 \pc Прежде чем начать эту незабываемую проповедь о царстве Божьем, которая была первым важным свершением его публичной деятельности, Иисус прочел из Писания следующий отрывок: «Будешь для меня царством священников, святых людей. Ибо Яхве судия наш, Яхве --- законодатель наш, Яхве --- царь наш; Он спасет нас. Яхве мой царь и мой Бог. Он великий Царь над всей землей. Любовь и милость над Израилем в этом царстве. Благословенна слава Господня, ибо он --- наш Царь».
\vs p137 8:5 Когда Иисус закончил читать, он сказал:
\vs p137 8:6 \pc «Я пришел возвестить установление царства Отца. И это царство примет почитающие Бога души евреев и неевреев, богатых и бедных, свободных и рабов, ибо Отец мой не взирает на лица; его любовь и милосердие надо всеми.
\vs p137 8:7 Отец Небесный посылает свой дух жить в умах людей, и когда я закончу мою работу на земле, подобным образом изольется Дух Истины на всю плоть. И дух Отца моего и Дух Истины утвердят вас в грядущем царстве духовного понимания и божественной праведности. Царство мое не от этого мира. Сын Человеческий не поведет армии в битву за установление престола власти или царства мирской славы. Когда наступит мое царство, вы узнаете Сына Человеческого как Принца Мира, как откровение вечного Отца. Дети этого мира борются за установление и расширение царств этого мира, но мои ученики войдут в царство небесное благодаря своим нравственным решениям и своим духовным победам; и, однажды войдя в него, они обретут радость, праведность и вечную жизнь.
\vs p137 8:8 Те, кто стремиться войти в царство, тем самым начинают бороться за величие натуры, подобное тому, что у моего Отца, вскоре обретут все, в чем они нуждаются. Но говорю вам со всей искренностью: если не ищете входа в царство с верой и простосердечным упованием малого ребенка, никогда не сможете войти в него.
\vs p137 8:9 Не обманывайтесь теми, кто приходят со словами о том, что царство здесь или там, ибо царство моего Отца не от вещей видимых и материальных. И это царство находится даже сейчас среди вас, ибо где дух Бога учит и наставляет души человеческие, там воистину есть царство небесное. И это царство Бога есть праведность, мир и радость в Духе Святом.
\vs p137 8:10 Иоанн воистину крестил вас в знак покаяния и отпущения ваших грехов, но когда вы войдете в царство небесное, будете крещены Святым Духом.
\vs p137 8:11 В царстве Отца моего не будет ни еврея, ни нееврея, а только те, кто ищут совершенства через служение, ибо говорю, что нужно сначала стать слугою для всех, чтобы быть великим в царстве моего Отца. Если вы готовы служить своим собратьям, в царстве моем вы будете рядом со мной, так же, как я благодаря служению в облике создания вскоре буду рядом с Отцом в его царстве.
\vs p137 8:12 Это новое царство подобно семени, растущему на доброй почве в поле. Оно не приносит быстро зрелого плода. Должно пройти время между установлением царства в душе человека и тем часом, когда царство принесет плод непреходящей праведности и вечного спасения.
\vs p137 8:13 И это царство, которое я возвещаю вам, не есть господство силы и богатства. Царство небесное --- это не еда и питье, а жизнь в возрастающей праведности и все большей радости служения моему Небесному Отцу. Ибо не сказал ли Отец о своих детях мирских: „Моя воля в том, чтобы они стали совершенны, так же, как я совершенен“.
\vs p137 8:14 Я пришел возвестить радостные вести о царстве. Я пришел не для того, чтобы увеличить тяжкую ношу тех, кто пожелает войти в него. Я возвещаю новый и лучший путь, и те, кто способен войти в грядущее царство, насладятся божественным отдыхом. Сколько бы ни стоило это вам в мирских ценностях, какую бы цену ни заплатили вы за то, чтобы войти в небесное царство, вы будете вознаграждены стократ радостью и духовным ростом в этом мире и вечной жизнью в грядущей эпохе.
\vs p137 8:15 Доступ в царство Отца не требует выступления армий, ниспровержения царств этого мира или сбрасывания рабских цепей. Небесное царство находится рядом с вами, вошедшие в него обретут обильную свободу и радостное спасение.
\vs p137 8:16 Власть этого царства вечна. Те, кто входят в него, вознесутся к Отцу; они будут сидеть по правую руку его в Раю, осененные его славой. И все, кто войдут в царство небесное, станут сынами Бога, и в наступающую эпоху взойдут к нему. И я пришел не звать мнящих себя праведниками, но грешников и всех, кто алчет и жаждет праведности божественного совершенства.
\vs p137 8:17 Иоанн пришел проповедовать покаяние, чтобы подготовить вас к царству; ныне я пришел, возвещая о вере как о даре Бога, как об условии вхождения в царство небесное. Если вы верите, что мой Отец любит вас бесконечной любовью, вы в царстве Бога».
\vs p137 8:18 \pc Сказав так, он сел. И все, кто слушал его, были изумлены его словами. Его ученики дивились. Но люди не были готовы получить добрые вести из уст этого Богочеловека. Примерно треть слышавших его поверили его посланию, хотя не могли полностью его понять; примерно треть приготовились в своих сердцах отвергнуть такое чисто духовное представление об ожидаемом царстве, в то время как остальные не смогли воспринять его учение, и многие из них были искренне уверены, что он «не в себе».
