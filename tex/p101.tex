\upaper{101}{Подлинная сущность религии}
\author{Мелхиседек}
\vs p101 0:1 Религия как человеческий опыт простирается от примитивного, основанного на страхе рабства эволюционирующего первобытного человека до высокой и величественной свободы веры тех цивилизованных смертных, кто прекрасно сознает свое сыновство по отношению к вечному Богу.
\vs p101 0:2 Религия является предком развитой этики и морали постепенной социальной эволюции. Однако религия как таковая --- это не просто нравственное движение, хотя внешние и социальные проявления религии подвержены сильному влиянию со стороны этической и нравственной движущей силы человеческого общества. Религия всегда вдохновляет совершенствующуюся человеческую природу, но тайной этой эволюции не является.
\vs p101 0:3 Религия, вера\hyp{}убеждение личности, всегда может восторжествовать над поверхностно противоречивой логикой отчаяния, рожденного в неверующем материальном уме. Истинный и подлинный внутренний голос, этот «великий свет, что светит всякому человеку, приходящему в мир», действительно существует. Причем это духовное водительство отличается от этического побуждения человеческой совести. Чувство религиозной уверенности есть нечто большее по сравнению с эмоциональным чувством. Уверенность, которую дает религия, превосходит рассуждения ума и даже логику философии. Религия \bibemph{есть} вера, упование и уверенность.
\usection{1. Истинная религия}
\vs p101 1:1 Истинная религия --- это не система философских верований, к которой можно прийти логическим путем и в поддержку которой можно привести естественные доказательства; не является она и фантастическим и мистическим переживанием неописуемых экстатических чувств, которые могут испытывать лишь романтические приверженцы мистицизма. Религия --- это не продукт разума, но при рассмотрении изнутри она вполне разумна. Религия --- отнюдь не производная от логики человеческой философии, но как опыт смертного вполне логична. Религия --- это ощущение божественного в сознании смертного существа, обладающего эволюционной природой, и представляет истинный опыт, связанный с вечными реальностями во времени, достижение духовного удовлетворения еще при жизни во плоти.
\vs p101 1:2 \P\ У Настройщика Мысли нет особого механизма, посредством которого он бы мог достичь самовыражения; ведь мистического религиозного дара для восприятия или выражения религиозных чувств не существует. Эти переживания доступны благодаря предопределенному природой механизму смертного разума. В этом и кроется одно из объяснений трудности, которую испытывает Настройщик при осуществлении прямого общения с материальным разумом, где он постоянно пребывает.
\vs p101 1:3 Божественный дух общается со смертным человеком не посредством чувств или эмоций, но в области высшего и наиболее одухотворенного мышления. Именно ваши \bibemph{мысли,} а не ваши чувства ведут вас к Богу. Божественная сущность может восприниматься только глазами разума. Однако разум, по\hyp{}настоящему видящий Бога и слышащий пребывающего в нем Настройщика, есть чистый разум. «Без святости никто не может видеть Господа». Всякое подобное внутреннее и духовное общение называется духовным пониманием. Такие религиозные переживания происходят от впечатления, которое оставляют в уме смертного совместные действия Настройщика и Духа Истины, действующих на основе и в среде идей, идеалов, пониманий и духовных стремлений совершенствующихся сыновей Бога.
\vs p101 1:4 Следовательно, религия живет и процветает не благодаря зрению и чувству, но благодаря вере и пониманию. Она заключается не в открытии новых фактов или в отыскании уникального опыта, но в обнаружении новых и духовных \bibemph{значений} в фактах, уже известных человечеству. Высший религиозный опыт не зависит от предшествующих ему деяний веры, традиций и власти; не является и религия порождением возвышенных чувств и чисто мистических эмоций. Скорее, это --- чрезвычайно глубокий и подлинный опыт духовного общения с духовными силами, действующими в уме человека, и, насколько такой опыт поддается определению в терминах психологии, он попросту является опытом переживания реальности веры в Бога, как реальности такого сугубо личного опыта.
\vs p101 1:5 \P\ Хотя религия и не есть продукт рационалистических рассуждений материальной космологии, она, тем не менее, творение всецело рационального понимания, возникающего в опыте человеческого разума. Религия не является порождением ни мистических медитаций, ни уединенных размышлений, хотя она всегда в большей или меньшей степени таинственна и всегда неопределима и необъяснима в терминах чисто интеллектуальных рассуждений и философской логики. Зачатки истинной религии возникают в области нравственного сознания человека и раскрываются в развитии человеческого духовного понимания, в той способности человеческой личности, которая возрастает вследствие присутствия открывающего Бога Настройщика Мысли в жаждущем Бога смертном разуме.
\vs p101 1:6 Вера объединяет нравственное понимание с тщательным различением ценностей, а предшествующее ему эволюционное чувство долга завершает родословную истинной религии. Религиозный опыт в конечном итоге приводит к определенному сознанию Бога и несомненной уверенности в продолжении существования верующей личности.
\vs p101 1:7 Таким образом, видно, что религиозные желания и духовные стремления обладают отнюдь не той природой, которая бы просто вынуждала людей \bibemph{хотеть} верить в Бога, но обладают такой природой и силой, что у людей возникает глубокое убеждение, что они верить в Бога \bibemph{должны.} Чувство эволюционного долга и обязательства, вытекающие из освещения откровения, производят на нравственную природу человека такое глубокое впечатление, что он в конце концов достигает такого расположения ума и такого душевного состояния, при которых он приходит к заключению, что \bibemph{не верить в Бога он не имеет права.} Более высокая и выходящая за пределы философии мудрость таких просвещенных и дисциплинированных индивидуумов окончательно убеждает их в том, что сомневаться в Боге или не доверять его доброте --- то же самое, что быть неверным \bibemph{реальнейшей и глубочайшей} вещи в человеческом разуме и в душе --- божественному Настройщику.
\usection{2. Факт религии}
\vs p101 2:1 Факт религии целиком и полностью заключается в религиозном опыте рациональных и обычных людей. Причем это --- единственный смысл, в котором религия вообще может считаться чем\hyp{}то научным или даже психологическим. Доказательство того, что откровение --- это откровение, является тем же фактом человеческого опыта, фактом, который состоит в том, что откровение синтезирует явно несходные естественные науки и религиозную теологию в последовательную и логическую вселенскую философию, единое и согласованное объяснение и науки, и религии и, таким образом, порождает гармонию разума и удовлетворение духа, которые в человеческом опыте отвечают на вопросы смертного ума, стремящегося узнать, \bibemph{как} Бесконечный осуществляет свою волю и планы в материи, с умами и воздействуя на дух.
\vs p101 2:2 Рассуждение --- это метод науки; вера --- метод религии; логика --- пробный метод философии. Откровение компенсирует отсутствие моронтийной точки зрения, предоставляя метод достижения единства в понимании реальности и отношений материи и духа, благодаря посредничеству ума. Причем истинное откровение никогда не представляет науку неестественной, религию --- неразумной, а философию --- нелогичной.
\vs p101 2:3 Рассуждение через изучение науки может привести через природу обратно к Первопричине, однако для того, чтобы преобразовать Первопричину науки в Бога спасения, необходима религиозная вера; откровение же необходимо для обоснования такой веры, такого духовного понимания.
\vs p101 2:4 Существует две основные причины верить в Бога, благоприятствующего человеческому спасению. Это:
\vs p101 2:5 \ublistelem{1.}\bibnobreakspace Человеческий опыт; личная уверенность и так или иначе сознаваемые надежда и упование, которые порождает пребывающий в человеке Настройщик Мысли.
\vs p101 2:6 \ublistelem{2.}\bibnobreakspace Откровение истины, данное прямым личным служением Духа Истины, пришествием в мир Божественных Сыновей или откровениями письменного слова.
\vs p101 2:7 \P\ Наука завершает свой поиск причины в гипотезе о Первопричине. Религия же не прекращает свой полет веры, пока не обретет уверенность в Боге спасения. Из глубокого научного исследования логически следует реальность и существование Абсолюта. Религия же беззаветно верит в существование и реальность Бога, благоприятствующего продолжению существования личности. Что совершенно не удается метафизике и чего отчасти не может даже философия, то делает откровение; то есть утверждает, что Первопричина науки и Бог спасения религии есть \bibemph{одно и то же Божество.}
\vs p101 2:8 \P\ Рассуждение --- это доказательство науки, вера --- доказательство религии, логика --- доказательство философии; откровение же подтверждается лишь человеческим \bibemph{опытом.} Наука дает знание; религия дает счастье; философия дает единство; откровение же подтверждает эмпирическую гармонию сего триединого подхода к вселенской реальности.
\vs p101 2:9 Созерцание природы может открыть только Бога природы, Бога движения. Природа являет лишь материю, движение и оживление --- жизнь. Сумма материи и энергии при определенных условиях проявляется в живых формах, однако хотя естественная жизнь как явление относительно непрерывна, она в плане индивидуальных особенностей полностью преходяща. Природа отнюдь не создает почву для логической веры в продолжение существования человеческой личности. Религиозный человек, находящий Бога в природе, уже нашел того же личного Бога в своей душе и притом сделал это в первую очередь.
\vs p101 2:10 \P\ Вера открывает Бога в душе. Откровение, замена моронтийного понимания в эволюционном мире, позволяет человеку видеть в природе того же самого Бога, которого вера являет в его душе. Таким образом, откровение с успехом перекидывает мост через пропасть между материальным и духовным, между творением и Творцом, между человеком и Богом.
\vs p101 2:11 Созерцание природы логически указывает в направлении разумного водительства и даже живого руководства, но никоим удовлетворительным образом не открывает личного Бога. С другой стороны, природа не раскрывает ничего, что мешало бы рассматривать вселенную как произведение Бога, о котором говорит религия. Бога нельзя найти только через природу, однако, поскольку человек все же нашел его, изучение природы становится полностью совместимым с высшим и более духовным толкованием вселенной.
\vs p101 2:12 \P\ Откровение как эпохальное явление периодично, а как личный человеческий опыт --- непрерывно. Божественное действует в смертной личности как Настройщик, дарованный Отцом, как Дух Истины Сына и как Святой Дух Вселенского Духа; при этом эти три надсмертных дара объединяются в человеческой основанной на опыте эволюции как служение Верховного.
\vs p101 2:13 Истинная религия --- это понимание реальности, дитя веры нравственного сознания, а не просто интеллектуальное согласие с некой совокупностью догматических доктрин. Истинная религия состоит в переживании того, что «Дух сам свидетельствует духу нашему, что мы --- дети Бога». Религия заключается отнюдь не в теологических утверждениях, но в духовном понимании и возвышенности душевного упования.
\vs p101 2:14 Твоя глубочайшая сущность --- божественный Настройщик --- создает внутри тебя голод и жажду праведности, определенное стремление к божественному совершенству. Религия --- это деяние веры, являющееся признанием этого внутреннего стремления к достижению божественности; отсюда и возникают то душевное упование и та уверенность, которые ты начинаешь сознавать как путь к спасению, как способ продолжения существования личности, и все те ценности, которые ты стал рассматривать как истинные и благие.
\vs p101 2:15 \P\ Становится ли религия реальностью в жизни человека, никогда не зависело и никогда не будет зависеть от большой учености или ясной логики. Это --- духовное понимание, и именно поэтому некоторые величайшие религиозные учителя мира и даже пророки иногда обладали столь малой мирской мудростью. Религиозная вера одинаково доступна и ученым, и неученым.
\vs p101 2:16 Религия должна всегда быть своим же собственным критиком и судьей; ее невозможно наблюдать, а тем более понимать, глядя снаружи. Твоя единственная уверенность в личном Боге заключается в твоем собственном понимании своей веры в духовное и в переживании его. Твоим собратьям, имевшим похожий опыт, не нужны аргументы относительно личности или реальности Бога, тогда как для всех остальных людей, не имеющих уверенности в Боге, никакие аргументы никогда не будут истинно убедительными.
\vs p101 2:17 Психология может действительно пытаться изучать явления религиозного отношения к социальному окружению, но она не может надеяться на то, что ей удастся проникнуть в подлинные и внутренние мотивы и действия религии. Только теология, область веры и метод откровения, могут вообще дать какое бы то ни было разумное описание сущности и содержания религиозного опыта.
\usection{3. Характерные особенности религии}
\vs p101 3:1 Религия жизнеспособна настолько, что продолжает существовать и при отсутствии учения. Она живет, несмотря на засорение ее ошибочными космологиями и ложными философиями, выдерживая даже метафизическую путаницу. И во всех исторических переменах, происходящих в религии, в ней сохраняется то, что необходимо для человеческого прогресса и продолжения существования: этическая совесть и нравственное сознание.
\vs p101 3:2 Рожденное верой понимание, или духовная интуиция --- это дар космического разума, связанного с Настройщиком Мысли, который является даром Отца человеку. Духовное рассуждение, разумение души --- это дарование Святого Духа, дар Творческого Духа человеку. Духовная философия, мудрость духовных реальностей --- это дарование Духа Истины, совместный дар совершивших пришествие Сынов детям человеческим. Причем соотнесение и взаимосвязь этих духовных даров делают человека в его потенциальной судьбе духовной личностью.
\vs p101 3:3 Это --- та же самая духовная личность в примитивной и зачаточной форме, обладание которой Настройщиком переживает естественную смерть во плоти. Эта составная сущность духовного происхождения в сочетании с человеческим опытом, благодаря живому пути, проложенному божественными Сынами, получает возможность пережить (хранимой Настройщиком) распад материального «я» разума и материи, когда такое кратковременное партнерство материального и духовного разъединяется прекращением жизненного движения.
\vs p101 3:4 Посредством религиозной веры душа человека раскрывает себя и побуждая смертную личность реагировать на определенные трудные интеллектуальные и сложные социальные ситуации, демонстрирует божественность своей возникающей сущности. Подлинно духовная вера (истинно нравственное сознание) открывается в том, что она:
\vs p101 3:5 \ublistelem{1.}\bibnobreakspace Заставляет этику и мораль развиваться вопреки врожденным и вредным животными наклонностям.
\vs p101 3:6 \ublistelem{2.}\bibnobreakspace Рождает возвышенное упование на доброту Бога даже перед лицом горького разочарования и сокрушительного поражения.
\vs p101 3:7 \ublistelem{3.}\bibnobreakspace Вырабатывает большую смелость и глубокую уверенность вопреки естественным несчастьям и физическим бедствиям.
\vs p101 3:8 \ublistelem{4.}\bibnobreakspace Являет необъяснимую уравновешенность и стойкое спокойствие, несмотря на тяжелые болезни и даже острое физическое страдание.
\vs p101 3:9 \ublistelem{5.}\bibnobreakspace Поддерживает непостижимое спокойствие и самообладание личности перед лицом дурного обращения и страшной несправедливости.
\vs p101 3:10 \ublistelem{6.}\bibnobreakspace Поддерживает божественное упование на окончательную победу вопреки жестокостям казалось бы слепой судьбы и кажущегося крайнего безразличия сил природы к человеческому благополучию.
\vs p101 3:11 \ublistelem{7.}\bibnobreakspace Упорствует в непоколебимой вере в Бога вопреки всем противоречащим ей доводам логики и успешно противостоит прочей интеллектуальной софистике.
\vs p101 3:12 \ublistelem{8.}\bibnobreakspace Продолжает проявлять бесстрашную веру в продолжение существования души, несмотря на обманчивые учения ложной науки и убедительные самообманы поверхностной философии.
\vs p101 3:13 \ublistelem{9.}\bibnobreakspace Живет и побеждает несмотря на губительные перегрузки сложных и фрагментарных цивилизаций современности.
\vs p101 3:14 \ublistelem{10.}\bibnobreakspace Способствует непрерывному сохранению альтруизма, несмотря на человеческий эгоизм, общественные антагонизмы, алчность промышленности и политическую несогласованность.
\vs p101 3:15 \ublistelem{11.}\bibnobreakspace Твердо придерживается возвышенной веры в единство вселенной и божественное водительство, несмотря на смущающее присутствие зла и греха.
\vs p101 3:16 \ublistelem{12.}\bibnobreakspace Продолжает почитать Бога вопреки всему. Не боится провозглашать: «Хоть он и убивает меня, я все равно буду ему служить».
\vs p101 3:17 \P\ Следовательно, благодаря трем явлениям мы знаем, что у человека есть божественный дух (или духи), пребывающий (или пребывающие) в нем: во\hyp{}первых, благодаря личному опыту --- религиозной вере; во\hyp{}вторых, благодаря откровению --- личному и расовому; и, в\hyp{}третьих, благодаря поразительному проявлению необычайных и неестественных реакций на его материальное окружение, подобных тем, которые иллюстрируются приведенными выше двенадцатью духоподобными действиями в действительных и сложных ситуациях реального человеческого бытия. Но есть и другие явления.
\vs p101 3:18 Причем именно такое жизненное и энергичное поведение веры в области религии и позволяет человеку утвердить личное обладание и духовную действительность этого завершающего дара человеческой природы, религиозного опыта.
\usection{4. Ограничения откровения}
\vs p101 4:1 Поскольку ваш мир, как правило, не знает об истоках, даже истоках физических, постольку давать время от времени наставления в космологии казалось разумным. Но это всегда создавало неприятности в будущем. Запрещая наделение незаслуженными или преждевременными знаниями, законы откровения сильно сдерживают нас. Любой космологии, представленной как часть данной откровением религии, суждено устаревать за очень короткое время. Поэтому те, кто будет изучать такое откровение в будущем, подвергаются искушению отвергнуть любой элемент подлинной религиозной истины, который оно может содержать, потому что обнаруживают ошибки на поверхности связанных с ним и представленных в нем космологий.
\vs p101 4:2 Человечество должно понимать, что мы, участвующие в откровении истины, весьма строго ограничены наставлениями наших руководителей. Мы не вольны предвосхищать научные открытия на тысячу лет вперед. Носители откровения должны действовать в соответствии с указаниями, которые являются частью установления об откровении. Мы не видим пути преодоления этой трудности ни теперь, ни в будущем. Мы полностью сознаем, что, хотя исторические факты и религиозные истины из этой серии раскрытия откровения всегда будут верны, многие из наших утверждений, касающихся естественных наук, через несколько лет будут нуждаться в пересмотре вследствие дополнительных научных достижений и новых открытий. Эти новые достижения мы предвидим уже сейчас, но нам запрещено включать подобные не открытые человеком факты в записи откровения. Пусть будет ясно, что откровения не обязательно вдохновлены. Так, космология этих откровений \bibemph{вдохновленной не является.} Она ограничена полученным нами позволением координировать и сортировать современные знания. Хотя божественное или духовное понимание --- это дар, \bibemph{человеческая мудрость должна развиваться.}
\vs p101 4:3 \P\ Истина --- это всегда откровение: самооткровение, когда оно возникает вследствие действия пребывающего в человеке Настройщика, и эпохальное откровение, когда оно представлено действием какой\hyp{}либо другой небесной силы, группы или личности.
\vs p101 4:4 В конечном счете о религии следует судить по ее плодам, в соответствии с манерой и степенью, в которой она проявляет присущее ей и божественное совершенство.
\vs p101 4:5 \P\ Истина может быть лишь относительно вдохновленной, тем не менее откровение --- неизменно духовное явление. Хотя утверждения со ссылкой на космологию вдохновленными не бывают никогда, подобные откровения обладают огромной ценностью, ибо они, по крайней мере на короткое время, проясняют знание, благодаря:
\vs p101 4:6 \ublistelem{1.}\bibnobreakspace Уменьшению путаницы вследствие упорного устранения ошибок.
\vs p101 4:7 \ublistelem{2.}\bibnobreakspace Координированию известных или почти известных фактов и наблюдений.
\vs p101 4:8 \ublistelem{3.}\bibnobreakspace Восстановлению важных частиц утраченного знания об эпохальных деяниях отдаленного прошлого.
\vs p101 4:9 \ublistelem{4.}\bibnobreakspace Предоставлению информации, которая заполняет важные пробелы в областях имеющегося знания.
\vs p101 4:10 \ublistelem{5.}\bibnobreakspace Представления космических данных таким способом, который бы озарил духовные учения, содержащиеся в сопровождающем их откровении.
\usection{5. Религия, расширенная откровением}
\vs p101 5:1 Откровение --- это метод, посредством которого в необходимом деле сортировки и отсеивания ошибок эволюции от истин духовного знания удается сэкономить века и века времени.
\vs p101 5:2 Наука оперирует \bibemph{фактами;} религия же интересуется только \bibemph{ценностями.} Благодаря просвещенной философии разум стремится объединить значения и фактов, и ценностей, тем самым достичь понятия о полной \bibemph{реальности.} Помните, что наука --- это область знания, философия --- царство мудрости, а религия --- сфера опыта веры. Однако религия, тем не менее, представляет две фазы проявления:
\vs p101 5:3 \ublistelem{1.}\bibnobreakspace Эволюционная религия. Опыт примитивного поклонения, религия, являющаяся порождением ума.
\vs p101 5:4 \ublistelem{2.}\bibnobreakspace Религия, данная откровением. Вселенская позиция, являющаяся порождением духа; вера и уверенность в сохранение вечных реальностей, в продолжение существования личности и в окончательное достижение космического Божества, чья цель и сделала все это возможным. Часть замысла вселенной такова, что рано или поздно эволюционной религии суждено принять духовное расширение, которое дает откровение.
\vs p101 5:5 \P\ И наука и религия начинают с выдвижения определенных общепринятых основ для логических умозаключений. Поэтому философия также должна начинать свой путь с предположения о реальности трех вещей:
\vs p101 5:6 \ublistelem{1.}\bibnobreakspace Материального тела.
\vs p101 5:7 \ublistelem{2.}\bibnobreakspace Сверхматериальной фазы человека, души или даже пребывающего в нем духа.
\vs p101 5:8 \ublistelem{3.}\bibnobreakspace Человеческого разума, механизма взаимоотношений и взаимосвязи между духом и материей, между материальным и духовным.
\vs p101 5:9 \P\ Ученые собирают факты, философы координируют идеи, а пророки возвышают идеалы. Чувства и эмоции --- неизменные спутники религии, но религией не являются. Религия может быть чувством опыта, но вряд ли она является опытом чувствования. Ни логика (рационалистическое объяснение), ни эмоции (чувства) не являются неотъемлемой частью религиозного опыта, хотя и те, и другие могут по\hyp{}разному сочетаться с проявлением веры в процессе углубления духовного понимания реальности согласно способностям и врожденным свойствам индивидуального ума.
\vs p101 5:10 Эволюционная религия есть порождение дара духа\hyp{}помощника разума локальной вселенной, отвечающего за создание и воспитание у развивающегося человека стремления к почитанию. Такие примитивные религии непосредственно связаны с этикой и моралью, с чувством человеческого \bibemph{долга.} Подобные религии основаны на убеждениях совести и приводят к стабилизации относительно этичных цивилизаций.
\vs p101 5:11 Религии же, данные личным откровением, поддерживаются духами пришествия, представляющими три лица Райской Троицы, и особым образом заняты расширением \bibemph{истины.} Эволюционная религия доводит до индивидуума понятие о личном долге; религия же, данная откровением, концентрируется на любви, на золотом правиле.
\vs p101 5:12 Эволюционная религия целиком и полностью покоится на вере. Откровение же обладает дополнительной уверенностью в своем расширенном представлении истин божественности и действительности, а также еще более ценным свидетельством подлинного опыта, который накапливается благодаря практически действующему союзу эволюционной веры и истины, данной откровением. Такой рабочий союз человеческой веры и божественной истины и является характером, прочно ставшим на путь, ведущий к подлинному обретению моронтийной личности.
\vs p101 5:13 \P\ Эволюционная религия дает лишь уверенность веры и подтверждение совести; религия же, данная откровением, дает уверенность веры плюс истину живого опыта в реальностях откровения. Третий этап религии, или третья фаза религиозного опыта, связан с моронтийным состоянием, более твердым пониманием моты. По мере моронтийного совершенствования истины религии, данной откровением, расширяются все больше и больше, и ты все больше и больше будешь познавать истину верховных ценностей, божественных добродетелей, вселенских отношений, вечных реальностей и предельных предназначений.
\vs p101 5:14 По мере моронтийного совершенствования уверенность, которую дает истина, во все большей степени заменяет уверенность, которую дает вера. И когда тебя окончательно возьмут в подлинно духовный мир, тогда уверенность, которую дает чисто духовное понимание, будет действовать вместо веры и истины или, точнее, в сочетании с этими прежними методами личностной уверенности и в дополнение к ним.
\usection{6. Развивающийся религиозный опыт}
\vs p101 6:1 Моронтийная фаза религии, данной откровением, связана с \bibemph{опытом продолжения существования,} и ее великим стремлением является достижение духовного совершенства. Существует также и более высокий побудительный мотив богопочитания, связанный с настоятельным призывом к усиленному этическому служению. Моронтийное понимание влечет за собой постоянное расширяющееся осознание Семеричного, Верховного и даже Предельного.
\vs p101 6:2 На протяжении всего религиозного опыта, от самого начала на материальном уровне вплоть до времени обретения полного духовного состояния, Настройщик остается тайной личного осознания реальности бытия Верховного; причем этот же самый Настройщик хранит секреты вашей веры в трансцендентное познание Предельного. Основанная на опыте личность развивающегося человека, соединенная с явленной в Настройщике сущностью экзистенциального Бога, и образует потенциальную завершенность верховного бытия и по своему существу является основой для сверхконечного проявления трансцендентной личности.
\vs p101 6:3 \P\ Нравственная воля включает в себя решения, основанные на осмысленном знании, углубленном мудростью и утвержденном религиозной верой. Подобные решения являются деяниями нравственной природы и свидетельствуют о существовании нравственной личности, предшественницы личности моронтийной и в конечном итоге обладающей истинно духовным статусом.
\vs p101 6:4 Эволюционный тип знания является лишь массой накопленного протоплазменного сохранившегося в памяти материала; это --- самая примитивная форма сознания у творения. Мудрость объемлет собой идеи, сформулированные из протоплазменной памяти в процессе ассоциирования и рекомбинации, причем подобные явления и отличают человеческий разум от простого животного разума. У животных есть знания, но лишь человек обладает способностью быть мудрым. Истина становится доступной одаренному мудростью индивидууму благодаря пришествию в такой ум духов Отца и Сынов, Настройщика Мысли и Духа Истины.
\vs p101 6:5 \P\ Христос\hyp{}Михаил во время своего пришествия на Урантию жил во власти эволюционной религии вплоть до времени своего крещения. С этого же момента до события его распятия (включительно) он продолжал свой труд под совокупным водительством эволюционной религии и религии откровения. С утра своего воскресения до своего вознесения он преодолел множество фаз моронтийной жизни перехода из мира материи в мир духа, который совершает смертный. После своего вознесения Михаил достиг совершенства в переживании Верховенства, осознании Верховного и, как единственная личность в Небадоне, обладающая неограниченной способностью ощущать реальность Верховного, незамедлительно достиг состояния верховного владычества в своей локальной вселенной.
\vs p101 6:6 Что же касается человека, то окончательное слияние и итоговое единство с пребывающим в нем Настройщиком --- синтез личности человека и сущности Бога --- делают его потенциально, живой частью Верховного и обеспечивают такому некогда смертному существу вечное неотъемлемое право бесконечного стремления к финальности вселенского служения Верховному и с Верховным.
\vs p101 6:7 \P\ Откровение учит смертного человека, что для того, чтобы предпринять такое величественное и захватывающее путешествие в пространстве посредством продвижения во времени, он должен начать с организации знаний в идеи\hyp{}решения; затем повелеть мудрости упорно трудиться над благородной задачей преобразования собственных идей во все более практичные, но, тем не менее, возвышенные идеалы, в представления, столь же разумные, как идеи, и столь же логичные, как идеалы, что Настройщик решается объединить и одухотворить их настолько, чтобы конечный разум смог сформировать из них такие представления, которые стали бы неотъемлемой частью человека и таким образом подготовили его к действию Духа Истины Сыновей, пространственно\hyp{}временному проявлению Райской истины --- истины всемирной. Умение привести в соответствие идеи\hyp{}решения, логические идеалы и божественную истину и есть обладание праведным характером, которое является предварительным условием, дающим смертному человеку возможность проникновения во все более расширяющиеся и все более духовные реальности моронтийных миров.
\vs p101 6:8 Учения Иисуса составили первую религию Урантии, которая с такой полнотой охватила гармоничное соотнесение знания, мудрости, веры, истины и любви, что полностью и одновременно обеспечила временное спокойствие, интеллектуальную убежденность, нравственное просвещение, философскую стабильность, этическую чувствительность, осознание Бога и положительную уверенность в продолжении существования личности. Вера Иисуса указала путь к финальности человеческого спасения, к предельному достижению смертных во вселенной, поскольку она обеспечила:
\vs p101 6:9 \ublistelem{1.}\bibnobreakspace Спасение от материальных оков в личном осознании сыновства по отношению к Богу, который есть дух.
\vs p101 6:10 \P\ \ublistelem{2.}\bibnobreakspace Спасение от интеллектуального рабства: человек узнает истину, и истина сделает его свободным.
\vs p101 6:11 \P\ \ublistelem{3.}\bibnobreakspace Спасение от духовной слепоты, осознание человеком братства смертных существ и моронтийное понимание братства всех творений вселенной; служение\hyp{}открытие духовной реальности и служение\hyp{}откровение о благости духовных ценностей.
\vs p101 6:12 \P\ \ublistelem{4.}\bibnobreakspace Спасение от неполноты собственного «я» через достижение духовных уровней вселенной и окончательное осознание гармонии Хавоны и совершенства Рая.
\vs p101 6:13 \P\ \ublistelem{5.}\bibnobreakspace Спасение от собственного «я», избавление от ограничений самосознания через достижение космических уровней Верховного разума и соотнесение с достижениями всех остальных сознающих себя существ.
\vs p101 6:14 \P\ \ublistelem{6.}\bibnobreakspace Спасение от времени, обретение вечной жизни бесконечного совершенствования в осознании Бога и служении Богу.
\vs p101 6:15 \P\ \ublistelem{7.}\bibnobreakspace Спасение от конечного, совершенное единство с Божеством в Верховном и через Верховного, посредством которого творение пытается свершить трансцендентное открытие Предельного на уровнях абсонитных, следующих после обретения статуса финалита.
\vs p101 6:16 \P\ Такое семеричное спасение равносильно полноте и совершенству реализации предельного переживания Отца Всего Сущего. Причем все это потенциально содержится в реальности веры, присущей человеческому переживанию религии. Возможно же это постольку, поскольку вера Иисуса была воспитана реальностями, существующими даже вне предельного, и открывала их; вера Иисуса приближалась к состоянию вселенского абсолюта настолько, насколько проявление такового возможно в эволюционирующем космосе времени и пространства.
\vs p101 6:17 Овладевая верой Иисуса, смертный человек может во времени предчувствовать реальности вечности. Иисус открыл Конечного Отца в человеческом опыте и его братья во плоти смертной жизни могут стать его последователями в этом же опыте открытия Отца. Даже такими, каковы они есть, они могут достичь того же удовлетворения в этом опыте познания Отца, как этого достиг Иисус таким, каким был он. После завершающего пришествия Михаила во вселенной Небадон были актуализированы новые потенциалы, и одним из них стало новое освещение вечного пути, который ведет к Отцу всех и который могут пройти даже смертные из материальной плоти и крови в своей начальной жизни на планетах пространства. Иисус был и остается новым и живым путем, идя по которому человек может вступить в божественное наследство, которое, Отец повелел, принадлежит человеку, стоит ему об этом только попросить. В Иисусе с избытком показаны и начала и концы опыта веры человечества, даже божественного человечества.
\usection{7. Личная философия религии}
\vs p101 7:1 Идея --- это лишь теоретический план действия, тогда как положительное решение есть план действия обоснованный. Стереотип --- это план действия, принятый без обоснования. Материалы, из которых строится личная философия религии, получаются как из внутреннего, так и из внешнего опыта индивидуума. Социальное положение, экономические условия, возможности получения образования, нравственные качества, влияние со стороны институтов, политические события, расовые тенденции и религиозные учения времени и места, в которых живет человек, --- все это становится факторами формирования личной философии религии. Даже врожденный темперамент и интеллектуальные наклонности, и те в значительной степени определяют структуру религиозной философии. Профессия, брак и родственники --- все это влияет на эволюцию личных жизненных норм человека.
\vs p101 7:2 Философия религии возникает из простого развития идей плюс экспериментальной жизни, по мере того как и то и другое модифицируется стремлением подражать товарищам. Благоразумность философских заключений зависит от острого, честного и проницательного мышления в сочетании с чувствительностью к значениям и точностью оценки. Нравственные трусы никогда не достигают высоких уровней философского мышления; для овладения новыми уровнями опыта и попытки исследовать неизвестные области интеллектуальной жизни требуется мужество.
\vs p101 7:3 Вскоре возникают новые системы ценностей; успешно формируются новые принципы и нормы; обычаям и идеалам придается новая форма; достигается некоторое понятие о личном Боге, вслед за которым расширяются представления об отношении к нему.
\vs p101 7:4 \P\ Великое различие между религиозной и нерелигиозной философией жизни заключается в природе и уровне признанных ценностей и в объекте верности. В эволюции религиозной философии существует четыре фазы: подобный опыт может стать просто приспособленческим, направленным на подчинение традиции и власти. Либо он может быть удовлетворен незначительными достижениями, достаточными лишь для придания устойчивости повседневной жизни и поэтому рано задерживается на таком второстепенном уровне. Подобные смертные верят, что лучше всего оставить все, как есть. Третья группа доходит до уровня логической интеллектуальности, но на нем и задерживается вследствие культурного рабства. Поистине жалко смотреть на то, как гигантские интеллекты столь прочно удерживаются в жестокой власти культурной зависимости. В равной степени жаль наблюдать и тех, кто променял свое культурное рабство на материалистические оковы науки, ложно именуемой таковой. Четвертый уровень философии достигает свободы от всех помех, которыми являются условности и традиции, и решается думать, действовать и жить честно, верно, бесстрашно и преданно.
\vs p101 7:5 Лакмусовой бумажкой для любой религиозной философии является то, делает ли она различие между реальностями материального и духовного миров, одновременно признавая что они едины в интеллектуальном стремлении и в общественном служении. Разумная религиозная философия отнюдь не смешивает Божие с кесаревым. И не признает эстетического культа чистого изумления в качестве замены религии.
\vs p101 7:6 Философия преобразует примитивную религию, которая во многом была сказкой, выдуманной совестью, в живой опыт в области восходящих ценностей космической реальности.
\usection{8. Вера и убеждение}
\vs p101 8:1 Убеждение достигает уровня веры тогда, когда оно мотивирует жизнь и формирует образ жизни. Принятие учения как истинного --- это не вера, а всего лишь убеждение. Не являются верой ни уверенность, ни убежденность. Состояние ума достигает уровней веры только тогда, когда она действительно занимает в образе жизни господствующее положение. Вера --- это живое свойство подлинного личного религиозного опыта. Человек верит истине, восхищается красотой и чтит доброту, но не почитает их; такое отношение спасительной веры сосредоточено на одном только Боге, который является олицетворением не только этого, но и бесконечно большего.
\vs p101 8:2 Убеждение всегда ограничивает и связывает, вера же расширяет и освобождает. Убеждение сковывает, а вера раскрепощает. Но живая религиозная вера больше, чем совокупность благородных убеждений, больше возвышенной философской системы; это --- живой опыт, связанный с духовными значениями, божественными идеалами и верховными ценностями; это --- познание Бога и служение человеку. Убеждения могут стать достоянием группы; вера же должна быть личной. Теологические убеждения можно предложить группе; вера же может возникнуть лишь в сердце отдельно взятого религиозного человека.
\vs p101 8:3 Вера искажает свой долг, подрывает доверие к себе, когда решается отвергать реальности и давать своим приверженцам вымышленное знание. Вера становится предателем, когда потворствует измене интеллектуальной честности и умаляет приверженность верховным ценностям и божественным идеалам. Вера никогда не уклоняется от обязанности решать проблемы смертной жизни. Живая вера отнюдь не способствует фанатизму, преследованиям или нетерпимости.
\vs p101 8:4 Вера не стесняет творческого воображения и не относится с неразумным предубеждением к научным открытиям. Вера оживляет религию и вынуждает религиозного человека героически являть своей жизнью золотое правило. Энтузиазм веры исходит из знания, и ее стремления являются прелюдией к возвышенному миру.
\usection{9. Религия и нравственность}
\vs p101 9:1 Ни одно исповедуемое откровение религии не может считаться подлинным, если оно не сумело признать требования долга, присущие этическим обязательствам, которые были созданы и взлелеяны предшествующей ему эволюционной религией. Откровение неизменно раздвигает этический горизонт эволюционирующей религии и одновременно неизменно расширяет нравственные обязательства, связанные со всеми прежними откровениями.
\vs p101 9:2 Решаясь критически судить примитивную религию человека (или религию первобытного человека), ты должен помнить, что судить таких дикарей и оценивать их религиозный опыт следует в соответствие со степенью их просвещенности и статусом их совести. Не делай ошибку, судя о религии другого человека, пользуясь своими критериями знания и истины.
\vs p101 9:3 Истинная религия есть то возвышенное и глубокое убеждение в душе, которое постоянно предостерегает человека о том, что для него было бы неправильно не верить в моронтийные реальности, составляющие его высшие этические и нравственные представления, его высшее толкование величайших ценностей жизни и глубочайших реальностей вселенной. Причем такая религия есть просто опыт соблюдения интеллектуальной верности высшим велениям духовного сознания.
\vs p101 9:4 Поиски красоты являются частью религии настолько, насколько они этичны, и в такой степени, в какой они обогащают понятие о нравственном. Искусство же религиозно только тогда, когда оно проникнуто целью, обусловленной высоким духовным побуждением.
\vs p101 9:5 Просвещенное духовное сознание цивилизованного человека связано не столько с некоторым специфическим интеллектуальным убеждением или каким\hyp{}либо особым образом жизни, сколько с открытием истины жизни, благого и правильного способа реагировать на постоянно повторяющиеся ситуации смертного бытия. Нравственное сознание --- это всего лишь название для обозначения человеческого признания и понимания тех этических и возникающих моронтийных ценностей, которым человек должен следовать в своей повседневной жизни, контролируя свое поведение и управляя им.
\vs p101 9:6 \P\ Несмотря на признание того, что религия несовершенна, существует по крайней мере два практических проявления ее сущности и назначения:
\vs p101 9:7 \ublistelem{1.}\bibnobreakspace Духовное побуждение и философское влияние религии имеют тенденцию вынуждать человека переносить свою оценку нравственных ценностей непосредственно на дела своих собратьев --- а это и есть этическая реакция религии.
\vs p101 9:8 \P\ \ublistelem{2.}\bibnobreakspace Религия создает для человеческого разума одухотворенное сознание божественной реальности, которое основано и благодаря вере получено из предшествовавших ему представлений моральных ценностей и которое взаимосвязано с наложенными на него представлениями моральных ценностей. Религия, таким образом, становится цензором земных дел, формой возвышенного морального упования и уверенности в реальности, углубленные реальности времени и более долговечные реальности вечности.
\vs p101 9:9 \P\ Вера становится связующим звеном между нравственным сознанием и духовным понятием непреходящей реальности. Религия становится стезей ухода человека от материальных ограничений временного и естественного мира к божественным реальностям вечного и духовного мира через посредство и с помощью метода спасения, постепенного моронтийного преобразования.
\usection{10. Религия как освободитель человека}
\vs p101 10:1 Разумный человек знает, что он --- дитя природы, часть материальной вселенной; более того, он не видит продолжения существования отдельной личности в движении и напряжении математического уровня вселенской энергии. Не может человек увидеть духовную реальность и через наблюдение физических причин и следствий.
\vs p101 10:2 Человек сознает также, что он является частью космоса, способного формировать и воспринимать идеи, однако, хотя понятие может существовать в течение времени, выходящего за пределы продолжительности смертной жизни, в самом понятии нет ничего, что указывало бы на продолжение существования личности задумавшей это понятие. Исчерпание возможностей логики и разума не откроют вечную истину о продолжении существования личности логику или мыслителю.
\vs p101 10:3 Материальный уровень закона обеспечивает причинно\hyp{}следственную последовательность, бесконечное последствие предшествующего ему действия; уровень разума же предполагает вечную неизменную способность формировать и воспринимать идеи, непрекращающийся поток концептуальной потенциальности, исходящий от существовавших прежде концепций. Но ни один из этих уровней вселенной не раскрывает ищущему смертному путь освобождения от неполноты статуса и невыносимой неопределенности бытия в состоянии переходящей реальности во вселенной, временной личности, обреченной на уничтожение после исчерпания ограниченных запасов жизненной энергии.
\vs p101 10:4 Только идя по моронтийному пути, ведущему к духовному пониманию, человек вообще может разрушить оковы, присущие его смертному статусу во вселенной. Энергия и разум ведут назад в Рай и к Божеству, но непосредственно от такого Райского Божества ни дарование энергии, ни дарование разума не исходят. Человек --- дитя Бога только в духовном смысле. Причем это так потому, что в настоящее время человек имеет дар Райского Отца и Райский Отец пребывает в нем лишь в духовном смысле. Человечество никогда не сможет открыть божественное, кроме как на стезе религиозного опыта и благодаря проявлению истинной веры. Принятие верой истины существования Бога позволяет человеку спастись от тесных оков материальных ограничений и дает ему разумную надежду на обретение надежного пути, ведущего из материального царства, в котором смерть, в духовное царство, в котором жизнь вечная.
\vs p101 10:5 \P\ Цель религии не в том, чтобы удовлетворить любопытство о Боге, а в том, чтобы обеспечить интеллектуальное постоянство и философскую уверенность, стабилизировать и обогащать человеческую жизнь путем соединения смертного с божественным, неполного с совершенным, человека и Бога. Именно через религиозный опыт понятия человека об идеальности наполняются реальностью.
\vs p101 10:6 \P\ Ни научных, ни логических доказательств божественности быть не может. Одним рассуждением невозможно обосновать ценности и добродетели религиозного опыта. Однако всегда истинными будут слова: Всякий, желающий исполнять волю Бога, поймет неоспоримость духовных ценностей. Таков лучший подход, которым можно воспользоваться на смертном уровне, чтобы представить доказательства реальности религиозного опыта. Такая вера дает единственно возможный способ избавления от механических тисков материального мира и от ошибочного искажения, вызванного неполнотой интеллектуального мира; это единственный найденный выход из тупика, в котором оказалось смертное мышление в отношении непрерывного продолжения существования отдельной личности. Это --- единственный путь к полноте реальности и вечности жизни во всемирном творении любви, закона, единства и поступательного достижения Божества.
\vs p101 10:7 Религия успешно исцеляет сознание человека от идеалистической обособленности или духовного одиночества; она делает человека сыном Бога, гражданином новой и исполненной смысла вселенной. Религия убеждает человека в том, что, идя за светом праведности, различимом в его душе, он тем самым отождествляет себя с замыслом Бесконечного и с целью Вечного. Такая освобожденная душа сразу начинает чувствовать себя как дома в этой новой вселенной, в своей вселенной.
\vs p101 10:8 Испытывая такое преобразование веры, ты перестанешь быть рабски покорной частью материального, космоса, но становишься освобожденным, наделенным волей сыном Отца Всего Сущего. Такой освобожденный сын уже не одинок в своей битве с неумолимой обреченностью на прекращение временного бытия; он более не сражается со всей природой при том, что неблагоприятные условия жестко действуют против него; он более не пребывает в нерешительности из\hyp{}за парализующего страха, что, возможно, доверился чистой иллюзии или связал свою веру с утопическим заблуждением.
\vs p101 10:9 Теперь сыновья Бога все вместе включены в сражение за победу реальности над частичными тенями бытия. Наконец все творения начинают осознавать, что Бог и все божественные духи почти безграничной вселенной в небесной борьбе за достижение вечности жизни и божественности статуса выступают на их стороне. Такие освобожденные верой сыновья, несомненно, включились во временную борьбу и выступают на стороне верховных сил и божественных личностей вечности; даже звезды, и те на путях своих сражаются за них; в конечном счете они смотрят на вселенную изнутри, с точки зрения Бога; и все преобразуется из неопределенностей материальной отчужденности в уверенность вечного духовного совершенствования. Даже само время становится всего лишь тенью вечности, отбрасываемой Райскими реальностями на движущиеся покровы пространства.
\vs p101 10:10 [Представлено Мелхиседеком Небадона.]
