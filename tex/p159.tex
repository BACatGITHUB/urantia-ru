\upaper{159}{Путешествие по Десятиградию}
\author{Комиссия срединников}
\vs p159 0:1 Придя в Магаданский лес, Иисус и двенадцать апостолов обнаружили там группу примерно из ста евангелистов и учеников, включая и женский отряд, уже ожидавших их и готовых идти с ними с проповедями по городам Десятиградия.
\vs p159 0:2 Утром в четверг 18 августа Учитель собрал своих последователей и распорядился, чтобы каждый из апостолов взял себе в помощники одного из двенадцати евангелистов и, разделив остальных евангелистов на двенадцать групп, вместе с ними отправился трудиться в города и селения Десятиградия. Женщинам же и остальным ученикам Иисус велел оставаться с ним. На это путешествие он определил четыре недели и дал указание своим последователям вернуться в Магадан не позднее пятницы 16 сентября. Иисус пообещал их часто навещать. На протяжении месяца эти двенадцать групп трудились в Герасе, Гамале, Гиппосе, Зафоне, Гадаре, Авиле, Едрее, Филадельфии, Есевоне, Диуме, Скифополе и многих других городах. За это время не случилось никаких чудес исцеления или иных необычайных событий.
\usection{1. Проповедь о прощении}
\vs p159 1:1 Однажды вечером в Гиппосе, отвечая на вопрос ученика, Иисус преподал урок о прощении. Учитель сказал:
\vs p159 1:2 \pc «Если у добросердечного человека есть сто овец, и одна из них заблудилась, то не оставит ли он девяносто девять и не пойдет ли искать заблудившуюся? И если он хороший пастух, то не будет ли искать потерявшуюся овцу, пока не найдет ее? Потом же, найдя свою потерявшуюся овцу, пастух берет ее на плечи и, радостно идя домой, созывает своих друзей и соседей: „Порадуйтесь со мной, я нашел мою пропавшую овцу“. Объявляю вам, что на небесах более радости об одном кающемся грешнике, нежели о девяносто девяти праведниках, не имеющих нужды в покаянии. Все равно Отец Небесный не желает, чтобы заблудился один из малых сих, а тем более погиб. В вашей религии Бог может принять кающихся грешников; в евангелии же царства Отец сам идет и ищет их еще до того, как они всерьез задумаются о покаянии.
\vs p159 1:3 Отец Небесный любит своих детей, а поэтому и вы должны научиться любить друг друга; Отец Небесный прощает вам грехи ваши; стало быть, и вы должны научиться прощать друг друга. Если брат твой согрешает против тебя, пойди к брату и деликатно и терпеливо покажи ему ошибку его. Сделай же это с ним с глазу на глаз. Если послушает тебя, то обрел ты брата твоего. Если же брат твой не послушает тебя, если будет упорствовать в заблуждении своем, тогда снова пойди к нему, возьми с собой одного или двух общих друзей, так чтобы у тебя было два или даже три свидетеля, которые подтвердили бы слова твои и убедились, что ты был справедлив и милосерд к обижающему тебя брату. Если же он откажется слушать собратьев твоих, можешь все рассказать братству, и тогда, если он откажется слушать братство, пусть оно поступает с ним, как сочтет правильным; да будет такой непокорный изгнан из царства. Хотя вы не можете претендовать на то, чтобы судить души собратьев ваших и хотя не можете прощать грехи или как\hyp{}либо притязать на права начальников над небесными воинствами, тем не менее в руки ваши предано поддержание временного порядка в царстве на земле. Хотя вы не можете вмешиваться в божественные установления, касающиеся вечной жизни, вы должны определять принципы поведения, когда от них зависит мирское благополучие братства на земле. Итак, во всех вопросах, связанных с порядком братства, что ни решите на земле, то будет признано и на небе. Хотя вам не дано определять вечную судьбу человека, вы можете издавать законы, касающиеся поведения группы, ибо где двое или трое из вас согласятся в отношении любого из этих дел и попросят меня, будет оно исполнено для вас, если только прошение ваше не будет несовместимо с волей Отца моего Небесного. И все это истинно навеки, ибо где двое или трое верующих, там и я среди них».
\vs p159 1:4 \pc Симон Петр, апостол, возглавляющий группу работников в Гиппосе, услышав эти слова Иисуса, спросил: «Господи, сколько раз прощать брату моему, согрешающему против меня? До семи раз?» И Иисус ответил Петру: «Не только до семи, но до семидесяти и семи раз. Поэтому царство небесное можно уподобить царю, повелевшему слугам своим возвратить ему долги. Когда же начали проводить проверку счетов, привели к нему одного из главных слуг и тот признался, что должен царю своему десять тысяч талантов. И вот этот придворный стал говорить, что для него настали трудные времена и что ему нечем уплатить по обязательству. Поэтому царь приказал забрать все его имущество, а детей продать, чтобы заплатить его долг. Услышав сей суровый приговор, этот главный слуга пал ниц пред царем, умоляя его смилостивиться над ним и дать еще времени: „Государь, потерпи еще немного, и я все тебе заплачу“. Царь же, взглянув на этого нерадивого слугу и его семью, исполнился сострадания. И приказал отпустить его и простить ему весь его долг.
\vs p159 1:5 Этот же главный слуга, получив помилование и прощение из рук царя, пошел по делу своему и, встретив одного из своих подчиненных, который был должен ему всего сто динариев, схватил его, стал душить, говоря: „Заплати мне все, что должен“. И тогда этот слуга пал перед главным слугой и, умоляя его, сказал: „Потерпи, и я скоро смогу заплатить тебе“. Но главный слуга не смилостивился над своим должником, а велел посадить его в темницу, пока не отдаст долга. Когда же товарищи его увидели, что произошло, то огорчились настолько, что пошли и рассказали обо всем царю, своему государю и повелителю. Выслушав их, царь призвал этого неблагодарного и не желающего прощать человека и сказал: „Злой и недостойный слуга! Когда ты искал сострадания, я, ничего не требуя, простил тебе весь долг твой. Почему же и ты не помиловал товарища своего, как я помиловал тебя?“ И разгневался царь настолько, что отдал своего неблагодарного главного слугу тюремщикам, чтобы те держали его, пока не отдаст все, что был должен. Так и Отец мой Небесный будет еще более милосерден к тем, кто, ничего не требуя, проявляет милосердие к своим собратьям. Как можете вы идти к Богу, прося о снисхождении к недостаткам вашим, когда сами привыкли наказывать братьев ваших за то, что они повинны в таких же человеческих слабостях? Говорю всем вам: даром получили блага царства, даром давайте собратьям вашим на земле».
\vs p159 1:6 \pc Таким образом Иисус объяснил опасность и показал несправедливость личного суда над своими собратьями. Порядок должен соблюдаться, правосудие отправляться, однако во всех этих вопросах мудрость братства должна преобладать. Иисус передал законодательную и судебную власть \bibemph{группе,} а не \bibemph{индивидууму.} Однако даже власть которой наделена группа, не должна стать инструментом осуществления личной власти. Всегда существует опасность, что на мнение индивидуума повлияют предрассудки или страсть. Коллективное же мнение, скорее может избежать опасности и устранить несправедливость личной предубежденности. Иисус всегда старался свести к минимуму элементы несправедливости, возмездия и мести.
\vs p159 1:7 \pc [Использование термина «семьдесят семь раз» в качестве примера милосердия и снисходительности встречается в Писании, где говорится о ликовании Ламеха из\hyp{}за того, что металлическое оружие есть у его сына Тувалкаина. Сравнивая эти более совершенные орудия с орудиями своих врагов, он воскликнул: «Если за Каина безоружного отмстилось всемеро, то за меня отмстится в семьдесят семь раз».]
\usection{2. Странный проповедник}
\vs p159 2:1 Иисус пришел в Гамалу, чтобы встретиться с Иоанном и теми, кто здесь трудился вместе с ним. Вечером после всех вопросов и ответов, Иоанн сказал Иисусу: «Учитель, вчера я ходил в Аштароф повидаться с человеком, который учит во имя твое и даже утверждает, что способен изгонять бесов. Однако этот человек никогда с нами не был и за нами не следует; поэтому я запретил ему делать подобное». Тогда Иисус сказал: «Не запрещай ему. Разве ты не понимаешь, что сие евангелие царства вскоре будет возвещаться по всему миру? Как можешь ты думать, что все, кто верит в евангелие, должны подчиняться твоему руководству? Радуйся, что наше учение уже начало проявлять себя за пределами нашего личного влияния. Разве ты не видишь, Иоанн: те, кто заявляют о том, что делают великие дела во имя мое, в конечном счете должны поддерживать наше дело? И, разумеется, не будут говорить обо мне плохого. Сын мой, в делах подобного рода тебе лучше будет считать: кто не против нас, тот с нами. В грядущих поколениях многие не вполне достойные будут совершать странные дела во имя мое, но я не буду им запрещать. Говорю тебе, даже когда чаша родниковой воды дается душе жаждущей, вестники Отца всегда делают запись о подобном служении любви».
\vs p159 2:2 Это наставление привело Иоанна в великое недоумение. Разве не слышал он, как Учитель сказал: «Кто не со мной, тот против меня?» Иоанн не понял, что в данном случае Иисус говорил о личном отношении человека к духовным учениям царства, тогда как в другом случае речь шла о внешних и многообразных отношениях между верующими, об отношениях, касающихся вопросов управления и сферы полномочий одной группы верующих над работой других групп, из которых в конечном итоге и будет образовано грядущее всемирное братство.
\vs p159 2:3 Однако в последующем, трудясь на благо царства, Иоанн часто рассказывал об этом случае. И все же апостолы не раз обижались на тех, кто осмеливался учить во имя Учителя. Им всегда казалось, что те, кто никогда не сидел у ног Иисуса, не достойны учить во имя его.
\vs p159 2:4 Человек, которому Иоанн запретил учить и трудиться во имя Иисуса, не подчинился апостолу. Он все равно продолжил свои труды и перед тем, как идти в Месопотамию, собрал в Канате довольно значительную группу верующих. Сей человек, Аден, уверовал в Иисуса благодаря рассказам умалишенного, которого Иисус исцелил недалеко от Хересы и который непоколебимо верил, что овладевшие им злые духи, которых изгнал Учитель, вошли в стадо свиней и, погнав их, сбросили со скалы и уничтожили.
\usection{3. Наставление учителям и верующим}
\vs p159 3:1 В Едрее, где трудился Фома со своими сподвижниками, Иисус провел сутки и во время вечерней беседы определил принципы, которыми должны руководствоваться те, кто проповедует истину, и которые должны вдохновлять всех, кто учит евангелию царства. Кратко и на современном языке суть учения Иисуса такова:
\vs p159 3:2 \pc Всегда уважайте личность человека. Никогда праведная цель не оправдывает применение силы; духовные победы могут быть одержаны только духовной силой. Это предписание, запрещающее использовать материальное воздействие, касается как психической силы, так и силы физической. Подавляющие доводы и умственное превосходство не должны использоваться для принуждения мужчин и женщин к царству. Человеческий ум не следует подавлять силой логических доводов или ослеплять изощренным красноречием. Хотя эмоции как фактор, влияющий на принятие людьми своих решений, устранить полностью нельзя, в наставлениях тех, кто желает продвинуть дело царства вперед, не следует прямо взывать к ним. Обращайтесь с призывом непосредственно к божественному духу, пребывающему в умах людей. Не взывайте к страху, жалости или к простому сочувствию. Обращаясь к людям, будьте справедливы, владейте собой и проявляйте сдержанность; с должным уважением относитесь к личности ваших учеников. Помните, что я сказал: «Се, стою при дверях и стучу, и если услышит кто, войду».
\vs p159 3:3 Ведя людей в царство, не подавляйте и не разрушайте их чувство собственного достоинства. Хотя чрезмерное чувство собственного достоинства может повредить подобающему смирению и привести к гордыне, тщеславию и высокомерию, утрата чувства собственного достоинства часто кончается параличом воли. Целью сего евангелия является восстановление чувства собственного достоинства у тех, кто его утратил, и обуздание --- у тех, у кого оно есть. Не делайте ошибку, только осуждая жизненные промахи ваших учеников; не забывайте также щедро одобрять то, что в их жизнях больше всего достойно похвалы. Помните: я не остановлюсь ни перед чем, чтобы восстановить чувство собственного достоинства у тех, кто его утратил и действительно желает обрести его вновь.
\vs p159 3:4 Смотрите, не уязвляйте чувство собственного достоинства робких и исполненных страха душ. Без сарказма относитесь к моим простодушным собратьям. Не будьте циничны с моими исполненными страха детьми. Праздность разрушает чувство собственного достоинства; посему призывайте собратьев ваших неустанно заниматься тем делом, которое они избрали и прилагайте все усилия, дабы предоставить работу тем, у кого ее нет.
\vs p159 3:5 Никогда не прибегайте к такому недостойному средству, как запугивание, пытаясь обратить мужчин и женщин к царству. Любящий отец не учит своих детей подчиняться своим справедливым требованиям, запугивая их.
\vs p159 3:6 Когда\hyp{}нибудь дети царства поймут, что сильные эмоции вовсе не равнозначны водительству божественного духа. Находиться под влиянием сильного и непонятного стремления что\hyp{}то предпринять или куда\hyp{}то идти совсем не обязательно означает, что подобные побуждения являются следствием водительства духа, пребывающего в человеке.
\vs p159 3:7 Предостерегайте всех верующих о моменте конфликта, который надлежит пройти всем, кто переходит из жизни, которой живут во плоти, в высшую жизнь, которой живут в духе. Живущие только в одном из этих миров находятся в более или менее спокойном состоянии, однако все обречены испытывать большую или меньшую неопределенность в момент перехода от одного уровня жизни на другой. Входя в царство, нельзя избежать ответственности или уклониться от обязательств перед ним, однако помните: иго евангелия --- благо, и бремя истины легко.
\vs p159 3:8 Мир полон жаждущих душ, погибающих от голода, когда хлеб жизни совсем рядом; люди умирают в поисках Бога, который живет внутри каждого из них. Они ищут сокровища царства, томя сердца и утруждая ноги, когда до живой веры им рукой подать. Вера для религии --- все равно, что паруса для корабля; она придает сил, а не отягчает бремя жизни. Для входящих в царство есть только одна борьба --- это сражаться в благом сражении веры. Для верующего есть только одна битва --- это битва с сомнениями --- неверие.
\vs p159 3:9 Проповедуя евангелие царства, вы просто учите дружбе с Богом. И братство это в равной степени привлекательно и для мужчин, и для женщин, ибо и те, и другие найдут в нем то, что наиболее истинно удовлетворяет их типичным желаниям и соответствует их идеалам. Говорите детям моим, что я не только нежен к их чувствам и терпим к их слабостям, но и безжалостен к греху и нетерпим к порокам. В присутствии Отца моего я, действительно, кроток и смирен, но я равно безжалостен и неумолим там, где намеренно вершится зло и царит греховный мятеж против воли Отца моего Небесного.
\vs p159 3:10 Не изображайте Учителя вашего как мужа скорби. Грядущие поколения должны также узнать блеск нашей радости, неутомимость нашей доброй воли и вдохновение нашего добродушного веселья. Мы возвещаем послание благой вести, которая заряжает своей преображающей мощью. Наша религия кипит новой жизнью и полна новым смыслом. Принимающие сие учение исполнены радостью и в сердцах своих призваны радоваться вечно. Все возрастающее счастье --- вот опыт всех уверенных в Боге.
\vs p159 3:11 Научите всех верующих стараться не полагаться на шаткие подпорки ложного сочувствия. Жалостью к себе сильного характера не воспитать; честно старайтесь избегать обманчивого влияния со стороны друзей по несчастью. Сочувствуйте храбрым и отважным, но избегайте чрезмерной жалости к тем трусливым душам, которые боятся жизненных испытаний. Не утешайте тех, кто уступает трудностям без борьбы. Не сочувствуйте собратьям вашим только лишь потому, что они в свою очередь станут сочувствовать вам.
\vs p159 3:12 \pc Когда дети мои обретают осознанную уверенность в существовании божественного присутствия, то такая вера расширяет разум, облагораживает душу, укрепляет личность, приумножает радость, углубляет духовное восприятие и увеличивает способность любить и быть любимым.
\vs p159 3:13 Научите всех верующих, что те, кто входит в царство, тем самым не становятся неуязвимыми для злоключений, происходящих во времени, или же обычных природных катастроф. Вера в евангелие не отвращает беду, но дает возможность \bibemph{не бояться,} когда беда одолеет вас. Если вы не боитесь верить в меня и без колебаний следовать за мною, то, поступая так, несомненно, встанете на тернистый путь. Я не обещаю избавить вас от потока несчастий, но обещаю пройти через него вместе с вами.
\vs p159 3:14 \pc И еще многому учил Иисус этих верующих перед тем, как они отошли ко сну. И внимавшие ему как сокровище хранили эти высказывания в сердцах своих и часто цитировали в назидание апостолам и ученикам, которым слышать их не довелось.
\usection{4. Беседа с Нафанаилом}
\vs p159 4:1 И затем пошел Иисус в Авилу, где трудился Нафанаил со своими сподвижниками. Нафанаила сильно смущали некоторые высказывания Иисуса, которые, казалось, принижали авторитет признанных еврейских писаний. Поэтому в эту ночь после обычного часа вопросов и ответов Нафанаил отвел Иисуса от остальных и спросил: «Учитель, можешь ли доверить мне знание истины о Писании? Я замечаю, ты учишь нас лишь части священных текстов --- как мне кажется, лучшим из них, --- из чего я заключаю: ты отвергаешь учения раввинов, которые указывают, что слова закона --- истинно слова Божии, бывшие с Богом на небе еще прежде времен Авраама и Моисея. Какова истина о Писании?» Услышав вопрос своего сбитого с толку апостола, Иисус ответил:
\vs p159 4:2 \pc «Ты правильно рассудил, Нафанаил; я не отношусь к Писанию, как раввины. И буду говорить с тобой об этом при условии, что ты ничего не расскажешь своим собратьям, ибо не все из них готовы принять это учение. Слов закона Моисеева и учений Писания до Авраама не существовало. Писание в том виде, в каком мы имеем его, было собрано воедино совсем недавно. Хотя в нем содержатся лучшие из высоких мыслей и чаяний еврейского народа, в нем также содержится много такого, что вовсе не отражает сущность и учения Отца Небесного; вот почему среди лучших писаний мне и приходится выбирать только те истины, которые должны быть тщательно подобраны для евангелия царства.
\vs p159 4:3 Писания эти --- плоды трудов людей, одни из которых были святыми, а другие нет. Эти книги отражают воззрения и уровень просвещенности тех времен, когда они возникли. Как откровение истины последние более совершенны, нежели первые. В Писании много заблуждений и в целом по своей сущности --- это человеческий документ, однако не ошибайся --- оно --- представляет собой лучшее собрание религиозной мудрости и духовной истины, которые только можно найти во всем мире в настоящее время.
\vs p159 4:4 Многие из этих книг были написаны не теми людьми, чье авторство им приписано, однако это ничуть не умаляет ценности истин, которые в них содержатся. Если история об Ионе и не является историческим фактом, если даже никакого Ионы никогда не было, все равно глубочайшая истина этого повествования, любовь Бога к Ниневее и так называемым язычникам совершенно не теряет своей ценности в глазах тех, кто любит своих собратьев. Писание священно, потому что в нем представлены мысли и описаны деяния людей, которые искали Бога и в этих трудах отразили свои высочайшие идеалы праведности, истины и святости. В Писании много, очень много истинного, однако в свете вашего сегодняшнего учения вы знаете, что эти тексты содержат и многое, неверно толкующее Отца Небесного, любящего Бога, открыть которого всем мирам я и пришел.
\vs p159 4:5 Нафанаил, никогда даже на минуту не позволяй себе верить текстам Писания, которые говорят вам, что Бог любви повелел вашим предкам пойти и в сражении убить всех своих врагов --- мужчин, женщин и детей. Подобные тексты --- это слова людей, не очень святых людей, а не слово Бога. Писания всегда отражают и будут всегда отражать интеллектуальный, моральный и духовный уровень тех, кто их создавал. Разве ты не заметил, что у пророков --- от Самуила к Исайе --- Яхве представляется все более благолепным и величественным. Ты также должен помнить, что Писание предназначено для религиозного наставления и духовного водительства. Это вовсе не сочинения историков или философов.
\vs p159 4:6 Наиболее прискорбна не эта ошибочная идея об абсолютном совершенстве текста Писания и непогрешимости его учений, но, скорее, вводящее в заблуждение неверное толкование этих священных текстов порабощенными традицией книжниками и фарисеями из Иерусалима. И ныне в своей решительной попытке противостоять более новым учениям евангелия царства они будут использовать как идею о непогрешимости Писания, так и свои проистекающие из нее неверные толкования. Никогда не забывай, Нафанаил, в откровении истины Отец не ограничивается каким\hyp{}то одним народом, каким\hyp{}то одним поколением. Многих из тех, кто искренне ищет истины, смущали и будут впредь смущать и приводить в уныние идеи о совершенстве Писания.
\vs p159 4:7 Авторитет истины --- это тот самый дух, что пребывает в ее живых проявлениях, а не мертвые слова менее просвещенных людей другого поколения, которым якобы было дано слово Божие. И даже если эти святые люди древности жили вдохновенными и проникнутыми духом жизнями, это не означает, что их \bibemph{слова} были в той же степени духовно вдохновенными. Сегодня мы не делаем записей учения евангелия царства, чтобы вы, когда я покину вас, из\hyp{}за несходства ваших толкований моих учений быстро не разделились на различные группы, соперничающие в борьбе за истину. Для сегодняшнего поколения лучше всего, чтобы мы являли эти истины в наших \bibemph{жизнях} и избегали каких бы то ни было записей.
\vs p159 4:8 Хорошо запомни слова мои, Нафанаил, ничто, чего коснулась человеческая природа, не может считаться непогрешимым. Разум человека действительно может отражать божественную истину, но при этом всегда относительной чистоты и частичной божественности. Создание может стремится к непогрешимости, но лишь один Творец обладает ей.
\vs p159 4:9 Однако величайшая ошибка учения о Писании --- это идея о том, что оно состоит из запечатанных книг тайны и мудрости, толковать которые смеют лишь мудрые умы нации. Откровения божественной истины запечатаны лишь человеческим невежеством, фанатизмом и узколобой нетерпимостью. Свет Писания лишь затуманивается предрассудками и затемняется суеверием. Ложный страх перед священностью помешал религии оградить себя здравым смыслом. Страх перед авторитетом священных текстов прошлого сильно мешает сегодняшним душам принять новый свет евангелия, свет, увидеть который так жаждали эти же самые знающие Бога люди прошлого поколения.
\vs p159 4:10 Самое же печальное из всего --- то, что некоторые проповедники святости подобного традиционализма эту истину знают. Они более или менее глубоко понимают эти недостатки Писания, однако нравственно трусливы и интеллектуально нечестны. Они знают истину о священных текстах, но предпочитают утаивать подобные подрывающие доверие факты от народа. Тем самым вместо того, чтобы обращаться к священным текстам как вместилищу нравственной мудрости, религиозного вдохновения и духовных учений познавших Бога людей иных поколений, они извращают и искажают Писание, делая его руководством по рабскому соблюдению мелочей повседневной жизни и авторитетом в делах недуховных».
\vs p159 4:11 \pc Речь Учителя просветила и потрясла Нафанаила. Наедине с собой Нафанаил долго размышлял о об этом разговоре, но до вознесения Иисуса никому ничего не рассказывал об этой беседе; и даже тогда боялся поведать всю историю наставлений Учителя.
\usection{5. Позитивная сущность религии Иисуса}
\vs p159 5:1 В Филадельфии, где трудился Иаков, Иисус учил учеников о позитивной сущности евангелия царства. Когда в ходе своих высказываний он упоминул, что в одних частях Писания больше истины, чем в других, и призвал своих слушателей питать свои души лучшей духовной пищей, Иаков перебил Учителя и спросил: «Не будешь ли ты так добр, Учитель, и не подскажешь ли, как нам для нашего личного назидания выбрать лучшие места из Писания?» И Иисус ответил: «Да, Иаков. Читая Писание, ищи вечно истинные и божественно прекрасные поучения, подобные этим:
\vs p159 5:2 „Сердце чистое сотвори во мне, Господи“.
\vs p159 5:3 \pc „Господь --- пастырь мой, я ни в чем не буду нуждаться“.
\vs p159 5:4 \pc „Возлюби ближнего твоего, как самого себя“.
\vs p159 5:5 \pc „Ибо я --- Господь Бог твой; держу тебя за правую руку твою и говорю тебе: не бойся, я помогу тебе“.
\vs p159 5:6 \pc „Не будут более народы учиться воевать“».
\vs p159 5:7 \pc Отсюда видно, как Иисус день за днем брал лучшее из еврейских писаний для наставления своих последователей и включал их в учения нового евангелия царства. Другие религии выдвигали мысль о близости Бога к человеку, Иисус же уподобил заботу Бога о человеке беспокойству любящего отца о благополучии зависящих от него детей, а затем сделал это положение краеугольным камнем своей религии. Эта идея об отцовстве Бога обязательно приводила к братству людей. Почитание Бога и служение человеку составили основу и сущность его религии. Иисус взял лучшее из еврейской религии и поместил его в замечательное обрамление новых учений евангелия царства.
\vs p159 5:8 Иисус привнес дух позитивного действия в пассивные доктрины еврейской религии. Вместо негативного подчинения обряду Иисус в соответствии со своей новой религией требовал позитивного действия от своих последователей. Религия Иисуса заключалась не просто в \bibemph{веровании,} но в активном \bibemph{действии,} именно к этому побуждало евангелие. Он вовсе не учил, что сущность его религии заключена в общественном служении, но что общественное служение является одним из конкретных следствий обладания духом истинной религии.
\vs p159 5:9 Иисус без колебаний брал лучшее из Писания и отвергал его менее ценное. Свой великий призыв: «Возлюби ближнего твоего, как самого себя» он взял из Писания, которое гласит: «Не мсти и не имей злобы на сынов народа твоего; но люби ближнего твоего, как самого себя». Иисус заимствовал только позитивную часть этой заповеди Писания, отвергнув ее негативную часть. Он возражал даже против негативного, или чисто пассивного непротивления. Он сказал: «Когда враг ударит тебя по одной щеке, не будь пассивен и не молчи, но прояви свою позитивную позицию и подставь другую щеку; то есть сделай все возможное, чтобы активным действием увести своего заблуждающегося брата с путей зла на светлые пути праведной жизни». Иисус требовал от своих последователей позитивной и агрессивной реакции на любую жизненную ситуацию. Обращение другой щеки или любой иной поступок, который может служить примером, требует инициативы, делает необходимым энергичное, активное и смелое проявление личности верующего.
\vs p159 5:10 Иисус не ратовал за негативную покорность унижениям со стороны тех, кто, возможно намеренно, пытался их причинить сторонникам непротивления злу, но учил стремиться к тому, чтобы его последователи были мудры и бдительны, быстро и позитивно отвечали добром на зло с тем, чтобы в конце концов победить добром зло. Не забывайте, истинное добро неизменно сильнее самого злого зла. Учитель учил позитивной мере праведности: «Кто хочет быть моим учеником, отрекись от себя, исполняй ежедневно полную меру своих обязанностей и следуй за мной». Именно так он жил сам, ибо «шел, благотворя». Эту сторону евангелия прекрасно поясняют многие притчи, которые он позднее рассказал своим последователям. Он никогда не призывал своих последователей терпеливо исполнять свой долг, но учил энергично и с энтузиазмом жить в полном соответствии со своими человеческими обязанностями и божественными привилегиями в царстве Бога.
\vs p159 5:11 Уча апостолов отдавать и рубашку, когда у них несправедливо отнимут плащ, Иисус вместо старого совета мстить --- «око за око» и т.д. имел в виду вовсе не предмет одежды в буквальном смысле, а идею сделать нечто \bibemph{позитивное} и спасти обидчика. Иисус питал отвращение и к идее мести, и к идее быть просто пассивным страдальцем или жертвой несправедливости. В этот раз он преподал им три правила борьбы со злом и сопротивления ему:
\vs p159 5:12 \ublistelem{1.}\bibnobreakspace Отвечать злом на зло --- позитивный, но неправедный метод.
\vs p159 5:13 \ublistelem{2.}\bibnobreakspace Терпеть зло, не жалуясь и не сопротивляясь --- исключительно негативный метод.
\vs p159 5:14 \ublistelem{3.}\bibnobreakspace Отвечать на зло добром, проявлять волю, стремясь стать хозяином положения, преодолевать зло добром --- позитивный и праведный метод.
\vs p159 5:15 \pc Однажды один из апостолов спросил: «Учитель, что мне делать, если незнакомый мне человек заставит меня нести его поклажу версту?» Иисус ответил: «Не садись и не вздыхай о том, чтобы стало легче, шепотом браня незнакомца. От подобного поведения праведность не придет. Если не можешь придумать чего\hyp{}нибудь более позитивного, то, по крайней мере, пронеси чужую поклажу еще версту. Это, несомненно, заставит неправедного и неблагочестивого незнакомца задуматься».
\vs p159 5:16 Евреи слышали о Боге, который прощает кающихся грешников и старается забыть их преступления, однако, пока не пришел Иисус, люди не слышали о Боге, который шел искать потерявшуюся овцу, беря на себя инициативу в поисках грешников, и радовался, когда видел их желание вернуться в дом Отца. Эта позитивная нота в религии Иисуса слышна даже в его молитвах. Он превратил негативное золотое правило в позитивное увещание человеческой справедливости.
\vs p159 5:17 Везде в своем учении Иисус неизменно избегал отвлекающих подробностей. Он остерегался пользоваться цветистым языком, и воздерживался от чисто поэтической игры слов. Как правило, большой смысл он вкладывал в краткие изречения. Например, Иисус изменял смысл многих современных ему понятий, таких как соль, закваска, рыбная ловля и малые дети. Он в высшей степени эффективно пользовался антитезой, сравнивая малое с бесконечным и т.д\ldots Его образы, такие как «слепой ведет слепого», поражали. Но величайшая сила его наглядного учения заключалась в его естественности. Иисус низвел философию религии с неба на землю. Элементарные потребности души он показывал с новым пониманием и с новым даром любви.
\usection{6. Возвращение в Магадан}
\vs p159 6:1 Четырехнедельное путешествие по Десятиградию было вполне успешным. Сотни душ были приняты в царство, и апостолы и евангелисты обрели ценный опыт осуществления своей деятельности без вдохновляющего непосредственного личного присутствия Иисуса.
\vs p159 6:2 Как было заранее договорено, в пятницу 16 сентября все труженики собрались в Магаданском лесу. Днем в субботу на встрече более ста верующих были подробно рассмотрены планы предстоящего расширения дела царства. Тут же присутствовали и вестники Давида, которые рассказали о том, как идут дела у верующих в Иудее, Самарии, Галилее и соседних с ними областях.
\vs p159 6:3 Немногие из последователей Иисуса в тот момент полностью оценили огромное значение службы отряда вестников. Они не только обеспечивали связь верующих друг с другом, с Иисусом и апостолами по всей Палестине, но и на протяжении этих тяжелых дней собирали средства для поддержания как Иисуса и его соратников, так и семей двенадцати апостолов и двенадцати евангелистов.
\vs p159 6:4 Приблизительно в это же время Авенир перенес центр своих действий из Хеврона в Вифлеем, где находился также штаб вестников Давида в Иудее. Давид организовал службу вестников, доставлявших послания за одну ночь между Иерусалимом и Вифсаидой. Эти гонцы каждый вечер покидали Иерусалим, передавали эстафету в Сихаре и в Скифополе и уже на следующее утро к завтраку прибывали в Вифсаиду.
\vs p159 6:5 Иисус и его соратники готовились к недельному отдыху перед тем, как начать последний этап своих трудов на благо царства. Это был их последний отдых, ибо путешествие по Перее было целиком посвящено проповедям и учению, и так, продолжалось вплоть до их прибытия в Иерусалим, где и произошли завершающие события земного пути Иисуса.
