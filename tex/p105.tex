\upaper{105}{Божество и реальность}
\vs p105 0:1 Бесконечность только частично доступна пониманию даже высоких чинов вселенских разумных существ, а завершенность реальности доступна их пониманию лишь относительно. Для человеческого разума, когда он стремится проникнуть в тайну\hyp{}вечность происхождения и предназначения всего, что называется \bibemph{реальным,} может оказаться полезным такой подход к проблеме, при котором вечность\hyp{}бесконечность воспринимается как почти беспредельный эллипс, порождаемый одной абсолютной причиной и функционирующий во всем этом вселенском круге нескончаемого разнообразия, причем всегда ища некий абсолютный и бесконечный потенциал предназначения.
\vs p105 0:2 Когда смертный интеллект пытается постичь понятие всей реальности в целом, такой конечный ум оказывается лицом к лицу с реальностью\hyp{}бесконечностью, вся реальность в целом \bibemph{есть} бесконечность, и, следовательно, никогда не может быть полностью осознана никаким разумом, способность которого к постижению понятий не является бесконечным.
\vs p105 0:3 Человеческий ум едва ли может сформировать адекватное понятие вечного существования, а без такого понимания невозможно описать даже наши понятия всей реальности в целом. Тем не менее, мы можем попытаться осуществить такое представление, хотя мы полностью отдаем отчет, что наши понятия должны подвергнуться глубокому искажению в процессе модификации, связанной с переводом на уровень понимания смертного сознания.
\usection{1. Философское понятие Я ЕСТЬ}
\vs p105 1:1 Абсолютную первопричину в бесконечности философы вселенных приписывают Отцу Всего Сущего, действующему как бесконечное, вечное и абсолютное Я ЕСТЬ.
\vs p105 1:2 Существует множество опасностей, связанных с представлением смертному разуму этой идеи бесконечного Я ЕСТЬ, так как это понятие настолько далеко от человеческого понимания, основанного на повседневном опыте, что влечет за собой серьезное искажение значений и создает неправильное представление о ценностях. Тем не менее, философское понятие Я ЕСТЬ может предоставить конечным существам некоторую основу для попытки подхода к частичному пониманию абсолютных первопричин и бесконечных предназначений. Но во всех наших попытках пролить свет на возникновение и осуществление реальности, проясним, что это понятие Я ЕСТЬ для всех личностных значений и ценностей является синонимом Первого Лица Божества, Отца всех личностей. Но этот постулат Я ЕСТЬ не может быть так же ясно идентифицирован в областях вселенской реальности, лишенных божественности.
\vs p105 1:3 \P\ \bibemph{Я ЕСТЬ есть Бесконечное; Я ЕСТЬ есть также бесконечность.} С последовательной, временной точки зрения, вся реальность берет свое начало в бесконечном Я ЕСТЬ, чье единичное существование в прошлой бесконечной вечности должно быть первым философским постулатом конечных созданий. Понятие Я ЕСТЬ означает \bibemph{неограниченную бесконечность,} недифференцированную реальность всего, что может существовать во всей бесконечной вечности.
\vs p105 1:4 Я ЕСТЬ как экзистенциальное понятие не является ни обожествленным, ни лишенным божественности, ни потенциальным, ни актуальным, ни личностным, ни не\hyp{}личностным, ни статическим, ни динамическим. Никакое определение не может быть применено к Бесконечному за исключением утверждения, что Я ЕСТЬ \bibemph{существует.} Философский постулат Я ЕСТЬ есть некое вселенское понятие, которое несколько более трудно усвоить, чем понятие Неограниченного Абсолюта.
\vs p105 1:5 Просто для конечного разума всегда должно существовать начало, и, хотя для реальности никогда не было истинного начала, все же существуют исходные отношения, которые реальность выражает бесконечности. До\hyp{}реальность, изначальная ситуация в вечности может представляться в таком виде: в некий бесконечно отдаленный гипотетический момент довечности Я ЕСТЬ может пониматься как вещь и как отсутствие вещи, как причина и как следствие, как волевой акт и как ответ на него. И в этот гипотетический момент вечности во всей бесконечности не существовало никакой дифференциации. Бесконечность была наполнена Бесконечным; Бесконечный окружает заключает в себе бесконечность. Таков был гипотетический статический момент вечности; актуальности еще содержались внутри своих потенциалов, и потенциалы еще не проявились внутри бесконечности Я ЕСТЬ. Но даже в такой предположительной ситуации мы должны допустить существование возможности своеволия.
\vs p105 1:6 \P\ Помните всегда, что постижение человеком Отца Всего Сущего есть его личный опыт. Как ваш духовный Отец Бог доступен для понимания вам и всем другим смертным; но \bibemph{почитаемое вами и основанное на опыте понятие об Отце Всего Сущего всегда должно быть уже, чем ваш философский постулат бесконечности Первоисточника и Центра, Я ЕСТЬ.} Когда мы говорим об Отце, мы подразумеваем Бога, доступного пониманию его созданий, как высших, так и низших, но Божество значительно больше, чем то, что может быть воспринято вселенскими созданиями. Бог, твой Отец и мой Отец, является той фазой Бесконечного, которую мы ощущаем в наших личностях как актуальную реальность, познаваемую на опыте, но Я ЕСТЬ всегда остается в качестве нашей гипотезы относительно всего того, что, как мы чувствуем, невозможно узнать о Первоисточнике и Центре. И даже такой гипотезы, вероятно, далеко недостаточно для объяснения непостижимой бесконечности изначальной реальности.
\vs p105 1:7 Вселенная вселенных с ее бесчисленным множеством обитающих там личностей есть громадный и сложный организм, но Первоисточник и Центр бесконечно более сложен, чем вселенные и личности, которые стали реальными в ответ на его преднамеренные установления. Когда ты испытываешь благоговейный трепет перед масштабом главной вселенной, обрати внимание, что даже это непостижимое творение есть не более чем частичное откровение Бесконечного.
\vs p105 1:8 Бесконечность, действительно, далека от опытного уровня смертного понимания, но даже в этот век на Урантии ваши представления о бесконечности развиваются, и они будут продолжать развиваться на всем протяжении вашей бесконечной деятельности, простирающейся в будущее вечности. Неограниченная бесконечность лишена смысла для конечных созданий, но бесконечность способна к самоограничению и восприимчива к выражению реальности на всех уровнях вселенского существования. И лик, который Бесконечное обращает ко всем вселенским личностям, есть лик Отца, любящего Отца Всего Сущего.
\usection{2. Я ЕСТЬ как триединый и как семеричный}
\vs p105 2:1 Рассматривая генезис реальности, всегда надо иметь в виду, что вся абсолютная реальность извечна и не имеет начала существования. Абсолютную реальность мы приписываем трем экзистенциальным лицам Божества, Райскому Острову и трем Абсолютам. Эти семь реальностей являются равнозначно вечными, несмотря на то, что мы прибегаем к языку пространства\hyp{}времени для представления людям их последовательного возникновения.
\vs p105 2:2 \P\ Согласно хронологическому описанию возникновения реальности, внутри Я ЕСТЬ должен существовать теоретически постулированный момент «первого» волевого выражения и «первой» ответной реакции. В наших попытках изобразить возникновение и образование реальности этот этап может восприниматься как самодифференциация \bibemph{Бесконечного} из \bibemph{Бесконечности,} но постулирование этой двойственной связи всегда должно быть расширено до представления о триедином посредством признания вечного континуума \bibemph{Бесконечности ---} Я ЕСТЬ.
\vs p105 2:3 Это самопревращение Я ЕСТЬ достигает своей высшей точки в многократной дифференциации обожествленной реальности и необожествленной реальности, потенциальной и актуальной реальности и некоторых других реальностей, которые едва ли можно классифицировать подобным образом. Эта дифференциация теоретически монистического Я ЕСТЬ вечно интегрируется при помощи одновременного возникновения связей внутри того же Я ЕСТЬ, предпотенциальной, предактуальной, предличностной, монотетичной предреальности, которая, хотя и будучи бесконечной, раскрывается как абсолютная в присутствии Первоисточника и Центра и как личность --- в беспредельной любви Отца Всего Сущего.
\vs p105 2:4 Посредством этих внутренних превращений Я ЕСТЬ создает основу для семеричной связи внутри себя. Философское (во времени) понятие единичного Я ЕСТЬ и промежуточное (во времени) понятие Я ЕСТЬ как триединого теперь могут быть расширены, чтобы заключить в себе Я ЕСТЬ как семеричного. Эта семеричная --- или семифазная --- природа может быть лучше всего представлена в связи с Семью Абсолютами Бесконечности:
\vs p105 2:5 \ublistelem{1.}\bibnobreakspace \bibemph{Отец Всего Сущего.} Я ЕСТЬ отец Вечного Сына. Это изначальная личностная связь актуальностей. Абсолютная личность Сына делает абсолютным факт Богоотцовства и создает возможность потенциального сыновства для всех личностей. Эта связь создает личность Бесконечного и завершает его духовное откровение в личности Первородного Сына. Эта фаза Я ЕСТЬ частично доступна опыту на духовном уровне даже для тех смертных, кто, будучи еще во плоти, почитают нашего Отца.
\vs p105 2:6 \P\ \ublistelem{2.}\bibnobreakspace \bibemph{Вселенский} \bibemph{Контролер.} Я ЕСТЬ причина вечного Рая. Это изначальная не\hyp{}личностная связь актуальностей; первоначальный не\hyp{}духовный союз. Отец Всего Сущего есть Бог\hyp{}любовь; Вселенский Контролер есть Бог\hyp{}паттерн. Эта связь создает возможность формы --- конфигурации --- и определяет главный паттерн не\hyp{}личностной и не\hyp{}духовной связи, главный паттерн, по которому делаются все копии.
\vs p105 2:7 \P\ \ublistelem{3.}\bibnobreakspace \bibemph{Творец Всего Сущего.} Я ЕСТЬ единое с Вечным Сыном. Это объединение Отца и Сына (в присутствии Рая) дает начало творческому циклу, который завершается появлением объединенной личности и вечной вселенной. С точки зрения конечного смертного, реальность берет свое истинное начало с появления в вечности --- создания Хавоны. Этот творческий акт Божества всецело совершается Богом Действия, который по сути представляет собой единство Отца\hyp{}Сына, выраженное на всех уровнях актуального. Следовательно, божественная способность творчества неизменно характеризуется единством, и это единство есть отражение вовне абсолютного тождества двуединства Отца\hyp{}Сына и Троицы --- Отца\hyp{}Сына\hyp{}Духа.
\vs p105 2:8 \P\ \ublistelem{4.}\bibnobreakspace \bibemph{Бесконечный Вседержитель.} Я ЕСТЬ самосвязующий. Это изначальный союз статики и потенциалов реальности. В этой связи уравновешивается все ограниченное и неограниченное. Эту фазу Я ЕСТЬ лучше всего понимать как Вселенский Абсолют --- объединяющий Божественный и Неограниченный Абсолюты.
\vs p105 2:9 \P\ \ublistelem{5.}\bibnobreakspace \bibemph{Бесконечный Потенциал.} Я ЕСТЬ самограничивающий. Это знак в бесконечности, представляющий вечное свидетельство волевого самоограничения Я ЕСТЬ, благодаря которому была достигнута троичность самовыражения и самооткровения. Эта сторона Я ЕСТЬ обычно понимается как Божественный Абсолют.
\vs p105 2:10 \P\ \ublistelem{6.}\bibnobreakspace \bibemph{Бесконечная Емкость.} Я ЕСТЬ статически\hyp{}реактивный. Это бесконечная матрица, возможность для всего будущего космического распространения. Эту сторону Я ЕСТЬ, по\hyp{}видимому, лучше всего понимать как сверхгравитационное присутствие Неограниченного Абсолюта.
\vs p105 2:11 \P\ \ublistelem{7.}\bibnobreakspace \bibemph{Всемирный Бесконечности.} Я ЕСТЬ как Я ЕСТЬ. Это стаз или самосвязь Бесконечности, вечный факт бесконечности\hyp{}реальности и универсальная истина реальности\hyp{}бесконечности. Поскольку эта связь различима как личность, она раскрывается вселенным в божественном Отце всех личностей --- даже абсолютных личностей. Поскольку эта связь выражена не\hyp{}личностно, она воспринимается вселенной как абсолютная связь чистой энергии и чистого духа в присутствии Отца Всего Сущего. Поскольку эта связь мыслится как абсолют, она раскрывается в первенстве Первоисточника и Центра; в нем мы все живем, передвигаемся и существуем --- от созданий пространства до граждан Рая; и это так же верно для главной вселенной, как и для бесконечно малых ультиматонов, так же верно для того, что будет, как для того, что есть и что было.
\usection{3. Семь абсолютов бесконечности}
\vs p105 3:1 Семь основных связей внутри Я ЕСТЬ увековечены как Семь Абсолютов Бесконечности. И хотя мы можем отобразить возникновение реальности и дифференциацию бесконечности в последовательном изложении событий, в действительности все семь Абсолютов неограниченно и одинаково вечны. Возможно, для смертного разума необходимо представить себе их начало, но такое представление должно быть отклонено осознанием того факта, что семь Абсолютов не имели начала; они --- вечны и такими были всегда. Семь Абсолютов являются предпосылкой реальности. Они описаны в этих текстах следующим образом:
\vs p105 3:2 \ublistelem{1.}\bibnobreakspace \bibemph{Первоисточник и Центр.} Первое Лицо Божества и главный не\hyp{}божественный паттерн, Бог, Отец Всего Сущего, творец, контролер и вседержитель, всеобщая любовь, вечный дух и бесконечная энергия; потенциал всех потенциалов и источник всех актуальностей; стабильность всего статичного и динамизм всех изменений; источник паттерна и Отец личностей. Вместе все семь Абсолютов эквивалентны бесконечности, но Отец Всего Сущего сам в действительности бесконечен.
\vs p105 3:3 \P\ \ublistelem{2.}\bibnobreakspace \bibemph{Второй Источник и Центр.} Второе Лицо Божества, Вечный и Первородный Сын; абсолютные личностные реальности Я ЕСТЬ и основа для реализации\hyp{}откровения «Я ЕСТЬ личность». Ни одна личность не может надеяться достичь Отца Всего Сущего иначе, чем через посредство его Вечного Сына; и личность не может достичь духовных уровней бытия без действия и помощи этого абсолютного паттерна для всех личностей. Во Втором Источнике и Центре дух неограничен, тогда как личность абсолютна.
\vs p105 3:4 \ublistelem{3.}\bibnobreakspace \bibemph{Райский Источник и Центр.} Второй небожественный паттерн, вечный Райский Остров; основа для реализации\hyp{}откровения «Я ЕСТЬ сила» и фундамент для установления гравитационного контроля во всех вселенных. По отношению ко всей актуализированной, недуховной, не\hyp{}личностной и не\hyp{}волевой реальности Рай является абсолютом паттернов. Как энергия духа связана с Отцом Всего Сущего через посредство абсолютной личности Матери\hyp{}Сына, так и вся космическая энергия держится под гравитационным контролем Первоисточника и Центра через посредство абсолютного паттерна Райского Острова. Рай не находится в пространстве; пространство существует по отношению к Раю, и длительность движения определяется при помощи его отношения к Раю. Вечный Остров находится в абсолютном покое; вся другая формированная и формирующая энергия находится в вечном движении; во всем пространстве только Неограниченный Абсолют находится в покое, и Неограниченный имеет одинаковый статус с Раем. Рай существует в фокусе пространства, Неограниченный наполняет его собой, и все существующее относительно имеет свое бытие внутри этой области.
\vs p105 3:5 \P\ \ublistelem{4.}\bibnobreakspace \bibemph{Третий Источник и Центр.} Третье Лицо Божества, Носитель Объединенных Действий; бесконечный интегратор Райской космической энергии с духовной энергией Вечного Сына; идеальный координатор мотивов воли и механики силы; унификатор всей актуальной и актуализирующейся реальности. Благодаря служению своего разнообразного потомства Бесконечный Дух раскрывает милосердие Вечного Сына, действуя в то же самое время как бесконечный манипулятор, навечно вплетая паттерн Рая в энергию пространства. Этот самый Носитель Объединенных Действий, этот Бог Действия является совершенным выражением безграничных планов и целей Отца\hyp{}Сына, причем сам он функционирует как источник разума и дарователь интеллекта созданиям необъятного космоса.
\vs p105 3:6 \P\ \ublistelem{5.}\bibnobreakspace \bibemph{Божественный Абсолют.} Причинные, потенциально личностные возможности вселенской реальности, тотальность всех Божественных потенциалов. Божественный Абсолют есть целенаправленный ограничитель неограниченных, абсолютных и небожественных реальностей. Божественный Абсолют ограничивает абсолютное и абсолютизирует ограниченное, он --- начинатель предназначения.
\vs p105 3:7 \P\ \ublistelem{6.}\bibnobreakspace \bibemph{Неограниченный Абсолют.} Статический, реактивный и незадействованный; нераскрытая космическая бесконечность Я ЕСТЬ; тотальность необожествленной реальности и финальность всех не\hyp{}личностных потенциалов. Действие Неограниченного ограничено пространством, но присутствие Неограниченного не имеет предела, оно бесконечно. Существует понятие периферии по отношению к главной вселенной, но присутствие Неограниченного безгранично; даже вечность не может исчерпать безмерный покой этого не\hyp{}божественного Абсолюта.
\vs p105 3:8 \P\ \ublistelem{7.}\bibnobreakspace \bibemph{Вселенский Абсолют.} Объединитель обожествленного и необожествленного; коррелятор абсолютного и относительного. Вселенский Абсолют (будучи статическим, потенциальным и связующим) компенсирует напряжение между всегда\hyp{}существующим и незавершенным.
\vs p105 3:9 \P\ Семь Абсолютов Бесконечности составляют начало реальности. Смертные умы могли бы полагать, что Первоисточник и Центр появился прежде всех абсолютов. Но такое утверждение, хотя и благодетельное, опровергается вечным сосуществованием Сына, Духа, трех Абсолютов и Райского Острова.
\vs p105 3:10 \bibemph{Истина} состоит в том, что Абсолюты есть проявления Я ЕСТЬ\hyp{}Первоисточник и Центр; \bibemph{факт} же состоит в том, что эти Абсолюты никогда не имели начала, и они с Первоисточником и Центром одинаково вечны. Связи абсолютов в вечности никогда не могут быть представлены без того, чтобы не возникли парадоксы в языке времени и в паттернах понятия пространства. Но несмотря на то, что представления о происхождении Семи Абсолютов Бесконечности очень запутаны, и факт, и истина состоят в том, что вся реальность основывается на их существовании в вечности и их связях в бесконечности.
\usection{4. Единство, двуединство и триединство.}
\vs p105 4:1 Философы вселенной постулируют существование Я ЕСТЬ в вечности как первоисточник всей реальности. И как сопутствующее этому обстоятельство они постулируют саморазделение Я ЕСТЬ на первичные связи, существующие в нем самом, --- семь фаз бесконечности. И одновременно с этим предположением имеется третий постулат\hyp{}появление в вечности Семи Абсолютов Бесконечности и увековечивание двуединого союза семи фаз Я ЕСТЬ и этих семи Абсолютов.
\vs p105 4:2 Таким образом,самораскрытие Я ЕСТЬ, начинаясь от статической самости, посредством саморазделения и образования внутренних связей переходит к абсолютным связям, связям с порождаемыми им самим Абсолютами. Двуединство оказывается, таким образом, существующим в вечном союзе Семи Абсолютов Бесконечности с семеричной бесконечностью саморазделившихся сторон самораскрывающегося Я ЕСТЬ. Эти двуединые связи, увековеченные для вселенных как семь Абсолютов, увековечивают главные основы для всей вселенской реальности.
\vs p105 4:3 Как уже говорилось, единство порождает двуединство, а двуединство порождает триединство, и триединство есть вечный прародитель всех вещей. Существует, конечно, три больших класса изначальных связей, и они таковы:
\vs p105 4:4 \ublistelem{1.}\bibnobreakspace \bibemph{Связи единства.} Связи, существующие внутри Я ЕСТЬ как единства, понимаются как результат троичного, а затем семеричного саморазделения.
\vs p105 4:5 \ublistelem{2.}\bibnobreakspace \bibemph{Связи двуединства.} Связи, существующие между Я ЕСТЬ как семеричным и Семью Абсолютами Бесконечности.
\vs p105 4:6 \ublistelem{3.}\bibnobreakspace \bibemph{Связи триединства.} Таковы функциональные союзы Семи Абсолютов Бесконечности.
\vs p105 4:7 \P\ Триединые связи возникают на основе двуединств вследствие неизбежности объединения Абсолютов между собой. Такие триединые союзы увековечивают потенциал всей реальности; они охватывают и обожествленную, и не\hyp{}обожествленную реальность.
\vs p105 4:8 Я ЕСТЬ как \bibemph{единство} является неограниченной бесконечностью. Двуединства увековечивают \bibemph{основы} реальности. Триединства выявляют реализацию бесконечности как вселенской \bibemph{функции.}
\vs p105 4:9 В семи Абсолютах пред\hyp{}экзистенциальное становится экзистенциальным, а экзистенциальное становится функциональным в триединствах, главных союзах Абсолютов. И увековечивание триединств сопутствует тому, что вселенная готова --- существуют потенциальности и присутствуют актуальности --- к диверсификации космической энергии, распространению Райского духа и наделению разумом наряду с дарованием личностности (и полнота вечности свидетельствует обо всем этом), благодаря чему все эти Божественные и Райские производные объединяются на тварном уровне посредством опыта, а на сверхтварном --- с помощью других методов.
\usection{5. Появление конечной реальности}
\vs p105 5:1 Так же, как первоначальная диверсификация Я ЕСТЬ должна быть отнесена за счет внутренне присущего неспровоцированного волевого акта, так и появление конечной реальности должно быть приписано волевым актам Райского Божества и ответным корректировкам функциональных триединств.
\vs p105 5:2 Может показаться, что до обожествления конечного вся диверсификация реальности происходит на абсолютных уровнях; но волевой акт, приводящий к появлению конечной реальности, означает ограничение абсолютности и подразумевает появление относительностей.
\vs p105 5:3 \P\ Хотя мы представляем это описание как хронологически последовательное изложение появления конечности как непосредственное производное абсолютности, необходимо иметь в виду, что трансцендентальности являются как предшествующими, так и последующими по отношению ко всему, что является конечным. Что касается конечного, то трансцендентальные пределы являются и причинными, и завершающими.
\vs p105 5:4 \P\ Возможности конечного внутренне присущи Бесконечности, но превращение возможности в вероятность и неизбежность должно быть приписано самосуществующей свободной воле Первоисточника и Центра, активирующего все триединые союзы. Только бесконечность воли Отца может так ограничить абсолютный уровень существования, чтобы выявить предельное или создать конечное.
\vs p105 5:5 С появлением относительной и ограниченной реальности возникает новый цикл реальности --- цикл развития --- величественное нисхождение с высот бесконечности в область конечного, вечно направленное вовнутрь к Раю и Божеству, всегда стремящееся к тем высоким предназначениям, которые соответствуют источнику бесконечности.
\vs p105 5:6 Эти непостижимые процессы знаменуют начало истории вселенной, знаменуют начало самого времени. Для тварного существа начало конечного \bibemph{есть} возникновение реальности; с точки зрения тварного разума, до возникновения конечного не существует никакой актуальности, доступной для понимания. Эта недавно появившаяся конечная реальность существует в двух первоначальных аспектах:
\vs p105 5:7 \ublistelem{1.}\bibnobreakspace \bibemph{Первичные максимумы,} верховно совершенная реальность --- вселенная и создания типа Хавоны.
\vs p105 5:8 \ublistelem{2.}\bibnobreakspace \bibemph{Вторичные максимумы,} реальность, сделанная верховно совершенной --- творение и создания типа сверхвселенной.
\vs p105 5:9 \P\ Итак это --- два первоначальных выражения конечной реальности: изначально совершенное и ставшее совершенным в результате эволюции. Они оба являются равнозначными с точки зрения взаимоотношений в вечности, но внутри временных границ они, по\hyp{}видимому, различны. Для того, что развивается, фактор времени означает развитие; вторичные конечные развиваются; отсюда --- то, что развивается, должно во времени казаться незавершенным. Но эти различия, которые столь важны по эту сторону Рая, в вечности не существуют.
\vs p105 5:10 Мы говорим о совершенных и ставших совершенными как о первичных и вторичных максимумах, но существует, однако, и другой тип: тринитизация и другие связи между первичными и вторичными максимумами приводят к появлению \bibemph{третичных максимумов,} вещей, значений и ценностей, которые не являются ни совершенными, ни доведенными до совершенства, тем не менее, они являются равнозначными по отношению к обоим предшествующим факторам.
\usection{6. Ответные реакции конечной реальности}
\vs p105 6:1 В целом все появление конечного существования представляет собой переход от потенциальностей к актуальностям внутри абсолютных союзов функциональной бесконечности. Среди множества ответных реакций на творческую актуализацию конечного могут быть названы:
\vs p105 6:2 \ublistelem{1.}\bibnobreakspace \bibemph{Отклик божества,} появление трех уровней верховенства опыта: актуальность верховенства личностного духа в Хавоне, возможность для верховенства личностной силы в будущей великой вселенной и способность разума, эволюционирующего в процессе опыта, к некоторому неизвестному функционированию, происходящему на некотором уровне верховенства, --- в будущей главной вселенной.
\vs p105 6:3 \P\ \ublistelem{2.}\bibnobreakspace \bibemph{Отклик вселенной,} вызывающий активацию архитектурных замыслов для пространственного уровня сверхвселенной, и эта эволюция все еще происходит во всей материальной организации семи сверхвселенных.
\vs p105 6:4 \P\ \ublistelem{3.}\bibnobreakspace \bibemph{Существа как ответная реакция} на распространение конечной реальности; это привело к появлению совершенных существ в чине вечных жителей Хавоны и достигших совершенства эволюционно восходящих из семи сверхвселенных. Но достижение совершенства в результате эволюционного (творящего во времени) опыта предполагает нечто иное, чем совершенство в качестве отправного пункта. Таким образом, в эволюционирующих созданиях возникает несовершенство. И это есть начало потенциального зла. Неспособность к адаптации, дисгармония и конфликт --- все это присуще эволюционному развитию --- от материальных вселенных до существ, наделенных личностью.
\vs p105 6:5 \P\ \ublistelem{4.}\bibnobreakspace \bibemph{Отклик божественности} на несовершенство, присущее временному запаздыванию эволюции, раскрывается в компенсирующем присутствии Бога Семеричного, благодаря действиям которого то, что совершенствуется, объединяется как с совершенным, так и достигшим совершенства. Это временное запаздывание неотделимо от эволюции, которая есть творчество во времени. Поэтому, а также по другим причинам, всемогущество Верховного зиждется на успехах божественности Бога Семеричного. Это временное запаздывание делает возможным участие живых созданий в божественном творении благодаря тому, что созданиям\hyp{}личностям позволено быть партнерами Божества в деле достижения максимального развития. Таким образом, даже материальный разум смертного создания становится партнером божественного Настройщика в совместном творении бессмертной души. К тому же Бог Семеричный обеспечивает способы компенсации изъянов врожденного совершенства, связанных с недостатком опыта, а также компенсирует ограничения несовершенства, существующие до начала восхождения.
\usection{7. Выявление трансцендентальностей}
\vs p105 7:1 Транцендентальности суббесконечны и субабсолютны, но при этом сверхконечны и сверхтварны. Трансцендентальности выявляются как объединяющий уровень, связующий сверхценности абсолютов с максимальными ценностями конечного. С точки зрения живого существа, то, что является трансцендентальным, по\hyp{}видимому, выявляется как следствие конечного; а с точки зрения вечности --- в предвидении конечного; существуют и те, кто рассматривают его как «пред\hyp{}эхо» конечного.
\vs p105 7:2 То, что является трансцендентальным, не обязательно чуждо развитию, но оно сверхэволюционно в конечном смысле; не чуждо оно и опыту, но представляет собой сверхопыт в той мере, в какой это понятие имеет смысл для живых созданий. Возможно, лучшей иллюстрацией такого парадокса является центральная вселенная совершенства: Едва ли она абсолютна --- только Райский Остров поистине абсолютен в «материализованном» смысле. Не является она и конечным эволюционирующим творением, как семь сверхвселенных. Хавона --- вечна, но не неизменна в том смысле, что не является вселенной, в которой отсутствует развитие. Она населена живыми созданиями (исконными жителями Хавоны), которые в действительности никогда не были созданы, ибо они существуют вечно. Таким образом, Хавона иллюстрирует нечто, не являющееся в точности конечным, но все же и не абсолютное. Далее, Хавона служит буфером между абсолютным Раем и конечными созданиями, поясняя к тому же функцию трансцендентальностей. Но сама Хавона не трансцендентальна: она --- Хавона.
\vs p105 7:3 Как Верховный ассоциируется с конечным, так и Предельный отождествляется с трансцендентальным. Но, хотя мы таким образом сравниваем Верховного с Предельным, их различает нечто большее, чем степень, различие кроется также и в качестве. Предельный есть нечто большее, чем проекция сверх\hyp{}Верховного на трансцендентальный уровень. Предельный является всем этим, но больше: Предельный есть выявление новых реальностей Божества, ограничение новых фаз, до той поры неограниченных.
\vs p105 7:4 \P\ Реальности, которые связаны с трансцендентальным уровнем, следующие:
\vs p105 7:5 \ublistelem{1.}\bibnobreakspace Божественное присутствие Предельного.
\vs p105 7:6 \ublistelem{2.}\bibnobreakspace Понятие главной вселенной.
\vs p105 7:7 \ublistelem{3.}\bibnobreakspace Архитекторы Главной Вселенной.
\vs p105 7:8 \ublistelem{4.}\bibnobreakspace Два чина Райских организаторов силы.
\vs p105 7:9 \ublistelem{5.}\bibnobreakspace Некоторые модификации могущества пространства.
\vs p105 7:10 \ublistelem{6.}\bibnobreakspace Некоторые ценности духа.
\vs p105 7:11 \ublistelem{7.}\bibnobreakspace Некоторые значения разума.
\vs p105 7:12 \ublistelem{8.}\bibnobreakspace Абсонитные качества и реальности.
\vs p105 7:13 \ublistelem{9.}\bibnobreakspace Всемогущество, всеведение и вездесущность.
\vs p105 7:14 \ublistelem{10.}\bibnobreakspace Пространство.
\vs p105 7:15 \P\ Вселенная, в которой мы теперь живем, может мыслиться как существующая на конечном, трансцендентальном и абсолютном уровнях. Это --- космическая сцена, на которой разворачивается бесконечная драма выражения личности и превращения энергии.
\vs p105 7:16 И все эти разнообразные реальности объединяются \bibemph{абсолютно} несколькими триединствами, \bibemph{функционально ---} Архитекторами Главной Вселенной и \bibemph{относительно ---} Семью Духами\hyp{}Мастерами, субверховными координаторами божественности Бога Семеричного.
\vs p105 7:17 Бог Семеричный представляет откровение личности и божественности Отца Всего Сущего созданиям, обладающим как максимальным, так и субмаксимальным статусом, но существуют другие семеричные связи Первоисточника и Центра, не имеющие отношения к выражению божественного духовного служения Бога, который есть дух.
\vs p105 7:18 \P\ В вечности прошлого силы Абсолютов, духи Божеств и личности Богов пришли в возбуждение в ответ на изначальное своеволие самосуществующего своеволия. В эту вселенскую эпоху мы все являемся свидетелями колоссальных последствий широко развернувшейся космической панорамы субабсолютного выражения безграничных потенциалов всех этих реальностей. И вообще, возможно, что продолжающаяся диверсификация первоначальной реальности Первоисточника и Центра может продвигаться вперед и вовне множество столетий, все вперед и вперед к отдаленным и непостижимым просторам абсолютной бесконечности.
\vs p105 7:19 [Представлено Мелхиседеком Небадона.]
