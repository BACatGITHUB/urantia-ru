\upaper{54}{Проблемы бунта Люцифера}
\author{Могучий Вестник}
\vs p054 0:1 Эволюционному человеку оказывается сложно до конца осознать важность и понять значения зла, заблуждения, греха и порока. Человек с трудом воспринимает, что противостоящие друг другу совершенство и несовершенство порождают потенциальное зло; что сталкивающиеся между собой истина и ложь создают заблуждение, помрачающее сознание; что божественный дар добровольного выбора выявляется в различных антагонистических областях греха и праведности; что настойчивые поиски божественности ведут к царству Бога, в противоположность тому, что постоянный отказ от нее ведет в мир порока.
\vs p054 0:2 Боги не создают зла, и не разрешают зло и бунт. Потенциальное зло является существующим во времени во вселенной, охватывающей различные уровни значений и ценностей совершенства. Грех потенциально присутствует во всех областях, где несовершенные существа наделены способностью выбирать между добром и злом. Само чреватое конфликтом присутствие истины и неправды, факта и лжи составляет потенциальную возможность заблуждения. Сознательный выбор зла есть грех; добровольный отказ от истины есть заблуждение, настойчивое следование греху и заблуждению есть порок.
\usection{1. Истинная и ложная свобода}
\vs p054 1:1 Изо всех запутанных проблем, возникающих из\hyp{}за бунта Люцифера, ни одна не вызывает таких больших осложнений, как неспособность незрелых эволюционных смертных провести различие между истинной и ложной свободой.
\vs p054 1:2 Истинная свобода --- это вековой поиск и награда за эволюционный прогресс. Ложная свобода --- это утонченный обман заблуждений времени и зла пространства. Длительная свобода основывается на реальности справедливости --- разумности, зрелости, братстве и равенстве.
\vs p054 1:3 Свобода --- это способ самоуничтожения космического существования, когда его движущие силы неразумны, неограниченны и бесконтрольны. Истинная свобода становится все больше и больше связанной с реальностью и всегда заботится о социальной беспристрастности, космической справедливости, вселенском братстве и о божественных обязанностях.
\vs p054 1:4 Свобода самоубийственна, когда она отделена от материальной справедливости, интеллектуальной честности, социальной терпимости, морального долга и духовных ценностей. Свободы не существует отдельно от космической реальности, и вся личностная реальность пропорциональна ее связям с божественностью.
\vs p054 1:5 Необузданное своеволие и не подчиняющееся никаким правилам самовыражение равносильны абсолютному эгоизму --- вершине безбожия. Свобода без связанного с ней и все большего овладения собственным «я» есть фикция, плод эгоистического смертного воображения. Свобода, мотивировкой для которой служит собственное «я», есть интеллектуальная иллюзия, жестокий обман. Своеволие, рядящееся в одежды свободы, есть предвестник жалкого рабства.
\vs p054 1:6 Истинная свобода --- сподвижница подлинного чувства собственного достоинства; ложная свобода --- союзница самолюбования. Истинная свобода --- плод самообладания; ложная свобода --- высокомерие самоутверждения. Самообладание ведет к альтруистическому служению; самолюбование стремится к эксплуатации других для эгоистического возвеличивания такого заблуждающегося индивидуума, так как хочет пожертвовать праведным достижением ради обладания несправедливой властью над себе подобными существами.
\vs p054 1:7 \pc Даже мудрость лишь тогда божественна и безопасна, когда она является космической по своему масштабу и духовной по своим побудительным причинам.
\vs p054 1:8 \pc Не существует большего заблуждения, чем тот вид самообмана, который ведет разумных существ к страстному желанию властвовать над другими существами для того, чтобы лишить этих лиц их естественных свобод. Золотое правило человеческой справедливости громко взывает против всех таких обманов, нечестности, эгоизма и неправедности. Лишь истинная и подлинная свобода совместима с царством любви и служением милосердия.
\vs p054 1:9 Как смеют своевольные создания во имя личной свободы покушаться на права своих собратьев, когда Верховные Правители вселенной отступают в милосердном уважении перед этими прерогативами воли и потенциалами личности! Ни одно существо, пользуясь своей мнимой личной свободой, не имеет права лишать любое другое существо тех привилегий существования, которые дарованы Творцами и должным образом уважаемы всеми их верными сподвижниками, подчиненными и подданными.
\vs p054 1:10 Эволюционный человек может оказаться вынужденным бороться за свои материальные свободы с тиранами и угнетателями в мире греха и порока или в ранние времена первобытной развивающейся сферы, но этого нет в моронтийных мирах или в сферах духа. Война есть наследие раннего эволюционного человека, но в мирах нормальной прогрессирующей цивилизации физическое сражение как способ улаживания расовых разногласий давно приобрело дурную славу.
\usection{2. Похищение свободы}
\vs p054 2:1 С Сыном и в Духе замышлял Бог вечную Хавону, и уже с тех пор был достигнут вечный паттерн равноправного участия в творении --- партнерского участия. Этот паттерн партнерского участия есть главный план для каждого из Сынов и Дочерей Бога, отправляющихся в пространство, чтобы попытаться скопировать во времени центральную вселенную вечного совершенства.
\vs p054 2:2 Каждому созданию каждого развивающегося мира, которое стремится исполнить волю Отца, предназначено стать партнером пространственно\hyp{}временных Творцов в этом великолепном начинании достижения совершенства посредством опыта. Если бы это не было верно, Отец едва ли одарил бы такие создания свободной творческой волей, и он бы не стал пребывать в них, реально осуществляя свое партнерство с ними посредством своего собственного духа.
\vs p054 2:3 \pc Безумием Люцифера была его попытка сделать то, что сделать нельзя, --- изъять время во вселенной опыта. Преступлением Люцифера была предпринятая им попытка лишить творческих прав каждую личность в Сатании, скрыто ограничить личное участие создания --- добровольное участие --- в длительной эволюционной борьбе за достижение --- и индивидуально, и коллективно --- статуса света и жизни. Поступая таким образом, этот бывший Владыка вашей системы временную цель своей собственной воли прямо противопоставил вечной цели воли Бога, которая раскрыта в даровании свободной воли всем личностным созданиям. Таким образом, бунт Люцифера грозил максимально возможно нарушить свободную волю восходящих и служителей системы Сатании --- бунт представлял собой угрозу навсегда лишить каждого из этих существ захватывающего переживания, связанного с внесением чего\hyp{}то личного и уникального в медленно воздвигаемый монумент опытной мудрости, который когда\hyp{}нибудь будет существовать как усовершенствованная система Сатании. Так и манифест Люцифера, маскируясь в одеяние свободы, высился в ясном свете разума как монументальная угроза окончательно похитить личную свободу и сделать это в масштабе, к которому за всю историю Небадона приближались лишь только дважды.
\vs p054 2:4 Коротко говоря, то, что Бог дал людям и ангелам, Люцифер хотел бы у них отобрать, а именно: божественное право участия в творении их собственной судьбы и судьбы их локальной системы обитаемых миров.
\vs p054 2:5 \pc Ни у одного существа во всей вселенной нет законной прерогативы лишать любое другое существо истинной свободы --- права любить и быть любимым, привилегии почитать Бога и служить своим собратьям.
\usection{3. Временная отсрочка правосудия}
\vs p054 3:1 Моральные создания эволюционных миров, обладающие волей, всегда озабочены не продуманным до конца вопросом, почему всемудрый Творец допускает зло и грех. Они не могут понять, что оба они неизбежны, если создание должно быть истинно свободно. Свободная воля развивающегося человека или утонченного ангела не является просто философским понятием, символическим идеалом. Способность человека выбирать добро или зло есть вселенская реальность. Эта свобода выбирать самостоятельно есть дар Верховных Правителей, и они не позволят ни существу, ни группе существ лишить какую\hyp{}либо отдельную личность в обширной вселенной этой божественно дарованной свободы --- даже ценой ограничения таких заблуждающихся и невежественных существ в их так называемой личной свободе.
\vs p054 3:2 Хотя сознательное и искреннее отождествление со злом (грех) эквивалентно небытию (уничтожению), между временем такого личностного отождествления со злом и исполнением наказания --- являющегося автоматическим результатом такого намеренного объятия зла, --- всегда должен существовать промежуток времени, достаточный для того, чтобы дать возможность вынести такое решение о вселенском статусе индивидуума, которое всецело удовлетворит всех связанных с ним вселенских личностей и которое будет настолько честным и справедливым, что получит одобрение самого грешника.
\vs p054 3:3 Но если такой вселенский бунтовщик вопреки реальности истины и добродетели отказывается одобрить вердикт и если виновный сознает в сердце своем справедливость осуждения, но отказывается в этом признаться, тогда исполнение приговора должно быть отложено с одобрения Древних Дней. И Древние Дней отказываются уничтожить любое существо до тех пор, пока не будут исчерпаны все моральные ценности и все духовные реалии в злодее и во всех тех связанных с ним существах, которые его поддерживают, и в тех, кто ему, возможно, симпатизирует.
\usection{4. Временная отсрочка, обусловленная милосердием}
\vs p054 4:1 Другая проблема, которую непросто объяснить в созвездии Норлатиадека, касается причин того, почему Люциферу, Сатане и падшим принцам позволялось так долго сеять раздоры, прежде чем они были арестованы, интернированы и осуждены.
\vs p054 4:2 Родители, родившие и воспитавшие детей, лучше способны понять, почему Михаил --- Творец\hyp{}отец может медлить в осуждении и уничтожении своих собственных Сынов. Притча Иисуса о блудном сыне является прекрасной иллюстрацией тому, как долго любящий отец может ждать раскаяния заблудшего сына.
\vs p054 4:3 То обстоятельство, что создание, творящее зло, может в действительности сознательно поступить плохо --- совершить грех, --- устанавливает факт наличия свободной воли и полностью оправдывает отсрочку исполнения правосудия, которая может иметь любую продолжительность, при условии, что предложенное милосердие способно привести к раскаянию и исправлению.
\vs p054 4:4 \pc Большинство свобод, которые искал Люцифер, он уже имел; остальные он должен был получить в будущем. Все эти драгоценные дары были потеряны в результате того, что была открыта дорога нетерпению и потворству желанию обладать всем тем, что желанно --- сейчас, причем обладать вопреки всем обязательствам уважать права и свободы всех других существ, составляющих вселенную вселенных. Этические обязательства являются врожденными, божественными и всемирными.
\vs p054 4:5 \pc Существует много причин, нам известных, почему Верховные Правители немедленно не уничтожили или не интернировали вождей бунта Люцифера. Существуют, без сомнения, и другие, возможно более веские причины, нам не известные. Милосердные детали этой отсрочки в исполнении правосудия были предложены лично Михаилом из Небадона. Если бы не любовь этого Творца\hyp{}отца к своим заблуждающимся Сынам, верховное правосудие сверхвселенной было бы осуществлено. Если бы такой эпизод, как бунт Люцифера, произошел в Небадоне в то время, как Михаил воплотился во плоти на Урантии, подстрекатели к такому злу, возможно, были бы немедленно и безусловно уничтожены.
\vs p054 4:6 Верховная справедливость может действовать мгновенно, когда она не сдерживается божественным милосердием. Но служение милосердия по отношению к детям времени и пространства всегда обеспечивает эту временную отсрочку, этот спасительный промежуток времени между севом и жатвой. Если посеянное семя есть добро, этот промежуток создает условия для испытания и укрепления характера; если посеянное семя есть зло, эта милостивая отсрочка дает время для раскаяния и исправления. Эта временная отсрочка в вынесении решения и исполнении приговора злодею присуща служению милосердия семи сверхвселенных. Это ограничение правосудия милосердием доказывает, что Бог есть любовь и что такой Бог любви господствует во вселенных и в милосердии управляет судьбой и осуждением всех его созданий.
\vs p054 4:7 Милосердные временные отсрочки осуществляются по указу свободной воли Творцов. Использование этого метода терпения в обращении с грешными бунтовщиками во вселенной должно приводить к добру. Хотя абсолютно верно, что добро не может происходить из зла в том, кто замышляет и осуществляет зло, равно верно и то, что все вещи (включая зло, потенциальное и явное) действуют вместе для добра всех существ, которые знают Бога, хотят исполнять его волю и являются восходящими к Раю в соответствии с его вечным планом и божественной целью.
\vs p054 4:8 Но эти милосердные отсрочки не бесконечны. Несмотря на длительную (соответственно тому, как исчисляется время на Урантии) задержку в вынесении решения относительно бунта Люцифера, мы можем свидетельствовать, что во время осуществления данного откровения на Уверсе происходило первое слушание дела Гавриил \bibemph{против} Люцифера, ожидающего своего решения, и вскоре после этого был выпущен указ Древних Дней, постановляющий, что отныне Сатана должен содержаться в мире заключения вместе с Люцифером. Этим Сатана был лишен возможности посещать в дальнейшем любой из падших миров Сатании. Правосудие в мире, где господствует милосердие, идет медленно, но верно.
\usection{5. Мудрость отсрочки}
\vs p054 5:1 Из многих известных мне причин, почему Люцифер и его сообщники не были раньше интернированы или осуждены, мне позволено перечислить следующие:
\vs p054 5:2 \ublistelem{1.}\bibnobreakspace Милосердие требует, чтобы каждый правонарушитель имел достаточно времени для того, чтобы сформулировать обдуманную и абсолютно сознательно вызванную позицию по отношению к своим злым мыслям и греховным действиям.
\vs p054 5:3 \pc \ublistelem{2.}\bibnobreakspace Верховное правосудие сдерживается любовью Отца; следовательно, правосудие никогда не уничтожит то, что милосердие может спасти. Каждому злодею даровано время, чтобы принять спасение.
\vs p054 5:4 \pc \ublistelem{3.}\bibnobreakspace Ни один любящий отец никогда не торопится покарать наказанием заблудшего члена своей семьи. Терпение не может выказываться независимо от времени.
\vs p054 5:5 \pc \ublistelem{4.}\bibnobreakspace Хотя правонарушение всегда вредно для семьи, мудрость и любовь убеждают честных детей терпеливо относиться к заблудшим братьям в течение времени, данного любящим отцом, в продолжение которого грешник может увидеть ошибки своего пути и принять спасение.
\vs p054 5:6 \pc \ublistelem{5.}\bibnobreakspace Независимо от отношения Михаила к Люциферу, безотносительно к тому, что он является Творцом\hyp{}отцом Люцифера, в компетенцию Сына\hyp{}Творца не входило применять дисциплинарное взыскание по отношению к отступническому Владыке Системы, потому что он тогда еще не завершил свой путь пришествий и не приобрел неограниченное владычество над Небадоном.
\vs p054 5:7 \pc \ublistelem{6.}\bibnobreakspace Древние Дней могли немедленно уничтожить этих бунтовщиков, но они редко наказывают правонарушителей, не проведя полного слушания их дел. В данном случае они отказались отменить решения Михаила.
\vs p054 5:8 \pc \ublistelem{7.}\bibnobreakspace Ясно, что Иммануил советовал Михаилу оставаться в стороне от бунтовщиков и позволить бунту идти естественным путем самоуничтожения. И мудрость Объединяющего Дней является отражением во времени объединенной мудрости Райской Троицы.
\vs p054 5:9 \pc \ublistelem{8.}\bibnobreakspace Верные Дней на Эдентии советовали Отцам Созвездий позволить бунтовщикам свободно идти до конца, говоря, что тем скорее будут искоренены все симпатии к этим злодеям в сердцах каждого теперешнего и будущего гражданина Норлатиадека --- каждого смертного, моронтийного или духовного создания.
\vs p054 5:10 \pc \ublistelem{9.}\bibnobreakspace На Иерусеме личный представитель Верховного Распорядителя Орвонтона советовал Гавриилу предоставить полную возможность каждому живому созданию обдуманно выбрать из тех положений, которые были включены в Декларацию Свободы Люцифера. Поднимая вопросы бунта, чрезвычайный Райский советчик Гавриила представил дело так, что, если такая полная и свободная возможность не будет дана всем созданиям Норлатиадека, то тогда должен быть выставлен карантин Рая от этих всевозможных нерешительных и охваченных сомнением созданий как средство самозащиты от всего созвездия. Чтобы держать Райские двери восхождения открытыми для существ Норлатиадека, необходимо обеспечить и полное развитие бунта, и полную определенность позиции всех существ, каким\hyp{}либо образом с ним связанных.
\vs p054 5:11 \pc \ublistelem{10.}\bibnobreakspace Божественная Служительница Спасограда выпустила в качестве своего третьего независимого объявления указ, призывающий не предпринимать ничего, чтобы хоть частично исправить, трусливо скрыть или иным образом замаскировать отвратительный облик бунтовщиков и бунта. Ангельским сонмам было указано работать, чтобы всецело изобличить грех и неограниченные возможности его проявления, ибо это самый быстрый способ совершенно и окончательно излечиться от чумы зла и греха.
\vs p054 5:12 \pc \ublistelem{11.}\bibnobreakspace На Иерусеме был организован чрезвычайный совет бывших смертных, состоящий из Могучих Вестников\hyp{}прославленных смертных, которые имели личный опыт участия в подобных ситуациях, вместе с их коллегами. Они сообщили Гавриилу, что, по меньшей мере, втрое большее число существ собьется с пути, если попытаться применить волевые или внесудебные меры подавления. Весь отряд советников Уверсы пришел к единому мнению, рекомендуя Гавриилу позволить бунту развиваться своим естественным путем, даже если потребуется миллион лет, чтобы уладить все его последствия.
\vs p054 5:13 \pc \ublistelem{12.}\bibnobreakspace Время относительно даже во вселенной времени: если урантийский смертный со средней продолжительностью жизни совершил бы преступление, которое повергло весь мир в ад кромешный, и если бы он был арестован, судим и казнен в течение двух или трех дней после совершения преступления, показалось бы это вам долгим сроком? И все же, в сравнении с продолжительностью жизни Люцифера, такое сопоставление лучше отвечало бы реальному положению вещей, даже если бы процесс по его делу, начавшийся сейчас, не был бы завершен еще в течение ста тысяч урантийских лет. С позиции Уверсы, где разбирательство не закончено, относительный промежуток времени можно было бы охарактеризовать, сказав, что преступление Люцифера было передано в суд в течение двух с половиной секунд после его совершения. С позиции Рая, это вынесение решения является одновременным с совершением преступления.
\vs p054 5:14 \pc Существует точное число причин произвольно не прекращать бунт Люцифера, которые частично могли бы быть вам понятны, но мне не позволено вам о них рассказывать. Я могу только сообщить, что на Уверсе мы преподаем сорок восемь причин, по которым злу позволяется пройти весь путь своего собственного морального банкротства и духовного вымирания. Я не сомневаюсь, что существует столько же дополнительных причин, мне не известных.
\usection{6. Триумф любви}
\vs p054 6:1 С какими бы трудностями ни встретились эволюционные смертные в своих попытках понять бунт Люцифера, всем здравомыслящим должно быть ясно, что методы обращения с бунтовщиками являются подтверждением божественной любви. Любящее милосердие, простертое к бунтовщикам, по\hyp{}видимому, вовлекло в испытания и несчастья многие невинные существа, но все эти несчастные личности могут спокойно положиться на всемудрых Судей, назначенных вынести решения относительно их судьбы в милосердии и по справедливости.
\vs p054 6:2 Во всех своих отношениях с разумными существами и Сын\hyp{}Творец, и его Райский Отец всецело преисполнены любовью. Невозможно понять многие аспекты отношения вселенских правителей к бунтовщикам и бунту --- к греху и грешникам, --- если не помнить, что Бог как Отец принимает главенство надо всеми аспектами выражения Божества во всех делах божественности с человечеством. Также необходимо вспомнить, что все Райские Сыны\hyp{}Творцы движимы милосердием.
\vs p054 6:3 \pc Если любящий отец большой семьи решает оказать милосердие одному из своих детей, виновному в тяжком правонарушении, вполне может случиться, что дарование милосердия этому недостойно себя ведущему ребенку создаст временные трудности для всех других добропорядочных детей. Такие случаи неизбежны; такой риск неотделим от реальной ситуации --- иметь любящего родителя и быть членом семейной группы. Каждому члену семьи идет на пользу праведное поведение любого члена семьи; аналогично, каждый член семьи страдает от немедленных последствий дурного поведения любого члена семьи. Семьи, группы, нации, расы, миры, системы, созвездия и вселенные представляют собой союзы основанные на связях внутри каждого самостоятельного сообщества, и, следовательно, каждый член любой такой группы, большой или малой, пожинает плоды добрых дел и страдает от последствий злых дел всех других членов данной группы.
\vs p054 6:4 Но одно следует прояснить: если вам пришлось страдать от дурных последствий греха, совершенного каким\hyp{}то членом вашей семьи, кем\hyp{}то из ваших сограждан или смертным собратом, даже от бунта в системе или где\hyp{}нибудь еще, --- не важно, что вам, может быть, придется вытерпеть из\hyp{}за правонарушения вашего сподвижника, собрата или начальника, --- вы можете быть спокойными в вечной уверенности, что такие несчастья --- преходящие страдания. Ни одно из этих общих последствий недостойного поведения в группе никогда не может представлять угрозу вашим вечным планам на будущее или хоть в малейшей степени лишить вас вашего божественного права восхождения к Раю и достижения Бога.
\vs p054 6:5 И существует воздаяние за эти испытания, отсрочки и разочарования, которые неизменно сопровождают грех бунта. Из многих важных последствий бунта Люцифера, которые могут быть названы, я лишь хочу обратить внимание на возвышенный путь тех восходящих смертных, граждан Иерусема, которые своим сопротивлением софизмам греха поставили себя в один ряд с теми, кто имеет шанс стать в будущем Могучими Вестниками, членами моего собственного чина. Каждое существо, которое выдержало испытание этой годины зла, немедленно продвинулось в своем административном статусе и повысило свою духовную ценность.
\vs p054 6:6 \pc Сначала казалось, что бунт Люцифера --- это абсолютное бедствие для системы и вселенной. Однако постепенно все больше стали проявляться и положительные последствия. По прошествии двадцати пяти тысяч лет системного времени (двадцати тысяч лет урантийского времени) Мелхиседеки начали учить, что добро, появившееся вследствие безумия Люцифера, сравнялось с причиненным злом. Сумма зла к этому времени стала почти постоянной, продолжая увеличиваться только в некоторых изолированных мирах, в то время как благотворные последствия продолжали приумножаться и распространяться по всей вселенной и сверхвселенной и даже до Хавоны. В настоящее время Мелхиседеки учат, что добро, появившееся в результате бунта Сатании, больше чем в тысячу раз превосходит всю совокупность зла.
\vs p054 6:7 Но такие чрезвычайные и благотворные плоды злых дел могли появиться только в результате мудрого, божественного и милосердного отношения всех стоящих выше Люцифера --- от Отцов Созвездия на Эдентии до Отца Всего Сущего в Раю. Течение времени увеличило силу добра, проистекающего из безумия Люцифера; и так как зло, которое должно быть наказано, окончательно сформировалось за сравнительно короткое время, ясно, что всемудрые и дальновидные правители вселенной были уверены, что нужно продлить время, в течение которого можно достигнуть еще более благоприятных результатов. Безотносительно ко многим другим доводам в пользу отсрочки ареста и вынесения приговора бунтовщикам Сатании, одного этого достаточно, чтобы объяснить, почему эти грешники не были интернированы раньше и почему они не были осуждены и уничтожены.
\vs p054 6:8 Близорукий и ограниченный временем разум смертных должен быть осмотрительным в критике временного промедления в действиях дальновидных и всемудрых руководителей вселенских дел.
\vs p054 6:9 Одно из человеческих заблуждений относительно этих проблем состоит в представлении, что все эволюционные смертные на развивающейся планете решили бы вступить на путь Рая, если бы их мир не был мучим грехом. Способность отказаться от продолжения существования в посмертии появилась не во времена бунта Люцифера. Смертный человек всегда обладал даром добровольного выбора по отношению к Райскому пути.
\vs p054 6:10 \pc Совершая восхождение и переживая опыт продолжения существования, вы расширите свои представления о вселенной и раздвинете свой горизонт значений и ценностей; и таким образом вы сможете лучше понять, почему таким существам, как Люцифер и Сатана, позволено продолжать бунт. Вам также станет более ясно, как из ограниченного во времени зла в конечном счете (если не немедленно) может произойти добро. После того, как вы достигнете Рая, вы будете действительно просвещены и утешены, слушая философов\hyp{}супернафимов, обсуждающих и объясняющих эти сложнейшие проблемы вселенского урегулирования. Но я сомневаюсь, что даже тогда в глубине души вы будете внутренне полностью удовлетворены. По крайней мере, я не был удовлетворен даже тогда, когда достиг высот вселенской философии. Я стал вполне понимать эти сложности лишь после того, как был назначен на административный пост в сверхвселенной, где благодаря реальному опыту приобрел способность адекватно воспринимать эти многосторонние проблемы космической справедливости и духовной философии. По мере того, как вы будете восходить к Раю, вы будете все больше и больше узнавать, что многие непонятные стороны вселенского управления могут быть осмыслены только после того, как вы увеличите свой опыт и достигнете более высокого уровня духовной проницательности. Для постижения космических ситуаций существенной является космическая мудрость.
\vsetoff
\vs p054 6:11 [Представлено Могучим Вестником, пережившим опыт существования в первом бунте системы во вселенных времени, в настоящий период приданным правительству сверхвселенной Орвонтона и действующим в этом деле по просьбе Гавриила из Спасограда.]
