\upaper{78}{Фиолетовая раса после времен Адама}
\author{Архангел}
\vs p078 0:1 Второй Эдем в течение почти тридцати тысяч лет был истоком цивилизации. Отсюда, из Месопотамии, Адамические народы продвигались далее, посылая свое потомство в различные концы земли, и впоследствии, смешавшись с Нодитами и племенами Сангика, стали известны как Андиты. Из этих мест пошли те мужчины и женщины, которые положили начало деяниям исторических времен и которые чрезвычайно ускорили развитие культуры на Урантии.
\vs p078 0:2 В этом тексте описывается планетарная история фиолетовой расы, от срыва Адама, что произошло около 35 000 года до н.э., до ее слияния с Нодитами и расами Сангика около 15 000 года до н.э., в результате чего возникли Андитские народы, и далее --- до, примерно, 2000 года до н.э., когда они окончательно исчезли из родной Месопотамии
\usection{1. Распределение рас и культур}
\vs p078 1:1 Хотя к моменту прибытия Адама разум и мораль этих рас были на низком уровне, физическая эволюция продолжалась; ее совершенно не затронули обстоятельства восстания Калигастии. Адам, безусловно, способствовал улучшению биологического состояния рас и значительному продвижению вперед народов Урантии, несмотря на то, что его предприятие не вполне удалось.
\vs p078 1:2 Деятельность Адама и Евы благотворно сказалось также на социальном, моральном и интеллектуальном прогрессе человечества; их потомки чрезвычайно ускорили процесс цивилизации. Но тридцать пять тысяч лет назад мир в целом обладал скудной культурой. Отдельные очаги цивилизации существовали здесь и там, но большая часть Урантии прозябала в дикости. Расы и культуры распределились следующим образом:
\vs p078 1:3 \ublistelem{1.}\bibnobreakspace \bibemph{Фиолетовая раса --- Адамиты и потомки Адама\hyp{}сына.} Главный центр Адамической культуры располагался во втором саду, находившемся в междуречьи Тигра и Евфрата; и в самом деле, это был исток Восточной и Индийской цивилизаций. второстепенный, или северный центр фиолетовой расы находился в поселениях потомков Адама\hyp{}сына, расположенных на востоке южного побережья Каспийского моря вблизи гор Копета. Из этих двух центров пошла в окружающие земли культура и жизненная плазма, что сразу же ускорило развитие всех рас.
\vs p078 1:4 \pc \ublistelem{2.}\bibnobreakspace \bibemph{Предшественники Шумеров и другие Нодиты.} Остатки древней культуры времен Даламатии также существовали в Месопотамии, неподалеку от устья рек. Эта группа по прошествии тысячелетий смешалась с Адамитами на севере, но частично сохранила свои Нодические традиции. Различные другие группы Нодитов, осевшие в Леванте, позднее были, в общем, поглощены распространившейся фиолетовой расой.
\vs p078 1:5 \pc \ublistelem{3.}\bibnobreakspace \bibemph{Андониты} обосновались в пяти или шести типичных поселениях к северу и востоку от центра Адама\hyp{}сына. Они были также рассеяны по всему Туркестану, да и во всей Евразии ими были населены отдельные местности, особенно в горных областях. Этим аборигенам до сих пор принадлежат северные земли Евразийского континента, включая Исландию и Гренландию, но они уже давным\hyp{}давно были вытеснены из долин Европы голубой расой, а из долин рек Азии --- распространяющейся желтой расой.
\vs p078 1:6 \pc \ublistelem{4.}\bibnobreakspace \bibemph{Красная раса,} вытесненная из Азии больше чем за пятьдесят тысяч лет до прибытия Адама, нашла пристанище в Америке.
\vs p078 1:7 \pc \ublistelem{5.}\bibnobreakspace \bibemph{Желтая раса.} Китайские народы установили прочный контроль над восточной Азией. Их наиболее развитые поселения были расположены к северо\hyp{}западу от современного Китая в районах, граничащих с Тибетом.
\vs p078 1:8 \pc \ublistelem{6.}\bibnobreakspace \bibemph{Голубая раса.} Голубая раса была разбросана по всей Европе, но главные центры ее культуры были расположены в плодородных долинах бассейна Средиземноморья и в северо\hyp{}западной Европе. Смешение с неандертальцами очень замедлило развитие культуры голубой расы, но, с другой стороны, из всех эволюционирующих народов Урантии это были самые энергичные и смелые народы, стремящиеся к освоению новых земель.
\vs p078 1:9 \pc \ublistelem{7.}\bibnobreakspace \bibemph{Индия до Дравидов.} Сложная смесь рас в Индии --- включающая все расы земли, но, в основном, зеленую, оранжевую и черную, --- обладала культурой, которая ненамного превосходила культуру отдаленных областей.
\vs p078 1:10 \pc \ublistelem{8.}\bibnobreakspace \bibemph{Цивилизация Сахары.} Находящиеся на более высоком уровне развития представители синей расы жили в процветающих поселениях в том регионе, где сегодня находится великая пустыня Сахара. Эта сине\hyp{}черная группа несла в себе многочисленные наследственные свойства затонувших оранжевых и зеленых народов.
\vs p078 1:11 \pc \ublistelem{9.}\bibnobreakspace \bibemph{Бассейн Средиземноморья.} За пределами Индии наиболее смешанная раса расположилась в области современного Средиземноморского бассейна. Здесь голубые люди с севера и выходцы из Сахары с юга встретились и перемешались с Нодитами и Адамитами с востока.
\vs p078 1:12 \pc Такова была картина мира около двадцати тысяч лет назад к началу великой экспансии фиолетовой расы. Второй сад в междуречьи Месопотамии был надеждой будущей цивилизации. Здесь, в юго\hyp{}западной Азии существовала потенциальная возможность зарождения великой цивилизации, возможность распространения в мир идей и идеалов, которые удалось сохранить с эпохи Даламатии и времен Эдема.
\vs p078 1:13 Адам и Ева оставили после себя небольшое, но могучее потомство, и небесные наблюдатели на Урантии с нетерпением ожидали того момента, когда станет ясно, как поведут себя эти потомки заблуждавшихся Материальных Сына и Дочери.
\usection{2. Адамиты во втором саду}
\vs p078 2:1 В течение тысячелетий сыны Адама трудились в бассейне рек Месопотамии, решая проблемы ирригации и контроля за наводнениями на юге, совершенствуя свои оборонительные сооружения на севере, пытаясь сохранить славные традиции первого Эдема.
\vs p078 2:2 Мужество, проявленное при управлении вторым садом, составляет одну из удивительных и вдохновляющих героических эпизодов Урантийской истории. Эти замечательные души никогда не теряли из вида цель Адамической миссии и поэтому храбро противодействовали влиянию окружающих племен, находившихся на более низком уровне развития, они постоянно и с готовностью посылали в качестве эмиссаров к расам земли самых лучших своих сынов и дочерей. Были времена, когда эта экспансия приводила к истощению родной культуры, но всегда эти народы, стоящие на более высоком уровне развития, находили в себе силы и восстанавливали ее.
\vs p078 2:3 Уровень развития цивилизации, общества и культуры Адамитов был гораздо выше, чем у эволюционирующих рас на Урантии. Только древние поселения Вана и Амадона и потомков Адама в какой\hyp{}то степени сравнимы с цивилизацией Адамитов. Но цивилизация второго Эдема была искусственным образованием --- \bibemph{она не являлась результатом эволюции ---} и была, следовательно, обречена на деградацию до тех пор, пока не достигнет естественного эволюционного уровня.
\vs p078 2:4 Адам оставил после себя замечательную интеллектуальную и духовную культуру, но она опережала технические достижения, поскольку каждая цивилизация ограничена тем, какие в ее распоряжении имеются природные ресурсы, гениальные люди, и есть ли у людей достаточно досуга для того, чтобы посвятить его изобретательству. Цивилизация фиолетовой расы опиралась на присутствие Адама и традиции первого Эдема. После смерти Адама и по мере того, как эти традиции с течением тысячелетий стали блекнуть, культурный уровень Адамитов постоянно снижался до тех пор, пока не сравнялся с уровнем окружающих народов и культурными возможностями естественно эволюционирующей фиолетовой расы.
\vs p078 2:5 Но около 19 000 года до н.э. Адамиты действительно являли собой нацию, насчитывавшую четыре с половиной миллиона, а миллионы их потомков уже влились в окружающие их народы.
\usection{3. Раннее распространение Адамитов}
\vs p078 3:1 Фиолетовая раса сохраняла миролюбивые Эдемические традиции в течение многих тысячелетий, чем и объясняется длительное отсутствие территориальной экспансии. Когда они начинали страдать от перенаселенности, вместо того, чтобы вести войну за захват большей территории, они посылали часть населения к другим расам в качестве учителей. Культурное влияние таких первых миграций было непродолжительным, но поглощение Адамитов\hyp{}учителей, торговцев и первопроходцев биологически укрепляло окружающие народы.
\vs p078 3:2 Некоторые Адамиты первые путешествия проделали на запад, к нильской долине, другие проникли на восток, в Азию, но таких было меньшинство. В более поздние времена особенно интенсивно происходило массовое переселение на север, а затем на запад. Это было, в основном, постепенное, но настойчивое продвижение на север, причем большинство Адамитов шло на север, а затем, обогнув Каспийское море, устремлялось на запад, в Европу.
\vs p078 3:3 Около двадцати пяти тысяч лет назад многие из наиболее чистокровных представителей Адамитов уже были на пути к северу. Но по мере того, как они проникали туда, они все больше и больше утрачивали Адамические свойства и ко времени оккупации ими Туркестана полностью смешались с другими расами, в частности, с Нодитами. Лишь очень немногие чистокровные представители фиолетовой расы проникли вглубь Европы или Азии.
\vs p078 3:4 Приблизительно от 30 000 года и до 10 000 года до н.э. по всей юго\hyp{}западной Азии происходили эпохальные по своему значению смешения рас. Обитатели нагорий Туркестана были мужественным и сильным народом. К северо\hyp{}западу от Индии продолжали существовать остатки культуры времен Вана. К северу от этих мест все еще жили лучшие из первых потомков Андонитов. И обе эти расы, обладающие высоким уровнем культуры и сильным характером, были поглощены мигрирующими на север Адамитами. Это слияние привело к восприятию многих новых идей; оно способствовало развитию цивилизации и чрезвычайно обогатило все аспекты искусства, науки и социальной культуры.
\vs p078 3:5 \pc Когда около 15 000 года до н.э. период ранних миграций Адамитов закончился, потомков Адама в Европе и Средней Азии уже было больше, чем в любой другой части мира, больше даже, чем в Месопотамии. Особенно значительным было проникновение в европейские голубые расы. Южные области современной России и Туркестана на всем протяжении южных границ были заняты огромной массой Адамитов, смешавшихся с Нодитами, Андонитами, красными и желтыми народами Сангика. Южная Европа и полоса, окаймляющая Средиземноморье, были заняты смешанной расой Андонитов и народов Сангика --- оранжевых, зеленых и синих --- с вкраплениями рода Адамитов. Малая Азия и центр восточной Европы были захвачены племенами, состоящими преимущественно из Андонитов.
\vs p078 3:6 Смешанная цветная раса, ассимилировавшая к этому времени значительное число мигрантов из Месопотамии, осела в Египте и была готова перенять исчезающую культуру долины Евфрата. Черные народы продвигались далее на юг, в Африку, и в конце концов оказались, как и красная раса, фактически в изоляции.
\vs p078 3:7 Цивилизация Сахары была разрушена засухой, а цивилизация бассейна Средиземноморья --- наводнением. Голубые народы пока еще не успели развить высокую культуру. Андониты все еще оставались разбросанными по просторам Арктики и Центральной Азии. Зеленая и оранжевая расы как таковые были истреблены. Синяя раса мигрировала на юг, в Африку, где началось ее медленное и продолжительное вырождение.
\vs p078 3:8 Народы Индии переживали период стагнации, и их цивилизация не развивалась; желтая раса упрочила свое господство в Центральной Азии; коричневая раса еще не начала создавать свою цивилизацию на близлежащих островах Тихого океана.
\vs p078 3:9 \pc Вышеописанное распределение рас, связанное с широкомасштабными климатическими изменениями, было всемирным процессом начального этапа эры Андитов в Урантийской цивилизации. Эти первые переселения народов длились более десяти тысяч лет, с 25 000 года до 15 000 года до н.э. Более поздние миграции (или миграции Андитов) продолжались с 15 000 года до 6000 года до н.э.
\vs p078 3:10 На то, чтобы первые волны Адамитов прокатились по Евразии, ушло так много времени, что их культура была, в значительной степени, утеряна в этих переходах. Только более поздние Андиты, продвигавшиеся достаточно быстро, смогли донести Эдемическую культуру на большее расстояние от Месопотамии.
\usection{4. Андиты}
\vs p078 4:1 Андиты возникли как результат смешения в основном чистокровной фиолетовой расы с Нодитами и с эволюционирующими народами. В общем, надо думать, что в жилах Андитов текло гораздо больше Адамической крови, чем у современных рас. В основном, термин Андиты обозначает те народы, у которых от одной шестой до одной восьмой совокупных природных свойств унаследовано от фиолетовой расы. Современные Урантиане, даже представители белых северных рас, обладают гораздо меньшей долей крови Адама.
\vs p078 4:2 Самые первые народы Андитов возникли более двадцати пяти тысяч лет назад в областях, прилегающих к Месопотамии, в результате смешения Адамитов и Нодитов. Доля фиолетовой крови уменьшалась по концентрическим кругам по мере удаления от второго сада, и по краю этого котла, в котором смешавалась кровь различных рас и племен, и образовалась раса Андитов. Впоследствии, когда мигрирующие Адамиты и Нодиты вступили в плодородные области Туркестана, они быстро смешались с местными жителями, находившимися на более высоком уровне развития, возникла смешанная раса, которая в дальнейшем и продвинула Андический тип далее на север.
\vs p078 4:3 Во всех отношениях Андиты были лучшим человеческим родом, который появился на Урантии со времен чистокровной фиолетовой расы. Они вобрали в себя, в основном, лучшие качества уцелевших остатков рас Адамитов и Нодитов, а позднее --- и некоторых из лучших родов желтых, голубых и зеленых людей.
\vs p078 4:4 \pc Первые Андиты не были Арийцами, они были предшественниками Арийцев. Они не были белыми, они были предшественниками белых. Это не был ни западный, ни восточный народ. Но именно Андическая наследственность дала этой многоязычной смеси так называемых белых рас ту всеобъемлющую однородность, которая получила название Европеоид).
\vs p078 4:5 \pc Более чистокровные линии фиолетовой расы сохранили Адамическую традицию миролюбия, это объясняет, почему ранние экспансии этой расы носили характер мирных миграций. Но как только Адамиты соединились с родами Нодитов, а к тому времени это были воинственные народы, их Андические потомки превратились в самых искусных и прозорливых (для своего времени) воинов, когда\hyp{}либо живших на Урантии. Отныне экспансия народов Месопотамии все больше и больше принимала милитаристский характер и стала больше походить на настоящий захват территории.
\vs p078 4:6 Андиты были предприимчивыми людьми, склонными к странствиям. Но кровь племен Сангика или Андонитов действовала на них как стабилизирующий фактор. Однако все равно их дальние потомки не успокоились до тех пор, пока не обошли земной шар и не открыли последний отдаленный континент.
\usection{5. Миграции Андитов}
\vs p078 5:1 Культура второго сад продолжала существовать в течение двадцати тысяч лет, но она постепенно приходила в упадок приблизительно до 15 000 года до н.э., пока, наконец, духовное возрождение священников\hyp{}Сифитов и руководство Амосада не положили начало блестящей эпохе. За мощными волнами развития цивилизации, которые позднее прокатились по Евразии, сразу же последовало великое возрождение Сада как результат широкомасштабного союза Адамитов с окружающими смешанными Нодитами, от которого и пошла раса Андитов.
\vs p078 5:2 Эти Андиты положили начало новому продвижению по всей Евразии и северной Африке. Культура Андитов господствовала от Месопотамии до Синцзяна, а постоянный исход в Европу непрерывно компенсировался новыми пришельцами из Месопотамии. Но говорить об Андитах как о сформировавшейся в Месопотамии расе можно только приблизительно с начала последних миграций смешанных потомков Адама. К этому времени даже расы во втором саду так перемешались, что больше не могли считаться Адамитами.
\vs p078 5:3 Цивилизация Туркестана постоянно оживлялась и обновлялась очередными пришельцами из Месопотамии, особенно поздними Андитами\hyp{}всадниками. В нагорьях Туркестана складывался так называемый Арийский праязык; это была смесь Андонического диалекта этого региона с языком потомков Адама\hyp{}сына и более поздних Андитов. Многие современные языки произошли от этого древнего говора племен Центральной Азии, которые покорили Европу, Индию и северные районы долин Месопотамии. Этот древний язык придал восточным языкам все то сходство, которое называется Арийским.
\vs p078 5:4 \pc К 12 000 году до н.э. три четверти Андитов жили в северной и восточной Европе, а при последнем и окончательном исходе из Месопотамии в Европу ушли шестьдесят пять процентов этих последних мигрантов.
\vs p078 5:5 \pc Андиты переселились не только в Европу, но и в северный Китай, и в Индию, в те же времена многие из них проникали как миссионеры, учители и торговцы в самые дальние уголки земли. Они значительно способствовали развитию северных племен народов Сангика Сахары. Но лишь некоторые учители и торговцы смогли проникнуть на юг Африки далее истоков Нила. Впоследствии потомки Андитов и Египтян прошли на юг вдоль и восточного, и западного берегов Африки значительно дальше экватора, но так и не достигли Мадагаскара.
\vs p078 5:6 Эти Андиты были так называемыми Дравидскими и более поздними --- Арийскими --- завоевателями Индии. Их влияние в Центральной Азии обусловило значительное развитие предков урало\hyp{}алтайских народов. Многие представители этой расы двигались в Китай через Синцзян и Тибет, и они привнесли ряд положительных качеств в наследственность более поздних китайских племен. Время от времени небольшие группы проникали в Японию, на Формозу, в Индонезию и южный Китай, хотя лишь единицы вошли в южный Китай, двигаясь вдоль побережья.
\vs p078 5:7 Сто тридцать два представителя этой расы отплыли из Японии на флотилии небольших лодок и через какое\hyp{}то время достигли Южной Америки. Вступив в брак с обитателями Анд, они стали предками позднейших правителей Инков. Они пересекали Тихий океан, часто делая остановки и высаживаясь на многочисленных островах, которые встречались на пути. Тогда острова Полинезии были и обширнее, и многочисленнее, чем теперь, и эти моряки\hyp{}Андиты вместе с теми, кто последовал за ними, биологически влияли на те группы туземцев, которые они встречали по пути. В результате проникновения Андитов на этих, ныне затопленных островах, возникло множество процветающих центров цивилизации. Долгое время остров Пасхи был религиозным и административным центром одной из групп этих исчезнувших островов. Но из Андитов, плававших в давние времена по Тихому океану, лишь только эти сто тридцать два человека достигли американского материка.
\vs p078 5:8 \pc Кочевые завоевания Андитов продолжались до их окончательного рассеяния, с 8000 до 6000 года до н.э. По мере того, как они покидали Месопотамию, биологические резервы в их родных местах непрерывно истощались, за счет чего заметно усиливались окружающие народы. И каждой нации, с которой они встречались, передавались юмор, искусство, любовь к путешествиям, музыка и ремесла. Они были искусными специалистами по одомашниванию животных и прекрасными земледельцами. Их присутствие на какое\hyp{}то время обычно положительно влияло на религиозные верования и нравственные обычаи более древних рас. Так культура Месопотамии незаметно распространилась в Европе, Индии, Китае, северной Африке и на островах Тихого океана.
\usection{6. Последнее рассеяние Андитов}
\vs p078 6:1 Последние три волны исхода Андитов из Месопотамии пришлись на время между 8000 и 6000 годом до н.э. Эти три великих исхода культурных народов из Месопотамии произошли под давлением горных племен на востоке и из\hyp{}за набегов равнинных жителей на западе. Жители долины Евфрата и прилегающих областей потянулись в свой последний исход по нескольким направлениям:
\vs p078 6:2 Шестьдесят пять процентов обогнули Каспийское море и пришли в Европу; они покорили только что появившиеся белые расы --- потомков голубых людей и более ранних Андитов --- и слились с ними.
\vs p078 6:3 Десять процентов, включая большую группу священников\hyp{}Сифитов, двинулись на восток, через горные области Элама к Иранскому плоскогорью и Туркестану. Многие их потомки впоследствии были вытеснены в Индию вместе с их Арийскими собратьями, проживавшими на севере.
\vs p078 6:4 Десять процентов жителей Месопотамии, пойдя на север, затем повернули на восток и пришли в Синцзян, где смешались с желтыми Андическими обитателями тех мест. Большинство талантливых потомков этого расового союза позднее пришли в Китай и много сделали для скорейшего развития северных племен желтой расы.
\vs p078 6:5 Десять процентов этих спасающихся бегством Андитов прошли через Аравию и вступили в Египет.
\vs p078 6:6 \pc Пять процентов Андитов, носители чрезвычайно высокой культуры прибрежных областей в устьях Тигра и Евфрата, которые не допускали браков с соседними низшими племенами, отказались покинуть свои земли. Эта группа представляла то, что уцелело от многочисленных родов Нодитов и Адамитов, находящихся на самом высоком уровне развития.
\vs p078 6:7 \pc К 6000 году до Рождества Христова Андиты почти полностью покинули этот регион, хотя гораздо позже их потомкам, в значительной степени уже смешавшимся с окружавшими их сангическими расами и с Андонитами Молой Азии, пришлось там сражаться с северными и восточными захватчиками.
\vs p078 6:8 Эпоха культуры второго сада пришла в упадок вследствие все увеличивающегося проникновения окружающих низших племен. Цивилизация двинулась на запад, к Нилу и островам Средиземноморья, там она продолжала процветать и развиваться долгое время после того, как ее источник в Месопотамии уже иссяк. И этот ничем не сдерживаемый наплыв народа, находящегося на низшем уровне развития, подготовил почву для позднейшего захвата всей Месопотамии северными варварами, которые вытеснили остатки потомков талантливого народа. Даже в более поздние годы носители этих остатков культуры все еще выражали свое возмущение присутствием невежественных и неотесанных оккупантов.
\usection{7. Наводнения в Месопотамии}
\vs p078 7:1 Живущие у рек привыкли к тому, что в определенное время реки разливаются; эти периодические наводнения в течение их жизни происходили каждый год. Но развивающиеся на севере геологические процессы угрожали долине Месопотамии новыми бедами.
\vs p078 7:2 После затопления первого Эдема в течение тысячелетий горы, окружавшие восточное побережье Средиземного моря, и горы к северо\hyp{}западу и к северо\hyp{}востоку от Месопотамии продолжали подниматься. Подъем гористых областей значительно ускорился около 5000 года до н.э., что, наряду со значительно увеличившимися снегопадами в северных горах, каждую весну вызывало беспрецедентные наводнения в долине Евфрата. Эти весенние паводки становились все опаснее и опаснее, так что, в конце концов, жители речных областей были вынуждены бежать в восточные нагорья. В течение почти тысячи лет из\hyp{}за обширных наводнений многие города были практически покинуты жителями.
\vs p078 7:3 \pc Почти пять тысяч лет спустя, когда иудейские священники во время Вавилонского пленения пытались проследить родословную еврейского народа назад к Адаму, они столкнулись с серьезными трудностями при согласовании отдельных событий этой истории; и одному из них пришло в голову оставить попытки это сделать --- и пусть весь мир уйдет под воду вместе со своими грехами во время Ноева потопа и это даст прекрасную возможность проследить родословную Авраама прямо до одного из трех оставшихся в живых сыновей Ноя.
\vs p078 7:4 Предания о том, что было время, когда вода покрывала всю поверхность земли, являются повсеместными. У многих рас есть легенды, что когда\hyp{}то в древние времена произошел всемирный потоп. Библейская история о Ное, ковчеге и потопе --- выдумка иудейского духовенства во времена Вавилонского плена. С тех пор, как на Урантии появилась жизнь, на ней никогда не было всемирного потопа. Единственный случай, когда поверхность земли была полностью покрыта водой, имел место в археозойской эпохе, прежде чем из воды стала появляться суша.
\vs p078 7:5 Но сам Ной --- лицо реальное; это винодел из Арама, речного поселения вблизи Эреха. Из года в год он вел записи дат подъема уровня реки. Над ним многие насмехались, потому что он ходил вверх и вниз по реке, поучая, что все дома должны быть построены из дерева, наподобие лодок, и что, как только приближается сезон наводнений, домашних животных следует каждую ночь приводить в дом. Каждый год он ходил в соседние речные поселения и предупреждал людей, что через столько\hyp{}то дней наступит наводнение. Наконец, в какой\hyp{}то год в результате проливных дождей ежегодное наводнение было особенно сильным, так что внезапный резкий подъем воды смыл всю деревню; и только Ной и его домочадцы спаслись в своем доме\hyp{}лодке.
\vs p078 7:6 \pc Эти наводнения завершили разрушение цивилизации Андитов. Когда закончился этот период наводнений, второго сада больше не существовало. Только на юге, среди Шумеров и сохранились следы былой славы.
\vs p078 7:7 Останки этой, одной из древнейших, цивилизации могут быть найдены в соответствующих областях Месопотамии, а также к северо\hyp{}востоку и северо\hyp{}западу от нее. Но под водами Персидского залива все еще существуют более древние свидетельства дней Даламатии, а первый Эдем лежит под водами восточной оконечности Средиземного моря.
\usection{8. Шумеры --- последние Андиты}
\vs p078 8:1 Когда последний исход Андитов положил конец биологическому существованию Месопотамской цивилизации, на своей родине вблизи устья рек оставалось жить лишь немногочисленные представители этой высшей расы. Это были Шумеры, и к 6000 году до н.э. по происхождению их можно было отнести, скорее, к Андитам, хотя их культура по своему характеру была исключительно культурой Нодитов и они придерживались древних традиций Далматии. Тем не менее, Шумеры прибрежных областей были в Месопотамии последними Андитами. Однако народы Месопотамии к этому позднему времени уже окончательно смешались, о чем свидетельствуют черепа, найденные в могильниках этой эпохи.
\vs p078 8:2 Именно на период наводнений приходится расцвет Суз. Первый, нижний, город был затоплен, так что второй, верхний, город стал наследником первого, центром своеобразного ремесленного производства того времени. Когда впоследствии интенсивность наводнения уменьшилась, центром керамического производства стал Ур. Около семи тысяч лет назад Ур находился на берегу Персидского залива, с тех пор речные отложения образовали сушу, простирающуюся до ее современных границ. Эти поселения меньше страдали от наводнений благодаря тому, что лучше было организовано регулирование уровня, и тому, что русла рек в устье были расширены.
\vs p078 8:3 \pc Мирные хлеборобы долин Евфата и Тигра долгое время страдали от набегов варваров Туркестана и Иранского плоскогорья. Но теперь усиливающаяся засуха высокогорных пастбищ вынудила кочевников сговориться и совместно напасть на долину Евфрата. Этот захват был тем более важен, потому что окружающие пастухи и охотники владели большими табунами прирученных лошадей. А лошади были тем фактором, который давал им огромное военное преимущество над богатыми южными соседями. За короткое время они опустошили всю Месопотамию, что привело к последнему исходу культурных народов, которые распространились по всей Европе, западной Азии и северной Африке.
\vs p078 8:4 В шеренгах завоевателей Месопотамии были многие лучшие Андические роды смешанных северных народов Туркестана, включая некоторые семьи потомков Адама\hyp{}сына. Эти менее развитые, но более сильные северные племена быстро и охотно ассимилировали остаток цивилизации Месопотамии и вскоре развились в те смешанные народы, которые обитали в долине Евфрата к началу исторических времен. Они быстро восстановили в Месопотамии многие стороны ушедшей цивилизации, переняли ремесла племен, живущих у рек, и многое из культуры Шумеров. Они даже пытались построить третью Вавилонскую башню и назвали впоследствии свое государство Вавилон.
\vs p078 8:5 Когда эти варвары\hyp{}всадники из северо\hyp{}восточных регионов опустошили всю долину Евфрата, они не смогли покорить остатки Андитов, которые жили вблизи устья реки, впадающей в Персидский залив. Эти Шумеры все\hyp{}таки сумели защитить себя благодаря своим более высоким интеллектуальным способностям, лучшему вооружению и наличию разветвленной сети военных каналов, которая дополняла их ирригационную систему соединенных между собой водоемов. Они представляли собой сплоченный народ, потому что у них была единая религия. Таким образом, они смогли поддерживать свою расовую и национальную целостность долгое время после того, как их соседи на северо\hyp{}западе распались на изолированные города\hyp{}государства. Ни одна из этих групп городов не была способна одержать победу над объединенными Шумерами.
\vs p078 8:6 И захватчики с севера скоро научились доверять и высоко ценить этих миролюбивых Шумеров как способных учителей и руководителей. У всех народов к северу от Месопотамии и на всем пространстве от Египта на западе до Индии на востоке их очень уважали и охотно нанимали в качестве учителей искусства и ремесел, управляющих торговыми делами и гражданских правителей.
\vs p078 8:7 После распада древней конфедерации Шумеров более поздние города\hyp{}государства управлялись отступниками из числа потомков священников\hyp{}Сифитов. Эти священники стали называть себя царями только тогда, когда они завоевали соседние города. Цари более поздних городов не смогли вплоть до времен Саргона образовать мощную конфедерацию из\hyp{}за религиозных разногласий. Каждый город полагал, что его бог является высшим по отношению ко всем остальным богам, и поэтому они отказывались подчиниться единому для всех вождю.
\vs p078 8:8 Конец этому длительному периоду правления слабовольных священников отдельными городами был положен Саргоном, священником Киша, который провозгласил себя царем и положил начало завоеванию всей Месопотамии и прилегающих территорий. И на некоторое время прекратили существование города\hyp{}государства, управляемые священниками, и прошло время, когда каждый город, находящийся под властью священников, имел своего собственного бога и свои собственные религиозные обряды.
\vs p078 8:9 После распада конфедерации Киша долгое время между этими городами, расположенными в долине, шла непрекращающаяся война за верховную власть. И власть в разное время захватывали различные города: Шумер, Аккад, Киш, Эрех, Ур и Суз.
\vs p078 8:10 Около 2500 года до н.э. Шумеры потерпели тяжелые поражения от северных Суитов и Гуитов. Пал Лагаш, столица Шумеров, построенная на холмах, образованных наносами наводнений. Тридцать лет после падения Аккада держался Эрех. К моменту установления правления Хаммурапи Шумеров поглотили северные Семиты, и Андиты Месопотамии исчезли со страниц истории.
\vs p078 8:11 С 2500 до 2000 года до н.э. на пространстве от Атлантического до Тихого океана неистовствовали кочевники. Последним всплеском Каспийской группы Месопотамских потомков Андонической и Андической рас явились Нериты. То, что не удалось сделать варварам при уничтожении Месопотамии, довершили последующие климатические изменения.
\vs p078 8:12 \pc Такова история фиолетовой расы после дней Адама, такова судьба их родины, расположенной между Тигром и Евфратом. Их древняя цивилизация пришла, в конце концов, в упадок в результате того, что эти края покинули народы, находящиеся на более высоком уровне развития, а на их место пришли их менее развитые соседи. Но задолго до того, как варвары\hyp{}всадники завоевали долину, многое из культуры Сада проникло в Азию, Африку и в Европу, создав там предпосылки, которые впоследствии привели Урантию к цивилизации двадцатого века.
\vsetoff
\vs p078 8:13 [Представлено Архангелом Небадона.]
