\upaper{155}{Бегство через северную Галилею}
\author{Комиссия срединников}
\vs p155 0:1 Высадившись неподалеку от Хересы в это богатое событиями воскресенье, Иисус и его двадцать четыре ученика прошли немного на север и заночевали в прекрасном парке к югу от Вифсаиды\hyp{}Юлии. Это место привала было им хорошо знакомо, ибо они и раньше уже останавливались здесь. Перед тем, как идти спать, Иисус собрал своих последователей вокруг себя и обсудил с ними планы их предстоящего путешествия через Батанию и северную Галилею к финикийскому берегу.
\usection{1. Почему мятутся язычники?}
\vs p155 1:1 Иисус сказал: «Вы все должны помнить, как псалмопевец, говоря об этих временах, сказал: „Почему мятутся язычники, и народы замышляют тщетное? Восстают цари земные и правители народа совещаются вместе против Господа и помазанника его, говоря: «Расторгнем узы милосердия и свергнем с себя оковы любви“.
\vs p155 1:2 Сегодня вы видите, как исполнилось сие на глазах ваших. Но вы не увидите исполнения остальной части пророчества псалмопевца, ибо у него были ложные представления о Сыне Человеческом и его миссии на земле. Царство мое основано на любви, провозглашено в милосердии и установлено бескорыстным служением. Отец мой отнюдь не восседает на небесах, насмехаясь и глумясь над язычниками. В великом недовольстве своем он не гневен. Истинно обетование о том, что Сын будет иметь наследников среди так называемых язычников (в действительности же своих невежественных и непросвещенных братьев). Я приму этих неевреев в распростертые объятия милосердия и любви. Вся полная любви доброта будет явлена так называемым язычникам, несмотря на то, что, к сожалению, было записано, что торжествующий Сын „поразит их жезлом железным и сокрушит их, как сосуд горшечника“. Псалмопевец призвал вас: „Служите Господу со страхом“. Я же призываю вас вступить в высокие привилегии божественного сыновства, даруемого верой; он приказывает вам радоваться с трепетом, а я же повелеваю вам радоваться с уверенностью. Он говорит: „Почтите Сына, чтобы он не прогневался и чтобы вам не погибнуть, когда возгорится гнев его“. Вы же, живущие со мной, хорошо знаете, что гнев и ярость не являются частью установления царства небесного в сердцах людей. Однако псалмопевец прозрел истинный свет, когда в завершении своего увещевания сказал: „Блаженны уповающие на Сына сего“».
\vs p155 1:3 Продолжая учить своих учеников, Иисус сказал: «Язычники озлились на нас не без причины. Поскольку кругозор их невелик и узок, они способны с энтузиазмом объединить свои силы. Цель их близка и более или менее очевидна; вот почему они стремятся к ней дерзко и решительно. Вы же, исповедующие вступление в царство небесное, в общем, учите слишком нерешительно и невнятно. Язычники стремятся прямо к цели; вы же грешите излишним хроническим томлением. Если желаете войти в царство, то почему не берете его духовным штурмом, как берут язычники город, который осадили? Вы вряд ли достойны царства, когда служение ваше в основном сводится к сожалению о прошлом, плачу о настоящем и тщетной надежде на будущее. Почему мятутся язычники? Потому что не знают истины. Почему от тщетного томления изнываете вы? Потому что истине \bibemph{не подчиняетесь.} Итак, оставьте ваше бесполезное томление и смело отправляйтесь делать то, что связано с установлением царства.
\vs p155 1:4 Во всем, что делаете, не будьте односторонни и узки. Фарисеи, пытающиеся уничтожить нас, действительно думают, что исполняют Божье служение. Традиция настолько сузила их мировосприятие, что они ослепли от предрассудков и ожесточились от страха. Посмотрите на греков; у них --- наука без религии, у евреев же --- религия без науки. Когда люди впадают в подобное заблуждение, принимая узкие и разрозненные обрывки за истину, их единственная надежда на спасение --- стать сообразным истине --- обратиться.
\vs p155 1:5 Особенно подчеркну эту вечную истину: если вы, будучи сообразным истине, научитесь своей жизнью являть пример прекрасной полноты праведности, тогда и ваши собратья будут следовать вам, чтобы и им достичь того, что удалось обрести вам. Мера притяжения ищущих истины к вам сообразна мере обретения вами истины, вашей праведности. Чем больше вы должны идти с вашим посланием к людям, тем больше это показывает вашу неспособность жить праведной жизнью, жизнью соответствующей истине».
\vs p155 1:6 И еще многому учил Учитель своих апостолов и евангелистов перед тем, как они, пожелав ему спокойной ночи, пошли спать.
\usection{2. Евангелисты в Хоразине}
\vs p155 2:1 В понедельник утром 23 мая Иисус дал указание Петру идти с двенадцатью евангелистами в Хоразин, а сам с одиннадцатью апостолами отправился в Кесарию Филиппову, пройдя вдоль Иордана к дороге, ведущей от Дамаска к Капернауму, оттуда --- на северо\hyp{}восток до пересечения с дорогой на Кесарию Филиппову, и далее в город, где они и провели, наставляя, две недели. В город они прибыли во вторник после полудня 24 мая.
\vs p155 2:2 Петр и евангелисты прожили в Хоразине две недели, проповедуя евангелие царства небольшой, но искренней группе верующих. Однако им не удалось привлечь много новых последователей. Ни один город в Галилее не дал царству так мало душ, как Хоразин. В соответствии с наставлениями Петра, двенадцать евангелистов мало говорили об исцелении --- о телесном --- зато с великим рвением учили и проповедовали духовные истины царства. Две проведенные в Хоразине недели стали для двенадцати евангелистов настоящим крещением несчастьями, ибо это был самый трудный и непродуктивный период в их служении. Не имея возможности испытать чувство удовлетворения, которое приносит обращение душ к царству, все они еще серьезнее и честнее занялись изучением своих собственных душ и их продвижения по духовным путям новой жизни.
\vs p155 2:3 Когда стало ясно, что искать пути в царство больше никто не хочет, Петр собрал своих сподвижников, и во вторник 7 июня они отправились в Кесарию Филиппову, чтобы присоединиться к Иисусу и апостолам. В среду около полудня они пришли в город и провели весь вечер, вспоминая то, что пережили среди неверующих жителей Хоразина. В этот вечер в своих рассуждениях Иисус еще раз рассказал им притчу о сеятеле и долго наставлял их о том, каково значение кажущихся неудач в жизненных делах.
\usection{3. В Кесарии Филипповой}
\vs p155 3:1 Хотя во время двухнедельного пребывания неподалеку от Кесарии Филипповой Иисус не занимался публичной деятельностью, апостолы провели в городе множество небольших вечерних собраний, и многие из верующих приходили в лагерь поговорить с Учителем. Однако в результате к группе верующих примкнуло очень немного новых последователей. Иисус беседовал с апостолами каждый день, и они стали намного яснее понимать, что теперь наступает новый этап дела провозглашения царства небесного. Они начали сознавать, что «царствие небесное --- не пища и не питие, но осознание духовной радости принятия божественного сыновства».
\vs p155 3:2 Пребывание в Кесарии Филипповой было для одиннадцати апостолов настоящим испытанием; им было трудно пережить эти две недели. Они были сильно подавлены, и им очень недоставало пламенного Петра, который неизменно вселял в них энтузиазм. Для них в эти времена верить в Иисуса и идти за ним было поистине великим испытанием. Хотя за две недели им удалось обратить лишь немногих, эти ежедневные беседы с Учителем принесли им огромную пользу.
\vs p155 3:3 Апостолы узнали, что евреи духовно закоснели и находятся на грани смерти, потому что облекли истину в форму вероучения; что, когда истина вместо того, чтобы служить путеводной звездой духовного водительства и прогресса, направлена на выражение самодовольной исключительности, тогда подобные учения утрачивают свою творческую и животворящую силу и в конечном счете становятся просто ортодоксальными и закоснелыми.
\vs p155 3:4 Они все больше учились у Иисуса подходить к человеческой личности с позиции ее возможностей во времени и в вечности. Они узнали, что лучший способ побудить многие души любить незримого Бога --- это научить их прежде всего любить тех собратьев, которые их окружают. И именно в этом был новый смысл изречения Учителя о бескорыстном служении собратьям: «Так как вы сделали это одному из братьев моих меньших, то сделали мне».
\vs p155 3:5 Одно из важнейших наставлений, сделанных в период этого пребывания в Кесарии, касалось происхождения религиозных традиций и, в частности, серьезной опасности, которую несет в себе допущение того, что чувство святости можно связывать с вещами несвященными, с обыкновенными идеями или повседневными событиями. Одна такая беседа позволила им усвоить учение о том, что истинная религия есть искренняя верность человека своим высочайшим и самым истинным убеждениям.
\vs p155 3:6 Иисус предостерег верующих о том, что если бы их религиозные стремления были только материальными, то углубляющиеся знания законов природы постепенно вытеснили бы из их представлений воображаемое сверхъестественное происхождение вещей и в конечном итоге лишило бы их веры в Бога. Но если бы их религия была духовной, то прогресс материальной науки никогда не смог бы поколебать их веру в вечные реалии и божественные ценности.
\vs p155 3:7 Они узнали, что когда религия в основе своей полностью духовна, это делает жизнь более значимой, исполняя ее высокими целями, облагораживая непревзойденными ценностями, вдохновляя высшими мотивами, постоянно утешая человеческую душу величественной и укрепляющей надеждой. Истинная религия служит ослаблению напряженности бытия; она дает веру и мужество для повседневной жизни и бескорыстного служения. Вера питает духовной энергией и способствует праведной плодотворности.
\vs p155 3:8 Иисус неоднократно учил своих апостолов, что ни одна цивилизация не сможет долго просуществовать после того, как ее религия утратила лучшие свои черты. Он никогда не уставал указывать двенадцати апостолам на великую опасность подмены религиозного опыта религиозными символами и ритуалами. Вся его земная жизнь была посвящена тому, чтобы оживотворить застывшие формы религии и превратить их в светозарную вольность просвещенного сыновства.
\usection{4. По пути в Финикию}
\vs p155 4:1 Утром в четверг 9 июня, получив известие о распространении царства, принесенное из Вифсаиды вестниками Давида, группа из двадцати пяти учителей истины покинула Кесарию Филиппову и начала свой путь к финикийскому берегу. Через Лузу они дошли до тропы, ведущей от Магдалы к Ливанской горе, обогнули болотистую местность, а оттуда направились к пересечению с дорогой, что вела к Сидону, и прибыли туда в пятницу после полудня.
\vs p155 4:2 Близ Лузы, где они остановились, чтобы пообедать в тени нависшего края скалы, Иисус произнес одну из самых замечательных речей, какую когда\hyp{}либо слышали его апостолы за все годы общения с ним. Не успели они сесть и преломить хлеб, как Симон Петр спросил Иисуса: «Учитель, если Отец Небесный ведает обо всем, а дух его --- наша поддержка в установлении царства небесного на земле, то почему мы бежим от угроз врагов наших? Почему отказываемся лицом к лицу встретиться с врагами истины?» Однако прежде, чем Иисус ответил Петру, в разговор вмешался Фома, сказав: «Учитель, я действительно хотел бы знать, что же неправильного в религии наших врагов в Иерусалиме. В чем по сути различие между их религией и нашей? Почему наши веры так отличаются друг от друга, тогда как все мы говорим о том, что служим одному и тому же Богу?» Когда Фома кончил говорить, Иисус сказал: «Хоть мне и не хочется оставлять без внимания вопрос Петра, ибо я полностью сознаю, как легко неверно понять причины, по которым как раз в это время я избегаю открытого столкновения с правителями евреев, все же гораздо полезнее для всех вас будет, если я предпочту ответить на вопрос Фомы. Что я и сделаю, как только вы отобедаете».
\usection{5. Беседа об истинной религии}
\vs p155 5:1 В этой памятной беседе о религии, кратко изложенной ниже современным языком, нашли отражение следующие истины:
\vs p155 5:2 \pc Хотя религии мира имеют двоякое происхождение --- естественное и данное в откровении --- в любое время и среди любого народа можно обнаружить три отчетливых типа религиозного поклонения. Эти три проявления религиозного побуждения суть таковы:
\vs p155 5:3 \ublistelem{1.}\bibnobreakspace \bibemph{Примитивная религия.} В какой\hyp{}то мере естественное и инстинктивное побуждение бояться непостижимых энергий и поклоняться высшим силам; главным образом, это религия материальной природы, религия страха.
\vs p155 5:4 \ublistelem{2.}\bibnobreakspace \bibemph{Религия цивилизации.} Развивающиеся религиозные представления и обычаи цивилизованных рас --- религия ума --- интеллектуальная теология власти установившейся религиозной традиции.
\vs p155 5:5 \ublistelem{3.}\bibnobreakspace \bibemph{Истинная религия --- религия откровения.} Откровение сверхъестественных ценностей, частичное прозрение в вечные реальности, проблески видения доброты и красоты бесконечной сущности Отца Небесного --- религия духа, каким он явлен в человеческом опыте.
\vs p155 5:6 \pc Учитель не стал умалять значение религии физических ощущений и суеверных страхов первобытного человека, хотя и выразил сожаление о том, что в религиозных формах наиболее развитых рас человечества сохраняется так много от этой примитивной формы поклонения. Иисус показал, что великая разница между религией ума и религией духа в том, что в то время как первая поддерживается властью духовенства, последняя целиком и полностью основана на человеческом опыте.
\vs p155 5:7 \pc И дальше продолжая наставления, Учитель разъяснил эти истины.
\vs p155 5:8 \pc Пока расы не станут более разумными и цивилизованными, сохранятся многие из тех наивных и суеверных ритуалов, что столь характерны для эволюционирующих религиозных обычаев примитивных и отсталых народов. Пока род человеческий не поднимется на уровень более высокого и общего признания реалий духовного опыта, большинство мужчин и женщин лично будут продолжать отдавать предпочтение тем религиям власти, которые требуют лишь интеллектуального согласия и этим отличаются от религии духа, влекущей за собой активное участие разума и души в переживаниях веры, связанных с попытками справиться с суровыми реалиями совершенствующегося человеческого опыта.
\vs p155 5:9 Принятие традиционных религий власти предлагает легкий выход для стремления человека искать удовлетворение желаний своей духовной природы. Устоявшаяся, выкристаллизовавшаяся и утвердившаяся религия власти предоставляет готовое убежище, где может укрыться смущенная и сбитая с толку душа человека, терзаемая страхом и мучимая неопределенностью. Такая религия в уплату за даруемые ей умиротворение и уверенность требует от своих приверженцев лишь пассивного и чисто интеллектуального согласия.
\vs p155 5:10 Еще долгое время на земле будут жить робкие, исполненные страха и нерешительные люди, которые предпочтут именно так получать религиозное утешение, хотя, связав таким образом свою судьбу с религиями власти, они рискуют независимостью личности, жертвуют чувством собственного достоинства и отказываются от права участвовать в этом самом волнующем и вдохновляющем из всех возможных человеческих переживаний: от личного искания истины, от радости одоления трудностей интеллектуального открытия; от решимости постигать реалии личного религиозного опыта, от верховного удовлетворения тем личным свершением; той поистине сознательной победой веры над интеллектуальными сомнениями, которая честно одержана в верховном достижении всего человеческого бытия --- в поисках человеком Бога для самого себя, собственными силами и отыскании его.
\vs p155 5:11 Религия духа означает усилие, борьбу, конфликт, веру, решимость, любовь, верность и совершенствование. Религия разума --- теология власти --- требует от своих формальных приверженцев малую толику этих проявлений или не требует их совсем. Традиция --- это надежное убежище и легкий путь для тех исполненных страхом и нерешительных душ, которые инстинктивно избегают духовных борений и умственных сомнений, связанных со столь смелым и рискованным плаванием веры по бурным волнам неизведанной истины в поисках дальних берегов духовных реалий, какие могут быть открыты развивающимся человеческим разумом и испытаны совершенствующейся человеческой душой.
\vs p155 5:12 \pc И Иисус продолжал: «В Иерусалиме религиозные лидеры сформировали из различных доктрин своих традиционных учителей и пророков прошлого официальную систему интеллектуальных верований, религию власти. Все аналогичные религии в основном апеллируют к разуму. И мы ныне готовимся вступить в смертельную схватку с подобной религией, поскольку вскоре начнем смело возвещать новую религию --- религию, которая в сегодняшнем понимании этого слова религией не является, религию, которая главным образом взывает к божественному духу моего Отца, пребывающему в разуме человека; религию, чья власть будет происходить из плодов, которые столь несомненно появятся в личном опыте всех, кто действительно и искренне уверует в истины сего более высокого духовного общения».
\vs p155 5:13 Обращаясь к каждому из двадцати четырех учеников и назвав каждого из них по имени, Иисус сказал: «И ныне который из вас предпочтет встать на сей легкий путь подчинения установленной и закоснелой религии, которую отстаивают фарисеи в Иерусалиме, и откажется терпеть трудности и преследования, сопровождающие миссию провозглашения людям лучшего пути спасения, при этом довольствоваться тем, что лично для себя открывает прекрасные реалии живого и личного опыта в вечных истинах и наивысшем величии царства небесного? Разве вы полны страха, слабы и ищите легких путей? Разве боитесь вы доверить свое будущее Богу истины, которого вы сыны? Разве не доверяете Отцу, которого вы дети? Неужели вернетесь на легкий путь определенности и интеллектуального застоя, присущих религии власти традиции, или же приготовитесь идти со мною вперед в неизвестное и тревожное будущее, провозглашая новые истины религии духа, царства небесного в сердцах людей?»
\vs p155 5:14 Все двадцать четыре слушателя Иисуса поднялись на ноги в знак того, что они единодушно и совершенно искренне откликнулись на этот один из тех редких эмоциональных призывов, с которыми Иисус когда\hyp{}либо обращался к ним, однако Иисус поднял руку и остановил их, сказав: «А теперь расходитесь, и пусть каждый оставшись наедине с Богом, поищет на мой вопрос беспристрастный ответ, а обретя такой истинный и искренний настрой души, свободно и смело скажет этот ответ Отцу моему и Отцу вашему, чья бесконечная жизнь, исполненная любви, и есть истинный дух религии, которую мы возвещаем».
\vs p155 5:15 Евангелисты и апостолы на короткое время разошлись. Их дух был приподнят, умы --- вдохновлены, а чувства взволнованы словами Иисуса. Однако когда Андрей созвал их, Учитель лишь сказал: «Продолжим наш путь. Мы пойдем в Финикию, где и пробудем какое\hyp{}то время; вы же должны молить Отца преобразить ваши эмоции ума и тела в более высокое чувство верности ума и приносящий большее удовлетворение опыт духа».
\vs p155 5:16 По дороге двадцать четыре молчали, однако вскоре разговорились друг с другом и к трем часам дня не могли идти дальше; они сделали привал, и Петр, подойдя к Иисусу, сказал: «Учитель, ты сказал нам слова жизни и истины. Мы хотим тебя послушать еще и просим тебя продолжить с нами разговор на эту тему».
\usection{6. Вторая беседа о религии}
\vs p155 6:1 Итак, пока они отдыхали в тени холма, Иисус продолжал учить их о религии духа и по существу сказал:
\vs p155 6:2 \pc Вы вышли из среды тех ваших собратьев, которые пожелали довольствоваться религией ума, которые жаждут уверенности и предпочитают подчинение. Вы же решили сменить ваше чувство определенности, которое дает вам религия власти, на уверенность духа, которую дает вам дерзновенная и совершенствующаяся вера. Вы решились протестовать против изнурительного рабства казенной религии и отказаться подчиняться власти традиций, запечатленной в записях, на которые ныне смотрят как на слово Божье. Наш Отец действительно говорил через Моисея, Илию, Исайю, Амоса и Осию, однако он не переставал давать миру слова истины и тогда, когда эти пророки древности закончили свои изречения. Отец мой не взирает на расы или поколения, и не допустит, чтобы слово истины было дано одной эпохе и утаено от другой. Не проявляйте безрассудство, называя божественным чисто человеческое, и умейте распознавать слова истины, пришедшие не через традиционных оракулов мнимого вдохновения.
\vs p155 6:3 \pc Я призвал вас родиться заново, родиться от духа. Я призвал вас из тьмы власти и летаргии традиции на необыкновенный свет осознания того, что можно сделать для себя величайшее из всех возможных для человеческой души открытие --- обрести высший опыт отыскания Бога для себя, в себе и своими силами и реально достичь всего этого в вашем собственном личном опыте. Итак, вы можете перейти из смерти в жизнь, от власти традиции к опыту познания Бога; таким образом вы перейдете из тьмы во свет, от унаследованной вами расовой веры к вере личной, достигаемой путем собственного опыта; благодаря этому вы перейдете от теологии разума, переданной вам вашими предками, к истинной религии духа, которая произрастет в душах ваших как вечный дар.
\vs p155 6:4 Из простой интеллектуальной веры во власть традиции ваша религия превратится в подлинный опыт той живой веры, что способна осознать реальность Бога, и всего, что имеет отношение к божественному духу Отца. Религия ума безнадежно связывает вас с прошлым; религия же духа заключается в непрерывном откровении и постоянно влечет вас к более высоким и более священным достижениям в духовных идеалах и вечных реалиях.
\vs p155 6:5 Хотя религия власти в текущий момент и может сообщить вам чувство твердой уверенности, за это преходящее удовлетворение придется заплатить высокую цену утраты вами духовной свободы и религиозной независимости. Отец мой в качестве платы за вход в царство небесное не требует от вас принуждать себя соглашаться с верой в то, что духовно неприемлемо, нечисто и неистинно. От вас не требуется, чтобы вы отступились от ваших представлений о милосердии, справедливости и истины ради подчинения изжившей себя системе религиозных форм и обрядов. Религия духа предоставляет вам вечную свободу следовать истине, куда бы водительство духа ни вело вас. И кто знает --- возможно, этот дух имеет сообщить сему поколению нечто такое, что другие поколения слушать отказывались.
\vs p155 6:6 Позор тем религиозным лжеучителям, что тянут алчущие души назад во тьму и далекое прошлое и там бросают их! Так что эти несчастные люди обречены страшиться всякого нового открытия и приходить в смятение от каждого нового откровения истины. Пророк, сказавший: «В совершенном мире пребывает тот, чей разум обращен к Богу», был не просто интеллектуально верующим в установленную теологию. Сей познавший истину человек открыл Бога, а не просто говорил о Боге.
\vs p155 6:7 Я призываю вас отказаться от обычая постоянно цитировать пророков древности и восхвалять героев Израиля, а вместо этого стремиться стать истинными пророками Всевышнего и духовными героями грядущего царства. Славить познавших Бога лидеров прошлого действительно может быть достойным занятием, но почему, поступая так, вы должны жертвовать высшим опытом человеческого бытия: опытом отыскания Бога для самих себя и познания его в ваших собственных душах?
\vs p155 6:8 У каждой человеческой расы --- свой собственный мысленный взгляд на человеческое существование; следовательно, религия разума должна всегда совпадать с этими различными расовыми точками зрения. Религии власти никогда не придут к единству. Человеческое единство и братство смертных может быть достигнуто лишь благодаря сверхдару религии духа и через него. Расовые умы могут отличаться, однако в человечестве в целом пребывает один и тот же божественный и вечный дух. Надежда достижения человеческого братства может сбыться лишь тогда, когда объединяющая и возвышающая религия духа --- религия личного духовного опыта --- оплодотворит своими идеями и возьмет верх над интеллектуальными религиями власти.
\vs p155 6:9 Религии власти могут лишь разделять, настраивая людей различных взглядов и вероисповедания друг против друга, религия же духа будет все больше сближать людей, делая их сознательно сочувствующими друг другу. Религии власти требуют от людей единообразия в вере, однако достигнуть его при современном состоянии мира невозможно. Религия же духа требует лишь единства опыта --- единообразия предназначения --- полностью допуская разнообразие верований. Религия духа требует только единообразия осознания истины, а не единообразия взглядов и мировоззрения. Религия духа не требует единообразия интеллектуальных воззрений, а лишь единства духовного чувства. Религии власти выкристаллизовываются в безжизненные символы веры; религия же духа переходит в возрастающую радость и свободу возвышенных деяний полного любви служения и милосердной помощи.
\vs p155 6:10 Однако следите за тем, чтобы никто из вас не смотрел с презрением на детей Авраамовых, потому что для них настали сии злые дни бесплодия, к которому ведет традиция. Наши предки отдавали себя настойчивым и страстным исканиям Бога и познали его так, как не познавала ни одна другая раса человечества со времен Адама, кому было известно много из этого, ибо он был Сыном Бога. Отец мой не забыл отметить долгую и неутомимую борьбу Израиля от дней Моисея --- найти Бога и узнать Бога. На протяжении изнурительной жизни многих поколений евреи не прекращали трудиться в поте лица своего, стенать и мучиться, терпеть страдания и переносить печали непонятого и презираемого народа --- и все это ради того, чтобы хоть немного приблизиться к открытию истины о Боге. И, несмотря на все неудачи и ошибки Израиля, наши отцы от Моисея до времен Амоса и Осии неуклонно всему миру открывали все более ясную и истинную картину вечного Бога. Именно так был приготовлен путь для еще более великого откровения об Отце, которое вы призваны разделить.
\vs p155 6:11 Никогда не забывайте: существует лишь одно переживание, приносящее больше удовлетворения и радости, нежели попытки открыть волю Бога живого, --- это высший опыт честного стремления исполнить эту божественную волю. Запомните и то, что волю Бога можно исполнять на земле в любом роде занятий. Одни призвания не святы, другие --- мирские. В жизнях тех, кто ведом духом, свято все; то есть подчинено истине, возвышено любовью, исполнено милосердия и ограничено справедливостью. Дух, который Отец мой и я посылаем в мир, --- не только Дух Истины, но и дух идеалистичной красоты.
\vs p155 6:12 Перестаньте искать слово Бога только на страницах древних записей теологической власти. Рожденные от духа Бога впредь будут узнавать слово Бога, откуда бы оно ни исходило. Божественная истина не должна обесцениваться потому, что путь, по которому она была дарована, кажется человеческим. Многие из братьев ваших обладают умом, воспринимающим теорию о Боге, но не способны духовно осознать присутствие Бога. Именно по этой причине я так часто учил вас тому, что царства небесного проще всего достичь путем принятия духовной позиции искреннего дитя. Но не к умственной незрелости ребенка я призываю вас, а к \bibemph{духовной простоте} такого легковерного и доверчивого чада. И не столь важно то, что вы должны знать о факте бытия Бога, сколь то, что непрерывно должна возрастать ваша способность \bibemph{ощущать присутствие Бога.}
\vs p155 6:13 Однажды найдя Бога в своей душе, вы вскоре начнете открывать его в душах других людей, а постепенно --- и во всех творениях и созданиях могучей вселенной. Однако есть ли надежда, что Отец явится как Бог верховной верности и божественных идеалов в душах людей, которые почти или совсем не уделяют времени вдумчивому размышлению над подобными вечными реалиями? Хотя разум и не есть средоточие духовной природы, он действительно служит вратами к ней.
\vs p155 6:14 Однако не совершите ошибку, пытаясь доказать другим людям, что вы нашли Бога; вы не способны представить подобного интеллектуального довода, хотя существуют два убедительных и надежных способа показать, что вы знаете Бога, и они таковы:
\vs p155 6:15 \ublistelem{1.}\bibnobreakspace Плоды духа Бога, явленные в вашей повседневной жизни.
\vs p155 6:16 \ublistelem{2.}\bibnobreakspace Тот факт, что весь ваш жизненный путь положительно доказывает, что во имя продолжения существования в посмертии, надеясь отыскать Бога вечности, чье присутствие вы предвкусили, живя во времени, вы безоглядно рискуете всем, что вы есть, и всем, что есть у вас.
\vs p155 6:17 \pc Помните, Отец мой всегда отвечает на малейший проблеск веры. Он замечает физические и суеверные чувства примитивного человека. А, что касается тех честных, но исполненных страха душ, чья вера настолько слаба, что ее едва хватает на то, чтобы лишь разумом подчиниться пассивной позиции соглашательства с религиями власти, то Отец всегда готов восславить и поощрить даже все подобные вялые попытки приблизиться к нему. Однако от вас, призванных из тьмы во свет, ожидают веры всем сердцем; и вера ваша должна возобладать над телом, разумом и духом.
\vs p155 6:18 Вы --- мои апостолы, и религия для вас не должна стать теологическим убежищем, куда вы можете спрятаться, опасаясь столкновения с суровыми реалиями духовного совершенствования и идеалистичного пути, но религия ваша должна стать фактом реального опыта, свидетельствующим о том, что Бог нашел вас, вас идеализировал, возвысил и одухотворил, и что вы примкнули к вечному делу отыскания Бога, который, таким образом, вас нашел и усыновил.
\vs p155 6:19 \pc И когда Иисус закончил говорить, он подозвал к себе Андрея и, указав на запад в сторону Финикии, сказал: «Продолжим наш путь».
