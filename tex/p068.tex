\upaper{68}{Рассвет цивилизации}
\author{Мелхиседек}
\vs p068 0:1 Это начало повествования об очень длительной успешной борьбе, которую вел род человеческий с времен, когда он был только немногим лучше животного состояния, и далее до времен, когда среди высших рас человечества зародилась подлинная, хотя и несовершенная цивилизация.
\vs p068 0:2 Цивилизация --- это расовое обретение; а не биологическое свойство, поэтому необходимо всех детей воспитывать в культурной среде, а каждому последующему поколению молодежи снова давать образование. Высшие проявления цивилизации --- научные, философские и религиозные --- не передаются от одного поколения к другому прямым наследованием. Эти культурные достижения сохраняются только просвещенным сбережением общественного наследия.
\vs p068 0:3 Общественная реализация плана кооперации была начата учителями Даламатии; в течение трехсот тысяч лет человечество воспитывалось на идее совместной деятельности. Больше всего пользы из этих ранних общественных учений извлек голубой человек; красный человек --- в какой\hyp{}то степени, а черный человек --- меньше всех. А потому в более поздние времена на Урантии желтая раса и белая раса обладали лучшим социальным развитием.
\usection{1. Оборонительная социализация}
\vs p068 1:1 Когда люди тесно общаются, они часто учатся любить друг друга, но примитивный человек не был от природы преисполнен духом братской любви и стремлением к общественным связям со своими сородичами. Скорее уж древние расы познавали, что «в единении сила», из горького опыта, и отсутствие природного чувства братства и сейчас препятствует братству людей на Урантии.
\vs p068 1:2 Объединение рано стало ценой выживания. Одинокий человек был беспомощен, если не нес племенного знака, подтверждающего его принадлежность к группе, которая обязательно отомстит за любой нанесенный ему вред. Даже во времена Каина путешествовать по чужим местам в одиночку, без знака группового сообщества, было смертельно опасно. Цивилизация застраховала человека от насильственной смерти, однако страховые премии выплачиваются только при соблюдении многочисленных законов общества.
\vs p068 1:3 Таким образом, примитивное общество было создано вследствие всеобщей потребности в повышении безопасности сообщества. И человеческое общество развивалось продолжительными циклами, как результат преодоления страха изоляции путем вынужденной кооперации.
\vs p068 1:4 \pc Примитивные человеческие существа рано поняли, что в группе они могут стать чем\hyp{}то намного более мощным и сильным, чем просто сумма усилий ее отдельных членов. Сто людей, объединившись и работая в унисон, могут передвинуть огромный камень; два десятка хорошо обученных стражей порядка могут обуздать разъяренную толпу. И поэтому общество было рождено не как простое объединение индивидуумов, но, скорее, как результат \bibemph{организации} разумных, вступающих в кооперацию членов. Но сотрудничество не является природным свойством человека; он учится сотрудничать сначала через страх, а позднее потому, что открывает, что сотрудничество прекрасно помогает встречать и преодолевать трудные времена и защищаться от предполагаемых опасностей вечности.
\vs p068 1:5 Таким образом, люди, которые рано организовались в примитивное сообщество, стали более успешно воздействовать на природу, равно как и защищаться от себе подобных; они обладали большими возможностями выживания; поэтому на Урантии цивилизация постоянно развивалась, несмотря на многие неудачи. И именно благодаря повышению уровня выживаемости в сообществе многие грубые ошибки человека не смогли до сих пор остановить или разрушить человеческую цивилизацию.
\vs p068 1:6 \pc То, что современное культурное общество является относительно поздним феноменом, хорошо подтверждается сохранившимися до наших дней примитивными общественными отношениями, характерными для австралийских аборигенов, бушменов и пигмеев в Африке. Среди этих отсталых народов до сих пор можно встретить враждебность, свойственную древним племенам, личное подозрение и другие антисоциальные признаки, которые были характерны для всех примитивных рас. Эти несчастные асоциальные люди, сохранившиеся с древних времен, красноречиво свидетельствуют о том, что естественный для человека индивидуализм не может успешно соперничать с более могущественными и мощными организациями и союзами социального развития. Эти отсталые и подозрительные антисоциальные расы, на диалектах которых говорят только в радиусе сорока или пятидесяти миль, показывают, в каком бы мире вы могли жить сегодня, если бы не просветительская деятельность облеченного в плоть штата Планетарного Принца и позднее, если бы не труды адамической группы реализаторов расового подъема.
\vs p068 1:7 Современная фраза «назад к природе» --- это заблуждение невежества, вера в реальность прошлого вымышленного «золотого века». Единственная основа легенды о золотом веке --- это исторический факт Даламатии и Эдема. Но эти более совершенные общества были далеко не такими, как это представляется в утопических мечтаниях.
\usection{2. Движущие силы общественного прогресса}
\vs p068 2:1 Цивилизованное общество --- результат ранних усилий человека преодолеть его нелюбовь к \bibemph{одиночеству.} Но это не обязательно свидетельствует об обоюдной симпатии, и неспокойное состояние некоторых современных примитивных групп хорошо показывает, через что прошли древние племена. Но хотя отдельные составляющие цивилизации и могут вступать в конфликт и сражаться друг с другом, хотя цивилизация и сама может казаться конгломератом противоречивых стремлений и сражений, она все\hyp{}таки служит доказательством подлинных деяний, а не смертельной монотонности застоя.
\vs p068 2:2 Хотя уровень интеллекта сильно способствовал культурному развитию, общество, по существу, создано для снижения элемента риска в образе жизни личности, и скорость развития общества определяется сокращением количества страданий и увеличением доли удовольствия в жизни. Так все общество медленно движется к своему предназначению --- вымиранию или выживанию --- в зависимости от того, является ли целью общества самоподдержание или самоудовлетворение. Самоподдержание создает общество, тогда как чрезмерное самоудовлетворение разрушает цивилизацию.
\vs p068 2:3 Общество заботится о самосохранении, самоподдержании и самоудовлетворении, но человеческая самореализация достойна стать немедленной целью многих культурных групп.
\vs p068 2:4 Стадный инстинкт нормального человека едва ли достаточен для развития такой социальной организации, которая сейчас существует на Урантии. Хотя это врожденное стадное чувство лежит в основе человеческого общества, во многом общительность человека --- дело наживное. Раньше всего сближению человеческих существ способствовали два исключительно важных фактора --- голод и сексуальная любовь; эти инстинктивные побуждения --- общие для человека и животного. Другими двумя эмоциями, которые соединяли человеческие существа и \bibemph{удерживали} их вместе, были тщеславие и страх, особенно страх призраков.
\vs p068 2:5 \pc История --- ни что иное, как свидетельство вековой борьбы человека за пищу. \bibemph{Примитивный человек думал только тогда, когда был голоден;} запасание пищи стало его первым самоотречением и самодисциплиной. С развитием общества голод перестал быть единственным стимулом к взаимовыгодному объединению. Другие многочисленные сильные желания, осознание различных потребностей --- все это вело к более тесному объединению человечества. Но сегодняшнее общество перегружено чрезмерным ростом мнимых потребностей человека. Западная цивилизация двадцатого века уже стонет под чудовищным гнетом роскоши и неумеренно возросшими человеческими вожделениями и желаниями. Современное общество переживает одну из наиболее опасных фаз обширных взаимных связей и сильно усложненной взаимной зависимости.
\vs p068 2:6 Голод, тщеславие и боязнь призраков оказывали непрерывное давление на общество, но сексуальное удовлетворение было мимолетно и нерегулярно. Сексуальное желание само по себе не побуждало примитивных мужчин и женщин принять на себя тяжелые обязанности по поддержанию дома. Ранний дом держался на сексуальной неугомонности мужчины, который был лишен возможности часто ее удовлетворять, и на преданной материнской любви женщины. Такое чувство отчасти свойственно самкам всех высших животных. Рождение беспомощного ребенка определило первое разделение мужской и женской деятельности; женщина должна была поддерживать постоянное место проживания, где она могла обрабатывать почву. И с самых ранних времен место, где находилась женщина, всегда считалось домом.
\vs p068 2:7 Женщина, таким образом, рано стала необходимой для развивающегося общества, не столько из\hyp{}за мимолетной сексуальной страсти, сколько вследствие \bibemph{потребности в пище;} она была незаменимым партнером в самоподдержании. Она обеспечивала пищу, была вьючным животным и спутником, который вынесет плохое обращение без яростного негодования, в дополнение ко всем этим желательным качествам она была постоянно присутствующим способом сексуального удовлетворения.
\vs p068 2:8 Почти все основополагающие ценности цивилизации уходят корнями в семью. Семья была первой успешной мирной группой, мужчиной и женщиной, учившимися усмирять свой антагонизм и в то же время обучающими своих детей стремлению к миру.
\vs p068 2:9 Брак в эволюции --- это гарантия выживания расы, а не просто реализация личного счастья; самоподдержание и самосохранение являются истинными целями домашнего очага. Самоудовлетворение, за исключением гарантии сексуальной связи, является случайным и несущественным. Природа требует только выживания; но культура цивилизации продолжает увеличивать удовольствия от брака и удовлетворение от семейной жизни.
\vs p068 2:10 \pc Если тщеславие понимать широко, как чувство гордости, честолюбие и славу, то мы можем понять не только, как эти склонности способствуют формированию человеческого союза, но и как они удерживают людей вместе, поскольку такие эмоции бесполезны без аудитории, перед которой их можно выставлять напоказ. Вскоре к тщеславию прибавились и другие эмоции и импульсы, которым нужна социальная арена, где они могли бы быть показаны и удовлетворены. Эти эмоции дали начало всем видам древнего искусства, обрядам и формам спортивных игр и соревнований.
\vs p068 2:11 Тщеславие очень способствовало рождению общества; но во времена данных откровений изворотливые усилия тщеславного поколения угрожают утопить и погубить всю сложную структуру высокоспециализированной цивилизации. Потребность в удовольствиях уже давно заместила потребность в пище; разумные общественные цели самоподдержания быстро переходят в низменные и угрожающие формы самоудовлетворения. Самоподдержание строит общество; необузданное самоудовлетворение неизменно разрушает цивилизацию.
\usection{3. Социализирующее влияние страха призраков}
\vs p068 3:1 Примитивные желания создают исходное общество, но страх призрака поддерживает его и привносит сверхчеловеческий аспект в его существование. Обычный страх был физиологическим по происхождению: страх физической боли, неудовлетворенного голода или каких\hyp{}нибудь стихийных бедствий; но страх призрака был новой и высокой разновидностью ужаса.
\vs p068 3:2 Вероятно, самым важным единичным фактором эволюции человеческого общества был сон о призраках. Хотя большинство снов приводили примитивный ум в сильное смущение, сон о призраках по\hyp{}настоящему терроризировал ранних людей, заставляя этих суеверных спящих искать помощи друг у друга в добровольном и искреннем союзе для взаимной защиты от неясных и невидимых воображаемых опасностей мира духов. Сон о призраках был одним из самых ранних появившихся различий между животным и человеческим типами разума. Животные не представляют себе продолжение существования в посмертии.
\vs p068 3:3 За исключением этого фактора призрака, все общество базировалось на жизненно важных потребностях и основных биологических побуждениях. Но страх призрака ввел новый фактор в цивилизацию, страх, который, несомненно, лежал вне элементарных потребностей личности и который был даже намного значительней борьбы за поддержание группы. Ужас перед духами умерших выявил новую и удивительную форму страха --- угнетающий и могущественный ужас, который способствовал объединению рыхлых общественных союзов ранних веков в относительно строго дисциплинированные и лучше контролируемые примитивные группы древних времен. Это бессмысленное суеверие, что\hyp{}то от него существует до сих пор, через суеверный страх перед чем\hyp{}то несуществующим и сверхъестественным подготовило умы людей к более позднему осознанию «страха перед Господом, который является началом мудрости». Беспочвенным страхам эволюции предназначено быть замещенными благоговением перед Божеством, благоговением, внушенным откровением. Ранний культ страха призрака превратился в мощные социальные узы, и с тех далеких дней человечество в большей или меньшей степени стало стремиться к достижению духовности.
\vs p068 3:4 \pc Голод и любовь объединили людей; тщеславие и страх призрака держали их вместе. Но только лишь эти эмоции, без воздействия способствующих миру откровений, не в состоянии долго сдерживать подозрение и раздражение, свойственные человеческим взаимоотношениям. Без помощи из сверхчеловеческих источников связь общества рвется при достижении определенных пределов, и те же самые влияния, которые мобилизовали общество --- голод, любовь, тщеславие и страх, --- бросают человечество в войну и кровопролитие.
\vs p068 3:5 Стремление человеческой расы к миру не является природным даром; оно проистекает из учения религий откровения, из накопленного опыта прогрессирующих рас, но, особенно, из учений Иисуса, Принца Мира.
\usection{4. Эволюция нравов}
\vs p068 4:1 Все современные социальные институты есть результат эволюции примитивных обычаев ваших диких предков; все сегодняшние условности есть измененные и расширенные обычаи вчерашнего дня. Для группы обычай то же самое, что и привычка для индивидуума; и групповые обычаи превращаются в народные обычаи, или племенные традиции --- в массовые условности. Все институты современного человеческого общества ведут свое скромное происхождение из этих ранних источников.
\vs p068 4:2 Следует иметь в виду, что нравы возникли из попытки приспособить групповой образ жизни к условиям существования масс; нравы были первым социальным институтом человека. И все племенные реакции выросли из попытки избежать боли и унижения, одновременно со стремлением к удовольствию и власти. Зарождение народных обычаев, так же как и зарождение языков, всегда бессознательно и непроизвольно, и потому всегда окутано тайной.
\vs p068 4:3 \pc Страх призрака привел примитивного человека к представлению о сверхъестественном и, таким образом, заложил надежный фундамент для тех могущественных влияний этики и религии, которые в свою очередь передавали неизменными нравы и обычаи общества от поколения к поколению. Одним из обстоятельств, в первую очередь влияющих на создание и формирование нравов, была вера в то, что мертвые ревниво относились к тому, как они жили и умерли, и поэтому жестоко карают тех живущих смертных, которые отваживаются безрассудно пренебрегать правилами жизни, которые чтили мертвые, когда были во плоти. В настоящее время, это лучше всего иллюстрируется благоговением желтой расы перед своими предками. Позднее развивающаяся религия заметно усилила страх призрака, укрепляя тем самым нравы, но развивающаяся цивилизация все больше и больше освобождала человечество от рабской зависимости от страха и предрассудков.
\vs p068 4:4 До освобождающего и расширяющего кругозор обучения учителями Даламатии древний человек был беспомощной жертвой ритуала нравов; примитивный дикарь был связан бесконечным обрядом. Все, что он ни делал, начиная с момента пробуждения утром и до того, как он засыпал в своей пещере ночью, должно было делаться именно так, а не иначе --- в соответствии с народными обычаями племени. Он был рабом тирании заведенного порядка; в его жизни не было ничего свободного, спонтанного или оригинального. Не было естественного движения к существованию, более высокому ментально, морально или социально.
\vs p068 4:5 Древний человек был очень зажат обычаем; дикарь был подлинным рабом заведенного порядка, но то и дело появлялись такие индивидуумы, которые отваживались искать новые пути мышления и улучшенные способы существования. Тем не менее, инерция жизни примитивного человека --- своеобразный биологический защитный тормоз, предохраняющий от внезапного низвержения в разрушительный дисбаланс стремительно продвигающейся вперед цивилизации.
\vs p068 4:6 Но обычаи не есть абсолютное зло; их эволюция должна продолжаться. Попытка полной смены обычаев путем радикальной революции оказывается почти фатальной для стабильности цивилизации. Обычай является связующей нитью, которая скрепляет цивилизацию. Путь человеческой истории усеян остатками отброшенных обычаев и устарелых социальных привычек; но ни одна цивилизация, отказавшаяся от своего уклада жизни без одновременного принятия лучших или более подходящих обычаев, не выдержала испытания временем.
\vs p068 4:7 Выживание общества зависит главным образом от прогрессирующей эволюции его нравов. Процесс эволюции обычая движим желанием экспериментировать; возникают новые идеи --- появляется конкуренция. Развивающаяся цивилизация принимает прогрессивные идеи и продолжает существовать; время и обстоятельства в конечном итоге избирают для выживания наиболее подходящую группу. Но это не означает, что каждое отдельное и единичное изменение в структуре человеческого общества происходит на благо. Нет, совсем не так! Оттого\hyp{}то в долгой прогрессивной борьбе цивилизации Урантии было так много регрессов.
\usection{5. Методы обработки земли --- искусство использования}
\vs p068 5:1 Земля --- место действия общества; люди --- актеры. Человек постоянно должен так регулировать свою деятельность, чтобы она соответствовала земельной ситуации. Эволюция нравов всегда зависит от соотношения земля --- человек. Хотя это сложно распознать, тем не менее, это верно. Методы обработки земли, или искусство землепользования, плюс уровень жизни человека, зависят от всего уклада жизни народа, его нравов. И то, как человек приспосабливается к потребностям жизни, и составляет культуру его цивилизации.
\vs p068 5:2 Самые древние человеческие культуры появились вдоль рек восточного полушария, и в прогрессивном развитии цивилизации было четыре важнейшие ступени. А именно:
\vs p068 5:3 \ublistelem{1.}\bibnobreakspace \bibemph{Стадия собирательства.} Потребность в пище, голод, привели к первой форме производственной организации --- примитивной цепочке собирающих пищу. Иногда такая цепочка голодных, проходя по земле и тщательно собирая пищу, могла достигать десяти миль в длину. Это была примитивная кочевая стадия культуры, и в настоящее время африканские бушмены ведут именно такой образ жизни.
\vs p068 5:4 \pc \ublistelem{2.}\bibnobreakspace \bibemph{Стадия охоты.} Изобретение орудий защиты позволило человеку стать охотником и, таким образом, в какой\hyp{}то мере освобождало от рабской зависимости от поисков пищи. Андонит, который сильно зашиб кулак в жестокой схватке, задумался и заново открыл идею использования длинной палки вместо руки и твердого кремня, привязанного к ее концу с помощью сухожилий, вместо кулака. Многие племена независимо друг от друга совершили открытия подобного рода, а такие молоты различной формы представляли собой огромный шаг вперед в развитии человеческой цивилизации. Отдельные племена австралийских аборигенов и сегодня незначительно ушли вперед от этого уровня.
\vs p068 5:5 Постепенно голубые люди стали опытными охотниками и звероловами: перегораживая реки, они в огромных количествах ловили рыбу, высушивая излишки на зиму. Для ловли дичи применялись многообразные искусные силки и ловушки, но примитивные расы не охотились на крупных животных.
\vs p068 5:6 \pc \ublistelem{3.}\bibnobreakspace \bibemph{Стадия пастушества.} Эта фаза цивилизации стала возможной благодаря одомашниванию животных. К наиболее молодым пастушеским народам можно отнести арабов и аборигенов Африки.
\vs p068 5:7 Пастушеский образ жизни еще больше сократил рабскую зависимость от поисков пищи. Человек учился жить на проценты со своего капитала, на приросте своих стад; а это давало больше свободного времени для культуры и прогресса.
\vs p068 5:8 Общество до пастушества было обществом равноправия полов, но распространение животноводства принизило женщину до уровня социального рабства. В древние времена снабжение животной пищей было обязанностью мужчины; делом женщины было обеспечение съедобными растениями. Поэтому, когда человечество вступило в пастушескую эру своего существования, статус женщины сильно понизился. Она по\hyp{}прежнему должна была тяжело трудиться, обеспечивая жизненные потребности в растениях, тогда как мужчине достаточно было просто пойти в свое стадо, чтобы иметь в избытке животную пищу. Мужчина стал, таким образом, сравнительно независим от женщины; на протяжении всего периода пастушества статус женщины устойчиво понижался. К концу этой эры по своему положению она стала лишь немногим выше животного, ее удел был работать и рожать потомство, почти такой же, как у животных из стада, --- работать и приносить молодняк. Люди пастушеских веков очень любили свой скот; тем более жаль, что они не смогли развить большей привязанности к своим женам.
\vs p068 5:9 \pc \ublistelem{4.}\bibnobreakspace \bibemph{Земледельческая стадия.} С этой эры в культуру вошли растения, и она являет собой высший тип материальной цивилизации. И Калигастия и Адам старались учить садоводству и земледелию. Адам и Ева были садоводами, а не пастухами, и занятие садоводством в те дни означало более высокий уровень культуры. Выращивание растений оказывает облагораживающее влияние на все расы человечества.
\vs p068 5:10 Земледелие более чем в четыре раза повысило соотношение земля\hyp{}человек в мире. Оно может сочетаться с пастушеством предыдущей культурной стадии. Когда частично совпадают три стадии, то мужчины занимаются охотой, а женщины возделывают землю.
\vs p068 5:11 Между пастухами и земледельцами всегда происходили распри. Охотник и пастух были воинственными; земледелец представлял собой более миролюбивый тип. Связь с животными подразумевает борьбу и насилие; связь с растениями вселяет терпение, спокойствие и мир. Земледелие и промышленное производство являются мирной деятельностью. Но у этой социальной деятельности в мире есть одна слабая сторона: она не возбуждает и не захватывает, она очень монотонна и однообразна.
\vs p068 5:12 \pc Человеческое общество эволюционировало от стадии охоты через стадию скотоводства к стадии оседлого земледелия. И каждая стадия развивающейся цивилизации была все менее кочевой; все большее число людей начинали вести оседлый образ жизни --- жить дома.
\vs p068 5:13 Сейчас земледелие дополняет индустрия, что, соответственно, увеличивает урбанизацию и количество не занятого в земледелии гражданского населения. Но индустриальная эра не сможет выжить, если ее лидеры не поймут, что даже высочайшие социальные достижения должны базироваться на прочном сельскохозяйственном фундаменте.
\usection{6. Эволюция культуры}
\vs p068 6:1 Человек --- создание почвы, дитя природы; как бы он не старался сбежать от земли, в конечном итоге ему это никогда не удается. «Из праха вы вышли, в прах и вернетесь» --- относится в полном смысле слова ко всему человечеству. Борьба за землю была, есть и всегда будет основной борьбой человека. Первые общественные объединения примитивных человеческих существ были необходимы, чтобы победить в схватках за землю. Соотношение земля\hyp{}человек лежит в основе всей социальной цивилизации.
\vs p068 6:2 Человеческий разум, с помощью мастерства и науки повышал плодородие земли; одновременно в какой\hyp{}то степени был взят под контроль естественный прирост населения, это давало больше свободного времени и помогало в построении культурной цивилизации.
\vs p068 6:3 \pc Человеческое общество контролируется законом, который гласит: население изменяется прямо пропорционально культуре земледелия и обратно пропорционально данному уровню жизни. В древние века, даже в большей степени, чем сейчас, закон спроса и предложения в отношении людей и земли определял оценочную стоимость того и другого. При изобилии земли, на незаселенных территориях, потребность в людях была огромной, а потому ценность человеческой жизни намного повышалась и, следовательно, людские потери были особенно чувствительны. При недостатке земли и связанной с этим перенаселенностью человеческая жизнь относительно обесценивалась, да так, что война, голод, чума вызывали меньшее беспокойство.
\vs p068 6:4 Когда падает плодородие почвы или увеличивается население, возобновляется неизбежная борьба; самые худшие черты человеческой натуры выходят на поверхность. Увеличение плодородия почвы, развитие ремесел и сокращение населения --- все это благоприятно сказывается на развитии лучших сторон человеческой природы.
\vs p068 6:5 \pc Общество, расположенное на границе цивилизации, занято неквалифицированной деятельностью; изящные искусства, подлинный научный прогресс и духовная культура в основном процветают в крупных центрах жизни, когда те обеспечиваются населением, занятым в сельском хозяйстве и промышленности, численность которого чуть меньше, чем соотношение земля --- человек. Города всегда повышают возможности их обитателей, либо на благо, либо во вред.
\vs p068 6:6 Уровень жизни всегда влиял на размер семьи. Чем выше уровень, тем меньше семья, и так до стадии равновесия или постепенного вымирания.
\vs p068 6:7 На протяжении всех веков уровень жизни определял скорее качество выживающего населения, чем просто его количество. Уровень жизни местных групп формирует новые социальные касты, новые нравы. Когда уровень жизни становится слишком усложненным или слишком роскошным, он быстро начинает быть самоубийственным. Каста является прямым результатом высокого социального напряжения из\hyp{}за острой конкуренции, вызванной высокой плотностью населения.
\vs p068 6:8 Древние расы часто прибегали к практике, направленной на ограничение населения; все примитивные племена убивали изуродованных и больных детей. До того, как жен стали покупать, младенцев женского пола часто убивали. Детей иногда душили при рождении, но чаще всего просто бросали на произвол судьбы. Отец близнецов обычно настаивал, чтобы один из них был убит, поскольку считалось, что множественные роды вызываются либо магией, либо изменой. Однако, как правило, близнецов одного пола щадили. Если это табу на близнецов и было когда\hyp{}то распространено почти по всему миру, то оно никогда не было свойственно укладу жизни андонитов; эти люди всегда считали близнецов добрым предзнаменованием.
\vs p068 6:9 Многие расы научились делать аборт, и эта операция стала обычной после введения табу на рождение ребенка у незамужних. Для них в течение долгого времени было в порядке вещей убивать свое потомство, но в более цивилизованных группах незаконнорожденные дети брались под опеку матерью девушки. Многие примитивные кланы были фактически уничтожены практикой абортов и детоубийства. Но несмотря на диктат нравов, лишь очень немногих детей когда\hyp{}либо убивали после хоть единственного кормления грудью --- материнский инстинкт слишком силен.
\vs p068 6:10 Даже в двадцатом веке все еще сохранялись остатки таких примитивных методов регулирования численности населения. В Австралии существует племя, где матери отказываются воспитывать более двух или трех детей. Еще недавно одно племя каннибалов поедало каждого пятого рождающегося ребенка. На Мадагаскаре некоторые племена все еще убивают всех детей, родившихся в определенные, так называемые несчастливые дни, это уносит из жизни почти двадцать пять процентов всех младенцев.
\vs p068 6:11 \pc В масштабе планеты перенаселенность никогда не была в прошлом, серьезной проблемой, но поскольку войны сокращаются, а наука все эффективнее контролирует болезни людей, перенаселенность может стать серьезной проблемой уже в ближайшем будущем. В это время подвергнется великому испытанию мудрость мировых лидеров. Хватит ли правителям Урантии проницательности и смелости поощрять рождение человеческих существ среднего, или нормального уровня развития вместо людей выше нормального уровня и чрезвычайно быстро увеличивающихся групп населения с уровнем развития ниже нормального? Нормального человека надо поощрять; он является костяком цивилизации и источником появляющихся в результате мутаций гениев расы. Численность населения с уровнем развития ниже нормального должна находиться под контролем общества; она не должна увеличиваться больше, чем это нужно для потребности низкотехнологического производства, для таких работ, где требуется интеллект чуть выше животного уровня; в то время как для человека с высоким уровнем развития низкоквалифицированный труд оборачивается рабством.
\vsetoff
\vs p068 6:12 [Представлено Мелхиседеком, иногда находящимся на Урантии.]
