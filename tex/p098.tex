\upaper{98}{Учения Мелхиседека на Западе}
\author{Мелхиседек}
\vs p098 0:1 Учения Мелхиседека проникали в Европу многими путями, но главным образом через Египет, и после глубокой эллинизации, а позднее христианизации стали частью западной философии. Идеалы Западного мира были в основном сократовскими, и его более поздняя религиозная философия стала философией Иисуса, измененной и приспособившейся благодаря контакту с развивавшимися западной философией и религией, которые достигли своей кульминации в христианской церкви.
\vs p098 0:2 В течение долгого времени салимские миссионеры продолжали свою деятельность в Европе и постепенно растворились в многочисленных периодически возникавших культах и ритуальных группах. Среди тех, кто сохранил салимские учения в наиболее чистом виде, необходимо отметить киников. Эти проповедники веры и упования на Бога еще действовали в Римской Европе в первом веке после Рождества Христа и позднее стали частью начинавшей формироваться христианской религии.
\vs p098 0:3 Многое из салимского учения распространялось в Европе еврейскими наемными солдатами, которые сражались в столь многих войнах Запада. В древности евреи славились своей военной доблестью так же, как и своими теологическими особенностями.
\vs p098 0:4 Основные доктрины греческой философии, еврейской теологии и христианской этики, по сути, были отголосками более ранних учений Мелхиседека.
\usection{1. Салимская религия у греков}
\vs p098 1:1 Салимские миссионеры могли бы создать некую религиозную структуру у греков, если бы не их строгое толкование своей клятвы посвящения, обязательства, наложенного на них Мелхиседеком, которое запрещало им организацию особых собраний для богопочитания и требовало от каждого учителя обещания не действовать в качестве священника, никогда не получать плату за религиозное служение, а только пищу, одежду и кров. Когда учителя Мелхиседека проникли в доэллиническую Грецию, то нашли там народ, который еще придерживался традиций Адама\hyp{}сына и времен Андитов, однако эти учения были крайне фальсифицированы понятиями и представлениями орд низших рабов, которых привозили к греческим берегам во все возраставшем количестве. Эта фальсификация и послужила причиной возврата к грубому анимизму с кровавыми ритуалами, причем низшие классы устраивали церемонии даже из казни осужденных на смерть преступников.
\vs p098 1:2 Раннее влияние салимских учителей было почти уничтожено так называемым арийским вторжением с юга Европы, а также с Востока. Эти эллинские захватчики принесли с собой антропоморфные представления о Боге, сходные с теми, что принесли их собратья\hyp{}арийцы в Индию. Заимствование этих представлений ознаменовало начало эволюции семейства греческих богов и богинь. Новая религия отчасти была основана на культах прибывавших эллинских варваров, но также пользовалась намного более древними мифами жителей Греции.
\vs p098 1:3 Эллинские греки нашли средиземноморский мир в состоянии, когда в нем в значительной степени господствовал культ матери, и навязали этим народам своего человекобога Дия\hyp{}Зевса, который, подобно Яхве у генотеических семитов, уже стал главой целого греческого пантеона подчиненных богов. Причем греки в конце концов в своем представлении о Зевсе достигли бы истинного монотеизма, если бы не сохранили веру в высшую власть Судьбы. Ведь Бог, являющийся конечной ценностью, должен быть сам вершителем судеб и творцом предопределения.
\vs p098 1:4 Вследствие этих факторов религиозной эволюции вскоре сформировалась популярная вера в беззаботных богов Олимпа, богов, обладающих скорее человеческими, нежели божественными качествами, богов, к которым умные греки никогда не относились очень серьезно. И никогда не испытывали к этим божествам, созданным ими самими, ни великой любви, ни великого страха. У них было патриотическое и национальное чувство к Зевсу и его семейству полулюдей и полубогов, но вряд ли они их почитали или поклонялись им.
\vs p098 1:5 Эллины настолько прониклись антисвященническими доктринами ранних учителей Салима, что в Греции не появилось сколь\hyp{}нибудь значимого духовенства. Даже изготовление изображений богов стало более созданием произведений искусства, нежели вопросом богопочитания.
\vs p098 1:6 Олимпийские боги --- пример типичного антропоморфизма человека. Однако греческая мифология была более эстетической, чем этической. Греческая религия была полезной, ибо изображала вселенную, управляемую группой божеств. Однако греческая мораль, этика и философия вскоре намного переросли понятие бога, и это несоответствие между интеллектуальным и духовным ростом было столь же опасным для Греции, каким оно оказалось для Индии.
\usection{2. Греческая философская мысль}
\vs p098 2:1 Мало уважаемая и поверхностная религия не может выжить, особенно тогда, когда у нее нет священнослужителей, которые способствовали бы развитию ее форм и наполняли бы сердца ее почитателей страхом и благоговением. Олимпийская религия не обещала спасения и не утоляла духовную жажду своих приверженцев, а потому была обречена на гибель. Через тысячелетие после своего появления она почти исчезла, и греки остались без национальной религии --- боги Олимпа утратили свою власть над лучшими умами.
\vs p098 2:2 Такова была ситуация, когда на протяжении шестого века до Рождества Христа Восток и Левант переживали возрождение духовного сознания и вновь стали тяготеть к монотеизму. Однако Запад в этом новом течении не участвовал; не принимали участия в этом религиозном ренессансе и Европа и северная Африка. Греки, однако, были заняты великолепным интеллектуальным совершенствованием. Они начали преодолевать страх и более не стремились к религии как к противоядию от него, однако не понимали, что истинная религия --- это лекарство от душевного голода, духовной тревоги и нравственного отчаяния. Они искали утешения для души в глубоком размышлении --- в философии и метафизике. И, отказавшись от размышлений о самосохранении --- спасении, --- обратились к самореализации и самопониманию.
\vs p098 2:3 Неутомимой мыслью греки пытались достигнуть того сознания уверенности, которое служило бы заменой веры в спасение, но потерпели полную неудачу. Понять новое учение могли лишь наиболее умные представители высших классов эллинских народов; рядовые же потомки рабов прежних поколений не обладали способностью к восприятию этой новой замены религии.
\vs p098 2:4 \pc Философы относились с презрением ко всем формам поклонения, несмотря на то, что практически все они в той или иной мере соглашались с предпосылками веры в салимское учение о «Разуме вселенной», «идее Бога» и «Великом Источнике». И коль скоро греческие философы признавали божественное и сверхконечное, то они были откровенными монотеистами; они почти не придавали никакого значения существованию целой плеяды Олимпийских богов и богинь.
\vs p098 2:5 Греческие поэты пятого и шестого веков, в особенности Пиндар, пытались реформировать греческую религию. Они возвысили ее идеалы, но были более художниками, нежели религиозными деятелями. И не сумели выработать метод, способствующий сохранению верховных ценностей.
\vs p098 2:6 Ксенофан учил о едином Боге, но его представление о божестве было слишком пантеистичным и не давало личностного Отца смертному человеку. Анаксагор был механистом, если не считать того, что он признавал Первопричину и Первоначальный Разум. Сократ и его последователи Платон и Аристотель учили, что добродетель --- это знание, а доброта --- здоровье души; что лучше пострадать от несправедливости, чем быть несправедливым самому; что отвечать на зло злом --- неправильно и что боги мудры и добры. Их главными добродетелями были: мудрость, смелость, умеренность и справедливость.
\vs p098 2:7 \pc Эволюция религиозной философии и эллинских, и еврейских народов демонстрирует два резко противоположных направления в деятельности церкви как института в формировании культурного прогресса. В Палестине человеческая мысль была настолько подчинена священникам и зависела от писаний в такой степени, что философия и эстетика были всецело погружены в религию и мораль. В Греции же почти полное отсутствие священников и «священных писаний» оставляло человеческий ум свободным и нестесненным, что привело к поразительному развитию глубокомыслия. Однако религия как личный опыт не смогла идти в ногу с интеллектуальными исследованиями природы и реальности космоса.
\vs p098 2:8 В Греции вера была подчинена мысли; в Палестине мысль повиновалась вере. В основном могущество христианства объясняется тем, что оно многое позаимствовало и у еврейской морали, и у греческой мысли.
\vs p098 2:9 В Палестине религиозная догма выкристаллизовалась настолько, что начала угрожать дальнейшему росту; в Греции же человеческая мысль стала столь абстрактной, что представление о Боге превратилось в туманные испарения пантеистического умозрения, мало чем отличавшегося от неличностной Бесконечности философов\hyp{}брахманов.
\vs p098 2:10 \pc Однако простые люди того времени не могли понять греческую философию самореализации и абстрактного Божества и не особенно интересовались ею, а, скорее, жаждали обещаний спасения в сочетании с личным Богом, который мог бы услышать их молитвы. Они изгоняли философов, преследовали остатки салимского культа (к тому времени оба учения сильно смешались) и были готовы погрузиться в ужасное и разнузданное безрассудство мистериальных культов, которые в то время распространились в странах Средиземноморья. В Олимпийском пантеоне возникли элевсинии --- греческая версия поклонения плодородию, процветало поклонение диониссийской природе; лучшим из культов было орфическое братство, нравственные проповеди и обещания спасения которого привлекли многих.
\vs p098 2:11 Вся Греция была увлечена новыми методами достижения спасения, этими эмоциональными и страстными церемониями. Ни одна нация никогда не достигла таких высот и артистизма в философии за столь короткое время; никто и никогда не создавал такую развитую систему этики практически без Божества и полностью лишенную обещания человеческого спасения, но ни одна нация никогда не погружалась столь быстро, глубоко и безудержно в такие глубины интеллектуального застоя, нравственной развращенности и духовной нищеты, как эти же самые греческие народы, бросившиеся в водоворот мистериальных культов.
\vs p098 2:12 \pc Религии долго существовали без поддержки со стороны философии, но не многие философии как таковые сохранялись в течение длительного времени без некоторого отождествления с религией. Философия для религии --- то же, что концепция для действия. Однако идеальное состояние для человека то, при котором философия, религия и наука, благодаря совместному действию мудрости, веры и опыта, сливаются в полное смысла единство.
\usection{3. Учения Мелхиседека в Риме}
\vs p098 3:1 После того как более древние формы поклонения семейным богам переросли в племенное почитание бога войны Марса, вполне естественно, что впоследствии соблюдение религии латинов явилось вопросом политическим в гораздо большей степени, чем в случае интеллектуальных систем греков и браманов или более духовных религий некоторых других народов.
\vs p098 3:2 В период великого монотеистического ренессанса евангелия Мелхиседека в шестом веке до Рождества Христова, в Италию проникло очень мало салимских миссионеров, а те, кому это сделать удалось, не могли преодолеть влияние быстро разраставшегося этрусского духовенства с его новой плеядой богов и храмов, из которых и была создана государственная религия Рима. Эта религия латинских племен отнюдь не была ни тривиальной и корыстной, как религия греков, ни суровой и тиранической, как религия евреев, и большей частью заключалась в соблюдении только форм, клятв и табу.
\vs p098 3:3 Римская религия подвергалась сильному влиянию обширных культурных заимствований из Греции. И в конце концов большинство Олимпийских богов было перенесено и включено в латинский пантеон. Греки издавна поклонялись огню домашнего очага; так, Гестия была целомудренной богиней домашнего очага, а Веста --- римской богиней дома. Зевс стал Юпитером; Афродита --- Венерой, и так произошло со множеством богов Олимпа.
\vs p098 3:4 Религиозная инициация римских юношей была событием их торжественного посвящения служению государству. Присяга и вступление в гражданство в действительности были религиозными церемониями. Латинские народы содержали храмы, алтари и святыни и во времена кризиса советовались с оракулами. Они сохраняли кости героев, а позднее и кости христианских святых.
\vs p098 3:5 Эта внешняя и бесстрастная форма псевдорелигиозного патриотизма была обречена на крах так же, как высокоинтеллектуальное и высокохудожественное поклонение греков, не устоявшее перед страстным и глубоко эмоциональным поклонением мистериальных культов. Величайшим из этих разрушительных культов была мистериальная религия секты Матери Бога, центр которой в те дни находился на том же самом месте, где теперь в Риме стоит церковь Святого Петра.
\vs p098 3:6 \pc Возникшее Римское государство покоряло политически, но в свою очередь само покорялось культами, ритуалами, мистериями, а также египетскими, греческими и левантийскими представлениями о боге. Эти заимствованные культы процветали всюду в Римском государстве вплоть до времени Августа, который исключительно по политическим и гражданским причинам предпринял героическую и отчасти успешную попытку уничтожить мистерии и возродить прежнюю политическую религию.
\vs p098 3:7 Один из священнослужителей государственной религии рассказал Августу о предпринятых ранее попытках салимских учителей распространить учение о едином Боге, высшем Божестве, стоящем во главе всех сверхъестественных существ; и идея эта завладела императором настолько, что он построил множество храмов, наполнил их прекрасными изображениями, реорганизовал духовенство государства, восстановил государственную религию, назначил себя действующим первосвященником всех и как император, не колеблясь, верховным богом провозгласил самого себя.
\vs p098 3:8 Новая религия поклонения Августу процветала и соблюдалась на протяжении всей его жизни по всей империи, кроме Палестины, родины евреев. И эта эра человеческих богов продолжалась до тех пор, пока в официальном римском культе не появился реестр, включавший более сорока возвысивших себя человеческих божеств, причем все они претендовали на чудесное рождение и другие сверхчеловеческие атрибуты.
\vs p098 3:9 \pc Последнее выступление сокращавшегося отряда салимских верующих было предпринято киниками, группой искренних проповедников, которые призывали римлян оставить свои дикие и бессмысленные религиозные ритуалы и вернуться к форме поклонения, олицетворяющей евангелие Мелхиседека в том измененном и искаженном виде, какой оно приобрело вследствие соприкосновения с философией греков. Однако народ в основном киников отвергал, предпочитая погружаться в ритуалы мистерий, которые не только предлагали надежду на личное спасение, но и удовлетворяли страсть к разнообразию, возбуждению и развлечению.
\usection{4. Мистериальные культы}
\vs p098 4:1 Большая часть народа греко\hyp{}римского мира, утратив свои примитивные семейные и государственные религии и будучи неспособной или несклонной к пониманию смысла греческой философии, обратилась к захватывающим и эмоциональным мистериальным культам Египта и Леванта. Простые люди желали обещаний спасения --- религиозного утешения на сегодняшний день и уверений в надежде на бесконечное существование после смерти.
\vs p098 4:2 Наиболее популярными стали три мистериальных культа:
\vs p098 4:3 \ublistelem{1.}\bibnobreakspace Фригийский культ Кибелы и ее сына Аттиса.
\vs p098 4:4 \ublistelem{2.}\bibnobreakspace Египетский культ Осириса и его матери Исис.
\vs p098 4:5 \ublistelem{3.}\bibnobreakspace Иранский культ поклонения Митре как спасителю и искупителю грешного человечества.
\vs p098 4:6 \pc Фригийские и египетские мистерии учили, что божественный сын (соответственно Аттис и Осирис) испытал смерть и был воскрешен божественной силой, и далее, что все, кто был должным образом посвящен в мистерию и благоговейно праздновал годовщину смерти и воскрешения бога, благодаря этому станут причастны его божественной природе и его бессмертию.
\vs p098 4:7 \pc Церемонии фригийцев были эффектными, но унизительными; их кровавые праздники свидетельствуют о том, какими деградировавшими и примитивными стали эти левантийские мистерии. Самым святым днем считалась черная пятница, «день крови», празднуемый в честь самоубийства Аттиса. После трех дней празднования жертвы и смерти Аттиса праздник обращался к радости в честь его воскрешения.
\vs p098 4:8 Ритуалы поклонения Исис и Осирису были более утонченными и впечатляющими, чем ритуалы фригийского культа. Этот египетский ритуал был основан на легенде о древнем боге Нила, боге, умершем и воскресшем, представление о котором происходило из наблюдения ежегодно повторяющегося прекращения роста растительности, вслед за которым наступает весеннее возрождение всех живых растений. Неистовство, с которым соблюдались эти мистериальные культы, и происходившие при исполнении их церемоний оргии, которые якобы должны были приводить к «воодушевлению» от осознания божественного, порой были отвратительнейшими.
\usection{5. Культ Митры}
\vs p098 5:1 Фригийские и египетские мистерии в конце концов уступили место величайшему из всех мистериальных культов --- поклонению Митре. Митраистский культ взывал ко многому в человеческой природе и постепенно вытеснил обоих своих предшественников. Митраизм распространился в Римской империи благодаря пропаганде, которую вели римские легионы, набранные в Леванте, где эта религия была популярной, ибо они несли эту веру всюду, куда бы ни шли. Причем этот новый религиозный ритуал по сравнению с более ранними мистериальными культами был великим шагом вперед.
\vs p098 5:2 Культ Митры возник в Иране и долго сохранялся на своей родине вопреки воинственной оппозиции последователей Зороастра. Однако ко времени, когда митраизм достиг Рима, он, благодаря заимствованию многих учений Зороастра, значительно усовершенствовался. Главным образом через культ Митры религия Зороастра и оказала влияние на христианство, возникшее позднее.
\vs p098 5:3 \pc Культ Митры изображал воинственного бога, который родился из огромной скалы, совершал доблестные подвиги и извергал воду из скалы, куда попадали его стрелы. Были там и потоп, от которого спасся один человек в специально построенной лодке, и последняя вечеря, которую Митра разделил с богом солнца перед тем как вознестись на небо. Этот бог солнца, или Sol Inviсtus (Непобедимое Солнце), являлся вырождающимся представлением зороастризма о божестве Ахурамазде. Митру представляли как выжившего защитника бога солнца в его борьбе с богом тьмы. И в знак признания того, что Митра убил мифического священного быка, его сделали бессмертным, возвысив до положения ходатая перед богами небесными за род человеческий,
\vs p098 5:4 Приверженцы этого культа поклонялись в пещерах и других тайных местах, распевали гимны, бормотали заклинания, ели мясо жертвенных животных и пили кровь. Поклонялись они три раза в день, при этом один раз в неделю, в день бога солнца, совершались особые церемонии, а ежегодный праздник Митры, который отмечался 25 декабря, соблюдался с особой тщательностью. Считалось, что причащение таинству дает вечную жизнь, немедленное перенесение после смерти в лоно Митры, дабы пребывать там в блаженстве до судного дня. В судный же день ключи Митры от неба откроют врата Рая, дабы принять верных, после чего все некрещенные из живых и мертвых будут уничтожены при возвращении Митры на землю. Учили, что, умирая, человек отправляется к Митре на суд и что во время конца света Митра призовет всех умерших из могил на страшный суд. Порочных уничтожит огонь, а праведные вместе с Митрой будут царствовать вечно.
\vs p098 5:5 Вначале это была религия только для мужчин, и существовало семь различных чинов, в которые верующие могли быть последовательно посвящены. Позднее в храмы Великой Матери, примыкавшие к храмам Митры, стали допускать жен и дочерей верующих. Женский культ был смесью митраистского ритуала и церемонии фригийского культа Кибелы, матери Аттиса.
\usection{6. Митраизм и христианство}
\vs p098 6:1 До прихода мистериальных культов и христианства личностная религия в цивилизованных странах северной Африки и Европы почти не развивалась как независимый институт и была, скорее, делом семьи, города\hyp{}государства, империи или политическим вопросом. Эллинические греки так и не создали централизованной системы поклонения; их ритуалы имели местный характер; у них не было священников и «священной книги». Во многом так же, как у римлян, их религиозным институтам для сохранения высших нравственных и духовных ценностей не хватало мощной движущей силы. И хоть верно то, что превращение религии в институт, как правило, снижало ее духовное качество, фактом остается и то, что до сих пор ни одной религии не удалось выжить без помощи определенной, большей или меньшей, организации.
\vs p098 6:2 Религия Запада, таким образом, продолжала слабеть вплоть до дней скептиков, киников, эпикурейцев и стоиков и, что самое главное, до прихода времен великого соперничества между митраизмом и новой христианской религией Павла.
\vs p098 6:3 \pc На протяжении третьего века после Рождества Христа митраистская и христианская церкви были очень похожи и по внешнему виду и по характеру своих ритуалов. Большинство таких мест почитания находилось под землей и в обоих были алтари, на заднем плане которых различным образом изображались страдания спасителя, который принес спасение человечеству, проклятому за грех.
\vs p098 6:4 У почитающих Митру всегда был обычай, входя во храм, окунать пальцы в святую воду. А поскольку в некоторых районах были такие, кто одновременно исповедовал обе религии, они ввели этот обычай в большинстве христианских церквей в окрестностях Рима. В обеих религиях практиковалось крещение и причащение святых даров хлеба и вина. Великое различие между митраизмом и христианством, помимо различия в характерах Митры и Иисуса, заключалось в том, что одна религия поощряла воинственность, а другая была крайне миролюбивой. Терпимость митраизма по отношению к другим религиям (за исключением более позднего христианства) привела к его окончательной гибели. Но решающим фактором в борьбе между христианством и митраизмом было принятие женщин в полноправные члены братства христианской веры.
\vs p098 6:5 \pc В конце концов номинально христианская вера стала на Западе господствующей. Греческая философия дала понятие об этических ценностях, митраизм --- ритуал совершения богопочитания, а христианство как таковое --- метод сохранения нравственных и социальных ценностей.
\usection{7. Христианская религия}
\vs p098 7:1 Сын\hyp{}Творец воплотился в подобие смертной плоти и даровал себя человечеству Урантии не затем, чтобы примирить его с гневным Богом, а для того, чтобы обратить все человечество к признанию любви Отца и осознанию их сыновства по отношению к Богу. В конце концов даже великий сторонник учения об искуплении, и тот отчасти понимал эту истину, ибо заявлял, что «Бог во Христе примирил с собою мир».
\vs p098 7:2 Происхождение и распространение христианской религии выходит за рамки данного текста. Достаточно сказать, что она построена вокруг личности Иисуса из Назарета, вочеловечившегося Сына Михаила из Небадона, известного Урантии как Христос, помазанник. Христианство распространялось в Леванте и на Западе последователями этого галилеянина, и их миссионерское рвение не уступало рвению их знаменитых предшественников --- сетитов и салимитов, а также рвению их искренних азиатских современников --- учителей буддизма.
\vs p098 7:3 Христианская религия как урантийская система веры возникла вследствие смешения следующих учений, влияний, верований, культов и личных позиций отдельных людей:
\vs p098 7:4 \ublistelem{1.}\bibnobreakspace Учений Мелхиседека, которые являются основным фактором во всех религиях Запада и Востока, возникших за последние четыре тысячи лет.
\vs p098 7:5 \pc \ublistelem{2.}\bibnobreakspace Иудейской системы морали, этики, теологии и веры как в Провидение, так и в верховного Яхве.
\vs p098 7:6 \pc \ublistelem{3.}\bibnobreakspace Зороастрийского представления о борьбе между космическими добром и злом, которое уже наложило свой отпечаток и на иудаизм, и на митраизм. Благодаря продолжительному взаимодействию, сопровождавшему борьбу между митраизмом и христианством, учения иранского пророка стали мощным фактором, определяющим теологическое и философское строение и структуру догм, догматов и космологии эллинизированных и латинизированных версий учений Иисуса.
\vs p098 7:7 \pc \ublistelem{4.}\bibnobreakspace Мистериальных культов, особенно митраизма, но также и почитания Великой Матери во фригийском культе. Даже легенды о рождении Иисуса на Урантии, и те вобрали в себя элементы римской версии чудесного рождения иранского героя\hyp{}спасителя Митры, свидетелями пришествия которого на землю считалась лишь горстка принесших дары пастухов, которым об этом приближающемся событии сообщили ангелы.
\vs p098 7:8 \pc \ublistelem{5.}\bibnobreakspace Исторического факта человеческой жизни Иешуа бен Иосифа, реальности Иисуса из Назарета как прославленного Христа, Сына Бога.
\vs p098 7:9 \pc \ublistelem{6.}\bibnobreakspace Личной точки зрения Павла Тарсянина. При этом следует записать, что во времена его юности митраизм был в Тарсе основной религией. Павел едва ли мечтал о том, что его благонамеренные послания к тем, кого он обратил, когда\hyp{}нибудь более поздние христиане будут считать «словом Божиим». Такие действующие из лучших побуждений учителя не должны отвечать за то, как в дальнейшем воспользовались этими писаниями их последователи.
\vs p098 7:10 \pc \ublistelem{7.}\bibnobreakspace Философской мысли эллинистских народов от Александрии и Антиохи до Греции и от Греции до Сиракуз и Рима. Философия греков была больше созвучна версии христианства, выдвинутой Павлом, чем любой другой религиозной системе того времени, и стала важным фактором успеха христианства на Западе. Греческая философия в сочетании с теологией Павла по\hyp{}прежнему формирует основу европейской этики.
\vs p098 7:11 \pc По мере проникновения первоначальных учений Иисуса на Запад, они становились западными, и по мере того, как эти учения становились западными, они начинали терять свою потенциально универсальную привлекательность для всех рас и типов людей. Сегодня христианство стало религией, хорошо приспособленной к социальным, экономическим и политическим нравам белых рас. С тех пор оно давно перестало быть религией Иисуса, хотя по\hyp{}прежнему героически изображает прекрасную религию об Иисусе для тех индивидуумов, кто искренне пытается следовать христианскому учению. Она прославила Иисуса как Христа, Мессию\hyp{}помазанника Божиего, но в значительной степени забыла личное евангелие Учителя: Отцовство Бога и вселенское братство всех людей.
\vs p098 7:12 \pc Такова длинная история об учениях Махивенты Мелхиседека на Урантии. Прошло почти четыре тысячи лет с тех пор, как этот спасительный Сын Небадона явился на Урантию, и за это время учение «священника Эл\hyp{}Элиона, Бога Всевышнего» проникло во все расы и народы. Причем Махивента с успехом достиг цели своего необычного пришествия, и когда Михаил приготовился явиться на Урантию, в сердцах мужчин и женщин уже существовало представление о Боге, то же самое представление о Боге, которое по\hyp{}прежнему пламенеет в живом духовном опыте многочисленных детей Отца Всего Сущего, живущих своими увлекательными временными жизнями на планетах, вращающихся в космическом пространстве.
\vsetoff
\vs p098 7:13 [Представлено Мелхиседеком Небадона.]
