\upaper{15}{Семь сверхвселенных}
\author{Вселенский Цензор}
\vs p015 0:1 По отношению к Отцу Всего Сущего --- как к Отцу --- вселенные фактически не существуют; он имеет дело с личностями; он Отец личностей. По отношению к Вечному Сыну и Бесконечному Духу --- как к сотрудничающим творцам --- вселенные локализованы и индивидуальны и управляются совместно Сынами\hyp{}Творцами и Творческими Духами. По отношению же к Райской Троице вне Хавоны существует всего семь обитаемых вселенных, семь сверхвселенных, осуществляющих управление кругом первого пост\hyp{}хавонного пространственного уровня. Семь Духов\hyp{}Мастеров излучают свое влияние из центрального Острова и таким образом делают огромное творение единым гигантским диском, осью которого является вечный Райский Остров, семью спицами --- лучи семи Духов\hyp{}Мастеров, а ободом --- внешние области великой вселенной.
\vs p015 0:2 В начале материализации вселенского творения была сформулирована семеричная схема организации и управления сверхвселенной. Первое пост\hyp{}хавонное творение было разбито на семь гигантских сегментов, и были разработаны и построены миры\hyp{}центры этих сверхвселенских правительств. Современная схема управления существовала почти от самой вечности, и правителей этих семи сверхвселенных справедливо называют Древними Дней.
\vs p015 0:3 Из огромной совокупности знаний о сверхвселенных я могу передать вам лишь малую часть, однако повсюду в этих областях действует метод разумного управления как физическими, так и духовными силами, причем присутствия всемирной гравитации действуют здесь с величественной мощью и в совершенной гармонии. Вначале важно получить адекватное представление о физическом строении и материальной организации сверхвселенских доменов, ибо тогда вы будете лучше подготовлены к пониманию значения изумительной организации, предусмотренной для духовного управления ими и интеллектуального совершенствования наделенных волей творений, живущих на бесчисленном множестве обитаемых планет, разбросанных повсюду в этих семи сверхвселенных.
\usection{1. Пространственный уровень сверхвселенной}
\vs p015 1:1 В записях, наблюдениях и воспоминаниях, ограниченных пределами в миллион или миллиард ваших коротких лет, Урантия и вселенная, к которой она принадлежит, переживают событие единого долгого и неисследованного устремления в новое пространство; однако, по записям Уверсы, согласно более старым наблюдениям, в соответствии с более продолжительным опытом и расчетами нашего чина и вследствие заключений, основанных на этих и других открытиях, мы знаем, что вселенные вовлечены в упорядоченную, хорошо понимаемую и совершенно управляемую процессию, в великолепном величии вращающуюся вокруг Первого Великого Источника и Центра и вселенной его пребывания.
\vs p015 1:2 Мы давно уже открыли, что семь сверхвселенных движутся по гигантской эллиптической орбите. Ваша солнечная система и другие миры со временем отнюдь не опрометчиво устремляются без карты и компаса в неисследованное пространство. Локальная вселенная, к которой принадлежит ваша система, следует по определенному и хорошо изученному, направленному против часовой стрелки курсу, по огромному эллипсу, который окружает центральную вселенную. Этот космический путь хорошо картирован и так же прекрасно известен сверхвселенским наблюдателям звезд, как известны астрономам Урантии орбиты планет, образующих вашу солнечную систему.
\vs p015 1:3 Урантия расположена в локальной вселенной и сверхвселенной, которые не полностью формированны, и ваша локальная вселенная находится в непосредственной близости к многочисленным частично завершенным физическим творениям. Вы принадлежите к одной из относительно молодых вселенных. Однако сегодня вы отнюдь не безудержно устремляетесь в неисследованное пространство и вовсе не слепо мчитесь в неизвестные области. Вы идете по упорядоченному и предопределенному пути пространственного уровня сверхвселенной. Сейчас вы проходите через то же самое пространство, которое ваша планетарная система или ее предшественники пересекли несколько эпох назад; и когда\hyp{}нибудь в отдаленном будущем ваша система или ее последователи снова пересекут пространство, идентичное тому, через которое вы столь стремительно проноситесь сейчас.
\vs p015 1:4 \pc В настоящее время с учетом направлений, принятых на Урантии, сверхвселенная номер один движется почти прямо на север, а если смотреть на восток, находится приблизительно напротив Райского местопребывания Великих Источников и Центров и центральной вселенной Хавона. Это положение, вместе с соответствующим ему на западе, представляют собой максимальное физическое приближение миров со временем к вечному Острову. Сверхвселенная номер два находится на севере и готовится к повороту в западном направлении, тогда как сверхвселенная номер три сейчас занимает самый северный сегмент великого пространственного пути, повернув уже в изгиб, ведущий в южном направлении. Сверхвселенная номер четыре совершает сравнительно прямой полет в южном направлении, и ее передние области сейчас находятся приблизительно напротив Великих Центров. Сверхвселенная номер пять почти покинула свое положение, противолежащее Центру Центров, и продолжает следовать прямым курсом на юг, непосредственно предшествующим повороту в восточном направлении; сверхвселенная номер шесть занимает большую часть южной кривой, сегмент который ваша сверхвселенная уже почти прошла.
\vs p015 1:5 Ваша локальная вселенная Небадон принадлежит Орвонтону, седьмой сверхвселенной, которая продолжает двигаться между сверхвселенными номер один и номер шесть, пройдя недавно (в нашем понимании времени) юго\hyp{}восточный изгиб пространственного уровня сверхвселенной. Солнечная система, к которой принадлежит Урантия, прошла южную кривизну несколько миллиардов лет назад, так что как раз сейчас вы выходите за пределы юго\hyp{}восточного изгиба и быстро движетесь по длинному и сравнительно прямому северному пути. На протяжении бессчетных периодов Орвонтон будет следовать этому почти прямому курсу на север.
\vs p015 1:6 Урантия принадлежит системе, расположенной близко к пограничной области вашей локальной вселенной, а ваша локальная вселенная в настоящее время пересекает периферию Орвонтона. За вами следуют другие, однако в пространстве вы далеко отстоите от тех систем, которые движутся по великому кругу в сравнительной близости к Великому Источнику и Центру.
\usection{2. Организация сверхвселенных}
\vs p015 2:1 Только Отец Всего Сущего знает положение и действительное число обитаемых миров в пространстве; он называет их по имени и числу. Я же могу назвать лишь приблизительное количество обитаемых или пригодных для обитания планет, ибо некоторые локальные вселенные имеют больше миров, подходящих для разумной жизни, чем другие. Не формированы и все задуманные локальные вселенные. Поэтому оценки, предлагаемые мной, годятся лишь для того, чтобы дать некоторое представление об огромности материального творения.
\vs p015 2:2 \pc В великой вселенной содержится семь сверхвселенных; устроены же они приблизительно так:
\vs p015 2:3 \pc \ublistelem{1.}\bibnobreakspace \bibemph{Система.} Основная единица сверхправительства состоит из приблизительно тысячи обитаемых или пригодных для обитания миров. Пылающие солнца, холодные миры, планеты, расположенные слишком близко к горячим солнцам, и другие миры, непригодные для обитания, в эту группу не включены. Эта тысяча миров, приспособленных для поддержания жизни, называется системой, однако в более молодых системах лишь сравнительно небольшое число миров могут быть обитаемыми. Во главе каждой обитаемой планеты стоит Планетарный Принц, а каждая локальная система в качестве центра имеет архитектурный мир и управляется Владыкой Системы.
\vs p015 2:4 \pc \ublistelem{2.}\bibnobreakspace \bibemph{Созвездие.} Сто систем (около 100 000 пригодных для обитания планет) образуют созвездие. Каждое созвездие имеет архитектурный мир\hyp{}центр и возглавляется тремя Сыновьями Ворондадеками, Всевышними. За каждым созвездием также наблюдает Верный Дней, посланник Райской Троицы.
\vs p015 2:5 \pc \ublistelem{3.}\bibnobreakspace \bibemph{Локальная вселенная.} Сто созвездий (около 10 000 000 пригодных для обитания планет) образуют локальную вселенную. Каждая локальная вселенная имеет величественный архитектурный мир\hyp{}центр и управляется одним из равноправных Сыновей\hyp{}Творцов в чине Михаила. Каждой вселенной даровано присутствие Объединяющего Дней, представителя Райской Троицы.
\vs p015 2:6 \pc \ublistelem{4.}\bibnobreakspace \bibemph{Малый сектор.} Сто локальных вселенных (около 1 000 000 000 пригодных для обитания планет) образуют малый сектор правительства сверхвселенной; у него есть чудесный мир\hyp{}центр, откуда его правители, Недавние Дней, управляют делами малого сектора. В центре каждого из малых секторов находятся трое Недавних Дней, три Верховные Личности Троицы.
\vs p015 2:7 \pc \ublistelem{5.}\bibnobreakspace \bibemph{Большой сектор.} Сто малых секторов образуют один большой сектор (100 000 000 000 пригодных для обитания миров). Каждый большой сектор обеспечен великолепным центром и возглавляется тремя Совершенствами Дней, Верховными Личностями Троицы.
\vs p015 2:8 \pc \ublistelem{6.}\bibnobreakspace \bibemph{Сверхвселенная.} Десять больших секторов (около 1 000 000 000 000 пригодных для обитания планет) образуют сверхвселенную. Каждая сверхвселенная обеспечена огромным и великолепным миром\hyp{}центром и управляется тремя Древними Дней.
\vs p015 2:9 \pc \ublistelem{7.}\bibnobreakspace \bibemph{Великая вселенная.} Семь сверхвселенных образуют современную формированную великую вселенную, состоящую приблизительно из семи триллионов пригодных для обитания миров и архитектурных миров и одного миллиарда обитаемых миров Хавоны. Сверхвселенными руководят и управляют косвенно и отражательно из Рая Семь Духов\hyp{}Мастеров. Миллиардом миров Хавоны непосредственно управляют Вечные Дней, причем одна такая Верховная Личность Троицы возглавляет каждый из этих совершенных миров.
\vs p015 2:10 \pc Не считая миров Рая и Хавоны, план вселенской организации предусматривает наличие следующих единиц:
\vs p015 2:11 Сверхвселенные 7
\vs p015 2:12 Большие секторы 70
\vs p015 2:13 Малые секторы 7 000
\vs p015 2:14 Локальные вселенные 700 000
\vs p015 2:15 Созвездия 70 000 000
\vs p015 2:16 Локальные системы 7 000 000 000
\vs p015 2:17 \pc Пригодные для обитания планеты 7 000 000 000 000
\vs p015 2:18 Каждая из семи сверхвселенных составлена приблизительно следующим образом:
\vs p015 2:19 Одна система охватывает примерно 1 000 миров
\vs p015 2:20 Одно созвездие (100 систем) 100 000 миров
\vs p015 2:21 Одна вселенная (100 созвездий) 10 000 000 миров
\vs p015 2:22 Один малый сектор (100 вселенных) 1 000 000 000 миров
\vs p015 2:23 Один большой сектор (100 малых секторов) 100 000 000 000 миров
\vs p015 2:24 Одна сверхвселенная (10 больших секторов) 1 000 000 000 000 миров
\vs p015 2:25 \pc Все подобные оценки в лучшем случае лишь приблизительны, ибо постоянно развиваются новые системы, тогда как другие формирования временно выбывают из материального существования.
\usection{3. Сверхвселенная Орвонтон}
\vs p015 3:1 Практически все звездные миры, видимые невооруженным глазом на Урантии, принадлежат седьмой части великой вселенной, сверхвселенной Орвонтон. Находясь большей частью вне вашей локальной вселенной, огромная звездная система Млечный Путь представляет собой центральное ядро Орвонтона. Это великое скопление солнц, темных островов пространства, двойных звезд, шаровидных скоплений, звездных облаков, спиральных и иных туманностей, а также несметное число отдельных планет образуют похожее на часы удлиненно\hyp{}круглое группирование, в которое входит примерно одна седьмая часть обитаемых эволюционных вселенных.
\vs p015 3:2 Если смотреть с астрономического местоположения Урантии через поперечное сечение близлежащих систем на великий Млечный Путь, то видно, что миры Орвонтона движутся по огромной вытянутой плоскости, ширина которой намного больше толщины, а длина --- намного больше ширины.
\vs p015 3:3 При наблюдении так называемого Млечного Пути обнаруживается сравнительное увеличение плотности звезд Орвонтона, если на небо смотреть в одном направлении, и уменьшение ее с другой стороны; количество звезд и других миров убывает в направлении от главной плоскости нашей материальной сверхвселенной. При благоприятном угле наблюдения сквозь основное тело этой области максимальной плотности вы смотрите в направлении вселенной пребывания и центра всех вещей.
\vs p015 3:4 \pc Из десяти основных секторов Орвонтона восемь так или иначе идентифицированы астрономами Урантии. Два других рассмотреть отдельно затруднительно, поскольку вам приходится наблюдать эти явления изнутри. Но если бы вы могли посмотреть на сверхвселенную Орвонтон из отдаленной точки в пространстве, вы бы сразу заметили десять больших секторов седьмой галактики.
\vs p015 3:5 Центр вращения вашего малого сектора расположен далеко в огромном и плотном звездном облаке Сагиттариус, вокруг которого движется ваша вселенная и все связанные с ней творения, и вы можете наблюдать, как с противоположных сторон громадной субгалактической системы Сагиттариус выходят два огромных потока звездных облаков в виде колоссальных звездных колец.
\vs p015 3:6 Ядро физической системы, к которой принадлежит ваше солнце и связанные с ним планеты, является центром бывшей туманности Андроновер. Эта бывшая спиральная туманность была несколько изменена разрывами гравитации, связанными с событиями, которые сопровождали рождение вашей солнечной системы и были вызваны приближением большой соседней туманности. Это почти столкновение превратило Андроновер в нечто, напоминающее сферическое скопление, но не уничтожило полностью двухстороннее движение солнц и связанных с ними физических групп. Сейчас ваша солнечная система занимает почти центральное положение в одном из рукавов этой искривленной спирали, расположенной приблизительно на середине пути от центра к краю звездного потока.
\vs p015 3:7 \pc Сектор Сагиттариус и все остальные сектора и подразделения Орвонтона вращаются вокруг Уверсы, и определенная путаница взглядов урантийских наблюдателей звезд объясняется иллюзиями и относительными искажениями, вызванными множеством вращательных движений, таких как:
\vs p015 3:8 \ublistelem{1.}\bibnobreakspace Вращение Урантии вокруг своего солнца.
\vs p015 3:9 \ublistelem{2.}\bibnobreakspace Обращение вашей солнечной системы вокруг ядра бывшей туманности Андроновер.
\vs p015 3:10 \ublistelem{3.}\bibnobreakspace Вращение звездного семейства Андроновер и связанных с ним скоплений вокруг составного центра вращения\hyp{}гравитации звездного облака Небадон.
\vs p015 3:11 \ublistelem{4.}\bibnobreakspace Движение локального звездного облака Небадон и связанных с ним творений вокруг центра Сагиттариус их малого сектора.
\vs p015 3:12 \ublistelem{5.}\bibnobreakspace Вращение ста малых секторов, в том числе и Сагиттариуса, вокруг их большого сектора.
\vs p015 3:13 \ublistelem{6.}\bibnobreakspace Кружение десяти больших секторов, так называемых звездных течений, вокруг центра Орвонтона --- Уверсы.
\vs p015 3:14 \ublistelem{7.}\bibnobreakspace Движение Орвонтона и шести связанных с ним сверхвселенных вокруг Рая и Хавоны, направленное против часовой стрелки обращением пространственного уровня сверхвселенной.
\vs p015 3:15 \pc Эти многочисленные движения суть разного свойства. Пути вашей планеты и вашей солнечной системы в пространстве являются генетическими, врожденными. Абсолютное движение против часовой стрелки Орвонтона также генетическое, присущее архитектурным планам главной вселенной. Однако промежуточные движения --- сложного свойства, вызванные отчасти сегментацией материи\hyp{}энергии на сверхвселенные, а отчасти --- разумным и целенаправленным действием Райских организаторов силы.
\vs p015 3:16 \pc Приближаясь к Хавоне, локальные вселенные сближаются друг с другом; контуров становится больше, увеличивается плотность наложения слоев друг на друга. Однако чем дальше от вечного центра, тем меньше и меньше систем, слоев, контуров и вселенных.
\usection{4. Туманности --- предки вселенных}
\vs p015 4:1 Хотя творение и организация вселенной вечно находятся под контролем бесконечных Творцов и их сподвижников, в целом это явление развивается в соответствии с предписанным способом и согласно гравитационным законам силы, энергии и материи. Однако с вселенским силовым зарядом связано нечто таинственное; мы вполне понимаем организацию материальных творений, начиная от ультиматонической стадии, но не до конца уяснили космическое происхождение ультиматонов. Мы уверены в том, что эти наследственные силы имеют Райское происхождение, поскольку они вечно движутся по заполненному пространству, точно совпадающему по форме с гигантскими очертаниями Рая. Этот силовой заряд пространства, предок всякой материализации, хотя и не реагирует на гравитацию Рая, но всегда отвечает на присутствие нижнего Рая, будучи заключенным именно в контуры, входящие в центр нижнего Рая и выходящие из него.
\vs p015 4:2 Райские организаторы силы преобразуют пространственное могущество в изначальную силу и превращают этот предматериальный потенциал в первичные и вторичные энергетические проявления физической реальности. Когда эта энергия достигает уровней реагирования на гравитацию, на сцене появляются управители мощи и их сподвижники по сверхвселенскому управлению и начинают свои непрекращающиеся манипуляции, предназначенные для установления множества контуров мощи и энергетических каналов вселенных со временем и пространством. Таким образом в пространстве появляется материя и подготавливаются условия для начала организации вселенной.
\vs p015 4:3 Эта сегментация энергии представляет собой явление, так и не объясненное физиками Небадона. Главная трудность, с которой они сталкиваются, заключается в относительной недоступности Райских организаторов силы, ибо живые управители мощи, хотя и знают, как обращаться с пространственной энергией, тем не менее не имеют ни малейшего представления о происхождении энергий, которыми они столь искусно и разумно манипулируют.
\vs p015 4:4 \pc Райские организаторы силы являются создателями туманностей; они способны порождать около своего пространственного присутствия огромные циклоны силы, которые, однажды возникнув, не могут быть остановлены или ограничены до тех пор, пока не будут мобилизованы всепронизывающие силы для последующего появления ультиматонических единиц вселенской материи. Так возникают спиральные и другие туманности, материнские диски звезд прямого происхождения и их различные системы. Во внешнем пространстве можно увидеть десять различных форм туманностей, фаз первичной эволюции вселенной, причем эти громадные энергетические диски того же происхождения, что и энергетические диски в семи сверхвселенных.
\vs p015 4:5 \pc Туманности сильно отличаются по размерам, количеству и агрегатной массе порождаемых ими звезд и планет. Солнцеобразующая туманность, расположенная к северу от границ Орвонтона, но находящаяся внутри пространственного уровня сверхвселенной, уже породила приблизительно сорок тысяч солнц, а материнский диск продолжает выбрасывать солнца, большинство которых по размерам превосходит ваше в несколько раз. От некоторых более крупных туманностей внешнего пространства происходит до ста миллионов звезд.
\vs p015 4:6 Туманности непосредственно не связаны ни с одной из административных единиц, таких как малые сектора или локальные вселенные, хотя некоторые локальные вселенные были формированы из продуктов одной туманности. Каждая локальная вселенная заключает в себе ровно одну стотысячную часть общего энергетического заряда сверхвселенной, независимо от своего отношения к туманности, ибо энергия не организуется туманностями, а распределяется повсеместно.
\vs p015 4:7 Не все спиральные туманности порождают солнца. Некоторые сохраняют контроль над множеством отделившихся от них звезд, и их спиральный вид обусловлен тем, что их солнца выходят из рукава туманности единым потоком, но возвращаются разными путями, отчего их легко наблюдать в одной точке, но гораздо сложнее рассматривать, когда они в большом рассеянии возвращаются разными путями, находясь на большом расстоянии от рукава туманности. В настоящее время в Орвонтоне не много активных туманностей, формирующих звезды, хотя Андромеда, находящаяся вне обитаемой сверхвселенной, очень активна. Эта далекая туманность видна невооруженным глазом; глядя на нее, остановитесь и подумайте о том, что свет, который вы видите, покинул эти далекие солнца почти миллион лет назад.
\vs p015 4:8 Галактика Млечный Путь состоит из огромного числа бывших спиральных и других туманностей, причем многие из них до сих пор сохраняют свои изначальные очертания. Но в результате внутренних катастроф и внешнего притяжения многие подверглись такому изменению и такой перестройке, что эти огромные агрегации кажутся гигантскими светящимися массами пылающих солнц, подобными Магелланову облаку. Шарообразный тип звездных скоплений преобладает у внешних границ Орвонтона.
\vs p015 4:9 Огромные звездные облака Орвонтона следует рассматривать как некие агрегации материи, сравнимые с отдельными туманностями, наблюдаемыми в областях пространства, внешних по отношению к галактике Млечный Путь. Многие из так называемых звездных облаков пространства, однако, состоят лишь из газобразного вещества. Энергетический потенциал этих звездных газовых облаков невероятно огромен, и некоторая часть его поглощается близлежащими солнцами и перераспределяется в пространстве в виде солнечных излучений.
\usection{5. Происхождение тел пространства}
\vs p015 5:1 Большая часть массы, содержащейся в солнцах и планетах сверхвселенной, зарождается в дисках туманностей, и лишь весьма небольшая часть массы сверхвселенной формирована прямым действием управителей мощи (как при создании архитектурных миров), хотя в открытом пространстве возникает постоянно изменяющееся количество материи.
\vs p015 5:2 Что же касается происхождения, то большинство солнц, планет и других миров могут быть отнесены к одной из следующих десяти групп.
\vs p015 5:3 \pc \ublistelem{1.}\bibnobreakspace \bibemph{Концентрические кольца сжатия.} Не все туманности спиральные. Многие гигантские туманности вместо того, чтобы превратиться в двойную звезду или развиваться в виде спирали, претерпевают сжатие, путем формирования множества колец. На протяжении длительных периодов такие туманности выглядят как огромное находящееся в центре солнце, окруженное многочисленными похожими на кольца гигантскими облаками материи, опоясывающими его.
\vs p015 5:4 \pc \ublistelem{2.}\bibnobreakspace \bibemph{Вихревые звезды} включают в себя солнца, которые извергаются из огромных сферических источников раскаленных газов. Выбрасываются они не как кольца, а как движущиеся слева направо и справа налево струи. Вихревые звезды также происходят от неспиральных туманностей.
\vs p015 5:5 \pc \ublistelem{3.}\bibnobreakspace \bibemph{Планеты, образовавшиеся в результате гравитационного взрыва.} Когда солнце рождается от спиральной туманности или брусковой туманности, оно нередко извергается на значительное расстояние. Такое солнце в значительной степени газообразно и впоследствии после некоторого охлаждения и сжатия может случайно оказатся около какой\hyp{}нибудь огромной массы материи, гигантского солнца или темного острова пространства. Такое сближение может быть недостаточным для того, чтобы вызвать столкновение, но все же достаточным для того, чтобы притяжение гравитации большего тела привело к началу приливно\hyp{}отливных катаклизмов на теле меньшей величины, вызывая таким образом серии связанных с приливом\hyp{}отливом сдвигов, возникающих одновременно на противоположных сторонах переживающей катаклизмы звезды. При достижении максимальной величины эти взрывные извержения образуют скопления материи различных размеров, которые могут выйти из зоны гравитации извергающего их солнца и таким образом стабилизироваться на своих собственных орбитах вокруг одного из двух тел, участвующих в этом событии. Позднее более крупные скопления материи объединяются и постепенно притягивают к себе тела меньших размеров. Таким образом образуются многие из твердых планет малых систем. Ваша солнечная система имеет именно такое происхождение.
\vs p015 5:6 \pc \ublistelem{4.}\bibnobreakspace \bibemph{Центробежные планетарные дочери.} Огромные звезды на определенных стадиях развития и при большом ускорении вращения начинают разбрасывать значительное количество вещества, которое впоследствии может соединиться и образовать небольшие миры, продолжающие вращаться вокруг породившего их солнца.
\vs p015 5:7 \pc \ublistelem{5.}\bibnobreakspace \bibemph{Миры с дефицитом гравитации.} Существует критический предел размера отдельных звезд. Когда солнце приближается к этому пределу, оно обречено на распад, если не замедлит скорость своего вращения; происходит расщепление солнца, и рождается новая двойная звезда этого вида. Впоследствии побочным продуктом такого гигантского распада может стать формирование большого числа малых планет.
\vs p015 5:8 \pc \ublistelem{6.}\bibnobreakspace \bibemph{Звезды, образовавшиеся в результате сжатия.} В небольших системах самая крупная из внешних планет иногда притягивает к себе соседние миры, а планеты, расположенные вблизи от солнца, начинают свой последний стремительный путь. В случае вашей солнечной системы подобный исход означал бы, что четыре внутренние планеты будут поглощены Солнцем, тогда как самая крупная планета Юпитер сильно увеличится в результате захвата остальных миров. Такой конец солнечной системы привел бы к созданию двух соседствующих, но неравных солнц, одному из типов двойной звезды. Подобные катастрофы происходят нечасто и, как правило, имеют место на краю звездных агрегаций сверхвселенной.
\vs p015 5:9 \pc \ublistelem{7.}\bibnobreakspace \bibemph{Кумулятивные миры.} Из огромного количества материи, движущейся в пространстве, постепенно могут аккумулироваться небольшие планеты. Они растут благодаря аккреции метеоритов и малым столкновениям. В определенных секторах пространства условия благоприятствуют подобным формам рождения планет. Многие обитаемые миры имеют именно такое происхождение.
\vs p015 5:10 Некоторые из плотных темных островов являются прямым следствием аккреций преобразующейся энергии в пространстве. Другая группа этих темных островов возникла благодаря аккумуляции огромного количества холодной материи, простых обломков и метеоров, движущихся в пространстве. Подобные агрегации материи никогда не были горячими и по своему составу очень похожи на Урантию, отличаясь от нее лишь плотностью.
\vs p015 5:11 \pc \ublistelem{8.}\bibnobreakspace \bibemph{Потухшие солнца.} Некоторые из темных островов пространства являются потухшими одиночными солнцами, испустившими всю пространственную энергию, которой они обладали. Формированные единицы материи приближаются к полной конденсации, фактически к полной консолидации; нужны эпохи, чтобы подобные огромные массы сильно сжатой материи могли вновь зарядиться в контурах пространства и, таким образом, приготовиться к новым циклам действия во вселенной после столкновения или же другого в равной степени возрождающего их к жизни космического события.
\vs p015 5:12 \pc \ublistelem{9.}\bibnobreakspace \bibemph{Миры, образовавшиеся в результате столкновения.} В областях более плотного скопления столкновения не редки. Подобные астрономические перестройки сопровождаются огромными энергетическими изменениями и преобразованиями материи. Особенно столкновения, в которых участвуют потухшие солнца, вызывают огромные флуктуации энергии. Возникающие в результате столкновений обломки часто образуют материальное ядро для последующего создания планетарных тел, пригодных для обитания смертных.
\vs p015 5:13 \pc \ublistelem{10.}\bibnobreakspace \bibemph{Архитектурные миры.} Это построенные по планам и указаниям с некоторой определенной целью миры, такие как Спасоград, центр вашей вселенной, и Уверса, место пребывания правительства нашей сверхвселенной.
\vs p015 5:14 \pc Существует множество других способов формирования звезд и возникновения планет, однако именно указанным способом создано подавляющее большинство звездных систем и планетарных семейств. Для того, чтобы описать все разнообразие звездных метаморфоз и планетарных эволюций, потребуется рассказать почти о ста различных способах формирования звезд и порождения планет. Ваши ученые, изучая звезды, будут наблюдать явления, подтверждающие эту звездную эволюцию, но им нечасто удается обнаружить признаки образования тех небольших темных скоплений материи, которые в дальнейшем будут обитаемыми планетами --- наиболее важными из многочисленных материальных творений.
\usection{6. Миры пространства}
\vs p015 6:1 Независимо от происхождения различные миры пространства можно разделить на следующие основные категории:
\vs p015 6:2 \ublistelem{1.}\bibnobreakspace Солнца --- звезды пространства.
\vs p015 6:3 \ublistelem{2.}\bibnobreakspace Темные острова пространства.
\vs p015 6:4 \ublistelem{3.}\bibnobreakspace Малые тела пространства --- кометы, метеоры и планетезимали.
\vs p015 6:5 \ublistelem{4.}\bibnobreakspace Планеты, в том числе и обитаемые миры.
\vs p015 6:6 \ublistelem{5.}\bibnobreakspace Архитектурные миры --- миры, созданные по указу.
\vs p015 6:7 \pc За исключением архитектурных миров все тела пространства имеют эволюционное происхождение --- эволюционное в том смысле, что они не были порождены указанием Божества, эволюционные в том смысле, что творческие деяния Бога раскрылись пространственно\hyp{}временным методом, благодаря действию многих сотворенных и выявленных разумных существ Божества.
\vs p015 6:8 \pc \bibemph{Солнца.} Это звезды пространства на всех их различных стадиях существования. Некоторые из них представляют собой одиночные развивающиеся пространственные системы; другие являются двойными звездами, сжимающимися или исчезающими планетарными системами. Звезды пространства существуют не менее чем в тысяче различных состояний и стадий. Вам известны солнца, испускающие свет, сопровождаемый теплом; однако есть и такие солнца, которые светят без тепла.
\vs p015 6:9 Триллионы и триллионы лет, в течение которых обычное солнце продолжает излучать тепло и свет, прекрасно подтверждают тот огромный запас энергии, который несет в себе каждая единица материи. Действительную же энергию, хранимую в этих невидимых частицах физической материи, трудно вообразить. Причем вся эта энергия под воздействием огромного теплового давления, и связанной с ним энергетической деятельностью преобладающей внутри пылающих солнц почти полностью выделяется в виде света. Другие условия позволяют этим солнцам преобразовывать и посылать большую часть пространственной энергии, которая приходит к ним по установленным пространственным контурам. Многие разновидности физической энергии и все формы материи притягиваются солнечным генератором и затем перераспределяются им. Таким образом солнца служат в качестве локальных ускорителей циркуляции энергии, действуя как автоматические станции регулирования энергии.
\vs p015 6:10 Сверхвселенная Орвонтон освещается и обогревается более, чем десятью триллионами пылающих солнц. Эти солнца --- звезды вашей наблюдаемой астрономической системы. Более двух триллионов звезд слишком далеки и малы, чтобы быть видимыми с Урантии. Однако в главной вселенной солнц столько же, сколько стаканов воды в океанах вашего мира.
\vs p015 6:11 \pc \bibemph{Темные острова пространства.} Это потухшие солнца и другие большие агрегации материи, лишенные света и тепла. Темные острова иногда обладают огромной массой и оказывают мощное влияние на равновесие во вселенной и на управление энергии. Плотность некоторых из этих больших масс почти невероятна. И эта огромная концентрация массы позволяет темным островам действовать как мощный противовес, прочно удерживающий большие соседствующие с ними системы. Во многих созвездиях они поддерживают гравитационное равновесие; многие физические системы, которые в противном случае быстро погрузились бы в близлежащие солнца и погибли бы, прочно удерживаются гравитационным захватом этих темных островов\hyp{}стражей. Благодаря этой их функции мы можем точно определить их местонахождение. Мы измерили постоянные гравитации светящихся тел и поэтому можем вычислить точные размеры и местоположение темных островов пространства, столь эффективно действующих, дабы надежно удерживать данную систему на ее пути.
\vs p015 6:12 \pc \bibemph{Малые тела пространства.} Метеоры и другие мелкие частицы материи, движущиеся и развивающиеся в пространстве, образуют огромный агрегат энергии и материального вещества.
\vs p015 6:13 Многие кометы являются неустановившимся, беспорядочным порождением солнечных материнских дисков, постепенно попадающим во власть царящего в центре солнца. Существует также множество других видов происхождения комет. Хвост кометы направлен от притягивающего ее к себе тела или солнца вследствие электрической реакции сильно разряженных газов и фактического давления света и других энергий, излучаемых солнцем. Это явление служит одним из убедительных доказательств реальности света и связанных с ним энергий; оно показывает, что у света есть вес. Свет --- это реальная материя, а не просто волны гипотетического эфира.
\vs p015 6:14 \pc \bibemph{Планеты.} Это более крупные скопления материи, движущиеся по орбите вокруг солнца или какого\hyp{}нибудь другого пространственного тела; диапазон их размеров --- от планетезималей до огромных газообразных, жидких и твердых миров. Холодные миры, образовавшиеся в результате скопления парящего в пространстве вещества, тогда более подходят для жизни разумных обитателей, когда эти миры оказываются надлежащим образом связанными с близлежащим солнцем. Потухшие солнца, как правило, непригодны для жизни; они слишком далеки от живого, пылающего солнца и сверх того слишком массивны; гравитация на их поверхности огромна.
\vs p015 6:15 В вашей сверхвселенной менее, чем одна из сорока холодных планет пригодна для обитания существ вашего чина. И, конечно, сверхгорячие солнца и далекие холодные миры не годятся для высших форм жизни. В вашей солнечной системе в настоящее время пригодны для жизни лишь три планеты. Урантия по размеру, плотности и местоположению во многих отношениях идеальна для обитания человека.
\vs p015 6:16 Физические законы энергии в основном универсальны, локальные же влияния во многом связаны с физическими условиями, преобладающими на отдельных планетах и в локальных системах. Для бесчисленных миров пространства характерно почти бесконечное разнообразие сотворенной жизни и других живых проявлений. Однако группе миров, соединенных в данной системе, присуще определенное сходство при том, что существует еще и вселенский паттерн разумной жизни. Между планетарными системами, принадлежащими к одному и тому же физическому контуру и на малом расстоянии следующими друг за другом в бесконечном движении вокруг круга вселенных, имеется физическая связь.
\usection{7. Архитектурные миры}
\vs p015 7:1 Хотя правительство каждой сверхвселенной работает вблизи центра эволюционных вселенных своего сегмента пространства, оно занимает мир, созданный согласно определенным указаниям и населенный уполномоченными на то личностями. Эти миры\hyp{}центры есть миры архитектурные, или пространственные тела, специально созданные с особой целью. Хотя эти миры пользуются светом близлежащих солнц, освещаются и обогреваются они независимо. Каждый из них имеет солнце, дающее свет без тепла (подобно спутникам Рая), и каждый обогревается за счет циркуляции определенных потоков энергии у поверхности мира. Эти миры\hyp{}центры принадлежат к одной из более крупных систем, расположенных близ астрономического центра соответствующих сверхвселенных.
\vs p015 7:2 \pc Время в центрах сверхвселенных стандартизовано. Стандартный день сверхвселенной Орвонтон равен почти тридцати дням урантийского времени, а орвонтонский год --- ста стандартным дням. Этот уверсный год --- стандартный в седьмой сверхвселенной; он на двадцать две минуты короче трех тысяч дней урантийского времени и равен приблизительно восьми и одной пятой ваших лет.
\vs p015 7:3 \pc Миры\hyp{}центры семи сверхвселенных сопричастны природе и величию Рая, своему центральному паттерну совершенства. В действительности все миры\hyp{}центры раеподобны. Это поистине небесные обители, от Иерусема к центральному Острову возрастающие в материальных размерах, моронтийной красоте и духовном великолепии. Причем все спутники этих миров\hyp{}центров также являются архитектурными мирами.
\vs p015 7:4 Различные миры\hyp{}центры обеспечены всеми видами материального и духовного творения. Все виды материальных, моронтийных и духовных существ чувствуют себя как дома в этих вселенских мирах встреч. По мере того, как смертные создания идут по пути восхождения во вселенной, переходя из материальных в духовные миры, они никогда не утрачивают понимания прежних уровней своего бытия и продолжают наслаждаться ими.
\vs p015 7:5 \pc \bibemph{Иерусем,} центр вашей локальной системы Сатания, имеет свои семь миров переходной культуры; каждый из них окружен семью спутниками, среди которых находятся семь миров\hyp{}обителей моронтийного задержания --- первое место посмертного пребывания человека. Так как на Урантии был принят термин «небо», он иногда обозначал эти семь миров\hyp{}обителей, причем первый мир\hyp{}обитель назывался первым небом, второй --- вторым и т.д. вплоть до седьмого.
\vs p015 7:6 \pc \bibemph{Эдентия,} центр вашего созвездия Норлатиадек, имеет семьдесят спутников общественной культуры и воспитания, где идущие по пути восхождения пребывают по завершении иерусемского процесса мобилизации, унификации и реализации личности.
\vs p015 7:7 \pc \bibemph{Спасоград,} столица вашей локальной вселенной Небадон, окружена десятью скоплениями\hyp{}университетами, по сорок девять миров в каждом. Здесь после своей созвездной социализации человек одухотворяется.
\vs p015 7:8 \pc \bibemph{Уминор третий,} центр вашего малого сектора Энса, окружен семью мирами высших физических исследований восходящей жизни.
\vs p015 7:9 \pc \bibemph{Умажор пятый,} центр вашего большого сектора Спландон, окружен семьюдесятью мирами продвинутого интеллектуального воспитания сверхвселенной.
\vs p015 7:10 \pc \bibemph{Уверса,} центр вашей сверхвселенной Орвонтон, непосредственно окруженный семью высшими университетами продвинутого духовного воспитания наделенных волей творений, идущих по пути восхождения. Каждое из этих семи скоплений чудесных миров состоит из семидесяти специализированных миров, содержащих тысячи тысяч хорошо обеспеченных и снабжаемых институтов и организаций, занимающихся вселенским воспитанием и духовной культурой; здесь пилигримы, жившие во времени, переобучаются и переэкзаменуются перед своим долгим полетом к Хавоне. Прибывающих пилигримов, живших во времени, всегда принимают на этих взаимодействующих мирах, но отбывающих выпускников всегда отправляют в Хавону прямо с берегов Уверсы.
\vs p015 7:11 Уверса --- духовный и административный центр приблизительно триллиона обитаемых и пригодных для обитания миров. Слава, величие и совершенство столицы Орвонтона превосходит все чудеса пространственно\hyp{}временных творений.
\vs p015 7:12 \pc Если бы все задуманные локальные вселенные и их составные части были созданы, в семи сверхвселенных было бы чуть менее пятисот миллиардов архитектурных миров.
\usection{8. Регулирование и управление энергией}
\vs p015 8:1 Миры\hyp{}центры сверхвселенных построены таким образом, что они способны действовать как эффективные регуляторы энергии и мощи в своих различных секторах, служа фокальными точками для направления энергии к входящим в ее состав локальным вселенным. Они оказывают мощное влияние на равновесие и управление физическими энергиями, циркулирующими в формированном пространстве.
\vs p015 8:2 Дальнейшие регулятивные функции выполняются сверхвселенскими центрами мощи и физическими контролерами, живыми и полуживыми разумными существами, созданными специально для этой цели. Эти центры мощи и контролеры трудны для понимания; их низшие чины неволевые, они не обладают волей, не выбирают, их действия весьма разумные, но, очевидно, автоматические и присущие их высоко специализированной организации. Центры мощи и физические контролеры сверхвселенных осуществляют управление и частичный контроль над тридцатью энергетическими системами, образующими домен тяготения. Для завершения окружения сверхвселенной контурами физической энергии, управляемыми центрами мощи Уверсы, необходимо немногим более 968 миллионов лет.
\vs p015 8:3 \pc Развивающаяся энергия имеет субстанцию; у нее есть вес, хотя вес всегда относителен и зависит от вращательной скорости, массы и антигравитации. Масса в материи имеет тенденцию замедлять скорость в энергии; и всюду присутствующая скорость энергии представляет собой: исходное дарование скорости минус торможение массой, встреченной в пути, плюс регулятивная функция живых контролеров энергии сверхвселенной и физическое воздействие близлежащих сильно нагретых или сильно заряженных тел.
\vs p015 8:4 Вселенский план поддержания равновесия между материей и энергией обуславливает необходимость вечного создания и уничтожения малых материальных единиц. Вселенские Управители Мощи обладают способностью сжимать и удерживать, либо расширять и освобождать различное количество энергии.
\vs p015 8:5 При достаточной длительности замедляющего влияния гравитация в конечном итоге преобразовала бы всю энергию в материю, если бы не действие двух факторов: во\hyp{}первых, антигравитационных влияний контролеров энергии и, во\hyp{}вторых, тенденции формированной материи к распаду при определенных условиях, существующих на очень горячих звездах, и при определенных специфических условиях, существующих в пространстве близ сильно заряженных энергией холодных тел из сжатой материи.
\vs p015 8:6 Когда масса становится переагрегатированной и угрожает нарушить равновесие энергии, истощить контуры физической мощи и собственная дальнейшая тенденция гравитации к чрезмерной материализации энергии не подавляется столкновением потухших гигантов пространства и происходящим вследствие этого мгновенным рассеиванием кумулятивных скоплений гравитации, тогда вмешиваются физические контролеры. В этих столкновениях огромные массы материи внезапно преобразуются в редчайшую форму энергии, и борьба за вселенское равновесие начинается снова. В конечном итоге более крупные физические системы становятся стабильными, физически устойчивыми и направляются в уравновешенные и установившиеся контуры сверхвселенных. После этого события в подобных установившихся системах столкновения или другие разрушительные катастрофы больше не возникают.
\vs p015 8:7 Во времена избытка энергии возникают возмущения мощи и тепловые флуктуации, сопровождаемые электрическими проявлениями. Во времена недостатка энергии возникают возрастающие тенденции, при которых материя начинает скапливаться, сжиматься и выходить из\hyp{}под контроля в наиболее точно сбалансированных контурах, в результате чего происходят перестройки в виде приливов или столкновений, которые быстро восстанавливают равновесие между циркулирующей энергией и более стабилизированной материей. Предсказывать и иными способами понимать такое вероятное поведение пылающих солнц и темных островов пространства --- одна из задач небесных наблюдателей звезд.
\vs p015 8:8 Мы способны осознавать большинство законов, управляющих равновесием во вселенной, и предсказывать многое из относящегося к ее стабильности. Практически наши прогнозы надежны, однако мы всегда сталкиваемся с определенными силами, которые не полностью подчиняются известным нам законам управления энергией и поведения материи. По мере того, как, двигаясь во вселенных, мы все больше удаляемся от Рая, предсказуемость всех физических явлений уменьшается. Выходя за границы личного управления Райских Правителей, мы все более оказываемся не способны вести расчеты в соответствии с установленными нормами и опытом, приобретенным в связи с наблюдениями физических явлений, происходящих исключительно в соседних с Раем астрономических системах. Даже в мирах семи сверхвселенных мы живем в среде силовых действий и энергетических реакций, наполняющих все наши сферы и простирающихся в едином равновесии во все области внешнего пространства.
\vs p015 8:9 Чем дальше мы идем, тем с большей неизбежностью сталкиваемся с изменчивыми и непредсказуемыми явлениями, которые столь точно характеризуют именно непостижимые действия присутствия Абсолютов и Божеств опыта. Причем эти явления должны свидетельствовать о некотором вселенском сверхконтроле над всеми вещами.
\vs p015 8:10 Кажется, что сейчас сверхвселенная Орвонтон останавливается; внешние вселенные, по\hyp{}видимому, готовятся к ни с чем несравнимой будущей активности; центральная же вселенная Хавона вечно устойчива. Гравитация и отсутствие тепла (холод) формируют и объединяют материю; тепло же и антигравитация разрывают материю и рассеивают энергию. Живые управители мощи и организаторы силы --- вот в чем тайна особого управления и разумного руководства бесконечными метаморфозами создания, разрушения и воссоздания вселенных. Туманности могут рассеиваться, солнца выгорать, системы исчезать, а планеты погибать, вселенные же не останавливаются.
\usection{9. Контуры сверхвселенной}
\vs p015 9:1 Всеобъемлющие контуры Рая действительно наполняют миры семи сверхвселенных. Эти контуры присутствия суть таковы: гравитация личности --- Отца Всего Сущего, духовная гравитация --- Вечного Сына, гравитация разума --- Носителя Объединенных Действий и материальная гравитация --- вечного Острова.
\vs p015 9:2 Помимо всеобъемлющих контуров Рая и кроме действий присутствия Абсолютов и Божеств опыта на пространственном уровне сверхвселенной функционируют всего две категории энергетических контуров или подразделений мощи: это --- контуры сверхвселенной и контуры локальных вселенных.
\vs p015 9:3 \pc Контуры сверхвселенной:
\vs p015 9:4 \ublistelem{1.}\bibnobreakspace Объединяющие контуры разума одного из Семи Райских Духов\hyp{}Мастеров. Такой контур космического разума ограничен одной сверхвселенной.
\vs p015 9:5 \ublistelem{2.}\bibnobreakspace Контур отражательного служения семи Отражательных Духов в каждой сверхвселенной.
\vs p015 9:6 \ublistelem{3.}\bibnobreakspace Тайные контуры Таинственных Помощников, некоторым образом взаимосвязанные и направленные Божеградом к Отцу Всего Сущего в Раю.
\vs p015 9:7 \ublistelem{4.}\bibnobreakspace Контур общения Вечного Сына с его Райскими Сыновьями.
\vs p015 9:8 \ublistelem{5.}\bibnobreakspace Мгновенное присутствие Бесконечного Духа.
\vs p015 9:9 \ublistelem{6.}\bibnobreakspace Передачи Рая, пространственные сообщения Хавоны.
\vs p015 9:10 \ublistelem{7.}\bibnobreakspace Энергетические контуры центров мощи и физических контролеров.
\vs p015 9:11 \pc Контуры локальных вселенных:
\vs p015 9:12 \pc \ublistelem{1.}\bibnobreakspace Дух пришествия Райских Сыновей, Утешитель миров, получивших пришествие. Дух Истины, дух Михаила на Урантии.
\vs p015 9:13 \ublistelem{2.}\bibnobreakspace Контур Божественных Служительниц, Духов\hyp{}Матерей локальных вселенных, Святой Дух вашего мира.
\vs p015 9:14 \ublistelem{3.}\bibnobreakspace Контур служения разума локальной вселенной, в том числе разнообразно действующие присутствия духов\hyp{}помощников разума.
\vs p015 9:15 \pc Когда в локальной вселенной возникает такая духовная гармония, что ее индивидуальные и объединенные контуры становятся неотличимыми от контуров сверхвселенной, когда действительно преобладает такая тождественность действия и единство служения, тогда локальная вселенная немедленно переходит в установленные контуры света и жизни, становясь сразу достойной вступления в духовную конфедерацию усовершенствованного союза сверхтворения. Требования же к вступлению в советы Древних Дней, к членству в конфедерации сверхвселенной таковы:
\vs p015 9:16 \ublistelem{1.}\bibnobreakspace \bibemph{Физическая устойчивость.} Звезды и планеты локальной вселенной должны пребывать в равновесии; периоды непосредственной звездной метаморфозы должны быть завершены. Вселенная должна двигаться по ясному пути; ее орбита должна быть надежно и окончательно установившейся.
\vs p015 9:17 \pc \ublistelem{2.}\bibnobreakspace \bibemph{Духовная верность.} Должно существовать состояние всеобщего признания Владыки Сына Бога, руководящего делами такой локальной вселенной, и преданность ему. Должно возникнуть состояние гармоничного сотрудничества между отдельными планетами, системами и созвездиями всей локальной вселенной.
\vs p015 9:18 \pc Ваша локальная вселенная не считается даже относящейся к устойчивому физическому чину сверхвселенной и в еще меньшей степени --- обладающей членством в признанном духовном семействе сверхправительства. Хотя у Небадона еще нет представительства на Уверсе, нас, входящих в состав правительства сверхвселенной, время от времени направляют в его миры с особыми заданиями, подобно мне, прибывшему на Урантию прямо с Уверсы. В решении сложных проблем мы оказываем всяческую помощь вашим руководителям и правителям; мы хотим видеть вашу вселенную пригодной к полному принятию в объединенные творения семейства сверхвселенной.
\usection{10. Правители сверхвселенных}
\vs p015 10:1 Центры сверхвселенных являются резиденцией высшего духовного правительства пространственно\hyp{}временных областей. Исполнительной ветвью сверхправительства, берущей начало в Советах Троицы, непосредственно руководит один из Семи Духов\hyp{}Мастеров верховного надзора, существ, которые восседают на местах Райской власти и управляют сверхвселенными через Семерых Верховных Исполнителей, находящихся в семи особых мирах Бесконечного Духа, самых отдаленных спутниках Рая.
\vs p015 10:2 Центры сверхвселенной являются местами пребывания Отражательных Духов и Помощников Отражательного Изображения. Из такого срединного положения эти удивительные существа осуществляют свои огромные отражательные операции и таким образом служат центральной вселенной вверху и локальной вселенной внизу.
\vs p015 10:3 \pc Каждую сверхвселенную возглавляет трое Древних Дней, объединенные главные руководители сверхправительства. В своей исполнительной ветви штат правительства сверхвселенной состоит из семи различных групп:
\vs p015 10:4 \ublistelem{1.}\bibnobreakspace Древние Дней.
\vs p015 10:5 \ublistelem{2.}\bibnobreakspace Совершенствователи Мудрости.
\vs p015 10:6 \ublistelem{3.}\bibnobreakspace Божественные Советники.
\vs p015 10:7 \ublistelem{4.}\bibnobreakspace Вселенские Цензоры.
\vs p015 10:8 \ublistelem{5.}\bibnobreakspace Могучие Вестники.
\vs p015 10:9 \ublistelem{6.}\bibnobreakspace Облеченные Высокой Властью.
\vs p015 10:10 \ublistelem{7.}\bibnobreakspace Не Имеющие Имени и Номера.
\vs p015 10:11 \pc Троим Древним Дней непосредственно помогает отряд из миллиарда Совершенствователей Мудрости, с которым связаны три миллиарда Божественных Советников. К каждой администрации сверхвселенной прикреплен миллиард Вселенских Цензоров. Эти три группы являются Равноправными Лицами Троицы, берущими начало непосредственно и божественно в Райской Троице.
\vs p015 10:12 Остальные три чина, Могучие Вестники, Облеченные Высокой Властью и Не Имеющие Имени и Номера --- это прославленные, совершившие восхождение смертные. Первые из принадлежащих к этим чинам поднялись благодаря процессу восхождения и прошли через Хавону во дни Грандфанды. По достижении Рая они были собраны в Отряд Финалитов, объяты Райской Троицей и впоследствии поставлены на божественное служение Древним Дней. Как класс эти три чина известны как Тринитизированные Сыны Достижения, имеющие двойственное происхождение, но ныне служащие Троице. Таким образом, исполнительная ветвь правительства сверхвселенной расширилась, включив в себя прославленных и ставших совершенными детей эволюционных миров.
\vs p015 10:13 Координационный совет сверхвселенной состоит из семи исполнительных групп, названных выше, а также следующих правителей секторов и других региональных руководителей.
\vs p015 10:14 \ublistelem{1.}\bibnobreakspace Совершенства Дней --- правители больших секторов сверхвселенной.
\vs p015 10:15 \ublistelem{2.}\bibnobreakspace Недавние Дней --- руководители малых секторов сверхвселенной.
\vs p015 10:16 \ublistelem{3.}\bibnobreakspace Объединяющие Дней --- Райские советники правителей локальных вселенных.
\vs p015 10:17 \ublistelem{4.}\bibnobreakspace Верные Дней --- Райские советники Всевышних правителей правительств созвездий.
\vs p015 10:18 \ublistelem{5.}\bibnobreakspace Сыны Троицы\hyp{}Учителя, которые могут оказаться на службе в центре сверхвселенной.
\vs p015 10:19 \ublistelem{6.}\bibnobreakspace Вечные Дней, которые могут оказаться присутствующими в центре сверхвселенной.
\vs p015 10:20 \ublistelem{7.}\bibnobreakspace Семь Помощников Отражательного Изображения --- представители семи Отражательных Духов и через них представители Семи Райских Духов\hyp{}Мастеров.
\vs p015 10:21 \pc Помощники Отражательного Изображения действуют также в качестве представителей многочисленных групп существ, которые пользуются влиянием в правительствах сверхвселенной, но в настоящее время по различным причинам не вполне активно проявляют свои индивидуальные способности. В состав этой группы входят: развивающееся сверхвселенское проявление личности Верховного Существа, Неограниченные Руководители Верховного, Ограниченные Наместники Предельного, неназванные отражатели связи Мажестона и сверхличностные духовные представители Вечного Сына.
\vs p015 10:22 \pc В мирах\hyp{}центрах сверхвселенных почти во все времена можно найти представителей всех групп сотворенных существ. Рутинная работа служения в сверхвселенных выполняется могучими секонафимами или другими членами обширного семейства Бесконечного Духа. В работе этих чудесных центров сверхвселенской администрации, управления, служения и распорядительного суждения разумные существа из каждой сферы вселенской жизни сплочены эффективной работой, мудрым руководством, полным любви служением и справедливым суждением.
\vs p015 10:23 В сверхвселенных нет никакого посольского представительства, и они полностью изолированы друг от друга. Они знают об общих делах лишь через информационный центр Рая, который опекают Семь Духов\hyp{}Мастеров. Их правители работают в советах божественной мудрости ради благополучия своих собственных сверхвселенных, независимо от того, что происходит в других частях вселенского творения. Эта изоляция сверхвселенных сохранится до тех пор, пока не будет достигнуто их согласование посредством более полной фактуализации сувернитета личности развивающегося Верховного Существа опыта.
\usection{11. Совещательная ассамблея}
\vs p015 11:1 Существа, представляющие автократию совершенства и демократию эволюции, встречаются лицом к лицу в таких мирах, как Уверса. Исполнительная ветвь сверхправительства берет начало в совершенных мирах; законодательная же происходит от цвета эволюционных вселенных.
\vs p015 11:2 Совещательная ассамблея сверхвселенной находится в мире\hyp{}центре. Этот законодательный или консультативный совет состоит из семи домов, в каждый из которых каждая из локальных вселенных, принитая в совет сверхвселенных, избирает своего представителя. Эти представители избираются высокими советами таких локальных вселенных из числа идущих по пути восхождения пилигримов\hyp{}выпускников Орвонтона, пребывающих на Уверсе и получивших право на перемещение в Хавону. Средний срок служения длится около ста лет стандартного сверхвселенского времени.
\vs p015 11:3 Я никогда не слышал о каком бы то ни было разногласии между распорядителями Орвонтона и ассамблеей Уверсы. За всю историю нашей сверхвселенной не было ни одного случая, чтобы совещательный орган принял рекомендацию, которую исполнительное подразделение сверхправительства не выполнило бы немедленно. Здесь в работе всегда царят совершеннейшая гармония и согласие; все это свидетельствует о том, что эволюционные существа, действительно, могут достигать высот совершенной мудрости, что позволяет им общаться с личностями совершенного происхождения и божественной природы. Присутствие совещательных ассамблей в центре сверхвселенной свидетельствует о мудрости и предвещает окончательный триумф всей огромной эволюционной концепции Отца Всего Сущего и его Вечного Сына.
\usection{12. Верховные суды}
\vs p015 12:1 Когда мы говорим об исполнительной и совещательной ветвях правительства Уверсы, вы по аналогии с определенными формами урантийского гражданского правительства можете догадаться, что у нас должна быть и третья, или судебная, ветвь; это действительно так, но у нее нет своего штата. Наши суды устроены следующим образом: в них в зависимости от природы и тяжести разбираемого дела председательствует Древний Дней, Совершенствователь Мудрости или Божественный Советник. Показания в пользу или против индивидуума, планеты, системы, созвездия или вселенной представляются и истолковываются Цензорами. Защищают детей времени и эволюционных планет Могучие Вестники, официальные наблюдатели сверхвселенского правительства в локальных вселенных и системах. Позицию высшего правительства представляют Облеченные Высокой Властью. И, как правило, приговор формулируется комиссией с переменным числом участников, состоящей из одинакового количества Не Имеющих Имени и Номера и группы всепонимающих, чутких личностей, избранных из совещательной ассамблеи.
\vs p015 12:2 Суды Древних Дней --- это высшие кассационные суды для вынесения духовного решения в отношении всех подвластных им вселенных. Сыны\hyp{}Владыки локальных вселенных обладают верховной властью в своих областях и подчинены сверхправительству лишь постольку, поскольку они добровольно передают дела Древним Дней для обсуждения или вынесения решения за исключением дел, предусматривающих уничтожение наделенных волей творений. Указы суда принимаются в локальных вселенных, приговоры же, предусматривающие уничтожение наделенных волей творений, всегда формулируются и приводятся в исполнение из центра сверхвселенной. Сыны локальных вселенных могут вынести решения о продолжении существования в посмертии смертного человека, но лишь Древние Дней могут выносить подлежащее исполнению решение о вечной жизни и смерти.
\vs p015 12:3 Во всех делах, не требующих суда и предоставления доказательств, решения выносят Древние Дней или их сподвижники, причем эти постановления всегда единогласны. Мы имеем дело с совершенными советами. В этих верховных и превосходных судах не бывает ни разногласий, ни мнений меньшинства.
\vs p015 12:4 За небольшим исключением каждое сверхправительство осуществляет юрисдикцию над всеми вещами и существами в своих владениях. Постановления и решения сверхвселенских властей не подлежат апелляции, так как они выражают совпадающие мнения Древних Дней и того Духа\hyp{}Мастера, который из Рая возглавляет судьбу данной вселенной.
\usection{13. Правительства секторов}
\vs p015 13:1 \bibemph{Большой сектор} представляет собой приблизительно одну десятую часть сверхвселенной и состоит из ста малых секторов, десяти тысяч локальных вселенных или ста миллиардов обитаемых миров. Эти большие секторы управляются тремя Совершенными Дней, Верховными Лицами Троицы.
\vs p015 13:2 Суды Совершенных Дней во многом устроены так же, как суды Древних Дней, но в отличие от них не выносят духовных решений в отношении миров. Работа этих правительств больших секторов в основном связана с интеллектуальным статусом необъятного творения. Большие секторы удерживают, выносят решения, распределяют и табулируют для доклада судам Древних Дней все рутинные или административные вопросы, касающиеся сверхвселенной и не являющиеся непосредственно связанными с духовным управлением миров или с выработкой планов Райских Правителей по восхождению смертных. Штат правительства большого сектора не отличается от штата правительства сверхвселенной.
\vs p015 13:3 Как чудесные спутники Уверсы связаны с вашей окончательной духовной подготовкой к Хавоне, так и семьдясят спутников Умажора пятого предназначены для вашего интеллектуального воспитания и развития. Со всего Орвонтона здесь собираются мудрые существа, которые без устали трудятся, дабы приготовить смертных, живущих во времени, к их дальнейшему продвижению к вечной жизни. Большая часть этой подготовки идущих по пути восхождения смертных проводится в семидесяти мирах обучения.
\vs p015 13:4 \pc Правительства \bibemph{малых секторов} возглавляют трое Недавних Дней. Их руководство главным образом связано с физическим управлением, унификацией, стабилизацией и рутинной координацией руководства локальных вселенных, входящих в состав малого сектора. В каждом малом секторе содержится до ста локальных вселенных, десять тысяч созвездий или около миллиарда обитаемых миров.
\vs p015 13:5 Миры\hyp{}центры малого сектора представляют собой огромные места встреч Мастеров\hyp{}Физических Контролеров. Эти миры\hyp{}центры окружены семью сферами наставления, которые образуют подготовительные школы сверхвселенной и являются центрами подготовки к физическому и административному знанию о вселенной вселенных.
\vs p015 13:6 Администраторы правительств малых секторов находятся в непосредственной юрисдикции правителей больших секторов. Недавние Дней получают все сообщения о наблюдениях и координируют все рекомендации, поступающие в сверхвселенную от Объединяющих Дней, которые выступают как наблюдатели и советники Троицы в мирах\hyp{}центрах локальных вселенных, а также от Верных Дней, которые точно так же прикреплены к советам Всевышних в центрах созвездий. Все подобные сообщения передаются Совершенным Дней в больших секторах для последующей отсылки в суды Древних Дней. Таким образом, режим Троицы простирается от созвездий локальных вселенных до центров сверхвселенной. Центр же локальной системы представителей Троицы не имеет.
\usection{14. Цели семи сверхвселенных}
\vs p015 14:1 Существует семь основных целей, раскрывающихся в эволюции семи сверхвселенных. Каждая из основных целей сверхвселенской эволюции найдет наиболее полное выражение лишь в одной из семи сверхвселенных, поэтому каждая сверхвселенная имеет особое назначение и уникальную природу.
\vs p015 14:2 Седьмая сверхвселенная Орвонтон, к которой принадлежит ваша локальная вселенная, известна главным образом своим огромным и щедрым дарованием милосердного служения смертным миров. Она знаменита тем, что в ней преобладает справедливость, смягченная милосердием, и правит власть, сдерживаемая терпением, которая к тому же легко поступается временем для достижения стабильности в вечности. Орвонтон --- показательная вселенная любви и милосердия.
\vs p015 14:3 Нам, однако, очень трудно описать наше понимание истинной природы эволюционной цели, которая раскрывается в Орвонтоне, но мы предполагаем и чувствуем, как в этом сверхтворении шесть уникальных целей космической эволюции, явленных в шести связанных с Орвонтоном сверхтворениях, взаимоувязаны в единое целое: по этой причине мы иногда и полагали, что развитая и законченная персонализация Верховного Бога в отдаленном будущем будет во всем постигаемом опытном величии его достигнутой к тому времени всемогущей независимой силы править с Уверсы семью совершенными сверхвселенными.
\vs p015 14:4 Орвонтон уникален по природе и индивидуален в предназначении так же, как и каждая из шести связанных с ним сверхвселенных. Очень многое из того, что происходит в Орвонтоне, однако, вам не явлено, причем многие из этих неявленных особенностей жизни Орвонтона должны найти полнейшее выражение в какой\hyp{}либо другой сверхвселенной. Семь целей сверхвселенской эволюции действуют во всех семи сверхвселенных, однако каждое сверхтворение даст полнейшее выражение лишь одной из этих целей. Для большего осознания этих сверхвселенских целей многое из того, что вы не понимаете, должно быть дано в откровении, но и тогда вы познаете лишь немногое. Весь этот рассказ представляет собой лишь беглый взгляд на огромное творение, частью которого являются ваши мир и локальная вселенная.
\vs p015 14:5 \pc Ваш мир называется Урантией, его номер в планетарной группе или системе Сатания --- 606. В настоящее время в этой системе содержится 619 обитаемых миров; еще более двухсот планет благополучно развиваются и в некотором будущем могут стать обитаемыми.
\vs p015 14:6 У Сатании есть мир\hyp{}центр, назваемый Иерусемом; в созвездии Норлатиадек его системный номер --- двадцать четыре. Ваше созвездие Норлатиадек состоит из ста локальных систем и имеет мир\hyp{}центр под названием Эдентия. Номер Норлатиадека во вселенной Небадон --- семьдесят. Локальная вселенная Небадон состоит из ста созвездий и имеет столицу, известную под названием Спасоград. В малом секторе Энса номер вселенной Небадон --- восемьдесят четыре.
\vs p015 14:7 Малый сектор Энса состоит из ста локальных вселенных и имеет столицу, называемую Уминор третий. В большом секторе Спландон номер этого малого сектора --- три. Спландон состоит из ста малых секторов и имеет мир\hyp{}центр, называемый Умажор пятый. Это пятый большой сектор сверхвселенной Орвонтон, седьмого сегмента великой вселенной. Таким образом, вы можете определить место вашей планеты на схеме организации и администрации вселенной вселенных.
\vs p015 14:8 Номер вашего мира Урантия в великой вселенной --- 5,342,482,337,666. Это --- регистрационный номер на Уверсе и в Раю, ваш номер в каталоге обитаемых миров. Я знаю регистрационный номер физического мира, однако размер его так велик, что для ума смертного он не имеет никакого практического значения.
\vs p015 14:9 \pc Ваша планета --- частица огромного космоса; она принадлежит к почти бесконечному семейству обитаемых миров, однако управляется точно так же и с такой же любовью, как если бы была единственным существующим обитаемым миром.
\vsetoff
\vs p015 14:10 [Представлено Вселенским Цензором из Уверсы.]
