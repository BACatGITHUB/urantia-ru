\upaper{10}{Райская Троица}
\author{Вселенский Цензор}
\vs p010 0:1 Райская Троица вечных Божеств помогает избавить Отца от абсолютизма личности. Троица идеально соединяет безграничное выражение бесконечной личной воли Бога с абсолютностью Божества. Вечный Сын и различные Сыны божественного происхождения вместе с Носителем Объединенных Действий и его вселенскими детьми эффективно обеспечивают освобождение Отца от ограничений, которые неотъемлемо присущи первенству, совершенству, неизменности, вечности, универсальности, абсолютности и бесконечности.
\vs p010 0:2 Райская Троица эффективно обеспечивает полное выражение и совершенное откровение вечной природы Божества. Стационарные Сыны Троицы дают также полное и совершенное откровение божественной справедливости. Троица --- это Божественное единство, и это единство вечно опирается на абсолютный фундамент божественного единства трех изначальных и равноправных и сосуществующих личностей: Бога Отца, Бога Сына и Бога Духа.
\vs p010 0:3 \P\ Глядя назад в бесконечное прошлое из нынешней точки на круге вечности, мы можем обнаружить лишь одну неотвратимую неизбежность в делах вселенной, и это Райская Троица. Я заключаю, что Троица была неизбежной. Осмысляя прошлое, настоящее и будущее, я считаю, что больше ничто во всей вселенной вселенных не было неизбежным. Нынешняя главная вселенная, рассматривать ли ее в ретроспективе или в перспективе, немыслима без Троицы. При наличии Райской Троицы мы можем допустить, что все могло бы происходить иначе или даже многими способами, но без Троицы, состоящей из Отца, Сына и Духа, мы не в состоянии понять, как Бесконечный мог достичь тройственной и равноправной персонализации при абсолютной единственности Божества. Никакое другое представление о творении не соответствует уровню Троицы, который содержит полноту абсолютности, присущей Божественному единству, в сочетании с полнотой добровольного освобождения, присущей троичной персонализации Божества.
\usection{1. Самораспределение Первоисточника и Центра}
\vs p010 1:1 По\hyp{}видимому, некогда в вечности Отец пошел по пути полного самораспределения. Самоотверженной, любящей и вызывающей любовь природе Отца Всего Сущего внутренне присуще нечто такое, что побуждает его сохранять за собой только ту власть и только те полномочия, которые он явно считает невозможным кому\hyp{}либо передать или даровать.
\vs p010 1:2 Отец Всего Сущего всегда слагал с себя все те роли, которые можно было даровать какому\hyp{}либо другому Творцу или творению. Он передал своим божественным Сынам и их сподвижникам --- разумным существам всю ту власть и все полномочия, которые можно было передать. Он действительно уступил своим Сынам\hyp{}Владыкам в их соответствующих вселенных все прерогативы исполнительной власти, которые можно было уступить. В делах локальных вселенных он сделал каждого Сына\hyp{}Владыку и Творца таким же совершенным, компетентным и авторитетным, каким является Вечный Сын в изначальной и центральной вселенной. Он раздал, поистине даровал с присущими ему величием и святостью всего себя и все свои атрибуты --- все, от чего он мог отказаться, --- всеми способами, во все времена, повсюду и каждой личности во всех вселенных, кроме той, что является центральным местом его пребывания.
\vs p010 1:3 \P\ Божественная личность не эгоцентрична; самораспределение и разделение личности характеризуют божественное добровольное Я. Творения жаждут общения с другими сотворенными личностями; Творцы испытывают побуждение разделить божественность со своими вселенскими детьми; личность Бесконечного раскрывается в виде Отца Всего Сущего, который разделяет реальность бытия с двумя равными ему личностями: Вечным Сыном и Носителем Объединенных Действий.
\vs p010 1:4 \P\ Наши знания относительно личности и божественных атрибутов Отца всегда будут зависеть от откровений Вечного Сына, ибо когда объединенный акт творения был совершен, когда Третье Лицо Божества обрело личностное существование, реализовав объединенные идеи своих божественных родителей, Отец перестал существовать как неограниченная личность. С возникновением Носителя Объединенных Действий и материализацией центрального ядра творения произошли некоторые вечные изменения. Бог передал себя как абсолютную личность своему Вечному Сыну. Таким образом, Отец дарует «бесконечную личность» своему единородному Сыну и оба они даруют «объединенную личность» своего вечного союза Бесконечному Духу.
\vs p010 1:5 По этим и другим причинам, недоступным пониманию конечного разума, человеку чрезвычайно трудно постичь бесконечную личность Бога как отца иначе, как в том виде, как она открыта всем вселенным в Вечном Сыне и вместе с Сыном повсюду действует в Бесконечном Духе.
\vs p010 1:6 Поскольку Райские Сыны Бога посещают эволюционные миры и иногда даже живут в них в подобии человеческой плоти и поскольку эти пришествия дают возможность смертному человеку действительно узнать что\hyp{}то о природе и характере божественной личности, люди планетарных сфер должны обращаться к пришествиям этих Райских Сынов за достоверной и надежной информацией относительно Отца, Сына и Духа.
\usection{2. Персонализация Божества}
\vs p010 2:1 Посредством тринитизации Отец отделяет от себя ту неограниченно духовную личность, которой является Сын, но, поступая так, он утверждает себя Отцом этого самого Сына и, тем самым, обретает безграничную способность становиться божественным Отцом всевозможных впоследствии создаваемых, выявляющихся или как\hyp{}либо иначе персонализирующихся обладающих волей разумных существ. Как \bibemph{абсолютная и неограниченная личность} Отец может функционировать только как Сын и через его посредство, но как \bibemph{личностный Отец} он продолжает даровать личность различным сонмам обладающих волей разумных существ разных уровней, и он вечно поддерживает личные отношения любви и союза с этой обширной семьей вселенских детей.
\vs p010 2:2 После того, как Отец даровал полноту себя личности своего Сына, и по завершению этого акта дарования из бесконечной мощи и природы, существующей, таким образом, в союзе Отца и Сына, эти вечные партнеры совместно даруют качества и атрибуты, составляющие еще одно существо, подобное им; и эта объединенная личность, Бесконечный Дух, завершает экзистенциальную персонализацию Божества.
\vs p010 2:3 Сын необходим для отцовства Бога. Дух необходим для братства Второго и Третьего Лица. Три лица образуют минимальную социальную группу, но это лишь наименьшее из всех многочисленных оснований для веры в неизбежность Носителя Объединенных Действий.
\vs p010 2:4 \P\ Первоисточник и Центр --- это бесконечная \bibemph{личность\hyp{}отец,} безграничная личность\hyp{}источник. Вечный Сын --- это неограниченная \bibemph{личность\hyp{}абсолют,} то божественное существо, которое выступает на протяжении всего времени и вечности как совершенное откровение личностной природы Бога. Бесконечный Дух --- это \bibemph{объединенная личность,} уникальный личностный результат вечного союза Отца и Сына.
\vs p010 2:5 \P\ Личность Первоисточника и Центра --- это личность бесконечности минус абсолютная личность Вечного Сына. Личность Третьего Источника и Центра --- это сверхсовокупный результат союза освобожденной личности Отца и абсолютной личности Сына.
\vs p010 2:6 \P\ Отец Всего Сущего, Вечный Сын и Бесконечный Дух являются уникальными личностями; ни одна из них не дублирует другую; каждая оригинальна; все они объединены.
\vs p010 2:7 \P\ Один лишь Вечный Сын испытывает полноту божественных отношений личности, сознание одновременно сыновства по отношению к Отцу и отцовства по отношению к Духу и божественного равенства и с Отцом\hyp{}предком, и с Духом\hyp{}сподвижником. Отец познал опыт обретения Сына, который равен ему, но Отец не знает никаких предшественников\hyp{}предков. Вечный Сын имеет опыт сыновства, осознание личностного предка и в то же время сознает себя одним из родителей Бесконечного Духа. Бесконечный Дух осознает свое двоякое личностное происхождение, но не является родителем по отношению ни к одной из равноправных личностей Божества. С появлением Духа экзистенциальный цикл персонализации Божества завершается; первичные личности Третьего Источника и Центра развиваются через опыт, и числом их семь.
\vs p010 2:8 Я происхожу от Райской Троицы. Я знаю Троицу как объединенное Божество; я знаю также, что Отец, Сын и Дух существуют и действуют в своих определенных личных качествах. Я точно знаю не только то, что они действуют лично и сообща, но так же и то, что они согласовывают свои действия, по\hyp{}разному группируясь, так что, в конечном счете, они действуют в семи различных одиночных и совместных ипостасях. И поскольку возможности такой божественной комбинации исчерпываются этими семью вариантами, то реальности вселенной неизбежно проявляются в семи вариантах ценностей, значений и личности.
\usection{3. Три личности Божества}
\vs p010 3:1 Несмотря на то, что есть только одно Божество, существует три несомненных и божественных персонализации Божества. Относительно повода дарования человеку божественных Настройщиков Отец сказал: «Давайте создадим смертного человека по нашему образу и подобию». В разных урантийских текстах неоднократно встречается это упоминание о действиях и деяниях множественного Божества, что ясно свидетельствует об осознании существования и деятельности трех Источников и Центров.
\vs p010 3:2 \P\ Нас учат, что в Троице Сын и Дух имеют одинаковые и равные отношения с Отцом. В вечности и как Божества, несомненно, это так, но во времени и как личности между ними, безусловно обнаруживаются очень разные отношения. Если смотреть из Рая на вселенные, то эти отношения кажутся очень схожими, но если смотреть из сферы пространства, то они представляются довольно разными.
\vs p010 3:3 Божественные Сыны --- это, поистине, «Слово Бога», но дети Духа --- воистину, «Деяние Бога.» Бог говорит через Сына и вместе с Сыном действует через Бесконечный Дух, причем во всей вселенской деятельности Сын и Дух находятся в истинно братских отношениях, действуя как два равных брата, испытывающих восхищение и любовь к уважаемому и божественно почитаемому общему Отцу.
\vs p010 3:4 Отец, Сын и Дух, безусловно, равны по природе, равноправны в существовании, но есть несомненные различия в их вселенской деятельности, и абсолютность каждого лица Божества явно ограничена, когда оно действует в одиночку.
\vs p010 3:5 \P\ До своей добровольной передачи личности, власти и атрибутов, составляющих Сына и Духа, Отец Всего Сущего, видимо, был (философски говоря) неограниченным, абсолютным и бесконечным Божеством. Но без Сына такой теоретический Первоисточник и Центр ни в каком смысле слова не мог считаться \bibemph{Отцом Всего Сущего;} отцовство не может быть реальным без сыновства. С другой стороны, чтобы быть абсолютным в полном смысле, в какой\hyp{}то бесконечно далекий момент Отец должен был существовать один. Но такого одиночного существования не было никогда; и Сын, и Дух равновечны Отцу. Первоисточник и Центр всегда был и всегда будет вечным Отцом Изначального Сына и вместе с сыном --- вечным прародителем Бесконечного Духа.
\vs p010 3:6 Мы замечаем, что Отец сложил с себя все прямые проявления абсолютности, кроме абсолютного отцовства и абсолютной воли. Мы не знаем, является ли воля неотъемлемым атрибутом Отца; можем только заметить, что он \bibemph{не} отказался от воли. Такая бесконечность воли должна была быть вечно присуща Первоисточнику и Центру.
\vs p010 3:7 Даруя абсолютность личности Вечному Сыну, Отец Всего Сущего освобождается от оков абсолютизма личности, но, поступая так, он совершает шаг, который навеки лишает его возможности действовать одному в качестве личности\hyp{}абсолюта. А после окончательной персонализации сосуществующего Божества --- Носителя Объединенных Действий --- возникает критически важная тройственная взаимозависимость трех божественных личностей в отношении тотальности функции Божества в абсолюте.
\vs p010 3:8 Бог --- это Отец\hyp{}Абсолют всех личностей во вселенной вселенных. Отец лично абсолютен в свободе действия, но во вселенных со временем и пространством, которые созданы, создаются и еще будут созданы, Отец не является, очевидно, абсолютным тотальным Божеством, кроме как в составе Райской Троицы.
\vs p010 3:9 \P\ Первоисточник и Центр функционирует за пределами Хавоны в являемых вселенных следующим образом:
\vs p010 3:10 \ublistelem{1.}\bibnobreakspace Как творец --- через Сынов\hyp{}Творцов, своих внуков.
\vs p010 3:11 \ublistelem{2.}\bibnobreakspace Как контролер --- через центр тяготения Рая.
\vs p010 3:12 \ublistelem{3.}\bibnobreakspace Как дух --- через Вечного Сына.
\vs p010 3:13 \ublistelem{4.}\bibnobreakspace Как разум --- через Объединенного Творца.
\vs p010 3:14 \ublistelem{5.}\bibnobreakspace Как Отец --- он поддерживает отеческий контакт со всеми созданиями через свой контур личности.
\vs p010 3:15 \ublistelem{6.}\bibnobreakspace Как личность --- он действует \bibemph{прямо} во всем творении своими особыми фрагментами, в смертном человеке --- Настройщиками Мысли.
\vs p010 3:16 \ublistelem{7.}\bibnobreakspace Как тотальное Божество он функционирует только в Райской Троице.
\vs p010 3:17 \P\ Все эти отказы Отца Всего Сущего от своих полномочий и их передача происходили исключительно добровольно и по его собственной инициативе. Всемогущественный Отец сознательно принимает эти ограничения вселенской власти.
\vs p010 3:18 \P\ Вечный Сын, по\hyp{}видимому, действует заодно с Отцом во всех духовных аспектах, за исключением дарования фрагментов Бога и других предличных действий. Сын не отождествляется впрямую ни с интеллектуальной деятельностью материальных созданий, ни с энергетической деятельностью материальных вселенных. Как абсолют Сын функционирует в качестве личности и только в сфере духовной вселенной.
\vs p010 3:19 \P\ Бесконечный дух поразительно универсален и невероятно разносторонен во всех своих действиях. Он действует в сферах разума, материи и духа. Носитель Объединенных Действий представляет союз Отца и Сына, но он функционирует также и сам по себе. Он не занимается непосредственно физическим тяготением, духовным тяготением или контуром личности, но в большей или меньшей степени участвует во всех прочих видах вселенской деятельности. Бесконечный Дух, по\hyp{}видимому, зависим от трех экзистенциальных и абсолютных контролей тяготения, но при этом осуществляет три сверхконтроля. Это троичное дарование всевозможно используется для того, чтобы превзойти и, вероятно, нейтрализовать манифестации первичных сил и энергий вплоть до сверхпредельных границ абсолютности. В определенных ситуациях этот сверхконтроль абсолютно превосходит даже основные манифестации космической реальности.
\usection{4. Божественное объединение в Троице}
\vs p010 4:1 Из всех абсолютных союзов Райская Троица (первое триединство) уникальна как исключительный союз личностного Божества. Бог выступает в качестве Бога только по отношению к Богу и к тем, кто может осознать Бога, но в качестве абсолютного Божества --- только в Райской Троице и по отношению к вселенской тотальности.
\vs p010 4:2 \P\ Вечное Божество всецело едино; тем не менее, есть три совершенно индивидуализированных лица Божества. Райская Троица делает возможным одновременное выражение всего разнообразия характерных черт и бесконечной власти Первоисточника и Центра и вечно равных ему и всего божественного единства вселенских функций неразделимого Божества.
\vs p010 4:3 Троица --- это союз бесконечных личностей, функционирующий в безличностном качестве, но не в противоречии с их личностями. Это приблизительно так, как если бы отец, сын и внук образовали бы совместное целостное безличностное единство, но подчиненное, тем не менее, воле каждого из них.
\vs p010 4:4 Райская Троица \bibemph{реальна.} Она существует как Божественное единство Отца, Сына и Духа; однако Отец, Сын или Дух по отдельности или же любые два из них могут функционировать во взаимосвязи с этой самой Райской Троицей. Отец, Сын и Дух могут сотрудничать не в составе Троицы, но не как три Божества. Как личности они могут совместно действовать так, как считают нужным, но это не Троица.
\vs p010 4:5 \P\ Всегда помните, что Бесконечный Дух выполняет функцию Носителя Объединенных Действий. И Отец, и Сын действуют в нем, через него и в качестве него. Но тщетно было бы пытаться разъяснить тайну Троицы: трое как один и в одном, а один как двое и действует за двоих.
\vs p010 4:6 \P\ Троица настолько связана с событиями во всей вселенной, что ее надо принимать во внимание при попытках объяснить тотальность любого изолированного космического события или личностного взаимоотношения. Троица функционирует на всех уровнях космоса, а смертный человек ограничен конечным уровнем; поэтому человек должен удовольствоваться конечным представлением о Троице как Троице.
\vs p010 4:7 Как смертные во плоти вы должны рассматривать Троицу в соответствии со своей индивидуальной просвещенностью и сообразно реакциям вашего разума и души. Вы можете лишь в очень малой степени узнать абсолютность Троицы, но по мере восхождения к Раю вы много раз испытаете изумление от все новых откровений и неожиданных открытий верховенства и предельности Троицы, если не ее абсолютности.
\usection{5. Функции Троицы}
\vs p010 5:1 Личные Божества имеют атрибуты, но едва ли можно говорить о том, что Троица имеет атрибуты. Правильнее считать, что этот союз божественных существ имеет такие \bibemph{функции,} как отправление правосудия, отношение тотальности, согласованные действия и космический сверхконтроль. Эти функции активно верховные, предельные и (в границах Божества) абсолютные касательно ко всем живым реальностям, обладающим личностью.
\vs p010 5:2 Функции Райской Троицы --- это не просто сумма очевидного дара божественности Отца и тех специфических атрибутов, которые индивидуально присущи личностному существованию Сына и Духа. Троица --- союз трех Райских Божеств --- имеет результатом эволюцию, выявление и обожествление новых значений, ценностей, сил и возможностей для вселенского откровения, действий и руководства. К живым союзам, человеческим семьям, социальным группам или Райской Троице не применимо простое арифметическое суммирование. Потенциал группы всегда намного выше простой суммы атрибутов составляющих ее индивидуумов.
\vs p010 5:3 \P\ Троица сохраняет уникальное отношение ко всей вселенной прошлого, настоящего и будущего именно как Троица. И функции Троицы лучше всего рассматривать в связи с ее отношением ко вселенной. Эти отношения к любой изолированной ситуации или событию одновременны и могут быть множественными:
\vs p010 5:4 \P\ \ublistelem{1.}\bibnobreakspace \bibemph{Отношение к Конечному.} Максимум самоограничения Троицы --- это ее отношение к конечному. Троица --- это не личность, равно как и Верховное Существо не является исключительной персонализацией Троицы, но Верховный --- это наибольшее приближение к личностно\hyp{}силовому средоточию Троицы, которое может быть понято конечными созданиями. Поэтому Троицу в отношении к конечному иногда называют Троицей Верховенства.
\vs p010 5:5 \P\ \ublistelem{2.}\bibnobreakspace \bibemph{Отношение к Абсонитному.} Райская Троица имеет отношение к тем уровням существования, которые выше конечных, но ниже абсолютных, и это отношение иногда называется Троицей Предельности. Ни Предельный, ни Верховный не представляют в полной мере Райскую Троицу, но в ограниченном смысле, каждый до своего соответствующего уровня, по\hyp{}видимому, представляет Троицу на протяжении доличностных эпох опытно\hyp{}силового развития.
\vs p010 5:6 \P\ \ublistelem{3.}\bibnobreakspace \bibemph{Абсолютное Отношение} Райской Троицы находится в связи с абсолютными существованиями и достигает апогея в действии тотального Божества.
\vs p010 5:7 \P\ Троица Бесконечная подразумевает согласованное действие всех триединых взаимоотношений Первоисточника и Центра --- необожествленных, равно как и обожествленных --- и поэтому очень трудна для понимания личностей. При рассмотрении Троицы как бесконечной не забывайте о семи триединствах; тем самым можно избежать некоторых трудностей понимания и частично разрешить некоторые парадоксы.
\vs p010 5:8 \P\ Но я не располагаю такими словами, которые позволили бы передать ограниченному человеческому разуму полную истину и вечное значение Райской Троицы и природы нескончаемого взаимосоюза трех существ бесконечного совершенства.
\usection{6. Стационарные Сыны Троицы}
\vs p010 6:1 Весь закон берет начало в Первоисточнике и Центре; \bibemph{он есть закон.} Исполнение духовного закона --- прерогатива Второго Источника и Центра. Откровение закона, распространение и истолкование божественных законов --- это функция Третьего Источника и Центра. Применение закона, правосудие относится к компетенции Райской Троицы и осуществляется определенными Сынами Троицы.
\vs p010 6:2 \P\ \bibemph{Правосудие} присуще вселенскому владычеству Райской Троицы, но добродетель, милосердие и истина являются вселенским служением божественных личностей, божественный союз которых составляет Троицу. Правосудие --- это не отношение Отца, Сына или Духа. Правосудие --- это отношение этих исполненных любви, милосердия и служения личностей как Троицы. Никто из Райских Божеств не занимается отправлением правосудия. Правосудие никогда не является личным делом; это всегда коллективная функция.
\vs p010 6:3 \P\ \bibemph{Свидетельства,} основа справедливости (правосудия в гармоническом сочетании с милосердием), предоставляются личностями Третьего Источника и Центра, объединенного представителя Отца и Сына по отношению ко всем сферам и ко всем разумам обладающих интеллектом существ всего творения.
\vs p010 6:4 \P\ \bibemph{Вынесение решения,} окончательное исполнение правосудия в соответствии со свидетельствами, представленными личностями Бесконечного Духа, --- это дело Стационарных Сынов Троицы, существ, получивших природу Троицы --- природу объединенных Отца, Сына и Духа.
\vs p010 6:5 \P\ Эта группа Сынов Троицы включает следующие личности:
\vs p010 6:6 \ublistelem{1.}\bibnobreakspace Тринитизированные Тайны Верховенства.
\vs p010 6:7 \ublistelem{2.}\bibnobreakspace Вечные Дней.
\vs p010 6:8 \ublistelem{3.}\bibnobreakspace Древние Дней.
\vs p010 6:9 \ublistelem{4.}\bibnobreakspace Совершенства Дней.
\vs p010 6:10 \ublistelem{5.}\bibnobreakspace Недавние Дней.
\vs p010 6:11 \ublistelem{6.}\bibnobreakspace Объединяющие Дней.
\vs p010 6:12 \ublistelem{7.}\bibnobreakspace Верные Дней.
\vs p010 6:13 \ublistelem{8.}\bibnobreakspace Совершенствователи Мудрости.
\vs p010 6:14 \ublistelem{9.}\bibnobreakspace Божественные Советники
\vs p010 6:15 \ublistelem{10.}\bibnobreakspace Вселенские Цензоры.
\vs p010 6:16 \P\ Мы --- дети трех Райских Божеств, функционирующих как Троица, ибо я принадлежу к десятому чину этой группы, Вселенским Цензорам. Эти чины не выражают позицию Троицы во всемирном смысле; они выражают эту коллективную позицию только в сфере вынесения подлежащих исполнению суждений --- правосудия. Они были специально созданы Троицей именно для той деятельности, исполнять которую они назначены, и они представляют Троицу только в тех функциях, для которых они персонализованы.
\vs p010 6:17 Древние Дней и их созданные Троицей сподвижники вершат праведный суд высшей справедливости в семи сверхвселенных. В центральной вселенной такие функции существуют только теоретически; там справедливость самоочевидна в совершенстве, и совершенство Хавоны исключает всякую возможность дисгармонии.
\vs p010 6:18 Правосудие --- это коллективная мысль праведности; милосердие --- это ее личностное выражение. Милосердие --- это отношение любви; точность --- это характерная черта действия закона; божественный суд --- это душа справедливости, всегда сообразующаяся с правосудием Троицы, всегда осуществляющая божественную любовь Бога. Справедливое правосудие Троицы и милосердная любовь Отца Всего Сущего, если их полностью осознавать и до конца понимать, совпадают. Но человек не имеет такого полного понимания божественного правосудия. Таким образом, в Троице, как ее может представить себе человек, личности Отца, Сына и Духа направлены на равноправное служение любви и закона в развивающихся с ростом опыта вселенных, cуществующих во времени.
\usection{7. Сверхконтроль Верховенства}
\vs p010 7:1 Первое, Второе и Третье лицо Божества равны между собой и едины. «Господь наш Бог --- един Бог.» В божественной Троице вечных Божеств есть совершенство замысла и единство исполнения. Отец, Сын и Носитель Объединенных Действий истинно и божественно едины. Воистину написано: «Я первый, и я последний, и нет Бога, кроме меня.»
\vs p010 7:2 \P\ С точки зрения смертного на конечном уровне, Райская Троица, как и Верховное Существо, имеет отношение только ко всеобщему --- всей планете, вселенной в целом, сверхвселенной в целом, всей великой вселенной. Тотальное отношение существует потому, что Троица является и совокупностью Божества и по многим другим причинам.
\vs p010 7:3 Верховное Существо --- это нечто меньшее, чем Троица, функционирующая в конечных вселенных, и нечто другое; но в некоторых пределах и в течение нынешней эры неполной силовой персонализации это эволюционное Божество все\hyp{}таки отражает позицию Троицы Верховенства. Отец, Сын и Дух персонально не связаны с Верховным Существом, но в течение нынешнего вселенского века они сотрудничают с ним как Троица. Мы понимаем, что они поддерживают подобные же отношения с Предельным. Мы часто строим догадки относительно того, каковы будут личные отношения между Райскими Божествами и Богом Верховным, когда он, наконец, разовьется, но в действительности мы этого не знаем.
\vs p010 7:4 \P\ Мы не считаем, что сверхконтроль Верховенства полностью предсказуем. Более того, оказывается, что эта непредсказуемость характеризуется некоторой неполнотой развития, несомненно, как результат неполноты Верховного и неполноты конечной реакции на Райскую Троицу.
\vs p010 7:5 Человеческий разум может непосредственно думать о тысяче и одной вещи --- катастрофических физических событиях, ужасных несчастных случаях, потрясающих бедах, мучительных болезнях и всемирных бедствиях --- и задавать вопрос, скоррелированы ли эти «кары божьи» в неведомом маневрировании вероятного функционирования Верховного Существа. Откровенно говоря, мы не знаем; в сущности, нам это точно не известно. Но мы замечаем, что с течением времени все эти различные и более или менее загадочные ситуации \bibemph{всегда} действуют во благо и к прогрессу вселенных. Возможно, обстоятельства существования и все необъяснимые превратности жизни вплетены в полный смысла драгоценный паттерн посредством функции Верховного и сверхконтроля Троицы.
\vs p010 7:6 Как сын Бога, ты можешь заметить личное отношение, исполненное любви, во всех действиях Бога Отца. Но ты не всегда сможешь понять, как много вселенских действий Райской Троицы идут на благо человеческому индивидууму в пространственных эволюционирующих мирах. В ходе вечного движения вперед действия Троицы откроются как полные смысла и исполненные заботы, но они не всегда представляются таковыми созданиям, живущим во времени.
\usection{8. Троица за пределами конечного}
\vs p010 8:1 Многие истины и факты, касающиеся Райской Троицы, могут быть поняты даже частично, только если осознать функцию, выходящую за пределы конечного.
\vs p010 8:2 Было бы нецелесообразно рассматривать функции Троицы Предельности, но можно раскрыть, что Бог Предельный является проявлением Троицы, понимаемым Трансценденталами. Мы склонны верить, что объединение главной вселенной есть выявляющий акт Предельного и, вероятно, отражает некоторые, но не все фазы абсонитного сверхконтроля Райской Троицы. Предельный является ограниченным проявлением Троицы в связи с абсонитным только в том смысле, что Верховный частично представляет Троицу в связи с конечным.
\vs p010 8:3 \P\ Отец Всего Сущего, Вечный Сын и Бесконечный Дух являются, в определенном смысле, составляющими личностями тотального Божества. Их союз в Райской Троице и абсолютная функция Троицы равны функции тотального Божества. И такая завершенность Божества превосходит как конечное, так и абсонитное.
\vs p010 8:4 Хотя никакая отдельная личность из Райской Троицы фактически не обладает всем потенциалом Божества, все три совместно обладают им. Три бесконечные личности --- это, по\hyp{}видимому, минимальное число существ, которое требуется для того, чтобы привести в действие предличный и экзистенциальный потенциал тотального Божества --- Божественного Абсолюта.
\vs p010 8:5 Мы знаем Отца Всего Сущего, Вечного Сына и Бесконечный Дух как \bibemph{личности,} но я не знаю лично Божественный Абсолют. Я люблю Бога Отца и поклоняюсь ему; я уважаю и почитаю Божественный Абсолют.
\vs p010 8:6 \P\ Некогда я обитал во вселенной, где некоторая группа существ учила, что финалиты в вечности в конечном счете должны стать детьми Божественного Абсолюта. Но я не склонен соглашаться с таким объяснением тайны, окутывающей будущее финалитов.
\vs p010 8:7 Отряд Финалитов включает, помимо прочих, тех смертных времени и пространства, которые достигли совершенства во всем, что имеет отношение к воле Бога. Как создания и в пределах способностей создания они полно и истинно знают Бога. Найдя, таким образом, Бога как Отца всех созданий, эти финалиты должны когда\hyp{}то начать искать сверхконечного Отца. Но этот поиск подразумевает понимание абсонитной природы предельных свойств и характера Райского Отца. Вечность покажет, возможно ли достичь этого, но мы убеждены, что даже если финалиты поймут эту предельность божественности, они, вероятно, будут не в состоянии достичь сверхпредельных уровней абсолютного Божества.
\vs p010 8:8 Возможно, что финалиты частично достигнут Божественного Абсолюта, но даже если бы и достигли, все равно в вечности вечностей проблема Вселенского Абсолюта будет продолжать увлекать, интриговать, озадачивать и бросать вызов восходящему и достигающему все более высоких уровней финалиту, ибо мы ощущаем, что по мере того, как материальные вселенные и их духовная администрация продолжают расширяться, пропорционально имеет тенденцию расти и непостижимость космических отношений Вселенского Абсолюта.
\vs p010 8:9 \P\ Лишь бесконечность может раскрыть Отца\hyp{}Бесконечного.
\vs p010 8:10 [Под покровительством Вселенского Цензора, действующего на основе полномочий от Древних Дней постоянно пребывающих на Уверсе.]
