\upaper{41}{Физические аспекты локальной вселенной}
\author{Архангел}
\vs p041 0:1 Характерное пространственное явление, отделяющее каждое локальное творение от всех остальных, есть присутствие Творческого Духа. Весь Небадон, несомненно, заполнен пространственным присутствием Божественной Служительницы Спасограда, причем такое присутствие столь же несомненно заканчивается у внешних границ нашей локальной вселенной. То, что заполнено Духом\hyp{}Матерью нашей локальной вселенной, \bibemph{есть} Небадон; то, что простирается за пределы ее пространственного присутствия, находится вне Небадона и является вненебадонскими областями пространства сверхвселенной Орвонтона --- другими локальными вселенными.
\vs p041 0:2 \pc Хотя по административному устройству великая вселенная четко разделена между правительствами центральной, сверх и локальной вселенных, хотя эти разделения астрономически соответствуют пространственному разграничению Хавоны и семи сверхвселенных, тем не менее, между локальными творениями таких четких физических демаркационных линий не существует. Ясно различимы (для нас) даже большие и малые сектора Орвонтона, но идентифицировать физические границы локальных вселенных не так\hyp{}то просто. Это объясняется тем, что данные локальные творения в административном плане организованы согласно определенным \bibemph{творческим} принципам, определяющим сегментацию общего заряда энергии у сверхвселенной, тогда как их физические компоненты, пространственные сферы --- солнца, темные острова, планеты и т.д. происходят прежде всего от туманностей, а те астрономически возникают в соответствии с определенными \bibemph{предтворческими} (трансцендентными) планами Архитекторов Главной Вселенной.
\vs p041 0:3 В области одной локальной вселенной могут быть заключены одна, несколько --- и даже множество --- туманностей, так же и Небадон был создан из звездно\hyp{}планетарного потомства Андроновера и других туманностей. Миры Небадона происходят от разных туманностей, однако все они обладали определенным минимумом общности пространственного движения, которое разумными усилиями управителей мощи было отрегулировано так, чтобы создать нашу теперешнюю агрегацию пространственных тел, в виде единого целого движущуюся по орбитам сверхвселенной.
\vs p041 0:4 Таково строение локального звездного облака Небадона, которое сегодня движется по все более устойчивой орбите вокруг созвездия Стрельца --- центра того малого сектора Орвонтона, к которому принадлежит наше локальное творение.
\usection{1. Центры мощи Небадона}
\vs p041 1:1 Спиральные и другие туманности --- материнские диски сфер пространства --- порождены Райскими организаторами силы; причем вслед за эволюцией небулярной гравитационной реакции их функции в сверхвселенной передаются центрам мощи и физическим контролерам, которые берут на себя всю ответственность за управление физической эволюцией возникающих поколений звездно\hyp{}планетарного потомства. Это физическое руководство предвселенной Небадона по прибытии нашего Сына\hyp{}Творца было немедленно скоординировано с его планом организации вселенной. Во владении этого Райского Сына Бога Верховные Центры Мощи и Мастера\hyp{}Физические Контролеры сотрудничали с появившимися позднее Руководителями Моронтийной Мощи и с другими, чтобы создать тот огромный комплекс линий связи, контуров энергии и путей мощи, которые прочно соединяют множество пространственных тел Небадона в единую административную единицу.
\vs p041 1:2 К нашей локальной вселенной постоянно приписано сто Верховных Центров Мощи четвертого чина. Эти существа принимают входящие линии мощи из центров Уверсы третьего чина и передают пониженные и модифицированные контуры центрам мощи наших созвездий и систем. Функционируя совместно, эти центры мощи образуют действенную систему управления и стабилизации, обеспечивающую сохранение баланса и распределение энергий, которые без нее были бы флуктуирующими и переменными. Центры мощи, однако, не связаны с преходящими и локальными энергетическими всплесками, такими как солнечные пятна и системные энергетические возмущения; свет и электричество не являются основными энергиями пространства, а лишь вторичными и побочными выражениями.
\vs p041 1:3 Сто локально\hyp{}вселенских центров расположено в Спасограде, где они действуют в самом центре энергии этого мира. Архитектурные миры, такие как Спасоград, Эдентия и Иерусем, освещаются, снабжаются теплом и энергией методами, которые делают их совершенно независимыми от солнц пространства. Эти миры были построены --- созданы по плану --- центрами мощи и физическими контролерами и предназначались для мощного влияния на распределение энергии. Основывая свою деятельность на таких точках сосредоточения управления энергией, центры мощи своим живым присутствием делают физические виды энергии пространства управляемыми и направляемыми по определенным каналам. Причем эти энергетические контуры служат основой всех физически\hyp{}материальных и моронтийно\hyp{}духовных явлений.
\vs p041 1:4 К каждому из первичных подразделений Небадона, состоящих из ста созвездий, приписано десять Верховных Центров Мощи пятого чина. В вашем созвездии Норлатиадеке они расположены не в мире\hyp{}центре, а в центре огромной звездной системы, которая образует физическое ядро созвездия. В Эдентии существует десять связанных между собой механических контролеров и десять франдаланков, которые поддерживают совершенную и постоянную связь с соседними центрами мощи.
\vs p041 1:5 Точно в фокусе гравитации каждой локальной системы установлен один Верховный Центр Мощи шестого чина. В системе же Сатании приписанный к ней центр мощи занимает темный остров пространства, расположенный в астрономическом центре системы. Многие из этих темных островов представляют собой огромные генераторы, которые аккумулируют определенные виды энергии пространства и направляют их в нужное русло, причем эти естественные процессы эффективно используются Центром Мощи Сатании, живая масса которого действует как связь с высшими центрами, направляющими потоки более материализованной мощи к Мастерам\hyp{}Физическим Контролерам на эволюционных планетах пространства.
\usection{2. Физические контролеры Сатании}
\vs p041 2:1 Хотя Мастера\hyp{}Физические Контролеры служат с центрами мощи во всей великой вселенной, тем не менее их функции в локальной системе, такой как Сатания, понять намного проще. Сатания --- это одна из сотни локальных систем, образующих административное устройство созвездия Норлатиадека, ближайшими соседями которого являются системы Сандматия, Ассунтия, Порогия, Сортория, Рантулия и Глантония. Системы Норлатиадека во многом отличаются друг от друга, однако все они являются эволюционными и развивающимися, очень похожими на Сатанию.
\vs p041 2:2 Сама же Сатания состоит из более семи тысяч астрономических групп, или физических систем, некоторые из которых имеют происхождение, сходное с происхождением вашей Солнечной системы. Астрономический центр Сатании --- это огромный темный остров пространства, который вместе с окружающими его мирами расположен недалеко от центра правительства системы.
\vs p041 2:3 \pc За исключением приписанного к Сатании центра мощи руководство всей системой физической энергии Сатании сосредоточено в Иерусеме. Мастер\hyp{}Физический Контролер, находящийся в этом мире\hyp{}центре, действует согласованно с центром мощи системы, служа главой связи инспекторов мощи, имеющих свой центр в Иерусеме, и функционируя во всей локальной системе.
\vs p041 2:4 Канализацией энергии в контуры и ее распределением по определенным каналам руководят пятьсот живых и разумных манипуляторов энергии, рассеянных по всей Сатании. Благодаря этим физическим контролерам руководящие центры мощи осуществляют полный и совершенный контроль над большинством основных видов энергий пространства, в том числе и над излучением горячих светил и темных миров, заряженных энергией. Эта группа живых существ может мобилизовывать, трансформировать, преобразовывать и передавать почти все виды физической энергии формированного пространства и управлять ими.
\vs p041 2:5 Жизни присуща способность к мобилизации и преобразованию вселенской энергии. Вам известно свойство растительной жизни преобразовывать материальную энергию света в многообразие видов растительного царства. Вы кое\hyp{}что знаете о методе, посредством которого эта растительная энергия может быть преобразована в явления животной деятельности, однако вам практически ничего не известно о методе управителей мощи и физических контролеров, наделенных способностью мобилизовывать, трансформировать, концентрировать многочисленные виды энергии пространства и направлять их в нужное русло.
\vs p041 2:6 \pc Эти существа, живущие в энергетических зонах, непосредственно не связаны ни с энергией как составным фактором живых творений, ни с областью физиологической химии. Иногда они бывают связаны с физическими предпосылками жизни, с разработкой тех энергетических систем, которые могут служить физическими носителями живой энергии элементарных материальных организмов. В каком\hyp{}то смысле физические контролеры имеют отношение к предшествующим проявлениям материальной энергии, так же как духи\hyp{}помощники разума связаны с преддуховными функциями материального разума.
\vs p041 2:7 \pc Эти разумные создания, осуществляющие управление мощью и руководство энергией, должны в каждом мире приспосабливать свои методы к физическому строению и архитектуре данной планеты. Они безошибочно используют расчеты и выводы своего штата физиков и других технических советников о локальном влиянии сильно нагретых солнц и других типов сверхзаряженных звезд. При этом во внимание должны приниматься даже холодные и темные гиганты пространства и клубящиеся облака звездной пыли; все эти материальные вещи связаны с практическими проблемами манипулирования энергией.
\vs p041 2:8 Руководство мощью\hyp{}энергией эволюционных обитаемых миров --- обязанность Мастеров\hyp{}Физических Контролеров, однако эти существа не отвечают за все сбои в потоках энергии на Урантии. Существует ряд причин, вызывающих подобные возмущения, некоторые из них выходят за пределы зоны контроля физических хранителей. Урантия расположена на линиях колоссальных энергий, эта небольшая планета находится в контуре чудовищной напряженности, и локальные контролеры, стараясь уравновесить эти линии энергии, иногда прибегают к помощи огромного числа существ своего чина. Когда дело касается физических контуров Сатании, они вполне справляются со своей задачей, но им не всегда удается изолировать мощные потоки Норлатиадека.
\usection{3. Наши звездные соседи}
\vs p041 3:1 В Сатании более двух тысяч сверкающих солнц, излучающих свет и энергию, ваше же солнце --- светило средней яркости. Из тридцати ближайших к вам солнц только три обладают большей яркостью. Вселенские Управители Мощи дают начало особым потокам энергии, текущим между отдельными звездами и системами, которые те образуют. Эти солнечные печи наряду с темными гигантами пространства служат центрам мощи и физическим контролерам промежуточными станциями для эффективной концентрации контуров энергии материальных творений и направления их в нужное русло.
\vs p041 3:2 Материальный состав всех солнц, темных островов, планет, спутников и даже метеоров совершенно одинаков. Средний диаметр этих солнц составляет около миллиона миль, ваше же солнечное светило несколько меньших размеров. Самая большая звезда во вселенной, звездное облако Антарес, в четыреста пятьдесят раз больше вашего солнца в диаметре и в шестьдесят миллионов раз больше в объеме. Однако места для размещения этих огромных солнц хватает с избытком. У них столько же свободного места в пространстве, сколько было бы у дюжины апельсинов, если бы они блуждали внутри Урантии и если бы планета представляла собой полый шар.
\vs p041 3:3 \pc Когда солнца, имеющие слишком большие размеры, выбрасываются из небулярного материнского диска, они делятся или формируют двойные звезды. Все солнца изначально только газообразны, хотя позже могут временно существовать и в полужидком состоянии. Когда ваше солнце достигло этого квазижидкого состояния при давлении, которое превращает материю в сверхгаз, оно было недостаточно большим, чтобы разделиться по экватору, что является одним из видов формирования двойных звезд.
\vs p041 3:4 При размерах, в десять раз меньших, чем у вашего солнца, эти огнедышащие сферы быстро сжимаются, уплотняются и остывают. Когда же их размеры в тридцать раз больше --- а точнее, масса материи, их образующей, в тридцать раз больше, чем у вашего солнца, --- они быстро разделяются на два отдельных тела и либо становятся центрами новых систем, либо, оставаясь в зоне влияния сил взаимной гравитации, вращаются вокруг общего центра, как это присуще одному из типов двойных звезд.
\vs p041 3:5 \pc Самым последним крупным космическим извержением в Орвонтоне был необычный взрыв двойной звезды, свет от которой достиг Урантии в 1572 году нашей эры. Эта вспышка была настолько сильной, что взрыв был отчетливо виден даже днем.
\vs p041 3:6 \pc Не все звезды твердые, однако многие из более старых звезд именно таковы. Некоторые из красноватых, тускло мерцающих звезд в центре своих огромных масс обладают плотностью, о которой можно дать представление, сказав, что на Урантии один кубический дюйм вещества такой звезды весил бы шесть тысяч фунтов. Огромное давление, сопровождающееся истечением тепла и циркулирующей энергии, привело к тому, что орбиты основных материальных единиц все больше и больше сближались друг с другом, так что теперь они почти достигли состояния электронной конденсации. Этот процесс охлаждения и сжатия может продолжаться до предельной и критической точки взрыва при ультиматонной конденсации.
\vs p041 3:7 Большинство солнц\hyp{}гигантов сравнительно молодо; большинство же звезд\hyp{}карликов (но не все) --- старые. Карлики, появившиеся в результате столкновения, могут быть очень молодыми и способны сиять интенсивным белым светом, так никогда и не пройдя через исходную красную стадию молодого свечения. Как очень молодые, так и очень старые звезды обычно излучают красноватое мерцание. Желтый оттенок свидетельствует о средней молодости и приближающейся стадии старости, ярко же белый свет говорит о здоровой и продолжительной зрелой жизни.
\vs p041 3:8 \pc Хотя все юные солнца не проходят через стадию пульсаций (по крайней мере видимую), тем не менее, глядя в пространство, вы можете наблюдать многие из этих наиболее молодых звезд, для завершения цикла гигантских, похожих на дыхательные, расширений и сжатий которых требуется от двух до семи дней. Ваше собственное солнце до сих пор сохраняет все менее и менее заметные следы мощных расширений, происходивших в дни его юности, однако прежний период пульсаций, составлявший трое с половиной суток, удлинился до теперешних циклов солнечных пятен с продолжительностью одиннадцать с половиной лет.
\vs p041 3:9 Природа переменных звезд совершенно разная. У одних двойных звезд приливы, обусловленные быстро изменяющимися расстояниями при движении двух тел по своим орбитам, вызывают и периодические флуктуации света. Эти изменения гравитации приводят к регулярно повторяющимся вспышкам, подобно тому как захват метеоров в результате нарастания вещества и энергии на поверхности приводит к сравнительно внезапным вспышкам света, которые быстро затухают до яркости, свойственной этой звезде. Иногда солнце на линии ослабленного гравитационного сопротивления захватывает целый поток метеоров, порой столкновения вызывают звездные вспышки, однако большинство подобных явлений целиком обусловлено внутренними флуктуациями.
\vs p041 3:10 В одной группе переменных звезд период флуктуаций света прямо пропорционален светимости, и астрономы, зная об этом, используют такие солнца в качестве маяков вселенной или точных опорных точек для дальнейших исследований далеких звездных скоплений. Пользуясь этим методом, можно точнейшим образом измерять межзвездные расстояния, величиной более чем миллион световых лет. Более совершенные методы пространственных измерений и улучшенный телескопический метод когда\hyp{}нибудь позволят полнее раскрыть десять великих подразделений сверхвселенной Орвонтона; по крайней мере, восемь из этих гигантских секторов вы будете наблюдать как огромные и достаточно симметричные звездные скопления.
\usection{4. Плотность солнца}
\vs p041 4:1 Масса вашего солнца, по расчетам ваших физиков составляет около двух октиллионов (2 х 10 в 27\hyp{}й степени) тонн, на самом деле она несколько больше. Сейчас оно находится посередине между самыми плотными и самыми разреженными звездами и имеет плотность, приблизительно в полтора раза большую, чем у воды. Однако ваше солнце ни жидкое, ни твердое --- оно газообразное --- и это правда, хотя трудно объяснить тот факт, как газообразное вещество может достичь и этой, и еще значительно больших плотностей.
\vs p041 4:2 \pc Газообразное, жидкое и твердое состояния определяются атомно\hyp{}молекулярными связями, плотность же есть отношение объема и массы. Плотность прямо пропорциональна количеству массы в пространстве и обратно пропорциональна величине пространства в массе, расстоянию между центральными ядрами вещества и частицами, которые вращаются вокруг этих центров, а также пространству внутри таких материальных частиц.
\vs p041 4:3 \pc Остывающие звезды могут быть газообразными и одновременно чрезвычайно плотными. Вы не знакомы с солнечными \bibemph{сверхгазами,} однако эта и другие необычные формы материи объясняют то, как плотность даже нетвердого солнца может быть равной плотности железа, почти такой же, как плотность Урантии, --- и все равно находиться в сильно нагретом газообразном состоянии, продолжая функционировать как солнце. Атомы в этих плотных сверхгазах исключительно малы и содержат мало электронов. Такие солнца израсходовали и значительную часть своих свободных ультиматонных запасов энергии.
\vs p041 4:4 Одно из ближайших к вам солнц, которое начало свою жизнь приблизительно с той же массой, что и у вашего солнца, сейчас сжалось почти до размеров Урантии, став в сорок тысяч раз плотнее, чем ваше солнце. Вес этого холодно\hyp{}горячего газообразно\hyp{}твердого вещества составляет около тонны на кубический дюйм. И все же это солнце излучает тусклый, красноватый свет --- старческое мерцание умирающего повелителя света.
\vs p041 4:5 Большинство солнц, однако, не такие плотные. Плотность одного из ближайших к вам в точности равна плотности вашей атмосферы на уровне моря. Если бы вы оказались внутри этого солнца, то не смогли бы ничего различить. Причем если бы температура позволяла и вы могли бы проникнуть внутрь большинства солнц, мерцающих в ночном небе, то увидели бы столько же вещества, сколько и в воздухе ваших земных жилищ.
\vs p041 4:6 Массивное солнце Велунтия, одно из самых больших в Орвонтоне, имеет плотность, составляющую всего одну тысячную от плотности атмосферы Урантии. Если бы по составу оно напоминало вашу атмосферу и не было сверхнагретым, то оно представляло бы собой такой вакуум, что люди в нем или на нем быстро бы задохнулись.
\vs p041 4:7 Еще один из гигантов Орвонтона сейчас имеет температуру поверхности чуть меньше трех тысяч градусов. Его диаметр более трехсот миллионов миль --- это размер, достаточный для того, чтобы разместить и ваше солнце, и современную орбиту земли. И все же при столь огромных размерах (более чем в сорок миллионов раз превышающих размеры вашего солнца) его масса больше массы вашего солнца всего раз в тридцать. Эти огромные солнца обладают настолько протяженно\hyp{}размытыми границами, что практически соприкасаются друг с другом.
\usection{5. Солнечная радиация}
\vs p041 5:1 То, что солнца пространства обладают не очень высокой плотностью, подтверждается непрерывными потоками исходящих от них световых энергий. Очень большая плотность за счет непрозрачности удерживала бы свет до тех пор, пока давление энергии света не достигло бы точки взрыва. Внутри солнца действует огромное давление света или газа, что вынуждает его выбрасывать потоки энергии, способные проникать в пространство на миллионы миль, обеспечивая энергией, освещая и обогревая далекие планеты. Пятнадцать футов поверхности с плотностью Урантии практически предотвращали бы истечение всех рентгеновских лучей и световых энергий от солнца до тех пор, пока возрастающее внутреннее давление аккумулирующихся энергий, вызванное распадом атомов, не преодолело бы гравитацию огромным, направленным вовне взрывом.
\vs p041 5:2 В присутствии движущихся газов свет при высоких температурах удерживаемый светонепроницаемыми стенками, становится взрывчатым. Свет реален. При расценках на энергию и электричество, принятых в вашем мире, стоимость солнечного света составила бы миллион долларов за фунт.
\vs p041 5:3 Внутри ваше солнце представляет собой огромный генератор рентгеновских лучей. Солнца поддерживаются изнутри непрекращающейся бомбардировкой этих мощных излучений.
\vs p041 5:4 Более полумиллиона лет потребуется возбужденному рентгеновским лучом электрону, чтобы преодолеть расстояние от самого центра среднего по размерам солнца до солнечной поверхности; отсюда электрон начинает свое космическое путешествие, чтобы, возможно, обогреть обитаемую планету, быть захваченным метеором, принять участие в рождении атома, быть притянутым сильно заряженным темным островом пространства или завершить свой полет в пространстве окончательным стремительным погружением в поверхность какого\hyp{}нибудь солнца, похожего на то, из которого он вылетел.
\vs p041 5:5 Рентгеновские лучи, испускаемые изнутри солнца, заряжают сильно разогретые и возбужденные электроны энергией, достаточной для того, чтобы пронести их через пространство, преодолеть чудовищное притяжение материи, встречающейся на их пути, и вопреки отклоняющим воздействиям гравитации устремиться к дальним мирам далеких систем. Огромной энергии, необходимой для преодоления силы гравитации солнца, достаточно для того, чтобы солнечный луч с неуменьшающейся скоростью двигался до тех пор, пока не столкнется со значительными массами вещества; после этого он быстро преобразуется в тепло, высвобождая при этом и другие виды энергии.
\vs p041 5:6 \pc Энергия, будь то в виде света или в иных формах, перемещается в пространстве по прямой. Подлинные частицы материального бытия пересекают пространство, подобно выстрелу из ружья. Они движутся по прямой и непрерывной линии, или процессии, за исключением тех случаев, когда на них действуют более мощные силы или когда они, как и всегда, подчиняются силам притяжения линейной гравитации, присущей материальной массе, и круговой гравитации Райского Острова.
\vs p041 5:7 \pc Может показаться, что солнечная энергия движется волнами, однако это обусловлено действием сосуществующих, но различных влияний. Данная форма организованной энергии распространяется не волнами, а по прямым линиям. Наличие второй или третьей формы силовой энергии может привести к тому, что наблюдаемый поток будет \bibemph{казаться} движущимся волнообразно, так же, как и при проливном дожде, сопровождающемся сильным ветром, вода иногда кажется падающей полосами или низвергающейся волнами. Дождевые капли падают по прямой линии непрерывного потока, однако действие ветра таково, что создает видимость полос воды и волн дождевых капель.
\vs p041 5:8 Воздействие некоторой вторичной или иных пока неоткрытых энергий, присутствующих в областях пространства вашей локальной вселенной, таково, что кажется, что излучения солнечного света осуществляют определенный волновой процесс и расчленены на бесконечно малые порции определенной длины и веса. Причем при практическом рассмотрении именно это и происходит. Вы едва ли можете надеяться на то, что достигнете лучшего понимания природы света до тех пор, пока не обретете более ясного представления о взаимодействии и взаимосвязи различных пространственных сил и солнечных энергий, действующих в областях Небадона. Ваше нынешнее замешательство вызвано также неполным осознанием вами этой проблемы, ибо она включает в себя взаимосвязанные действия личностного и безличностного управления главной вселенной --- присутствие, поступки и согласование Носителя Объединенных Действий и Неограниченного Абсолюта.
\usection{6. Кальций --- путешественник в пространстве}
\vs p041 6:1 При расшифровке спектральных явлений следует помнить, что пространство не пусто; что свет, преодолевая пространство, иногда несколько модифицируется различными формами энергии и материи, которые циркулируют во всяком формированном пространстве. Некоторые из указывающих на наличие неизвестного вещества линий, которые появляются в спектрах вашего солнца, вызваны модификациями хорошо известных элементов, движущихся в пространстве в раздробленной форме; это атомы\hyp{}жертвы жестоких схваток в битвах солнечных элементов. Пространство наполнено этими блуждающими изгоями, особенно натрием и кальцием.
\vs p041 6:2 Во всем Орвонтоне кальций фактически является главным элементом материи, пронизывающей пространство. По всей нашей сверхвселенной разбросаны мельчайшие твердые частицы. Камень является буквально основным строительным материалом планет и миров пространства. Космическое облако, огромное пространственное покрывало, состоит большей частью из модифицированных атомов кальция. Атом кальция один из преобладающих и самых устойчивых элементов. Он не только выдерживает солнечную ионизацию --- расщепление --- но и сохраняет ассоциативную идентичность даже после бомбардировки разрушительными рентгеновскими лучами и раздробляющего воздействия высоких солнечных температур. Кальций обладает специфическими особенностями и продолжительностью существования большей, чем у всех наиболее распространенных форм материи.
\vs p041 6:3 \pc Как и предполагали ваши физики, эти разрозненные частицы солнечного кальция буквально едут на световых лучах на разные расстояния, и, таким образом, чрезвычайно облегчается их широкое распространение по всему пространству. Атом натрия при определенных изменениях тоже способен передвигаться вместе с энергией или светом. Способность же к этому кальция тем более замечательна, что масса этого элемента почти в два раза больше массы натрия. Локальное проникновение кальция в пространство обусловлено тем фактом, что кальций покидает солнечную фотосферу в модифицированной форме и буквально летит на испускаемых солнцем лучах. Из всех солнечных элементов кальцию, несмотря на его сравнительную громоздкость --- в его состав входит двадцать вращающихся электронов, --- бегство изнутри солнца в миры пространства удается с наибольшим успехом. Этим же объясняется, почему на солнце существует кальциевый слой --- газообразная каменная поверхность толщиной в шесть тысяч миль, --- и это несмотря на то, что под ним находятся девятнадцать более легких элементов и множество более тяжелых.
\vs p041 6:4 При солнечных температурах кальций --- активный и изменчивый элемент. На двух внешних электронных оболочках, расположенных близко друг к другу, атом кальция имеет два подвижных и слабо удерживаемых электрона. В начальные моменты кальций утрачивает свой внешний электрон, после чего начинает виртуозно жонглировать девятнадцатым электроном между девятнадцатой и двадцатой электронной орбитой. Перебрасывая этот девятнадцатый электрон с его собственной орбиты на орбиту его утраченного спутника и обратно двадцать пять тысяч раз в секунду, возбужденный атом кальция способен частично преодолевать гравитацию и, таким образом, успешно двигаться на исходящих лучах света и энергии, солнечных лучах, навстречу свободе и приключению. Этот атом кальция движется наружу попеременными рывками, продвигаясь вперед, захватывает и отпускает солнечный луч около двадцати пяти тысяч раз в секунду. Именно поэтому камень и является главным компонентом миров пространства. Кальций --- самый искушенный беглец из солнечной тюрьмы.
\vs p041 6:5 О подвижности акробата\hyp{}электрона кальция свидетельствует тот факт, что, когда силы солнечных температур и рентгеновских лучей выталкивают его на более высокую орбиту, он остается там всего около одной миллионной доли секунды; однако прежде чем электрическая гравитационная мощь атомного ядра вернет его на старую орбиту, он способен совершить миллион оборотов вокруг центра атома.
\vs p041 6:6 \pc Ваше солнце рассталось с огромным количеством своего кальция, утратив его гигантские запасы во времена своих конвульсивных извержений в процессе формирования солнечной системы. Много солнечного кальция сейчас находится и во внешней коре солнца.
\vs p041 6:7 \pc Следует помнить, что спектральный анализ показывает лишь состав солнечной поверхности. Например: на солнечных спектрах имеется множество линий железа, но железо не является основным элементом солнца. Это явление почти полностью объясняется современной температурой солнечной поверхности; составляя немногим менее 6000 градусов, эта температура весьма благоприятна для фиксации спектра железа.
\usection{7. Источники солнечной энергии}
\vs p041 7:1 Внутренняя температура многих солнц (и даже вашего) намного больше, чем принято считать. Внутри солнца практически не существует целых атомов; они в большей или меньшей степени расщеплены интенсивной бомбардировкой рентгеновских лучей, что характерно для столь высоких температур. Независимо от того, какие материальные элементы могут присутствовать во внешних слоях солнца, элементы, находящиеся внутри, вследствие разлагающего действия разрушительных рентгеновских лучей, очень похожи. Рентгеновский луч --- это великий нивелировщик существования атомов.
\vs p041 7:2 Температура поверхности вашего солнца составляет почти 6000 градусов, однако по мере углубления внутрь солнца она быстро возрастает, достигая в центральных областях невероятной отметки почти 35 000 000 градусов. (Все температуры указаны в принятой у вас шкале Фаренгейта.).
\vs p041 7:3 \pc Все эти явления свидетельствуют об огромном расходе энергии, источники же солнечной энергии, перечисленные в порядке своей важности, таковы:
\vs p041 7:4 \ublistelem{1.}\bibnobreakspace Распад атомов и, в конечном счете, электронов.
\vs p041 7:5 \ublistelem{2.}\bibnobreakspace Преобразование элементов, сопровождающееся излучением радиоактивных видов энергий.
\vs p041 7:6 \ublistelem{3.}\bibnobreakspace Аккумулирование и передача определенных всемирных пространственных энергий.
\vs p041 7:7 \ublistelem{4.}\bibnobreakspace Материя и метеоры пространства, непрерывно падающие на пылающие солнца.
\vs p041 7:8 \ublistelem{5.}\bibnobreakspace Солнечное сжатие; охлаждение и последующее сжатие солнца выделяют энергии и тепла иногда больше, нежели материя пространства.
\vs p041 7:9 \ublistelem{6.}\bibnobreakspace Воздействие гравитации при высоких температурах преобразует некоторую законтуренную мощь в излучаемые виды энергии.
\vs p041 7:10 \ublistelem{7.}\bibnobreakspace Повторный захват света и иной материи, которые, покинув солнце, притягиваются к нему вновь вместе с другими видами энергии, имеющими внесолнечное происхождение.
\vs p041 7:11 \pc Существует регулирующий поверхностный слой горячих газов (с температурой иногда в миллионы градусов), который окружает солнца и действует так, что стабилизирует потери тепла и иными способами предотвращает опасные флуктуации теплового рассеяния. Во время активной жизни солнца внутренняя температура в 35 000 000 градусов остается почти постоянной, совершенно независимо от нарастающего падения температуры внешней.
\vs p041 7:12 \pc Вы можете попытаться мысленно представить себе температуру в 35 000 000 градусов, сочетаемую с определенными давлениями гравитации, как электронную точку кипения. При таком давлении и такой температуре все атомы разрушаются и расщепляются на свои электронные и другие, породившие их, компоненты; расщеплены могут быть даже электроны и другие ассоциации ультиматонов, но сами ультиматоны солнцами разрушены быть не могут.
\vs p041 7:13 Эти солнечные температуры разгоняют до огромных скоростей ультиматоны и электроны или, по крайней мере, те из последних, что продолжают существовать при этих условиях. Что значит высокая температура в плане ускорения ультиматонной и электронной деятельности, вы поймете, когда сделаете паузу в своих размышлениях и подумаете о том, что в одной капле простой воды содержится более миллиарда триллионов атомов. Это равно энергии более ста лошадиных сил, воздействующей непрерывно в течение двух лет. Тепла, выделяемого сейчас солнцем солнечной системы каждую секунду, достаточно для того, чтобы за секунду довести до кипения всю воду во всех океанах Урантии.
\vs p041 7:14 \pc Светить вечно могут лишь те солнца, которые действуют в прямых каналах основных потоков энергии вселенной. Такие солнечные печи пылают бесконечно, будучи способными восполнять потери вещества за счет поглощения пространственной силы и аналогичной циркулирующей энергии. Звезды же, сильно удаленные от этих основных каналов подзарядки, обречены на истощение энергии --- постепенное охлаждение и выгорание.
\vs p041 7:15 К таким умершим или умирающим солнцам молодость может вернуться благодаря удару, вызванному столкновением, или подзарядке островами пространства, обладающими несветовой энергией, либо благодаря притягиванию гравитацией соседних солнц и систем меньшего размера. Большинство умерших солнц возрождается благодаря этим и другим эволюционным методам. Те же, которые в итоге не подвергнутся подзарядке, будут обречены на разрушение в результате взрыва массы, когда гравитационная конденсация достигнет критического уровня ультиматонной конденсации давления энергии. Подобные исчезающие солнца, таким образом, становятся энергией редчайшей формы, изумительно приспособленной для подзарядки энергией других благоприятнее расположенных солнц.
\usection{8. Реакции солнечной энергии}
\vs p041 8:1 На солнцах, расположенных в пространственно\hyp{}энергетических каналах, солнечная энергия выделяется вследствие различных сложных цепных ядерных реакций, наиболее распространенной из которых является водородно\hyp{}углеродно\hyp{}гелиевая. В этой метаморфозе углерод выступает в качестве катализатора энергии, так как в результате процесса превращения водорода в гелий он не претерпевает никаких изменений. При определенных условиях, возникающих при высокой температуре, водород проникает в ядра углерода. А поскольку углерод не может удерживать более четырех таких протонов, то при достижении этого состояния насыщения он начинает испускать протоны с той же скоростью, с какой к нему прибывают новые. В этой реакции прибывающие частицы водорода истекают в виде атома гелия.
\vs p041 8:2 \pc Уменьшение содержания водорода увеличивает светимость солнца. У солнц, обреченных на выгорание, пик светимости достигается в точке исчерпания водорода. После этой точки яркость поддерживается проистекающим процессом сжатия под действием гравитации. В конце концов такая звезда станет так называемым белым карликом, сильно уплотненной сферой.
\vs p041 8:3 \pc В больших солнцах --- небольших сферических туманностях --- при исчерпании водорода и начале гравитационного сжатия в случае, если такое тело недостаточно светонепроницаемо, чтобы сохранять внутреннее давление, необходимое для удержания внешних областей газа, возникает внезапный коллапс. Гравитационно\hyp{}энергетические изменения порождают огромное количество крошечных частиц, лишенных электронного потенциала, и такие частицы легко покидают внутреннюю зону солнца, что через несколько дней приводит к коллапсу гиганта. Именно такое истечение этих «частиц\hyp{}беглецов» и вызвало около пятидесяти тысяч лет назад коллапс гигантской новой в туманности Андромеды. Это огромное звездное тело сколлапсировало за сорок минут урантийского времени.
\vs p041 8:4 Как правило, сильнейшие сгустки материи продолжают существовать около остатков остывающего солнца в виде протяженных облаков небулярных газов. Этим и объясняется происхождение многих типов диффузных туманностей, таких как Крабовидная, которая возникла около девятисот лет назад и до сих пор демонстрирует материнскую сферу в виде одиночной звезды недалеко от центра диффузной массы, составляющей туманность.
\usection{9. Стабильность солнца}
\vs p041 9:1 Солнца более крупных размеров сохраняют такое гравитационное воздействие на свои электроны, что свет покидает их лишь с помощью мощных рентгеновских лучей. Эти лучи\hyp{}помощники пронизывают все пространство и участвуют в поддержании основных ультиматонных ассоциаций энергии. Огромные потери энергии в первое время существования солнца после достижения им максимальной температуры --- выше 35 000 000 градусов вызваны не столько излучением света, сколько истечением ультиматонов. Во времена юности солнца эти ультиматонные энергии вырываются в пространство, чтобы осуществить электронную и энергетическую материализацию в виде настоящего энергетического взрыва.
\vs p041 9:2 \pc Атомы и электроны подчиняются воздействию гравитации. Ультиматоны же локальной гравитации взаимодействию, вызванному притяжением материи, \bibemph{не} подчиняются, но они полностью подвластны абсолютной или Райской гравитации, тенденции, движению всеобщего и вечного цикла вселенной вселенных. Энергия ультиматонов не подчиняется линейному или непосредственному гравитационному притяжению близрасположенных или отдаленных материальных масс, но она постоянно движется точно по контуру огромного эллипса необъятного творения.
\vs p041 9:3 \pc Центр вашего солнца ежегодно излучает почти сто миллиардов тонн реального вещества, тогда как солнца\hyp{}гиганты в начале своего роста, то есть в свой первый миллиард лет, теряют вещество со скоростью поистине чудовищной. Жизнь солнца становится стабильной по достижении максимума внутренней температуры и с началом освобождения субатоматомных энергий. Именно в этой критической точке и начинаются конвульсивные пульсации крупных солнц.
\vs p041 9:4 Стабильность солнца целиком зависит от равновесия между действием сил гравитации и температурой --- когда огромные давления уравновешиваются невообразимыми температурами. Упругость внутреннего солнечного газа удерживает лежащие сверху различные виды вещества, причем когда гравитация и тепло находятся в равновесии, вес внешних слоев вещества в точности равен температурному давлению находящихся ниже внутренних газов. Во многих наиболее молодых звездах продолжающаяся гравитационная конденсация приводит к постоянному повышению внутренних температур, а с увеличением нагрева внутренней части солнца внутренее давление рентгеновских лучей ветров сверхгаза становится настолько большим, что центробежные силы солнца начинают выбрасывать свои верхние слои в пространство, компенсируя этим дисбаланс между гравитацией и теплотой.
\vs p041 9:5 Ваше солнце уже давно достигло относительного равновесия между своими циклами расширения и сжатия, теми возмущениями, которые вызывают гигантские пульсации многих молодых звезд. Сейчас ваше солнце заканчивает свой шестимиллиардный год. В настоящее время оно действует необычайно экономично. С теперешней интенсивностью оно будет светить еще более двадцати пяти миллиардов лет. Возможно, оно переживет еще и сравнительно эффективный период заката, по продолжительности равный сумме периодов своей молодости и стабильной деятельности.
\usection{10. Происхождение обитаемых миров}
\vs p041 10:1 Некоторые из переменных звезд в состоянии (или почти в состоянии) максимальной пульсации находятся в процессе порождения дочерних систем, многие из которых в конце концов будут очень похожи на ваше солнце и вращающиеся вокруг него планеты. Ваше солнце находилось точно в таком же состоянии мощной пульсации, когда к нему приблизилась массивная система Ангона и внешняя поверхность солнца стала извергать настоящие потоки --- непрерывные пласты материи. Это продолжалось с нарастающей силой вплоть до максимального сближения, когда были превышены пределы сцепления солнечной материи и произошел отрыв ее огромного куска, из которого и возникла солнечная система. При аналогичных обстоятельствах максимальное приближение притягивающего тела иногда способно оторвать целые планеты, даже четверть или треть массы солнца. Эти огромные сгустки формируют некоторые своеобразные окруженные облаком типы миров, сферы, во многом похожие на Юпитер и Сатурн.
\vs p041 10:2 Большинство солнечных систем, однако, имеет происхождение, совершенно отличное от вашего, причем именно так дело обстоит даже для тех систем, которые были созданы гравитационно\hyp{}приливным методом. Но независимо от любого метода построения мира, гравитация всегда создает творения типа солнечной системы; то есть находящееся в центре солнце или темный остров с планетами, спутниками, спутниками спутников и метеорами.
\vs p041 10:3 \pc Физические свойства отдельных миров во многом определяются видом их происхождения, астрономической ситуацией и физическим окружением. Возраст, размеры, частота вращения и скорость движения в пространстве также являются определяющими факторами. Как для миров, возникших в результате сжатия газа, так и для миров, возникших в результате скопления твердых тел, характерно наличие гор и на ранних этапах их существования (в случае, если они не очень малы) --- воды и воздуха. Миры, возникшие в результате расплавления и раскола, а также столкновения, иногда лишены длинных горных цепей.
\vs p041 10:4 На ранних этапах существования всех этих новых миров часто происходят землетрясения, и для всех из них характерны сильнейшие физические возмущения; что особенно присуще сферам, возникшим в результате сжатия газа, мирам, рожденным из небулярных колец, оставшихся после более ранней конденсации и сжатия некоторых отдельных солнц. Планеты, подобно Урантии имеющие двоякое происхождение, переживают менее бурную и неспокойную молодость. Но все равно, ваш мир пережил раннюю стадию мощных потрясений, для которых характерны вулканы, землетрясения, наводнения и ужасные бури.
\vs p041 10:5 \pc Урантия находится в сравнительной изоляции на краю Сатании; причем ваша солнечная система (если не считать еще одной системы) является самой далекой от Иерусема, в то время как сама Сатания находится перед самой отдаленной системой Норлатиадека; сейчас это созвездие пересекает внешнюю границу Небадона. Вы поистине были одним из самых незначительных творений до тех пор, пока пришествие Михаила не возвело вашу планету на почетное место, сделав ее объектом огромного всеобщего интереса. Иногда последнее становится первым, а ничтожнейшее величайшим.
\vsetoff
\vs p041 10:6 [Представлено Архангелом в сотрудничестве с Главой Центров Мощи Небадона.]
