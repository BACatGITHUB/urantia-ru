\upaper{146}{Первое путешествие с проповедями по Галилее}
\vs p146 0:1 Первое путешествие с публичными проповедями по Галилее началось в воскресенье 18 января 28 года н.э. и продолжалось около двух месяцев, закончившись возвращением в Капернаум 17 марта. В этом путешествии Иисус и двенадцать апостолов, которым помогали бывшие апостолы Иоанна, проповедовали в Риммоне, Иотапате, Раме, Зевулоне, Ироне, Гишале, Хоразине, Мадоне, Кане, Наине и в Ендоре. В этих городах они останавливались и проповедовали, тогда как во многих других более мелких городах проповедовали евангелие царства, проходя через них.
\vs p146 0:2 Это был первый раз, когда Иисус позволил своим соратникам свободно проповедовать. В этом путешествии он предостерегал их только в трех случаях, советуя держаться подальше от Назарета и быть осторожными, проходя через Капернаум и Тиверию. Апостолы были очень довольны тем, что могут наконец свободно проповедовать и безвозбранно учить, и они с головой окунулись в работу, с великой горячностью и радостью проповедуя евангелие, служа больным и крестя верующих.
\usection{1. Проповедь в Риммоне}
\vs p146 1:1 Небольшой городок Риммон когда\hyp{}то был посвящен культу вавилонского бога воздуха Раммана. Многие из древних вавилонских, а позднее и зороастрийских учений были по\hyp{}прежнему частью верований жителей Риммона; поэтому Иисус и двадцать четыре апостола посвятили большую часть своего времени тому, что разъясняли, чем отличаются эти старые верования и новое евангелие царства. Здесь Петр прочитал одну из великих проповедей начала своего служения «Аарон и золотой телец».
\vs p146 1:2 Хотя многие из жителей Риммона стали верующими в учения Иисуса, они доставляли массу хлопот своим братьям в последующие годы. Весьма трудно за короткое время одной жизни обратить поклоняющихся природе в совершенное братство поклонения духовному идеалу.
\vs p146 1:3 \P\ Многие из лучших вавилонских и персидских представлений о свете и тьме, добре и зле, времени и вечности позднее вошли в доктрины так называемого христианства, и включение их в состав христианского учения способствовало тому, что оно было быстрее принято народами Ближнего Востока. Подобным же образом усвоение многих теорий Платона об идеальном духе или невидимых патернах всего видимого и материального, позднее введенных в еврейскую теологию Филоном, сделало христианское учение Павла более понятным и приемлемым для западных греков.
\vs p146 1:4 \P\ Именно в Риммоне Тодан впервые услышал евангелие царства; позднее он перенес эту весть в Месопотамию и еще дальше. Он был среди первых, кто проповедовал благую весть людям, жившим за Евфратом.
\usection{2. В Иотапате}
\vs p146 2:1 Хотя простой люд Иотапаты с радостью слушал Иисуса и его апостолов и многие приняли евангелие царства, но особо примечательным посещение Иотопаты сделала беседа Иисуса с двадцатью четырьмя апостолами, которая состоялась во второй вечер их пребывания в этом небольшом городке. Учения Учителя о молитве, благодарении и богопочитании привели ум Нафанаила в смущение, и в ответ на его вопрос Иисус произнес пространную речь, где дал дальнейшее разъяснение своего учения. В кратком изложении на современном языке эту беседу можно представить, выделив следующие положения:
\vs p146 2:2 \P\ \ublistelem{1.}\bibnobreakspace Сознательная и устойчивая склонность к пороку в человеческом сердце постепенно разрушает молитвенную связь души человека с духовными контурами, соединяющими человека с его Творцом. Конечно, Бог слышит, о чем просит его дитя, однако, когда человеческое сердце умышленно и настойчиво таит в себе порочные помыслы, постепенно происходит утрата личного общения между земным чадом и его небесным Отцом.
\vs p146 2:3 \P\ \ublistelem{2.}\bibnobreakspace Молитва, несовместимая с известными и установленными законами Бога, для Райских Божеств --- скверна. Если человек не слушает Богов, говорящих со своим творением по законам духа, ума и материи, то сам факт подобного умышленного и сознательного пренебрежения со стороны создания отвратит слух духовных личностей от личных прошений таких непослушных закону и непокорных смертных. Иисус процитировал из Пророка Захарии: «Но они не хотели внимать, отвратились и уши свои отяготили, чтобы не слушать. И сердце свое окаменили, чтобы не слышать закона моего и слов, которые я посылал духом моим чрез пророков; поэтому за злые мысли их великий гнев пал на их грешные головы. И было: они взывали о помощи, но ни одно ухо не слышало». Затем Иисус процитировал высказывание мудреца, который сказал: «Кто отклоняет ухо свое от слушания божественного закона, того и молитва --- скверна».
\vs p146 2:4 \P\ \ublistelem{3.}\bibnobreakspace Смертные, открывая со своей стороны канал связи между Богом и человеком, получают немедленный доступ к вечному потоку божественного служения созданиям миров. Когда человек слышит, как дух Бога говорит в человеческом сердце, тогда и Бог одновременно слышит молитву этого человека. Даже прощение греха происходит точно так же безотлогательно. Отец Небесный простил вас еще прежде, чем вы подумали просить его, однако такое прощение недоступно вашему личному религиозному опыту до тех пор, пока вы не простите ваших собратьев\hyp{}людей. \bibemph{Фактически} Божие прощение отнюдь не обусловлено вашим прощением своих собратьев, однако в вашем \bibemph{опыте} оно обусловлено именно этим. И эта особенность синхронии божественного и человеческого прощения была, таким образом, признана и связана воедино в молитве, которой Иисус учил апостолов.
\vs p146 2:5 \P\ \ublistelem{4.}\bibnobreakspace Во вселенной существует основной закон справедливости, обойти который милосердие бессильно. Бескорыстно раздаваемые великолепия Рая не могут быть восприняты глубоко эгоистичным созданием миров времени и пространства. Даже бесконечная любовь Бога не может насильно дать спасения вечной жизни любому смертному созданию, которое этого не желает. Милосердие обладает великой широтой дарования, но все же существуют установления правосудия, которые даже любовь, соединенная с милосердием, не могут полностью отменить. И снова Иисус процитировал из еврейских писаний: «Я звала вас и вы не послушали; простирала руку мою, и не было внимающего. Вы отвергли все мои советы, и обличений моих не приняли, и за этот мятежный дух неизбежно станет так, что вы будете звать меня, и не получите ответа. Отвергнув путь жизни, вы будете усердно меня искать во времена страдания вашего, но меня не найдете».
\vs p146 2:6 \P\ \ublistelem{5.}\bibnobreakspace Те, кто хотят получать милосердие, должны проявлять милосердие; не судите, да не судимы будете. С каким духом других судить будете, с таким и вас будут судить. Милосердие не может полностью отменить вселенскую справедливость. И в конце концов правдой окажутся слова: «Кто отклоняет ухо свое от вопля бедного, тот и сам будет вопить о помощи, и никто не услышит его». Искренность любой молитвы --- вот залог того, что она будет услышана; духовная мудрость и совместимость с законами вселенной всякого прошения определяют время, способ и полноту ответа. Мудрый отец не дает \bibemph{буквального} ответа на глупые прошения своих несведущих и неопытных детей, хотя дети могут испытывать большое наслаждение и истинное душевное удовлетворение, обращаясь с подобными абсурдными просьбами.
\vs p146 2:7 \P\ \ublistelem{6.}\bibnobreakspace Когда вы полностью посвятите себя исполнению воли Отца Небесного, тогда на все ваши прошения будет ответ, ибо молитвы ваши будут полностью согласованы с волей Отца; воля же Отца всегда явлена в его необъятной вселенной. Чего истинный сын желает и что бесконечный Отец велит --- СОВЕРШАЕТСЯ. Такая молитва не может остаться без ответа, и ни на какое другое прошение не может быть полного ответа.
\vs p146 2:8 \P\ \ublistelem{7.}\bibnobreakspace Мольба праведного есть проявление веры дитя Бога, которая открывает двери в сокровищницу доброты, истины и милосердия Отца, и благие дары эти давно ожидают, когда сын приблизится к ним и сам возьмет их. Молитва не изменяет божественного отношения к человеку, но изменяет отношение человека к неизменному Отцу. Прямой путь к божественному уху дает \bibemph{побудительная причина} молитвы, а не социальный, экономический и внешнепоказной религиозный статус молящегося.
\vs p146 2:9 \P\ \ublistelem{8.}\bibnobreakspace Молитва не может быть использована для того, чтобы избежать временных трудностей либо преодолеть пространственные преграды. Молитва отнюдь не служит для самовозвеличивания или получения несправедливого преимущества перед собратьями. Глубоко эгоистичная душа не может молиться в истинном смысле этого слова. Иисус сказал: «Да будет высшее утешение твое в сущности Бога, и он обязательно исполнит искреннее желание сердца твоего». «Предай Господу путь твой; уповай на него; и он совершит». «Ибо Господь слышит мольбу нуждающегося, и призрит на молитву беспомощных».
\vs p146 2:10 \P\ \ublistelem{9.}\bibnobreakspace «Я пришел от Отца; итак, если сомневаетесь, о чем просить Бога, просите во имя мое, и я представлю прошение ваше согласно вашим подлинным нуждам и желаниям и в соответствие с волей Отца моего». Берегитесь великой опасности стать эгоцентричными в молитвах ваших. Избегайте молиться много о себе; молитесь же более о духовном развитии братьев ваших. Избегайте молиться о материальном; молитесь же в духе и об изобилии даров духа.
\vs p146 2:11 \P\ \ublistelem{10.}\bibnobreakspace Молясь о больных и страждущих, не ждите, что прошения ваши заменят полное любви и разумное служение потребностям страждущих сих. Молитесь о благополучии ваших семей, друзей и собратьев; особенно же молитесь о проклинающих вас и с любовью просите за преследующих вас. «Когда же молиться, я не скажу. Лишь дух, пребывающий в вас, может подвигнуть вас к произнесению прошений, которые выражают ваши внутренние отношения с Отцом духов».
\vs p146 2:12 \P\ \ublistelem{11.}\bibnobreakspace Многие обращаются к молитве только в беде. Подобная практика безрассудна и обманчива. Да, вы поступаете правильно, молясь в состоянии тревоги, но вы также не должны забывать говорить с Отцом вашим, как сыновья, даже тогда, когда на душе у вас все хорошо. Да будут настоящие прошения ваши всегда тайны. И да не услышат люди ваши личные молитвы. Молитвы благодарения подходят для группы молящихся, однако молитва души --- сугубо личное дело. Существует лишь одна форма молитвы, подходящая для всех детей Бога, и она такова: «И все же да будет воля твоя».
\vs p146 2:13 \P\ \ublistelem{12.}\bibnobreakspace Все верующие в сие евангелие должны искренне молиться о распространении царства небесного. Из всех молитв еврейского писания Иисус наиболее одобрительно отозвался о прошении псалмопевца: «Сердце чистое сотвори во мне, Боже, и дух правый обнови внутри меня. Очисти меня от грехов тайных и удержи раба твоего от самонадеянного проступка». Иисус весьма пространно истолковал отношение молитвы к небрежной и оскорбительной речи и процитировал: «Положи, Господи, охрану устам моим, и огради двери уст моих». «Человеческий язык, --- сказал Иисус, --- есть орган, который немногие люди могут укротить, но дух внутри человека может обратить этот непокорный орган в добрый глас терпимости и вдохновляющего служителя милосердия».
\vs p146 2:14 \P\ \ublistelem{13.}\bibnobreakspace Иисус учил, что молитва о божественном водительстве по стезе земной жизни --- следующая по важности после прошения о знании воли Отца. В действительности это --- молитва о божественной мудрости. Иисус никогда не учил, что человеческое знание и особое искусство можно обрести с помощью молитвы. Но он учил, что молитва --- это посредник, в повышении способности человека воспринимать присутствие божественного духа. Уча своих соратников молиться в духе и истине, Иисус объяснял, что он говорит об искренней молитве, молитве, соответствующей степени просвещенности молящегося, о молитве, идущей из глубины сердца, о молитве разумной, честной и безустанной.
\vs p146 2:15 \P\ \ublistelem{14.}\bibnobreakspace Иисус предостерег своих последователей от мысли о том, что молитвы их будут более действенны благодаря витиеватому многословию, красноречивым выражениям, посту, покаянию или жертвам. Однако он призвал верующих в него использовать молитву как средство, ведущее через благодарение к истинному богопочитанию. Иисус высказал сожаление о том, что в молитвах и почитании его последователей так мало духа благодарения. В связи с этим он процитировал Писание и сказал: «Благо есть благодарить Господа и петь хвалу имени Всевышнего, возвещать милость его всякое утро и верность всякую ночь, ибо Бог возрадовал меня делами своими. Во всем воздавать благодарение буду согласно воле Божией».
\vs p146 2:16 \P\ \ublistelem{15.}\bibnobreakspace Затем Иисус сказал: «Не беспокойтесь постоянно и чрезмерно об обыденных нуждах ваших. Не тревожьтесь о проблемах вашего земного бытия, но в молитве и прошении с духом благодарения открывайте нужды ваши перед Отцом вашим Небесным». Затем он процитировал Писание: «Я буду славить имя Бога в песне и буду превозносить его в славословии. И будет это благоугоднее Господу, нежели в жертву принесенный вол или телец с копытами».
\vs p146 2:17 \P\ \ublistelem{16.}\bibnobreakspace Иисус учил своих последователей, что, произнеся свои молитвы к Отцу, они должны какое\hyp{}то время оставаться в состоянии молчаливой восприимчивости, дабы предоставить пребывающему в них духу более полную возможность говорить со слушающей душой. Дух Отца наилучшим образом говорит с человеком, когда ум человеческий пребывает в состоянии подлинного богопочитания. Мы почитаем Бога благодаря помощи пребывающего в нас духа Отца и озарению ума человеческого через служение истины. Почитание, учил Иисус, еще больше уподобляет человека существу, которого тот почитает. Богопочитание --- это опыт преображения, посредством которого конечное постепенно приближается к присутствию Бесконечного и в конце концов достигает его.
\vs p146 2:18 \P\ И еще много других истин об общении человека с Богом поведал Иисус своим апостолам, но многие из них не смогли полностью понять его учение.
\usection{3. Остановка в Раме}
\vs p146 3:1 В Раме у Иисуса была памятная беседа с пожилым греческим философом, который учил, что науки и философии достаточно, чтобы удовлетворить потребности, возникающие с человеческим опытом. Иисус терпеливо и с симпатией слушал этого греческого учителя, признавая истинность многого из того, что тот сказал, но, когда философ закончил свою речь, отметил, что в своем описании человеческого бытия тот не сумел объяснить «откуда, зачем и куда», и при этом добавил: «Мы начинаем там, где кончаешь ты. Религия --- это откровение душе человека, имеющей дело с духовными сущностями, которые одним только умом никогда нельзя ни понять, ни до конца осознать. Интеллектуальные искания могут открыть жизненные факты, тогда как евангелие царства раскрывает \bibemph{истины} бытия. Ты говорил о материальных тенях истины; не послушаешь ли теперь и ты мой рассказ о вечных и духовных сущностях, которые отбрасывают сии преходящие тени материальных фактов смертного бытия?» Более часа Иисус учил этого грека спасительным истинам евангелия царства. Пожилой философ оказался восприимчивым к взглядам Учителя и, будучи в глубине души искренним и честным быстро поверил евангелию спасения.
\vs p146 3:2 Апостолы были несколько смущены тем, как открыто соглашался Иисус со многими утверждениями грека, но Иисус позже сказал им наедине: «Дети мои, не удивляйтесь тому, что я был терпим к философии грека. Истинная и подлинная внутренняя уверенность нисколько не боится открытого анализа, а истина не негодует против честной критики. Не забывайте никогда, что нетерпимость суть маска, скрывающая тайные сомнения в истинности своих убеждений. Никого и никогда не смущают взгляды ближнего, если сам он совершенно уверен в истине того, во что искренне верит. Смелость --- это уверенность в бескомпромиссной честности относительно того, во что, по открытому признанию человека, он верит. Искренние люди не боятся критического изучения своих истинных убеждений и благородных идеалов».
\vs p146 3:3 \P\ Во второй вечер пребывания в Раме Фома задал Иисусу такой вопрос: «Учитель, как впервые уверовавшему в твое учение действительно узнать об истинности сего евангелия царства и по\hyp{}настоящему убедиться в ней?»
\vs p146 3:4 И Иисус сказал Фоме: «Ваша уверенность в том, что вы вошли в семью Отца в царстве и что вы вместе с детьми царства будете жить вечно, --- целиком и полностью вопрос личного опыта, вопрос веры в слово истины. Духовная уверенность есть эквивалент вашего личного религиозного опыта в вечных реальностях божественной истины и, выражаясь иначе, равна вашему разумному пониманию реальностей истины плюс ваша духовная вера и минус ваши честные сомнения.
\vs p146 3:5 По природе своей Сын наделен жизнью Отца. Будучи наделены живым духом Отца, вы, таким образом, сыновья Бога. Вы продолжаете жить после вашей жизни в материальном мире плоти, потому что вы отождествлены с живым духом Отца, даром вечной жизни. Правда, многие имели сию жизнь до того, как я пришел от Отца, и еще больше получили дух этот, потому что поверили моему слову; но я объявляю: когда я вернусь к Отцу, он пошлет свой дух в сердца всех людей.
\vs p146 3:6 Хотя вы не можете наблюдать действие божественного духа в умах ваших, существует практический способ определить степень, в которой вы передали контроль над вашими душевными силами учению и водительству сего пребывающего в вас духа Отца Небесного, и это --- сила любви вашей к вашим собратьям. Сей дух Отца вкушает от любви Отца и, овладевая человеком, неизменно ведет к божественному почитанию и сердечному отношению к своим собратьям. Сначала вы верили, что вы --- сыны Бога, поскольку мое учение помогло вам глубже осознать внутреннее водительство пребывающего в вас присутствия Отца; но вскоре Дух Истины изольется на всякую плоть и будет жить среди людей и учить всех людей так же, как я ныне живу среди вас и говорю вам слова истины. И сей Дух Истины, говоря от имени духовных даров душ ваших, поможет вам узнать, что вы --- сыновья Божии. Он будет неизменно свидетельствовать вместе с пребывающим в вас присутствием Отца, вашим духом, который будет тогда пребывать во всех людях, как ныне пребывает он в некоторых из них, говоря вам, что вы в действительности сыновья Бога.
\vs p146 3:7 Каждое земное дитя, следующее водительству этого духа, в конце концов узнает волю Бога, и кто подчиняется воле Отца моего, тот будет жить вечно. Путь из земной жизни в вечное бытие не был указан вам, но такой путь есть, был всегда, и я пришел, дабы возродить этот путь. Кто входит в царство, тот уже имеет жизнь вечную и никогда не умрет. Однако многое из этого вы поймете лучше, когда я вернусь к Отцу и вы сможете посмотреть на ваш сегодняшний опыт ретроспективно».
\vs p146 3:8 И все, кто слышал эти благие слова, сильно ободрились. Еврейские учения о вечной жизни праведных были запутанными и неясными, и, услышав эти ясные и точные слова об уверенности в вечном спасении всех истинно верующих, последователи Иисуса обрели новые силы и новое вдохновение.
\vs p146 3:9 \P\ Апостолы продолжали проповедовать и крестить верующих и, как и прежде, посещали отдельные дома, утешали упавших духом и служили больным и страждущим. Апостольское сообщество расширилось, ибо у каждого из апостолов Иисуса теперь был соподвижник из числа апостолов Иоанна; Авенир был соратником Андрея; и план этот по преимуществу осуществлялся до тех пор, пока они не пришли в Иерусалим на следующий праздник Пасхи.
\vs p146 3:10 \P\ Особое наставление, данное Иисусом во время их пребывания в Зевулоне, главным образом было связано с дальнейшими обсуждениями взаимных обязательств в царстве и включало в себя учение, предназначенное разъяснить разницу между личным религиозным опытом и дружескими отношениями, присущими общественно\hyp{}религиозным обязательствам. Это был один из немногочисленных случаев, когда Учитель обсуждал социальные аспекты религии. На протяжении всей своей земной жизни Иисус почти не давал своим последователям наставлений относительно обобществления религии.
\vs p146 3:11 В Зевулоне жили люди смешанной расы, которых едва ли можно было отнести как к евреям, так и к неевреям, и несмотря на то, что они слышали об исцелении больных в Капернауме, лишь немногие из них действительно уверовали в Иисуса.
\usection{4. Евангелие в Ироне}
\vs p146 4:1 В Ироне, как и во многих еще более мелких городах Галилеи и Иудеи, была синагога, и на начальных этапах своего служения Иисус имел обыкновение произносить речи в этих синагогах в день субботы. Иногда он произносил речь на утренней службе, а Петр или один из других апостолов проповедовал в послеполуденный час. Иисус и апостолы также часто учили и проповедовали на вечерних собраниях в синагоге в будние дни. Хотя религиозные лидеры в Иерусалиме становились все более враждебно настроены к Иисусу, их влияние и управление не распространялось на синагоги за пределами города. Лишь в более поздний период публичного служения Иисуса они смогли создать такое широко распространенное негативное отношение к нему, что оно привело к тому, что почти повсеместно синагоги были закрыты для его учения. Но в это время все синагоги Галилеи и Иудеи были для него открыты.
\vs p146 4:2 В те дни Ирон был местом широкой добычи полезных ископаемых, и поскольку Иисус никогда не сталкивался с жизнью рудокопа, почти все время своего пребывания в Ироне он провел в шахтах. Пока апостолы ходили по домам и проповедовали в публичных местах, Иисус работал в шахтах с тружениками подземелья. Слава Иисуса как целителя достигла даже этого отдаленного селения, поэтому многие больные и страждущие пытались облегчить страдания от его рук, и многим очень помогло его целительное служение. Однако ни в одном из этих случаев Учитель не совершал так называемых чудес исцеления, кроме случая с прокаженным.
\vs p146 4:3 \P\ На третий день пребывания в Ироне спустя несколько часов после полудня, когда Иисус возвращался к своему жилищу, случилось так, что он проходил по узкому переулку. Когда он приблизился к жалкой лачуге некого больного проказой, страдалец, слышавший о его славе целителя, осмелился обратиться к нему, когда тот проходил мимо его двери и упав на колени сказал: «Господи, если только хочешь, можешь меня очистить. Я слышал послание учителей твоих и войду в царство, если смогут меня очистить». Прокаженный говорил так потому, что среди евреев прокаженным запрещалось даже посещать синагогу или каким\hyp{}либо иным способом участвовать в общественных богослужениях. Этот человек действительно верил, что не может быть принят в грядущее царство, пока не найдет исцеления от своей проказы. И когда Иисус увидел его в несчастье и услышал его слова веры, полной надежды, человеческое сердце Иисуса было тронуто, а его божественный ум исполнился сострадания. Когда же Иисус посмотрел на него, человек пал ниц и стал молиться. Тогда Учитель протянул руку и, коснувшись его, сказал: «Повелеваю --- очистись». И тот был немедленно исцелен; и проказа более не причиняла ему страданий.
\vs p146 4:4 Подняв человека на ноги, Иисус приказал ему: «Смотри, никому не рассказывай о своем исцелении, но тихо пойди по делу своему и покажи себя священнику и принеси жертвы, какие повелел Моисей, во свидетельство очищения твоего». Но этот человек не сделал так, как велел ему Иисус. Вместо этого он стал рассказывать всему городу о том, что Иисус его исцелил от проказы, а поскольку этого человека в селении знали все, народ ясно увидел, что он был очищен от своей болезни. Не ходил он и к священникам, как посоветовал ему Иисус. А так как весть о том, что Иисус его исцелил, на следующий день распространилась очень быстро, Учителя осадили такие толпы больных, что он был вынужден встать рано утром и покинуть селение. Хотя Иисус больше не возвращался в этот город, он еще два дня оставался в его предместье недалеко от шахт, продолжая давать дальнейшие наставления верующим рудокопам о евангелие царства.
\vs p146 4:5 Очищение прокаженного до сих пор было первым так называемым чудом, которое Иисус совершил умышленно и намеренно. И это был случай настоящей проказы.
\vs p146 4:6 \P\ Из Ирона они пошли в Гишалу и, возвещая евангелие, провели там два дня, а затем отправились в Хоразин, где, проповедуя благую весть, пробыли почти неделю; однако в Хоразине им не удалось обратить многих верующих к царству. Ни в одном месте, где учил Иисус, не встречался он с таким единодушным неприятием своего послания. Пребывание в Хоразине было весьма удручающим для большинства апостолов, и Андрею с Авениром было очень трудно поддерживать дух своих товарищей. Итак, пройдя тихо через Капернаум, они пошли дальше к селению Мадон, где им стало полегче. В умах большинства апостолов возобладала мысль о том, что им не удалось добиться успеха в этих недавно посещенных ими городах потому, что Иисус настойчиво требовал от них в своих учениях и проповедях воздерживаться от упоминания о нем как о целителе. Как же им хотелось, чтобы он очистил еще одного прокаженного или каким\hyp{}либо иным способом явил свою силу и завладел вниманием народа! Однако их горячий призыв не тронул сердце Иисуса.
\usection{5. Снова в Кане}
\vs p146 5:1 Апостолы чрезвычайно воодушевились, когда Иисус объявил: «Завтра мы пойдем в Кану». Апостолы понимали,что в Кане их будут слушать с сочувствием, ибо Иисуса там хорошо знали. Они весьма успешно вершили свое дело обращения людей к царству, когда на третий день в Кану прибыл некий Тит, известный житель Капернаума, который до некоторой степени верис в учение Исуса и чей сын был опасно болен. Он услышал, что Иисус был в Кане, и поэтому поспешил туда, чтобы встретиться с ним. Верующие в Капернауме думали, что Иисус может исцелить любую болезнь.
\vs p146 5:2 Разыскав Иисуса в Кане, этот знатный человек стал его упрашивать поспешить в Капернаум и исцелить его больного сына. Апостолы стояли рядом в напряженном ожидании, и Иисус, посмотрев на отца больного мальчика, сказал: «Доколе буду терпеть вас? Сила Бога --- среди вас, но вы отказываетесь верить, пока не увидите знамения и не узрите чудес». Но этот знатный человек умолял Иисуса, говоря: «Господи мой, я верую, но приди, пока не умер сын мой, ибо, когда я покинул его, он был уже при смерти». Склонив голову и на мгновение погрузившись в молчаливое раздумье, Иисус вдруг сказал: «Возвращайся домой; твой сын будет жить». Тит поверил слову Иисуса и поспешил назад в Капернаум. Когда же он возвращался, его слуги вышли к нему навстречу и сказали: «Радуйся, ибо твой сын поправился --- он жив». Тогда Тит спросил у них, в котором часу мальчику стало лучше, и когда слуги ответили: «Вчера около седьмого часа лихорадка оставила его», отец вспомнил, что приблизительно в этот час Иисус и сказал: «Твой сын будет жить». И с тех пор Тит верил всем сердцем и верила также вся его семья. Этот же сын стал могущественным служителем царства и позднее лишился жизни вместе с теми, кто страдал в Риме. Хотя вся семья Тита, ее друзья и даже апостолы считали это событие чудом, оно чудом не было. По крайней мере, это не было чудом исцеления болезни тела. Это был просто случай предвидения действия естественного закона, как раз такого знания, каким Иисус часто пользовался после своего крещения.
\vs p146 5:3 И снова Иисусу пришлось поспешить прочь из Каны а поскольку его служение в этом селении привлекало излишнее внимание, вызванное уже вторым случаем подобного рода Жители города помнили о воде и вине, и теперь, когда все думали, что он на столь значительном расстоянии исцелил сына знатного человека, шли к нему и не только несли к нему больных и страждущих, но и направляли посыльных, дабы он исцелил страдальцев на расстоянии. И увидев, что вся округа пришла в движение, Иисус сказал: «Идем в Наин».
\usection{6. Наин и сын вдовы}
\vs p146 6:1 Эти люди верили в знамения; они были поколением, жаждавшим сверхъестественного. К этому времени народ центральной и южной Иудеи от всего, что имело отношение к Иисусу и его личному служению, ждал только чудес. Десятки, сотни честных людей, страдавших чисто нервными расстройствами и подверженных эмоциональным срывам, приходили к Иисусу, а затем возвращались домой к своим друзьям, объявляя, что Иисус их исцелил. И подобные случаи ментального исцеления эти невежественные и простодушные люди считали исцелениями тела, случаями чудодейственного излечения.
\vs p146 6:2 \P\ Когда Иисус собрался уходить из Каны и идти в Наин, за ним последовало великое множество верующих и много любопытных. Они желали видеть чудеса и необычные проявления, и не хотели испытать разочарование. Приблизившись к вратам города, Иисус и его апостолы увидели похоронную процессию на пути к находившемуся неподалеку кладбищу, куда несли единственного сына овдовевшей матери, жительницы Наина. Эту женщину глубоко уважали, и половина селения шла за несущими гроб с этим мальчиком, которого считали умершим. Когда похоронная процессия приблизилась к Иисусу и его последователям, вдова и ее друзья узнали Учителя и стали умолять его вернуть сына к жизни. Их ожидание чуда достигло такой степени, что они считали, будто Иисус может излечить любую человеческую болезнь, а раз так, то почему бы такому целителю не воскресить и мертвого? Осаждаемый подобными просьбами, Иисус сделал шаг вперед и, приподняв покрывало носилок, осмотрел мальчика. Увидев, что молодой человек на самом деле не умер, он понял, какую трагедию могло предотвратить его присутствие; поэтому, повернувшись к матери, он сказал: «Не плачь. Твой сын не умер, но спит. Он вернется к тебе». И затем, взяв молодого человека за руку, сказал: «Проснись и встань». И юноша, которого считали умершим, тотчас поднялся, сел и стал говорить, и Иисус отослал их к домам их.
\vs p146 6:3 Иисус старался успокоить толпу и напрасно пытался объяснить, что мальчик на самом деле мертвым не был, что он отнюдь не вернул его из могилы, но бесполезно. Толпа, сопровождавшая его, и все селение Наин просто неистовствовали. Многих охватил страх, других --- паника, третьи же пали и стали молиться и рыдать о грехах своих. Шумную толпу удалось рассеять лишь спустя несколько часов после наступления темноты. И конечно же, несмотря на заявление Иисуса о том, что мальчик не был мертв, все твердили, будто было совершено чудо, воскрешен мертвый. Хотя Иисус говорил им, что мальчик просто глубоко спал, они объясняли, что такова его манера говорить, и обращали внимание на то, что Иисус всегда из великой скромности старался скрыть свои чудеса.
\vs p146 6:4 Так слух о том, что Иисус воскресил из мертвых сына вдовы, разнесся по всей Галилее и достиг Иудеи, и многие слышавшие эту весть поверили ей. Иисус так и не сумел даже всех своих апостолов полностью убедить, что сын вдовы на самом деле не был мертв, когда он велел ему очнуться и встать. Однако он вполне убедил их воздержаться от упоминаний об этом случае во всех последующих записях, и лишь Лука описал его как случай, поведанный ему. И снова Иисуса стали осаждать как врача настолько, что рано утром на следующий день он отправился в Ендор.
\usection{7. В Ендоре}
\vs p146 7:1 В Ендоре Иисус на несколько дней избавился от шумных толп, требовавших исцеления тела. Во время их пребывания в этом месте Учитель для наставления апостолов вспомнил историю о царе Сауле и волшебнице из Ендора. Иисус ясно показал своим апостолам, что заблудшие и мятежные срединники, которые часто выдавали себя за якобы души умерших, вскоре будут укрощены, так что не смогут более совершать эти странные деяния. Он сказал своим последователям, что после того, как он вернется к Отцу и они изольют дух свой на всякую плоть, подобные полудуховные существа --- так называемые нечистые духи --- более не смогут овладевать слабыми и злонамеренными среди смертных.
\vs p146 7:2 Иисус дал своим апостолам дальнейшие разъяснения о том, что духи умерших человеческих существ не возвращаются в мир, откуда они произошли, чтобы общаться со своими живыми собратьями. Что только по прошествии диспенсационной эпохи станет возможным возвращение на землю развивающегося духа смертного человека, и то лишь в исключительных случаях и только в качестве духовного руководства планеты.
\vs p146 7:3 После двух дней отдыха Иисус сказал своим апостолам: «Пойдем завтра в Капернаум, где остановимся и будем учить, пока округа не успокоится. Дома они к этому времени отчасти отойдут от подобного рода волнения»
