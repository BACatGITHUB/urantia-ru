\upaper{121}{Время пришествия Михаила}
\author{Комиссия срединников}
\vs p121 0:1 Действуя под руководством совета двенадцати членов Объединенного братства срединников Урантии, при совместной поддержке главы нашего чина и назначенного Мелхиседека, я, срединник второго рода, некогда прикрепленный к Апостолу Андрею, уполномочен записать рассказ о жизни и деяниях Иисуса из Назарета в том виде, как их наблюдали братья по чину и как они частично были записаны человеком, хранителем которого я был некоторое время. Зная, как тщательно его Учитель избегал оставлять после себя какие бы то ни было письменные свидетельства, Андрей неуклонно отказывался копировать свои записи. Такое же отношение других апостолов Иисуса сильно задержало написание Евангелий.
\usection{1. Запад в первом веке после Христа}
\vs p121 1:1 Иисус пришел в мир в эпоху, которая вовсе не была периодом духовного упадка. Во времена его рождения Урантия испытывала невиданный расцвет духовной мысли и религиозной жизни, подобного которому не бывало за всю предшествующую историю со времен Адама, ни в какую\hyp{}либо последующую эру. В то время, когда Михаил воплотился на Урантии, в мире создались самые благоприятные условия из когда\hyp{}либо существовавших прежде или впоследствии для пришествия Сына\hyp{}Творца. В столетия, непосредственно предшествовавшие этим временам, греческая культура и греческий язык распространились по всему Западному миру и Ближнему Востоку. Евреи, будучи левантийской расой, по природе своей отчасти западной, отчасти восточной, являли наиболее благоприятную культурную и лингвистическую среду для успешного распространения новой религии и на Востоке, и на Западе. Эти чрезвычайно благоприятные обстоятельства подкреплялись полным терпимости правлением римлян в Средиземноморье.
\vs p121 1:2 Все это разнообразие влияний в мире хорошо иллюстрируют деяния Павла, который, будучи иудеем из иудеев, проповедовал Евангелие еврейского мессии на греческом языке, оставаясь при этом римским гражданином.
\vs p121 1:3 Ничего подобного цивилизации времен Иисуса не появлялось на Западе ни до, ни после той поры. Европейская цивилизация объединялась и направлялась тройственным влиянием:
\vs p121 1:4 \ublistelem{1.}\bibnobreakspace Римской политической и социальной системой.
\vs p121 1:5 \ublistelem{2.}\bibnobreakspace Греческими языком, культурой и, в определенной степени, философией.
\vs p121 1:6 \ublistelem{3.}\bibnobreakspace Быстро распространяющимся влиянием еврейских религиозных и моральных учений.
\vs p121 1:7 \pc Во времена, когда родился Иисус, весь средиземноморский мир был единой империей. Впервые в мировой истории хорошие дороги соединяли между собой множество больших и важных центров. Моря были очищены от пиратов и эра великого расцвета торговли и путешествий быстро набирала силы. Вплоть до девятнадцатого века после Христа Европа более не знала подобного успеха в торговле и путешествиях.
\vs p121 1:8 Несмотря на внутренний мир и внешнее благополучие греко\hyp{}римского мира, большинство жителей империи прозябали в нищете и убожестве. Немногочисленный высший класс был богат; большинство же простых людей составляло жалкий и обнищавший низший класс. В те времена не существовало счастливого и преуспевающего среднего класса, он лишь зарождался в римском обществе.
\vs p121 1:9 В недавнем прошлом завершились первые битвы между Римским и Парфянским государствами, и Сирия оказалась в руках римлян. Во времена Иисуса Палестина и Сирия переживали период процветания, относительного спокойствия и широкого коммерческого обмена с землями как Востока, так и Запада.
\usection{2. Еврейский народ}
\vs p121 2:1 Евреи являлись частью древней семитской расы, включавшей также вавилонян, финикийцев и недавно ставших врагами Рима карфагенян. В начале первого столетия после Христа евреи были самой влиятельной группой среди семитских народов, они же занимали особо стратегически важное географическое положение в мире, которое во многом определялось путями и организацией торговли.
\vs p121 2:2 Множество больших дорог, соединявших древние народы, проходило через Палестину, таким образом оказавшуюся перекрестком трех континентов. Путешественники, торговцы и армии из Вавилона, Ассирии, Египта, Сирии, Греции, Парфии и Рима волна за волной наводняли Палестину. С незапамятных времен множество караванных путей с Востока проходили через различные части этого района к тем немногим удобным морским портам восточного побережья Средиземного моря, откуда корабли развозили грузы по всему прибрежному Западу. Более половины этих караванов проходили через маленький городок Назарет в Галилее или же вблизи него.
\vs p121 2:3 Хотя Палестина и была очагом еврейской религиозной культуры, местом, где зародилось христианство, евреи были рассеяны по всему миру, обитая среди различных народов и занимаясь торговлей в каждой из римских и парфянских провинций.
\vs p121 2:4 Греция дала миру язык и культуру, Рим построил дороги и объединил империю, но лишь рассеяние евреев с их более чем двумястами синагогами и хорошо организованными религиозными общинами, разбросанными повсюду в римском мире, создало те культурные центры, где новое евангелие царства небесного первоначально было воспринято и откуда затем распространилось по всему свету.
\vs p121 2:5 Вокруг каждой еврейской синагоги собирался некий круг верующих\hyp{}неевреев, «набожных» или «богобоязненных» людей, и именно они составили большинство обращенных Павлом в христианство. Даже при Иерусалимском храме был богато украшенный двор, где собирались неевреи. Очень тесные взаимосвязи поддерживались между культурой, коммерцией и религиозной жизнью Иерусалима и Антиохии. В Антиохии последователей Павла впервые стали называть «христианами».
\vs p121 2:6 В централизации иудейского храмового богослужения в Иерусалиме заключался секрет выживания их монотеизма и того, что новое и более широкое представление о едином Боге всех народов и Отце всех смертных будет взлелеяно и распространено дальше по всему миру. Служба в Иерусалимском храме была символом выживания религиозной культурной идеи невзирая на нескончаемую череду сменяющих друг друга иностранных властителей, гонения и преследования евреев.
\vs p121 2:7 \pc Евреи тех времен, хотя и находились под римским владычеством, обладали, в принципе, правом самоуправления и, помня о недавних подвигах освободительной борьбы Иуды Маккавея и его ближайших последователей, с трепетом ожидали скорого появления еще более великого освободителя, долгожданного Мессии.
\vs p121 2:8 Секрет выживания Палестины, Иудейского царства в виде полунезависимого государства определялся внешней политикой Рима, который стремился сохранить контроль над палестинскими торговыми путями между Сирией и Египтом, а также западными оконечностями караванных путей между Востоком и Западом. Рим вовсе не хотел, чтобы в Леванте возник какой\hyp{}нибудь центр власти, ибо это могло воспрепятствовать его будущей экспансии в этом регионе. Политика интриг, целью которой было стравливать Сирию Селевкидов и Египет Птолемеев друг с другом, вынуждало Рим поддерживать Палестину и превратить ее в отдельное и независимое государство. Политика Рима, упадок Египта и все возрастающее ослабление Селевкидов наряду с ростом Парфянского царства объясняют, почему в течение жизни нескольких поколений маленькая и не обладающая никакой властью группа евреев могла сохранять свою независимость как от Селевкидов на Севере, так и от Птолемеев на Юге. Эту случайно доставшуюся им свободу и независимость от политического влияния соседних и более могущественных народов евреи приписывали тому, что они являются «избранным народом», который пользуется прямым покровительством Яхве. Эта убежденность в своем расовом превосходстве сделала еще тяжелей для них римское господство, когда оно в конце концов обрушилось на эту землю. Но даже в это печальное время евреи отказались понять, что их всемирная миссия была духовной, а не политической.
\vs p121 2:9 \pc Во времена Иисуса евреи были исполнены тревожных предчувствий и подозрений, потому что ими правил чужак, идумеец Ирод, захвативший власть в Иудее путем ловких интриг с римскими правителями. И хотя Ирод открыто провозглашал лояльность к иудейским обрядам, он продолжал строить храмы многим чужим богам.
\vs p121 2:10 Дружеские отношения Ирода с правителями Рима обеспечили евреям безопасное передвижение по миру и таким образом открыли пути усиленному проникновению евреев даже в самые отдаленные части Римской империи, а также к другим народам с новым евангелием царства небесного. Правление Ирода также во многом благоприятствовало дальнейшему смешению иудейской и эллинистической философии.
\vs p121 2:11 Ирод построил гавань Кесарию, которая еще более способствовала превращению Палестины в перекресток путей всего цивилизованного мира. Он умер в 4 году до н.э., а его сын Ирод Антипа правил Галилеей и Переей во времена юности и проповедничества Иисуса, вплоть до 39 года н.э. Антипа, как и его отец, был великим строителем. Он реконструировал многие города Галилеи, в том числе важный торговый центр Сефорис.
\vs p121 2:12 Иерусалимские религиозные деятели и учителя\hyp{}раввины недолюбливали галилеян. Во времена рождения Иисуса Галилея была скорее нееврейской, чем еврейской.
\usection{3. Среди неевреев}
\vs p121 3:1 Хотя социальные и экономические условия жизни в римском государстве были не столь уж высоки, распространение мирной жизни и преуспеяния были вполне благоприятны для пришествия Михаила. В первом веке после Христа общество Средиземноморья состояло из пяти четко обозначенных слоев:
\vs p121 3:2 \ublistelem{1.}\bibnobreakspace \bibemph{Аристократия.} Высшие классы, обладающие деньгами и официальной властью, привилегированные и правящие группы.
\vs p121 3:3 \pc \ublistelem{2.}\bibnobreakspace \bibemph{Деловые круги.} Крупные коммерсанты и банкиры, торговцы --- крупные импортеры и экспортеры --- международные купцы.
\vs p121 3:4 \pc \ublistelem{3.}\bibnobreakspace \bibemph{Немногочисленный средний класс.} Несмотря на то, что группа эта была действительно немногочисленна, она оказалась достаточно влиятельной и составила моральную основу ранней христианской церкви, которая, в свою очередь, поощряла эти группы в их занятиях ремеслами и торговлей. К этому классу торговцев среди евреев принадлежали и многие фарисеи.
\vs p121 3:5 \pc \ublistelem{4.}\bibnobreakspace \bibemph{Свободные пролетарии.} У этой группы почти что не было общественного статуса, во всяком случае, он был весьма невысок. Несмотря на то, что они гордились своей свободой, пролетарии находились в довольно невыгодном положении, потому что им приходилось соревноваться с трудом рабов. Высшие классы относились к ним с презрением, считая абсолютно непригодными ни для чего, кроме «сохранения расы».
\vs p121 3:6 \pc \ublistelem{5.}\bibnobreakspace \bibemph{Рабы.} Половину населения Римского государства составляли рабы; многие из них были выдающимися людьми и быстро пробили себе дорогу в среде свободных пролетариев и даже купцов. Большинство же находилось на среднем или довольно низком уровне.
\vs p121 3:7 Взятие в рабство было характерной чертой римских военных завоеваний, и рабами становились даже представители высокоразвитых народов. Власть хозяина над рабом была безграничной. Ранняя христианская церковь в большой степени состояла из представителей низших классов и рабов.
\vs p121 3:8 Лучшие из рабов нередко получали жалованье и, скопив денег, могли купить свободу. Многие из таких освободившихся рабов добивались высокого положения в государстве, церкви и деловом мире. Именно благодаря таким возможностям ранняя христианская церковь была столь терпимой к подобным формам рабства.
\vs p121 3:9 \pc В Римской империи первого века после Христа не существовало никаких серьезных социальных проблем. Большая часть населения считала себя принадлежащей к той социальной группе, в какой ей выпала доля родиться. К тому же всегда были открыты пути, по которым талантливые и одаренные личности могли подняться на более высокие ступени римского общества, хотя, как правило, люди были в целом довольны своим социальным положением. Они не обладали классовым сознанием и к тому же не смотрели на классовые различия как на нечто несправедливое или ошибочное. Христианство никоим образом не было экономическим движением, ставящим себе целью улучшить положение бедствующих и угнетенных слоев.
\vs p121 3:10 Несмотря на то, что женщина пользовалась в Римской империи большей свободой, чем в Палестине, где ее положение было во многом зависимым, преданность семье и привязанность друг к другу у евреев значительно превосходили те же чувства у представителей нееврейского мира.
\usection{4. Нееврейская философия}
\vs p121 4:1 По степени развития нравственности, неевреи были в чем\hyp{}то ниже иудеев, но в сердцах наиболее благородных неевреев было вполне достаточно природной доброты и потенциальной человеческой привязанности, чтобы на этой почве семя христианства смогло взойти и принести обильный урожай нравственного характера и духовных достижений. Нееврейский мир находился под влиянием четырех великих философий, в той или иной степени происходящих от раннего греческого платонизма. Этими философскими школами были:
\vs p121 4:2 \ublistelem{1.}\bibnobreakspace \bibemph{Эпикурейство.} Это философское учение было направлено на поиски счастья. Лучшие из эпикурейцев вовсе не предавались чувственным излишествам. По крайней мере, эта доктрина помогла освободить римлян от закостенелого фатализма, она учила, что люди способны улучшить свое земное положение, если захотят что\hyp{}либо для этого предпринять. Она достаточно эффективно боролась с предрассудками и невежеством.
\vs p121 4:3 \pc \ublistelem{2.}\bibnobreakspace \bibemph{Стоицизм.} Стоицизм был высшей философией высших классов. Стоики верили в то, что управляющий нами Разум\hyp{}Судьба властен над всей природой. Они учили, что душа человека божественна; что она заключена в темницу тела физической природы. Душа человека становилась свободной тогда, когда существовала в гармонии с природой, с Богом; таким образом, добродетель сама по себе являлась наградой. Стоицизм достиг вершин высочайшей нравственности, его идеалы не превзошла ни одна созданная людьми философская система. Но несмотря на то, что стоики признавали себя «порождениями Бога», им не удалось познать его и, таким образом, они так и не нашли его. Стоицизм остался философией; он так и не стал религией. Его последователи стремились настроить свои умы на пребывание в гармонии с Вселенским Разумом, но не смогли осознать себя детьми любящего Отца. Павел был заметно склонен к стоицизму, когда писал: «Я научился довольствоваться тем положением, в котором нахожусь».
\vs p121 4:4 \pc \ublistelem{3.}\bibnobreakspace \bibemph{Кинизм.} Несмотря на то, что киники возводят свою философию к Диогену Афинскому, многое в их доктрине унаследовано от остатков учения Махивенты Мелхиседека. Поначалу кинизм был скорее религией, чем философией. По крайней мере, киники сделали свою религо\hyp{}философию демократичной. На площадях и рынках они настойчиво проповедовали свою доктрину, провозглашая, что «человек может спасти себя, если захочет». Они призывали к простоте и добродетели и учили людей встречать смерть безбоязненно. Эти странствующие проповедники много сделали для того, чтобы подготовить духовно изголодавшихся людей к будущим проповедям христианских миссионеров. Их народные проповеди по манере и стилю во многом были схожи с посланиями Павла.
\vs p121 4:5 \pc \ublistelem{4.}\bibnobreakspace \bibemph{Скептицизм.} Скептицизм утверждал, что знание иллюзорно, а убеждение и уверенность невозможны. Это был чисто негативный подход, который так никогда широко и не распространялся.
\vs p121 4:6 \pc Все эти философии были полурелигиозными. Зачастую они воодушевляли, облагораживали, были высоко этичными, но обычно были далеки от простых людей. Это были философии, возможно за исключением кинизма, для сильных и мудрых, а вовсе не религии спасения, обращенные равно и к бедным, и к слабым.
\usection{5. Религии неевреев}
\vs p121 5:1 На протяжении предыдущих веков религия оставалась по преимуществу делом племени или нации; редко она становилась предметом заботы индивидуума. Боги были родовыми или национальными, а не личными. Подобные религиозные системы не отвечали личным духовным устремлениям среднего человека.
\vs p121 5:2 Во время жизни Иисуса религии Запада включали:
\vs p121 5:3 \ublistelem{1.}\bibnobreakspace \bibemph{Языческие культы.} Это было сочетание эллинистической и латинской мифологии, патриотизма и традиции.
\vs p121 5:4 \pc \ublistelem{2.}\bibnobreakspace \bibemph{Почитание императора.} Обожествление человека как символа государства возмущало евреев и ранних христиан, что и повлекло за собой жестокое преследование обеих церквей римским правительством.
\vs p121 5:5 \pc \ublistelem{3.}\bibnobreakspace \bibemph{Астрология.} Эта псевдонаука, пришедшая из Вавилона, превратилась в религию в греко\hyp{}римской империи. Даже в двадцатом столетии человек не освободился от этого суеверия полностью.
\vs p121 5:6 \pc \ublistelem{4.}\bibnobreakspace \bibemph{Религии мистерий.} На духовно голодный мир обрушился поток мистериальных культов, новых и странных религий из Леванта, очаровавших простых людей и суливших им \bibemph{личное} спасение. Эти религии быстро стали укоренившейся верой низших классов греко\hyp{}римского мира. Они во многом подготовили путь для быстрого распространения значительно превосходящего их христианского учения, давшего величественную концепцию Божества, которая соединяла в себе привлекательную для интеллектуалов теологию и прямой путь к спасению для всех, включая и невежественного, но духовно изголодавшегося обычного человека того времени.
\vs p121 5:7 \pc Религии\hyp{}мистерии положили конец национальным верованиям и в итоге породили многочисленные личные культы. Мистерий было много, но для них всех характерно следующее:
\vs p121 5:8 \pc \ublistelem{1.}\bibnobreakspace В каждой из них присутствовала некая мифическая легенда, тайна --- отсюда и название мистерия. Как правило, тайна эта относилась к повествованию о жизни, смерти и возвращении вновь к жизни какого\hyp{}либо Бога, что можно хорошо понять на примере митраизма, который на время оказался соперником и современником создаваемого Павлом христианского культа.
\vs p121 5:9 \pc \ublistelem{2.}\bibnobreakspace Мистерии были вненациональными и межрасовыми. Они обращались к личности и делали это по\hyp{}братски, тем самым сопутствуя росту религиозных братств и многочисленных сектантских обществ.
\vs p121 5:10 \pc \ublistelem{3.}\bibnobreakspace Что касается службы, то они отличались сложными церемониями посвящений и впечатляющими причастиями во время богослужений. Иногда эти тайные ритуалы и процедуры вызывали ужас и отвращение.
\vs p121 5:11 \pc \ublistelem{4.}\bibnobreakspace Независимо от природы подобных церемоний или степени излишеств, мистерии неизменно обещали своим посвященным \bibemph{спасение,} «освобождение от зла, жизнь после смерти и продолжение жизни в блаженных царствах по ту сторону этого рабского и печального мира».
\vs p121 5:12 \pc Но не стоит путать учение Христа с мистериями. Популярность мистерий отражает подлинное стремление человека выжить, показывая насущную потребность в личной религии и добродетели. Несмотря на то, что мистерии не удовлетворяли этим стремлениям должным образом, они подготовили дорогу для последующего появления Иисуса, который поистине принес в этот мир хлеб и воду жизни.
\vs p121 5:13 Павел, пытаясь использовать широкую популярность лучших мистерий, отчасти изменил учение Иисуса, дабы подобным переложением сделать их более приемлемыми для большего числа будущих новообращенных. Но даже компромиссное изложение учений Иисуса Павлом (христианство) было гораздо выше того лучшего, что было в мистериях:
\vs p121 5:14 \ublistelem{1.}\bibnobreakspace Павел учил нравственному искуплению, этическому спасению. Христианство указывало на новую жизнь и провозглашало новый идеал. Павел отказался от магических ритуалов и колдовских церемоний.
\vs p121 5:15 \pc \ublistelem{2.}\bibnobreakspace Христианство представляло собой религию, которая приводила к конечному разрешению человеческих проблем, потому что не только предлагала спасение от печали и даже смерти, но и обещала избавление от греха, за которым следовал дар праведного характера обладающего качествами вечной жизни.
\vs p121 5:16 \pc \ublistelem{3.}\bibnobreakspace Мистерии были построены на мифах. Христианство, каким проповедовал его Павел, было основано на историческом факте: пришествии Михаила, Сына Бога, к человечеству.
\vs p121 5:17 \pc У неевреев мораль не обязательно была связана с философией или религией. За пределами Палестины далеко не всегда людям приходило в голову, что служитель религии должен вести нравственную жизнь. Иудаизм и вслед за ним учение Иисуса, а позже христианство Павла были первыми европейскими религиями, соединившими мораль и этику и настаивавшими на том, что религиозные деятели должны обращать внимание и на то, и на другое.
\vs p121 5:18 Именно к людям, обладающим столь несовершенными философскими системами и запутавшимся в столь сложных религиозных культах, и пришел Иисус, рожденный в Палестине. К этим людям он обратил свою проповедь личной религии --- сыновства с Богом.
\usection{6. Иудейская религия}
\vs p121 6:1 В конце первого столетия до Христа религиозная мысль в значительной степени была подвержена влиянию греческих культурных традиций и греческой философии и частично видоизменялась в результате этого влияния. Долгое противостояние между взглядами восточной и западной школ иудейской мысли в целом закончилось принятием Иерусалимом, остальным Западным миром и Левантом западной, то есть видоизмененной эллинистической, точки зрения.
\vs p121 6:2 В Палестине времен Иисуса были распространены три основных языка: простые люди говорили на одном из диалектов арамейского; священники и раввины на древнееврейском; высшие классы евреев и наиболее образованная часть общества в основном говорили по\hyp{}гречески. Немалую роль в преобладающем влиянии греческого крыла иудейской культуры и теологии сыграли ранние переводы иудейских писаний на греческий в Александрии. Вскоре должны были появиться и труды христианских учителей на том же языке. Возрождение иудаизма начинается с перевода на греческий иудейских писаний. Именно эти тенденции предопределили обращение христианского культа Павла к Западу, а не к Востоку.
\vs p121 6:3 Учение эпикурейцев оказало незначительное влияние на эллинизированные верования евреев, тогда как философия Платона и стоические доктрины самоотрицания, напротив, оставили в них заметный след. Огромное влияние стоицизма можно увидеть в «Четвертой Книге Маккавеев»; проникновение и платонической, и стоической доктрин просматривается и в «Притчах Соломона». Эллинизированные евреи столь аллегорически толковали иудейское писание, что смогли довольно легко согласовать иудейскую теологию со столь чтимой ими аристотелевской философией. Но все это привело в результате к чудовищной путанице, царившей до той поры, пока этими проблемами не занялся Филон Александрийский; он начал трудиться над тем, чтобы систематизировать и согласовать греческую философию и иудейскую теологию так, чтобы создать компактную и последовательную систему религиозной жизни и верований. Именно это позднее учение, совмещающее греческую философию и иудейскую теологию, преобладало в Палестине во времена, когда там жил и учил Иисус, и именно это учение Павел использовал как основу для построения своего более прогрессивного и просвещающего культа --- христианства.
\vs p121 6:4 Филон был великим учителем; со времен Моисея не было человека, который оказывал бы столь глубокое влияние на этическую и религиозную мысль западного мира. Семь выдающихся учителей из числа людей смогли сочетать лучшие элементы современных им систем этических и религиозных учений; это Сефард, Моисей, Зороастр, Лао\hyp{}Цзы, Будда, Филон и Павел.
\vs p121 6:5 В попытке Филона сочетать греческую мистическую философию и римскую стоическую доктрину с основанной на законах теологией иудеев были и противоречивые моменты, многие из которых, хотя далеко не все, Павел отследил и мудро исключил из своей дохристианской теологии. Филон расчистил пути для Павла, что позволило ему полнее восстановить концепцию Райской Троицы, столь долго остававшуюся сокрытой в иудейской теологии. Только в одном случае Павлу не удалось идти в ногу с Филоном или превзойти учение этого богатого и высокообразованного еврея из Александрии: речь идет о доктрине искупления. Филон в своих учениях пытался освободиться от доктрины, гласящей, что прощение возможно только с пролитием крови. Не исключено также, что он прозревал реальное присутствие Настройщиков Мысли гораздо яснее, чем Павел. Теория же Павла о первородном грехе, его доктрины о наследственной вине и врожденном зле и следующем из этого искуплении частично проистекали из митраистских представлений и имели довольно мало общего с иудейской теологией, философией Филона или учением Иисуса. Некоторые части учения Павла о первородном грехе и искуплении были его собственным детищем.
\vs p121 6:6 Евангелие от Иоанна, последний из рассказов о земной жизни Иисуса, было адресовано западным людям и построено в свете воззрений поздних александрийских христиан, которые были последователями учений Филона.
\vs p121 6:7 \pc Примерно в пору жизни Христа отношение к евреям в Александрии странно изменилось, и в этой бывшей еврейской цитадели поднялась жестокая волна гонений, докатившаяся даже до Рима, откуда были изгнаны многие тысячи евреев. Но эта кампания клеветы оказалась недолговечной, и вскоре имперское правительство восстановило свободы евреев в империи.
\vs p121 6:8 Разбросанные ли по всему миру, занесенные ли в разные части света делами торговли, гонимые ли преследованиями, евреи в сердцах своих хранили образ центра, святого Иерусалимского храма, и это чувство сплачивало их. Еврейская теология сохранилась именно в том виде, как ее толковали и осуществляли на практике в Иерусалиме, несмотря на то, что несколько раз она была спасена от забвения своевременной деятельностью некоторых вавилонских учителей.
\vs p121 6:9 Почти два с половиной миллиона этих разбросанных по свету евреев обычно съезжались в Иерусалим на празднование национальных религиозных торжеств. И невзирая на теологические или философские различия, существующие между восточными (вавилонскими) и западными (эллинистическими) евреями, они все стекались в Иерусалим как в центр своего богослужения, в вечном ожидании прихода Мессии.
\usection{7. Евреи и неевреи}
\vs p121 7:1 Ко времени Иисуса евреи выработали четкую концепцию о своем происхождении, истории и судьбе. Они построили несокрушимую стену между собой и нееврейским миром; к неевреям они относились с презрением. Они поклонялись букве закона и пестовали добродетельный образ собственного народа, основанный на ложной гордости своим происхождением. Они создали предвзятое представление об обещанном Мессии, и в своих ожиданиях рисовали облик Спасителя, неотъемлемого от национальной и расовой истории. Евреи той поры считали свою теологию неизменной, незыблемой во веки веков.
\vs p121 7:2 Учение и деяния Христа, проповедовавшие терпимость и доброту, входили в противоречие с издавна установившимся отношением евреев к другим народам, которых они считали язычниками. Поколениями евреи воспитывали в себе то отношение ко внешнему миру, которое сделало для них невозможным принятие наставлений Учителя о духовном братстве людей. Они не желали разделять Яхве с неевреями на равных и, похоже, не собирались принимать как Сына Бога того, кто учил столь новым и странным доктринам.
\vs p121 7:3 Книжники, фарисеи и священнослужители создали для евреев чудовищное ярмо ритуальности и законничества, и это ярмо было гораздо реальнее и действеннее, чем Римская политическая власть. Евреи времен Иисуса не только держались в жестком подчинении \bibemph{закону,} но в равной степени были скованы рабскими условиями \bibemph{традиции,} внедрившейся во все сферы личной и общественной жизни. Эти скрупулезные нормы поведения определяли жизнь каждого правоверного еврея, и нет ничего странного в том, что евреи довольно скоро отвергли своего собрата, осмелившегося игнорировать священные традиции и позволившего себе пренебрегать их столь ревностно хранимыми нормами поведения. Едва ли евреи в состоянии были относиться с почтением к учению того, кто без всяких колебаний порывал с догмами, которые они приписывали самому Отцу Аврааму. Моисей вручил им их закон, и они не собирались идти ни на какие компромиссы.
\vs p121 7:4 К первому столетию после Христа устное толкование закона авторитетными учителями, книжниками, стало пользоваться большим уважением, чем само писание. Тем легче было определенным религиозным лидерам призвать народ к неприятию новой проповеди.
\vs p121 7:5 Все эти обстоятельства сделали невозможным для иудеев исполнение их божественного предназначения --- стать носителями новой проповеди религиозной свободы и духовного освобождения. Они не смогли разорвать путы традиции. Иеремия говорил о «законе, запечатленном в сердцах людей», Иезекииль повествовал о «новом духе, который воцарится в душе человека», а Псалмопевец молился о том чтобы Бог «сотворил сердце чистым и обновил дух праведный\ldots » Но когда иудаизм, религия праведного труда и покорности закону, пал жертвой косности и инерции традиционализма, движение религиозной эволюции сместилось к Западу, к европейским народам.
\vs p121 7:6 Итак, иные люди оказались призванными к тому, чтобы привнести в мир более высокую теологию, систему учений, вмещающую философию греков, закон римлян, мораль евреев и проповедь личной святости и духовной свободы, сформулированную Павлом и основанную на учении Иисуса.
\vs p121 7:7 \pc Созданный Павлом христианский культ опирался на принципы еврейской морали. Евреи смотрели на историю как на провидение Бога --- Яхве за работой. Греки привнесли в новое учение более ясные представления о вечной жизни. На теологическую и философскую стороны доктрины Павла оказали влияние не только учение Иисуса, но также Платон и Филон. Этика его была вдохновлена не только Христом, но и стоиками.
\vs p121 7:8 Евангелие Иисуса, представленное в культе антиохийского христианства Павла, оказалось смешанным со следующими учениями:
\vs p121 7:9 \ublistelem{1.}\bibnobreakspace Философскими постулатами обращенных в иудаизм греков, включавшими и некоторые концепции о вечной жизни.
\vs p121 7:10 \ublistelem{2.}\bibnobreakspace Привлекательными учениями преобладавших мистериальных культов, в особенности митраистскими доктринами возмездия, искупления и спасения с помощью жертвы, принесенной кем\hyp{}либо из богов.
\vs p121 7:11 3.Стойкой моралью незыблемой религии евреев.
\vs p121 7:12 \pc Средиземноморская Римская империя, Парфянское царство и соседние им народы времен Иисуса обладали весьма примитивными и неразвитыми представлениями о географии, астрономии, здоровье и болезнях; разумеется, они были ошеломлены новыми поразительными высказываниями плотника из Назарета. Представления о вселении духов, плохих и хороших, относились не только к людям, но и к каждому дереву или камню. Это была волшебная эпоха, и все верили в чудеса, происходящие на каждом шагу.
\usection{8. Предшествующие письменные свидетельства}
\vs p121 8:1 Насколько это возможно, твердо придерживаясь данных нам наказов, мы приложим все усилия к тому, чтобы использовать и в определенной степени упорядочить уже существующие записи, относящиеся к жизни Иисуса на Урантии. Хотя у нас был доступ к утерянным записям апостола Андрея и мы получили огромную пользу от сотрудничества с сонмом небесных существ, пребывавших на Земле во времена пришествия Михаила, особенно --- с его отныне персонализированным Настройщиком Мысли, нашей задачей было использовать и так называемые Евангелия от Матфея, Марка, Луки и Иоанна.
\vs p121 8:2 Записи Нового Завета появились на свет при следующих обстоятельствах:
\vs p121 8:3 \ublistelem{1.}\bibnobreakspace \bibemph{Евангелие от Марка.} Иоанн Марк сделал самую раннюю (не считая записей Андрея), самую краткую и простую запись событий жизни Иисуса. Он представил Учителя в образе служителя, человека среди людей. Хотя сам Марк и был свидетелем многих из описываемых событий, все же его труд был Евангелием, записанным со слов Симона\hyp{}Петра. Сначала он и был связан с Петром; позже --- с Павлом. Марк писал Евангелие по настоянию Петра и по просьбе Римской церкви. Зная, насколько последовательно отказывался Учитель от записи своего учения, пребывая во плоти на Земле, Марк, так же как и другие апостолы и ученики, сомневался, стоит ли браться за этот труд. Но Петр чувствовал, что Римской церкви необходима поддержка в виде такого письменного свидетельства, и Марк согласился приняться за его подготовку. До смерти Петра в 67 г. н.э. он составил множество заметок и в соответствии с очерками, одобренными Петром и предназначенными для Римской церкви, начал записывать свое повествование вскоре после смерти Петра. Евангелие было закончено примерно к концу 68 г. н.э. В основу его были положены воспоминания самого Марка и воспоминания Петра. С тех пор запись претерпела значительные изменения. Многие части были исключены из нее, а некоторые добавлены позднее и заменили последнюю, пятую часть первоначального Евангелия, утерянную до того, как с первого списка были сделаны копии. Эти записки Марка, наряду с заметками Андрея и Матфея, стали письменной основой всех последующих Евангельских повествований, пытавшихся рассказать о жизни и учении Иисуса.
\vs p121 8:4 \pc \ublistelem{2.}\bibnobreakspace \bibemph{Евангелие от Матфея.} Так называемое Евангелие от Матфея --- это рассказ о жизни Учителя, записанный в назидание еврейским христианам. Автор этой записи последовательно стремится показать, что многое в жизни Иисуса показывало, как «могли быть исполнены слова пророка». Евангелие от Матфея изображает Иисуса как сына Давидова, выказывающего большое уважение к закону и пророкам.
\vs p121 8:5 Апостол Матфей не писал этого Евангелия. Оно было написано Изадором, одним из его учеников, который использовал в работе не только личные воспоминания Матфея о событиях, но и некие записи, которые последний сделал сразу после Распятия. Запись Матфея была сделана на арамейском языке; Изадор писал по\hyp{}гречески. Приписывая труд Матфею, никто не намеревался обманывать. В те времена это был обычный для учеников способ отдать дань уважения учителю.
\vs p121 8:6 Изначальный текст Матфея был несколько изменен и дополнен в 40 г. н.э., еще до того, как он покинул Иерусалим, чтобы начать евангельскую проповедь. Это были личные записки, и последняя копия с них погибла при пожаре Сирийского монастыря в 416 г.
\vs p121 8:7 Изадор бежал из Иерусалима в 70 г. после осады города армиями императора Тита, взяв с собой в Пеллу копию записок Матфея. В 71 г., живя в Пелле, Изадор написал Евангелие от Матфея. У него имелись также и первые четыре из пяти частей рассказа Марка.
\vs p121 8:8 \pc \ublistelem{3.}\bibnobreakspace \bibemph{Евангелие от Луки.} Лука, врач из Антиохии в Писидии, был неевреем, обращенным Павлом, и записал совсем другую историю о жизни Учителя. Он стал последователем Павла и узнал о жизни и учении Иисуса в 47 г. В своих записях Лука во многом старается передать «благодать Господа нашего Иисуса Христа», объединяя все, что узнал от Павла и других. Лука преподносит образ Учителя --- «друга мытарей и грешников». Он собрал свои многочисленные заметки в виде Евангелия лишь после смерти Павла. Записи его были сделаны в 82 г. в Ахаие. Он предполагал написать три книги, повествующие об истории Христа и христианстве, но умер в 90 году, не успев закончить вторую из намеченных работ, «Деяния Апостолов».
\vs p121 8:9 Составляя свое Евангелие, Лука прежде всего опирался на историю жизни Иисуса, рассказанную ему Павлом. Таким образом, в каком\hyp{}то смысле его Евангелие является Евангелием по Павлу. Но у него были и другие источники. Он не только разговаривал с десятками живых свидетелей тех многочисленных эпизодов жизни Иисуса, которые им описаны, но и использовал списки, то есть первые четыре пятых Евангелия от Марка, рассказ Изадора и краткую запись, сделанную в 78 году в Антиохии верующим по имени Седес. У Луки был также сильно искаженный и переписанный много раз список неких заметок, якобы принадлежавших перу апостола Андрея. сильно искаженный
\vs p121 8:10 \pc \ublistelem{4.}\bibnobreakspace \bibemph{Евангелие от Иоанна.} Евангелие от Иоанна рассказывает о деятельности Иисуса в Иудее и окрестностях Иерусалима многое такое, чего нет в других записях. Этот текст и представляет собой так называемое Евангелие от Иоанна, сына Зеведеева, хотя Иоанн и не писал его, а лишь вдохновил его написание. С момента первой записи текст много раз переписывался, дабы придать ему видимость написанного самим Иоанном. В то время, когда делалась запись, у Иоанна имелись списки других Евангелий, и он видел, сколь многое в них упущено; соответственно, в 101 г. он вдохновил своего товарища, Натана, греческого еврея из Кесарии, начать работу над Евангелием. Иоанн пользовался своими воспоминаниями и уже имеющимися тремя списками. Сам Иоанн не делал никаких заметок. Послание, известное как «Первое послание апостола Иоанна», написано самим Иоанном и представляет собой сопроводительное письмо к работе, сделанной Натаном под его руководством.
\vs p121 8:11 \pc Евангелисты описали нам Иисуса таким, каким они видели и помнили его, узнали о нем, и таким образом, как эти далекие события впоследствии были представлены под влиянием христианской теологии Павла. Однако эти записи, какими бы несовершенными они ни были, оказались способными изменить ход истории Урантии на почти что две тысячи лет.
\vsetoff
\vs p121 8:12 \bibemph{Благодарности.} Выполняя свое задание --- восстановить учение и рассказать о деяниях Иисуса из Назарета, --- я свободно пользовался всеми доступными источниками записи и планетарной информации. Моим основным мотивом было желание подготовить запись, которая не только просветила бы ныне живущих людей, но стала бы полезной и всем будущим поколениям. Из огромного запаса доступной мне информации я выбрал то, что более всего подходило для этой цели. Насколько возможно, я пользовался информацией из чисто земных источников. Только в случаях, когда этих источников оказывалось явно недостаточно, я обращался к внечеловеческой памяти. Когда оказывалось, что идеи и концепции, касающиеся жизни и учения Иисуса, приемлемо изложены человеком, я, без колебаний, отдавал предпочтение такому повествованию, отвечающему чисто человеческому складу мышления. Несмотря на то, что я стремился придать словесному воплощению рассказа ту форму, которая в наибольшей степени отвечала бы нашему пониманию подлинного значения и ценности жизни и учения Учителя, я, насколько это было возможно, твердо придерживался в своих рассказах современных земных концепций и образа мышления. Я знаю, что концепции, порожденные разумом человека, будут более понятными и полезными для умов других людей. Когда я оказывался не в силах найти необходимые понятия в человеческих записях или в человеческих выражениях, я пользовался источниками памяти земных существ моего уровня, срединников. Когда же и этого, второго источника информации оказывалось недостаточно, я, без колебаний, прибегал к помощи сверхпланетных источников информации.
\vs p121 8:13 Свод записей, собранный мной и помогавший мне в подготовке этого рассказа о жизни и учении Иисуса, --- помимо записи апостола Андрея --- включает в себя жемчужины мысли и верховные понятия учений Иисуса, собранные от более чем двух тысяч человек, живших на земле со времен Иисуса до того времени, как было написано, а точнее изложено по\hyp{}новому, это откровение. Разрешение прибегнуть к откровению было использовано лишь тогда, когда человеческие источники и представления оказывались не в состоянии предать мысли нужные формы. Комиссия по откровению запрещает мне прибегать к внечеловеческим источникам в поиске любой информации или понятий до тех пор, пока не доказано, что мои попытки найти необходимое понятие исключительно в земных источниках не удались.
\vs p121 8:14 Итак, я, вместе с одиннадцатью моими соратниками\hyp{}срединниками и под наблюдением назначенного Мелхиседека, подготовил этот рассказ в соответствии со своим пониманием того, как лучше расположить и передать эти сведения, хотя большая часть идей и даже некоторые особо выразительные слова, использовавшиеся мною, были рождены в умах множества людей разных рас и многих поколений, живших и живущих ныне на земле. Я скорее был собирателем и редактором этих рассказов, чем рассказчиком. Я, без колебаний, заимствовал те идеи и концепции, в основном человеческие, которые помогали мне создать возможно более выразительное повествование о жизни Иисуса и делали меня способным изложить рассказ о его несравненном учении в наиболее подходящих выражениях, которые могли бы оказать огромную помощь всем, кто стремится понять его и которые вдохновляли бы всю Вселенную. От имени Объединенного Братства Срединников Урантии я выражаю глубокую благодарность всем источникам, которые использовались в последующим повествовании для более глубокого и подробного изложения жизни Иисуса на земле.]
