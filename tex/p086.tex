\upaper{86}{Ранняя эволюция религии}
\author{Вечерняя Звезда}
\vs p086 0:1 Эволюция религии от предшествующей примитивной тяги к почитанию происходила не благодаря откровению. Для обеспечения такого развития совершенно достаточно нормального функционирования человеческого разума при направляющем воздействии шестого и седьмого помощников разума, дара вселенского духа.
\vs p086 0:2 Самый ранний дорелигиозный страх человека перед силами природы постепенно становился религиозным по мере того, как в человеческом сознании природа персонифицировалась, одухотворялась и, в конечном счете, обожествлялась. Религия примитивного типа была поэтому естественным биологическим следствием психологической инертности развивающегося животного разума после того, как такие разумы обрели понятия о сверхъестественном.
\usection{1. Случайность: удача и неудача}
\vs p086 1:1 Помимо естественной тяги к почитанию, корни ранней эволюционной религии лежали в человеческом опыте, связанном со случайностью --- так называемой удачей, повседневными случайными событиями. Первобытный человек добывал пищу охотой. Результаты охоты неизбежно бывают разными, и это порождает тот опыт, который человек расценивает как \bibemph{удачу} и \bibemph{неудачу.} Невезение было важнейшим фактором в жизни мужчин и женщин, постоянно живущих на суровой грани опасного и изнурительного существования.
\vs p086 1:2 Интеллектуальная ограниченность дикарей приводит к концентрации внимания на случайности до такой степени, что удача становится неотъемлемым фактором в их жизни. Первобытные урантийцы вели борьбу за существование, а не за уровень жизни; они вели жизнь, полную опасностей, в которой важную роль играла случайность. Постоянный страх перед неизвестными и невиданными бедствиями висел над этими дикарями, как туча безысходности, начисто затмевающая всякую радость; они жили, постоянно страшась сделать что\hyp{}то, что принесет неудачу. Суеверные дикари всегда боялись полосы везения; в таких удачах они видели некое предвестие бед.
\vs p086 1:3 Эта постоянно присутствующая боязнь неудачи оказывала парализующее воздействие. Зачем упорно трудиться и пожинать неудачу --- не получать ничего взамен чего\hyp{}то, --- когда можно просто плыть по течению и неожиданно встретить удачу --- получить что\hyp{}то просто так? Не думающие люди забывают об удачах --- принимая их как должное, --- но сохраняют болезненные воспоминания о неудачах.
\vs p086 1:4 Древний человек жил в неуверенности и в постоянном страхе перед случайностью --- неудачей. Жизнь была захватывающей игрой случая; существование становилось подобным игре в рулетку. Не удивительно, что более или менее цивилизованные люди по\hyp{}прежнему верят в удачу и упорно проявляют склонность к азартным играм. У первобытного человека чередовались два всепоглощающих интереса: страсть к получению чего\hyp{}то за просто так и страх не получить ничего взамен чего\hyp{}то. И эта азартная игра в существование была самым главным, что интересовало и пленяло ум древнего дикаря.
\vs p086 1:5 У живших значительно позже скотоводов сохранилось такое же отношение к случайности и удаче, а затем и земледельцы все больше и больше осознавали, что урожай непосредственно зависит от множества обстоятельств, которые человек может контролировать лишь в малой степени или не может совсем. Земледелец оказывался жертвой засухи, наводнений, града, бурь, вредных насекомых и болезней растений, а также жары и холода. И в зависимости от того, как все эти природные факторы влияли на благосостояние человека, они рассматривались как удачи или неудачи.
\vs p086 1:6 Это понятие случайности и удачи чрезвычайно распространено в философии всех древних народов. Даже в относящейся к более поздним временам Премудрости Соломона говорится: «Я обратился и увидел, что не проворные выигрывают бег, не сильные достают победы, не мудрые --- хлеб, и не разумные --- богатство, и не искусные --- благорасположение; но судьба и случайность выпадают на долю их всех. Ибо человек не знает своей судьбы; как рыбы попадаются в пагубную сеть и как птицы запутываются в силках, так сыны человеческие низвергаются в бедственные времена, когда они неожиданно обрушиваются на них».
\usection{2. Персонификация судьбы}
\vs p086 2:1 Обеспокоенность была естественным состоянием дикаря. Когда мужчины и женщины становятся жертвами чрезмерной обеспокоенности, то они просто возвращаются к тому, что было естественным состоянием для их далеких предков; а когда обеспокоенность становится поистине мучительной, то она препятствует активной деятельности и неизменно ведет к эволюционным изменениям и биологической адаптации. Боль и страдания играют существенную роль в прогрессивной эволюции.
\vs p086 2:2 Борьба за существование настолько мучительна, что некоторые отсталые племена даже и сейчас стонут и причитают при каждом новом восходе солнца. Первобытный человек постоянно спрашивал: «Кто мучает меня?» Не находя материального источника своих невзгод, он останавливался на духовном объяснении. И так из страха перед таинственным, трепета перед невидимым и ужаса перед неизвестным рождалась религия. Страх перед природой становился, таким образом, фактором в борьбе за существование, во\hyp{}первых, из\hyp{}за случайностей и, во\hyp{}вторых, из\hyp{}за таинственности.
\vs p086 2:3 \pc Первобытный разум был логическим, но не обладал в достаточной степени понятиями для разумных ассоциативных связей; ум дикаря был непросвещенным, совершенно неизощренным. Если одно событие следовало за другим, дикарь рассматривал их как причину и следствие. То, что цивилизованный человек считает суеверием, было просто лишь невежеством дикаря. Человечество не скоро постигло, что между намерениями и результатами не обязательно есть связь. Люди только лишь начинают осознавать, что между действиями и их последствиями есть еще жизненные обстоятельства. Дикарь пытается персонифицировать все неосязаемое и абстрактное, и, таким образом, природа и судьба персонифицируются в образе призраков, а позже --- богов.
\vs p086 2:4 \pc Человек, естественно, склонен верить в то, что он считает наилучшим для себя, в то, что в его непосредственных или долгосрочных интересах; собственные интересы в значительной степени затмевают логику. Разум дикаря отличается от разума цивилизованных людей скорее содержанием, чем сущностью, скорее количественно, чем качественно.
\vs p086 2:5 Но продолжать приписывать трудно постижимому сверхъестественные причины --- это не что иное, как праздный и удобный способ избегать всякой напряженной интеллектуальной деятельности. Удача --- это просто термин, выдуманный для того, чтобы скрывать за ним все необъяснимое в любые эпохи человеческого существования; он обозначает те явления, которые люди не могут или не желают понять. Случайность --- это слово, которое показывает, что человек слишком невежествен или слишком ленив, чтобы установить причины. Люди рассматривают естественное происшествие как случайность или неудачу, только когда они лишены любознательности и воображения, когда у расы недостаточно инициативы и решительности. Исследование жизненных явлений рано или поздно рушит веру человека в судьбу, удачу и так называемые случайности, заменяя ее представлением о вселенной, в которой царит закон и порядок и всем следствиям предшествуют определенные причины. Таким образом, страх существования заменяется радостью жизни.
\vs p086 2:6 Дикарь рассматривает всю природу как живую, как кем\hyp{}то населенную. Цивилизованный человек все еще пинает и проклинает неодушевленные предметы, которые преграждают ему путь и о которые он ударяется. Первобытный человек никогда не относился к чему бы то ни было как случайности; во всем и всегда видел умысел. Для первобытного человека власть рока, удачное стечение обстоятельств, мир духов были такими же неупорядоченными и бессистемными, как и первобытное общество. Удача рассматривалась как определяемая прихотью и настроением реакция мира духов; позже --- как благосклонность богов.
\vs p086 2:7 Но не все религии произошли от анимизма. Одновременно с анимизмом существовали другие представления о сверхъестественном, и эти представления тоже вели к религиозному почитанию. Натурализм --- это не религия; это порождение религии.
\usection{3. Смерть --- необъяснимое явление}
\vs p086 3:1 Смерть была величайшим потрясением для развивающегося человека, непостижимым сочетанием судьбы и тайны. Не священность жизни, а потрясение от смерти внушало страх и, таким образом, успешно взращивало религию. У диких народов смерть обычно была насильственной, так что не насильственная смерть казалась еще более таинственной. Смерть как естественный и ожидаемый конец жизни непонятна сознанию первобытных людей, и потребовались многие и многие века, чтобы человек понял ее неизбежность.
\vs p086 3:2 \pc Древний человек воспринимал жизнь как данность, смерть же он рассматривал как некую кару свыше. У всех народов существуют легенды о людях, которые не умирали, --- отголосок традиций древнего отношения к смерти. В человеческом разуме уже существовало смутное представление о неясном и беспорядочном мире духов, мире, из которого исходило все, что есть в человеческой жизни необъяснимого, и смерть была добавлена к этому длинному перечню необъяснимых явлений.
\vs p086 3:3 Поначалу верили, что всякая человеческая болезнь и естественная смерть вызваны воздействием духов. Даже и в настоящее время некоторые цивилизованные народы рассматривают болезнь как нечто, порожденное «ненавистником», и полагаются на целительное воздействие религиозных обрядов. Более поздние и более сложные теологические построения по\hyp{}прежнему объясняют смерть воздействием духовного мира, что ведет к учениям о первородном грехе и грехопадении человека.
\vs p086 3:4 Именно сознание бессилия перед лицом могущественных сил природы наряду с признанием слабости человека перед карой свыше в виде болезней и смерти побуждали дикаря просить помощи у нематериального мира, в котором он смутно видел источник этих таинственных превратностей жизни.
\usection{4. Представление о преодолении смерти}
\vs p086 4:1 Представление о сверхматериальной стороне человеческой личности родилось из бессознательной и чисто случайной ассоциативной связи между событиями повседневной жизни и сновидениями. Казалось, что если умершего вождя одновременно увидели во сне несколько членов его племени, то это убедительное доказательство того, что старый вождь действительно каким\hyp{}то образом вернулся. Дикарям, просыпавшимся после таких снов в поту, дрожа от ужаса, все это представлялось очень реальным.
\vs p086 4:2 Тот факт, что источником возникновения веры в последующее существование были сны, объясняет склонность всегда представлять себе невидимое в образе чего\hyp{}то видимого. И вскоре это новое представление о последующей жизни, обусловленное сновидениями, становится эффективным противоядием от страха смерти, связанного с биологическим инстинктом самосохранения.
\vs p086 4:3 Внимание древнего человека очень занимало его дыхание, особенно в холодном климате, где при выдохе возникало подобие облака. \bibemph{Дыхание жизни} рассматривалось как феномен, различающий живых и мертвых. Он знал, что дыхание может покинуть тело, а его сновидения, где с ним, пока он спал, происходили совершенно невероятные события, убеждали его, что в человеке есть нечто нематериальное. Самое примитивное представление о человеческой душе как о призраке возникло из системы понятий, связанных с дыханием и снами.
\vs p086 4:4 Со временем дикарь стал представлять себя как сочетание двух компонентов --- тела и дыхания. Дыхание без тела --- это дух, призрак. Имея совершенно определенное человеческое происхождение, призраки, или духи считались сверхчеловеческими сущностями. И эта вера в существование бесплотных духов, казалось, объясняла наличие необычного, удивительного, редкого и непостижимого.
\vs p086 4:5 \pc Примитивное учение о жизни после смерти не обязательно подразумевало веру в бессмертие. Люди, едва умевшие считать до двадцати, едва ли могли представить себе бесконечность и вечность; они, скорее, думали о повторяющихся инкарнациях.
\vs p086 4:6 Вера в переселение душ и реинкарнацию была особенно свойственна оранжевой расе. Эта идея реинкарнации возникла в результате того, что между предками и их потомками замечали наследственное сходство черт. Обычай называть детей в честь дедов и других предков был вызван верой в реинкарнацию. Некоторые народы в более поздние времена верили, что человек умирает от трех до семи раз. Это верование (оставшееся от учений Адама о мирах\hyp{}обителях) и многие другие следы данной через откровение религии обнаруживаются в абсурдных в прочих отношениях учениях варваров двадцатого века.
\vs p086 4:7 \pc У древнего человека не было идеи ада или последующего наказания. Дикарь представлял следующую жизнь точно такой же, как эта, за вычетом всех неудач. Позже возникло представление о различной судьбе, ожидающей хорошие и плохие души, --- рае и аде. Но поскольку многие первобытные народы верили, что человек вступает в следующую жизнь именно таким же, каким он ушел из этой жизни, их не радовала мысль о том, что они станут старыми и дряхлыми. Пожилые предпочитали, чтобы их убили раньше, чем они станут немощными.
\vs p086 4:8 Почти у каждой группы было свое особое представление о судьбе, ожидающей покинувшую тело душу\hyp{}призрак. Греки верили, что у слабых людей должны быть слабые души; поэтому они придумали Гадес как подобающее место для приема таких немощных душ; считалось, что эти хиловатые субъекты отбрасывают более короткую тень. Древние андиты полагали, что дух человека возвращается на родину предков. Китайцы и египтяне некогда верили, что душа и тело остаются вместе. У египтян это приводило к тому, что они старательно сооружали гробницы и стремились сохранить тело. Даже современные народы пытаются приостановить разложение трупа. Иудеи представляли себе, что призрачная копия человека спускается в Шеол; она не может вернуться на землю к живым. Именно они сделали этот важный шаг вперед в учении об эволюции души.
\usection{5. Представление о душе\hyp{}призраке}
\vs p086 5:1 Нематериальную часть человека называли по\hyp{}разному: призрак, дух, тень, фантом, привидение и, наконец, \bibemph{душа.} Душа была тем двойником, которого видел во сне древний человек; она была во всех отношениях точно похожа на самого человека, за исключением того, что она не реагировала на прикосновения. Вера в являвшихся во сне двойников прямо привела к мысли, что у всего одушевленного и неодушевленного есть души так же, как и у людей. Это представление долго поддерживало веру в духов природы; эскимосы по\hyp{}прежнему полагают, что у всего в природе есть дух.
\vs p086 5:2 Душу\hyp{}призрак можно было услышать и увидеть, но нельзя было осязать. Постепенно жизнь, которая являлась во сне, настолько развила и расширила представления людей об этом эволюционирующем мире духов, что смерть, в конце концов, стали рассматривать как «испускание духа». У всех первобытных племен, за исключением тех, кто был лишь немногим выше животных, развилась некая система взглядов о душе. По мере развития цивилизации это суеверное воззрение на душу разрушается, и в новом понятии о душе как о сотворчестве разума человека, знающего Бога, и живущего в нем божественного духа, Внутреннего Настройщика, человек целиком опирается на откровение и личный религиозный опыт.
\vs p086 5:3 Древние люди обычно не проводили различия между понятиями живущего в человеке духа и имеющей эволюционную природу души. У дикаря была полная неясность, является ли бесплотная душа врожденной по отношению к телу или же она --- внешний элемент, в собственности которого находится тело. Так как это явление сложно, а логическое мышление дикарям не было свойственно, то совершенно объяснимы явные несообразности в их представлениях о душах, призраках и духах.
\vs p086 5:4 Считалось, что душа связана с телом, как испускаемый аромат --- с цветком. Древние верили, что душа может покинуть тело разными способами, последствиями чего могут стать:
\vs p086 5:5 \ublistelem{1.}\bibnobreakspace Обычный временный обморок.
\vs p086 5:6 \ublistelem{2.}\bibnobreakspace Сон, естественное сновидение.
\vs p086 5:7 \ublistelem{3.}\bibnobreakspace Состояние комы и потеря сознания, связанные с болезнью и несчастными случаями.
\vs p086 5:8 \ublistelem{4.}\bibnobreakspace Смерть, уход души навсегда.
\vs p086 5:9 \pc Дикарь усматривал в чихании бесплодную попытку души выскочить из тела. В состоянии бодрствования и бдительности тело было способно пресечь попытку побега души. Впоследствии чихание всегда сопровождалось какой\hyp{}нибудь религиозной фразой, такой как «Благослови вас Бог!»
\vs p086 5:10 \pc На раннем этапе эволюции сон рассматривали как доказательство того, что душа\hyp{}призрак может временно покидать тело, и верили, что ее можно призвать обратно, если произносить или выкрикивать имя спящего. При других формах потери сознания считалось, что душа находится на большем удалении и, возможно, пытается уйти навсегда --- надвигающая смерть. В снах усматривали опыт души, который она обрела в течение сна, временно покинув тело. Дикарь верит, что его сны так же реальны, как и любая часть его опыта, обретенного в состоянии бодрствования. У древних было принято будить спящих постепенно, чтобы у души было время возвратиться обратно в тело.
\vs p086 5:11 На протяжении веков люди с благоговейным трепетом относились к ночным видениям, не были исключением и иудеи. Они искренне верили, что во сне с ними говорит Бог, несмотря на запрет, наложенный на эту идею Моисеем. А Моисей был прав, потому что личности из духовного мира не используют обычные сновидения как средство общения с материальными существами.
\vs p086 5:12 Древние верили, что души могут вселяться в животных и даже в неодушевленные предметы. Ярче всего это проявилось в представлениях об оборотне, человеке\hyp{}волке. Днем человек мог быть законопослушным гражданином, но когда он засыпал, его душа могла вселяться в волка или в какое\hyp{}либо другое животное и хищно рыскать по ночам в поисках жертв.
\vs p086 5:13 Первобытные люди считали, что душа связана с дыханием и что ее качества могут передаваться или переноситься через дыхание. Храбрый вождь дышал на новорожденного ребенка, передавая, таким образом, мужество. У ранних христиан обряд сошествия Святого Духа сопровождался дуновением на кандидатов. Сказано в Псалмах: «Словом Господа были созданы небеса, а все силы небесные --- дуновением его уст». Долгое время было принято, чтобы старший сын постарался поймать последний вздох своего умирающего отца.
\vs p086 5:14 Позже начали боятся и почитать тень так же, как и почитали дыхание. В собственном отражении в воде тоже усматривали доказательство существования своего двойника, а к зеркалам относились с суеверным трепетом. Даже и теперь многие цивилизованные люди поворачивают зеркала к стене в случае чьей\hyp{}то смерти. Некоторые отсталые племена по\hyp{}прежнему верят, что создание картин, рисунков, скульптур или статуй забирает из тела всю душу или часть души; поэтому это запрещено.
\vs p086 5:15 Обычно душу отождествляли с дыханием, но в представлении разных народов она помещалась в голове, волосах, сердце, печени, крови, жире. «Голос крови Авеля вопиет из земли» --- это отражение существовавшей некогда веры, что душа находится в крови. Семиты учили, что душа пребывает в слое жира на теле, и у многих племен религия запрещала есть жир животных. Охота за головами, равно как и снятие скальпа, была способом завладеть душой врага. В более поздние времена к глазам стали относиться как к зеркалу души.
\vs p086 5:16 Те, кто придерживались учения о трех или четырех душах, верили, что потеря одной души вызывает недомогание, двух --- болезнь, трех --- смерть. Одна душа жила в дыхании, одна --- в голове, одна --- в волосах и одна --- в сердце. Больным советовали гулять на свежем воздухе в надежде вобрать обратно свои заблудившиеся души. Считалось, что величайшие из шаманов заменяют больную душу заболевшего человека на новую, происходит «новое рождение».
\vs p086 5:17 У детей Бадонана выработалась вера в две души --- дыхание и тень. Древние нодиты считали, что человек состоит из двух сущностей: души и тела. Этот философский взгляд на человека позже отразился в воззрениях греков. Сами греки верили в существование трех душ; растительная находится в желудке, животная --- в сердце, интеллектуальная --- в голове. Эскимосы верят, что человек состоит из трех частей: тела, души и имени.
\usection{6. Среда духов\hyp{}призраков}
\vs p086 6:1 Человек унаследовал природную окружающую среду, приобрел социальную среду и измыслил среду призраков. Государство --- это реакция человека на природную среду, дом и семья --- на социальную среду, церковь --- на иллюзорную среду призраков.
\vs p086 6:2 На очень раннем этапе истории человечества вера в реальность воображаемого мира призраков и духов стала всеобщей, и этот недавно воображенный мир духов приобрел в первобытном обществе реальную силу. Появление такого нового фактора в мыслях и поступках человека навсегда видоизменило умственную и нравственную стороны жизни всего человечества.
\vs p086 6:3 В сию основную посылку, вытекающую из иллюзий и невежества, человеческий страх впихивал все последующие суеверия и религию первобытных народов. Это было единственной религией человека вплоть до времени откровения, и сегодня у многих народов мира есть только такая незрелая эволюционная религия.
\vs p086 6:4 С ходом эволюции удача стала ассоциироваться с хорошими духами, а неудача --- с плохими. Затруднения, связанные с вынужденным приспособлением к меняющейся окружающей среде, рассматривались как неудачи, недовольство духов. Из врожденной тяги первобытного человека к почитанию и из его превратных представлений о случайностях постепенно развивалась религия. У цивилизованного человека существует целая система позволяющая предохранять от случайностей судьбы и преодолевать их; на место вымышленных духов и капризных богов современная наука ставит страховщика с математическими подсчетами.
\vs p086 6:5 Каждое новое поколение посмеивается над глупыми суевериями своих предков и, при этом, само продолжает впадать в логические и религиозные заблуждения, которые дадут просвещенным потомкам основание для дальнейших усмешек.
\vs p086 6:6 \pc Но, наконец\hyp{}то, ум первобытного человека был занят мыслями, которые вышли за пределы его врожденных биологических побудительных мотивов; человек, наконец, приблизился к тому, чтобы создать искусство жизни, основанное на чем\hyp{}то большем, чем реакции на материальные стимулы. Появлялись начала примитивной философской жизненной стратегии. Близилось возникновение сверхъестественных норм жизни, поскольку если в гневе дух посылает неудачу, а в хорошем расположении --- удачу, тогда человек должен соответствующим образом регулировать свое поведение. Развилось, наконец, понятие о том, что хорошо, и том, что плохо, и все это задолго до времен какого\hyp{}либо откровения на земле.
\vs p086 6:7 Возникновение этих понятий положило начало долгой и требующей больших затрат борьбе за то, чтобы умилостивить вечно недовольных духов; рабской зависимости от страха, порождаемого эволюционной религией; долго продолжавшемуся растрачиванию человеческих сил на гробницы, храмы, жертвоприношения и жречество. Пришлось заплатить огромную и ужасающую цену, но она была заплачена не зря, ибо человек действительно начал отличать хорошее от плохого; родилась человеческая этика!
\usection{7. Функции примитивной религии}
\vs p086 7:1 Дикарь испытывал потребность чувствовать себя застрахованным, и поэтому за свой полис магического страхования от неудач он с готовностью платил обременяющие его страховые взносы в виде страха, суеверия, ужаса и даров священнослужителям. Первобытная религия была просто платой за страхование от опасностей, которые крылись в лесах; цивилизованный же человек платит материальные взносы за страхование от несчастных случаев, связанных с техникой, и от критических ситуаций, возможных при современном образе жизни.
\vs p086 7:2 Современное общество изымает страховое дело из сферы священства и религии и переводит его в сферу экономики. Религия все больше занимается страхованием загробной жизни. Современные люди, по крайней мере думающие, уже не платят обременительных страховых взносов, чтобы контролировать удачу. Религия постепенно поднимается на более высокие философские уровни по сравнению с ее прежней функцией в системе страхования от неудач.
\vs p086 7:3 Но эти древние понятия о религии не сделали людей фаталистами и безнадежными пессимистами; они верили, что, по крайней мере, могут что\hyp{}то предпринять, чтобы повлиять на судьбу. Религия страха перед призраками внушила людям, что они должны \bibemph{регулировать свое поведение,} что есть сверхматериальный мир, имеющий власть над человеческой судьбой.
\vs p086 7:4 Современные цивилизованные народы только еще начинают утрачивать страх перед призраками, деятельностью которых объясняли удачу и обычное неравенство существования. Человечество приближается к освобождению от того рабства, к которому приводит вера, что все неудачи происходят из\hyp{}за духов\hyp{}призраков. Но, отказываясь от ложного учения о духах как о причине превратностей судьбы, люди обнаруживают удивительную готовность принять почти столь же ошибочное учение, предлагающее объяснять всякое человеческое неравенство неадекватностью политики, социальной несправедливостью и промышленной конкуренцией. Но новое законодательство, рост благотворительности и дальнейшая реорганизация промышленности, как бы замечательны сами по себе они ни были, не оградят от фактов, связанных с рождением, и несчастных случайностей на протяжении жизни. Только понимание фактов и мудрые действия с учетом законов природы позволят человеку добиться того, чего он хочет, и избежать того, чего он не хочет. Научные знания, ведущие к научно обоснованным действиям, являются единственным средством защиты от так называемых случайных несчастий.
\vs p086 7:5 \pc Промышленность, войны, рабство и гражданское правление возникли как реакция на социальную эволюцию человека в природной среде; подобным же образом религия появилась как реакция на иллюзорную среду воображаемого мира духов. Религия была эволюционным результатом стремления к самосохранению, и она действовала, несмотря на ложность ее первоначальных представлений и крайнюю нелогичность.
\vs p086 7:6 Через посредство мощной и вызывающей трепет силы ложного страха примитивная религия подготовила в человеческом уме почву для пришествия настоящей духовной силы, имеющей сверхъестественную природу, --- Настройщика Мысли. И божественный Настройщик с тех пор постоянно трудился, чтобы превратить страх перед Богом в любовь к Богу. Эволюция может идти медленно, но она неизменно приносит результат.
\vsetoff
\vs p086 7:7 [Передано Вечерней Звездой Небадона.]
