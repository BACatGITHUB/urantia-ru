\upaper{177}{Среда, день отдыха}
\author{Комиссия срединников}
\vs p177 0:1 Каждую среду Иисус и его апостолы имели обыкновение отдыхать от своих трудов, если деятельность по обучению народа не препятствовала этому. В эту среду они позавтракали несколько позже обычного, и в лагере воцарилась зловещая тишина; в первую половину этой утренней трапезы говорили мало. Наконец Иисус сказал: «Я желаю, чтобы сегодня вы отдохнули. Не спеша, обдумайте все, что происходило со времени нашего прихода в Иерусалим, и поразмыслите о том, что скоро предстоит, о чем я ясно вам рассказал. Непременно постарайтесь, чтобы ваши жизни были исполнены истины и чтобы с каждым днем в вас росла благодать».
\vs p177 0:2 После завтрака Учитель сообщил Андрею, что в этот день намерен отсутствовать, и предложил, чтобы апостолам было позволено проводить время по их собственному усмотрению, с той оговоркой, что ни при каких обстоятельствах они не должны входить в ворота Иерусалима.
\vs p177 0:3 Когда Иисус в одиночестве собрался отправиться в горы, Давид Зеведеев обратился к нему со словами: «Ты хорошо знаешь, Учитель, что фарисеи и правители стремятся уничтожить тебя, и все же собираешься один отправиться в горы. Безрассудно так поступать; поэтому я пошлю с тобой троих хорошо подготовленных людей для того, чтобы следить, как бы с тобой не случилась какая\hyp{}нибудь беда». Иисус посмотрел на троих хорошо вооруженных дюжих галилеян и сказал Давиду: «Ты действуешь из лучших побуждений, но заблуждаешься --- ты не можешь понять, что Сыну Человеческому не нужна ничья защита. Ни один человек не поднимет на меня руку вплоть до того часа, когда я буду готов отдать свою жизнь в соответствии с волей моего Отца. Эти люди не могут сопровождать меня. Я желаю пойти один, чтобы я мог пообщаться с Отцом».
\vs p177 0:4 Услышав эти слова, Давид и его вооруженные стражи удалились; но как только Иисус отправился один, к нему подошел Иоанн Марк, держа в руках небольшую корзину с пищей и водой, и сказал, что если он намеревается отсутствовать весь день, то может вдруг проголодаться. Учитель благосклонно улыбнулся Иоанну и протянул руку за корзиной.
\usection{1. Один день наедине с Богом}
\vs p177 1:1 Когда Иисус собрался уже взять корзину с едой из рук Иоанна, молодой человек отважился сказать: «Но, Учитель, ты можешь поставить корзину на землю, отвернуться, пока молишься, и уйти без нее. Кроме того, если бы я пошел с тобой и понес еду, то тебе было бы удобнее молиться, а я непременно буду хранить молчание. Я не буду задавать никаких вопросов и побуду возле корзины, когда ты уйдешь один молиться».
\vs p177 1:2 Произнося эту речь, смелость которой удивила некоторых из находившихся поблизости слушателей, Иоанн позволил себе не отдавать корзину. Так они и стояли, Иоанн и Иисус, оба держась за корзину. Вскоре Учитель отпустил и, глядя на юношу, сказал: «Поскольку ты всем сердцем страстно желаешь пойти со мной, тебе не будет отказано в этом. Мы отправимся вдвоем и хорошо пообщаемся. Ты можешь задать мне любой вопрос, который в твоем сердце, и мы будем ободрять и утешать друг друга. Cначала ты можешь нести еду, а когда устанешь, то я помогу тебе. Пойдем со мной».
\vs p177 1:3 В тот вечер Иисус вернулся в лагерь лишь после заката. Учитель провел этот последний спокойный день на земле, общаясь с этим жаждущим истины юношей и разговаривая со своим Райским Отцом. Это событие стало известно на небе как «день, проведенный молодым человеком в горах с Богом». Этот случай вовеки являет пример готовности Творца по\hyp{}братски общаться с созданием. Даже юноша, если его сердце действительно преисполнено верховного желания, может удостоиться внимания Бога вселенной и наслаждаться исполненным любви общением с ним, наяву испытать незабываемый восторг от пребывания в горах наедине с Богом, причем целый день. Таков был уникальный опыт Иоанна Марка в эту среду в горах Иудеи.
\vs p177 1:4 Иисус много общался с Иоанном, свободно беседовал о делах мира сегодняшнего и грядущего. Иоанн рассказал Иисусу, как сильно он сожалеет о том, что ему недостаточно много лет, чтобы быть одним из апостолов, и выразил свою величайшую признательность за то, что с момента их первой проповеди на реке Иордан возле Иерихона ему позволено было всюду следовать с ними, за исключением путешествия в Финикию. Иисус предостерег юношу, чтобы он не терял присутствия духа перед лицом надвигающихся событий, и заверил его, что он станет великим вестником царства.
\vs p177 1:5 Иоанн Марк испытывал глубокое волнение при воспоминании об этом дне, проведенном с Иисусом в горах, но никогда не забывал о последнем указании Учителя, данном непосредственно перед возвращением в Гефсиманский лагерь. Он сказал: «Итак, Иоанн, мы хорошо провели время, настоящий день отдыха, но смотри, не рассказывай ни одному человеку то, что я говорил тебе». И Иоанн Марк никогда и словом не обмолвился о том, что происходило в этот день, проведенный им с Иисусом в горах.
\vs p177 1:6 На протяжении немногих оставшихся часов земной жизни Иисуса Иоанн Марк никогда не позволял себе надолго терять Учителя из виду. Юноша, таясь, постоянно находился поблизости; он спал только тогда, когда спал Иисус.
\usection{2. Ранний период жизни в семье}
\vs p177 2:1 В этот день в ходе бесед с Иоанном Марком Иисус уделил немало времени сравнению их опыта раннего детства и последующего отрочества. Хотя родители Иоанна были богаче, чем родители Иисуса, многое в их отроческом опыте было очень сходно. Иисус рассказал много такого, что помогло Иоанну лучше понять своих родителей и других членов семьи. Когда юноша спросил Учителя, как тот мог знать, что он окажется «великим вестником царства», Иисус сказал:
\vs p177 2:2 «Я знаю, что ты докажешь свою преданность евангелию царства, потому что я могу положиться на твою сегодняшнюю веру и любовь, когда эти качества основываются на таком воспитании, которое было у тебя дома в детстве. Ты вырос в семье, где родители испытывают друг к другу искреннюю любовь, и поэтому не стал объектом чрезмерной любви, которая может вести к неоправданному возвеличиванию и преувеличенному представлению о собственной значимости. И твоя личность не подверглась пагубному влиянию соперничающих друг с другом родителей, прибегающих к всевозможным, не имеющим отношения к любви ухищрениям, чтобы завоевать твое доверие и привязанность. Родители относились к тебе с такой любовью, которая воспитывает достойную похвалы уверенность в себе и вырабатывает нормальное чувство защищенности. Но еще тебе повезло, что твоим родителям была наряду с любовью присуща мудрость, и именно мудрость побудила их по большей части воздержаться потворствовать слабостям и развлечениям, которые оказываются доступными благодаря богатству, а вместо этого они отправили тебя учиться в синагогальную школу вместе с твоими окрестными сверстниками, и способствовали тому, чтобы ты учился жить в этом мире, позволяя тебе обретать собственный опыт. Ты пришел со своим другом Амосом к Иордану, где мы проповедовали, а ученики Иоанна крестили. Вы оба пожелали пойти с нами. Когда ты вернулся в Иерусалим, твои родители дали согласие; родители Амоса не позволили ему; они так сильно любили своего сына, что отказали ему в получении того благословенного опыта, который обрел ты, даже того, который есть у тебя сегодня. Амос мог бы присоединиться к нам, сбежав из дома, но, поступив так, он нанес бы обиду любви, и пожертвовал бы преданностью. Даже если такой образ действий был бы разумным, пришлось бы заплатить слишком дорого за опыт, независимость и свободу. Мудрые родители, такие, как твои, стараются, чтобы их детям по достижении твоего возраста не приходилось наносить обиду любви или подавлять преданность ради того, чтобы, обрести независимость и наслаждаться живительной свободой.
\vs p177 2:3 Любовь, Иоанн, это верховная реалия вселенной, когда она исходит от премудрых, но в том виде, в каком она проявляется в опыте смертных родителей, это опасное и зачастую полуэгоистичное чувство. Когда ты вступишь в брак и будешь иметь собственных детей, которых предстоит воспитывать, убедись, что твоя любовь руководствуется мудростью и направляется разумом.
\vs p177 2:4 Твой друг Амос верит в это евангелие царства так же твердо, как и ты, но я не могу всецело положиться на него; я не могу с уверенностью знать, что он будет делать в последующие годы. Его домашняя жизнь в детстве была не такой, какая может воспитать абсолютно надежного человека. Амос слишком похож на одного из апостолов, не получившего нормального, исполненного любви, мудрого домашнего воспитания. Вся твоя последующая жизнь будет более счастливой и предсказуемой, потому что ты провел свои первые восемь лет в нормальной семье, где жизнь была хорошо упорядочена. У тебя сильный и твердый характер потому, что ты вырос в доме, где господствовала любовь и царила мудрость. Подобное воспитание в детстве вырабатывает тот тип преданности, который дает мне уверенность, что ты до конца будешь идти тем путем, на который ступил».
\vs p177 2:5 Больше часа Иисус и Иоанн продолжали говорить о семейной жизни. Далее Учитель объяснил Иоанну, что все ранние интеллектуальные, общественные, моральные и даже духовные представления ребенка полностью зависят от его родителей и связанной с ними семейной жизни, поскольку семья олицетворяет для маленького ребенка все, откуда он может поначалу узнать об отношениях как человеческих, так и божественных. Ребенок должен извлекать свои первые впечатления о вселенной из материнской заботы, его первые понятия о небесном Отце полностью зависят от отца земного. Последующая жизнь ребенка бывает счастливой или несчастной, легкой или тяжелой в зависимости от ментальной и эмоциональной жизни в детстве, которая обусловлена социальными и духовными отношениями в семье. То, что происходит в течение первых нескольких лет существования, оказывает огромное влияние на всю жизнь человека в посмертии.
\vs p177 2:6 \P\ По нашему искреннему убеждению, евангелие учения Иисуса, основанного на отношениях отца и ребенка, едва ли получит повсеместное признание во всем мире до тех пор, пока семейная жизнь современных цивилизованных народов не исполнится большей любви и мудрости. Несмотря на то, что родители двадцатого века обладают большими знаниями и ближе стоят к пониманию истины, как сделать семью совершеннее и облагородить семейную жизнь, но факт остается фактом, что очень немногие семьи так же хороши для воспитания мальчиков и девочек, как семья Иисуса в Галилее и семья Иоанна Марка в Иудее, хотя принятие евангелия Иисуса немедленно приводит к улучшению семейной жизни. Исполненная любви жизнь разумной семьи и искренняя преданность истинной религии оказывают глубокое обоюдное воздействие друг на друга. Такая семейная жизнь укрепляет религию, а истинная религия всегда прославляла семью.
\vs p177 2:7 Правда, в современных, лучше организованных семьях практически исчезли многие достойные порицания факты, оказывающие тормозящее воздействие, и другие сковывающие черты, присущие еврейским семьям прошлого. Безусловно, стало больше непосредственной свободы и гораздо больше личной независимости, но эта свобода не сдерживается любовью, не мотивируется преданностью и не направляется разумной дисциплиной, диктуемой мудростью. Пока мы учим ребенка молиться: «Отче наш, сущий на небесах», на всех земных отцах лежит громадная ответственность --- жить и организовывать семейную жизнь так, чтобы слово \bibemph{отец} бережно сохранялось в умах и сердцах подрастающих детей.
\usection{3. Днем в лагере}
\vs p177 3:1 Апостолы провели большую часть этого дня, прогуливаясь по Масличной горе и беседуя с учениками, которые расположились в этом же лагере, но после полудня им все сильнее хотелось, чтобы Иисус вернулся. Время шло, и они все больше начинали тревожиться за его безопасность; без него они чувствовали себя невыразимо одинокими. Весь день много спорили о том, следовало ли отпускать Учителя в горы одного в сопровождении лишь мальчика\hyp{}посыльного. Хотя свои мысли никто открыто не выражал, среди них не было ни одного, за исключением Иуды Искариота, кто не желал бы сам быть на месте Иоанна Марка.
\vs p177 3:2 \P\ Примерно в середине дня Нафанаил произнес перед полудюжиной апостолов и примерно столькими же учениками речь о «Верховном желании», которая заканчивалась так: «Плохо то, что большинство из нас нерешительны. Мы любим Учителя не так, как он любит нас. Если бы все мы захотели пойти с ним так же сильно, как захотел этого Иоанн Марк, он непременно взял бы нас всех. Мы просто стояли рядом, а в это время к Учителю приблизился юноша и протянул ему корзину, но когда Учитель взялся за нее, юноша ее не отпустил. Так что Учитель оставил нас здесь и отправился в горы с корзиной и юношей, вот и все».
\vs p177 3:3 \P\ Около четырех часов к Давиду Зеведееву прибыли гонцы и принесли ему весть от его матери из Вифсаиды и от матери Иисуса. За несколько дней до этого Давид решил, что первосвященники и правители убьют Иисуса. Давид знал, что они полны решимости погубить Учителя, и был практически убежден, что Иисус не воспользуется своей божественной силой для своего спасения и не позволит своим последователям применить силу для его защиты. Придя к такому выводу, он без промедления отправил вестника к своей матери, призывая ее немедленно прибыть в Иерусалим и привести Марию, мать Иисуса, и всех членов его семьи.
\vs p177 3:4 Мать Давида поступила так, как просил ее сын, и теперь гонцы вернулись к Давиду с известием о том, что его мать и вся семья Иисуса находятся на пути к Иерусалиму и должны прибыть или завтра поздно вечером, или послезавтра рано утром. Поскольку Давид сделал это по собственной инициативе, то счел за лучшее хранить все в тайне. Поэтому он никому не рассказал, что семья Иисуса находится на пути к Иерусалиму.
\vs p177 3:5 \P\ Вскоре после полудня в лагерь прибыли более двадцати греков, встречавшихся с Иисусом и двенадцатью апостолами в доме у Иосифа Аримафейского, и Петр и Иоанн несколько часов совещались с ними. Эти греки, по крайней мере, некоторые из них, были весьма искушенными в знании царства, поскольку в Александрии их наставлял Родан.
\vs p177 3:6 В тот вечер, вернувшись в лагерь, Иисус беседовал с этими греками и дал бы двадцати грекам посвящение точно так же, как дал посвящение семидесяти вестникам, если бы такой поступок не взволновал его апостолов и многих из его наиболее выдающихся учеников.
\vs p177 3:7 \P\ Пока все это происходило в лагере, в Иерусалиме первосвященники и старейшины были поражены тем, что Иисус не вернулся, чтобы снова обратиться к народу. Правда, накануне, уходя из храма, он сказал: «Я оставляю вам дом ваш пуст». Но они не могли понять, почему он с готовностью отказался от огромного преимущества, которое давало ему доброжелательное настроение народа. Хотя они боялись, что он возбудит мятеж народа, последние слова Учителя, обращенные к толпе, были призывом всеми разумными способами поддерживать согласие с властью тех, «кто сидят на месте Моисея». Но в городе в этот день все были очень заняты, поскольку одновременно и готовились к Пасхе, и разрабатывали свои планы уничтожения Иисуса.
\vs p177 3:8 \P\ В лагерь приходило не много людей, поскольку его местоположение тщательно хранилось в тайне всеми, кто знали, что Иисус предполагал оставаться там вместо того, чтобы каждый вечер уходить в Вифанию.
\usection{4. Иуда и первосвященники}
\vs p177 4:1 Вскоре после того, как Иисус и Иоанн Марк покинули лагерь, Иуда Искариот исчез, оставив своих собратьев, и вернулся лишь в конце дня. Этот сбитый с толку и исполненный недовольства апостол, несмотря на конкретную просьбу своего Учителя воздерживаться от посещения Иерусалима, спешно отправился на встречу с врагами Иисуса в дом первосвященника Каиафы. Проходило неофициальное собрание синедриона, назначенное на это утро в начале одиннадцатого. Это собрание было созвано для того, чтобы обсудить характер обвинений, которые предстояло выдвинуть против Иисуса, и решить, каким образом передать его римским властям, чтобы обеспечить необходимое светское утверждение смертного приговора, который они ему уже вынесли.
\vs p177 4:2 Накануне Иуда поведал кое\hyp{}кому из своих родственников и некоторым саддукеям --- друзьям семьи его отца, что он пришел к выводу, что Иисус, хотя и был мечтателем и идеалистом, имеющим наилучшие намерения, не является ожидаемым спасителем Израиля. Иуда заявил, что очень хотел бы найти какой\hyp{}нибудь приемлемый способ покинуть это движение. Его друзья льстиво заверили его, что еврейские правители приветствовали бы такой поступок как выдающееся событие и что ничто не оказалось бы для него недосягаемым. Они дали ему понять, что он тотчас получил бы от синедриона высокие почести и смог бы, наконец, смыть пятно от своей хоть и вызванной наилучшими побуждениями, но «прискорбной связи с невежественными галилеянами».
\vs p177 4:3 Иуда не мог до конца поверить в то, что могущественные деяния Учителя творятся силой принца дьяволов, но теперь он был совершенно уверен, что Иисус не воспользуется своей силой, чтобы возвысить себя; он, наконец, пришел к убеждению, что Иисус позволит еврейским правителям уничтожить себя, и для него была невыносимо унизительна мысль, что он связал себя с движением, которому уготовано поражение. Он отказывался мириться с мыслью об очевидной неудаче. Он хорошо понимал твердый характер своего Учителя и остроту его величественного и милосердного ума, однако ему нравилось, пусть даже не во всем, соглашаться с предположением одного из своих родственников о том, что Иисус, будучи фанатиком, имеющим наилучшие намерения, в действительности, вероятно, психически нездоров; что он всегда производил впечатление странного и непонятного человека.
\vs p177 4:4 И теперь, как никогда раньше, Иуда почувствовал, что начинает испытывать необъяснимое возмущение из\hyp{}за того, что Иисус никогда не назначал его на более почетный пост. Все это время он ценил предоставленную ему честь быть апостольским казначеем, но теперь стал чувствовать, что его не ценят; что его способности не получают признания. Его вдруг охватило негодование из\hyp{}за того, что Петр, Иаков и Иоанн имели честь близко общаться с Иисусом, и, направляясь к дому первосвященника, он в большей степени был настроен поквитаться с Петром, Иаковом и Иоанном, нежели озабочен мыслями об измене Иисусу. Но постепенно, именно тогда главное место в его уме начала занимать новая и всепоглощающая мысль: он отправился добывать себе почести, и если одновременно с этим можно еще и поквитаться с теми, кто вызвал величайшее разочарование в его жизни, то --- тем лучше. Он был охвачен в одно и то же время смятением, гордостью, отчаянием и решимостью. Таким образом, совершенно очевидно, что не ради денег направлялся Иуда в дом Каиафы договариваться о предательстве Иисуса.
\vs p177 4:5 Подходя к дому Каиафы, Иуда принял окончательное решение покинуть Иисуса и его сподвижников\hyp{}апостолов; и решив, таким образом, бросить дело царства небесного, он вознамерился добыть себе как можно больше тех почестей и славы, которые, как он думал раньше, впервые связав себя с Иисусом и новым евангелием царства, должны были когда\hyp{}нибудь выпасть на его долю. Всем апостолам когда\hyp{}то было свойственно то же честолюбие, что и Иуде, но со временем они научились восхищаться истиной и любить Иисуса, по крайней мере, больше, чем Иуда.
\vs p177 4:6 Изменника представлял Каиафе и еврейским правителям его двоюродный брат, который объяснил, что Иуда, поняв свою ошибку, заключавшуюся в том, что позволил ввести себя в заблуждение хитроумным учением Иисуса, пришел к выводу, что желает публично и официально отречься от своей связи с галилеянином и в то же время просить вернуть ему доверие и расположение его иудейских собратьев. Представляющий Иуду человек стал дальше объяснять, что Иуда осознал, что для мира Израиля будет лучше, если Иисус будет взят под стражу. И в знак своего сожаления об участии в таком ошибочном движении и в доказательство искренности своего возврата к учениям Моисея он пришел предложить себя синедриону в качестве человека, который может помочь начальнику стражи, имеющему приказ об аресте Иисуса, негласно взять того под стражу и избежать, таким образом, какой\hyp{}либо опасности возникновения смуты в народе или же необходимости отложить этот арест до конца Пасхи.
\vs p177 4:7 Закончив говорить, двоюродный брат представил Иуду, который приблизился к первосвященнику и сказал: «Я сделаю все, что пообещал мой двоюродный брат, но что вы готовы дать мне за эту услугу?» Иуда, казалось, не уловил выражения презрения и даже отвращения, промелькнувшего на лице жестокосердного и тщеславного Каиафы; он был слишком озабочен собственной славой и жаждой удовлетворения своего тщеславия.
\vs p177 4:8 И тогда Каиафа сказал, свысока глядя на изменника: «Иуда, иди к начальнику стражи и договорись с ним, как доставить к нам твоего Учителя сегодня или завтра вечером, а когда он будет предан тобой в наши руки, ты получишь свою награду за эту услугу». Услышав это, Иуда удалился от первосвященников и правителей и стал обсуждать с начальником храмовой стражи, каким образом схватить Иисуса. Иуда знал, что Иисуса тогда не было в лагере, и понятия не имел о том, когда тот вернется в этот вечер, поэтому они решили схватить Иисуса на следующий вечер (в четверг), когда жители Иерусалима и все собравшиеся паломники лягут спать.
\vs p177 4:9 Иуда вернулся в лагерь к своим сподвижникам, опьяненный мыслями о величии и славе, которых у него не было уже давно. В свое время он присоединился к Иисусу в надежде стать когда\hyp{}нибудь великим человеком в новом царстве. В конце концов, он понял, что не будет никакого нового царства в том виде, как он себе его представлял. Но его радовало то, что он оказался достаточно прозорливым, чтобы променять свое разочарование из\hyp{}за невозможности обрести славу в ожидаемом новом царстве на немедленное приобретение славы и награды при старом строе, который, как он теперь полагал, уцелеет и который, как он был уверен, уничтожит Иисуса и все то, за что тот ратовал. В конечном счете, что касается сознательных мотивов, то предательство Иисуса Иудой было трусливым поступком эгоистичного перебежчика, думающего лишь о собственной безопасности и прославлении, не обращая внимания на последствия его поведения для его Учителя и его бывших сподвижников.
\vs p177 4:10 Но все это было не ново. Иуде давно было присуще это нарочитое, упорное, эгоистичное и мстительное сознание, побуждающее его постоянно вынашивать в уме и лелеять в сердце злое и пагубное чувство мести и предательства. Иисус любил Иуду точно так же, как и других апостолов, и доверял ему так же, как и другим, но Иуда в ответ не смог проявить истинное доверие и не испытывал искренней любви. И насколько опасным может стать честолюбие, когда оно целиком связано со своекорыстием и в первую очередь движимо гнетущим и давно сдерживаемым чувством мести! Какой разрушительной силой оказывается разочарование в жизни тех неумных людей, которые, сосредоточившись на теневых и мимолетных соблазнах времени, закрывают глаза на высшие и действительно подлинные непреходящие достижения вечных миров божественных ценностей и истинных духовных реалий. Иуда умом страстно желал мирской славы и постепенно полюбил это желание всем сердцем; другие апостолы умом также жаждали той же мирской славы, но сердцем они любили Иисуса и прикладывали все усилия к тому, чтобы научиться любить те истины, которым он их учил.
\vs p177 4:11 Еще с тех пор, как Ирод обезглавил Иоанна Крестителя, Иуда подсознательно был критически настроен по отношению к Иисусу, хотя в то время он этого не осознавал. В глубине души Иуда всегда возмущался тем, что Иисус не спас Иоанна. Не следует забывать, что прежде, чем стать последователем Иисуса, Иуда был учеником Иоанна. И все эти накапливавшиеся человеческие обиды и горькие разочарования, которые копились в душе Иуды и постепенно окрашивались ненавистью, ясно оформились в подсознании и готовы были вылиться и полностью захлестнуть его. И это тогда, когда он решился оградить себя от поддерживающего влияния своих собратьев и в то же время подвергся воздействию хитрых измышлений и тонких насмешек врагов Иисуса. Каждый раз, когда Иуда позволял воспарить своим надеждам, а Иисус их рушил каким\hyp{}нибудь поступком или словом, на сердце Иуды оставался шрам от горькой обиды; и поскольку эти шрамы множились, вскоре это многократно раненное сердце утратило всякую истинную привязанность к тому, кто был источником этих неприятных переживаний человека, действующего из лучших побуждений, но трусливого и эгоцентричного. Иуда был трусом, хотя он и не понимал этого. Поэтому он всегда был склонен приписывать Иисусу трусость в качестве мотива, побуждавшего того так часто отказываться от власти или славы, когда они, казалось бы, были легко досягаемы. И каждый смертный человек прекрасно знает, как любовь, даже некогда искренняя, со временем может в результате разочарований, ревности и постоянных обид превратиться в настоящую ненависть.
\vs p177 4:12 Наконец\hyp{}то, первосвященники и старейшины смогли вздохнуть с облегчением. Им не придется публично хватать Иисуса, а союз с изменником Иудой гарантировал, что Иисус не избежит их судилища, как это много раз удавалось ему в прошлом.
\usection{5. Последний вечер общения}
\vs p177 5:1 Поскольку была среда, в лагере был вечер общения. Учитель пытался приободрить своих удрученных апостолов, но это было почти невозможно. Все они начинали понимать, что надвигаются тягостные и прискорбные события. Они не смогли повеселеть даже когда Учитель стал вспоминать полные событий годы их исполненного любви общения. Иисус подробно расспросил всех апостолов об их семьях и, посмотрев в сторону Давида Зеведеева, осведомился, не получал ли он в последнее время вестей от матери, младшей сестры или других членов семьи. Давид потупил взгляд; он боялся отвечать.
\vs p177 5:2 В этот раз Иисус предупредил своих последователей, чтобы они осторожно относились к поддержке толпы. Он вспомнил об их опыте в Галилее, когда огромные массы людей с восторгом следовали за ними, а потом так же рьяно обратились против них и вернулись к своим прежним привычным верованиям и образу жизни. И затем он сказал: «Итак, вы не должны обманываться насчет настроения огромных толп, которые слушали нас в храме и, казалось, поверили в наши учения. Эти толпы слушают истину и верят в нее лишь поверхностно, умом, но немногие из них позволяют словам дойти до сердца и пустить там живые корни. Когда действительно приходит беда, нельзя полагаться на поддержку тех, кто приняли евангелие только умом и не пережили его сердцем. Когда правители евреев договорятся уничтожить Сына Человеческого и придут к полному единодушию, вы увидите, как толпа или разбежится в страхе, или же будет стоять в безмолвном изумлении в то время, как обезумевшие и ослепленные правители поведут на казнь учителей евангелия истины. А затем, когда на вас обрушатся бедствия и преследования, одни, питающие, как вы считали, любовь к истине, рассеются, а другие отрекутся от евангелия и бросят вас. Некоторые, кто были очень близки нам, уже приняли решение уйти. Сегодня вы отдыхали, готовясь к надвигающимся на нас временам. Будьте же начеку и молитесь, чтобы завтра вы укрепились в преддверии близящихся дней».
\vs p177 5:3 В лагере царила необъяснимо напряженная атмосфера. Появлялись и исчезали безмолвные вестники, общавшиеся только с Давидом Зеведеевым. До исхода вечера некоторые уже знали, что Лазарь спешно бежал из Вифании. После возвращения в лагерь Иоанн Марк пребывал в многозначительном молчании, несмотря на то, что провел целый день в обществе Учителя. Все усилия склонить его к беседе лишь ясно показывали, что Иисус велел ему ничего не рассказывать.
\vs p177 5:4 Их пугало даже хорошее настроение Учителя и его необычная общительность. Все они ясно чувствовали надвигающееся одиночество, которое, как они понимали, вот\hyp{}вот обрушится на них --- абсолютно внезапно и с ужасающей неотвратимостью. Они лишь смутно понимали, что именно предстоит, и никто не чувствовал в себе готовности посмотреть в лицо испытаниям. Учитель весь день отсутствовал; им ужасно не хватало его.
\vs p177 5:5 Вечером в эту среду состояние их духа упало до самой низкой точки и оставалось таковым до часа смерти их Учителя. Хотя следующий день еще на сутки приближал трагическую пятницу, но все же он был с ними, и тревожные часы проходили не столь удручающе.
\vs p177 5:6 Близилась полночь, когда Иисус, зная, что это будет последняя ночь, проведенная на земле вместе с его избранной семьей, сказал, отправляя их спать: «Ложитесь спать, братья мои, и да будет мир с вами, пока мы не встанем завтра и не наступит еще один день, чтобы исполнить волю Отца и возрадоваться от сознания того, что мы его сыновья».
