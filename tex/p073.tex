\upaper{73}{Эдемский Сад}
\author{Солония}
\vs p073 0:1 Упадок культуры и духовная нищета вследствие предательства Калигастии и последующие общественные беспорядки мало повлияли на физический и биологический уровень народов Урантии. Несмотря на культурный и моральный регресс, который стремительно последовал сразу же после предательства Калигастии и Далигастии, эволюция органического мира продолжала идти быстрыми темпами. И почти сорок тысяч лет назад в истории планеты наступило время, когда Носители Жизни, находившиеся в то время по долгу службы на Урантии, зафиксировали, что чисто биологическое развитие рас Урантии приближается к своей высшей точке. Мелхиседеки\hyp{}исполнители, согласившись с этой точкой зрения, с готовностью поддержали обращение Носителей Жизни к Всевышним Эдентии, в котором содержалась просьба произвести инспекцию Урантии, а именно: получить разрешение отправить туда реализаторов биологического подъема, Материального Сына и Материальную Дочь.
\vs p073 0:2 Этот запрос был адресован Всевышним Эдентии, поскольку с момента ниспровержения Калигастии и временного отсутствия власти в Иерусеме многие дела Урантии находились в их непосредственном ведении.
\vs p073 0:3 Табамантия, полновластный надзиратель ряда десятичных экспериментальных миров, отправился инспектировать эту планету, и после изучения состояния развития рас должным образом рекомендовал, чтобы на Урантию направили Материальных Сынов. Не прошло и ста лет после этой инспекции, как туда прибыли Адам и Ева --- Материальные Сын и Дочь локальной системы, и взялись за трудное дело --- попытаться распутать запутанные дела планеты, из\hyp{}за бунта отставшей в своем развитии и находящейся по приговору в духовной изоляции.
\usection{1. Нодиты и амадониты}
\vs p073 1:1 Для обычной планеты прибытие Материального Сына, как правило, означает приближение замечательной эпохи открытий, материального прогресса и интеллектуального просвещения. Постадамическая эра --- это эпоха великих научных достижений для большинства миров, но только не для Урантии. Хотя эта планета и была населена расами, которые подходили для этого по своему физическому развитию, но ее племена погрязли в дикости и находились в моральном упадке.
\vs p073 1:2 Спустя десять тысяч лет после бунта все преимущества правления Принца были сведены на нет, расы в этом мире немного бы потеряли, если бы этот заблудший Сын никогда и не приходил бы на Урантию. Только среди нодитов и амадонитов сохранялись традиции Даламатии и культуры, связанной с Планетарным Принцем.
\vs p073 1:3 \bibemph{Нодиты} были потомками членов штата Принца, примкнувших к бунту; свое имя они получили от Нода, их первого начальника, который одно время был председателем комиссии по производству и торговле Даламатии. \bibemph{Амадониты} же происходили от тех андонитов, которые остались верными Вану и Амадону. Слово «амадониты» обозначает скорее культурную и религиозную принадлежность, чем принадлежность к какой\hyp{}либо расе. С расовой точки зрения, амадониты являются, по существу, \bibemph{андонитами.} Термин «нодиты» обозначает как культурную, так и расовую принадлежность, ибо сами нодиты составляют восьмую расу Урантии.
\vs p073 1:4 Традиционно между нодитами и амадонитами существовала вражда. Эта вражда постоянно разгоралась, как только представители этих двух групп пытались заняться каким\hyp{}либо общим делом. Даже позднее, в делах, связанных с Эдемом, им было чрезвычайно трудно работать вместе в мире и согласии.
\vs p073 1:5 Вскоре после распада Даламатии последователи Нода разделились на три большие группы. Центральная группа оставалась в районе их первоначального расселения вблизи верховьев рек Персидского залива. Восточная группа заселила гористую местность Элама прямо на восток от долины Евфрата. Западная группа обосновалась на сирийском северо\hyp{}восточном побережье Средиземного моря и на прилегающих территориях.
\vs p073 1:6 Нодиты легко смешались с сангикскими расами и оставили дееспособное потомство. А некоторые из потомков мятежных даламатийцев впоследствии присоединились к Вану и его верным последователям, живущим на землях к северу от Месопотамии. Здесь, вблизи озера Ван и южного Каспия и возникло новое племя: нодиты, смешавшись с амадонитами, стали «могущественными мужами древности».
\vs p073 1:7 Перед прибытием Адама и Евы эти группы --- нодиты и амадониты --- были наиболее культурными и передовыми расами на земле.
\usection{2. Планирование Сада}
\vs p073 2:1 В течение почти ста лет перед инспекцией Табамантии Ван и его сподвижники из своих горных центров мировой культуры и этики проповедовали о пришествии обетованного Сына Бога, реализатора подъема, учителя истины, достойного заменить вероломного Калигастию. Хотя большинство людей, населяющих в то время мир, проявляли мало интереса или были совершенно безразличны к подобным предсказаниям, те, кто непосредственно соприкасался с Ваном и Амадоном, отнеслись к таким высказываниям серьезно и начали готовиться к реальной встрече обетованного Сына.
\vs p073 2:2 Ван рассказал своим ближайшим соратникам историю о Материальных Сынах Иерусема; то, что он знал о них еще до того, как прибыл на Урантию. Ему было хорошо известно, что эти адамические Сыны жили в простых, но прелестных домах, расположенных в садах, и за восемьдесят три года до прибытия Адама и Евы он предложил своим соратникам вместе посвятить себя возвещению пришествия и заложить сад для встречи с ними.
\vs p073 2:3 Из своих горных пристанищ и разбросанных повсюду шестидесяти одного поселения Ван и Амадон набрали больше трех тысяч добровольных и энергичных работников, которые на торжественном собрании решили посвятить себя подготовке к пришествию обетованного --- или, по крайней мере, долгожданного --- Сына.
\vs p073 2:4 Ван разделил добровольцев на сто групп, поставив во главе каждой руководителя и дав ему помощника, который служил офицером связи в личном штабе Вана; Амадон выполнял функции его собственного помощника. Эти группы со всей ответственностью начали подготовительную работу, а одна группа --- комитет, которому было поручено найти место для Сада, --- отправилась на поиски идеально подходящего места.
\vs p073 2:5 \pc Хотя Калигастия и Далигастия были практически лишены большей части своей силы и не могли творить зло, они делали все возможное, чтобы сорвать и воспрепятствовать работе по созданию Сада. Но их злонамеренные махинации, в большинстве своем, уже в самом начале были сведены на нет благородными усилиями почти десяти тысяч верных срединников, которые без устали трудились над развитием этого предприятия.
\usection{3. Местоположение Сада}
\vs p073 3:1 Комитет, которому было поручено найти место для Сада, отсутствовал почти три года. Он сообщил, что склоняется в пользу трех возможных вариантов: остров в Персидском заливе; район междуречья, там впоследствии создали второй сад, и узкий длинный полуостров --- почти остров --- выступающий на запад от восточного побережья Средиземного моря.
\vs p073 3:2 Комитет почти единогласно высказался за третий вариант. Место было выбрано, и в течение двух лет на этот средиземноморский полуостров перенесли культурный центр мира, включая дерево жизни. Когда прибыли Ван и его команда, все жители полуострова, кроме одной группы, добровольно покинули его.
\vs p073 3:3 \pc На средиземноморском полуострове были целебные для здоровья климатические условия, без резких колебаний температуры. Ровная погода обусловливалась тем, что область была окружена горами и, по существу, представляла собой остров во внутреннем море. Если для соседних нагорий были характерны обильные дожди, то собственно в Эдеме дожди шли редко. Но каждую ночь, благодаря развитой сети ирригационных каналов, «поднимался туман» и освежал растительность Сада.
\vs p073 3:4 Береговая линия этого куска суши высоко поднималась над морем, а перешеек, соединяющий его с материком был в самом узком месте шириной всего двадцать семь миль. Большая река, орошающая Сад, начиналась в горных районах полуострова и текла на восток по перешейку к материку, а затем, по низинам Месопотамии к открытому морю. Она питалась четырьмя притоками, берущими свое начало на прибрежных возвышенностях Эдемского полуострова и являющимися теми «четырьмя истоками» реки, которые «вытекают из Эдема»; позднее их стали путать с притоками рек, окружающих второй сад.
\vs p073 3:5 Горы вокруг Эдема изобиловали драгоценными камнями и металлами, хотя на это почти не обращали внимание. Главной была идея прославления садоводства и возвеличивания земледелия.
\vs p073 3:6 Место, выбранное для Сада, было, пожалуй, самым прелестным уголком на всей земле, а климат его был идеальным. Больше нигде не было места, которое столь легко можно было бы превратить в рай ботанического мира. В этом месте было собрано все лучшее цивилизации Урантии. За пределами полуострова лежал мир темноты, невежества и дикости. Эдем был единственным светлым пятном Урантии; он действительно был красивейшим местом и вскоре превратился в настоящую поэму торжества утонченного и доведенного до совершенства ландшафта.
\usection{4. Основание Сада}
\vs p073 4:1 Когда Материальные Сыны, реализаторы биологического подъема, начинают обитать в эволюционирующем мире, место их пребывания часто именуют Эдемским Садом, поскольку оно характеризуется красотой цветов и великолепием растительного мира Эдентии, главной планеты созвездия. Ван был хорошо осведомлен об этих обычаях и надлежащим образом позаботился, чтобы весь полуостров был бы отдан под Сад. Прилегающие земли материка предполагалось отвести под пастбища и животноводство. В парке не могли находиться никакие другие животные, кроме птиц и различных одомашненных видов. Согласно инструкциям Вана, Эдем должен был быть садом и только садом. Никогда ни одно животное не было убито на его территории. Все мясные продукты, потребляемые работниками Сада в пищу в течение всех лет его создания, привозились из стад, содержавшихся под охраной на материке.
\vs p073 4:2 Первоочередной задачей было возведение кирпичной стены поперек перешейка. После того, как это выполнили, стало возможным беспрепятственно производить работы по благоустройству ландшафта и строительству домов.
\vs p073 4:3 Зоологический сад устроили сразу за основной стеной, построив на некотором расстоянии от нее меньшую стену. В пространстве между двумя стенами обитали всевозможные дикие животные, дополнительно защищая Сад от враждебных нападений. В зверинце было двенадцать больших отделений, и огражденные стенами дороги, проложенные между этими участками, вели к двенадцати воротам Сада, реке и прилегающим к ней пастбищам, которые занимали центральную территорию.
\vs p073 4:4 Сад обустраивали только добровольцы, наемные рабочие никогда для этого не привлекались. Они возделывали Сад и пасли свои стада, дававшие им пропитание; кроме того, им приносили продукты верующие соседи. И это грандиозное предприятие было доведено до конца, несмотря на трудности, связанные с беспорядками в мире в эти смутные времена.
\vs p073 4:5 Однако когда Ван, не зная, как скоро могут прибыть долгожданные Сын и Дочь, предложил молодежи также обучаться работе по осуществлению этого предприятия, в случае, если прибытие Сынов будет отложено, --- это вызвало глубокое разочарование. Предложение было воспринято как показатель отсутствия веры у Вана, что привело к сильным волнениям, в результате многие покинули свою работу. Но Ван продолжал осуществлять свой план подготовки, набирая, тем временем, молодых добровольцев на место дезертиров.
\usection{5. Дом Сада}
\vs p073 5:1 В центре Эдемского полуострова находилась святыня Сада, храм Отца Всего Сущего, выстроенный из благородного камня. К северу от него был устроен центр управления, к югу --- построены дома для работников и их семей. На западе отводилась территория для школ, предполагалось, что они обязательно будут в системе образования Сына, а «к востоку от Эдема» были возведены жилища для обетованного Сына и его прямого потомства. Архитектурные планы Эдема предусматривали наличие жилья и обширной территории для одного миллиона человек.
\vs p073 5:2 Ко времени прибытия Адама, хотя Сад был готов всего на четверть, в нем уже провели тысячи миль ирригационных каналов и больше двенадцати тысяч миль мощеных больших и малых дорог. В различных местах воздвигли свыше пяти тысяч кирпичных зданий, окруженных бесчисленными деревьями и другой растительностью. Максимальное число домов любой застройки в парке равнялось семи. Здания Сада была просты, но необыкновенно изящны. Малые и большие дороги были прочно построены, ландшафтные работы выполнены самым изысканным образом.
\vs p073 5:3 Санитарно\hyp{}техническое оборудование Сада далеко превосходило все, что до этого когда\hyp{}либо использовалось на Урантии. Здоровые свойства питьевой воды Эдема сохранялись благодаря строгому соблюдению правил, разработанных для поддержания ее чистоты. В начальный период случалось множество неприятностей вследствие пренебрежения этими правилами, но Ван постепенно убедил своих сподвижников, как важно не допускать попадания чего бы то ни было в систему водоснабжения Сада.
\vs p073 5:4 Пока не была закончена система канализации, жители Эдентии тщательно закапывали все отходы или разлагающиеся продукты. Инспекторы Амадона совершали ежедневный обход в поисках возможных возбудителей болезней. Жители Урантии все\hyp{}таки не осознавали важности профилактики болезней вплоть до более поздних времен --- до девятнадцатого и двадцатого веков. До распада адамического правления была построена система кирпичных водопроводов для удаления сточных вод; она проходила под стенами и впадала в реку Эдема на расстоянии около мили за пределами внешней или малой стены Сада.
\vs p073 5:5 Ко времени прибытия Адама большинство растений этой части мира уже росли в Эдеме. Многие фрукты, злаки и орехи уже тогда подверглись существенной селекции. Многие современные овощи и злаки впервые были выращены здесь, но впоследствии множество разнообразных съедобных растений было утрачено для мира.
\vs p073 5:6 Около пяти процентов Сада было занято под искусственно выращенные культуры, пятнадцать процентов было обработано лишь частично, остальная часть находилась в относительно первозданном виде, ожидая прибытия Адама --- предполагалось, что лучше всего завершить обустройство парка в соответствии с его собственными представлениями.
\vs p073 5:7 Итак, Эдемский Сад был подготовлен к приему обетованного Адама и его супруги. И Сад этот сделал бы честь любому миру, находящемуся под совершенным управлением и нормальным контролем. Адам и Ева были вполне удовлетворены общим планом Эдема, хотя и внесли значительные изменения в обстановку своего собственного жилища.
\vs p073 5:8 Хотя работы по украшению Сада не были закончены ко времени прибытия Адама, это место уже представляло собой жемчужину растительного мира. А уже в начальный период его пребывания в Эдеме весь Сад приобрел новый вид и новые пропорции, подчеркивающие его красоту и великолепие. Никогда, ни до, ни после этого времени, Урантия не создавала столь прекрасное и совершенное творение растениеводства и земледелия.
\usection{6. Дерево жизни}
\vs p073 6:1 В центре храма Сада Ван посадил неусыпно охраняемое дерево жизни, листья которого были предназначены для «исцеления народов», а плоды служили им пропитанием на земле. Ван хорошо знал, что Адам и Ева также будут нуждаться в этом даре Эдентии, необходимом им для поддержания жизни после физического воплощения на Урантии.
\vs p073 6:2 В столицах систем Материальные Сыны не нуждаются в дереве жизни для своего пропитания. Только в своем планетарном перевоплощении они становятся зависимыми от этого атрибута физического бессмертия.
\vs p073 6:3 \pc Если выражение «древо познания добра и зла», возможно, является выражением, обозначающим многообразие человеческого опыта, то «дерево жизни» не миф, это реальность, действительно существовавшая на Урантии в течение долгого времени. Когда Всевышние Эдентии одобрили назначение Калигастии Планетарным Принцем Урантии и ста граждан Иерусема в качестве его административного штата, они послали с Мелхиседеками куст Эдентии, и это растение выросло, чтобы стать деревом жизни на Урантии. Эта форма неразумной жизни присуща сферам центров созвездий, она обнаружена и в центральных мирах локальных и сверхвселенных, а кроме того --- на сферах Хавоны, но не в столицах систем.
\vs p073 6:4 Это сверхрастение запасало определенную пространственную энергию, которая являлась противоядием от веществ, вызывающих процесс старения живого организма. Плод дерева жизни похож на сверххимический аккумулятор, таинственным образом высвобождающий в процессе еды вселенскую силу продления жизни. На Урантии этот вид пищи совершенно бесполезен для обычных эволюционных существ, но он, в частности, полезен для ста материализованных членов штата Калигастии и для тех ста модифицированных андонитов, которые отдали свою жизненную плазму штату Принца, за что и стали обладателями дополнителя жизненной силы. Это дало им возможность использовать в пищу плоды дерева жизни, чтобы безгранично продлить их дотоле смертное существование.
\vs p073 6:5 \pc В правление Принца дерево росло из почвы в круглом центральном дворе храма Отца. Когда разразился бунт, оно было вновь выращено из отростка центрального ствола Ваном и его товарищами в их временном лагере. Затем этот куст Эдентии был принесен в их убежище в горах, где служил Вану и Амадону в течение более чем ста пятидесяти тысяч лет.
\vs p073 6:6 Когда Ван и его сподвижники подготовили Сад для Адама и Евы, они пересадили дерево Эдентии в Эдемский Сад, где оно снова стало расти в круглом центральном дворе, но уже другого храма Отца. И Адам и Ева время от времени вкушали этот плод, чтобы поддержать двуединую форму своей физической жизни.
\vs p073 6:7 \pc Когда планы Материального Сына потерпели крушение, Адаму и его семье не было разрешено унести ствол дерева из Сада. Когда нодиты вторглись в Эдем, им было сказано, что они станут «как боги, если отведают плод этого дерева». К своему большому удивлению, нодиты нашли дерево неохраняемым. Они свободно питались его плодами в течение ряда лет, но это на них совершенно не сказалось. Все они были обычными смертными материального мира и не обладали даром, который является необходимым дополнителем к действию плода дерева. Они были в ярости из\hyp{}за своей неспособности воспользоваться деревом жизни, а в результате одной из междоусобных войн, и храм, и дерево, погибли в огне. Только каменная стена простояла до тех пор, пока впоследствии Сад не погрузился под воду. Это второй храм Отца, который был уничтожен.
\vs p073 6:8 И теперь все живое на Урантии должно было развиваться своим естественным путем, путем жизни и смерти. Адам, Ева, их дети и дети их детей вместе со своими сподвижниками --- все со временем умерли, становясь, таким образом, подвластными схеме восхождения, присущей локальной вселенной, в которой за материальной смертью следует воскресение в мирах\hyp{}обителях.
\usection{7. Судьба Эдема}
\vs p073 7:1 После того, как первый сад был оставлен Адамом, его в разное время занимали нодиты, кутиты и суниты. Позже он стал обиталищем северных нодитов, которые были против союза с адамитами. Полуостров был завоеван этими низкоразвитыми нодитами, и после того, как Адам покинул Сад, они владели им почти четыре тысячи лет, а потом в связи с бурной активностью окружающих вулканов и погружения перешейка между Сицилией и Африкой, восточное дно Средиземного моря опустилось и унесло под воду весь Эдемский полуостров. Одновременно с этим обширным погружением значительно поднялась береговая линия восточного Средиземноморья. Это был конец самого прекрасного создания природы, когда\hyp{}либо существовавшего на Урантии. Погружение было постепенным, потребовалось несколько столетий, чтобы весь полуостров скрылся под водой.
\vs p073 7:2 Мы ни в коей мере не можем рассматривать исчезновение Сада как результат неудачного осуществления божественных планов или как следствие ошибок Адама и Евы. Мы считали погружение Эдема под воду только случайным феноменом природы, но все\hyp{}таки нам кажется, что затопление Эдема было приурочено как раз к тому времени, когда фиолетовая раса достигла количества необходимого для того, чтобы предпринять работу по реабилитации народов мира.
\vs p073 7:3 \pc Мелхиседек посоветовал Адаму не начинать программу расового подъема и не смешиваться с другими народами, пока его собственная семья не увеличится до полумиллиона. Никогда не предполагалось, что Сад должен быть постоянным местом жительства адамитов. Им надлежало стать эмиссарами новой жизни для всего мира. Они должны были быть мобилизованы для бескорыстного дара пришествия бедствующим расам земли.
\vs p073 7:4 Инструкции, данные Мелхиседеком Адаму, подразумевали, что тому следует создать расовые, континентальные и административные центры, которые возглавят его сыны и дочери, в то время как сам он вместе с Евой должны трудиться в различных столицах мира как советники и координаторы всемирного служения делу биологического подъема, интеллектуального прогресса и восстановления морали.
\vsetoff
\vs p073 7:5 [Представлено Солонией, серафическим «голосом в Саду».]
