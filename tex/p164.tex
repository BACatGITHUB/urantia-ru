\upaper{164}{Праздник посвящения}
\author{Комиссия срединников}
\vs p164 0:1 Пока в Пелле строился лагерь, Иисус, взяв с собой Нафанаила и Фому, тайно пошел в Иерусалим на праздник посвящения. Только после того, как они у Вифании переправились через Иордан, оба апостола догадались, что Учитель направляется в Иерусалим. Поняв, что он, действительно, намерен присутствовать на празднике посвящения, они стали его горячо отговаривать и, приводя всевозможные доводы, старались его переубедить. Но все их усилия были напрасны; Иисус твердо решил посетить Иерусалим. На все их просьбы и предостережения, подчеркивавшие безрассудство поступка и опасность оказаться в руках Синедриона, Иисус лишь ответил: «Я хочу перед тем, как настанет мой час, дать этим учителям в Израиле еще одну возможность увидеть свет».
\vs p164 0:2 Они шли к Иерусалиму, и оба апостола продолжали выражать свои опасения и высказывать свои сомнения в благоразумии такого явно самонадеянного предприятия. Около половины пятого они дошли до Иерихона и приготовились здесь заночевать.
\usection{1. История о добром самарянине}
\vs p164 1:1 В тот вечер около Иисуса и двух апостолов собралось довольно много людей, они задавали разные вопросы, на некоторые из них отвечали апостолы, некоторые же обсуждались Учителем. В этот вечер некий законник, пытавшийся вовлечь Иисуса в опасный спор, сказал: «Учитель, я хочу тебя спросить, что мне делать, чтобы наследовать жизнь вечную?» Иисус ответил: «В законе и у пророков что написано? Как читаешь Писание?» Законник же, зная учения и Иисуса, и фарисеев, ответил: «Возлюби Господа Бога всем сердцем твоим, и всею душою твоею, всем разумением твоим и всей крепостью твоею, и ближнего твоего, как самого себя». Тогда Иисус сказал: «Правильно ты отвечал; если действительно будешь так поступать, это приведет тебя к жизни вечной».
\vs p164 1:2 Но законник, задавая этот вопрос, был не вполне искренен и, желая оправдать себя и вместе с тем надеясь поставить Иисуса в тупик, решился задать еще один вопрос. Подойдя ближе к Учителю, он сказал: «Однако, Учитель, я хочу, чтобы ты сказал мне, кто мой ближний». Законник задал этот вопрос, надеясь спровоцировать Иисуса на какое\hyp{}нибудь высказывание, противоречившее еврейскому закону, согласно которому ближний человек --- это лишь «дети его народа». Всех же остальных евреи считали «нееврейскими псами». Сам законник был в какой\hyp{}то степени знаком с учениями Иисуса и поэтому хорошо знал, что Учитель думал иначе; таким образом, он надеялся заставить его сказать нечто такое, что могло быть истолковано как выпад против священного закона.
\vs p164 1:3 Но Иисус разгадал намерение законника и для того, чтобы не попасть в западню, стал рассказывать собравшимся историю --- историю, которую хорошо бы понял любой житель Иерихона, слушавший его. Иисус сказал: «Некий человек шел из Иерусалима в Иерихон и попал в руки жестоких разбойников, которые его ограбили, раздели, избили, и, бросив его чуть живого, ушли. Немного позже случайно по той же дороге шел один священник и когда наткнулся на израненного, то, увидев его жалкое состояние, прошел мимо по другой стороне дороги. Также и левит подошел, посмотрел и прошел мимо. И вот приблизительно в это же время некий самарянин, ехавший в Иерихон, заметил этого несчастного и, увидев, что того ограбили и избили, исполнился сострадания и, подойдя к нему, перевязал раны, возливая масло и вино, и, посадив его на своего осла, привез в гостиницу и позаботился о нем. А на другой день дал содержателю гостиницы деньги и сказал: „Хорошенько позаботься о друге моем, и если издержишь больше, я, когда возвращусь, отдам тебе“. Теперь же позволь мне спросить тебя: кто из этих троих оказался ближним ограбленному»? И законник, увидев, что попал в свои же сети, ответил: «Оказавший ему милость». Тогда Иисус сказал: «Иди и поступай так же».
\vs p164 1:4 Законник ответил: «Оказавший ему милость», чтобы избежать даже произнести это отвратительное ему слово «самарянин». Законник был вынужден дать именно такой ответ на вопрос: «Кто мой ближний?», который желал получить Иисус и который, если бы Иисус его так сформулировал, сразу же навлек бы на него обвинение в ереси. Иисус не только нарушил планы незадачливого законника, но и поведал своим слушателям историю, которая была прекрасным наставлением всем его последователям и одновременно сокрушительным упреком всем евреям за их отношение к самарянам. И рассказ этот способствовал распространению братской любви среди всех, кто впоследствии поверил в евангелие Иисуса.
\usection{2. В Иерусалиме}
\vs p164 2:1 Иисус посетил праздник кущей, чтобы провозгласить евангелие паломникам со всех концов империи; теперь же он шел на праздник посвящения с единственной целью: предоставить синедриону и еврейским лидерам еще одну возможность увидеть свет. Главное событие этих нескольких дней, проведенных в Иерусалиме, произошло в пятницу ночью в доме Никодима. Здесь собрались двадцать пять еврейских лидеров, которые верили в учение Иисуса. Среди них было четырнадцать человек, которые на тот момент являлись членами синедриона или же были ими до недавнего времени. На этой встрече присутствовали Эбер, Матадорм и Иосиф Аримафейский.
\vs p164 2:2 В данном случае все слушатели Иисуса были образованными людьми, но они, и оба апостола, были поражены широтой и глубиной суждений Учителя. Со времен, когда Иисус учил в Александрии, в Риме и на островах Средиземного моря, он никогда не выказывал такой учености и не проявлял такое понимание человеческих проблем, и мирских и религиозных.
\vs p164 2:3 После этой короткой встречи все разошлись озадаченные личностью Учителя, очарованные его любезными манерами и влюбленные в него. Они старались дать Иисусу совет в связи с его желанием обратить оставшихся членов синедриона. Учитель внимательно, но молча выслушал все их предложения. Он хорошо знал, что ничего из предполагаемого не осуществится. Он догадывался, что большинство еврейских лидеров никогда не примут евангелие царства, и все же дал им всем еще одну возможность сделать выбор. Однако в ту ночь, вместе с Нафанаилом и Фомой отправляясь на масличную гору, чтобы там заночевать, он еще не решил, как поступит, дабы еще раз привлечь внимание синедриона к своему делу.
\vs p164 2:4 В ту ночь Нафанаил и Фома мало спали; они были чрезвычайно взволнованы тем, что услышали в доме Никодима. Они много думали о том, что ответил Иисус на предложение бывших и действующих членов синедриона вместе с ним предстать перед семидесятью. Учитель сказал: «Нет, братья мои, это ни к чему. Вы лишь умножите гнев, который падет на ваши же головы, но ничуть не смягчите ненависть, которую они питают ко мне. Пусть каждый из вас идет и занимается делом Отца, как укажет ему дух, я же еще раз обращу их внимание к царству, так, как повелит мне Отец мой».
\usection{3. Исцеление слепорожденного}
\vs p164 3:1 На следующее утро все трое пошли завтракать в дом Марфы в Вифании, а затем сразу отправились в Иерусалим. В это субботнее утро, подойдя к храму, Иисус и двое апостолов увидели известного нищего, слепого от рождения, который сидел на своем обычном месте. Хотя нищие не просили и не получали милостыню по субботам, им позволялось сидеть на своих привычных местах. Иисус остановился и внимательно посмотрел на нищего. Пока он пристально разглядывал слепого от рождения человека, ему пришла на ум мысль о том, как еще раз привлечь к своей миссии на земле внимание синедриона и других еврейских лидеров и религиозных учителей.
\vs p164 3:2 Когда Учитель, глубоко задумавшись, стоял перед слепым, Нафанаил, размышляя над возможной причиной слепоты этого человека, спросил: «Учитель, кто согрешил, этот человек или его родители, что он родился слепым?»
\vs p164 3:3 \pc Раввины учили, что все подобные случаи слепоты от рождения есть следствие греха. Дети не только зачинались и рождались в грехе, но ребенок мог родиться слепым в наказание за какой\hyp{}нибудь особый грех, совершенный его отцом. Они даже учили, что ребенок может сам грешить еще в утробе матери. Они также учили, что подобные недостатки могут быть вызваны грехом или же иной порочной слабостью матери в то время, когда она носила ребенка.
\vs p164 3:4 Во всех этих землях продолжали верить в перевоплощение. Более древние еврейские учителя вместе с Платоном, Филоном и многими из ессеев терпимо относились к теории, согласно которой люди в одной инкарнации могли пожинать то, что посеяли в предыдущей жизни; таким образом, считалось, что в одной жизни они искупали грехи, совершенные в предыдущих. Учителю было трудно заставить людей поверить, что у их душ никаких предыдущих жизней не было.
\vs p164 3:5 Однако, как бы непоследовательно это ни выглядело, хотя подобная слепота считалась следствием греха, евреи придерживались мнения, что подаяние милостыни таким слепым нищим в высшей степени достойно похвалы. Эти нищие имели обыкновение постоянно монотонным голосом просить у прохожих: «О добросердечный, заработай заслугу, помоги слепому».
\vs p164 3:6 \pc Иисус вступил в дискуссию по этому вопросу с Нафанаилом и Фомой не только потому, что уже решил использовать этого слепого в качестве средства в тот день еще раз как можно заметнее привлечь внимание еврейских лидеров к своей миссии, но также потому, что всегда призывал своих апостолов искать истинные причины всех явлений, и естественных и духовных. Он всегда предостерегал их избегать общепринятого стремления объяснять духовными причинами повседневные физические события.
\vs p164 3:7 Иисус уже решил использовать этого нищего в том, что он задумал предпринять в тот день, однако перед тем, как сделать что\hyp{}нибудь для слепого, которого звали Иосия, стал отвечать на вопрос Нафанаила. Учитель сказал: «Не согрешил ни этот человек, ни родители его, чтобы на нем явились дела Бога. Слепота нашла на него в результате естественного хода событий, однако мы должны делать дела Пославшего меня теперь, пока еще день, ибо обязательно придет ночь, когда невозможно будет делать дело, которое мы должны исполнить. Доколе я в мире, я свет миру, но уже совсем скоро меня с вами не будет».
\vs p164 3:8 Сказав это, Иисус обратился к Нафанаилу и Фоме: «Сотворим зрение этому слепому в сей день субботний, чтобы у книжников и фарисеев было полное основание, которого они ищут, дабы обвинить Сына Человеческого». Затем, наклонившись, он плюнул на землю, смешал глину со слюной и, говоря обо всем этом так, чтобы слепой мог слышать, подошел к Иосии и, наложив эту глину на его незрячие глаза, сказал: «Сын мой, пойди и смой эту глину в купели Силоам и сразу прозреешь». И, умывшись в купели Силоам, как велел ему Иисус, Иосия вернулся к своим друзьям и семье зрячим.
\vs p164 3:9 Иосия всю жизнь был нищим и ничего другого делать не умел; поэтому когда первый восторг от прозрения прошел, он вернулся на свое обычное место, где всегда просил милостыню. Его друзья, соседи и все, кто его знал прежде, увидев, что он видит, сказали: «Не тот ли это Иосия, слепой нищий?» Иные говорили: это он; иные же: «Нет, похож на него, но зрячий». Однако когда спросили его самого, тот ответил: «Это я».
\vs p164 3:10 Когда же стали спрашивать у него, как он прозрел, он ответил: «Человек, называемый Иисус, проходил этим путем и, говоря обо мне со своими друзьями, смешал слюну с глиной, помазал мои глаза и велел, чтобы я пошел и умылся в купели Силоам. Я сделал то, что сказал мне этот человек, и сразу прозрел. А было это всего несколько часов назад. Я еще не понимаю значения многого из того, что вижу». Когда же люди, которые начали собираться вокруг него, спросили, где им найти странного человека, который исцелил его, Иосия смог ответить лишь, что он не знает.
\vs p164 3:11 \pc Это --- одно из самых необычайных чудес, совершенных Учителем. Этот человек об исцелении не просил. Он не знал, что Иисус, повелевший ему умыться в Силоамской купели и обещавший ему прозрение, был галилейским пророком, который проповедовал в Иерусалиме во время праздника кущей. Этот человек мало верил в то, что прозреет, однако люди в те дни имели великую веру в целебную силу плюновения великого или святого человека; из разговора же Иисуса с Нафанаилом и Фомой Иосия заключил, что его будущий благодетель --- человек великий, ученый учитель или святой пророк; поэтому и поступил, как велел ему Иисус.
\vs p164 3:12 Иисус же смешал глину со слюной и велел ему умыться в священной купели Силоам по трем причинам:
\vs p164 3:13 \ublistelem{1.}\bibnobreakspace Случившееся не было чудом, явившимся следствием веры человека. Это было чудо, которое Иисус решил совершить, преследуя свою собственную цель, но сделал все так, чтобы это пошло во благо этому человеку.
\vs p164 3:14 \pc \ublistelem{2.}\bibnobreakspace Поскольку слепой не просил об исцелении и так как вера его была слаба, эти материальные деяния были совершены, чтобы его ободрить. Он верил в предрассудок относительно целебной силы плюновения и знал, что купель Силоам была полусвященным местом. Однако он бы туда не пошел, если бы не было нужды смыть глину, которой его помазали. Это действие являлось в какой\hyp{}то мере элементом обряда, что и побудило его действовать.
\vs p164 3:15 \pc \ublistelem{3.}\bibnobreakspace Но у Иисуса была и третья причина прибегнуть к материальным средствам в с этом уникальном деле: то было чудо, сотворенное в полном соответствии с его собственным выбором; посредством него Иисус желал научить своих последователей того времени и всех последующих веков воздерживаться от пренебрежения материальными средствами в исцелении больных. Он хотел научить их перестать считать чудеса единственным методом лечения человеческих болезней.
\vs p164 3:16 \pc В это субботнее утро и в Иерусалиме рядом с храмом Иисус вернул этому человеку зрение чудесным деянием, преследуя главную цель --- превратить этот акт в открытый вызов синедриону и всем еврейским учителям и религиозным лидерам. Таким образом он открыто заявил о разрыве с фарисеями. Во всем, что делал Иисус, он был всегда позитивен. Именно с целью привлечь внимание синедриона к этим вопросам Иисус и привел в эту субботу сразу после полудня двух своих апостолов к этому человеку и намеренно вызвал толки, которые заставили фарисеев обратить внимание на чудо.
\usection{4. Иосия перед синедрионом}
\vs p164 4:1 Уже через несколько часов после исцеления Иосии возле храма разгорелись такие споры, что начальники синедриона решили собраться на совет в особом отведенном месте храма. Сделали же это они вопреки установленному правилу, запрещавшему собрания синедриона в субботу. Иисус знал, что нарушение субботы будет одним из главных обвинений против него, когда настанет последнее испытание, и желал предстать перед судом синедриона за исцеление слепого в субботу, после того как само заседание высшего еврейского суда, осуждающее его за милосердное деяние, будет проведено именно в субботу, тем самым впрямую нарушая свои же собственные законы.
\vs p164 4:2 Но они не призвали Иисуса предстать перед ними; потому что боялись. Вместо этого они немедленно послали за Иосией. После нескольких предварительных вопросов, представитель синедриона (присутствовало около пятидесяти членов) приказал Иосии рассказать, что с ним произошло. После своего исцеления в то утро Иосия от Фомы, Нафанаила и других узнал, что фарисеи разгневаны его исцелением в субботу и что они, вероятно, причинят неприятности всем, кто в этом участвовал; однако Иосия еще не понимал, что Иисус был тем, кого называли Избавителем. Поэтому, когда фарисеи спросили его, он сказал: «Этот человек подошел, наложил глину на мои глаза, сказал мне умыться в Силоамской купели, и теперь я вижу».
\vs p164 4:3 Один из старейших фарисеев произнес длинную речь и потом сказал: «Не может быть от Бога этот человек, потому что он, как видите, не соблюдает субботу. Он нарушает закон, во\hyp{}первых, делая глину, а затем, посылая этого нищего умыться в Силоамской купели в субботу. Такой человек не может быть учителем, посланным от Бога».
\vs p164 4:4 Тогда один из младших по возрасту, тайно веривший в Иисуса, сказал: «Если этот человек не послан Богом, то как он может делать подобное? Мы знаем, что простой грешник не может творить такие чудеса. Мы все знаем этого нищего --- он родился слепым; теперь же он видит. Неужели вы все равно будете говорить, что сей пророк совершает все эти чудеса силою принца бесовского?». И против каждого фарисея, посмевшего обвинить и осудить Иисуса, вставал другой и задавал запутывающие и смущающие вопросы, так что в конце концов среди них возникла серьезная распря. Председательствующий увидел, куда все клонится, и, чтобы уладить спор, решил дальше допрашивать человека сам. Повернувшись к Иосии, он спросил: «Ты что скажешь об этом человеке, об этом Иисусе, который, как ты утверждаешь, отверз твои очи?» И Иосия ответил: «Я думаю, что он --- пророк».
\vs p164 4:5 Руководители были очень встревожены и, не зная, что делать, решили послать за родителями Иосии, чтобы узнать, действительно ли он родился слепым. Они не желали верить, что нищий был исцелен.
\vs p164 4:6 В Иерусалиме хорошо знали, что не только Иисусу было запрещено входить во все синагоги, но и все, верившие в него, были так же, как и он, из них изгнаны и отлучены от собрания Израилева; это означало поражение во всех правах и привилегиях любого рода во всем еврействе за исключением права покупать предметы первой необходимости.
\vs p164 4:7 Поэтому когда родители Иосии, бедные и задавленные страхом души, предстали перед августейшим синедрионом, то боялись говорить открыто. Председатель суда спросил: «Это ли ваш сын? Правильно ли мы понимаем, что он родился слепым? Если это правда, то как он теперь видит?» И тогда отец Иосии, а за ним и мать его, ответили: «Мы знаем, что это сын наш и что он родился слепым, а как стал видеть или кто отверз ему очи, мы не знаем. Спросите его; он в совершенных летах; пусть сам о себе скажет».
\vs p164 4:8 Тогда призвали Иосию предстать перед ними во второй раз. Им не удавалось придерживаться принятого распорядка судопроизводства, и некоторые из них начинали чувствовать себя неловко, потому что делали все это в субботу; поэтому, призвав Иосию, они попытались заманить его в ловушку, попробовав атаковать с другой стороны. Представитель суда обратился к прозревшему и сказал: «Почему ты не воздаешь хвалу Богу за это? Почему не говоришь нам всю правду о том, что случилось? Мы все знаем, что этот человек грешник. Почему ты отказываешься видеть истину? Мы знаем, что и ты и этот человек виновны в нарушении субботы. Но искупишь ли свой грех, признав Бога твоим исцелителем, если по\hyp{}прежнему утверждаешь, что отверзлись очи твои сегодня?»
\vs p164 4:9 Но Иосия был неглуп и не страдал отсутствием юмора, а потому ответил представителю суда: «Грешник ли он, я не знаю; одно знаю, что я был слеп, а теперь вижу». И так как они не смогли заманить Иосию в ловушку, то стали его допрашивать дальше, говоря: «Как же он отверз твои очи? Что на самом деле сделал с тобой? Что тебе сказал? Просил ли тебя верить в него?»
\vs p164 4:10 Иосия с некоторым раздражением ответил: «Я уже в точности рассказал вам, как все случилось; если не верите моему свидетельству, то почему хотите об этом услышать снова? Уж не хотите ли и вы стать его учениками?» Когда Иосия сказал это, в синедрионе произошло замешательство, почти неистовство, ибо руководители набросились на Иосию, злобно восклицая: «Себя называй учеником этого человека, а мы Моисеевы ученики и учителя законов Бога. Мы знаем, что Бог говорил через Моисея, а что до сего человека, Иисуса, мы не знаем, откуда он».
\vs p164 4:11 Тогда Иосия встал на скамью и, обращаясь ко всем, кто мог слышать, сказал: «Слушайте вы, называющие себя учителями всего Израиля! Объявляю вам: в сем --- великое чудо, ибо вы утверждаете, что не знаете, откуда сей человек, и вместе с тем из свидетельства, которое услышали, со всей определенностью знаете, что он отверз мои очи. Мы все знаем, что Бог не делает подобного для неверующих, что Бог совершает такое лишь по просьбе истинного молитвенника\hyp{}\hyp{}для того, кто свят и праведен. Вы знаете, что от начала мира не слышали, чтобы отверзались очи слепого от рождения. Так посмотрите же все на меня и осознайте, что было сделано в этот день в Иерусалиме! Говорю вам, если бы этот человек был не от Бога, то сего не смог бы совершить». Члены же синедриона, уходя, в злобе и смятении кричали ему: «Ты ли рожденный в грехе осмеливаешься учить нас? Может быть, ты по\hyp{}настоящему и не был слепым от рождения и даже если глаза твои отверзлись в субботу, сделано это было силой принца бесовского». И они сразу пошли в синагогу и изгнали Иосию вон.
\vs p164 4:12 Иосия пришел на этот суд, почти ничего не зная об Иисусе и природе его исцеления. В основном смелые доказательства, которые он так умно и отважно выдвигал перед высшим трибуналом всего Израиля, возникали в его уме по мере того, как судилище разворачивалось столь несправедливым и неправедным образом.
\usection{5. Проповедь на крыльце Соломоновом}
\vs p164 5:1 Все время, пока в одном из помещений храма происходило нарушающее субботу заседание синедриона, Иисус был рядом, уча народ на крыльце Соломоновом и надеясь, что его вызовут в синедрион, где он сможет рассказать благую весть о свободе и радости божественного сыновства в царстве Бога. Но члены синедриона боялись посылать за ним. Эти внезапные появления Иисуса перед народом в Иерусалиме всегда приводили их в замешательство. Тот самый повод, которого они столь страстно искали, Иисус теперь им давал, но они боялись вызвать его предстать перед синедрионом в качестве свидетеля и еще больше боялись его арестовывать.
\vs p164 5:2 В Иерусалиме тогда была середина зимы, и народ пытался хоть как\hyp{}то укрыться на крыльце Соломоновом; и когда Иисус был там, люди задавали ему множество вопросов, а он учил их более двух часов. Некоторые из еврейских учителей, пытаясь заманить его в западню, публично спрашивали его: «Долго ли будешь держать нас в неопределенности. Если ты Мессия, то почему не говоришь нам об этом прямо?» Иисус сказал: «Я говорил о себе и об Отце моем много раз, но вы мне не верили. Разве не видите, что дела, которые творю я во имя Отца моего, свидетельствуют обо мне? Однако многие из вас не верят, потому что не принадлежат к моему стаду. Учитель истины привлекает лишь тех, кто алчет истины и жаждет праведности. Овцы мои слушаются голоса моего, и я знаю их, и они идут за мною. И всем, кто следует моему учению, я даю жизнь вечную; они вовек не погибнут, и никто не вырвет их из руки моей. Отец мой, который дал мне детей сих, величее всех, так что никто не может похитить их из руки Отца моего. Отец и я --- одно». Некоторые из неверующих евреев бросились туда, где еще строили храм, схватить каменья, чтобы побить Иисуса, но верующие удержали их.
\vs p164 5:3 Иисус же продолжал свою проповедь: «Много добрых дел показал я вам от Отца, так что теперь хочу спросить: за которое из них думаете побить меня камнями?» Тогда один из фарисеев ответил: «Не за доброе дело хотим побить тебя камнями, но за богохульство, так как ты, будучи человек, смеешь делать себя равным Богу». И Иисус ответил: «Вы обвиняете Сына Человеческого в богохульстве, потому что отказались верить мне, когда я объявил вам, что я послан Богом. Если я не творю дел Бога, не верьте мне; а если творю, даже если не верите в меня, думаю, в дела мои поверите. Но чтобы вы уверились в том, что я возвещаю, позвольте мне снова заявить, что Отец во мне, и я в Отце, и что, как Отец пребывает во мне, так и я пребуду в каждом, верующем в сие евангелие». И когда народ услышал эти слова, многие бросились хватать камни, чтобы кидать в него, но он вышел через пределы храма; и, встретив Нафанаила и Фому, присутствовавших на заседании синедриона, вместе с ними стал дожидаться около храма, когда Иосия выйдет из палаты, где проходил совет.
\vs p164 5:4 Иисус и оба апостола не ходили искать Иосию к его дому, пока не услышали, что тот был изгнан из синагоги. Придя же в его дом, Фома вызвал его во двор, и Иисус, говоря с ним, сказал: «Иосия, веришь ли ты в Сына Бога?» И Иосия ответил: «Скажи мне, кто он, чтобы мне веровать в него». И Иисус сказал: «И видел ты его и слышал, и он --- тот, кто сейчас говорит с тобою». И Иосия воскликнул: «Верую, Господи» и, пал пред ним, и поклонился ему.
\vs p164 5:5 Когда Иосия узнал о своем изгнании из синагоги, то сначала был сильно удручен, но потом, когда Иисус велел ему немедленно приготовиться идти с ними в лагерь в Пелле, очень ободрился. Этот простодушный человек из Иерусалима был действительно изгнан из еврейской синагоги, но вот Творец вселенной ведет его присоединиться к духовной элите того времени и того поколения.
\vs p164 5:6 И оставил Иисус Иерусалим, и уже не возвращался туда до момента, когда приготовился покинуть сей мир. Вместе с двумя апостолами и Иосией Учитель пошел назад в Пеллу. И Иосия оказался один из получателей чудотворнного служения Учителя, который принес много плодов, ибо стал проповедником евангелия, и оставался им до конца жизни.
