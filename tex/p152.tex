\upaper{152}{События, приведшие к кризису в Капернауме}
\author{Комиссия срединников}
\vs p152 0:1 История об исцелении Амоса, безумца из Хересы, уже достигла Вифсаиды и Капернаума, и огромная толпа ожидала Иисуса, когда в четверг перед полуднем его лодка причалила к берегу. В толпе были новые наблюдатели из иерусалимского синедриона, которые пришли в Капернаум, чтобы найти повод для ареста и осуждения Учителя. Когда Иисус разговаривал с теми, кто собрался поприветствовать его, некий Иаир, один из управителей синагоги, протиснулся через толпу и, припав к ногам Иисуса, взял его за руку и стал умолять поспешить уйти вместе с ним, говоря: «Учитель, моя дочь малая, единственное мое дитя, лежит в доме моем при смерти. Я умоляю тебя прийти и исцелить ее». Услышав просьбу этого отца, Иисус сказал: «Я пойду с тобой».
\vs p152 0:2 Когда Иисус шел рядом с Иаиром, за ними следовала огромная толпа, слышавшая просьбу отца и желавшая увидеть, что же произойдет. Недалеко от дома управителя, в то время, когда они спешили по узкой улочке и толпа стесняла их, Иисус вдруг остановился и воскликнул: «Кто\hyp{}то прикоснулся ко мне». Когда же те, кто был рядом с ним, стали отрицать, что прикасались к нему, Петр сказал: «Учитель, ты видишь, что тебя теснит толпа, угрожая раздавить нас, и ты говоришь: кто\hyp{}то прикоснулся ко мне. Что ты имеешь в виду?» Тогда Иисус сказал: «Я спросил, кто прикоснулся ко мне, ибо почувствовал, что из меня вышла живая энергия». Иисус посмотрел вокруг себя и его взгляд остановился на стоявшей рядом женщине, которая, выйдя вперед, упала к его ногам и сказала: «Уже несколько лет я страдаю страшным кровотечением. Много претерпела от разных врачей; израсходовала все свое имущество, но никто не смог меня излечить. Потом я услышала о тебе и подумала: если я смогу хотя бы прикоснуться к краю его одежды, то обязательно выздоровею. Поэтому я вместе с идущей толпой протискивалась вперед до тех пор, пока, стоя рядом с тобой, Учитель, не коснулась края твоей одежды и не выздоровела; я знаю, что я исцелена от болезни моей».
\vs p152 0:3 Услышав это, Иисус взял руку женщины и, подняв ее, сказал: «Дочь моя, вера твоя исцелила тебя; иди с миром». Именно ее \bibemph{вера,} а не ее \bibemph{прикосновение} вернула ей здоровье. Случай же этот является хорошим примером многочисленных кажущихся чудесными исцелений, которые сопровождали земную жизнь Иисуса, но которых он никоим образом сознательно не желал. Со временем стало ясно, что эта женщина была действительно исцелена от своей болезни. Вера ее была такова, что непосредственно впитала созидающую силу, присущую личности Учителя. С верой, которой она обладала, требовалось лишь приблизиться к Учителю; касаться же его одежды было совершенно не нужно; это был всего лишь момент суеверия ее веры. Иисус подозвал эту женщину, Веронику из Кесарии Филипповой, к себе, дабы исправить две ошибки, которые могли остаться в ее мыслях и у тех, кто был свидетелем этого исцеления: ему не хотелось, чтобы Вероника ушла, думая, что страх, который она испытывала, пытаясь получить исцеление тайком, заслуживал одобрения или что ее суеверие, суть которого заключалась в том, что свое исцеление она связывала с прикосновением к его одежде, было действенным. Он желал, чтобы все знали: исцеление свершила ее чистая и живая \bibemph{вера.}
\usection{1. В доме Иаира}
\vs p152 1:1 Разумеется, Иаир очень нервничал из\hyp{}за этой задержки на пути к его дому; поэтому теперь они шли намного быстрее. Однако еще прежде, чем они вошли во двор управителя синагоги, один из его слуг вышел и сказал: «Не утруждай Учителя; дочь твоя умерла». Но Иисус, казалось, не обращал внимания на слова слуги, ибо, взяв с собой Петра, Иакова и Иоанна, повернулся к убитому горем отцу и сказал: «Не бойся; только веруй». Войдя в дом, он увидел, что там собрались уже плакальщики и свирельщики, производившие неподобающий шум; родственники уже рыдали и плакали. Удалив всех плакальщиков из комнаты, он вошел в нее с отцом девочки, ее матерью и тремя своими апостолами. Он сказал плакальщикам, что девочка не умерла, но они с презрением посмеялись над ним. Тогда Иисус повернулся к матери и сказал: «Твоя дочь не умерла; она только спит». Когда же в доме утихло, Иисус, подойдя к месту, где лежала девочка, взял ее руку и сказал: «Говорю тебе, дочь, проснись и встань!» Услышав эти слова, девочка сразу поднялась и пошла по комнате. И вскоре после того, как она пришла в себя от потрясения, Иисус велел дать ей что\hyp{}нибудь поесть, ибо она долгое время была без пищи.
\vs p152 1:2 Поскольку в Капернауме возникли большие волнения, направленные против Иисуса, он собрал семью девочки и объяснил, что та находилась в состоянии комы, которая последовала за продолжительной лихорадкой, и что он просто разбудил ее, а из мертвых не воскрешал. Точно так же он все объяснил своим апостолам, но, увы, бесполезно; все они считали, что Иисус воскресил девочку из мертвых. То, что говорил Иисус в объяснение многих из этих событий, кажущихся чудесами, оказывало на его последователей слабое действие. Они ожидали чуда и не упустили возможности приписать Иисусу еще одно необычное проявление. Иисус и апостолы вернулись в Вифсаиду после того, как он специально приказал всем никому ничего не рассказывать.
\vs p152 1:3 \pc Когда он вышел из дома Иаира, следом за ним пошли двое слепых, которых вел немой мальчик и которые умоляли об исцелении. Приблизительно в это же время слава Иисуса как целителя достигла своей вершины. Куда бы он ни приходил, всюду его ожидали больные и страждущие. Теперь Учитель выглядел сильно уставшим, и все его друзья стали беспокоиться, что если он будет продолжать работу учителя и целителя, это может закончиться настоящим истощением.
\vs p152 1:4 \pc Апостолы Иисуса, и тем более простые люди, не могли понять природы и особенностей этого Богочеловека. Ни одно из последующих поколений также не было способно оценить, что же представлял собой на земле Иисус из Назарета. И ни науке, ни религии больше не представится возможность исследовать эти замечательные события по той простой причине, что подобная экстраординарная ситуация больше никогда не повторится ни в этом, ни в каком\hyp{}либо другом мире Небадона. Никогда больше ни в одном из миров всей этой вселенной не появится существо в подобии смертной плоти, одновременно воплотившее в себе все свойства созидательной энергии в сочетании с духовными дарами, превосходящими время и почти все другие материальные ограничения.
\vs p152 1:5 Ни до появления Иисуса на земле, ни после этого не было возможно столь непосредственно и явно получать результаты, благодаря сильной и живой вере смертных мужчин и женщин. Чтобы повторить эти явления, потребуется оказаться в непосредственном присутствии Михаила, Творца, и тому быть таким же, каким он был в те дни, --- Сыном Человеческим. Так же и сегодня, когда его отсутствие исключает подобные материальные проявления, вы должны воздерживаться накладывать любые ограничения на возможное проявление его \bibemph{духовной силы.} Хотя Учитель как материальное существо отсутствует, он, тем не менее, присутствует как духовное влияние на сердца людей. Уйдя из этого мира, Иисус сделал возможным, чтобы дух его жил рядом с духом его Отца, пребывающим в разуме всех людей.
\usection{2. Насыщение пяти тысяч}
\vs p152 2:1 Иисус продолжал учить народ днем, а ночью давал наставления апостолам и евангелистам. В пятницу он объявил недельный перерыв, чтобы все его последователи перед тем, как начать подготовку к путешествию в Иерусалим на Пасху, могли несколько дней провести дома или со своими друзьями. Однако больше половины его учеников отказались оставить его, а толпа, сопровождавшая его, росла с каждым днем, так что Давид Зеведеев хотел даже разбить новый лагерь, но Иисус отказался дать на это согласие. В субботу у Учителя не было возможности отдохнуть, поэтому в воскресенье утром 27 марта он попытался уйти от народа. Несколько евангелистов остались говорить с народом, тогда как Иисус и двенадцать апостолов собирались незаметно укрыться на противоположном берегу озера, где они хотели обрести столь необходимый им покой в прекрасном парке к югу от Вифсаиды\hyp{}Юлии. Этот район был излюбленным местом отдыха жителей Капернаума; всем им были известны эти парки на восточном берегу.
\vs p152 2:2 Но люди не могли с этим смириться. Они видели, куда направилась лодка Иисуса, и, наняв все имевшиеся плавучие средства, пустились ему вслед. Те же, кто не смог достать лодку, пошли пешком вдоль северного берега озера.
\vs p152 2:3 К концу дня более тысячи человек нашли Иисуса в одном из парков, и он коротко поговорил с ними, передав слово Петру. Многие из этих людей принесли с собой еду и, совершив вечернюю трапезу, разбились на небольшие группы, и апостолы и ученики Иисуса учили их.
\vs p152 2:4 В понедельник после полудня в толпе уже было больше трех тысяч человек. И все равно до самого вечера народ продолжал прибывать, неся с собой самых разных больных. Сотни заинтересовавшихся людей запланировали по пути на праздник Пасхи остановиться в Капернауме, чтобы увидеть и услышать Иисуса, и они просто отказывались уходить ни с чем. В среду к полудню в этом парке к югу от Вифсаиды\hyp{}Юлии собралось около пяти тысяч мужчин, женщин и детей. Погода была приятная, ибо сезон дождей в этой местности подходил к концу.
\vs p152 2:5 \pc Филипп приготовил для Иисуса и двенадцати апостолов трехдневный запас пищи, хранившийся у юноши Марка --- мальчика, исполняющего всевозможные поручения. После полудня этого уже третьего дня практически у половины собравшихся еда, принесенная ими, почти закончилась. У Давида же Зеведеева здесь не было палаточного лагеря, чтобы накормить и разместить всех прибывших. Филипп также не приготовил запасов пищи для такой огромной массы людей. Однако люди, несмотря на то, что были голодны, уходить не собирались. Об Иисусе тихо шептались, что он, желая избежать неприятностей, которые могли причинить ему Ирод и иерусалимские правители, выбрал этот находившийся вне юрисдикции всех его врагов тихий уголок в качестве подходящего места для провозглашения себя царем. Возбуждение толпы с каждым часом росло. Иисусу же не говорили ни слова, хотя, разумеется, о том, что происходило, он знал все. Подобными идеями были увлечены даже двенадцать апостолов и особенно самые молодые из евангелистов. Апостолами, которые поддерживали эту попытку провозгласить Иисуса царем, были Петр, Иоанн, Симон Зилот и Иуда Искариот. Против этого возражали Андрей, Иаков, Нафанаил и Фома. Матфей, Филипп и близнецы Алфеевы заняли уклончивую позицию. Зачинщиком же этого плана сделать Иисуса царем был Иоав, один из молодых евангелистов.
\vs p152 2:6 \pc Такова была обстановка в среду около пяти часов вечера, когда Иисус попросил Иакова Алфеева призвать Андрея и Филиппа. Иисус сказал: «Что нам делать с толпой? Уже три дня они с нами, и многие из них голодны. У них нет пищи». Переглянувшись с Андреем, Филипп ответил: «Учитель, ты должен отослать этих людей, чтобы они могли пойти в окрестные селения и купить себе пищи», Андрей же, опасаясь осуществления плана провозглашения Иисуса царем, быстро присоединился к Филиппу и сказал: «Да, Учитель, я думаю, что лучше всего будет, если ты велишь толпе разойтись, так чтобы люди могли пойти своим путем и купить еды, а ты бы смог какое\hyp{}то время отдохнуть». В это время к беседе подключились остальные из двенадцати. Тогда Иисус сказал: «Но я не желаю отсылать их голодными; не могли бы вы накормить их?» Этого Филипп выдержать не смог и прямо сказал: «Учитель, в этом пустынном месте где нам купить хлеба для этой толпы? Да и на двести динариев не хватит хлеба на ужин».
\vs p152 2:7 Однако прежде чем апостолы смогли высказаться, Иисус повернулся к Андрею и Филиппу и сказал: «Я не хочу отсылать этих людей. Вот они, как овцы без пастуха. Я хочу накормить их. Какая у нас есть пища?» Пока Филипп разговаривал с Матфеем и Иудой, Андрей разыскал мальчика Марка, чтобы выяснить, что осталось от их запасов провизии. Он вернулся к Иисусу и сказал: «У мальчика осталось только пять ячменных хлебов и две вяленые рыбки». Петр тут же добавил: «И нам тоже нужно поесть этим вечером».
\vs p152 2:8 Минуту Иисус простоял молча. Взгляд у него был отсутствующий. Апостолы ничего не говорили. Вдруг Иисус повернулся к Андрею и сказал: «Принеси мне хлебы и рыбу». И когда Андрей принес корзину Иисусу, Учитель сказал: «Прикажите людям сесть на траву группами по сто человек, назначьте в каждой группе старшего, а всех евангелистов приведите сюда к нам».
\vs p152 2:9 Иисус взял хлебы в свои руки и, воздав благодарение, преломил хлеб и дал его своим апостолам, которые передали его своим товарищам, а те в свою очередь отнесли его толпе. Точно так же Иисус разломил и разделил рыбу. И толпа ела и насытилась. Когда же люди кончили есть, Иисус сказал ученикам: «Соберите оставшиеся разломленные куски, чтобы ничего не пропало». Когда же закончили собирать куски, у них оказалось полных двенадцать корзин. Тех же, кто ел на этом необычайном пиру, насчитывалось до пяти тысяч мужчин, женщин и детей.
\vs p152 2:10 \pc Это было первым и единственным естественным чудом, которое Иисус совершил сознательно, обдумав его заранее. Верно, его ученики были склонны называть чудесами многие вещи, которые чудесами не были, однако это была подлинно сверхъестественная помощь. В данном случае, как нас учили, Михаил приумножил элементы пищи так, как он делает это всегда, за исключением того, что был устранен фактор времени и зримый канал жизни.
\usection{3. Провозглашение царем}
\vs p152 3:1 Насыщение пяти тысяч при помощи сверхъестественной энергии было еще одним из тех событий, когда человеческая жалость плюс созидающая сила осуществили то, что произошло. Теперь, когда толпа была досыта накормлена, а слава Иисуса немедленно возросла благодаря этому изумительному чуду, на то, чтобы взять и провозгласить Учителя царем, не нужно было личных указаний. Казалось, что мысль об этом распространилась в толпе, подобно инфекции. Реакция народа на столь быстрое и эффектное удовлетворение его физических потребностей была глубокой и неудержимой. Долгое время евреев учили: когда явится Мессия, Сын Давидов, он сделает так, что по земле снова потекут молочные реки в кисельных берегах и что им будет дарован хлеб жизни так же, как якобы упала манна небесная на их праотцов в пустыне. Разве не сбылось полностью это ожидание прямо у них на глазах? Когда эта голодная, никогда досыта не евшая толпа насытилась чудо\hyp{}хлебом, ее единая реакция была такова: «Вот наш царь». Чудотворный избавитель Израиля пришел. В глазах этих простодушных людей способность накормить давала право на власть. Поэтому неудивительно, что, закончив пир, толпа как один поднялась и воскликнула: «Провозгласим его царем!»
\vs p152 3:2 Этот мощный порыв привел в восторг Петра и тех апостолов, кто все еще питал надежду увидеть, как Иисус заявляет свое право на власть. Однако этим ложным надеждам не суждено было долго прожить. Не успело стихнуть эхо мощного возгласа, доносившееся с соседних гор, как Иисус встал на огромный камень и, воздев правую руку, чтобы привлечь внимание, сказал: «Дети мои, у вас хорошие намерения, однако вы близоруки и думаете только о материальном». Наступила короткая пауза; сей могучий галилеянин стоял в величественной позе в чарующем зареве восточного заката. С головы до пят он выглядел, как настоящий царь, и продолжал говорить с затаившей дыхание толпой: «Вы хотите сделать меня царем не потому, что души ваши озарила великая истина, но потому, что ваши животы насытились хлебом. Сколько же раз я говорил вам, что царство мое не от мира сего? Сие царство небесное, которое мы возвещаем, суть духовное братство, и правит им отнюдь не человек, восседающий на материальном троне. Отец мой Небесный есть премудрый и всемогущий Правитель этого духовного братства сыновей Бога на земле. Неужели мне до такой степени не удалось открыть вам Отца духов, что вы желаете сделать царем его Сына во плоти? Итак, сейчас же ступайте к домам вашим. Если же вам нужен царь, то да воцарится Отец света в сердце каждого из вас как духовный Правитель над всем».
\vs p152 3:3 \pc С этими словами Иисус отослал ошеломленную и пришедшую в уныние толпу. Многие, верившие в него, от него отвратились и более не следовали за ним. Апостолы лишились дара речи и молча стояли, собравшись у двенадцати корзин с остатками пищи; только посыльный, мальчик Марк, проговорил: «И он отказался быть нашим царем». Иисус же перед тем, как уйти, чтобы уединиться в горах, повернулся к Андрею и сказал: «Отведи братьев твоих в дом Зеведея и молись с ними, особенно же о брате твоем Симоне Петре».
\usection{4. Ночное видение Петра}
\vs p152 4:1 Апостолы без своего Учителя --- предоставленные сами себе --- вошли в лодку и стали молча грести к Вифсаиде на западном берегу озера. Никто из двенадцати не был так сломлен и удручен, как Симон Петр. Не было произнесено почти ни слова; все думали об Учителе, находившемся в одиночестве в горах. Неужели он их покинул? Раньше он никогда не отсылал их всех сразу и не отказывался идти вместе с ними. Что все это могло значить?
\vs p152 4:2 На них спустилась тьма, поднялся сильный встречный ветер, и они совершенно не могли продвигаться вперед. После нескольких часов тьмы и тяжелой гребли Петр утомился и от усталости крепко уснул. Андрей и Иаков положили его на мягком сидении на корме лодки. Пока остальные апостолы боролись с ветром и волнами, Петру снился сон; видение того, как Иисус шел к ним прямо по морю. Когда Учитель, казалось, проходил рядом с лодкой, Петр вскричал: «Спаси нас, Учитель, спаси нас!» И те, кто был на корме лодки, услышали, как он сказал некоторые их этих слов. В ночном видении Петру приснилось, что он услышал, как Иисус сказал: «Ободрись; это я; не бойся». Для смятенной души Петра это было подобно бальзаму галаадскому; это успокоило его встревоженный дух, так что (во сне) он прокричал Учителю: «Господи! Если это действительно ты, повели мне прийти и ходить с тобой по воде». Когда же Петр пошел по воде, неистовые волны испугали его, и он, начав тонуть, закричал: «Господи, спаси меня!» И многие из двенадцати апостолов слышали, как он издал этот крик. Потом Петру снилось, что Иисус пришел к нему на помощь и, протянув свою руку, поддержал его и сказал: «О маловерный, зачем ты усомнился?»
\vs p152 4:3 Под влиянием только что увиденного во сне Петр встал с сидения, на котором спал, и действительно перешагнул через борт лодки в воду. Очнулся же он ото сна, когда Андрей, Иаков и Иоанн нагнулись через борт лодки и вытащили его из моря.
\vs p152 4:4 Для Петра это ощущение всегда было реальным. Он искренне верил, что Иисус приходил к ним той ночью. В этом он лишь в какой\hyp{}то степени убедил Иоанна Марка, что и объясняет, почему Марк не включил часть этой истории в свое повествование. Врач же Лука, тщательно изучивший этот вопрос, заключил, что данный эпизод был видением Петра, и поэтому, готовя свое повествование, отказался ввести в него эту историю.
\usection{5. Снова в Вифсаиде}
\vs p152 5:1 В четверг утром перед рассветом они поставили свою лодку на якорь недалеко от дома Зеведея и проспали почти до полудня. Первым проснулся Андрей и, прогуливаясь возле моря, набрел на Иисуса, который сидел у края воды вместе с их посыльным мальчиком. Хотя многие из толпы и молодые евангелисты всю ночь и большую часть следующего дня искали Иисуса в горах на востоке, он и мальчик Марк вскоре после полуночи отправились пешком вокруг озера и через реку назад в Вифсаиду.
\vs p152 5:2 \pc Из пяти тысяч тех, кто чудом насытился и хотел сделать его царем, когда животы их были сыты, а сердца пусты, лишь около пятисот человек продолжали следовать за ним. Однако до того, как они узнали, что Иисус вернулся в Вифсаиду, он попросил Андрея собрать двенадцать апостолов и их сподвижников, включая женщин, и сказал: «Я желаю говорить с ними». Когда же все были готовы, Иисус сказал:
\vs p152 5:3 \pc «Доколе буду терпеть вас? Неужели вы все так медлительны в духовном познании и в вас недостает живой веры? Все эти месяцы я учил вас истинам царства, и, тем не менее, вами движут материальные мотивы, а не духовные соображения. Неужели вы даже не читали в Писании, как Моисей увещевал неверующих детей Израиля, говоря: „Не бойтесь, стойте и увидите спасение Господне“? Псалмопевцем сказано: „Уповайте на Господа“. „Будь терпелив, надейся на Господа и мужайся. Он укрепит сердце твое“. „Возложи на Господа заботы твои, и он поддержит тебя. Уповай на него во все времена и изливай ему сердце свое, ибо Бог --- прибежище твое“. „Живущий под кровом Всевышнего под сенью Всемогущего покоится“. „Лучше уповать на Господа, нежели надеяться на князей человеческих“.
\vs p152 5:4 И ныне все ли вы видите, что чудотворство не обращает души к духовному царству? Мы накормили толпу, но люди от этого не взалкали хлеба жизни и не возжаждали вод духовной праведности. Когда голод их был утолен, они искали не входа в царство небесное, но пытались провозгласить Сына Человеческого царем, как провозглашают царей мира сего затем лишь, чтобы они могли продолжать есть хлеб без необходимости трудиться для этого. И все это, в чем многие из вас так или иначе принимали участие, никоим образом не открывает Отца Небесного и не способствует распространению его царства на земле. Разве недостаточно у нас врагов среди религиозных лидеров страны и без того, что может вызвать недовольство гражданских правителей? Я молюсь о том, чтобы Отец умастил глаза ваши, дабы вы могли видеть, и открыл уши ваши, дабы вы могли слышать, чтобы вы полностью уверовали в евангелие, которому я учил вас».
\vs p152 5:5 \pc Затем Иисус объявил, что он желал бы удалиться со своими апостолами и отдохнуть несколько дней перед тем, как отправиться в Иерусалим на Пасху, и запретил кому бы то ни было из учеников или толпы следовать за ним. Поэтому они на лодке отбыли в Генисаретскую область, чтобы дня два\hyp{}три отдохнуть и выспаться. Иисус готовился к великому переломному моменту своей жизни на земле и потому много времени провел, общаясь с Отцом Небесным.
\vs p152 5:6 Вести о насыщении пяти тысяч и попытке сделать Иисуса царем возбудили величайший интерес и разбудили страхи у религиозных лидеров и гражданских правителей по всей Галилее и Иудее. Это великое чудо не сделало ничего для развития евангелия царства в душах материально настроенных и сомневающихся верующих. Однако оно выявило тенденции апостолов и избранных учеников Иисуса искать чуда и сделать его царем. Это знаменательное событие завершило начальный период учения, воспитания и целительства и тем самым подготовило путь для вступления в последний год провозглашения более высоких и более духовных аспектов нового евангелия царства о божественном сыновстве, духовной свободе и вечном спасении.
\usection{6. В Генисарете}
\vs p152 6:1 Отдыхая в доме богатого верующего в районе Генисарета, Иисус каждый день после полудня беседовал с двенадцатью апостолами. Вестники царства были, серьезными и сдержанными людьми, утратившими иллюзии. Однако даже после всего, что произошло, как показали дальнейшие события, этим двенадцати все же не удалось до конца избавиться от своих впитанных с молоком матери и давно лелеемых представлений о пришествии еврейского Мессии. События нескольких предшествующих недель развивались слишком быстро, чтобы эти ошеломленные рыбаки поняли все их значение. Ведь для того, чтобы произошли радикальные и глубокие перемены в основных, фундаментальных представлениях мужчин и женщин об общественных отношениях, философских взглядах и религиозных убеждениях, требуется время.
\vs p152 6:2 Пока Иисус и двенадцать апостолов отдыхали в Генисарете, толпа рассеялась, одни ушли к своим домам, другие отправились в Иерусалим на Пасху. Меньше, чем через один месяц от полных энтузиазма и открытых последователей Иисуса, которых в одной Галилее насчитывалось более пятидесяти тысяч, осталось не более пятисот человек. Иисус стремился, чтобы апостолы на собственном опыте познали переменчивость народных восхвалений, дабы они не поддавались бы искушению полагаться на подобные проявления недолговечной религиозной истерии после того, как он оставит их в деле царства одних, но эта его попытка удалась лишь отчасти.
\vs p152 6:3 \pc Во вторую ночь пребывания в Генисарете Учитель снова рассказал апостолам притчу о сеятеле и прибавил такие слова: «Как видите, дети мои, призыв к человеческим чувствам преходящ и ведет к крайнему разочарованию; призыв исключительно к интеллекту человека точно так же пуст и бесплоден; лишь обращаясь к духу, живущему в разуме человека, вы можете надеяться достигнуть непреходящего успеха и добиться тех удивительных преображений в характере человека, которые проявятся в богатом урожае подлинных плодов духа в повседневной жизни тех, кто рождением от духа был избавлен от тьмы сомнений и перенесен во свет веры --- в царство небесное».
\vs p152 6:4 \pc Иисус учил обращаться к чувствам как методу, способному привлечь и сосредоточить внимание интеллекта. Пробужденный таким образом и оживленный разум он называл вратами души, где покоится духовная сущность человека, которая и должна распознавать истину и отвечать на духовный призыв евангелия, дабы достичь постоянных результатов в истинном преображении характера.
\vs p152 6:5 Иисус, таким образом, попытался подготовить апостолов к предстоящему потрясению --- к перелому в отношении к нему народа, до которого оставалось всего несколько дней. Он объяснил двенадцати апостолам, что религиозные правители Иерусалима вступят в сговор с Иродом Антипой, чтобы уничтожить их. Двенадцать апостолов начали отчетливее (хотя и не до конца) понимать, что Иисус не собирался воссесть на престол Давида. Они яснее увидели, что духовную истину нельзя утвердить материальными чудесами. Они стали осознавать, что насыщение пяти тысяч и порыв народа сделать Иисуса царем, было пиком поисков сверхъестественного и, ожидания чудотворства, высшей точкой популярности Иисуса среди народа. Они в какой\hyp{}то мере различали и смутно предвидели приближающиеся времена духовного отсева и жестоких несчастий. Эти двенадцать человек медленно начинали осознавать подлинную сущность их задачи как посланников царства и стали готовить себя к тяжелым и суровым испытаниям последнего года земного служения Иисуса.
\vs p152 6:6 \pc Перед тем, как покинуть Геннисарет, Иисус объяснил им суть чудесного насыщения пяти тысяч, рассказав, почему он осуществил это необычайное проявление творческой силы, и уверил их в том, что он не поддавался своему чувству сострадания толпе до тех пор, пока не удостоверился, что это «соответствовало воле Отца».
\usection{7. В Иерусалиме}
\vs p152 7:1 апреля в воскресенье Иисус в сопровождении только двенадцати апостолов отправился в путь из Вифсаиды в Иерусалим. Чтобы избавиться от толп и как можно меньше привлекать к себе внимание, они пошли дорогой, проходившей через Герасу и Филадельфию. В этом путешествии Иисус запретил апостолам учить публично; он также не позволил им учить или проповедовать во время пребывания в Иерусалиме. Поздно вечером в среду 6 апреля они прибыли в Вифанию, находившуюся недалеко от Иерусалима. На эту одну ночь они остановились в доме Лазаря, Марфы и Марии, но на следующий день расстались. Иисус с Иоанном жили у верующего по имени Симон, чей дом находился недалеко от дома Лазаря в Вифании. Иуда же Искариот и Симон Зилот остановились у друзей в Иерусалиме, тогда как остальные апостолы жили по двое в разных домах.
\vs p152 7:2 Во время этой Пасхи Иисус входил в Иерусалим только один раз, и было это в великий день праздника. Авенир приводил в Вифанию многих верующих из Иерусалима для встречи с Иисусом. Во время этого пребывания в Иерусалиме двенадцать апостолов узнали, каким неприязненным становилось отношение к их Учителю. Все они ушли из Иерусалима с мыслью, что кризис приближается.
\vs p152 7:3 В воскресенье 24 апреля Иисус и апостолы покинули Иерусалим и отправились в Вифсаиду, путем, проходившем через приморские города Иоппию, Кесарию и Птолемаиду. Отсюда они шли через Раму и Хоразин и прибыли в Вифсаиду в пятницу 29 апреля. Придя домой, Иисус немедленно послал Андрея испросить у управителя синагоги разрешения прочесть проповедь на следующий день, в субботу, во время послеполуденной службы. Иисус хорошо знал, что это будет последний раз, когда ему разрешат говорить в капернаумской синагоге.
