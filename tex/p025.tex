\upaper{25}{Сонмы Вестников Пространства}
\author{Облеченный Высокой Властью}
\vs p025 0:1 Промежуточное положение в семье Бесконечного Духа занимают Сонмы Вестников Пространства. Эти разносторонние существа функционируют как связующие звенья между высшими личностями и духами\hyp{}служителями. Сонмы вестников включают следующие чины небесных существ:
\vs p025 0:2 \ublistelem{1.}\bibnobreakspace Сервиталы Хавоны.
\vs p025 0:3 \ublistelem{2.}\bibnobreakspace Вселенские Примирители.
\vs p025 0:4 \ublistelem{3.}\bibnobreakspace Технические Советчики.
\vs p025 0:5 \ublistelem{4.}\bibnobreakspace Хранители Записей в Раю.
\vs p025 0:6 \ublistelem{5.}\bibnobreakspace Небесные Протоколисты.
\vs p025 0:7 \ublistelem{6.}\bibnobreakspace Моронтийные Компаньоны.
\vs p025 0:8 \ublistelem{7.}\bibnobreakspace Райские Компаньоны.
\vs p025 0:9 \pc Из перечисленных семи групп только три --- сервиталы, примирители и Моронтийные Компаньоны --- создаются как таковые; остальные четыре представляют уровни достижений ангельских чинов. Сонмы вестников служат различным образом во вселенной вселенных --- в соответствии с присущей им природой и достигнутым статусом, но они всегда подчиняются руководству тех, кто правит сферами их назначения.
\usection{1. Сервиталы Хавоны}
\vs p025 1:1 Хотя они и называются сервиталами, эти «срединные создания» центральной вселенной не являются слугами в прямом «лакейском» смысле этого слова. В духовном мире нет такой вещи, как работа лакея; всякая служба священна и радостна, и высшие чины существ не смотрят свысока на нижние чины.
\vs p025 1:2 \pc Сервиталы Хавоны есть результат совместной творческой работы Семи Духов\hyp{}Мастеров и их сподвижников --- Семи Верховных Управителей Мощи. Это творческое сотрудничество более всего подходит к тому, чтобы быть паттерном для длинного списка воспроизводства двуединого чина в эволюционирующих вселенных, простирающегося от сотворения Яркой и Утренней Звезды в результате связи Сына\hyp{}Творца и Творческого Духа до полового размножения в мирах, подобных Урантии.
\vs p025 1:3 Численность сервиталов огромна, и их создание никогда не прекращается. Они появляются группами по одной тысяче в третий момент после собрания Духов\hyp{}Мастеров и Верховных Управителей Мощи в их общей области, находящейся в далеком северном секторе Рая. Каждый четвертый сервитал по своему типу более материален, чем остальные; то есть из каждой тысячи семьсот пятьдесят, по\hyp{}видимому, действительно принадлежат к духовному типу, а двести пятьдесят являются по своей природе полуматериальными. Эти \bibemph{четвертые создания} в чем\hyp{}то походят на чин материальных существ (материальных, как это присуще Хавоне), напоминая больше управителей физической мощи, чем Духов\hyp{}Мастеров.
\vs p025 1:4 \pc В личностных отношениях духовное преобладает над материальным, хотя в настоящее время на Урантии кажется, что это не так; и в производстве Сервиталов Хавоны господствует закон преобладания духа; установленное соотношение --- три духовных существа к одному полуматериальному.
\vs p025 1:5 \pc Вновь сотворенные сервиталы вместе с вновь появившимися Проводниками Выпускников --- все проходят курсы обучения, которые старшие проводники постоянно ведут на каждом из семи контуров Хавоны. Затем сервиталам поручаются определенные виды деятельности, к которым они лучше всего приспособлены, а поскольку они бывают двух типов --- духовные и полуматериальные, --- диапазон работ, которые эти разносторонние существа могут выполнять, почти безграничен. Высшие или духовные группы избираются на службу Отца, Сына и Духа и на работу Семи Духов\hyp{}Мастеров. Время от времени большое число их посылается служить в миры обучения, окружающие сферы\hyp{}центры семи сверхвселенных, миры, которые посвящены завершающей стадии воспитания и духовному развитию восходящих душ, живущих во времени, которые готовятся для продвижения к контурам Хавоны. И духовные сервиталы, и их более материальные собратья также назначаются помощниками и сподвижниками Проводников Выпускников для помощи и обучения различных чинов восходящих созданий, которые достигли Хавоны и которые стремятся достичь Рая.
\vs p025 1:6 Сервиталы Хавоны и Проводники Выпускников проявляют необыкновенную преданность своей работе и трогательную привязанность друг к другу, привязанность, которую, хоть она и духовная, вы можете понять, только сравнив ее с таким феноменом, как человеческая любовь. Есть некий божественный пафос в отделении сервиталов от проводников, как это часто случается, когда сервиталы направляются с миссиями за пределы центральной вселенной; но они идут на это с радостью, а не с сожалением. Радость удовлетворения от выполнения высокого долга есть чувство, которое у духовных существ затмевает все остальные. Не может быть сожаления, если есть осознание должным образом исполненного божественного долга. И когда восходящая человеческая душа предстает перед Верховным Судьей, решение, имеющее непреходящее значение, не будет определяться материальными успехами или количественными достижениями; вердикт, объявленный в высоком суде, гласит: «Прекрасно, добрый и \bibemph{верный} слуга; ты был преданным в самом главном; ты будешь правителем вселенских реальностей».
\vs p025 1:7 На сверхвселенской службе Сервиталы Хавоны всегда назначаются в ту область, которую возглавляет Дух\hyp{}Мастер, с которым они наиболее схожи в общих и особых прерогативах духа. Они служат только в образовательных мирах, окружающих столицы семи сверхвселенных, и последний отчет Уверсы показывает, что почти 138 миллиардов сервиталов служили на ее 490 спутниках. Они занимаются бесконечно разнообразной деятельностью, связанной с работой этих образовательных миров, составляющих сверхуниверситеты сверхвселенной Орвонтона. Здесь они --- ваши компаньоны; они следуют по вашему пути, чтобы изучить вас и воодушевить реальностью и несомненностью вашего окончательного перехода от вселенных времени к сферам вечности. И в этом контакте сервиталы приобретают тот предварительный опыт служения восходящим созданиям времени, который столь полезен для их последующей работы на контурах Хавоны в качестве сподвижников Проводников Выпускников или --- как, например перенесенные сервиталы --- в качестве самих Проводников Выпускников.
\usection{2. Вселенские Примирители}
\vs p025 2:1 Когда создается один Сервитал Хавоны одновременно порождается семь Вселенских Примирителей, по одному в каждой сверхвселенной. Это творческое деяние включает определенный сверхвселенский метод отражательного отклика на события, происходящие в Раю.
\vs p025 2:2 В мирах\hyp{}центрах семи сверхвселенных функционируют семь отражений Семи Духов\hyp{}Мастеров. Трудно описать природу этих Отражательных Духов человеческому разуму. Они --- истинные личности; и все же каждый член сверхвселенской группы обладает совершенной отражательностью по отношению только к одному из Семи Духов\hyp{}Мастеров. И всякий раз, когда Духи\hyp{}Мастера вступают в союз с управителями мощи с целью сотворения группы Сервиталов Хавоны, происходит одновременное сосредоточение на одном из Отражательных Духов в каждой из сверхвселенских групп, и тотчас в мирах\hyp{}центрах сверхтворений, вполне оформившись, появляется равное число Вселенских Примирителей. Если в процессе сотворения сервиталов Дух\hyp{}Мастер Номер Семь проявляет инициативу, никто иной как Отражательный Дух седьмого чина становится беременным примирителями; и одновременно с сотворением одной тысячи сервиталов орвонтонского типа в столице каждой сверхвселенной появляется одна тысяча примирителей седьмого чина. В таких случаях, отражающих семеричную природу Духов\hyp{}Мастеров, возникают семь сотворенных чинов примирителей, служащих в каждой сверхвселенной.
\vs p025 2:3 Примирители, имеющие предрайский статус, при прохождении службы не меняют сверхвселенные, так как ограничены своими исконными зонами творения. Отряд каждой сверхвселенной, включающий одну седьмую часть каждого сотворенного чина, таким образом, очень долгое время, находится под влиянием одного из Духов\hyp{}Мастеров, не испытывая влияния других, ибо, хотя все семь \bibemph{отражаются} в столицы сверхвселенных, только один является \bibemph{господствующим} в каждом сверхтворении.
\vs p025 2:4 Каждое из семи сверхтворений в действительности заполнено тем из Духов\hyp{}Мастеров, который осуществляет контроль за его предназначением. Таким образом, каждая сверхвселенная подобна гигантскому зеркалу, отражающему природу и характер руководящего Духа\hyp{}Мастера, и все это продолжается в каждой дочерней локальной вселенной благодаря присутствию и функционированию Творческих Духов\hyp{}Матерей. Влияние такого окружения на эволюционный рост столь глубоко, что примирители на своих постсверхвселенских путях совокупно выражают сорок девять полученных с опытом точек зрения или уровней понимания, и в каждой проблема рассматривается под своим углом --- и поэтому неполна --- но все взаимно дополняют друг друга и вместе стремятся охватить круг Верховенства.
\vs p025 2:5 \pc В каждой сверхвселенной Вселенские Примирители объединены в удивительно и изначально разделенные на группы по четыре --- союзы, в которых они продолжают служение. В каждой группе трое --- духовные личности, а один, как четвертое создание сервиталов, --- полуматериальное существо. Эта четверка образует примирительную комиссию, в которую входят:
\vs p025 2:6 \ublistelem{1.}\bibnobreakspace \bibemph{Судья\hyp{}Арбитр.} Единогласно назначен остальными тремя как самый компетентный и квалифицированный для того, чтобы выступать как судейский глава группы.
\vs p025 2:7 \ublistelem{2.}\bibnobreakspace \bibemph{Дух\hyp{}Защитник.} Назначен судьей\hyp{}арбитром представлять свидетельские показания, защищать права всех личностей, фигурирующих в любом деле, назначенном к судебному рассмотрению примирительной комиссией.
\vs p025 2:8 \ublistelem{3.}\bibnobreakspace \bibemph{Божественный Исполнитель.} Примиритель по его врожденным свойствам определен осуществлять контакт с материальными существами сфер и исполнять решения комиссии. Божественные исполнители, будучи четвертыми созданиями --- квазиматериальными существами, --- почти способны быть видимыми в узком поле зрения смертных рас.
\vs p025 2:9 \ublistelem{4.}\bibnobreakspace \bibemph{Протоколист.} Оставшийся член комиссии автоматически становится протоколистом, клерком суда. Он удостоверяет, что все записи должным образом приготовлены для архивов сверхвселенной и для записей локальной вселенной. Если комиссия служит в эволюционирующем мире, с помощью исполнителя подготавливается третий отчет для материальных записей правительства соответствующей системы подведомственной области.
\vs p025 2:10 \pc Во время заседания комиссия функционирует как группа, состоящая из трех членов, так как защитник не присутствует во время рассмотрения и принимает участие в составлении вердикта только на заключительном слушании. Поэтому эти комиссии иногда называют судейскими трио.
\vs p025 2:11 \pc Примирители очень важны для поддержания плавного хода дел вселенной вселенных. Пересекая пространство со скоростью серафимов, равной тройной скорости, они служат членами передвижных судов миров, комиссий, занимающихся быстрым разрешением незначительных проблем. Если бы не эти мобильные и чрезвычайно справедливые комиссии, трибуналы сфер безнадежно погрязли бы в незначительных проблемах своих областей.
\vs p025 2:12 Эти судейские трио не рассматривают дела, имеющие непреходящее значение; душе, вечным перспективам созданий времени никогда не угрожает опасность из\hyp{}за их действий. Примирители никогда не имеют дела с проблемами, простирающимися за пределы преходящего существования и космического благополучия созданий, живущих во времени. Но если комиссия однажды приняла проблему в судебное производство, ее решение является окончательным и всегда единогласным; обжалованию решение судьи\hyp{}арбитра не подлежит.
\usection{3. Широко разветвленная Служба Примирителей}
\vs p025 3:1 Центр группы примирителей находится в столице их сверхвселенной, где размещается их первичный резервный отряд. Их вторичные резервы располагаются в столицах локальных вселенных. Более молодые и менее опытные члены комиссий начинают свою работу в низших мирах --- мирах, подобных Урантии, и допускаются к решению более трудных проблем после того, как приобретут достаточный опыт.
\vs p025 3:2 Чин примирителей вполне заслуживает доверия; никогда ни один не сбивался с пути. Хотя их суждения и не безошибочны в мудрости и в рассудительности, они, безусловно, надежны и безупречно верны. Они берут свое начало в центрах сверхвселенных и, в конце концов, возвращаются туда, продвигаясь по следующим уровням вселенского служения:
\vs p025 3:3 \ublistelem{1.}\bibnobreakspace \bibemph{Примирители Миров.} Когда руководящие личности отдельных миров сталкиваются со значительными трудностями или в конечном счете не приходят к единому мнению относительно правильности действий в данных обстоятельствах, тогда, если дело не является столь существенно важным, что должно быть представлено на рассмотрение официально учрежденным трибуналам данной сферы, по ходатайству двух личностей, по одному от каждой стороны, примирительная комиссия немедленно начинает действовать.
\vs p025 3:4 Когда эти административные и правовые проблемы, связанные с судопроизводством, отданы в руки примирителей для изучения и вынесения решения, их власть становится верховной. Но они не выносят решения, пока не выслушаны все свидетельские показания, и абсолютно не ограничено их право вызывать свидетелей --- откуда угодно и когда угодно. Хотя их решения не могут быть обжалованы, иногда дело принимает такой оборот, что комиссия прекращает вести свои протоколы на определенном пункте, заканчивает прения, и передает весь вопрос на рассмотрение более высоким судам сферы.
\vs p025 3:5 Решения примирителей фиксируются в документах планеты и, в случае необходимости, реализуются божественным исполнителем. Его власть очень велика и диапазон деятельности в обитаемом мире очень широк. Божественные исполнители искусно манипулируют тем, что делается в интересах того, что должно быть. Иногда их работа действительно направлена на благосостояние сферы, а иногда их действия в мирах пространства и времени трудно объяснить. Хотя исполнение решений суда не противоречит естественному закону или сложившимся обычаям сферы, зачастую исполнители совершают свои удивительные дела и осуществляют наказы примирителей в соответствии с высшими законами управления системой.
\vs p025 3:6 \ublistelem{2.}\bibnobreakspace \bibemph{Примирители Центров Систем.} От службы в эволюционных мирах эти комиссии четырех продвигаются к службе в центрах систем. Здесь у них много работы, и они становятся чуткими друзьями людей, ангелов и других духовных существ. Эти судебные трио заняты не столько личностными трудностями, сколько разногласиями групп, размолвками, возникающими между различными чинами созданий; а в центрах систем живут и духовные, и материальные существа, и смешанные типы, такие, как Материальные Сыны.
\vs p025 3:7 В момент, когда Творцы порождают развивающихся индивидуумов, обладающих способностью выбора, возникает отклонение от плавной работы божественного совершенства; случаются и определенные разногласия, и, следовательно, должны быть созданы условия для справедливого сглаживания этих различных, но искренних точек зрения. Мы все должны помнить, что всемудрые и всесильные Творцы могли бы сделать локальную вселенную столь же совершенной, как Хавона. В центральной вселенной нет необходимости в деятельности примирительных комиссий. Но Творцы в своей всемудрости решили не делать это. И в то время как они творят вселенные, которые изобилуют различиями и разнообразными трудностями, они также обеспечивают механизмы и средства для сглаживания этих различий и создания гармонии во всей этой кажущейся неразберихе.
\vs p025 3:8 \ublistelem{3.}\bibnobreakspace \bibemph{Примирители Созвездий.} Со службы в системах примирители выдвигаются для вынесения решений по проблемам созвездия, занимаясь там незначительными трудностями, возникающими между сотней его систем обитаемых миров. Лишь немногие проблемы, возникающие в центрах созвездий, подпадают под их юрисдикцию, но они постоянно заняты, проходя от системы к системе, собирая свидетельские показания и подготавливая предварительные документы. Если спор честен, если трудности возникают из\hyp{}за искреннего несовпадения мнений и откровенного несходства позиций, независимо от того, как мало лиц может быть вовлечено в этот спор и насколько очевидно тривиально разногласие, примирительная комиссия может и готова всегда высказать свою точку зрения.
\vs p025 3:9 \ublistelem{4.}\bibnobreakspace \bibemph{Примирители Локальных Вселенных.} В этой более обширной работе вселенной члены комиссии оказывают большую помощь и Мелхиседекам, и Сынам\hyp{}Повелителям, и правителям созвездий, и сонмам личностей, занимающимся согласованием и управлением сотней созвездий. Различные чины серафимов и других, живущих в сферах\hyp{}центрах локальной вселенной, также пользуются помощью и решениями судебных трио.
\vs p025 3:10 Почти невозможно объяснить природу тех разногласий, которые могут возникать в различных делах системы, созвездия или вселенной. Конечно, трудности появляются, но они совершенно не похожи на мелочные тяжбы и муки материального существования, какими полна жизнь в эволюционирующих мирах.
\vs p025 3:11 \ublistelem{5.}\bibnobreakspace \bibemph{Примирители Малых Секторов Сверхвселенной.} От проблем локальных вселенных примирители продвигаются к изучению вопросов, возникающих в малых секторах своей сверхвселенной. Чем дальше они восходят вовнутрь от отдельных планет, тем меньшими становятся материальные обязанности божественного исполнителя; постепенно он принимает новую роль интерпретатора милосердия\hyp{}справедливости, сохраняя --- будучи квазиматериальным --- комиссию как целое в полных сочувствия контактах с материальными аспектами ее исследований.
\vs p025 3:12 \ublistelem{6.}\bibnobreakspace \bibemph{Примирители Больших Секторов Сверхвселенной.} Характер работы членов комиссии продолжает изменяться по мере их продвижения. Становится все меньше и меньше спорных вопросов, которые следует урегулировать, и все больше и больше таинственных явлений, которые надо объяснять и интерпретировать. От этапа к этапу они развиваются от арбитров, улаживающих разногласия, до \bibemph{объясняющих тайны ---} судей, становящихся учителями\hyp{}интерпретаторами. Они были когда\hyp{}то арбитрами тех, кто по незнанию допускал разногласия и недоразумения; теперь же они становятся наставниками тех, кто достаточно умен и терпим, чтобы избежать столкновений разума и войны мнений. Чем выше образование создания, тем больше он уважает знание, опыт и мнения других.
\vs p025 3:13 \ublistelem{7.}\bibnobreakspace \bibemph{Примирители Сверхвселенных.} Здесь члены комиссий становятся равноправными --- четырьмя понимающими друг друга и идеально функционирующими арбитрами\hyp{}учителями. Божественный исполнитель лишается своей карательной власти и становится физическим голосом духовного трио. К этому времени эти советники и учителя уже профессионально знакомы с большинством актуальных проблем и трудностей, которые встречаются при ведении дел сверхвселенных. Таким образом, они становятся замечательными советчиками и мудрыми учителями восходящих пилигримов, которые живут в учебных сферах, окружающих миры\hyp{}центры сверхвселенных.
\vs p025 3:14 \pc Все примирители служат под общим руководством Древних Дней и под непосредственным началом Помощников Изображения до той поры, пока они не продвинутся к Раю. Во время пребывания в Раю они отчитываются перед Духом\hyp{}Мастером, который возглавляет сверхвселенную, из которой они произошли.
\vs p025 3:15 В реестрах сверхвселенных не числятся те примирители, которые вышли из их юрисдикции, а такие комиссии широко распространены по всей великой вселенной. Из последней записи в реестре на Уверсе следует, что число таких комиссий, действующих в Орвонтоне, почти восемнадцать триллионов --- т.е. включает свыше семидесяти триллионов членов. Но это всего лишь весьма малая часть множества примирителей, которые были созданы в Орвонтоне; их численность в целом намного больше и эквивалентна общему числу Сервиталов Хавоны, не считая тех, кто находится в процессе превращения в Проводников Выпускников.
\vs p025 3:16 Время от времени, по мере того, как число примирителей сверхвселенных увеличивается, они переносятся в советы совершенства в Рай, откуда впоследствии выходят в качестве членов отряда согласования, выделенного Бесконечным Духом для вселенной вселенных; это изумительная группа существ, число и эффективность которых постоянно увеличивается. Посредством опытного восхождения и Райского обучения они приобрели уникальную способность опознавать появляющуюся реальность Верховного Существа и странствуют по вселенной вселенных, выполняя специальные задания.
\vs p025 3:17 Члены примирительной комиссии никогда не расходятся. Группа четырех служит вечно вместе в том виде, в каком они были первоначально объединены друг с другом. Даже в их прославленном служении они продолжают функционировать как квартеты аккумулированного космического опыта и усовершенствованной опытной мудрости. Они навечно связаны как воплощение верховной справедливости времени и пространства.
\usection{4. Технические Советчики}
\vs p025 4:1 Эти юридические и технические разумы духовного мира не были созданы таковыми. Из первых супернафимов и омниафимов Бесконечным Духом был выбран миллион наиболее организованных разумов в качестве ядра этой громадной и разносторонней группы. И с той древней поры всем, кто стремится стать Техническими Советчиками, необходимо иметь реальный опыт применения законов совершенства к планам эволюционирующего творения.
\vs p025 4:2 \pc Технические Советчики набираются из рангов следующих личностных чинов:
\vs p025 4:3 \ublistelem{1.}\bibnobreakspace Супернафимы.
\vs p025 4:4 \ublistelem{2.}\bibnobreakspace Секонафимы.
\vs p025 4:5 \ublistelem{3.}\bibnobreakspace Терциафимы.
\vs p025 4:6 \ublistelem{4.}\bibnobreakspace Омниафимы.
\vs p025 4:7 \ublistelem{5.}\bibnobreakspace Серафимы.
\vs p025 4:8 \ublistelem{6.}\bibnobreakspace Некоторые типы восходящих смертных.
\vs p025 4:9 \ublistelem{7.}\bibnobreakspace Некоторые типы восходящих срединников.
\vs p025 4:10 \pc В настоящее время, не считая всех смертных и срединников, которые имеют временное прикрепление, число Технических Советчиков, зарегистрированных на Уверсе и действующих в Орвонтоне, ненамного превышает шестьдесят один триллион.
\vs p025 4:11 Технические Советчики часто функционируют как индивидуумы, но они организованы в группы по семь для прохождения службы и поддержания общих центров в сферах назначения. В каждой группе, по крайней мере, пять имеют постоянный статус, а два могут быть временно связаны с этой группой. Восходящие смертные и восходящие срединные создания служат в этих консультативных комиссиях, когда следуют по пути Райского восхождения, но они не проходят курсов обучения, обычных для Технических Советчиков, и никогда не становятся постоянными членами этого чина.
\vs p025 4:12 Те смертные и срединники, которые временно служат вместе с советчиками, выбираются для такой работы потому, что они, как эксперты, разбираются в понятиях вселенского закона и верховной справедливости. По мере того, как ты путешествуешь к твоей Райской цели, постоянно приобретая дополнительное знание и возрастающее мастерство, тебе непрерывно дается возможность поделиться с другими мудростью и опытом, которые ты уже накопил; на всем пути внутрь к Хавоне ты исполняешь роль ученика\hyp{}учителя. Ты будешь прокладывать свой путь через восходящие уровни этого обширного университета опыта посредством передачи тем, кто ниже тебя, вновь обретенного знания, полученного на пути восхождения. Во вселенском масштабе ты не считаешься обладающим знанием и истиной до тех пор, пока не продемонстрируешь способность и готовность передать это знание и истину другим.
\vs p025 4:13 После долгого обучения и получения реального опыта любому из духов\hyp{}служителей, обладающему более высоким статусом, чем статус херувима, позволяется получить постоянное назначение в качестве Технического Советчика. Все кандидаты вступают в этот чин служения добровольно; но, приняв однажды на себя такие обязанности, они не могут от них отказаться. Только Древние Дней могут перевести этих советчиков на другую работу.
\vs p025 4:14 \pc Обучение Технических Советчиков, начавшееся в колледжах Мелхиседеков локальных вселенных, продолжается при судах Древних Дней. От этого сверхвселенского обучения они переходят далее в «школы семи кругов», расположенные в путеводных мирах контуров Хавоны. И с путеводных миров они принимаются в «колледжи этики права и методов Верховенства» --- Райскую специальную школу совершенствования Технических Советчиков.
\vs p025 4:15 Эти советчики --- нечто большее, чем юридические эксперты; они --- студенты и учителя \bibemph{прикладного} права, законов вселенной, примененных к жизни и предназначению всех, кто населяет громадные области широко раскинувшегося творения. Со временем они становятся живыми юридическими библиотеками пространства и времени, причем они предотвращают бесконечные хлопоты и ненужные задержки, информируя личности, живущие во времени, о формах и видах процедур, наиболее приемлемых для правителей вечности. Они способны дать такой совет работникам пространства, который позволит тем действовать в согласии с требованиями Рая; они являются учителями всех созданий в том, что касается методов Творцов.
\vs p025 4:16 Такие живые библиотеки прикладного права не могли быть сотворены; такие существа должны были быть развиты посредством реального опыта. Бесконечные Божества экзистенциальны; и это компенсирует в них отсутствие опыта; они знают все даже до того, как испытают все на опыте, но они не передают это неопытное знание своим низшим созданиям.
\vs p025 4:17 \pc Предназначение Технических Советчиков --- предотвращать задержки, содействовать прогрессу и советовать, как достичь цели. Всегда существует \bibemph{наилучший} и \bibemph{правильный} способ достижения; всегда существует метод совершенства --- божественный метод, и эти советчики знают, как направить всех нас, чтобы мы смогли найти этот лучший способ.
\vs p025 4:18 Эти чрезвычайно мудрые и практичные существа всегда тесно связаны со службой и работой Вселенских Цензоров. Умелый отряд таких существ дан Мелхиседекам. Правители систем, созвездий, вселенных и секторов сверхвселенной также все щедро обеспечены этими техническими или юридическими справочными умами духовного мира. Особая группа действует в качестве юридических советников Носителей Жизни, консультируя этих Сынов, насколько допустимы отклонения от установленного порядка распространения жизни, или иным образом информируя об их прерогативах и свободе действия. Они являются советчиками для всех классов существ по вопросам, касающимся должного использования и методов всех свершений духовного мира. Но они --- лично и непосредственно --- не имеют дела с материальными существами миров.
\vs p025 4:19 Помимо консультирования относительно правовых норм, Технические Советчики в равной степени занимаются эффективной интерпретацией всех законов, касающихся сотворенных существ --- материальных, интеллектуальных и духовных. Они находятся в распоряжении Вселенских Примирителей и всех, кто действительно желает знать истину закона; другими словами, кто желает знать, как Верховенство Божества будет реагировать в любой конкретной ситуации, учитывая факторы установленного материального, интеллектуального и духовного порядка. Они пытаются даже прояснить методы Предельного.
\vs p025 4:20 Технические Советчики --- избранные и испытанные существа; я никогда не знал ни одного из них, кто бы сбился с пути. На Уверсе у нас нет записей о том, что они когда\hyp{}либо были признаны виновными в неуважении к божественным законам, которые они столь эффективно интерпретируют и столь красноречиво разъясняют. Предел областям их служения не известен, и не поставлен предел их продвижению. Они продолжают быть советчиками даже у ворот Рая; вся вселенная законов и опыта открыта для них.
\usection{5. Хранители Записей в Раю}
\vs p025 5:1 В Хавоне отдельные старшие главы протоколистов из среды третичных супернафимов избираются Хранителями Записей --- хранителями официальных архивов Острова Света, тех архивов, которые отличаются от живых реестров в разумах хранителей знания, которых иногда называют «живыми библиотеками Рая».
\vs p025 5:2 Ангелы\hyp{}протоколисты обитаемых планет --- источники всех индивидуальных записей. Повсюду во вселенных функционируют другие протоколисты, которые занимаются и письменными документами и живыми записями. На пути от Урантии до Рая встречаются оба вида записей: в локальной вселенной больше письменных записей и меньше живых; в Раю --- больше живых и меньше письменных; на Уверсе они имеются в равном количестве.
\vs p025 5:3 Каждое значительное событие в формированном и обитаемом творении подлежит регистрации. В то время как события местного значения регистрируются лишь в локальных записях, с событиями, которые имеют более широкую значимость, поступают следующим образом. С планет, систем и созвездий Небадона все, что имеет вселенскую важность, фиксируется в Спасограде; а из таких вселенских столиц эти эпизоды передаются далее для записи в более высоких документах, которые относятся к делам правительств секторов и сверхвселенных. Рай также имеет соответствующую сводку по данным сверхвселенных и Хавоны; и этот аккумулированный исторический рассказ о вселенной вселенных находится под охраной возвышенных третичных супернафимов.
\vs p025 5:4 Хотя некоторые из этих существ были посланы в сверхвселенные для службы в качестве Глав Записей, руководящих деятельностью Небесных Протоколистов, ни один не был выведен из постоянного реестра их чина.
\usection{6. Небесные Протоколисты}
\vs p025 6:1 Существуют протоколисты, которые выполняют все записи в двух экземплярах, делая оригинальную духовную запись и полуматериальный дубликат --- то, что можно назвать копией через копирку. Они могут поступать таким образом, потому что обладают характерной для них способностью одновременно манипулировать и духовной, и материальной энергией. Небесные Протоколисты не были созданы таковыми; они --- восходящие серафимы из локальных вселенных. Они принимаются, классифицируются и назначаются в сферы своей работы советами Глав Записей в центрах семи сверхвселенных. Там также находятся школы для обучения Небесных Протоколистов. Школой на Уверсе руководят Совершенствователи Мудрости и Божественные Советники.
\vs p025 6:2 Когда протоколисты продвигаются во вселенской службе, они продолжают исполнять свои двойные записи, делая, таким образом, их доступными для всех классов существ --- от материальных чинов до высоких духов света. В вашем преходящем опыте, когда вы совершаете восхождение с этого материального мира, вы всегда сможете посмотреть или каким\hyp{}либо иным образом познакомиться с записями, повествующими об истории и традициях сферы вашего статуса.
\vs p025 6:3 Протоколисты составляют проверенный и испытанный отряд. Я никогда не был извещен о нарушении долга Небесными Протоколистами, и никогда в их записях не было обнаружено фальсификации. Они подвергаются двойной проверке, записи внимательно изучаются их возвышенными собратьями из Уверсы и Могучими Вестниками, которые удостоверяют правильность квазиматериальных копий оригинальных духовных записей.
\vs p025 6:4 В то время как продвигающихся протоколистов, размещенных в Орвонтоне на низших сферах, насчитывается триллионы и триллионы, число тех, кто достигает статуса на Уверсе, меньше восьми миллионов. Старшие протоколисты или протоколисты\hyp{}выпускники являются сверхвселенскими хранителями и экспедиторами сберегаемых записей миров пространства и времени. Их постоянные центры находятся в круглых жилищах, окружающих область записей на Уверсе. Они никогда не передают охрану этих записей другим, какие\hyp{}то протоколисты могут отсутствовать, но большинство --- никогда.
\vs p025 6:5 Как и те супернафимы, которые стали Хранителями Записей, отряд Небесных Протоколистов имеет постоянное назначение. Если серафим и супернафим однажды приняты на свою службу, они останутся, соответственно, Небесными Протоколистами и Хранителями Записей до того дня, когда появится новая и модифицированная администрация полной персонализации Бога Верховного.
\vs p025 6:6 На Уверсе старшие Небесные Протоколисты могут показать записи всего, что имеет космическую важность во всем Орвонтоне со стародавних времен прибытия Древних Дней, в то время как на вечном Острове Хранители Записей охраняют архивы этой сферы, которые свидетельствуют о делах Рая со времен персонификации Бесконечного Духа.
\usection{7. Моронтийные Компаньоны}
\vs p025 7:1 Эти дети Духов\hyp{}Матерей локальных вселенных --- друзья и сподвижники всех, кто живет восходящей моронтийной жизнью. Они не обязательны для реальной работы восходящего создания в процессе продвижения и никоим образом не заменяют работы серафимов\hyp{}хранительниц, которые часто сопровождают своих смертных товарищей в Райском путешествии. Моронтийные Компаньоны просто любезные хозяева для тех, кто только начинает долгое восхождение внутрь. Они также являются искусными устроителями игр и им умело помогают в этой работе руководители восстановления.
\vs p025 7:2 Хотя на моронтийных учебных мирах Небадона вы столкнетесь с выполнением серьезных и все более трудных задач, вам всегда будут регулярно предоставляться периоды для отдыха и восстановления. На протяжении всего путешествия к Раю всегда будет находиться время для отдыха и игр духа; и в пути света и жизни всегда есть время для богопочитания и для нового достижения.
\vs p025 7:3 Моронтийные Компаньоны --- столь дружески расположенные сподвижники, что, когда ты, наконец, покинешь последнюю фазу моронтийного опыта, когда ты подготовишься вступить на путь сверхвселенского духовного восхождения, то искренне пожалеешь, что эти общительные создания не смогут тебя сопровождать, --- они служат только лишь в локальных вселенных. На каждом этапе пути восхождения все личности, с которыми возможно установить контакт, будут общительными и дружелюбными, но, пока ты не встретишься с Райскими Компаньонами, ты не найдешь другой более дружественной и общительной группы.
\vs p025 7:4 Работа Моронтийных Компаньонов будет полнее отражена в повествованиях, касающихся дел твоей локальной вселенной.
\usection{8. Райские Компаньоны}
\vs p025 8:1 Райские Компаньоны --- это смешанная или сборная группа, набранная из существ ранга серафимов, секонафимов, супернафимов и омниафимов. Хотя они служат в течение времени, которое вы посчитали бы чрезвычайно долгим, они не имеют постоянного статуса. Когда их служение завершено, как правило (но не всегда), они возвращаются к обязанностям, которые они выполняли, когда были призваны на Райское служение.
\vs p025 8:2 Члены ангельских сонмов назначаются на эту службу Духом\hyp{}Матерью локальной вселенной, Отражательными Духами сверхвселенной и Маджестоном Рая. Они призываются на центральный Остров и назначаются Райскими Компаньонами одним из Семи Духов\hyp{}Мастеров. За исключением постоянного статуса в Раю, это временное служение Райского общения --- самая большая честь, когда\hyp{}либо оказанная духам\hyp{}служителям.
\vs p025 8:3 Эти избранные ангелы преданы служению общения и назначены как сподвижники ко всем классам существ, которые могут оказаться в Раю одни, --- главным образом, к восходящим смертным, но также и ко всем другим, кто одинок на центральном Острове. Райские Компаньоны в своем общении не имеют никакой особенной цели или назначения, они просто компаньоны. Почти каждое существо, которое вы, смертные, встретите во время своей Райской жизни --- за исключением ваших собратьев\hyp{}пилигримов, --- должно будет сделать что\hyp{}то определенное с вами или для вас; но эти компаньоны призваны только быть с вами и общаться с вами как друзья личностей. В их служении им часто помогают милостивые и блестящие Райские Граждане.
\vs p025 8:4 Смертные происходят от очень общительных рас. Творцы хорошо знают, что «плохо человеку быть одному», и, соответственно, созданы условия для общения, даже в Раю.
\vs p025 8:5 \pc Если ты как восходящий смертный достигнешь Рая в сопровождении компаньона или близкого сподвижника твоего земного пути, или если твоему серафиму\hyp{}хранительнице предназначения случится прибыть с тобой или ожидать тебя там, то никакого постоянного компаньона тебе не будет назначено. Но если ты прибудешь один, компаньон непременно будет приветствовать тебя, когда ты проснешься на Острове Света от последнего сна времени. Даже если известно, что тебя будет сопровождать кто\hyp{}нибудь из группы восходящих, будут назначены временные компаньоны, чтобы приветствовать тебя на вечных берегах и сопроводить к месту, приготовленному для того, чтобы принять тебя и твоих сподвижников. Ты можешь быть уверен, что будешь тепло встречен, когда испытаешь воскрешение в вечности на вечных берегах Рая.
\vs p025 8:6 В последние дни пребывания восходящего на последнем контуре Хавоны назначаются принимающие компаньоны, и они внимательно изучают записи о происхождении смертного и его насыщенном событиями восхождении через миры пространства и контуры Хавоны. Когда они приветствуют смертных, живших во времени, они уже хорошо осведомлены о путях этих прибывших пилигримов и немедленно оказываются участливыми и интересными компаньонами.
\vs p025 8:7 В течение твоей дофиналитной жизни в Раю, если ты по какой\hyp{}либо причине отделен от твоего сподвижника в восходящем пути --- смертного или серафима, --- к тебе немедленно будет назначен Райский Компаньон для совета и общения. Компаньон, однажды назначенный к восходящему смертному, пребывающему в Раю в одиночестве, остается с ним до тех пор, пока тот либо не воссоединится со своими восходящими сподвижниками, либо не будет должным образом зачислен в Отряд Финалитов.
\vs p025 8:8 \pc Райские Компаньоны назначаются в порядке очереди, но никогда восходящий не поручается заботе компаньона, природа которого не соответствует сверхвселенскому типу смертного, даже если наступает очередь такого компаньона. Если урантийский смертный прибывает сегодня в Рай, к нему будет назначен первый в очереди компаньон, который или происходит из Орвонтона или иным образом обладает природой Седьмого Духа\hyp{}Мастера. Поэтому омниафимы никогда не служат у восходящих созданий из семи вселенных.
\vs p025 8:9 \pc Райские Компаньоны оказывают много дополнительных услуг: если восходящий смертный достигает центральной вселенной в одиночку и, пересекая Хавону, терпит неудачу в какой\hyp{}либо фазе Божественного искания, он должным образом отсылается обратно во вселенные времени и немедленно делается запрос в резервы Райских Компаньонов. Один из членов этого чина назначается следовать за претерпевшим крушение пилигримом, быть с ним, утешать и ободрять его и остается с ним до тех пор, пока тот не вернется в центральную вселенную, чтобы вновь начать Райское восхождение.
\vs p025 8:10 Если восходящий пилигрим претерпевает неудачу в Божественном восхождении в то время, как он пересекает Хавону вместе с восходящим серафимом, ангелом\hyp{}хранительницей смертного пути, решение сопровождать своего смертного сподвижника принимает именно она. Эти серафимы всегда делают это добровольно, и им разрешено сопровождать своих старинных смертных товарищей обратно на службу пространства и времени.
\vs p025 8:11 Но иначе дело обстоит с двумя тесно связанными восходящими смертными: если один достигает Бога в то время, как другой временно терпит неудачу, и первый непременно хочет вернуться на эволюционирующие творения вместе с личностью, испытавшей разочарование, то это не разрешается. В таком случае делается запрос в резервы Райских Компаньонов и выбирается один из добровольцев, чтобы сопровождать разочарованного пилигрима. Затем доброволец\hyp{}Райский Гражданин становится связанным с добившимся успеха смертным, который, пребывая на центральном Острове, ожидает возвращения в Хавону потерпевшего поражение собрата и преподает тем временем в некоторых Райских школах, где рассказывает увлекательную историю своего эволюционного восхождения.
\vsetoff
\vs p025 8:12 [Под покровительством Облеченного Высокой Властью из Уверсы.]
