\upaper{195}{После Пятидесятницы}
\vs p195 0:1 Последствия проповеди Петра в день Пятидесятницы были таковы, что полностью определили всю будущую политику и планы большинства апостолов в деле возвещения евангелия царства. Петр был подлинным основателем христианской церкви; Павел нес христианскую весть неевреям, а верующие греки распространяли ее по всей Римской империи.
\vs p195 0:2 Хотя скованные традициями и находящиеся во власти священников евреи, как народ, отказались принять и Иисусово евангелие отцовства Бога и братства людей, и весть Петра и Павла о воскресении и вознесении Христа (позже --- христианство), остальная Римская империя восприняла развивающееся христианское учение. Западной цивилизации в то время были присущи интеллектуальность, усталость от войн и скептицизм по отношению ко всем существующим религиям и мировым философиям. Народы западного мира, воспринявшие греческую культуру, несли в себе почитаемые традиции великого прошлого. Их окружало великое наследие --- развитая философия, искусство, литература и успехи в политике. Но при всех этих достижениях, у них не было питающей душу религии. Их духовная жажда оставалась неудовлетворенной.
\vs p195 0:3 И тут человеческому обществу вдруг открылись учения Иисуса, содержащиеся в христианском послании. Так изголодавшимся сердцам западных народов были предложены новые устои жизни. Эта ситуация вела к немедленному конфликту между прежними религиозными верованиями и новой христианизированной версией послания Иисуса миру. Такой конфликт должен был окончиться решительной победой или нового, или старого, или же, в той или иной степени, \bibemph{компромиссом.} История показывает, что борьба закончилась компромиссом. Христианство несло в себе так много, что не было народа, способного усвоить все это за одно или два поколения. Это был не просто духовный призыв, подобный тому, что Иисус обращал к душам людей; христианство с первых шагов заняло твердую позицию в вопросе религиозных ритуалов, образования, магии, медицины, искусства, литературы, права, форм правления, морали, сексуальных норм, многоженства и, в какой\hyp{}то степени, даже рабовладения. Христианство появилось не просто как новая религия --- которую ждали вся Римская империя и весь Восток, --- но как \bibemph{новое устройство человеческого общества.} И претендуя на такую роль, оно быстро вызвало общественно\hyp{}моральное столкновение эпох. Идеалы Иисуса, по новому истолкованные греческой философией ии введенные в христианскую общину, бросали теперь смелый вызов традициям человечества, воплощенным в этике, морали и религиях западной цивилизации.
\vs p195 0:4 \P\ На первых порах в христианство обращались только представители низших социальных и имущественных слоев. Но уже к началу второго века лучшие представители греко\hyp{}римской культуры постепенно стали все больше тяготеть к этим новым устоям христианской веры, к этому новому понятию цели жизни и смысла существования.
\vs p195 0:5 Почему же это новое учение еврейского происхождения, потерпевшее почти полное фиаско на родной земле, так стремительно и успешно овладело самыми лучшими умами Римской империи? Победа христианства над философскими религиями и мистериальными культами объяснялась:
\vs p195 0:6 \ublistelem{1.}\bibnobreakspace Организацией. Павел был выдающимся организатором, не отставали от него и его преемники.
\vs p195 0:7 \P\ \ublistelem{2.}\bibnobreakspace Христианство было сильно эллинизировано. Оно вобрало в себя лучшее из греческой философии так же, как и лучшее из еврейской теологии.
\vs p195 0:8 \P\ \ublistelem{3.}\bibnobreakspace Но самое главное, оно содержало новый и великий идеал, отзвук пришествия и жизни Иисуса и отражение его вести о спасении для всего человечества.
\vs p195 0:9 \P\ \ublistelem{4.}\bibnobreakspace Христианские руководители готовы были пойти на такие компромиссы с митраизмом, что добрая половина его приверженцев приняли культ Антиохии.
\vs p195 0:10 \P\ \ublistelem{5.}\bibnobreakspace Подобным же образом, следующее и последующие поколения христианских руководителей пошли в дальнейшем на такие компромиссы с язычеством, что даже римский император Константин обратился в новую религию.
\vs p195 0:11 \P\ Но христиане заключили практичную сделку с язычниками --- они приняли языческую пышность ритуалов, но взамен побудили язычников принять эллинизированую версию Паулинистического христианства. Их сделка с язычниками была много лучше сделки с культом Митры, но даже и в том предыдущем компромиссе они оказались больше чем победителями, поскольку им удалось уничтожить величайшую безнравственность и другие многочисленные предосудительные моменты, связанные с практикой персидских религиозных таинств.
\vs p195 0:12 Мудро это или нет, но ранние руководители христианства намеренно поступались \bibemph{идеалами} Иисуса с целью сохранить и распространить многие из его \bibemph{идей.} И это им великолепно удалось. Но не поймите неверно! Эти идеалы Учителя, которыми приходилось поступаться, в неявном виде по\hyp{}прежнему присутствуют в его евангелии и со временем в полной мере овладеют миром.
\vs p195 0:13 В результате сближения христианства с язычеством старая система одержала много незначительных побед в вопросах ритуала, но христиане одержали верх в том отношении, что:
\vs p195 0:14 \ublistelem{1.}\bibnobreakspace Человеческой нравственности стала отводиться новая и гораздо более высокая роль.
\vs p195 0:15 \P\ \ublistelem{2.}\bibnobreakspace Миру было дано новое и существенно расширенное представление о Боге.
\vs p195 0:16 \P\ \ublistelem{3.}\bibnobreakspace Надежда на бессмертие стала частью того, что гарантировала признанная религия.
\vs p195 0:17 \P\ \ublistelem{4.}\bibnobreakspace Изголодавшаяся душа человека получила Иисуса из Назарета.
\vs p195 0:18 \P\ Многие из великих истин, которым учил Иисус, почти растворились в этих ранних компромиссах, но они все\hyp{}таки дремлют в подвергшейся влиянию язычества христианской религии, которая, в свою очередь, была Паулинистической версией жизни и учений Сына Человеческого. Но еще до того, как христианство подверглось влиянию язычества, оно было сильно эллинизировано. Христианство многим, очень многим обязано грекам. Именно грек из Египта так смело выступил в Никее и так бесстрашно противостоял собору, что собор не осмелился затемнить представление о природе Иисуса настолько, что подлинная правда о его пришествии подверглась бы опасности быть утраченной для мира. Этого грека звали Афанасий, и если бы не красноречие и логика этого верующего, то восторжествовали бы сторонники Ария.
\usection{1.\bibnobreakspace Влияние греков}
\vs p195 1:1 По\hyp{}настоящему эллинизация христианства началась в тот знаменательный день, когда апостол Павел предстал перед советом ареопага в Афинах и рассказал афинянам о <<Неизвестном Боге>>. Там, под сенью Акрополя, этот римский гражданин возвестил грекам свою версию новой религии, возникшей на еврейской земле Галилеи. И странным образом было нечто сходное в греческой философии и во многих учениях Иисуса. И философия, и учения Иисуса стремились к \bibemph{проявлению человека как индивидуума.} Греки --- в плане социальном и политическом; Иисус --- в плане нравственном и духовном. Греки учили интеллектуальному либерализму, ведущему к политической свободе; Иисус учил духовному либерализму, ведущему к религиозной свободе. Эти две идеи, сложенные вместе, являли собой новую и мощную программу освобождения человека; они предвещали социальную, политическую и духовную свободу человека.
\vs p195 1:2 Христианство возникло и восторжествовало над всеми противостоящими религиями, прежде всего, в силу двух причин:
\vs p195 1:3 \ublistelem{1.}\bibnobreakspace Сознание греков готово было воспринимать хорошие новые идеи даже от евреев.
\vs p195 1:4 \ublistelem{2.}\bibnobreakspace Павел и его преемники были готовы идти на компромиссы но при этом были практичны и дальновидны; они были искусными негоциантами теологии.
\vs p195 1:5 \P\ Во времена, когда Павел в Афинах убедительно проповедовал на тему <<Распятые Христос и Он>>, греки испытывали духовный голод; они вопрошали о духовной истине, она интересовала их, и они действительно ее искали. Не забывайте, что поначалу римляне боролись против христианства, тогда как греки приняли его, и именно греки впоследствии буквально вынудили римлян принять эту новую религию в модифицированном к тому времени виде в качестве части греческой культуры.
\vs p195 1:6 Греки чтили красоту, евреи --- святость, но оба эти народа любили истину. Веками греки серьезно обдумывали и горячо обсуждали все человеческие проблемы --- общественные, экономические, политические и философские --- все кроме религии. Мало кто из греков обращал много внимания на религию; даже к своей собственной религии они относились не слишком серьезно. Евреев же веками мало заботили все эти разные интеллектуальные сферы, они обращали свои помыслы лишь к религии. Они относились к своей религии серьезно, слишком серьезно. Совместный плод многовековой мысли этих двух народов, озаренный содержанием вести Иисуса, стал теперь движущей силой нового устройства человеческого общества и, в известной степени, новой системы верований и религиозной практики людей.
\vs p195 1:7 \P\ Когда Александр распространял эллинистическую цивилизацию по ближневосточному миру, страны западного средиземноморья уже находились под влиянием греческой культуры. Греков прекрасно устраивала их религия и их политика, пока они жили в небольших городах\hyp{}государствах, но когда македонский царь решился расширить пределы Греции до размеров империи, простирающейся от Адриатики до Инда, начались сложности. Искусство и философия Греции находились на достаточно высоком уровне для имперской экспансии, но этого нельзя было сказать о греческом государственном управлении и о религии. После того, как города\hyp{}государства Греции превратились в империю, их, в сущности, местные боги стали казаться несколько странными. Греки действительно искали \bibemph{одного Бога,} Бога более великого и хорошего, когда к ним пришла христианская версия более древней еврейской религии.
\vs p195 1:8 Эллинистическая империя как такова не смогла бы просуществовать долго. Ее культурное влияние продолжалось, но сохраниться она смогла, только когда получила с запада римский политический талант управления империей, а с востока --- религию, в которой Бог обладал имперским величием.
\vs p195 1:9 В первом веке после рождества Христова эллинистическая культура достигла наивысшего уровня; начался ее упадок; ученость развивалась, но творческий гений переживал спад. Именно в это время идеи и идеалы Иисуса, частично воплотившиеся в христианстве, оказались частью спасения греческой культуры и учености.
\vs p195 1:10 Александр покорял Восток культурным даром греческой цивилизации; Павел завоевывал Запад христианской версией евангелия Иисуса. И повсюду, где на Западе преобладала греческая культура, устанавливалось и эллинизированное христианство.
\vs p195 1:11 \P\ Восточная версия вести Иисуса, хотя она и точнее соответствовала его учениям, продолжала придерживаться бескомпромиссной позиции Авенира. Она никогда не добивалась таких же успехов, как эллинизированная версия, и впоследствии исчезла под натиском исламского движения.
\usection{2.\bibnobreakspace Римское влияние}
\vs p195 2:1 Римляне полностью восприняли греческую культуру, заменив правление по жребию на представительное правление. И теперь это изменение оказалось благоприятным для христианства, поскольку Рим принес в весь Западный мир новый элемент терпимости к чужим языкам, народам и даже религиям.
\vs p195 2:2 Первоначальные преследования христиан в Риме в значительной степени были вызваны исключительно неудачным использованием в их проповеди термина <<царство>>. Римляне терпимо относились к самым разным религиям, но очень болезненно воспринимали все, что наводило на мысль о политическом соперничестве. Так что после того, как эти ранние преследования, в очень большой степени вызванные недоразумением, прекратились, открылся широкий простор для религиозной пропаганды. Римлянина интересовало управление государством; его мало заботили и искусство, и религия, но и к тому, и к другому он относился с необыкновенной терпимостью.
\vs p195 2:3 Восточное право было суровым и деспотичным; греческое право было гибким и артистичным; римское право было величественным и вызывающим почтение. Римское образование воспитывало беспримерную невозмутимую благонадежность. Древние римляне были личностями политически верными и преданными. Они были честны, ревностны и преданны своим идеалам, но у них не было религии, достойной таковой называться. Ничего удивительного, что греческие учителя смогли убедить их принять христианство Павла.
\vs p195 2:4 И сами римляне были великим народом. Они смогли управлять Западом потому, что хорошо управляли собой. Такие, как у них, несравненные честность, преданность и стойкое самообладание были идеальной почвой для принятия и развития христианства.
\vs p195 2:5 Этим греко\hyp{}римлянам было легко стать преданными официальной церкви точно так же, как они были политически преданы государству. Римляне боролись с церковью только тогда, когда видели в ней соперника государству. Рим, у которого были мало развиты национальная философия и собственная культура, воспринял греческую культуру как свою собственную и смело принял Христа в качестве своей нравственной философии. Христианство стало нравственной культурой Рима, но едва ли стало его религией --- в смысле индивидуального опыта духовного развития тех, кто принимали новую религию в массовом порядке. Правда, многие конкретные люди не довольствовались тем, что лежало на поверхности всей этой государственной религии, а находили питающие душу подлинные ценности скрытых значений, содержащихся в туманных истинах эллинизированого и подвергшегося влиянию язычества христианства.
\vs p195 2:6 \P\ Стоики с их твердым призывом к <<природе и совести>> еще больше подготовили весь Рим к принятию Христа, по крайней мере, в интеллектуальном смысле. Римлянин по своей сути и по образованию был законоведом; он чтил даже законы природы. И теперь через христианство он распознал в законах природы законы Бога. Народ, который смог породить Цицерона и Вергилия, созрел для эллинизированого христианства Павла.
\vs p195 2:7 Итак, эти романизированные греки вынудили и евреев, и христиан подвести философскую базу под свою религию, согласовать ее идеи и систематизировать ее идеалы, приспособить религиозные обычаи к существующему течению жизни. И всему этому чрезвычайно способствовал перевод еврейского Писания на греческий язык и последующая запись Нового Завета по\hyp{}гречески.
\vs p195 2:8 Греки, в противоположность евреям и многим другим народам, уже издавна более или менее верили в бессмертие, в некое продолжение жизни после смерти, а поскольку это положение лежало в основе учения Иисуса, христианство, несомненно, должно было обладать для них большой притягательностью.
\vs p195 2:9 Череда успехов греческой культуры и римской политики сплотил земли Средиземноморья, превратив их в единую империю с единым языком и единой культурой, и подготовил Западный мир к восприятию концепции единого Бога. Такого Бога мог дать иудаизм, но иудаизм как религия был неприемлем для этих романизированных греков. Филон помог некоторым смягчить их неодобрение, но лишь христианство открыло им лучшее понятие о едином Боге, и они с готовностью приняли его.
\usection{3.\bibnobreakspace Под властью Римской империи}
\vs p195 3:1 После укрепления римского политического владычества и распространения христианства у христиан был единый Бог, великая религиозная концепция, но не было империи. У греко\hyp{}римлян была великая империя, но не было Бога, понятие о котором послужило бы основой для общеимперского религиозного культа и духовного единения. Христиане приняли империю; империя приняла христианство. Римляне дали единство политической власти; греки --- единство культуры и учености; христианство --- единство религиозной мысли и культа.
\vs p195 3:2 Рим преодолел традицию национализма посредством имперского универсализма и впервые в истории создал возможность того, чтобы разные народы и национальности, по крайней мере номинально, приняли единую религию.
\vs p195 3:3 Христианство получило признание в Риме во время, когда шла острая борьба между энергичным учением стоиков и обещающими спасение мистическими культами. Христианство, несущее целительное утешение и возможность освобождения, пришло к духовно изголодавшимся людям, в языке которых отсутствовало слово <<бескорыстие>>.
\vs p195 3:4 \P\ Христианству придавало величайшую силу то, как верующие жили --- жизнью, исполненной служения, --- и даже то, как они умирали за свою веру в начальный период жестоких преследований.
\vs p195 3:5 \P\ Учение Христа о любви к детям вскоре положило конец распространенному обычаю оставлять умирать нежеланных грудных детей, особенно девочек.
\vs p195 3:6 \P\ Первоначальный порядок христианского богослужения, в значительной степени, был заимстван из еврейской синагоги и видоизменен по аналогии с митраистским ритуалом; позже туда было внесено много языческой пышности. Исповедовавшие иудаизм греки, принявшие христианство, и составили костяк ранней христианской церкви.
\vs p195 3:7 \P\ Второй век после рождества Христова был самым лучшим периодом во всей мировой истории для того, чтобы достойная религия могла добиться успеха в западном мире. В течение первого века христианство, где борьбой, где компромиссами подготовилось к тому, чтобы пустить корни и стремительно распространиться. Христианство приняло императора; позже он принял христианство. Это была великолепная эпоха для распространения новой религии. Везде --- религиозная свобода; все и повсюду путешествовали, и мысль не встречала преград.
\vs p195 3:8 Духовное влияние номинально принимаемого христианства пришло в Рим слишком поздно, чтобы предотвратить давно уже начавшееся падение нравов или воспрепятствовать безусловно имевшему место и усиливавшемуся генетическому вырождению. Эта новая религия была культурной необходимостью для имперского Рима, и крайне прискорбно, что она не стала средством духовного спасения в более широком смысле.
\vs p195 3:9 Даже замечательная религия не могла спасти огромную империю от неизбежных последствий отсутствия личного участия индивидуума в делах управления, от чрезмерного патернализма, непомерных налогов и грубых злоупотреблений при их сборе, несбалансированной торговли с Левантом, истощавшей золотой запас, безумной страсти к развлечениям, римской стандартизации, деградации женщины, рабства и ухудшения расы, эпидемий и бедствий и от государственной церкви, настолько казенной и официальной, что уже практически духовно бесплодной.
\vs p195 3:10 Впрочем, в Александрии ситуация была не такой уж плохой. Ранние школы по\hyp{}прежнему сохраняли многое из учений Иисуса, не идя на компромиссы. Пантен учил Климента, а затем вслед за Нафанаилом отправился возвещать о Христе в Индию. Хотя при создании христианства и пришлось пожертвовать некоторыми идеалами Иисуса, следует со всей справедливостью отметить, что к концу второго века практически все выдающиеся умы греко\hyp{}римского мира стали христианами. Близилась окончательная победа.
\vs p195 3:11 И эта Римская империя просуществовала достаточно долго, чтобы обеспечить христианству выживание даже после крушения империи. Но мы часто задаем себе вопрос, что происходило бы в Риме и в мире, если бы вместо греческого христианства приняли евангелие царства.
\usection{4.\bibnobreakspace Европейское средневековье}
\vs p195 4:1 Церкви, тесно связанной с обществом и политикой, суждено было вместе с ними испытать интеллектуальный и духовный упадок так называемых европейских <<темных веков>>. В течение этого времени религия становилась все более и более клерикальной, аскетической и казенной. В духовном смысле христианство пребывало в спячке. На протяжении всего этого периода параллельно с этой дремлющей и секуляризованной религией постоянно существовало мистическое течение, фантастический духовный опыт на грани ирреального, философски родственного пантеизму.
\vs p195 4:2 В эти темные и безысходные столетия религия снова стала доступной фактически лишь из вторых рук. Отдельного человека практически не было видно на фоне подавляющего авторитета, традиции и диктата церкви. Возникла новая духовная опасность, связанная с формированием такого явления, как лики <<святых>>, которые, как считалось, имели особое влияние в божественных судах и которые поэтому, если их хорошенько попросить, могли ходатайствовать за человека перед Богами.
\vs p195 4:3 Но хотя христианство было бессильно остановить наступление средневековья, оно уже достаточно подвергалось огосударствлению и влиянию язычества, так что хорошо было подготовлено к тому, чтобы пережить этот длительный период нравственной темноты и духовного застоя. И оно продолжало существовать на протяжении всей долгой ночи западной цивилизации и по\hyp{}прежнему оказывало моральное воздействие на мир, когда забрезжила эпоха возрождения. Реабилитация христианства, последовавшая за окончанием средневековья, привела к появлению многочисленных сект христианских учений, верований, соответствующих специфическим интеллектуальным, эмоциональным и духовным типам человеческой личности. И многие из этих особых общин христиан, или религиозных братств, по\hyp{}прежнему существуют к моменту этого повествования.
\vs p195 4:4 \P\ История христианства начинается с его возникновения в результате непреднамеренного превращения религии Иисуса в религию об Иисусе. Далее следует история его эллинизации, влияния язычества, секуляризации, огосударствления, интеллектуального вырождения, духовного упадка, нравственной спячки, угрозы исчезновения, затем --- восстановления, дробления и, ближе к нашему времени, относительной реабилитации. Такая история свидетельствует о том, что оно жизнеспособно и обладает колоссальными ресурсами для восстановления. И это же самое христианство наличествует сейчас в цивилизованном мире западных народов и ведет там значительно более опасную борьбу за существование, чем в те судьбоносные кризисы, которые были характерны для его прошлой борьбы за господство.
\vs p195 4:5 \P\ Религии теперь бросает вызов новая эпоха научного сознания и материалистических тенденций. В этой грандиозной борьбе между секулярным и духовным религия Иисуса, в конечном счете, восторжествует.
\usection{5.\bibnobreakspace Современная проблема}
\vs p195 5:1 Двадцатый век поставил новые проблемы перед христианством и всеми прочими религиями. Чем выше поднимается цивилизация, тем насущнее становится задача <<искать прежде всего небесные реальности>> при любом стремлении человека стабилизировать общество и способствовать решению его материальных проблем.
\vs p195 5:2 Когда истина расчленена на части, отделена, изолирована и подвергается излишнему анализу, она часто начинает сбивать с толку и даже вводить в заблуждение. Живая истина правильно учит стремящихся к ней только тогда, когда она принимается в целостности и как живая духовная реалия, а не как факт материальной науки или озарение через искусство, служащее мостом между материальным и духовным.
\vs p195 5:3 Религия --- это раскрытие человеку его божественного и вечного предназначения. Религия --- это чисто личный и духовный опыт, и ее всегда необходимо отличать от других высших форм человеческого мышления, таких как:
\vs p195 5:4 \ublistelem{1.}\bibnobreakspace Логическое отношение человека к объектам материальной реальности.
\vs p195 5:5 \ublistelem{2.}\bibnobreakspace Эстетическое восприятие человеком прекрасного как антипода уродливого.
\vs p195 5:6 \ublistelem{3.}\bibnobreakspace Этическое признание человеком общественных обязательств и политического долга.
\vs p195 5:7 \ublistelem{4.}\bibnobreakspace Даже человеческая нравственность сама по себе не есть религия.
\vs p195 5:8 \P\ Религия предназначена для того, чтобы находить во вселенной те ценности, которые порождают веру, доверие и убежденность; высшее проявление религии --- богопочитание. Религия открывает душе верховные ценности --- в противоположность относительным ценностям, открываемым посредством разума. Такое сверхчеловеческое озарение можно обрести только через подлинный религиозный опыт.
\vs p195 5:9 Прочная общественная система без нравственности, основанной на духовных реалиях, возможна не более, чем солнечная система без силы гравитации.
\vs p195 5:10 Не пытайтесь за одну короткую жизнь во плоти удовлетворить любопытство или полностью утолить обуревающую душу жажду свершений. Будьте терпеливы! не поддавайтесь искушению очертя голову броситься в недостойные и низменные предприятия. Обуздайте свою энергию и сдерживайте свои страсти; храните спокойствие в ожидании того, что перед вами величественно откроет бесконечный путь, полный свершений и захватывающих открытий.
\vs p195 5:11 \P\ Сбитые с толку относительно происхождения человека, не предавайте забвению его вечное предназначение. Не забывайте, что Иисус любил даже маленьких детей и что он всегда подчеркивал огромную ценность человеческой личности.
\vs p195 5:12 \P\ Глядя на мир, помните, что черные пятна зла заметны вам на преобладающем белом фоне добра. Не правда, что вы просто видите белые пятнышки добра, жалко проглядывающие на черном фоне зла.
\vs p195 5:13 Когда можно оглашать и возвещать столько благой истины, зачем же людям так сильно сосредоточиваться на зле лишь потому, что оно есть? Достоинства духовных ценностей истины более отрадны и возвышающи, чем феномен зла.
\vs p195 5:14 \P\ В религии Иисус предлагал следовать путем опыта, подобно тому, как современная наука следует технике эксперимента. Мы находим Бога через водительство духовного прозрения, но мы подходим к этому прозрению души через любовь к прекрасному, стремление к истине, верность долгу и почитание божественной добродетели. Но из всех этих ценностей любовь действительно ведет к подлинному прозрению.
\usection{6.\bibnobreakspace Материализм}
\vs p195 6:1 Ученые непреднамеренно ввергли человечество в материалистическое безрассудство; они вызвали бездумное ажиотажное изъятие вкладов из нравственного банка веков, но этот банк человеческого опыта обладает обширными духовными ресурсами; он может удовлетворить все предъявляемые ему требования. Только неразумные люди впадают в панику по поводу духовных запасов человечества. Когда светско\hyp{}материалистическоебезрассудство закончится, религия Иисуса не окажется несостоятельной. Духовный банк царства небесного будет выплачивать веру, надежду и моральную защищенность всем, кто обращается за ними <<от Его имени>>.
\vs p195 6:2 Какие бы видимые конфликты ни существовали между материализмом и учением Иисуса, вы можете быть уверены, что в грядущие века учения Учителя полностью восторжествуют. В действительности, истинная религия не может вступать ни в какие споры с наукой; она не имеет никакого отношения к материальным явлениям. Религия относится к науке просто с безразличием, хотя и с симпатией, но ей совершенно не безразличен \bibemph{ученый.}
\vs p195 6:3 Поиск только лишь знания без сопутствующей интерпретации через мудрость и духовное озарение религиозного опыта ведет, в конечном счете, к пессимизму и человеческому отчаянию. Малое знание действительно приводит в замешательство.
\vs p195 6:4 Ко времени написания этого текста худшая эпоха материализма закончилась; уже восходит заря более правильного понимания. Высочайшие умы научного мира уже не придерживаются чисто материалистической философии, но рядовые люди все еще тяготеют к этому из\hyp{}за прежних учений. Но эта эпоха физического реализма --- лишь преходящий эпизод в человеческой жизни на земле. Современная наука никак не затронула истинную религию --- учений Иисуса, воплотившихся в жизнях верующих. Единственное, что сделала наука, --- разрушила наивные иллюзии, связанные с неправильным пониманием жизни.
\vs p195 6:5 Наука --- это количественный рост опыта, религия --- качественный рост опыта в жизни человека на земле. Наука имеет дело с явлениями; религия --- с истоками, ценностями и целями. Определить \bibemph{причины,} как объяснение физических явлений, значит признать непонимание предельных причин, что, в конечном счете, лишь приводит ученого прямо к великой первопричине --- Вселенскому Райскому Отцу.
\vs p195 6:6 Резкий переход от эры чудес к эре машин сбил человека с толку. Разумность и стройность ложных механистических философий противоречат их механистическим утверждениям. Фаталистическая живость ума материалиста вечно опровергает его утверждения, что вселенная представляет собой слепое и бесцельное энергетическое явление.
\vs p195 6:7 Как механистический натурализм некоторых казалось бы образованных людей, так и бездумный секуляризм человека с улицы интересуются исключительно \bibemph{вещами;} они лишены всех подлинных ценностей, опор и духовной удовлетворенности, а также веры, надежды и вечной уверенности. Одна из величайших проблем современной жизни заключается в том, что человек считает себя слишком занятым, чтобы находить время для духовного размышления и религиозного служения.
\vs p195 6:8 Материализм превращает человека в бездушный автомат и делает его просто арифметическим символом, беспомощно занимающим место в математической формуле лишенной романтики и механистической вселенной. Но откуда же взялась вся эта обширная математическая вселенная без Главного Математика? Наука может говорить о законе сохранения материи, религия же обосновывает закон сохранения человеческих душ --- ее интересует их опыт, связанный с духовными реальностями и непреходящими ценностями.
\vs p195 6:9 Современный социолог\hyp{}материалист изучает некое сообщество людей, дает его описание и оставляет людей такими, какими они были. Девятнадцать веков назад необразованные галилеяне своими глазами видели, как Иисус дал свою жизнь как духовный вклад во внутренний опыт человека, а потом пошли и перевернули вверх дном всю Римскую империю.
\vs p195 6:10 Но религиозные лидеры совершают огромную ошибку, пытаясь призывать современного человека на духовную битву средневековыми трубными звуками. Религия должна обеспечить себя новыми, современными лозунгами. Ни демократия, ни какая\hyp{}либо другая политическая панацея не заменят духовного прогресса. Ложные религии могут бежать от реальности, но Иисус в своем евангелии подвел смертного человека прямо ко входу в вечную реальность духовного прогресса.
\vs p195 6:11 Утверждение, что разум <<возник>> из материи, ничего не объясняет. Если бы вселенная была просто механизмом и разум был бы неотделим от материи, никогда не могло бы быть двух различных истолкований какого\hyp{}либо наблюдаемого явления. Понятия истины, красоты и добродетели не присущи ни физике, ни химии. Машина не может знать, а тем более, знать истину, жаждать праведности и лелеять добродетель.
\vs p195 6:12 Наука может быть физической, но разум способного распознавать истину ученого в то же время сверхматериален. Материя не знает истины, она не может любить милосердие и наслаждаться духовными реальностями. Нравственные убеждения, основанные на духовном прозрении и коренящиеся в человеческом опыте, так же реальны и несомненны, как математические выводы, основанные на физических наблюдениях, но на другом и более высоком уровне.
\vs p195 6:13 Если бы люди были лишь машинами, они реагировали бы на материальный мир более или менее единообразно. Не существовало бы индивидуальности, а тем более личности.
\vs p195 6:14 \P\ Существование абсолютного механизма Рая в центре вселенной вселенных при наличии неограниченной воли Второго Источника и Центра делает несомненным фактом то, что детерминизм не является единственным законом космоса. Существует материализм, но не исключительно он, существует механицизм, но он ограничен; существует детерминизм, но он не единственный фактор.
\vs p195 6:15 Конечная материальная вселенная стала бы, в конечном счете, единообразной и детерминистичной, если бы не объединенное присутствие разума и духа. Влияние космического разума постоянно привносит спонтанность даже в материальные миры.
\vs p195 6:16 Свобода и инициатива в любой сфере существования прямо пропорциональны степени духовного влияния и контроля космического разума; то есть --- в человеческом опыте --- степени действенности исполнения <<воли Отца>>. Так что, когда вы однажды начинаете искать Бога, это убедительно доказывает, что Бог уже нашел вас.
\vs p195 6:17 Искреннее стремление к добродетели, красоте и истине ведет к Богу. И каждое научное открытие демонстрирует существование во вселенной и свободы, и единообразия. Сделавший открытие обладал свободой его сделать. Та вещь, которую открыли, реальна и, очевидно, единообразна, иначе она не могла бы быть отождествлена как некая вещь.
\usection{7.\bibnobreakspace Уязвимость материализма}
\vs p195 7:1 Как глупо со стороны материалистически мыслящего человека позволять таким уязвимым теориям, как теории о механистичности вселенной, лишать его обширных духовных ресурсов личного опыта истинной религии. Факты никогда не вступают в противоречие с подлинной духовной верой; теории --- могут. Науке лучше следовало бы заниматься разрушением суеверий, чем пытаться ниспровергнуть религиозную веру --- человеческую веру в духовные реальности и божественные ценности.
\vs p195 7:2 Науке следует делать для человека в материальном плане то же самое, что религия делает для него в плане духовном: расширять жизненные горизонты и обогащать его личность. Истинная наука не может иметь никаких длительных противоречий с истинной религией. <<Научный метод>> --- это просто интеллектуальное мерило для измерения материальных событий и физических достижений. Но, будучи материальным и полностью интеллектуальным, он совершенно бесполезен при оценке духовных реальностей и религиозного опыта.
\vs p195 7:3 Непоследовательность современного механициста в следующем: если бы эта вселенная была чисто материальной, а человек был лишь машиной, то такой человек был бы совершенно неспособен осознать себя в качестве таковой машины, и, аналогично, такой человек\hyp{}машина совершенно не осознавал бы факта существования такой материальной вселенной. Материалистическое уныние и безысходность механистической науки не позволили осознать факт существования духа в разуме ученого, сверхчеловеческие озарения которого и позволяют формулировать эти ошибочные и внутренне противоречивые \bibemph{концепции} материалистической вселенной.
\vs p195 7:4 Райские ценности вечности и бесконечности, истины, красоты и добродетели скрыты внутри наблюдаемых явлений пространственно\hyp{}временной вселенной. Но чтобы обнаружить и распознать эти духовные ценности, требуется глаз веры, которой обладает духовно родившийся смертный.
\vs p195 7:5 Реальности и ценности духовного прогресса --- это не <<психологическая проекция>>, не просто величественная мечта материального разума. Такие вещи есть духовные предвещания) внутреннего Настройщика --- духа Бога, живущего в разуме человека. И не допускайте, чтобы соприкосновения с мельком увиденными выводами <<относительности>> поколебали бы ваши представления о вечности и бесконечности Бога. И во всех своих настойчивых стремлениях, связанных с необходимостью \bibemph{самовыражения,} не совершайте ошибку --- не забывайте обеспечивать \bibemph{самовыражение Настройщика,} проявление подлинной и лучшей части вашего Я.
\vs p195 7:6 Если бы эта вселенная была только материальной, материальный человек никогда не смог бы прийти к представлению о механистическом характере такого исключительно материального существования. Уже это \bibemph{механистическое представление} о вселенной само по себе есть нематериальный феномен разума, и весь разум имеет нематериальное происхождение, независимо от того, до какой степени он может казаться материально обусловленным и механистически контролируемым.
\vs p195 7:7 Частично развившийся психическое устройство смертного человека не страдает от переизбытка мудрости и логичности. Человеческая самонадеянность часто преобладает над разумом и не подчиняется логике.
\vs p195 7:8 Сам пессимизм самого пессимистического материалиста уже служит достаточным доказательством того, что мир пессимиста не является полностью материальным. И пессимизм, и оптимизм --- это понятийные реакции разума, осознающего наряду с \bibemph{фактами} также и \bibemph{ценности.} Если бы вселенная действительно была такой, какой представляют ее материалисты, то человек, будучи человеческой машиной, был бы тогда лишен всякой возможности осознать сам этот \bibemph{факт.} Без осознания рожденным в духе разумом понятия \bibemph{ценностей} человек совершенно не осознавал бы факт материальности вселенной и феномен механистичности функционирования вселенной. Одна машина не может осознавать природу или ценность другой машины.
\vs p195 7:9 Механистическая философия жизни и вселенной не может быть научной, потому что наука признает и рассматривает только материю и факты. Философия неизбежно бывает наднаучной. Человек --- это материальный факт природы, но его \bibemph{жизнь ---} это феномен, выходящий за рамки материальных уровней природы, поскольку в ней проявляются контролирующие свойства разума и созидательные качества духа.
\vs p195 7:10 Искреннее стремление человека стать механицистом представляет собой трагический феномен, равноценный тщетной попытке этого человека совершить интеллектуальное и моральное самоубийство. Но он не может этого сделать.
\vs p195 7:11 Если бы вселенная была только материальной, а человек был бы лишь машиной, не было бы и науки, дающей ученому смелость делать такие утверждения о механистичности вселенной. Машины не могут измерять, классифицировать или оценивать сами себя. Такая научная деятельность может осуществляться только существом, имеющим надмашинный статус.
\vs p195 7:12 Если реальность вселенной --- это лишь одна огромная машина, тогда, чтобы осознать такой \bibemph{факт} и обрести \bibemph{понимание} такой \bibemph{оценки,} человек должен находиться за пределами вселенной и вне ее.
\vs p195 7:13 \P\ Если человек --- это только машина, каким же образом этот человек начинает \bibemph{верить} и утверждать, будто бы \bibemph{знает,} что он только машина? Опыт самоосознанной оценки собственного <<Я>> никогда не был свойствен простой машине. Обладающий самосознанием признанный механицист --- это самый лучший ответ механицизму. Если бы материализм был истиной, не могло бы быть обладающего самосознанием механициста. Правда также и то, что человеку должна быть присуща нравственность прежде, чем он может совершить безнравственные поступки.
\vs p195 7:14 \P\ Само отстаивание материализма уже подразумевает надматериальный разум, который и позволяет защищать такие догмы. Механизм может портиться, но он никогда не может совершенствоваться. Машины не могут думать, творить, мечтать, стремиться, идеализировать, жаждать истины или праведности. Они не делают движущим мотивом своей жизни страстное желание служить другим машинам и не избирают в качестве цели вечного движения вперед величественную задачу поиска Бога и стремление быть подобным ему. Машины никогда не обладают интеллектом, эмоциями, эстетическим и этическим чувствами, нравственностью или духовностью.
\vs p195 7:15 Искусство доказывает, что человек не механистичен, но оно не доказывает, что он духовно бессмертен. Искусство --- это человеческая моронтия, область, лежащая между человеком материальным и человеком духовным. Поэзия --- это стремление бежать от материальных реальностей к духовным ценностям.
\vs p195 7:16 В высокоразвитой цивилизации искусство делает науку более человечной, а само, в свою очередь, обретает духовность от истинной религии --- от понимания духовных и вечных ценностей. Искусство представляет собой человеческую пространственно\hyp{}временную оценку реальности. Религия \bibemph{есть} божественное вместилище космических ценностей и подразумевает вечный прогресс в духовном восхождении. Искусство во все времена становится опасным, только когда оно закрывает глаза на духовные стандарты вечных божественных паттернов, которые отражаются вечностью, как тени реальности данного времени. Истинное искусство эффективно манипулирует материальными вещами в жизни; религия же --- облагораживающе преобразует материальные факты жизни и непрестанно осуществляет духовную оценку искусства.
\vs p195 7:17 \P\ Как глупо полагать, что автомат мог бы постичь философию автоматизма, и как нелепо, если бы он взялся создавать такую концепцию относительно других своих собратьев\hyp{}автоматов!
\vs p195 7:18 \P\ Любая научная интерпретация материальной вселенной не представляет никакой ценности, если она не дает должного признания \bibemph{ученому.} Не может быть подлинным признание искусства, не приносящее признание \bibemph{художнику.} Оценка морали ничего не стоит, если она не распространяется и на \bibemph{моралиста.} Признание философии мало поучительно, если оно не обращает внимание на \bibemph{философа,} а религия не может существовать без реального опыта \bibemph{верующего,} который в своем опыте и с помощью него стремится найти Бога и узнать его. Точно так же вселенная вселенных утрачивает смысл без <<Я ЕСТЬ>>, бесконечного Бога, сотворившего ее и непрестанно управляющего ею.
\vs p195 7:19 \P\ Механицисты --- гуманисты --- имеют склонность пассивно плыть по материальным течениям. Идеалисты и спиритисты \bibemph{осмеливаются} с умом и энергично воспользоваться веслами, чтобы изменить по видимости чисто материальное направление потоков энергии.
\vs p195 7:20 \P\ Наука живет логическим мышлением разума; музыка выражает ритм эмоций. Религия --- это духовный ритм души, находящейся в пространственно\hyp{}временной гармонии с тактами высшей и вечной мелодии Бесконечности. Религиозный опыт в человеческой жизни --- это нечто действительно надлогическое.
\vs p195 7:21 В языке алфавит представляет собой материальный механизм, тогда как слова, выражающие смысл тысячи мыслей, великих идей и благородных идеалов --- любви и ненависти, малодушия и мужества, --- отражают деятельность разума в рамках материального и духовного закона, направляемую волей личности и ограниченную врожденным уровнем дарований.
\vs p195 7:22 Вселенная не похожа на законы, механизмы и закономерности, которые ученый открывает и которые он называет наукой, она, скорее, похожа на любознательного, думающего, выбирающего, творческого, устанавливающего общности и различия \bibemph{ученого,} который таким образом наблюдает явления вселенной и классифицирует математические факты, присущие механистическим аспектам материальной стороны творения. Не похожа вселенная и на искусство художника, а, скорее, похожа на увлеченного, мечтающего, горящего и движущегося вперед \bibemph{художника,} который старается выйти за пределы мира материальных вещей в стремлении достичь духовной цели.
\vs p195 7:23 Ученый, а не наука постигает реальности развивающейся и прогрессирующей вселенной энергии и материи. Художник, а не искусство демонстрирует существование преходящего моронтийного мира, занимающего положение между материальным существованием и духовной свободой. Верующий, а не религия доказывает существование духовных реальностей и божественных ценностей, которые встречаются при движении вперед в вечности.
\usection{8.\bibnobreakspace Светский тоталитаризм}
\vs p195 8:1 Но даже после того, как материализм и механицизм более или менее побеждены, разрушительное влияние секуляризма двадцатого века по\hyp{}прежнему будет наносить вред духовному опыту миллионов доверчивых душ.
\vs p195 8:2 Современный секуляризм взлелеян двумя общемировыми факторами. Отцом секуляризма была недалекая безбожная позиция так называемой науки девятнадцатого и двадцатого века --- атеистической науки. Матерью современного секуляризма была тоталитарная средневековая христианская церковь. Секуляризм берет начало от нарастающего протеста против почти полного господства в западной цивилизации узаконенной христианской церкви.
\vs p195 8:3 В момент написания данного откровения преобладающий интеллектуальный и философский климат в жизни и Европы, и Америки, несомненно, светский --- гуманистический. На протяжении трехсот лет западная мысль становилась все более светской. Религия все больше становилась чисто номинальным фактором, и в значительной степени сводилась лишь к соблюдению ритуалов. Большинство считающих себя христианами в западной цивилизации фактически невольно являются секуляристами.
\vs p195 8:4 Потребовалась огромная сила, мощное воздействие, чтобы освободить мысль и жизнь западных народов от иссушающих тисков тоталитарного церковного господства. Секуляризм разорвал оковы церковного господства, а теперь он сам, в свою очередь, угрожает установить новый безбожный тип господства над сердцами и умами современного человека. Тираническая и диктаторская политическая система --- прямой результат научного материализма и философского секуляризма. Секуляризм, освободив человека от господства узаконенной церкви, тут же продает его в рабство тоталитарному государству. Секуляризм освобождает человека из церковного рабства только для того, чтобы обманом уготовить ему тиранию политического и экономического рабства.
\vs p195 8:5 \P\ Материализм отрицает Бога, секуляризм просто игнорирует его; по крайней мере, такова была его более ранняя позиция. Позже секуляризм занял более воинственную позицию, претендуя на то место, которое занимала религия, с тоталитарным господством которой он некогда боролся. Секуляризм двадцатого века склонен утверждать, что человеку не нужен Бог. Но берегитесь! Эта безбожная философия человеческого общества приведет лишь к беспорядкам, вражде, несчастьям, войне и всемирным бедствиям.
\vs p195 8:6 \P\ Секуляризм никогда не сможет принести человечеству мир. Ничто не может занять в человеческом обществе место Бога. Но помните! Не надо торопиться отказываться от благотворных достижений светской борьбы против церковного тоталитаризма. Сегодняшняя западная цивилизация обрела много свобод и благ вследствие борьбы за независимость от церкви. Огромная ошибка секуляризма заключалась в следующем: восстав против почти полного контроля религиозных властей над жизнью и добившись освобождения от такой церковной тирании, секуляристы пошли дальше, начав восстание против самого Бога, иногда --- подспудно, иногда --- открыто.
\vs p195 8:7 Борьбе с официальной церковью мы обязаны поразительным творческим потенциалом американского индустриализма и беспрецедентным материальным прогрессом западной цивилизации. А поскольку секуляристское восстание зашло слишком далеко, забыв Бога и \bibemph{истинную} религию, последовала неожиданная расплата в виде мировых войн и международных конфликтов.
\vs p195 8:8 Нет необходимости приносить веру в Бога в жертву тем благам, которые принесло современное секуляристское восстание: терпимость, общественное служение, демократическое правление, гражданские свободы. Секуляристам не было необходимости бороться с истинной религией для того, чтобы развивать науку и образование.
\vs p195 8:9 Но секуляризм --- не единственный родоначальник всех этих достижений последнего времени, обогативших жизнь. За достижениями двадцатого века стоят не только наука и секуляризм, но и незамеченное и непризнанное духовное влияние жизни и учений Иисуса из Назарета.
\vs p195 8:10 Без Бога, без религии научный секуляризм никогда бы не смог скоординировать свои силы, установить согласие между различными и часто соперничающими интересами, расами и национальными чувствами. Секуляристское человеческое общество, несмотря на свои беспрецедентные материальные успехи, постепенно распадается. Главная связующая сила, противостоящая этому антагонистическому распаду, --- это национализм. И национализм же есть главная преграда на пути к миру во всем мире.
\vs p195 8:11 Неотъемлемо присущая секуляризму слабость заключается в том, что он отказывается от этики и религии ради политики и власти. Утвердить братство людей, игнорируя или отрицая при этом отцовство Бога, просто невозможно.
\vs p195 8:12 Секулярный общественный и политический оптимизм суть лишь иллюзия. Без Бога ни свобода и независимость, ни собственность и богатство не приведут к миру.
\vs p195 8:13 Полная секуляризация науки, образования, промышленности и общества может привести только к катастрофе. За первую треть двадцатого века урантийцы убили больше людей, чем было убито за все время существования христианства вплоть до этого времени. И это только начало жатвы ужасного урожая материализма и секуляризма; впереди еще больше ужасов смерти и разрушения.
\usection{9.\bibnobreakspace Проблема христианства}
\vs p195 9:1 Не забывайте о ценности вашего духовного наследства --- реке истины, текущей через столетия к бесплодным временам эпохи материализма и секуляризма. При всех своих достойных похвалы попытках избавиться от суеверных верований прошлых эпох неукоснительно твердо придерживайтесь вечной истины. Но будьте терпеливы! Когда сегодняшняя борьба; против предрассудков закончится, истины евангелия Иисуса продолжат восхитительно освещать новый и лучший путь.
\vs p195 9:2 Но христианству, испытавшему влияние язычества и обобществления, необходимо заново соприкоснуться с учениями Иисуса в их первоначальном виде; оно чахнет из\hyp{}за отсутствия нового видения жизни Учителя на земле. Новому и более полному раскрытию религии Иисуса суждено победить империю материалистического секуляризма и низвергнуть мировое господство механистического натурализма. Урантия сейчас стоит на грани одной из самых удивительных и увлекательных эпох социальной реорганизации, нравственного оживления и духовного просвещения.
\vs p195 9:3 Учения Иисуса, хотя и сильно видоизмененные, пережили в период своего рождения мистические культы, в темные века --- невежество и суеверие, а теперь постепенно одерживают победу над материализмом, механицизмом и секуляризмом двадцатого века. А времена великих испытаний и угрозы поражения всегда бывают временами великих откровений.
\vs p195 9:4 \P\ Религия нуждается в новых лидерах, одухотворенных мужчинах и женщинах, которые возьмут на себя смелость полагаться только на Иисуса и его несравненные учения. Если христианство будет упорно пренебрегать своей духовной миссией, продолжая заниматься социальными и материальными проблемами, то для духовного возрождения придется дожидаться прихода новых учителей религии Иисуса, которые посвятят себя исключительно духовному возрождению людей. И тогда эти рожденные в духе души быстро обеспечат водительство и вдохновляющие идеи, необходимые для социальной, нравственной, экономической и политической реорганизации мира.
\vs p195 9:5 Современная эпоха откажется принять религию, которая противоречит фактам и не созвучна высочайшим представлениям об истине, красоте и добродетели. Настает час нового открытия истинных и подлинных основ сегодняшнего искаженного и несущего на себе печать компромиссов христианства --- нового открытия подлинной жизни и учений Иисуса.
\vs p195 9:6 \P\ Первобытный человек жил жизнью, полной суеверной зависимости от религиозного страха. Современные цивилизованные люди страшатся мысли, что попадут под власть сильных религиозных убеждений. Мыслящий человек всегда боялся находиться \bibemph{во власти} религии. Когда сильная и трогающая душу религия угрожает обрести над ним власть, он неизменно пытается сделать ее рациональной, традиционной и институционной, надеясь, таким образом, обрести над ней контроль. В результате таких действий даже богооткровенная религия превращается в созданную человеком и находящуюся под его властью. Современные интеллектуальные мужчины и женщины избегают религии Иисуса из\hyp{}за страха перед тем, что она сделает \bibemph{с} ними и \bibemph{из} них. И все эти страхи вполне обоснованы. Религия Иисуса действительно обретает власть над верующими и преображает их, требуя, чтобы люди посвящали свою жизнь стремлению узнать волю Отца Небесного и направляли свою жизненную энергию на служение человеческому братству.
\vs p195 9:7 Эгоистичные мужчины и женщины просто не станут платить такую цену даже за величайшее духовное сокровище, когда\hyp{}либо предлагавшееся смертному человеку. Только когда печальные разочарования, сопутствующие неразумным и обманчивым эгоистическим устремлениям, и понимание бесплодности формальной религии в достаточной степени лишат человека иллюзий, тогда он будет склонен всем сердцем обратиться к евангелию царства, к религии Иисуса из Назарета.
\vs p195 9:8 Мир нуждается в религии из первых рук. Даже христианство --- лучшая из религий двадцатого века --- это не только религия \bibemph{об} Иисусе, но в очень большой степени --- религия, которую люди узнают из вторых рук. Они полностью принимают религию в том виде, в каком она преподносится их признанными религиозными учителями. Какое пробуждение пережил бы мир, если бы он только смог увидеть Иисуса таким, каким тот действительно жил на земле, и из первых уст услышать его живительные учения! Слова, описывающие красоту, не могут взволновать так же сильно, как ее созерцание, и слова о религии не могут вдохновлять души людей так, как опыт осознания присутствия Бога. Но ожидающая вера всегда будет держать дверь человеческой души открытой для вечных духовных реалий божественных ценностей потусторонних миров.
\vs p195 9:9 \P\ Христианство решилось принизить свои идеалы перед лицом человеческой жадности, безумной тяги к войне и жажды власти; но религия Иисуса выступает как незапятнанный и божественный духовный призыв ко всему лучшему, что есть в человеке, возвыситься над всеми этими наследиями животной эволюции и через благодать достичь нравственных высот истинного человеческого предназначения.
\vs p195 9:10 Христианству угрожает постепенная гибель из\hyp{}за формализма, заорганизованности, интеллектуализма и прочих недуховных тенденций. Современная христианская церковь --- не то братство активных верующих, кому Иисус поручил постоянно осуществлять духовное преображение последующих поколений человечества.
\vs p195 9:11 Так называемое христианство стало общественным и культурным движением, а не только религиозным верованием и ритуалом. В поток современного христианства стекается много воды из болота язычества и трясины варварства; в этот нынешний культурный поток впадают воды из многих старых культурных водоразделов, а не только с высокого Галилейского плоскогорья, где должен был быть его единственный источник.
\usection{10.\bibnobreakspace Будущее}
\vs p195 10:1 Христианство поистине сослужило великую службу этому миру, но что сейчас больше всего необходимо --- так это Иисус. Миру снова нужно увидеть на земле Иисуса, живущего через опыт духовно родившихся смертных, которые успешно открывают Учителя всем людям. Бесполезно говорить о возрождении первоначального христианства; надо идти вперед с того места, где находишься. Современная культура должна принять духовное крещение через новое открытие жизни Иисуса и обрести новое понимание его евангелия вечного спасения. И когда Иисус будет таким образом возвышен, он привлечет к себе всех людей. Последователи Иисуса должны быть больше, чем победителями, они должны быть кипучими источниками вдохновения и более полной жизни для всех людей. Религия --- это только лишь возвышенный гуманизм, но она становится божественной, когда на личном опыте осознается реальность присутствия Бога.
\vs p195 10:2 Красота и величие, человечность и божественность, простота и уникальность жизни Иисуса на земле представляют такую поразительную и привлекательную картину спасения человека и открытия Бога, что теологи и философы всех времен должны будут воздерживаться от попыток на основе такого трансцендентального пришествия Бога в облике человека осмеливаться формулировать символы веры и создавать теологические системы духовного рабства. В лице Иисуса вселенная сотворила такого смертного человека, в котором дух любви восторжествовал над материальными препятствиями времени и преодолел физическое начало.
\vs p195 10:3 \P\ Всегда помните --- Бог и человек нужны друг другу. Они взаимно необходимы для полного и окончательного обретения вечного опыта личности в божественной судьбе финальности вселенной.
\vs p195 10:4 <<Царство Бога внутри вас>> --- это были, возможно, самые великие слова из всех, сказанных Иисусом, наряду с возвещением, что его Отец --- живой и любящий дух.
\vs p195 10:5 \P\ При завоевании душ для Учителя человека и его мир преображает не первый этап принуждения, исполнения долга или следования обычаю, а \bibemph{второй} этап свободного служения и вольнолюбивого рвения, который означает, что приверженец Иисуса стремится объять своего брата любовью и вести его под духовным руководством вперед к высшей и божественной цели человеческого существования. Сейчас христианство с готовностью проходит \bibemph{первую} версту, но человечество изнемогает и идет, спотыкаясь в нравственной темноте, потому что очень немногие готвы вторую версту --- слишком мало из признающих себя последователями Иисуса действительно живут и любят так, как он учил своих учеников жить и любить и служить.
\vs p195 10:6 Призыв к замечательному делу строительства нового и преображенного человеческого общества через духовное возрождение братства царства Иисуса должен так взволновать всех верующих в него, как ничто не волновало их с тех пор, как они общались на земле с Иисусом во плоти.
\vs p195 10:7 Никакая общественная система или политический режим, которые отрицают реальность Бога, не могут внести никакого конструктивного и прочного вклада в развитие человеческой цивилизации. Но сегодняшняя раздробленность и секуляризованность христианства --- это самое большое препятствие для его дальнейшего развития; особенно это касается Востока.
\vs p195 10:8 \P\ Церковность раз и навсегда несовместима с этой живой верой, ростом в духе и собственным неопосредованным опытом товарищей Иисуса по вере, которые, пребывая вместе с Иисусом в братстве людей, изведали при этом духовное братство царства небесного. Похвальное желание сохранять традиции прошлых достижений часто ведет к защите изживших себя систем вероисповедания. Благонамеренное желание пестовать древние системы мысли во многом препятствует появлению новых и адекватных средств и методов, предназначенных удовлетворить духовную жажду развивающегося и совершенствующегося разума современных людей. Более того, христианские церкви двадцатого века совершенно непроизвольно оказываются огромным препятствием для немедленного успешного распространения подлинного евангелия --- учений Иисуса из Назарета.
\vs p195 10:9 Многие искренние люди, которые с радостью принесли бы свою преданность в дар евангельскому Христу, чувствуют, что им трудно воодушевленно поддержать церковь, в которой так мало духа его жизни и учений и которую, как их ошибочно учили, он основал. Иисус не основывал так называемую христианскую церковь, но, как мог согласно своей природе, он \bibemph{лелеял} ее как наилучшего существующего истолкователя трудов его жизни на земле.
\vs p195 10:10 Если бы только христианская церковь решилась последовать предначертаниям Учителя, тысячи молодых людей, кажущихся безразличными, устремились бы вперед, чтобы включиться в такое духовное начинание, и без колебаний пошли бы до конца в этом великом захватывающем деле.
\vs p195 10:11 Христианство вплотную сталкивается с роковыми проблемами, выраженными в одном из его собственных лозунгов: <<Дом, разделившийся внутри себя, не может устоять>>. Нехристианский мир едва ли капитулирует перед христианством, разделенным на секты. Живой Иисус --- это единственная надежда на возможное объединение христианства. Истинная церковь --- братство Иисуса --- невидимая, духовная, и ей присуще \bibemph{единство,} но не обязательно \bibemph{единообразие.} Единообразие --- это отличительное свойство физического мира механистической природы. Духовное единство --- это плод единения с живым Иисусом через веру. Зримая церковь должна прекратить далее препятствовать прогрессу незримого и духовного братства царства Бога. А этому братству суждено стать \bibemph{живым организмом} в противоположность институционализованной общественной организации. Оно вполне может использовать такие общественные организации, но не должно ими подменяться.
\vs p195 10:12 Но к христианству даже двадцатого века нельзя относиться с презрением. Оно есть продукт совместного нравственного гения узнавших Бога людей из многих народов, живших в течение многих веков, и оно поистине было одной из величайших движущих сил добра на земле, и поэтому ни один человек не должен относиться к нему пренебрежительно, несмотря на органически свойственные ему и приобретенные изъяны. Христианству по\hyp{}прежнему удается вызывать в умах мыслящих людей глубокие нравственные эмоции.
\vs p195 10:13 Но нет никакого оправдания участию церкви в коммерции и политике; такие нечестивые связи являются вопиющим предательством по отношению к Учителю. А те, кто действительно любят истину, не скоро забудут, что эта могущественная институционализованная церковь часто осмеливалась подавлять только что родившуюся веру и преследовать тех, кто нес истину, но являлся при этом в необычном и неортодоксальном обличии.
\vs p195 10:14 Несомненно, что такая церковь не могла бы продолжать существовать, если бы в мире не было людей, предпочитающих именно такой порядок богослужения. Многие духовно ленивые души жаждут древней и авторитетной религии ритуала и священных традиций. Человеческого развития и духовного прогресса едва ли достаточно для того, чтобы все люди смогли обойтись без религиозной власти. И невидимое братство царства вполне может включить в себя таких людей из разных социальных классов и обладающих разными характерами, если только они проявят готовность поистине стать сынами Бога, ведомыми духом. Но в этом братстве Иисуса нет места для соперничества сект, межгрупповой неприязни или притязаний на моральное превосходство и духовную непогрешимость.
\vs p195 10:15 Эти разные сообщества христиан могут соответствовать потребностям многообразных типов потенциальных верующих, принадлежащих к разным народам западной цивилизации, но из\hyp{}за такой разделенности христианство оказывается во многом бессильно, когда пытается нести евангелие Иисуса восточным народам. Эти народы еще не понимают, что существует \bibemph{религия Иисуса} отдельно и несколько в стороне от христианства, которое все более и более становится \bibemph{религией об Иисусе.}
\vs p195 10:16 Великая надежда Урантии кроется в возможности нового раскрытия Иисуса и нового, расширенного представления его спасительной вести, которая духовно объединила бы в исполненном любви служении многочисленные семейства тех, кто ныне считают себя его последователями.
\vs p195 10:17 Даже секулярное образование могло бы помочь этому великому делу духовного возрождения, если бы оно уделяло больше внимания обучению молодежи тому, как планировать жизнь и совершенствовать характер человека. Любое образование должно ставить себе задачу благоприятствовать и содействовать достижению верховной цели жизни --- развитию величественной и гармоничной личности. Совершенно необходимо воспитывать нравственную дисиплину, а не чувство неизменного самодовольства. На такой основе религия, благодаря своему духовному стимулу, может способствовать развитию и обогащению человеческой жизни и стать порукой существования в вечной жизни.
\vs p195 10:18 Христианство --- это импровизационная религия, и поэтому оно должно действовать не спеша. Для интенсивной духовной деятельности необходимо дождаться нового раскрытия и более массового принятия подлинной религии Иисуса. Но христианство --- мощная религия, судя по тому, что простые ученики распятого плотника распространили те учения, которые за триста лет покорили Римский мир, а затем восторжествовали и над повергшими Рим варварами. То же самое христианство покорило --- впитало и возвысило --- все течения еврейской теологии и греческой философии. И даже после того, как вследствие избыточного влияния мистерий и язычества христианская религия почти на тысячелетие впала в коматозное состояние, она воскресла и фактически вновь покорила весь западный мир. В христианстве вполне достаточно учений Иисуса, чтобы стать бессмертным.
\vs p195 10:19 Если бы христианство смогло усвоить больше учений Иисуса, то оно гораздо больше смогло бы помочь современному человеку в решении его новых и все более сложных проблем.
\vs p195 10:20 Огромной помехой для христианства является то, что в умах всего мира оно стало восприниматься как часть общественной системы, индустриальной жизни и нравственных стандартов западной цивилизации; и, таким образом, невольно стало казаться, что христианство поддерживает общество, которое сгибается под тяжестью своей вины --- терпимого отношения к науке без идеализма, политике без принципов, богатству без работы, удовольствию без удержу, знанию без характера, силе без совести и индустрии без нравственности.
\vs p195 10:21 Надежда современного христианства в том, что оно прекратит поддерживать общественные системы и индустриальную политику западной цивилизации и в то же время смиренно склонится перед крестом, который оно так доблестно превозносит, чтобы заново узнать от Иисуса из Назарета величайшие истины, которые когда\hyp{}либо может услышать смертный человек, --- живое евангелие отцовства Бога и братства человека.
