\upaper{82}{Эволюция брака}
\vs p082 0:1 Брак --- половые отношения --- следствие бисексуальности. Брак --- это способ, каким человек приспособился к бисексуальности, в свою очередь семейная жизнь --- конечный результат, вытекающий из всех апробированных эволюционных и адаптивных способов приспособиться. Брак --- фундаментальный институт; он не присущ биологической эволюции, однако является основой всей эволюции общества и потому в той или иной форме обязательно продолжит свое существование. Брак дал человечеству семью, а семья --- это славный венец всей долгой и напряженной эволюционной борьбы.
\vs p082 0:2 Хотя религиозные, социальные и образовательные институты --- все жизненно необходимы для выживания культурной цивилизации, \bibemph{семья --- главный цивилизатор.} Ведь большинству жизненных принципов ребенок учится у своей семьи и соседей.
\vs p082 0:3 Люди древности не обладали очень богатой общественной цивилизацией, однако ту, что у них была, они верно и успешно передавали следующему поколению. И вы должны понимать, что большинство цивилизаций прошлого и развивались при минимальном влиянии со стороны других институтов, поскольку семья как институт действовала эффективно. Сегодня человеческие расы обладают богатым социально\hyp{}культурным наследием, и его следует мудро и умело передавать следующим поколениям. Как воспитательный институт семья должна сохраняться.
\usection{1. Половой инстинкт}
\vs p082 1:1 Несмотря на пропасть, отделяющую личность мужчины от личности женщины, полового влечения достаточно, чтобы обеспечить их сближение для рождения себе подобных. Этот инстинкт эффективно действовал задолго до того, как люди стали испытывать те чувства, которые позднее получили названия --- любовь, преданность и супружеская верность. Половое влечение --- это врожденная наклонность, а брак --- ее эволюционное последствие.
\vs p082 1:2 Половое желание и влечение не были господствующими страстями у первобытных народов; они просто принимали их как должное. Вся репродуктивная деятельность была свободна от той соблазнительной прелести, которой наделяет ее воображение. Всепоглощающая сексуальная страсть у народов, обладающих более высокой цивилизацией, в основном объясняется расовыми смешениями, особенно в тех случаях, когда эволюционирующая природа стимулировалась художественным воображением и любовью к красоте, свойственными Нодитам и Адамитам. Однако это наследие Андитов было усвоено эволюционирующими расами столь ограниченно, что у них не развилось в достаточной степени чувство самообладания, чтобы обуздывать животные страсти, которые, благодаря дарованию более острого полового сознания и более сильного полового инстинкта, пробудились и усилились. Из всех эволюционных рас красный человек обладал самой совершенной культурой половых отношений.
\vs p082 1:3 \P\ Регулирование половых отношений в браке свидетельствует:
\vs p082 1:4 \ublistelem{1.}\bibnobreakspace Об относительном прогрессе цивилизации. Цивилизация все больше требовала, чтобы удовлетворение полового инстинкта оказывало благотворное влияние и соответствовало нравам.
\vs p082 1:5 \P\ \ublistelem{2.}\bibnobreakspace О количестве андической крови у любого народа. У таких групп половая сфера стала олицетворять как самое высокое, так и самое низменное и в физическом, и в эмоциональном плане.
\vs p082 1:6 \P\ Для сангических рас была характерна нормальная животная страсть, но они проявляли мало воображения и мало ценили красоту и внешнюю привлекательность противоположного пола. То, что называют сексуальной привлекательностью, фактически отсутствует даже у современных примитивных рас; у этих, не подвергшихся смешению народов, есть определенный половой инстинкт, но нет достаточной сексуальной страсти, которая бы создавала серьезные проблемы, требующие общественного регулирования.
\vs p082 1:7 У людей половой инстинкт --- одно из главных физиологических побуждений; это и есть чувство, которое под видом личного удовлетворения на самом деле эффективно вынуждает эгоистичного человека ставить благополучие расы и ее увековечение намного выше своего индивидуального комфорта и личной свободы от ответственности.
\vs p082 1:8 Как институт, брак от своих истоков до новейшего времени отображает социальную эволюцию биологического стремления к самоувековечению. Увековечение совершенствующегося человеческого вида гарантируется присутствием у рас полового чувства, стремления, которое называют сексуальным влечением. Это великое биологическое устремление становится источником всевозможных видов связанных с ним инстинктов, чувств и обычаев --- физических, интеллектуальных, нравственных и социальных.
\vs p082 1:9 У первобытных людей добыча пищи была определяющим побуждением; когда же уровень цивилизации начинает гарантировать изобилие пищи, роль полового влечения как главного побуждения часто становится основной и потому постоянно нуждается в общественном регулировании. У животных инстинктивная периодичность управляет половым влечением, однако, поскольку человек в столь значительной степени --- существо, контролирующее самого себя, то и половое желание у него не периодично; поэтому общество вынуждено приучать индивидуума к сдержанности.
\vs p082 1:10 Ни одно человеческое чувство или побуждение, даже когда оно становится необузданным или им чрезмерно увлекаются, не может причинить столько вреда и горя, как мощное половое влечение. Разумное подчинение этого побуждения общественным правилам является высшим мерилом подлинности любой цивилизации. Сдержанность, большая и еще большая сдержанность --- вот постоянно усиливающееся требование человечества, идущего по пути прогресса. Скрытность, неискренность и лицемерие могут замаскировать сексуальные проблемы, но не решают их и не способствуют развитию этики.
\usection{2. Ограничительные табу}
\vs p082 2:1 Рассказ об эволюции брака --- это просто история управления половыми отношениями посредством социальных, религиозных и гражданских ограничений. Природа не признает индивидуумов; она не знает так называемой морали; ее интересует только размножение вида и исключительно это. Природа неумолимо настаивает на размножении, но безучастно предоставляет обществу решать вытекающие из этого проблемы, тем самым создавая эволюционирующему человечеству постоянную и острую проблему. Это социальное противоречие заключается в непрекращающейся борьбе между основными инстинктами и совершенствующейся этикой.
\vs p082 2:2 \P\ У древних рас почти или совсем не было регулирования отношений между полами. Вследствие такой сексуальной свободы проституции не существовало. Сегодня у пигмеев и других отсталых народностей нет института брака, и изучение этих народов обнаруживает в сфере половых отношений простые обычаи соединения пар, которым и следуют примитивные расы. Однако древние народы всегда следует изучать и судить о них в свете моральных норм нравов их времени.
\vs p082 2:3 Однако свободная любовь уже никогда не допускалась в обществах, стоящих на более высоком уровне, чем первобытное. Одновременно с началом формирования общественных групп стали вырабатываться нормы семейных отношений и брачные ограничения. Половые отношения, таким образом, в своем развитии преодолели множество переходных периодов --- от состояния почти полной сексуальной свободы до норм двадцатого века, для которых характерны достаточно строгие ограничения, действующие в половой сфере.
\vs p082 2:4 На самых ранних этапах племенного развития нравственные нормы и ограничительные табу были весьма грубыми, но они сдерживали представителей противоположных полов --- это способствовало спокойствию, порядку и труду --- с этого и началась долгая эволюция брака и семьи. Основанные на половом различии традиции одеваться, украшать себя и религиозные обычаи происходят от этих древних табу, которые определяли диапазон сексуальных свобод и, таким образом, в конечном итоге сформировали понятия порока, преступления и греха. Однако обычай отменять все ограничения, регулирующие половые отношения, в дни больших праздников и особенно в день майских игр сохранялся еще очень долго.
\vs p082 2:5 \P\ Женщины всегда должны были подчиняться более строгим табу, чем мужчины. Древние нравы давали незамужним женщинам такую же степень сексуальной свободы, как и мужчинам, однако от жен всегда требовалось быть верными своим мужьям. Первобытный брак не особенно сокращал сексуальные свободы мужчин, зато усиливал табу сексуальной свободы для жен. Замужние женщины всегда носили какой\hyp{}нибудь знак, выделявший их в особый класс, например, особую прическу, другую одежду, покрывало, украшения и кольца или держались обособленно.
\usection{3. Древние нравы брака}
\vs p082 3:1 Брак является институтом\hyp{}реакцией общественного организма на постоянно присутствующее биологическое давление неослабевающего стремления человека к размножению --- самораспространению. Половые отношения всегда естественны, и по мере развития общества от простого к сложному происходила соответствующая эволюция нравов в сфере половых отношений, генезис института брака. Всюду, где эволюция общества достигает ступени, на которой вырабатываются нравы, брак становится развивающимся институтом.
\vs p082 3:2 В браке всегда существовали и всегда будут существовать две отдельные сферы: нравы --- законы, регулирующие внешние аспекты брака, и, наоборот, скрытые и личные отношения мужчин и женщин. Индивидуум всегда восставал против навязываемых обществом правил, регулирующих половые отношения; это и является причиной извечной проблемы пола: поддержка собственного существования --- дело индивидуума, но осуществляется группой; самоувековечение --- цель общества, но достигается благодаря индивидуальному влечению.
\vs p082 3:3 Нравы, когда их уважают, обладают достаточной силой, чтобы ограничивать и контролировать половое влечение, о чем свидетельствует опыт всех рас. Нормы супружеской жизни всегда были истинным показателем силы нравов и функциональной целостности гражданского правительства в данное время. Однако древние нравы в сфере половых отношений представляли собой массу несовместимых и грубых правил. Родители, дети, родственники и общество --- все имели свои, противоречащие друг другу интересы в правилах, регулирующих супружескую жизнь. Но, несмотря на все это, те расы, что поощряли и практиковали супружество, естественным образом поднялись до более высоких уровней, выжили и их численность даже увеличилась.
\vs p082 3:4 \P\ В первобытные времена брак определял уровень общественного положения; обладание женой являлось знаком отличия. Первобытный человек смотрел на день своей свадьбы как на день своего вступления в самостоятельность и зрелость. В одно время к браку относились как к обязанности перед обществом; в другое --- как к религиозному долгу; а в третье --- как к политическому требованию давать граждан государству.
\vs p082 3:5 Во многих древних племенах в качестве условия заключения брака требовалось совершить подвиг похищения; более развитые народы заменили подобные грабительские набеги атлетическими состязаниями и соревнованиями. Победители в таких состязаниях награждались главным призом --- возможностью выбирать лучших невест года. У охотников за головами юноша не мог жениться до тех пор, пока не становился обладателем, по крайней мере, одной головы, правда, такие головы иногда можно было и купить. Когда покупать жен перестали, их стали завоевывать в состязаниях на смекалку; такой обычай по\hyp{}прежнему существует во многих племенах чернокожих людей.
\vs p082 3:6 С развитием цивилизации некоторые племена предоставили женщинам право судить суровые брачные испытания мужской выносливости; таким образом, женщины могли проявлять благосклонность к избранным ими мужчинам. Эти брачные испытания включали в себя состязания в искусстве охоты, в борьбе и в способности обеспечивать семью. Долгое время существовал обычай, согласно которому жених должен был войти в семью невесты, по крайней мере, на один год, чтобы, живя и трудясь в ней, доказать, что он достоин той, которую хотел взять в жены.
\vs p082 3:7 От жены требовалась способность выполнять тяжелую работу и рожать детей. Она должна была уметь выполнять определенную часть сельскохозяйственной работы за заданное время. И если женщина уже родила ребенка до брака, это повышало ее ценность; тем самым доказывалась ее плодовитость.
\vs p082 3:8 \P\ Древние народы считали, что быть неженатым или незамужней позорно или даже грешно, это объясняет причину очень ранних браков; поскольку брак необходим, значит, чем раньше, тем лучше. Повсеместно распространено было также поверье, что холостые и незамужние не могут войти в страну духов, а это еще больше стимулировало браки между детьми, которые заключались даже при их рождении, а иногда и до него, в зависимости от того, какого пола будут будущие дети. Древние считали, что женатыми и замужними должны быть даже мертвые. Первоначально сваты использовались для заключения браков между умершими. И один родитель через таких посредников договаривался о браке умершего сына с умершей дочерью из другой семьи.
\vs p082 3:9 У более поздних народов брачным, как правило, считался возраст достижения половой зрелости, однако распространение этого обычая было непосредственно связано с прогрессом цивилизации. На ранних этапах эволюции общества как у мужчин, так и у женщин возникли особые социальные группы давших обет безбрачия; они основывались и поддерживались людьми, у которых в той или иной степени отсутствовало половое влечение.
\vs p082 3:10 Многие племена позволяли членам правящей группы иметь половые отношения с невестой перед тем, как отдать ее мужу. Каждый из этих людей дарил девушке подарок, откуда и возник обычай дарить подарки на свадьбу. А у некоторых племен было принято, чтобы молодые женщины зарабатывали свое приданое, которое состояло из подарков, полученных в награду за ее сексуальные услуги на выставке невест.
\vs p082 3:11 Некоторые племена женили молодых людей на вдовах или женщинах более старшего возраста, а затем, когда впоследствии они оставались вдовцами, позволяли им брать в жены молодых девушек, тем самым обеспечивая, как они выражались, то, что оба родителя не окажутся невеждами, что, как они полагали, произошло бы в случае, если бы в половые отношения вступали двое молодых. Другие племена разрешали половые отношения лишь между людьми сходных возрастов. Ограничения, допускавшие браки лишь в определенном возрасте, и породили первые идеи инцеста. (В Индии даже сейчас нет возрастных ограничений для вступления в брак.)
\vs p082 3:12 \P\ При определенных нравах следовало чрезвычайно опасаться вдовства, поскольку вдов либо убивали, либо позволяли им совершить самоубийство на могилах своих мужей, ибо считалось, что они должны перейти в страну духов вместе со своими супругами. Оставшуюся в живых вдову почти неизменно обвиняли в смерти мужа. И некоторые племена сжигали таких женщин заживо. Если же вдова продолжала жить, ее жизнь становилась жизнью постоянной скорби и невыносимых социальных ограничений, поскольку повторные браки, как правило, не одобрялись.
\vs p082 3:13 В древности поощрялись многие обычаи, которые теперь считаются безнравственными. Так, первобытные жены нередко очень гордились любовными связями своих мужей с другими женщинами. Целомудрие у девушки очень мешало замужеству, и рождение ребенка до брака крайне повышало желанность девушки как жены, поскольку мужчина был уверен, что он получит плодовитую спутницу.
\vs p082 3:14 Многие первобытные племена одобряли пробный брак до того, как женщина забеременеет, и лишь затем исполняли обычную брачную церемонию; у других групп свадьбу не играли до рождения первого ребенка. Если жена оказывалась бесплодной, то ее родители должны были ее выкупить, и брак аннулировался. Нравы требовали, чтобы каждая пара имела детей.
\vs p082 3:15 Эти первобытные пробные браки были полностью лишены какого\hyp{}либо сходства с сексуальной свободой, а просто являлись честным испытанием плодовитости. Заключившие договор вступали в постоянный брак, как только способность к воспроизведению потомства была установлена. Когда современные пары заключают брак с задней мыслью об удобном разводе, который они оформят, если не будут полностью удовлетворены своей супружеской жизнью, то в действительности они вступают в некую форму пробного брака и притом гораздо более низкого свойства, нежели честные предприятия их менее цивилизованных предков.
\usection{4. Брак и имущественные отношения}
\vs p082 4:1 Брак всегда был тесно связан как с собственностью, так и с религией. Собственность придавала браку прочность, а религия делала его нравственным.
\vs p082 4:2 Первобытный брак был своеобразным вложением капитала, экономической сделкой; скорее деловым вопросом, нежели ухаживанием. Древние женились ради выгоды и благополучия рода; по этой причине их браки планировались и устраивались родом, родителями и старшими. И то, что имущественные интересы были эффективным средством упрочения института брака подтверждается тем, что у древних племен брак был более устойчивым, чем у многих современных народов.
\vs p082 4:3 С развитием цивилизации и дальнейшим признанием обществом института частной собственности воровство стало считаться великим преступлением. Супружеская измена признавалась видом воровства, посягательством на имущественные права мужа, а потому в древнейших кодексах и нравственных нормах особо не оговаривалась. Сначала женщина была собственностью отца, который передавал свое право ее мужу, и все узаконенные половые отношения вытекали из этих существовавших прежде прав собственности. Ветхий Завет рассматривает женщину как форму собственности; Коран учит, что она --- существо низшее. Мужчина имел право одалживать свою жену другу или гостю, и у некоторых народов этот обычай сохраняется до сих пор.
\vs p082 4:4 Сексуальная ревность у современного человека --- отнюдь не врожденное чувство; она --- продукт развивающихся нравов. Первобытный человек не ревновал свою жену, а просто охранял свою собственность. Причина, по которой за сексуальные проступки с жены спрашивалось строже, чем с мужа, заключалась в том, что ее супружеская неверность затрагивала вопросы наследования и наследства. Весьма рано в развитии цивилизации незаконные дети стали пользоваться дурной славой. Сначала за нарушение супружеской верности наказывали только женщину; позднее нравы стали также предписывать наказание ее партнера, и в течение многих веков оскорбленный муж или опекающий отец имели полное право убить мужчину, посягнувшего на то, что они считали своим. Современные народы сохраняют эти нравы, которые, согласно неписанному закону, допускают так называемые преступления, совершенные во имя сохранения чести.
\vs p082 4:5 Поскольку табу целомудрия возникло как форма ограничений, касающихся имущества, оно вначале относилось только к замужним женщинам, а не незамужним девушкам. Позднее целомудрия требовал скорее отец, нежели человек, который ухаживал за девушкой; для отца девственница была коммерческой собственностью --- за нее больше платили. Когда же спрос на целомудрие возрос, вошло в обычай платить отцу выкуп за невесту в знак признательности за то, что он, как положено, вырастил девственную невесту для будущего мужа. Однажды возникнув, идея о женском целомудрии настолько укоренилась в в роде человеческом, что обычным делом стало буквально содержать девушек в клетке, фактически годами держать их в неволе, чтобы гарантировать их девственность. Поэтому позднее нормы и испытания, подтверждающие девственность, автоматически породили класс профессиональных проституток; это были отвергнутые невесты --- женщины, которых матери женихов девственницами не сочли.
\usection{5. Эндогамия и экзогамия}
\vs p082 5:1 Первобытные люди весьма рано заметили, что смешение рас улучшает качество потомства. Дело не в том, что брак между кровными родственниками всегда был плох, а в том, что брак между людьми, не состоящими в родстве, был всегда заметно лучше; поэтому нравы имели тенденцию ограничивать половые отношения между близкими родственниками. Было признано, что брак между людьми, не состоящими в родстве, значительно повышает селективные возможности для эволюционных изменений и развития. Индивидуумы, рожденные у родителей, не состоящих между собой в родстве, были более разносторонними и обладали большей способностью выживать во враждебном мире; рожденные же от родителей, состоявших в кровном родстве, постепенно вымирали вместе со своими традициями. Процесс этот был медленным; первобытные люди не задумывались о подобных проблемах. Но более поздние и развивающиеся народы поступали осознанно, ибо заметили, что слабость потомства порой была следствием чрезмерного распространения браков между кровными родственниками.
\vs p082 5:2 Хотя браки между кровными родственниками из хороших родов иногда и приводили к образованию сильных племен, особенно яркие случаи дурных последствий браков между кровными родственниками, при которых передавались наследственные дефекты, производили более сильное впечатление, так что совершенствовавшиеся нравы все настойчивее формулировали табу, запрещавшие браки между близкими родственниками.
\vs p082 5:3 \P\ Религия издавна была эффективной формой защиты от экзогамных браков; многие религиозные учения запрещали браки с иноверцами. Женщины, как правило, одобряли практику эндогамных браков, а мужчины --- экзогамных. Собственность всегда влияла на институт брака, и иногда стремление удержать собственность внутри клана порождало обычаи, вынуждавшие женщин выбирать мужей в племенах своих отцов. Подобные обычаи умножили браки между двоюродными братьями и сестрами. Эндогамные половые отношения практиковались также при попытке сохранить секреты ремесла; искусные работники стремились удержать профессиональное мастерство в своей семье.
\vs p082 5:4 \P\ Находясь в изоляции, высшие группы всегда возвращались к кровосмесительным связям. Так Нодиты больше ста пятидесяти тысяч лет были одной из крупных групп, где были приняты браки между близкими родственниками. Позднее на обычай заключать браки между родственниками оказали огромное влияние традиции фиолетовой расы, у которой сначала в случае необходимости допускались половые отношения между братом и сестрой. И браки между братом и сестрой были распространены в древнем Египте, Сирии, Месопотамии и всюду в местах, где когда\hyp{}то жили Андиты. Стараясь сохранить чистоту царской крови, египтяне долго практиковали браки между братом и сестрой, в Персии такой обычай сохранялся еще дольше. У жителей Месопотамии до дней Авраама браки между двоюродными братом и сестрой были обязательными; двоюродные братья обладали преимущественным правом брать в жены своих двоюродных сестер. Сам Авраам женился на единокровной сестре, однако подобные союзы по обычаям более поздних евреев не допускались.
\vs p082 5:5 Впервые начали отказываться от заключения браков между братом и сестрой, когда вошло в обычай многоженство, так как жена\hyp{}сестра высокомерно властвовала над другой женой или другими женами. Нравы некоторых племен запрещали брак с вдовой умершего брата, но требовали, чтобы оставшийся в живых брат производил потомство вместо своего покойного брата. Против эндогамных браков, какой бы степени родства они ни были, биологического инстинкта не существует; подобные ограничения целиком определяются табу.
\vs p082 5:6 \P\ Экзогамный брак, в конце концов, стал господствующим, поскольку мужчина отдавал ему предпочтение; брак с женщиной со стороны давал большую свободу от родственников жены. Близкое знакомство порождает неуважение; поэтому, когда в браке стал доминировать элемент личного выбора, стало обычаем отдавать предпочтение партнеру вне племени.
\vs p082 5:7 Многие племена, в конце концов, запретили браки внутри клана; некоторые ограничили выбор супруга определенными кастами. Табу, запрещавшее брак с женщиной из своего тотема, породило обычай похищать женщин из соседних племен. Позднее браки стали зависеть скорее от территории проживания, нежели от степени родства. В эволюции перехода от эндогамного брака к современному экзогамному было множество этапов. Даже после того, как табу стали запрещать эндогамные браки между простыми людьми, представителям знатных родов и царям дозволялось жениться на близких родственниках ради сохранения чистоты царской крови. Нравы, как правило, позволяли суверенным правителям пользоваться определенными свободами в сексуальных вопросах.
\vs p082 5:8 Присутствие более поздних андических народов во многом повлияло на возрастающее стремление сангических рас заключать браки вне своего племени. Однако преобладание экзогамных браков не было возможным до тех пор, пока живущие по соседству племена не научились сосуществовать в относительном мире.
\vs p082 5:9 Экзогамный брак сам по себе способствовал миру; браки между племенами уменьшали вражду. Экзогамный брак обуславливал взаимодействие племен и военные союзы и постепенно стал господствующим, потому что обеспечивал большую силу и являлся созидателем нации. Экзогамному браку также крайне благоприятствовал рост торговых связей; путешествия и исследования расширили границы половых взаимоотношений и чрезвычайно способствовали взаимному оплодотворению расовых культур.
\vs p082 5:10 Необъяснимая пестрота брачных обычаев во многом объясняется ничем иным как существующими у различных рас традициями заключать браки между людьми, не состоящими в родстве, которые сопровождались похищением или выкупом жен у чужих племен, что в конечном счете привело к объединению нравов отдельных племен. То, что эти табу в отношении эндогамного брака обуславливались социальными мотивами, а не биологическими, убедительно подтверждается примерами табу на браки между родственниками, действие которых распространялось на очень дальние степени родства жены или мужа, вплоть до случаев, где уже не было никакой кровной связи.
\usection{6. Смешение рас}
\vs p082 6:1 Сегодня чистых рас в мире не существует. Первые и изначально эволюционировавшие цветные народы сейчас в мире представляют только две сохранившиеся расы --- желтая и черная; но даже эти две расы сильно смешаны с вымершими цветными народами. Хотя так называемая белая раса произошла преимущественно от древнего голубого человека, она во многом так же, как и красный человек обеих Америк, в той или иной степени смешана со всеми остальными расами.
\vs p082 6:2 Из шести цветных сангических рас три были первичными, а три --- вторичными. Хотя первичные расы --- голубая, красная и желтая --- во многих отношениях превосходили три вторичных, следует помнить, что вторичные расы имели множество полезных качеств, которые бы значительно усилили первичные народы, если бы их лучшие наследственные свойства могли быть усвоены.
\vs p082 6:3 Нынешние предрассудки против «полукровок», «метисов» и «нечистокровных» возникают потому, что современные расовые смешения большей частью происходят между носителями наихудших наследственных качеств участвующих в смешении рас. Однако потомство с неудовлетворительными качествами получается и тогда, когда в брак вступают вырождающиеся представители одной и той же расы.
\vs p082 6:4 Если бы современные расы Урантии могли избавиться от бремени, которым являются их низшие слои, состоящие из вырожденцев, антиобщественных, слабоумных и отверженных элементов, то ограниченное расовое смешение не вызывало бы большой неприязни. А если бы подобные расовые смешения происходили между высшими типами нескольких рас, то возражений было бы еще меньше.
\vs p082 6:5 Гибридизация высших и несходных родов --- вот секрет создания новой и более сильной породы. Причем это истинно и для растений, и для животных, и для людей. Гибридизация увеличивает силу и повышает плодовитость. Расовые смешения средних или высших слоев различных народов сильно повышают \bibemph{творческий} потенциал, что видно на примере современного населения Северо\hyp{}Американских Соединенных Штатов. Когда же такие браки заключаются между представителями низших слоев, творческие возможности сокращаются, что видно на примере современных народов южной Индии.
\vs p082 6:6 Расовые смешения чрезвычайно способствуют внезапному появлению \bibemph{новых} качеств, а если такое смешение является союзом высших представителей рас, то и эти новые качества будут качествами \bibemph{высшего} свойства.
\vs p082 6:7 Пока современные расы столь перегружены низшими и вырождающимися родами, расовое смешение в большом масштабе будет крайне вредным, однако большинство возражений против подобных экспериментов основано на общественных и культурных предрассудках, а не на биологических соображениях. Даже среди низших родов рожденные в браках с другими низшими родами --- лучше, чем их предки. Гибридизация способствует улучшению вида благодаря роли, которую играют \bibemph{доминантные гены.} Смешение рас повышает вероятность увеличения числа полезных \bibemph{доминант,} присутствующих у гибрида.
\vs p082 6:8 \P\ За последние сто лет на Урантии произошло больше расовых гибридизаций, чем за предыдущие тысячи лет. Опасность большой дисгармонии вследствие смешения человеческих рас сильно преувеличивалась. Главные беды, идущие от «полукровок», объясняются общественными предрассудками.
\vs p082 6:9 Эксперимент Питкерна по смешению белой и полинезийских рас дал вполне хорошие результаты, поскольку белые мужчины и полинезийки обладали достаточно хорошими расовыми наследственными качествами. Смешение высочайших типов белой, красной и желтой рас немедленно приведет к появлению многих новых и биологически эффективных свойств. Эти три народа принадлежат к первичным сангическим расам. Смешение же белой и черной рас приводит не к таким положительным результатам; но опять\hyp{}таки подобное потомство мулатов не столько обладает нежелательными качествами, сколько таковыми их стараются представить социальные и расовые предрассудки. С физической точки зрения, такие черно\hyp{}белые метисы являются превосходным человеческим видом, несмотря на небольшое отставание в некоторых других аспектах.
\vs p082 6:10 \P\ Когда первичная сангическая раса смешивается со вторичной, последняя значительно улучшается за счет первой. И в малом масштабе --- на протяжении длительных периодов времени --- против подобного жертвенного вклада первичных рас в улучшение вторичных групп не может быть серьезных возражений. С биологической точки зрения, вторичные сангики в некоторых отношениях превосходят первичные расы.
\vs p082 6:11 Во всяком случае реальную опасность, угрожающую человеческим видам, следует искать в неограниченном умножении низших и вырождающихся наследственных свойств различных цивилизованных народов, а не в мнимой угрозе, вызванной смешением рас.
\vs p082 6:12 [Представлено Главным Архангелом, находящимся на Урантии.]
