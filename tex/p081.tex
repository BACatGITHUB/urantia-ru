\upaper{81}{Развитие современной цивилизации}
\author{Архангел}
\vs p081 0:1 Несмотря на успехи и неудачи миссий Калигастии и Адама, связанные с ошибками в исполнении планов улучшения мира, в целом биологическая эволюция человеческих видов продолжала продвигать расы по ступеням человеческого прогресса и расового развития. Эволюцию можно задержать, но остановить невозможно.
\vs p081 0:2 Влияние фиолетовой расы, хотя и менее многочисленной, чем планировалось, произвело в цивилизации сдвиг, который после дней Адама намного превысил результат развития, достигнутый человечеством за все свое предыдущее существование, продолжавшееся почти миллион лет.
\usection{1. Колыбель цивилизации}
\vs p081 1:1 В течение тридцати пяти тысяч лет после эпохи Адама колыбель цивилизации находилась в юго\hyp{}западной Азии и простиралась от долины Нила на восток и несколько севернее через северную Аравию и Месопотамию в Туркестан. Причем решающим фактором, определяющим возникновение цивилизации в этой области, был \bibemph{климат.}
\vs p081 1:2 Великие климатические и геологические изменения в северной Африке и западной Азии прервали первые миграции Адамитов в Европу, отделив их от нее расширившимся Средиземным морем, и направили поток миграции на север и на восток в Туркестан. Около 15 000 лет до н.э., когда прекратились подъемы суши и связанные с ними перемены климата, развитие мировой цивилизации зашло в тупик, но это не касалось культурного потенциала и биологических ресурсов Андитов, по\hyp{}прежнему отгороженных на востоке горами в Азии, а на западе --- разраставшимися лесами Европы.
\vs p081 1:3 Теперь изменение климата было готово завершить то, что не смогли сделать другие силы, то есть заставить евразийского человека бросить охоту и заняться более передовой деятельностью --- скотоводством и земледелием. Эволюция может быть медленной, но она чрезвычайно эффективна.
\vs p081 1:4 Поскольку рабы практически повсеместно использовались первыми земледельцами, охотники и скотоводы раньше относились к ним свысока. Веками считалось, что возделывать землю унизительно, оттуда и идет представление, что труд на земле --- проклятие, тогда как это величайшее из всех благ. Даже во дни Каина и Авеля жертвоприношения скотоводов ценились выше, нежели дары земледельцев.
\vs p081 1:5 Обычно человек из охотника превращался в земледельца, пройдя через эру скотоводства, это было справедливо и для Андитов, но гораздо чаще эволюционное давление, вызванное климатическими изменениями, вынуждало целые племена охотников сразу становиться успешными земледельцами. Однако такой процесс прямого перехода от охоты к земледелию происходил лишь в тех районах, где степень расового смешения с фиолетовой расой была высокой.
\vs p081 1:6 Эволюционирующие народы (особенно китайцы), наблюдая за прорастанием случайно увлажненных семян или семян, положенных в могилы в качестве пищи для умерших, еще в глубокой древности научились сажать семена и возделывать урожай. Однако по всей юго\hyp{}западной Азии вдоль плодородных речных пойм и на прилегающих к ним равнинах Андиты пользовались более совершенными методами земледелия, унаследованными от своих предков, которые сделали земледелие и садоводство своим главным занятием на территории второго сада.
\vs p081 1:7 Тысячелетиями потомки Адама на высокогорьях северной Месопотамии выращивали пшеницу и ячмень, выведенные в саду. Здесь встречались, торговали и смешивались общались потомки Адама и Адама\hyp{}сына.
\vs p081 1:8 Эти вынужденные перемены в условиях жизни и заставили достаточно большую часть рода человеческого стать всеядной в плане потребления продуктов питания. Сочетание же пшеницы, риса и растительной пищи с мясом животных положительно повлияло на результат укрепления здоровья и силы этих древних народов.
\usection{2. Орудия цивилизации}
\vs p081 2:1 Рост культуры основан на развитии орудий производства. И орудия, которыми пользовался человек на своем пути от первобытного состояния, были эффективны ровно настолько, насколько они освобождали силы человека для исполнения более высоких задач.
\vs p081 2:2 Вы, ныне живущие в окружении современной многообещающей культуры и начинающегося прогресса в общественных делах, действительно имеющие немного свободного времени, которое можете отдать \bibemph{размышлениям} об обществе и цивилизации, не должны забывать о том, что у ваших далеких предков почти не было никакого досуга, который они могли бы посвятить раздумьям или социальному анализу.
\vs p081 2:3 \pc Первые четыре великих достижения человеческой цивилизации таковы:
\vs p081 2:4 \ublistelem{1.}\bibnobreakspace Укрощение огня.
\vs p081 2:5 \ublistelem{2.}\bibnobreakspace Одомашнивание животных.
\vs p081 2:6 \ublistelem{3.}\bibnobreakspace Порабощение пленных.
\vs p081 2:7 \ublistelem{4.}\bibnobreakspace Частная собственность.
\vs p081 2:8 \pc Хотя огонь, первое великое открытие, в конечном итоге отворил двери в мир науки, он в этом отношении имел для первобытного человека лишь небольшое значение. Человек отказывался признавать естественные причины в качестве объяснения обыкновенных явлений.
\vs p081 2:9 Когда же возник вопрос, откуда взялся огонь, простую историю об Андоне и кремне вскоре заменила легенда о том, как некий Прометей похитил его с неба. Древние искали сверхъестественные объяснения всем природным явлениям, находящимся выше их личного понимания; и многие современные люди продолжают делать это. Для деперсонализации так называемых природных явлений потребовались века, и она еще не завершена. Однако искренний, честный и бесстрашный поиск истинных причин породил современную науку, превратив астрологию в астрономию, алхимию --- в химию, а волшебство --- в медицину.
\vs p081 2:10 \pc В домашинную эпоху единственный способ, посредством которого человек мог выполнить работу, не делая ее сам, заключался в использовании животных. Одомашнивание животных дало ему в руки живые орудия, разумное использование которых подготовило путь для появления и земледелия, и транспорта. И без этих животных человек не смог бы уйти от своего первобытного состояния и подняться до уровней цивилизации, возникших впоследствии.
\vs p081 2:11 Большинство животных, лучше всего поддающихся одомашниванию, находилось в Азии, особенно в ее центральных и юго\hyp{}западных районах. Это была одна из причин, почему цивилизация в этих местах развивалась быстрее, чем в других регионах мира. Многие из таких животных ранее уже дважды одомашнивались и в эпоху Андитов были приручены еще раз. Только собака, с давнейших времен прирученная голубым человеком, всегда оставалась рядом с охотниками.
\vs p081 2:12 Андиты Туркестана --- первые народы, которые широко использовали лошадь в домашнем хозяйстве, и это --- еще одна причина, почему их культура так долго доминировала. К пятому тысячелетию до н.э. земледельцы Месопотамии, Туркестана и Китая стали разводить овец, коз, коров, верблюдов, лошадей, птицу и слонов. И в качестве вьючных животных использовали волов, верблюдов, лошадей и яков. Одно время человек сам был вьючным животным. Так, у одного вождя синей расы когда\hyp{}то было сто тысяч носильщиков.
\vs p081 2:13 \pc Институты рабства и частного владения пришли вместе с земледелием. Рабство повысило уровень жизни хозяина и обеспечивало больший досуг, необходимый для общественной культуры.
\vs p081 2:14 Дикарь --- раб природы, однако развитие науки постепенно предоставляет человеку все большую свободу. Благодаря животным, огню, ветру, воде, электричеству и другим пока еще не открытым источникам энергии человек освободил и будет продолжать освобождать себя от необходимости беспрестанно трудиться. Несмотря на преходящие беды, вызванные множеством изобретенных машин, конечная выгода, извлекаемая из подобных механических изобретений, неоценима. Цивилизация не может процветать, а тем более устанавливаться, пока у человека не будет \bibemph{досуга} думать, планировать, изобретать новые и лучшие пути действий.
\vs p081 2:15 \pc Вначале человек просто подбирал себе кров, жил под выступами скал или в пещерах. Затем научился создавать семейные хижины, используя такие природные материалы, как дерево и камень. И наконец вступил в творческую стадию строительства домов, научился изготавливать кирпич и другие строительные материалы.
\vs p081 2:16 Народы высокогорий Туркестана были первыми из более современных рас, которые стали строить дома из дерева, дома, почти не отличавшиеся от бревенчатых хижин американских пионеров\hyp{}переселенцев. На равнинах же человеческие жилища складывали из кирпича, а позднее --- из обожженных кирпичей.
\vs p081 2:17 Древние расы, жившие вдоль рек, делали свои хижины, втыкая по кругу длинные жерди в землю; затем верхушки жердей связывали, образуя остов хижины, и переплетали поперечными прутьями, так что все сооружение напоминало огромную перевернутую корзину. Это сооружение затем могло быть обмазано глиной и, высохнув на солнце, превращалось в весьма прочное и защищающее от непогоды жилище.
\vs p081 2:18 От этих\hyp{}то первых хижин впоследствии и возникла самостоятельная идея плетения всевозможных корзин. А какую\hyp{}то группу наблюдение результатов обмазывания таких каркасов из жердей мокрой глиной подтолкнуло к идее изготовления гончарных изделий. Техника повышения прочности гончарных изделий путем обжига была открыта, когда одна из таких обмазанных глиной примитивных хижин случайно сгорела. Искусства древности зачастую возникали от случайностей повседневной жизни первобытных народов. По крайней мере, так было почти всегда и во всем, что касалось эволюционного развития человечества до прихода Адама.
\vs p081 2:19 Хотя гончарное дело впервые было введено штатом Принца около полумиллиона лет тому назад, изготовление глиняных сосудов практически прекратилось больше чем на сто пятьдесят тысяч лет. И лишь дошумерские Нодиты, жившие на берегу залива, продолжали делать глиняные сосуды. Искусство изготовления гончарных изделий возродилось во времена Адама. Распространение этого искусства совпало с расширением районов пустынь Африки, Аравии и центральной Азии и, последовательно и постепенно совершенствуясь, технология разошлась от Месопотамии по всему восточному полушарию.
\vs p081 2:20 Цивилизации эпохи Андитов не всегда можно выявить по этапам развития у них гончарного дела и других искусств. Равномерность человеческой эволюции чрезвычайно осложнялась укладом жизни и Даламатии, и Эдема. И часто случалось, что поздние вазы и орудия уступают по качеству аналогичным ранним предметам более чистых Андических народов.
\usection{3. Города, производство и торговля}
\vs p081 3:1 Климатическое разрушение богатых, пригодных для охоты и скотоводства открыто\hyp{}луговых пастбищ Туркестана, которое началось около 12 000 лет до н.э., вынудило людей, живших в этих районах, обратиться к новым формам производства и обработки сырья. Одни занялись разведением домашнего скота, другие стали земледельцами или рыбаками, однако высший тип умнейших Андитов решил заняться торговлей и производством. Обычным делом стало, когда целые племена посвящали себя развитию какой\hyp{}либо одной производственной отрасли. От долины Нила до Гиндукуша и от Ганга до Желтой реки главным занятием развитых племен стало возделывание земли и торговля как побочная отрасль.
\vs p081 3:2 Рост торговли и производства из сырья различных товаров прямо способствовал созданию тех первых и относительно мирных общин, которые в значительной степени влияли на распространение культуры и искусств цивилизации. До наступления эры широкой мировой торговли общины были племенными --- разросшимися семейными группами. Торговля объединяла различные типы людей и, таким образом, способствовала ускоренному взаимному оплодотворению культуры.
\vs p081 3:3 Около двенадцати тысяч лет тому назад начиналась эра независимых городов. Примитивные торговые и ремесленные города всегда были окружены зонами земледелия и скотоводства. Хотя и верно, что промышленность способствовала повышению уровня жизни, не следует заблуждаться относительно удобств городской жизни древних. Древние народы опрятностью и чистоплотностью особенно не отличались, и среднее первобытное поселение поднималось на 1\hyp{}2 фута каждые двадцать пять лет вследствие простого накопления грязи и мусора. Некоторые же из древних городов весьма быстро поднимались над окружающей территорией еще и потому, что мазанки из необожженной глины служили недолго, а по обычаю новые жилища строили прямо на обломках старых.
\vs p081 3:4 \pc Особенностью этой эры первых ремесленных и торговых городов было широкое использование металлов. Вы уже обнаружили бронзовую культуру в Туркестане, относящуюся ко времени до девятого тысячелетия до н.э., и Андиты также рано научились работать с железом, золотом и медью. Однако вдали от наиболее развитых центров цивилизации условия были уже совершенно иные. Отчетливых периодов, таких как каменный, бронзовый и железный век, не было; в различных местах все три периода существовали одновременно.
\vs p081 3:5 Золото было первым металлом, который искал человек; он легко обрабатывался и сначала из него делали только украшения. Следующим металлом, который стали использовать, была медь, однако ее широко не применяли, пока не научились смешивать с оловом, получив более прочную бронзу. Открытие, которое заключалось в том, что из сплава меди с оловом получается бронза, было сделано одним из потомков Адама\hyp{}сына в Туркестане; его высокогорный медный рудник случайно оказался рядом с месторождением олова.
\vs p081 3:6 \pc С появлением обработки сырья и началом производства торговля стала самым могущественным средством распространения культуры цивилизации. Открытие сухопутных и морских торговых путей чрезвычайно способствовало путешествиям и взаимовлиянию культур, сплаву цивилизаций. К пятому тысячелетию до н.э. лошадь повсеместно использовалась в цивилизованных и полуцивилизованных землях. У этих более поздних рас были не только одомашненные лошади, но и всевозможные повозки и колесницы. Уже несколько веков использовали и колесо, однако только теперь колесные транспортные средства стали применяться повсюду --- и в торговле, и в военных целях.
\vs p081 3:7 Путешествующий торговец и бродячий исследователь сделали для прогресса исторической цивилизации больше, нежели все остальные вместе взятые. Военные завоевания, колонизация и миссионерская деятельность, осуществляемая более поздними религиями, также являлись факторами распространения культуры; однако они все были вторичными по сравнению с торговыми связями, которые постоянно углублялись быстро развивавшимися искусствами и достижениями производства.
\vs p081 3:8 Вливание рода Адамитов в человеческие расы не только ускорило темпы развития цивилизации, но и чрезвычайно стимулировало их склонность к путешествиям и исследованиям, так что большая часть Евразии и северной Африки вскоре была заселена быстро размножавшимися смешанными потомками Андитов.
\usection{4. Смешанные расы}
\vs p081 4:1 Ближе к началу исторических времен всю Евразию, северную Африку и острова Тихого океана заселили уже смешанные человеческие расы. Причем современные расы появились в результате многократного смешивания пяти основных человеческих рас Урантии.
\vs p081 4:2 У каждой урантийской расы были свои отличительные особенности строения тела. Так, Адамиты и Нодиты были длинноголовыми, а Андониты широкоголовыми. Сангические расы были среднеголовыми, причем желтые и голубые люди, как правило, были ближе к широкоголовым. Голубые расы после смешения с расой Андонитов стали явно широкоголовыми. Вторичные Сангики были средне\hyp{}и длинноголовыми.
\vs p081 4:3 Хотя эти пропорции черепа и полезны в определении происхождения рас, скелет как нечто целое в этом смысле намного надежнее. В начале развития урантийских рас существовало пять отчетливых типов строения скелета:
\vs p081 4:4 \ublistelem{1.}\bibnobreakspace Андонический, скелет коренных жителей Урантии.
\vs p081 4:5 \ublistelem{2.}\bibnobreakspace Скелет первичных Сангиков, красных, желтых и голубых.
\vs p081 4:6 \ublistelem{3.}\bibnobreakspace Скелет вторичных Сангиков, оранжевых, зеленых и синих.
\vs p081 4:7 \ublistelem{4.}\bibnobreakspace Скелет Нодитов, потомков Даламатян.
\vs p081 4:8 \ublistelem{5.}\bibnobreakspace Адамический, скелет фиолетовой расы.
\vs p081 4:9 \pc По мере углублявшегося взаимовлияния этих пяти великих рас в результате непрерывного смешения андонический тип все больше затмевался наследственными свойствами Сангиков. Саами и эскимосы --- продукт смешения андонической и голубой сангической рас. Строение их скелетов более всего сохранило признаки коренного андонического типа. Однако Адамиты и Нодиты настолько перемешались с другими расами, что их можно идентифицировать лишь как представителей обобщенного европеоидного типа.
\vs p081 4:10 Поэтому если, в принципе, при раскопках будут обнаружены человеческие останки, относящиеся к последним двадцати тысячам лет, то будет невозможно четко выделить пять исходных типов строений скелета. Сопоставление строений таких скелетов покажет лишь, что теперь человечество можно условно разделить приблизительно на три класса.
\vs p081 4:11 \ublistelem{1.}\bibnobreakspace \bibemph{Европеоидный ---} андическая смесь нодических и адамических рас, подвергшаяся дальнейшему изменению в результате смешения с первичными и (некоторыми) вторичными Сангиками, а также значительного андонического влияния. К этой группе относят белые расы Запада и некоторые индийские и урало\hyp{}алтайские народы. Объединяющим фактором в этой группе является большая или меньшая доля андического наследия.
\vs p081 4:12 \pc \ublistelem{2.}\bibnobreakspace \bibemph{Монголоидный ---} первичный сангический тип, включающий в себя исходную красную, желтую и голубую расы. К этой группе относятся китайцы и америнды. В Европе монголоидный тип подвергся изменению в результате смешения с вторичными Сангиками и Андонитами, а еще больше вследствие смешения с Андитами. К этому классу относятся малайцы и другие индонезийские народы, хотя в их жилах и течет большая доля крови вторичных Сангиков.
\vs p081 4:13 \pc \ublistelem{3.}\bibnobreakspace \bibemph{Негроидный ---} вторичный сангический тип, первоначально включавший в себя оранжевую, зеленую и синюю расу. Лучший пример этого типа --- негры; этот тип встречается в Африке, Индии и Индонезии, всюду, где жили вторичные сангические расы.
\vs p081 4:14 \pc В северном Китае существует определенная смесь европеоидного и монголоидного типов; в Леванте произошло смешение европеоидного и негроидного типов; в Индии же так же, как в Южной Америке, представлены все три типа. Особенности строения скелетов трех дошедших до нас типов по\hyp{}прежнему сохраняются и помогают идентифицировать более поздних предков современных человеческих рас.
\usection{5. Культурное общество}
\vs p081 5:1 Биологическая эволюция и культурная цивилизация не всегда взаимосвязаны; органическая эволюция в любую эпоху может беспрепятственно продолжаться в самый разгар упадка культуры. Однако если рассматривать продолжительные периоды человеческой истории, то обнаруживается, что эволюция и культура в конечном итоге становятся взаимосвязанными как причина и следствие. Эволюция может продолжаться и при отсутствии культуры, однако культурная цивилизация не может процветать без убедительных результатов предшествующего ей развития рас. Адам и Ева практически не привнесли опыта цивилизации, чуждого прогрессу человеческого общества, но адамическая кровь усилила врожденные способности рас и ускорила ход экономического и производственного развития. Пришествие Адама улучшило умственные способности рас и тем самым значительно ускорило процессы естественной эволюции.
\vs p081 5:2 Благодаря земледелию, одомашниванию животных и развитию архитектуры человечество постепенно избавлялось от худших качеств в непрекращающейся борьбе за существование и стало задумываться над тем, как сделать жизнь более привлекательной, а это и было началом борьбы за все более высокий уровень материального благосостояния. Благодаря производству и промышленности человек постепенно добивается того, что смертная жизнь становится более приятной.
\vs p081 5:3 Однако культурное общество --- вовсе не огромный благотворительный клуб наследственных привилегий, куда свободно допускаются все люди на основе полного равенства по факту своего рождения. Скорее это --- высшая и постоянно развивающаяся гильдия земных работников, допускающая в свои ряды только избранных тружеников, которые стараются сделать мир лучшим местом, где их дети и дети их детей смогут жить и совершенствоваться в последующие века. И сия гильдия цивилизации взимает высокую плату за вход, предписывает строгую и неукоснительную дисциплину, подвергает суровым наказаниям всех инакомыслящих и не подчинявшихся общим правилам и вместе с тем почти не дарует личных свобод и привилегий, за исключением свобод и привилегий усиленной защиты от общих опасностей и от угроз, которым подвергается вся раса.
\vs p081 5:4 Общественное объединение есть форма гарантии выживания, которая, как поняли люди, весьма полезна; поэтому большинство людей готово платить ту цену самопожертвования и сокращения личных свобод, которую взимает общество со своих членов взамен за усиленную коллективную безопасность. Коротко говоря, современный общественный механизм --- это осуществляемый путем проб и ошибок план страхования, направленный на достижение некоторой степени гарантии и защиты от возврата к ужасным и антиобщественным условиям, характерным для первых опытов рода человеческого.
\vs p081 5:5 Общество, таким образом, становится совместным органом для обеспечения: гражданской свободы --- через законодательные основы, экономической свободы --- посредством капитала и предприимчивости, социальной свободы --- средствами культуры, а свободы от насилия --- посредством охраны правопорядка.
\vs p081 5:6 \bibemph{Сила не дает права, но она обеспечивает соблюдение повсеместно признаваемых социальных прав каждого следующего поколения.} Главная миссия правительства состоит в установлении права, справедливом и честном упорядочении классовых различий и обеспечении равных возможностей согласно правилам, предписываемым законом. Каждое право человека связано с его определенной обязанностью перед обществом; групповая привилегия есть страховой механизм, неизменно требующий полной уплаты в виде соответствующего группового служения. И должны защищаться как групповые права, так и индивидуальные, включая и регулирование сексуальных наклонностей.
\vs p081 5:7 Свобода, подчиненная групповому регулированию, --- вот законная цель общественной эволюции. Свобода без ограничений --- это абсурдная и нереальная мечта неустойчивых и непостоянных человеческих умов.
\usection{6. Сохранение цивилизации}
\vs p081 6:1 Хотя биологическая эволюция всегда развивалась в направлении совершенствования, культурная эволюция в основном распространялась из долины Евфрата волнами, которые с течением времени, постепенно ослабевая, прекратились, когда все прямые потомки Адама наконец ушли обогащать цивилизации Азии и Европы. Полного смешения рас не произошло, однако их цивилизации очень сильно перемешались. Культура медленно распространялась по всему миру. И эту цивилизацию необходимо сохранять и лелеять, ибо сегодня новых источников культуры не существует, и нет уже Андитов, которые бы оживили и стимулировали медленный прогресс эволюции цивилизации.
\vs p081 6:2 \pc Цивилизация, в настоящее время развивающаяся на Урантии, основана на следующих факторах, из которых она и возникла:
\vs p081 6:3 \ublistelem{1.}\bibnobreakspace \bibemph{Естественные обстоятельства.} Природа и уровень материального развития в огромной степени определяются доступными природными ресурсами. Климат, погода и многочисленные физические условия являются факторами эволюции культуры.
\vs p081 6:4 В начале эры Андитов во всем мире существовали лишь две обширные и плодородные открытые области, пригодные для охоты. Одна из них находилась в Северной Америке и была заселена Америндами; другая находилась севернее Туркестана и была частично заселена желтой андонической расой. Главными в эволюции высшей культуры на юго\hyp{}западе Азии являлись расовый и климатический факторы. Андиты были великим народом, однако решающим фактором, определявшим путь развития их цивилизации, была усиливающаяся засуха в Иране, Туркестане и Синьцзяне, которая и \bibemph{вынудила} их изобретать и применять новые и более совершенные методы добывания средств к существованию из своих земель, чье плодородие постепенно уменьшалось.
\vs p081 6:5 Очертания континентов и другие особенности расположения территорий были действенными факторами в вопросах войны и мира. Очень немногие обитатели Урантии имели такую же благоприятную возможность для непрерывного и спокойного развития, которой обладали народы Северной Америки --- защищенные практически со всех сторон огромными океанами.
\vs p081 6:6 \pc \ublistelem{2.}\bibnobreakspace \bibemph{Средства производства.} В условиях бедности культура не развивается никогда; для прогресса цивилизации необходим досуг. При отсутствии материального богатства отдельные люди могут развить в себе высоко нравственный и духовный характер, однако культурная цивилизация возможна лишь при наличии таких условий материального процветания, которые благоприятствуют досугу, и честолюбия.
\vs p081 6:7 В первобытные времена жизнь на Урантии была суровой и трудной. И желая избежать непрерывной борьбы и бесконечного утомительного труда, человечество постоянно стремилось переместиться в места благоприятного тропического климата. Хотя эти теплые регионы и освобождали в какой\hyp{}то мере от напряженной борьбы за существование, расы и племена, стремившиеся к облегчению таким путем, редко использовали свой незаслуженный досуг для развития цивилизации. К социальному прогрессу неизменно приводят замыслы и планы тех рас, которые своим разумным трудом научились добывать средства к существованию из земли меньшими усилиями и за более короткое время и, таким образом, приобрели возможность наслаждаться заслуженным и полезным досугом.
\vs p081 6:8 \pc \ublistelem{3.}\bibnobreakspace \bibemph{Научное знание.} Материальным результатам цивилизации всегда должно предстоять накопление научных данных. Лишь по истечении длительного времени после создания лука и стрелы и использования животных в качестве тягловой силы человек научился покорять ветер и воду, а затем и применять пар и электричество. Однако орудия цивилизации медленно совершенствовались. Вслед за одомашниванием животных, гончарным делом, ткачеством и обработкой металлов наступил век литературы и книгопечатания.
\vs p081 6:9 Знание --- сила. Изобретение всегда предшествует ускорению культурного развития во всемирном масштабе. Печатный станок принес науке и изобретательству наибольшую выгоду, и взаимодействие всех видов культурной и изобретательской деятельности чрезвычайно ускорило ход культурного развития.
\vs p081 6:10 Наука учит человека говорить на новом языке математики и приучает его мыслить с предельной точностью. Устраняя ошибки, она также укрепляет философию, а разрушая суеверия, очищает религию.
\vs p081 6:11 \pc \ublistelem{4.}\bibnobreakspace \bibemph{Человеческие ресурсы.} Для распространения цивилизации необходима человеческая сила. При прочих равных условиях многочисленный народ поглотит цивилизацию менее многочисленной расы. Следовательно, неспособность довести численность народа до определенного уровня мешает полной реализации национальной судьбы, однако при увеличении населения наступает момент, когда дальнейший рост становится самоубийственным. Увеличение численности, при котором плотность населения превышает оптимальную, означает либо снижение уровня жизни, либо немедленное расширение территориальных границ путем мирного проникновения или военного захвата, насильственной оккупации.
\vs p081 6:12 Разрушительные последствия войны порой ужасают вас, однако вы вынуждены признать необходимость рождения большого числа смертных для обеспечения широких возможностей социального и нравственного развития; при таком подходе к планетарной рождаемости вскоре возникает серьезная проблема перенаселения. Большинство обитаемых миров малы размерами. Урантия имеет среднюю величину или, быть может, чуть меньше средней. Стабилизация оптимальной численности нации укрепляет культуру и предотвращает войны. Нация, знающая, когда следует прекратить рост своей численности, --- это мудрая нация.
\vs p081 6:13 Однако континент, обладающий богатейшими природными запасами и самым современным производством, будет развиваться медленно, если интеллектуальные способности его народа находятся в состоянии упадка. Знания можно получить путем образования, мудрость же, без которой истинная культура обойтись не может, достигается лишь через опыт теми мужчинами и женщинами, которые обладают врожденным интеллектом; такие люди способны учиться у опыта и могут стать истинно мудрыми.
\vs p081 6:14 \pc \ublistelem{5.}\bibnobreakspace \bibemph{Эффективность материальных ресурсов.} Многое зависит от мудрости, проявляемой в использовании природных ресурсов, научных знаний, средств производства и человеческого потенциала. Основным фактором ранней цивилизации была \bibemph{сила,} проявляемая мудрыми общественными учителями; первобытному человеку цивилизация была буквально навязана его высшими современниками. Хорошо организованные и более развитые меньшинства, в основном, и управляли этим миром.
\vs p081 6:15 Сила не дает права, однако она дает то, что есть, и то, что было в истории. Лишь недавно общество Урантии достигло уровня, когда оно готово полемизировать об этике силы и права.
\vs p081 6:16 \pc \ublistelem{6.}\bibnobreakspace \bibemph{Эффективность языка.} Распространения цивилизации зависит от разовития языка. Живые и развивающиеся языки обеспечивают распространение цивилизованного мышления и планирования. В древности в области языка были достигнуты значительные успехи. И сегодня для того, чтобы облегчить выражение совершенствующейся мысли, существует огромная необходимость дальнейшего развития языка.
\vs p081 6:17 Язык развился из групповых связей, при том что каждая местная группа вырабатывала свою собственную систему словесного обмена. Язык создавался, проходя через этапы жестов, знаков, криков, имитирующих звуков, интонации и акцентов, вплоть до вокализации возникших впоследствии алфавитов. Язык --- это величайший и наиболее полезный инструмент мышления человека, однако он не процветал до тех пор, пока у социальных групп не появился некоторый досуг. Склонность вольно обращаться с языком рождает новые слова --- жаргон. Если большинство принимает жаргон, тогда частое использование делает его языком. Происхождение диалектов объясняется потаканием разговору на ломанном языке с детьми в семейной группе.
\vs p081 6:18 Языковые различия всегда были великой преградой на пути распространения мира. Победа над диалектами должна предшествовать распространению культуры в расе, на континенте и во всем мире. Универсальный язык способствует миру, обеспечивает развитие культуры, усиливает состояние счастья. Даже тогда, когда количество языков мира сокращается до нескольких, совершенное владение ими наиболее передовыми культурными народами оказывает сильное влияние на достижение мира и процветания во всем мире.
\vs p081 6:19 Хотя в создании международного языка на Урантии достигнут незначительный прогресс, многое оказалось возможным благодаря установлению международного торгового обмена. Все международные отношения следует укреплять, на чем бы они ни были основаны, будь то язык, торговля, искусство, наука, соревнования или религия.
\vs p081 6:20 \pc \ublistelem{7.}\bibnobreakspace \bibemph{Эффективность механических устройств.} Прогресс цивилизации непосредственно связан с развитием орудий, машин и транспортных путей и обладанием ими. Более совершенные инструменты, оригинальные и эффективные машины определяют выживание соперничающих групп на арене развивающейся цивилизации.
\vs p081 6:21 В древности единственным видом энергии, использовавшемся для обработки земли, была человеческая сила. Процесс замены человека волом был длительным, так как лишал человека работы. Позднее вытеснять человека стали машины, и каждый подобный шаг вперед прямо способствует развитию общества, поскольку освобождает силу человека для выполнения более ценных задач.
\vs p081 6:22 Наука, ведомая мудростью, может стать великим социальным освободителем человека. Век машин может стать гибельным лишь для нации, чей интеллектуальный уровень слишком низок, не способен найти мудрые методы и эффективные способы успешного приспособления к переходным трудностям, вызванным внезапной потерей работы большим числом людей вследствие стремительных темпов изобретения новых видов избавляющих от труда машин.
\vs p081 6:23 \pc \ublistelem{8.}\bibnobreakspace \bibemph{Характер факельщиков.} Общественное наследие позволяет человеку стоять на плечах всех своих предшественников, внесших свою лепту во всеобщую культуру и знания. И в сем деле передачи факела культуры следующему поколению семья всегда будет основным институтом. Затем следуют игра и общественная жизнь, а школа занимает последнее, но в равной степени незаменимое место в сложном и высоко организованном обществе.
\vs p081 6:24 Насекомые рождаются полностью обученными и подготовленными к жизни --- на самом деле весьма узкому и чисто инстинктивному существованию. Человеческое дитя рождается без каких бы то ни было знаний; поэтому человек обладает возможностью через управление образованием и воспитанием младшего поколения значительно изменить эволюционный путь развития цивилизации.
\vs p081 6:25 В двадцатом веке величайшими факторами, способствующими дальнейшему развитию цивилизации и культурному прогрессу, являются заметный рост путешествий по миру и беспрецедентные усовершенствования в методах связи. Однако уровень образования не во всем и не всегда отвечал развитию структуры общества; современное понимание этики отставало, часто не соответствовало интеллектуальному росту и достижениям в науке. А что касается духовного развития и охраны института семьи, то современная цивилизация не стронулась с мертвой точки.
\vs p081 6:26 \pc \ublistelem{9.}\bibnobreakspace \bibemph{Расовые идеалы.} Идеалы одного поколения определяют судьбу следующих поколений. От \bibemph{качества} факельщиков общества зависит, пойдет ли цивилизация вперед или двинется вспять. Семьи, церкви и школы одного поколения предопределяют характерные наклонности следующего поколения. Нравственный и духовный импульс расы или нации во многом определяет темпы культурного развития этой цивилизации.
\vs p081 6:27 Идеалы возвышают источник общественного потока. И ни один поток не поднимается выше своего источника, какой бы способ давления или контроля над направлением ни использовался. Движущая сила даже самых материальных аспектов культурной цивилизации кроется в наименее материальных достижениях общества. Разум может управлять механизмом цивилизации, мудрость может направлять ее, духовный же идеализм является энергией, которая действительно возвышает и продвигает человеческую культуру с одного уровня достижений на другой.
\vs p081 6:28 Сначала жизнь была борьбой за существование; сейчас это --- борьба за уровень жизни; в дальнейшем она станет борьбой за качество мышления, наступающую земную цель человеческого бытия.
\vs p081 6:29 \pc \ublistelem{10.}\bibnobreakspace \bibemph{Координирование специалистов.} Цивилизация чрезвычайно продвинулась вперед благодаря раннему разделению труда и возникшей из него позднее специализации. Теперь цивилизация зависит от эффективного координирования труда специалистов. По мере развития общества возникает необходимость отыскать некий метод, объединяющий действия различных специалистов.
\vs p081 6:30 Число специалистов в области управления обществом, специалистов в области искусства, технических и производственных специалистов будет и дальше множиться, и они будут наращивать свое мастерство и знания. И это разнообразие способностей и различия в труде в конечном итоге ослабят и разобщат человеческое общество, если не будут выработаны эффективные средства координирования и сотрудничества. Однако потенциал разума, обладающего способностью к такой изобретательности и такой специализации, должен быть совершенно достаточен для того, чтобы выработать адекватные методы контроля и решения всех проблем, вытекающих из быстрого роста изобретательства и ускоренного хода культурного развития.
\vs p081 6:31 \pc \ublistelem{11.}\bibnobreakspace \bibemph{Механизм решения вопроса занятости.} Следующая эпоха социального развития будет характеризоваться лучшим и более эффективным согласованием и координированием постоянно растущей и расширяющейся специализации. И по мере того, как труд будет становиться все более и более разнообразным, потребуется разработать определенный метод, руководствуясь которым, отдельные люди будут направляться на подходящее для них место работы. Машины --- не единственная причина безработицы у цивилизованных народов Урантии. Сложность экономики и неуклонный рост промышленной и профессиональной специализации усложняет проблемы занятости.
\vs p081 6:32 Подготовить людей к работе --- не достаточно; в обществе со сложной структурой необходимо также обеспечить эффективные методы поиска места работы. До обучения граждан высокоспециализированным способам добывания средств к существованию их следует научить одной или нескольким неквалифицированным специальностям, ремеслам или развить призвания, иными словами, привить навыки, которые они могут использовать, пока временно лишены работы по своей специальности. Ни одна цивилизация не может выдержать длительного содержания большого класса безработных. Со временем, принимая содержание из общественной казны, даже лучшие из граждан становятся развращенными и деморализованными. Даже частная благотворительность, и та становится пагубной, когда предлагается здоровым людям в течение длительного времени.
\vs p081 6:33 Подобное высокоспециализированное общество не будет терпимо относиться к древним общинным и феодальным традициям старинных народов. Верно, многие обыкновенные профессии могут быть приемлемо и выгодно поставлены под общественный контроль, однако наилучшим способом управления высокообразованными и обладающими узкой специальностью людьми может быть какой\hyp{}то метод разумного сотрудничества. Доведенное до современных стандартов координирование и основанное на братских традициях регулирование позволят достигнуть более долгосрочного сотрудничества, нежели старые и относительно примитивные методы коммунизма или диктаторские регулятивные институты, основанные на силе.
\vs p081 6:34 \pc \ublistelem{12.}\bibnobreakspace \bibemph{Готовность к сотрудничеству.} Одним из огромных препятствий на пути развития человеческого общества является конфликт между интересами и благополучием больших, более социализированных человеческих групп и менее многочисленных противоположно настроенных асоциальных человеческих объединений, не говоря уже об отдельных антиобщественно настроенных индивидуумах.
\vs p081 6:35 Ни одна национальная цивилизация не будет долговечной, если ее образовательные методы и религиозные идеалы не вдохновляют на высший тип разумного патриотизма и любовь к своему народу. Без подобного разумного патриотизма и культурной солидарности все нации склонны к распаду вследствие провинциальной зависти и местнического эгоизма.
\vs p081 6:36 Сохранение мировой цивилизации зависит от людей, которые учатся жить вместе в мире и братстве. Без эффективного координирования промышленной цивилизации угрожают опасности сверхспециализации: однообразие, узость и тенденция к порождению недоверия и зависти.
\vs p081 6:37 \pc \ublistelem{13.}\bibnobreakspace \bibemph{Эффективное и мудрое руководство.} Многое, очень многое в цивилизации зависит от вдохновенного и эффективного духа сотрудничества. При подъеме большой тяжести от десяти человек не намного больше пользы, чем от одного, если они не поднимают тяжесть вместе --- все в один и тот же момент. Подобное взаимодействие --- социальное сотрудничество --- зависит от руководства. Культурные цивилизации прошлого и настоящего были основаны на разумном сотрудничестве граждан с мудрыми и прогрессивными лидерами; и до тех пор, пока человек не разовьется до более высоких уровней, цивилизация будет продолжать зависеть от мудрого и энергичного руководства.
\vs p081 6:38 Высокие цивилизации --- это плод разумного соотношения материального богатства, интеллектуального величия, нравственных ценностей, способности к социальным контактам и понимания космоса.
\vs p081 6:39 \pc \ublistelem{14.}\bibnobreakspace \bibemph{Социальные перемены.} Общество не есть божественное установление; это --- явление поступательной эволюции; и развитие цивилизации всегда задерживается, когда ее лидеры не способны произвести те перемены в организации общества, которые необходимы для того, чтобы идти в ногу с научными достижениями эпохи. Однако при всем при том нельзя ни отказываться от вещей на том лишь основании, что они стары, ни безоговорочно принимать какую\hyp{}либо идею, только лишь потому, что она свежа и нова.
\vs p081 6:40 Человек не должен бояться экспериментировать с механизмами структуры общества. Однако подобные предприятия в области культурного регулирования должны всегда контролироваться теми, кто полностью сведущ в истории общественной эволюции, причем эти новаторы должны пользоваться мудрыми советами тех, кто обладал практическим опытом в областях, которые становятся предметом социальных или общественных экспериментов. \bibemph{Ни одно большое общественное или экономическое изменение не должно предприниматься резко.} Время необходимо для всех человеческих перемен, какими бы они ни были --- физическими, социальными или экономическими. Мгновенными могут быть только нравственные и духовные перемены, но даже они для полной отработки своих материальных и социальных последствий требуют времени. Расовые идеалы --- вот главная опора и гарантия в критическое время, когда цивилизация совершает переход с одного уровня на другой.
\vs p081 6:41 \pc \ublistelem{15.}\bibnobreakspace \bibemph{Предотвращение распада во время переходного периода.} Общество является порождением сменяющих друг друга эпох проб и ошибок; оно есть то, что осталось после селективных корректировок и перестроек на последовательных этапах векового восхождения человечества с животных на человеческие уровни планетарного положения. Великой опасностью для всякой цивилизации --- в любой момент времени --- является угроза распада во время перехода от установившихся методов прошлого к новым и лучшим, но непроверенным порядкам будущего.
\vs p081 6:42 Для прогресса крайне необходимо управление. Для выживания наций требуются мудрость, понимание и предвидение. Ничто и никогда по\hyp{}настоящему не угрожает цивилизации, пока она не начинает утрачивать способное руководство. И число таких мудрых руководителей никогда не превышало одного процента от общей численности населения.
\vs p081 6:43 Поднимаясь по этим ступеням эволюционной лестницы, цивилизация и оказалась там, где могли пробудиться те могучие силы, которые достигли апогея в стремительном развитии культуры двадцатого века. И лишь придерживаясь этих основ, человек может надеяться на сохранение современных цивилизаций, обеспечив их непрерывное развитие и непременное выживание.
\vs p081 6:44 \pc Такова суть долгой, долгой борьбы народов земли за установление цивилизации со времен Адама. Современная культура есть конечный результат этой потребовавшей огромных усилий эволюции. До изобретения книгопечатания прогресс шел относительно медленно, так как одно поколение не могло быстро извлекать пользу из достижений своих предшественников. Однако теперь человеческое общество стремительно движется вперед под действием энергии, аккумулированной за все века, через которые цивилизация прошла в борьбе.
\vsetoff
\vs p081 6:45 [Под покровительством Архангела Небадона.]
