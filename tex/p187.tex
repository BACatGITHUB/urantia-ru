\upaper{187}{Распятие}
\vs p187 0:1 После того, как двое разбойников были доставлены, солдаты под командой центуриона зашагали к месту распятия. Центурион, которому подчинялись двенадцать солдат, был тем самым командиром, что предыдущей ночью вел римских солдат к Гефсимании арестовывать Иисуса. У римлян было правило приставлять по четыре солдата к каждому человеку, подлежащему распятию. Двоих разбойников, как обычно, бичевали перед тем, как повести на распятие, но Иисуса телесным наказаниям больше не подвергали; командир, наверняка, подумал, что его достаточно бичевали еще до вынесения приговора.
\vs p187 0:2 Оба вора, которых должны были распять вместе с Иисусом, были сообщниками Вараввы, и их предали бы смерти вместе с предводителем, если бы Пилат согласно еврейскому обычаю не освободил его по случаю Пасхи. Таким образом, Иисуса распяли вместо Вараввы.
\vs p187 0:3 То, что сейчас предстоит свершить Иисусу --- принять смерть на кресте, --- он делает по собственной доброй воле. Предрекая этот опыт, он говорил: <<Отец любит и поддерживает меня, потому что я добровольно готов отдать свою жизнь. Но я обрету ее снова. Никто не отнимает у меня жизнь --- я отдаю ее сам. В моей власти лишиться жизни, и в моей власти обрести ее. Я получил это веление от моего Отца>>.
\vs p187 0:4 Было почти девять часов утра, когда солдаты повели Иисуса из претории в сторону Голгофы. За Иисусом следовало много тайно ему сочувствовавших, но большинство из более чем двухсот человек составляли его враги и праздные зеваки, просто желающие поглазеть на ужасающее зрелище распятия. Из еврейских старейшин лишь несколько человек отправились смотреть, как Иисус умрет на кресте. Зная, что Пилат приговорил к смерти и передал его римским солдатам, они на совете в храме принялись обсуждать, как поступить с его последователями.
\usection{1.\bibnobreakspace На пути к Голгофе}
\vs p187 1:1 Выходя со двора претории, солдаты возложили поперечную перекладину от креста на плечи Иисуса. Было принято заставлять приговоренного самого нести перекладину к месту распятия. Приговоренный нес не весь крест, а только короткую перекладину. Длинные прямые брусья для трех крестов уже были доставлены на Голгофу, и к моменту прихода солдат и приговоренных были прочно врыты в землю.
\vs p187 1:2 Согласно обычаю, командир во главе отряда нес небольшие белые дощечки, на которых углем были написаны имена преступников и характер преступлений, за которые им вынесен приговор. Для двух воров у центуриона были таблички с их именами, ниже которых было написано лишь одно слово: <<Разбойник>>. По обычаю, после того, как жертву прибивали к перекладине и поднимали до положенного места на вертикальном брусе, эту табличку прибивали к верхней части креста, прямо над головой у преступника, чтобы все присутствующие могли знать, за какое преступление распяли приговоренного. Надпись, которую нес центурион, чтобы прибить к кресту Иисуса, была написана самим Пилатом на латинском, греческом и арамейском языках и гласила: <<Иисус Назорей --- царь иудейский>>.
\vs p187 1:3 Некоторые из представителей еврейских властей, присутствовавшие в тот момент, когда Пилат писал эту надпись, решительно протестовали против того, что Иисус именуется <<царем иудейским>>. Но Пилат напомнил им, что такая формулировка была частью обвинения, приведшего к вынесенному приговору. Когда евреи поняли, что не смогут переубедить Пилата, они стали просить, чтобы надпись хотя бы была изменена и гласила: Он говорил: <<Я --- царь иудейский>>. Но Пилат был непреклонен; он не пожелал вносить изменений в написанное. На все продолжающие просьбы он лишь отвечал: <<Что я написал, то написал>>.
\vs p187 1:4 Обычно было принято следовать к Голгофе по самой длинной дороге, чтобы побольше людей могли увидеть приговоренного преступника, но в этот день пошли самым коротким путем до Дамасских ворот, от которых дорога шла на север, и по этой дороге они быстро дошли до Голгофы, установленного места распятий в Иерусалиме. За Голгофой располагались виллы богачей, а по другую сторону от дороги находилось богатое еврейское кладбище.
\vs p187 1:5 \P\ Не в обычае евреев было распинать преступников. И греки, и римляне переняли этот вид казни у финикийцев. Даже Ирод, при всей его жестокости, не прибегал к распятию. Римляне никогда не распинали римского гражданина; только рабы и покоренные народы подвергались этой позорной казни. Во время осады Иерусалима, всего через сорок лет после распятия Иисуса, вся Голгофа была уставлена тысячами и тысячами крестов, на которых изо дня в день погибал цвет еврейской нации. Поистине, жуткие всходы от посева сего дня.
\vs p187 1:6 \P\ Когда процессия с осужденными проходила по узким улицам Иерусалима, многие добросердечные еврейские женщины, которые раньше слышали от Иисуса слова ободрения и сострадания и знали о его жизни, посвященной исполненному любви служению, не могли сдержать слез при виде того, как его ведут на постыдную казнь. Когда он проходил мимо, многие эти женщины плакали и рыдали. А когда некоторые из них даже решились идти подле него, Учитель, обратившись к ним, сказал: <<Дщери иерусалимские, не плачьте обо мне, но плачьте о себе и о детях ваших. Мой труд почти завершен --- скоро я отправляюсь к Отцу моему --- но грядут для Иерусалима времена ужасных бедствий. Смотрите, приходят дни, в которые скажете: >>Блаженны неплодные и те, чья грудь никогда не вскармливала их младенца. В те дни вы будете молить камни с гор упасть на вас, чтобы избавить вас от ужасов ваших бедствий<<.
\vs p187 1:7 От этих женщин Иерусалима действительно требовалось мужество, чтобы открыто выражать сочувствие к Иисусу, ибо законом строго\hyp{}настрого запрещалось проявлять доброжелательность к тому, кого ведут на распятие. Толпе дозволялось глумиться, дразнить и осмеивать осужденного, но не разрешалось выражать никакого сочувствия. Хотя Иисус и испытывал признательность за выражение сочувствия в этот горький час, когда его друзья скрывались, он не хотел, чтобы эти добросердечные женщины навлекли на себя немилость властей, осмелившись выказать сострадание в его адрес. Даже в этот момент Иисус мало думал о себе, но только о грядущих страшных днях трагедии Иерусалима и всей еврейской нации.
\vs p187 1:8 Пока Учитель брел к месту распятия, он очень устал; он был почти без сил. Он ничего не ел и не пил с Последней Вечери в доме у Ильи Марка; и ему ни минуты не давали поспать. Кроме того, вплоть до момента вынесения приговора непрерывно одно за другим следовали судебные разбирательства, не говоря уж об унизительных наказаниях и вызванных ими физических страданиях и потери крови. И все это сопровождалось величайшей душевной болью, сильнейшим духовным напряжением и чувством неизбывного человеческого одиночества.
\vs p187 1:9 Иисус шел, с трудом неся перекладину, и вскоре после того, как они миновали ворота и вышли из города, физические силы на мгновение покинули его, и он упал под тяжестью своей ноши. Солдаты стали кричать на него и бить его ногами, но он не мог встать. При виде этого центурион, зная о том, что уже претерпел Иисус, приказал солдатам прекратить. Затем он велел прохожему, некоему Симону Киринеянину, снять перекладину с плеч Иисуса и нести ее весь остальной путь до Голгофы.
\vs p187 1:10 \P\ Этот Симон проделал долгий путь из Кирены, расположенной в Северной Африке, чтобы побывать на праздновании Пасхи. Он остановился вместе с другими киринеянами за пределами городских стен поблизости от города и направлялся в город на проходившие в храме службы, когда римский центурион приказал ему нести перекладину Иисуса. Симон остался и находился там все последние часы жизни Учителя на кресте, разговаривая со многими его друзьями и с его врагами. После воскресения еще до своего ухода из Иерусалима он стал бестрепетным верующим в евангелие царства, а когда вернулся домой, то привел в царство небесное и свою семью. Оба его сына, Александр и Руф, стали очень деятельными проповедниками нового евангелия в Африке. Но Симон так никогда и не узнал, что Иисус, чью ношу он нес, и еврейский наставник, что некогда помог его пострадавшему сыну, это один и тот же человек.
\vs p187 1:11 \P\ В начале десятого процессия с осужденными на казнь добралась до Голгофы, и теперь римским солдатам предстояло прибить гвоздями к крестам двух разбойников и Сына Человеческого.
\usection{2.\bibnobreakspace Распятие}
\vs p187 2:1 Солдаты сначала веревками привязали к перекладине руки Учителя, затем прибили их гвоздями. Подняв перекладину на столб, они прочно прибили ее к вертикальной части креста, затем привязали его ноги и одним длинным гвоздем прибили к столбу обе ступни. На столбе на соответствующей высоте был большой деревянный штырь, своего рода седло, удерживающее вес тела. Крест был невысокий, и ноги Учителя находились всего лишь примерно в трех футах от земли. Поэтому он мог слышать все высказываемые в его адрес насмешки и отчетливо видеть выражение лиц всех тех, кто так безрассудно осмеивали его. И все присутствующие тоже легко могли слышать все, что говорил Иисус в часы томительной пытки и медленной смерти.
\vs p187 2:2 С тех, кого предстояло распять, было принято снимать всю одежду, но так как евреи решительно возражали против полного обнажения человеческого тела прилюдно, римляне всегда предоставляли набедренную повязку всем тем, кого распинали у Иерусалима. Таким образом, после того, как с Иисуса сняли одежды, его облачили именно таким образом прежде, чем распять на кресте.
\vs p187 2:3 К распятию прибегали как к жестокой и вызывающей долгие мучения каре, поскольку жертва иногда не умирала несколько дней. В Иерусалиме общество было крайне отрицательно настроено против распятия, и существовало сообщество еврейских женщин, которые всегда посылали кого\hyp{}нибудь из своих рядов на место распятия, чтобы предложить жертве вина с наркотическими снадобьями, уменьшающими страдания. Но когда Иисус попробовал это болеутоляющее вино, как ни сильна была его жажда, он не стал его пить. Учитель предпочел сохранять свое человеческое сознание до самого конца. Он пожелал встретить смерть, пусть даже в этой жестокой и бесчеловечной форме, и преодолеть ее, претерпев по доброй воле, покорно и в полной мере весь человеческий опыт.
\vs p187 2:4 Перед тем, как распять Иисуса, распяли двоих разбойников, которые непрерывно проклинали и плевали на своих палачей. Когда же к перекладине прибивали Иисуса, единственными его словами были: <<Отче, прости им, ибо не ведают, что творят>>. Он не мог бы так милосердно и заботливо заступаться за своих палачей, если бы подобные чувства самоотверженной любви не лежали в основе всей его жизни, исполненной беззаветного служения. Идеи, движущие мотивы и стремления всей жизни наиболее полно раскрываются в критические моменты.
\vs p187 2:5 После того, как Учителя распяли на кресте, центурион прибил у него над головой табличку с надписью на трех языках: <<Иисус Назорей --- царь иудейский>>. Евреи были в ярости от этого, как они полагали, оскорбления. Но Пилат был раздражен их непочтительным поведением; он чувствовал, что его заставили бояться и унизили, и прибег к такой мелкой мести. Он мог написать <<Иисус, бунтарь>>. Но он хорошо знал, какое отвращение иерусалимские евреи питают даже к самому названию <<Назарет>>, и был полон решимости именно таким образом унизить их. Он также понимал, что евреи будут больно уязвлены, увидев, что этот казненный галилеянин именуется <<царем иудейским>>.
\vs p187 2:6 Многие из еврейских правителей, узнав о том, как Пилат попытался высмеять их, поместив эту надпись на кресте Иисуса, поспешили на Голгофу, но так и не решились даже попытаться снять ее, поскольку на страже стояли римские солдаты. Не имея возможности снять табличку, эти правители смешались с толпой и изо всех сил старались вызвать насмешки и издевки, чтобы никто серьезно не воспринял эту надпись.
\vs p187 2:7 Апостол Иоанн с матерью Иисуса Марией, Руфью и Иудой добрались до места казни сразу после того, как Иисуса распяли на кресте, именно в тот момент, когда центурион прибивал табличку над головой у Учителя. Иоанн единственный из апостолов присутствовал при распятии, но даже он пробыл там не все время, поскольку, приведя на место казни мать Иисуса, вскоре побежал в Иерусалим за своей матерью и ее друзьями.
\vs p187 2:8 Когда Иисус увидел мать, Иоанна и своих братьев и сестер, он улыбнулся, но ничего не сказал. Тем временем четверо солдат, поставленных у креста Учителя, по обычаю, поделили между собой его одежду: один взял сандалии, другой --- головную повязку, третий --- пояс, четвертый --- плащ. После этого остался еще хитон --- одеяние, сделанное из цельного куска материи и длиной почти до колен, --- который предстояло поделить, разрезав на четыре части, но когда солдаты увидели, какое это необыкновенное одеяние, они решили разыграть его, бросив жребий. Иисус смотрел, как они делят его одежду, а бездумная толпа глумилась над ним.
\vs p187 2:9 \P\ Хорошо, что одеждой Учителя завладели римские солдаты. Иначе, если бы эта одежда попала в руки его последователей, у тех появился бы соблазн сделать из нее реликвию для суеверного почитания. Учитель хотел, чтобы у его последователей не осталось ничего материального, связанного с его жизнью на земле. Он хотел оставить человечеству только память о человеческой жизни, посвященной высокому духовному идеалу исполнения воли Отца.
\usection{3.\bibnobreakspace Очевидцы распятия}
\vs p187 3:1 В эту пятницу около половины десятого утра Иисус был распят на кресте. К одиннадцати часам собралось уже больше тысячи человек, чтобы посмотреть на зрелище распятия Сына Человеческого. На протяжении всех этих ужасных часов невидимое воинство вселенной стояло, в молчании взирая на необыкновенное явление, как Творец умирает смертью творения, причем самой позорной смертью приговоренного преступника.
\vs p187 3:2 Во время распятия к кресту время от времени подходили Мария, Руфь, Иуда, Иоанн, Саломея (мать Иоанна) и преданные верующие женщины, в том числе, Мария --- жена Клеопы и сестра матери Иисуса, Мария Магдалина и Ревекка, жившая некогда в Сепфорисе. Они и остальные друзья Иисуса хранили молчание, видя его величайшее терпение и стойкость, и неотрывно смотрели на его мучительные страдания.
\vs p187 3:3 Многие из проходивших мимо качали головами и осуждали его, говоря: <<Ты, который собирался разрушить храм и вновь построить его в три дня, спаси\hyp{}ка себя самого. Если ты Сын Бога, что же ты не сойдешь со своего креста?>> Подобным же образом злословили его некоторые из правителей евреев, говоря: <<Других спасал, а самого себя не может спасти>>. Другие говорили: <<Если ты царь евреев, сойди с креста, и мы уверуем в тебя>>. А потом они еще сильнее стали осмеивать его и говорить: <<Он уповал на Бога, что Бог спасет его. Он даже утверждал, что он Сын Бога --- взгляните теперь на него --- распятого между двумя ворами>>. Даже эти двое воров тоже бранили его и бросали ему упреки.
\vs p187 3:4 Поскольку Иисус оставлял их насмешки без ответа и близился полдень этого особого дня приготовления к Пасхе, к половине двенадцатого насмехающаяся и глумящаяся толпа в основном разошлась; на месте распятия осталось не больше пятидесяти человек. Солдаты собрались поесть и выпить своего дешевого кислого вина, так как им предстояло еще долго стоять на страже возле казненных. Выпив вина, они насмешливо подняли тост за Иисуса, сказав: <<Привет и удачи тебе! За царя иудейского>>. И они были поражены спокойным отношением Учителя к их насмешкам и издевательствам.
\vs p187 3:5 Увидев, что они едят и пьют, Иисус посмотрел ни них и сказал: << Жажду>>. Услышав слова Иисуса, что он хочет пить, командир стражников смочил вином из своей бутылки губку, служившую затычкой, и поднял ее на острие копья к Иисусу, чтобы тот мог смочить свои пересохшие губы.
\vs p187 3:6 Иисус стремился жить, не прибегая к своей сверхъестественной силе, и точно так же он решил и умереть на кресте как простой смертный. Он жил как человек и пожелал умереть как человек --- исполняя волю Отца.
\usection{4.\bibnobreakspace Вор на кресте}
\vs p187 4:1 Один из разбойников злословил Иисуса, говоря: <<Если ты Сын Бога, что же ты не освободишь себя и нас?>> Но когда он упрекнул Иисуса, то другой вор, много раз раньше слышавший, как Учитель учил, сказал: <<Неужели у тебя нет даже страха перед Богом? Разве ты не видишь, что мы заслуженно страдаем за свои деяния, но этот человек страдает незаслуженно? Лучше бы нам искать прощения своих грехов и спасения для своих душ>>. На эти слова вора Иисус повернулся к нему и одобрительно улыбнулся. Увидев обращенное к нему лицо Иисуса, преступник собрал все свое мужество, раздул теплящийся огонек своей веры и сказал: <<Господи, помяни меня, когда придешь в царство твое>>. И тогда Иисус сказал: <<Истинно, истинно говорю тебе, ныне же будешь со мною в Раю>>.
\vs p187 4:2 Учитель нашел в себе силы, несмотря на острую боль человеческой смерти, выслушать религиозную исповедь уверовавшего разбойника. Когда этот вор протянул руку, прося спасения, он его нашел. Много раз до этого он был склонен поверить в Иисуса, но только в эти последние мгновения, пребывая в сознании, он всем сердцем обратился к учению Учителя. Увидев, как Иисус встретил смерть на кресте, этот разбойник уже не мог преодолеть убеждения, что сей Сын Человеческий действительно Сын Бога.
\vs p187 4:3 \P\ Во время этого обращения в веру и принятия Иисусом вора в царство апостол Иоанн отсутствовал, поскольку ушел в город, чтобы привести на место распятия свою мать и ее друзей. Лука впоследствии слышал эту историю от обращенного в веру римского командира стражников.
\vs p187 4:4 Апостол Иоанн поведал о распятии так, как он помнил это событие через две трети столетия после того, как оно произошло. Другие повествования были основаны на рассказе командовавшего стражниками римского центуриона, который, благодаря тому, что увидел и услышал, впоследствии поверил в Иисуса и стал полноправным членом братства царства небесного на земле.
\vs p187 4:5 \P\ Молодого человека, раскаявшегося разбойника, втянули в свое время в полную насилия и беззакония жизнь те, кто превозносили разбойные действия как действенный патриотический протест против политического гнета и социальной несправедливости. И такого рода взгляды в сочетании с жаждой приключений приводили к тому, что многие, в остальном, добропорядочные юноши участвовали в этих захватывающих разбойных делах. Этот юноша смотрел на Варавву как на героя. Теперь он понял, что ошибался. Здесь на кресте рядом с собой он увидел действительно выдающегося человека, подлинного героя. Это был герой, который возжег в нем пыл и внушил высочайшие идеи морального самоуважения, оживил все его идеалы отваги, мужества и храбрости. Созерцание Иисуса вызвало в его сердце всепоглощающее чувство любви, преданности и подлинного величия.
\vs p187 4:6 И если бы любой другой человек из этой глумящейся толпы почувствовал в своей душе рождение веры и воззвал бы к милосердию Иисуса, то он был бы принят с такой же любовью и участием, какие были проявлены к этому уверовавшему разбойнику.
\vs p187 4:7 \P\ Вскоре после того, как раскаявшийся вор услышал обещание Учителя, что в свое время они встретятся в Раю, из города вернулся Иоанн со своей матерью и в сопровождении почти дюжины верующих женщин. Иоанн встал возле Марии, матери Иисуса, поддерживая ее. По другую сторону стоял ее сын Иуда. Было полуденное время, Иисус посмотрел на них и сказал своей матери: <<Женщина, се сын твой!>> И, обращаясь к Иоанну, он сказал: <<Сын мой, се мать твоя!>> А затем он обратился к ним обоим, сказав: <<Я желаю, чтобы вы удалились от этого места>>. И Иоанн с Иудой повели Марию прочь от Голгофы. Иоанн отвел мать Иисуса в тот дом, где он обитал в Иерусалиме, и затем поспешил обратно к месту распятия. После Пасхи Мария вернулась в Вифсаиду, где и прожила в доме у Иоанна всю свою оставшуюся естественную жизнь. Мария не прожила и года после смерти Иисуса.
\vs p187 4:8 После ухода Марии остальные женщины отступили на небольшое расстояние и оставались возле Иисуса до самой его смерти на кресте, и находились там до тех пор, пока тело Учителя не было снято с креста для погребения.
\usection{5.\bibnobreakspace Последний час на кресте}
\vs p187 5:1 Вскоре после двенадцати часов небо потемнело от песчаной тучи, хотя еще не наступило то время года, когда бывали такие природные явления. Жители Иерусалима знали, что это признак приближения одной из тех песчаных бурь с иссушающим ветром, которые шли из Аравийской пустыни. К часу дня небо потемнело настолько, что скрылось солнце, и остатки толпы поспешили обратно в город. Когда вскоре после этого Учитель скончался, на горе оставалось меньше тридцати человек: только тринадцать римских солдат и примерно пятнадцать верующих. В этой группе верующих, кроме двоих: Иуды, брата Иисуса, и Иоанна Зеведеева, вернувшегося к месту распятия незадолго перед тем, как Учитель скончался, были одни женщины.
\vs p187 5:2 В начале второго среди сгущающейся тьмы свирепой песчаной бури Иисус стал терять человеческое сознание. Последние слова милосердия, прощения и напутствия были им уже сказаны. Его последнее желание --- касающееся заботы о его матери --- было уже высказано. В этот час близящейся смерти человеческое сознание Иисуса обратилось к повторению многих текстов из еврейского Писания, в особенности --- Псалмов. Последняя мысль в человеческом сознании Иисуса была связана с мысленным повторением той части Книги псалмов, которая известна сейчас как двадцатый, двадцать первый и двадцать второй псалмы. Его губы часто шевелились, хотя он был слишком слаб, чтобы вслух произносить слова тех текстов, которые он так хорошо знал наизусть и которые звучали у него в уме. Лишь несколько раз стоявшие поблизости уловили какие\hyp{}то слова, такие, как: <<Я знаю, Господь спасет своих помазанников>>, <<Твоя рука настигнет всех моих врагов>> и <<Боже мой, Боже мой, для чего ты меня оставил?>> У Иисуса ни на миг не было ни малейшего сомнения в том, что он жил в соответствии с волей Отца; и он никогда не сомневался, что сейчас расстается со своей жизнью во плоти в соответствии с волей своего Отца. У него не было чувства, что Отец оставил его; он просто повторял в своем угасающем сознании многие тексты из Писания, в том числе и этот двадцать второй псалом, начинающийся словами <<Боже мой, Боже мой, для чего ты меня оставил?>> И случилось так, что это была одна из трех фраз, прозвучавших достаточно ясно, чтобы их могли расслышать стоявшие поблизости.
\vs p187 5:3 \P\ Последняя просьба, обращенная Иисусом\hyp{}человеком к своим собратьям, прозвучала около половины второго, когда он сказал вторично: <<Жажду>>, и опять командир стражников увлажнил его губы той же самой губкой, смоченной в кислом вине, которое в те дни часто называли уксусом.
\vs p187 5:4 \P\ Песчаная буря усиливалась, и небеса становились все темнее. Но солдаты и немногочисленные верующие по\hyp{}прежнему стояли поблизости. Солдаты прижались к земле возле креста и сбились в кучу, защищаясь от больно бьющего песка. Мать Иоанна и другие находились на небольшом расстоянии, где их отчасти защищала нависающая скала. Когда Учитель испустил последний вздох, у подножия его креста стояли Иоанн Зеведеев, брат Иуда, сестра Руфь, Мария Магдалина и Ревекка из Сефориса.
\vs p187 5:5 Было почти три часа, когда Иисус громким голосом воскликнул: <<Свершилось! Отче, в руки твои предаю дух мой!>> И сказав так, он, преклонив главу, предал дух. Увидев, как умер Иисус, римский центурион ударил себя в грудь и сказал: <<Это был поистине праведный человек; воистину, должно быть, он был Сыном Божиим>>. И с того часа он стал верить в Иисуса.
\vs p187 5:6 \P\ Иисус умер царственно --- так же, как и жил. Он свободно признал свой царский титул и оставался хозяином положения на протяжении всего этого трагического дня. Он добровольно пошел на свою постыдную казнь после того, как обеспечил безопасность своих избранных апостолов. Он благоразумно удержал Петра от применения насилия, которое могло привести к непредсказуемым последствиям, и сделал так, чтобы Иоанн смог быть возле него до самого конца его человеческого существования. Он открыл свою истинную природу жестокому синедриону и как Сын Бога напомнил Пилату об источнике своей суверенной власти, которой он обладал как Сын Бога. Он шел к Голгофе, неся перекладину от своего собственного креста, и закончил свое исполненное любви пришествие, передав Райскому Отцу свой дух, обретенный им как смертным человеком. После такой жизни --- и при такой смерти --- Учитель мог с полным правом сказать: <<Свершилось>>.
\vs p187 5:7 \P\ Так как это был день приготовления к Пасхе и к субботе, евреи не хотели оставить тел на кресте на Голгофе. Поэтому они обратились к Пилату с просьбой чтобы всем троим перебить голени и покончить с ними, чтобы до захода солнца их можно было снять с крестов и бросить в могилы для преступников. Выслушав эту просьбу, Пилат тотчас же отправил троих солдат перебить ноги Иисусу и двоим разбойникам и покончить с ними.
\vs p187 5:8 Придя на Голгофу, солдаты поступили, как было приказано, с двумя разбойниками, но, к своему большому удивлению, обнаружили, что Иисус уже мертв. Однако, чтобы убедиться в его смерти, один из солдат копьем проткнул ему левый бок. Хотя жертвы распятия очень часто оставались живыми на кресте даже два или три дня, захлестнувшие Иисуса эмоциональные страдания и жестокие душевные муки положили конец его человеческой жизни во плоти меньше, чем через пять с половиной часов.
\usection{6.\bibnobreakspace После распятия}
\vs p187 6:1 В кромешной тьме песчаной бури около половины четвертого Давид Зеведеев отправил последних вестников с известием о смерти Учителя. Последних своих гонцов он послал в дом Марфы и Марии в Вифанию, где, как он полагал, остановилась мать Иисуса вместе с остальными членами своей семьи.
\vs p187 6:2 После смерти Учителя Иоанн отправил женщин в сопровождении Иуды в дом Илии Марка, где они и провели всю субботу. Сам Иоанн, которого к этому времени уже хорошо знал римский центурион, оставался на Голгофе до тех пор, пока туда не пришли Иосиф и Никодим с приказом от Пилата, разрешающим им забрать тело Иисуса.
\vs p187 6:3 Так для обширной вселенной закончился день трагедии и горя, мириады разумов содрогнулись от ужасающего зрелища распятия человеческого воплощения их любимого Владыки; они были ошеломлены этим проявлением бессердечия смертных и людской порочности.
