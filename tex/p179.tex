\upaper{179}{Тайная вечеря}
\author{Комиссия срединников}
\vs p179 0:1 В четверг днем, напоминая Учителю о приближающейся Пасхе и спрашивая его о планах ее празднования, Филипп имел в виду пасхальную вечерю, которую надлежало вкушать вечером на следующий день, то есть в пятницу. По обычаю, приготовления к празднованию Пасхи начинались не позднее полудня предыдущего дня. А поскольку евреи считали, что день начинается с захода солнца, это означало, что субботнюю пасхальную вечерю следовало вкушать в пятницу ночью, незадолго до наступления полуночи.
\vs p179 0:2 Поэтому апостолы были в полном недоумении и не смогли понять слов Учителя о том, что праздновать Пасху они будут на один день раньше. Апостолы (по крайней мере некоторые из них), думали, что Иисус знает о том, что его возьмут под арест еще до пасхальной вечери в пятницу ночью, и поэтому собирает их на особую вечерю в этот четверг. Другие полагали, что это --- просто особое событие, которое должно предшествовать празднованию Пасхи.
\vs p179 0:3 Апостолы знали, что другие Пасхи Иисус праздновал без агнца; они знали, что сам Иисус не участвовал ни в одной жертвенной службе еврейской религиозной системы. Он много раз вкушал от пасхального агнца как гость, однако всегда, когда хозяином был он, агнца не подавали. Если бы апостолы увидели, что агнца нет даже в пасхальную ночь, это не вызвало бы у них большого удивления, а поскольку сия вечеря была устроена на день раньше, отсутствие агнца вообще не навело их ни на какие мысли.
\vs p179 0:4 Выслушав приветствия отца и матери Иоанна Марка, апостолы сразу пошли в верхнюю комнату, а Иисус остался беседовать с семьей Марка.
\vs p179 0:5 О том, что Учитель собирается отмечать праздник только со своими двенадцатью апостолами, было известно заранее; поэтому никаких слуг, которые бы им прислуживали, не было.
\usection{1. Жажда привилегий}
\vs p179 1:1 Когда Иоанн Марк проводил апостолов наверх, они увидели большую и просторную комнату, полностью подготовленную для вечери, и отметили, что на одном конце стола приготовлены хлеб, вино, вода и травы. Этот длинный стол, кроме того его конца, где стояли хлеб и вино, был окружен тринадцатью ложами для возлежания, как раз такими, какими пользовались во время празднования Пасхи в зажиточных еврейских домах.
\vs p179 1:2 Войдя в верхнюю комнату, двенадцать апостолов заметили сразу за дверью кувшины с водой, умывальницы и полотенца для омовения их пыльных ног; а поскольку никакого слуги, который мог бы оказать им эту услугу, не было, апостолы, как только Иоанн Марк их оставил, посмотрели друг на друга и каждый стал думать про себя: «Кто омоет мне ноги?» При этом каждый из них также думал, что не его будут присутствующие считать за их слугу.
\vs p179 1:3 Стоя и обдумывая это про себя, они осмотрели, как расположены места у стола, и обратили внимание, что справа от более высокого ложа хозяина расположено одно ложе, а одиннадцать расставлены вокруг стола так, что последнее из них было напротив второго почетного места по правую руку от места хозяина.
\vs p179 1:4 С минуты на минуту ожидая появления Иисуса, апостолы не знали, занять ли себе места самим или дождаться прихода Учителя и положиться на его решение, кому где сидеть. Пока же они мешкали, Иуда подошел к почетному месту слева от места хозяина и дал понять, что он намерен возлежать здесь в качестве привилегированного гостя. Этот поступок Иуды сразу вызвал горячий спор между остальными апостолами. Не успел Иуда захватить почетное место, как Иоанн Зеведеев завладел следующим привилегированным местом --- справа от хозяина. Симон Петр же так возмутился присвоением избранных мест Иудой и Иоанном, что, пока остальные разгневанные апостолы смотрели на это, обошел вокруг стола и занял последнее ложе, как раз напротив места, выбранного Иоанном Зеведеевым. Так как другие заняли первые места, Петр решил выбрать последнее, причем сделал это не просто в знак протеста против неподобающей гордыни своих братьев, но с надеждой, что Иисус, придя и увидев его на наименее почетном месте, позовет его пересесть выше и, таким образом, переместит тех, кто посмел возвысить самого себя.
\vs p179 1:5 Поскольку первое и последнее места были заняты, остальные апостолы стали занимать места и располагаться, одни ближе к Иуде, другие ближе к Петру, пока не расселись все. Расположились же они на этих ложах вокруг подковообразного стола в следующем порядке: справа от Учителя --- Иоанн; слева --- Иуда, Симон Зилот, Матфей, Иаков Зеведеев, Андрей, близнецы Алфеевы, Филипп, Нафанаил, Фома и Симон Петр.
\vs p179 1:6 \pc Они собрались праздновать, по крайней мере в духе, событие, которое предшествовало даже Моисею и восходило к временам, когда отцы их были рабами в Египте. Эта вечеря --- их последнее свидание с Иисусом, но даже в такой торжественной обстановке под влиянием Иуды апостолы снова подпадают под власть своей старой склонности --- стремления к славе, привилегиям и личному возвышению.
\vs p179 1:7 \pc Когда Учитель появился в дверях, где он задержался на минуту, они еще продолжали злобные взаимные обвинения, и на лице Иисуса постепенно проступило разочарование. Ничего не сказав, он пошел к своему месту и не стал менять порядок, в котором расселись апостолы.
\vs p179 1:8 Теперь они были готовы к вечере; только оставались не омытыми ноги их, и им явно не хватало хорошего настроения. Когда пришел Учитель, они еще продолжали нелицеприятные замечания друг о друге, не говоря уже о мыслях тех, у кого все же было достало самообладания воздержаться от высказывания своих чувств.
\usection{2. Начало вечери}
\vs p179 2:1 После того, как Учитель подошел к своему месту, в течение нескольких минут не было произнесено ни слова. Иисус окинул всех взором и, сняв напряжение улыбкой, сказал: «Очень желал я есть с вами сию пасху. Еще раз хотел я есть с вами прежде моего страдания и, понимая, что час мой настал, устроил сию вечерю с вами сегодня ночью, ибо, что касается завтрашнего дня, все мы в руках Отца, чью волю я и пришел исполнить. Не буду снова есть с вами, пока не воссядете со мной в царстве, которое Отец мой отдаст мне, когда совершу то, зачем он послал меня в этот мир».
\vs p179 2:2 Смешав вино и воду, апостолы поднесли чашу Иисусу, который принял ее из рук Фаддеуса и держа, произнес благодарственную молитву. Закончив же, Иисус сказал: «Примите чашу сию и разделите между собой; вкусив же от нее, поймите, что я не буду более пить с вами от плода виноградного, ибо эта есть наша последняя вечеря. Когда же снова воссядем так, будет это в грядущем царстве».
\vs p179 2:3 Иисус заговорил с апостолами подобным образом, поскольку знал, что его час настал. Он понимал, что пришло время, когда он должен вернуться к Отцу, и что труд его на земле почти завершен. Учитель знал, что он явил любовь Отца на земле и показал его милосердие к человечеству; что он совершил то, ради чего пришел в мир, получив всю силу и власть равно на небе и на земле. Подобно тому, Иисус знал, что Ииуда Искариот окончательно решился предать его этой ночью в руки врагов. Он вполне сознавал, что это вероломное предательство было делом рук Иуды, но что оно было угодно и Люциферу, Сатане и принцу тьмы Калигастии. Однако Иисус боялся искавших его духовного свержения не больше, чем боялся он тех, кто искал его физической смерти. Учителя беспокоило только одно: безопасность и спасение его избранных последователей. Так, полностью понимая, что Отец все отдал в его власть, Учитель приготовился теперь представить притчу о братской любви.
\usection{3. Омовение ног апостолам}
\vs p179 3:1 По еврейскому обычаю, после вкушения первой пасхальной чаши хозяин вставал из\hyp{}за стола и умывал руки. Позже во время еды и после второй чаши умывать руки вставали все гости. Так как апостолы знали, что Учитель никогда не соблюдал эти обряды ритуального омовения рук, им было очень любопытно узнать, что Иисус собирается делать, когда после того, как они вкусили от первой чаши, он встал из\hyp{}за стола и молча направился к месту рядом с дверью, где находились кувшины с водой, умывальницы и полотенца. И любопытство их переросло в изумление, когда они увидели, что Учитель снял с себя верхнюю одежду, препоясался полотенцем и стал наливать воду в одну из умывальниц для ног. Вообразите возлежал Симон Петр, встал на колени как слуга и приготовился мыть ноги Симона. Когда Учитель преклонил колени, все двенадцать апостолов как один встали на ноги; даже предатель Иуда, и тот на мгновение настолько забыл о своем бесчестии, что поднялся вместе со свже удивление этих двенадцати человек, только недавно отказавшихся омыть ноги друг другу и занятых неподобающим спором о почетных местах за столом, когда они увидели, как Иисус, обойдя никем не занятый край стола, подошел к последнему месту на празднике, где оими собратьями\hyp{}апостолами, выражая удивление, уважение и крайнее изумление.
\vs p179 3:2 Симон Петр стоял, глядя на обращенное к нему лицо своего Учителя. Иисус ничего не сказал; ему и не требовалось ничего говорить. Его поза ясно показывала, что он собирается омыть ноги Симона Петра. Несмотря на слабости своего характера, Петр любил Учителя. Сей галилейский рыбак был первым человеком, который всем сердцем поверил в божественность Иисуса и полностью и публично исповедал эту веру. С тех пор Петр никогда по\hyp{}настоящему не сомневался в божественной природе Учителя. Поскольку Петр так чтил и почитал Иисуса в сердце своем, неудивительно, что душа его негодовала при мысли о том, что Иисус встал перед ним на колени как покорный слуга и, как раб, предложил ему умыть ноги. Немного погодя, когда к Петру вернулось достаточно самообладания, чтобы обратиться к Учителю, он высказал то, что чувствовал в сердце каждый из его собратьев\hyp{}апостолов.
\vs p179 3:3 После нескольких минут великого смущения Петр сказал: «Учитель, действительно ли ты хочешь умыть мои ноги?» Тогда, посмотрев Петру в глаза, Иисус сказал: «Возможно, ты не до конца понимаешь, что я собираюсь сделать, но уразумеешь после смысл всего этого». Тогда Симон Петр, глубоко вздохнул и сказал: «Учитель, ты не умоешь ног моих вовек!» И каждый из апостолов кивнул головой, одобряя твердое заявление Петра, отказавшегося позволить Иисусу таким образом унизить себя перед ними.
\vs p179 3:4 Драматизм этой необычной сцены вначале тронул сердце даже Иуды Искариота; однако, когда его тщеславный ум рассудил о происходящем, он заключил, что этот уничижительный жест окончательно доказал, что Иисус никогда не станет избавителем Израиля и что он не ошибся, предав дело Учителя.
\vs p179 3:5 Когда все они, изумленные, стояли, затаив дыхание, Иисус сказал: «Петр, я заявляю: если не умою ног твоих, не будешь иметь части со мною в том, что я готовлюсь совершить». Услышав эти слова, сказанные Иисусом, который продолжал стоять перед ними на коленях, Петр решился слепо покориться желанию того, кого он чтил и любил. По мере того, как Симону Петру становилось ясно, что этому предлагаемому служению придавался некий смысл, определявший будущую связь с делом Учителя, он не только примирился с мыслью о том, что Иисусу следует позволить омыть ему ноги, но и в характерной для себя импульсивной манере воскликнул: «Тогда, Учитель, умой не только мои ноги, но и руки, и голову».
\vs p179 3:6 Приготовившись омывать ноги Петра, Учитель сказал: «Чистому нужно только ноги умыть. Вы, сидящие со мной этой ночью, чисты, но не все. Однако пыль с ног ваших следовало смыть прежде, чем вы сели со мной за трапезу. И кроме того, я исполню это служение вам как притчу, поясняющую смысл новой заповеди, которую я вскоре вам дам».
\vs p179 3:7 Таким же образом Учитель молча обошел вокруг стола и омыл ноги всех двенадцати учеников, не исключая и Иуды. Когда же умыл им ноги и надел одежду свою, Иисус возвратился на место хозяина и, оглядев смущенных апостолов, сказал:
\vs p179 3:8 \pc «Действительно ли понимаете, что я сделал вам? Вы называете меня Учителем и правильно говорите, ибо я точно то. Итак, если Учитель умыл ноги вам, почему вы не хотели умыть ноги друг другу? Какой урок надлежит вам извлечь из сей притчи, в которой Учитель столь охотно исполняет служение, которое братья его не хотят исполнить друг другу? Истинно, истинно говорю: слуга не больше господина своего; и посланник не больше пославшего его. В моей жизни среди вас вы видели путь служения, и блаженны вы, имеющие благодатную смелость служить так. Почему же вы никак не поймете, что секрет величия в духовном царстве совсем не похож на силовые методы материального мира?
\vs p179 3:9 Когда сегодня ночью я вошел в эту комнату, вам мало было гордо отказаться умывать друг другу ноги, но вы еще и начали между собой спорить, кто займет почетные места за моим столом. Таких почестей ищут фарисеи и дети мира сего, но не так должно быть среди посланников царства небесного. Разве не знаете вы, что за столом моим нет привилегированных мест? Разве не понимаете, что я люблю каждого из вас так же, как люблю остальных. Разве не знаете, что ближайшие ко мне места (как люди смотрят на подобные почести), не имеют ничего общего с положением вашим в царстве небесном? Вы знаете, что цари неевреев господствуют над своими подданными, и имеющих такую власть иногда называют благодетелями. Но в царстве небесном не будет так. Желающий быть большим среди вас да будет как меньший; и начальствующий, как служащий. Кто больше: возлежащий за трапезой или служащий? Не принято ли считать, что возлежащий? Однако вы видите, что я среди вас, как служащий. Если вы готовы вместе со мной стать служащими в исполнении воли Отца, то в грядущем царстве воссядете со мной в силе, по\hyp{}прежнему исполняя волю Отца в будущей славе».
\vs p179 3:10 Когда Иисус кончил говорить, близнецы Алфеевы принесли хлеб и вино с горькими травами и пастилу из сухих фруктов --- следующие блюда Тайной Вечери.
\usection{4. Последние слова предателю}
\vs p179 4:1 Несколько минут апостолы ели молча, однако глядя на веселое настроение Учителя постепенно втянулись в разговор, и довольно скоро трапеза проходила так, будто и не было ничего необычного, помешавшего хорошему настроению и всеобщему согласию во время этого необычайного события. Прошло некоторое время, приблизительно в середине второй перемены блюд Иисус посмотрел на них и произнес: «Я уже сказал вам, как желал этой вечери с вами, и, зная, что силы зла вступили в заговор, дабы убить Сына Человеческого, решил вкусить от сей вечери с вами в этой тайной комнате и за день до Пасхи, поскольку завтра ночью к этому времени меня уже не будет с вами. Я неоднократно говорил вам, что должен вернуться к Отцу. Ныне час мой настал, однако вовсе не требовалось, чтобы один из вас предал меня в руки моих врагов».
\vs p179 4:2 Благодаря притче об умывании ног и последовавшей за ней речи Учителя двенадцать апостолов уже лишились большей части своих притязаний и самоуверенности, и услышав эти слова, они стали смотреть друг на друга, спрашивая смущенными голосами: «Не я ли?» Когда же этот вопрос задали все, Иисус сказал: «Хоть я и должен идти к Отцу, совсем не надо было, чтобы один из вас во исполнение воли Отца стал предателем. Таковы плоды зла, сокрытого в сердце не сумевшего возлюбить истину всею душою своею. Как обманчива гордыня ума, предшествующая духовному падению! В течение многих лет бывший другом моим, ныне вкушающий мой хлеб и сейчас обмакивающий со мной в блюдо, будет хотеть предать меня».
\vs p179 4:3 Когда Иисус сказал это, все апостолы стали снова спрашивать: «Не я ли?» И когда Иуда, сидевший слева от Учителя, снова спросил: «Не я ли?», Иисус, обмакнув хлеб в блюдо с травами, дал его Иуде со словами: «Ты сказал». Но остальные не слышали, как Иисус говорил с Иудой. Иоанн, возлежавший по правую руку Иисуса, наклонился и спросил Учителя: «Кто это? Мы должны знать, кто оказался недостоин оказанного ему доверия?» Иисус ответил: «Уже я сказал вам: тот, кому я, обмакнув, подал кусок хлеба». Однако обмакнуть кусок хлеба и подать его сидящему слева было для хозяина настолько естественно, что никто из апостолов не обратил на это никакого внимания, хотя Учитель и говорил так ясно. Иуда же, болезненно осознававший, что смысл слов Учителя связан с его поступком, испугался, как бы его братья также теперь не поняли, что он --- предатель.
\vs p179 4:4 Петр был глубоко взволнован сказанным и, склонившись над столом, обратился к Иоанну: «Спроси у него, кто это, или, если он сказал тебе, скажи мне, кто предатель».
\vs p179 4:5 Иисус же прекратил их перешептывания, сказав: «Я опечален тем, что это зло произошло, и до сего часа надеялся, что сила истины восторжествует над ложью зла, но такие победы не одерживаются без веры искренне возлюбившего истину. Я не сказал бы это на этой нашей последней вечере, но я желаю предостеречь вас об этих печалях и тем самым подготовить к тому, что нас ожидает. Я рассказал вам об этом, потому что желаю, чтобы вы, когда я уйду, помнили, что я знал обо всех этих заговорах и предупреждал вас о предательстве меня. Я делаю все это лишь затем, чтобы вы укрепились и были готовы к искушениям и испытаниям, ожидающим вас впереди».
\vs p179 4:6 Сказав это, Иисус наклонился к Иуде и сказал: «Что решил делать, делай скорее». Услышав эти слова, Иуда встал из\hyp{}за стола и, поспешно покинув комнату, ушел в ночь делать то, что решился совершить. Увидев, как поспешно вышел Иуда после того, как Иисус переговорил с ним, другие апостолы решили, что тот ушел добыть для вечери что\hyp{}нибудь еще либо исполнить какое\hyp{}нибудь другое поручение Учителя, ибо они думали, что Иуда по\hyp{}прежнему держит казну у себя.
\vs p179 4:7 \pc Иисус знал, что удержать Иуду от предательства нельзя было ничем. Он начал с двенадцатью апостолами --- теперь у него было одиннадцать. Из этих апостолов он избрал шестерых, и хотя Иуда был среди тех, кто был избран первозванными апостолами, Учитель все равно принял его и вплоть до самого последнего часа делал все возможное, чтобы освятить и спасти его, точно так же, как трудился он во имя мира и спасения остальных.
\vs p179 4:8 Эта вечеря, с ее столь волнующими и трогательными событиями, была последним призывом Иисуса к предающему Иуде, но призыв этот не помог. Предостережение, сделанное даже самым тактичным образом и выраженное самым доброжелательным тоном, как правило, только усиливает ненависть и разжигает злую решимость до конца исполнить эгоистичные замыслы, когда любовь уже действительно мертва.
\usection{5. Учреждение вечери воспоминания}
\vs p179 5:1 Когда апостолы поднесли Иисусу третью чашу вина, «чашу благословения», тот поднялся с ложа и, взяв чашу в руки, ее благословил и сказал: «Возьмите чашу эту и пейте из нее все. Сия будет чашей в мое воспоминание. Сия есть чаша благословения новой диспенсации благодати и истины. Она будет вам символом дара и служения божественного Духа Истины. Я не буду более пить эту чашу с вами до того дня, когда буду пить вместе с вами новое вино в вечном царстве Отца».
\vs p179 5:2 С глубоким почтением и в полном молчании вкушая от этой чаши благословения, все апостолы чувствовали, что происходит нечто из ряда вон выходящее. Старая Пасха была празднованием перехода их отцов из состояния расового рабства в состояние индивидуальной свободы; теперь же Учитель учреждал новую вечерю воспоминания как символ новой диспенсации, в которой порабощенный индивидуум переходит от оков формализма и эгоизма в духовную радость братства и сообщества освобожденных верующих сынов живого Бога.
\vs p179 5:3 Когда апостолы выпили эту чашу воспоминания, Учитель взял хлеб и, воздав благодарение, преломил его и велел им передать его по кругу, сказав: «Примите сей хлеб воспоминания и ешьте его. Я говорил вам, что я хлеб жизни. И сей хлеб жизни есть единая жизнь Отца и Сына в одном даре. Слово Отца, явленное в Сыне, воистину есть хлеб жизни». Вкусив от хлеба воспоминания, символа живого слова истины, воплощенного в его подобии смертной плоти, все апостолы сели.
\vs p179 5:4 \pc Учреждая эту вечерю воспоминания, Учитель, по своему обыкновению, прибегнул к притчам и символам. Символами он пользовался потому, что хотел учить некоторым великим духовным истинам так, чтобы его последователям было трудно дать его словам точное толкование и определенное значение. Тем самым он пытался воспрепятствовать тому, чтобы последующие поколения формализовали его учение и сковали его духовный смысл мертвыми цепями традиций и догм. Учреждая единственный обряд, или таинство, связанное с миссией всей своей жизни, Иисус постарался \bibemph{намекать} на смысл этого обряда, а не связывать себя \bibemph{точными определениями.} Он не желал разрушать индивидуальное понимание божественного общения учреждением точной его формы; он также не желал ограничивать духовное воображение верующего, ставя его в формальные рамки. Напротив, он пытался освободить рожденную заново душу человека, дав ей крылья радости новой и живой духовной свободы.
\vs p179 5:5 Несмотря на попытку Учителя таким образом установить это новое таинство воспоминания, те, кто стал его последователем в последующие века, сделали все, чтобы воспрепятствовать исполнению его ясного желания, ибо его простая духовная символика той последней ночи во плоти была сведена к точным толкованиям и подчинена почти строго математической установленной формуле. Из всех учений Иисуса ни одно не подвергалось большей формализации традицией.
\vs p179 5:6 Эта вечеря воспоминания, вкушаемая верующими в Сына и знающими Бога, не нуждается в том, чтобы связывать ее символику с какими бы то ни было человеческими пустыми и неверными толкованиями смысла божественного присутствия, ибо во всех подобных случаях Учитель \bibemph{действительно присутствует.} Вечеря воспоминания --- это символическое свидание верующего с Михаилом. Когда вы таким образом становитесь сознающими дух, Сын действительно присутствует, и дух его братается с пребывающей в верующем частицей его Отца.
\vs p179 5:7 \pc По истечении нескольких минут, проведенных апостолами в размышлении, Иисус продолжил свою речь: «Делая это, вспоминайте жизнь, которую я прожил на земле среди вас, и радуйтесь тому, что мне надлежит продолжать жить на земле с вами и через вас служить. Не спорьте между собой, кому быть большим. Будьте все как братья. Точно так же, когда царство приумножится и охватит собой большие массы верующих, воздерживайтесь бороться за первенство или требовать главенства среди таких групп».
\vs p179 5:8 Это великое событие произошло в верхней комнате в доме друга. Ни в вечере, ни в здании не было ничего от священной формы или обрядового посвящения. Вечеря воспоминания была установлена без церковной санкции.
\vs p179 5:9 Установив таким образом вечерю воспоминания, Иисус сказал апостолам: «Всякий раз творите сие в мое воспоминание. Вспоминая же меня, сначала вспоминайте мою жизнь во плоти, вспоминайте, что однажды я был с вами, а затем верой осознайте, что все вы некогда будете вкушать со мной в вечном царстве Отца. Сия есть новая Пасха, которую я оставляю вам, память о моей жизни во плоти, о слове вечной истины; и о моей любви к вам, излиянии моего Духа Истины на всякую плоть».
\vs p179 5:10 Завершая это празднование старой, но бескровной Пасхи и по случаю установления новой вечери воспоминания они вместе пропели сто восемнадцатый псалом.
