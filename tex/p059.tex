\upaper{59}{Эра морской жизни на Урантии}
\author{Носитель Жизни}
\vs p059 0:1 Мы считаем, что история Урантии началась около миллиарда лет назад и проходила через пять основных эр:
\vs p059 0:2 \ublistelem{1.}\bibnobreakspace \bibemph{Эра, предшествующая жизни, ---} это первые четыреста пятьдесят миллионов лет, начиная примерно с того времени, когда планета достигла своих современных размеров, до установления жизни. Этот период ваши ученые назвали \bibemph{Археозойским.}
\vs p059 0:3 \pc \ublistelem{2.}\bibnobreakspace \bibemph{Эра зарождения жизни} распространяется на следующие сто пятьдесят миллионов лет. Это промежуток времени между эрой предшествующей жизни, или эрой катаклизмов, и последующим периодом более высоко развитой морской жизни. Эта эра известна вашим исследователям как \bibemph{Протерозойская.}
\vs p059 0:4 \pc \ublistelem{3.}\bibnobreakspace \bibemph{Эра морской жизни} охватывает следующие двести пятьдесят миллионов лет и более известна вам как \bibemph{Палеозойская.}
\vs p059 0:5 \pc \ublistelem{4.}\bibnobreakspace \bibemph{Эра ранней жизни на суше ---} это последующие сто миллионов лет, известные как \bibemph{Мезозойская} эра.
\vs p059 0:6 \pc \ublistelem{5.}\bibnobreakspace \bibemph{Эра млекопитающих} продолжается последние пятьдесят миллионов лет. Эта новейшая эра известна как \bibemph{Кайнозойская.}
\vs p059 0:7 \pc Таким образом, эра морской жизни охватывает около одной четверти вашей планетарной истории. Она может быть подразделена на шесть долгих периодов, каждый из которых характеризуется определенным хорошо выраженным развитием геологической и биологической сфер.
\vs p059 0:8 В начале этой эры дно морей, обширные континентальные шельфы и многочисленные мелководные прибрежные бассейны были покрыты обильной растительностью. Простейшие и примитивные формы животной жизни уже развились из существовавших ранее растительных организмов, и ранние животные организмы постепенно распространялись вдоль изрезанных береговых линий различных масс суши, пока многие внутренние моря не заполнились примитивной морской жизнью. Поскольку очень немногие из этих ранних организмов имели внешние панцири, крайне мало их сохранилось в ископаемом состоянии. Тем не менее, все было подготовлено для вступительных глав великой «каменной книги», которая сохранила записи о жизни, методично развивающейся в течение последующих веков.
\vs p059 0:9 Континент Северной Америки замечательно богат отложениями с ископаемыми останками, относящимися ко всей эре морской жизни. Самые первые и древние слои отделены от более поздних пластов предшествующего периода сильно эродированными отложениями, которые отчетливо разделяют эти две стадии планетарного развития.
\usection{1. Ранняя морская жизнь в мелководных морях. Эра трилобитов}
\vs p059 1:1 К началу этого периода относительного спокойствия на земной поверхности жизнь не выходит за пределы различных внутренних морей и океанской береговой линии, поскольку еще не появилось никаких форм сухопутных организмов. Примитивные морские животные хорошо прижились и готовы к последующему эволюционному развитию. Типичными организмами, сохранившимися от этой первой стадии животной жизни, являются амебы, появившиеся в конце предшествующего переходного периода.
\vs p059 1:2 \pc \bibemph{400\,000\,000} лет назад морская жизнь, как растительная, так и животная, достаточно хорошо распространена по всей планете. Климат на планете становится чуть теплее и мягче. Происходит общее затопление морских берегов различных континентов, особенно Северной и Южной Америки. Появляются новые океаны, а старые водоемы сильно увеличиваются в размерах.
\vs p059 1:3 Теперь растительность впервые выбирается на сушу и вскоре уже существенно адаптируется к неморской среде обитания.
\vs p059 1:4 \bibemph{Внезапно,} без постепенных переходных предковых форм, появляются первые многоклеточные животные. Возникли трилобиты, которые веками будут доминировать в морях. Что касается морской жизни --- это эпоха трилобитов.
\vs p059 1:5 В более поздний период этого отрезка времени большая часть Северной Америки и Европы поднялась из моря. Земная кора временно стабилизировалась; горы, или вернее довольно высокие возвышения суши простерлись вдоль атлантических и тихоокеанских берегов, в Вест\hyp{}Индии и южной Европе. Весь Карибский регион был высоко поднят.
\vs p059 1:6 \pc \bibemph{390\,000\,000} лет назад суша все еще была поднята. В отдельных районах восточной и западной Америки и западной Европы можно найти скальные пласты, отложения тех времен, и это древнейшие породы, которые содержат ископаемых трилобитов. Эти породы, несущие ископаемых, откладывались в многочисленных заливах, вдававшихся в массивы суши в форме длинных пальцев.
\vs p059 1:7 В продолжении нескольких миллионов лет Тихий океан начал вторгаться на американские континенты. Затопление суши происходило главным образом за счет колебаний коры, хотя и окраинное опускание земли, или континентальный оползень, также было тому причиной.
\vs p059 1:8 \pc \bibemph{380\,000\,000} лет назад Азия опускалась, а все другие континенты испытывали кратковременное поднятие. Но по мере развития этой эпохи недавно возникший Атлантический океан залил обширную береговую линию. Северная Атлантика, или Полярные моря, соединились тогда с южными водами Мексиканского залива. Когда это южное море затопило Аппалачский желоб, на востоке его волны разбивались о подножья гор, таких же высоких, как Альпы, но в целом, континенты представляли собой однообразные равнины, совершенно лишенные живописных красот.
\vs p059 1:9 \pc В эти эпохи сформировалось четыре вида отложений:
\vs p059 1:10 \ublistelem{1.}\bibnobreakspace Обломочные породы --- породы, отложенные около береговых линий.
\vs p059 1:11 \ublistelem{2.}\bibnobreakspace Песчаники --- породы, отложенные в мелких водах, где сила волн препятствовала накоплению глины.
\vs p059 1:12 \ublistelem{3.}\bibnobreakspace Сланцевые глины --- осадки в более глубоких и более спокойных водах.
\vs p059 1:13 \ublistelem{4.}\bibnobreakspace Известняки --- включая отложения панцирей трилобитов в глубоких водах.
\vs p059 1:14 \pc Ископаемые трилобиты тех времен представлены определенными сходными основными формами с некоторыми хорошо выраженными отличиями. Ранние животные, развивающиеся из трех первоначальных имплантаций жизни, обладали характерными чертами; те, которые появились в западном полушарии, в чем\hyp{}то отличались от животных из евразийской группы и от австралазийского, или австрало\hyp{}антарктического типа.
\vs p059 1:15 \pc \bibemph{370\,000\,000} лет назад произошло мощное и почти полное погружение Северной и Южной Америки, за ним последовало затопление Африки и Австралии. Только отдельные районы Северной Америки оставались выше уровня этих мелководных Кембрийских морей. Спустя пять миллионов лет моря стали отступать перед поднимающейся сушей. И все эти явления опускания и поднятия суши происходили внешне незаметно, чрезвычайно медленно в течение миллионов лет.
\vs p059 1:16 Пласты этой эпохи, содержащие ископаемых трилобитов, встречаются повсеместно на всех континентах, кроме центральной Азии. Во многих районах эти породы лежат горизонтально, но в горах они наклонены и искривлены из\hyp{}за давления и деформации. И во многих местах такое давление изменило первоначальную структуру этих отложений. Песчаники превратились в кварцы, сланцевые глины --- в сланцы, а известняки --- в мрамор.
\vs p059 1:17 \pc \bibemph{360\,000\,000} лет назад суша все еще продолжала подниматься. Северная и Южная Америка была высоко поднята. Поднимались Западная Европа и Британские острова, кроме отдельных областей Уэльса, которые находились глубоко под водой. В эти эпохи не существовало больших ледовых щитов. Псевдоледниковые отложения, с которыми связывают эти пласты в Европе, Африке, Китае и Австралии, появились благодаря изолированным горным ледникам или перемещенным ледниковым наносам более позднего происхождения. Мировой климат был морским, а не континентальным. Южные моря были теплее, чем сейчас и простирались к северу по Северной Америки до полярных районов. Гольфстрим проходил через центральную часть Северной Америки, поворачивал к востоку, омывая и согревая берега Гренландии, превращая этот покрытый сейчас ледовым панцирем континент в подлинный тропический рай.
\vs p059 1:18 \pc Морская жизнь была почти однотипной во всем мире: это морские водоросли, одноклеточные организмы, простые губки, трилобиты и другие ракообразные --- креветки, крабы и омары. К концу этого периода появилось до трех тысяч разновидностей плеченогих, из которых сохранилось только двести. Эти животные --- разновидности ранней жизни, которая дошла до настоящего времени практически неизменной.
\vs p059 1:19 Но доминирующими живыми созданиями были трилобиты. Это были раздельнополые животные многочисленных форм; будучи плохими пловцами, они медленно парили в воде или ползали по дну моря, сворачиваясь для защиты, когда их атаковали враги, возникшие позднее. Они увеличились в длину с двух дюймов до одного фута и сформировали четыре различные группы: плотоядные, растительноядные, всеядные и «грунтоеды». Способность последних питаться в основном неорганическим веществом, а они были самыми поздними многоклеточными животными, способными на это, объясняет их многочисленность, и продолжительность существования.
\vs p059 1:20 Такова биогеологическая картина Урантии в конце длительного периода мировой истории, охватывающего пятьдесят миллионов лет, названного вашими геологами \bibemph{Кембрийский.}
\usection{2. Стадия первого континентального потопа. Эпоха беспозвоночных животных}
\vs p059 2:1 Все периодические явления поднятия и затопления суши, характерные для тех времен, были постепенными и происходили без катаклизмов, практически без вулканической активности или с малой вулканической активностью. Во время всех этих последовательных поднятий и опусканий суши материнский Азиатский континент лишь частично разделял судьбу остальных массивов суши. Он особенно в ранние периоды истории испытал многие затопления, погружаясь сначала в одном направлении, потом в другом; но на нем не сохранилось однородных скальных отложений, которые можно найти на других континентах. В относительно недавние времена Азия была наиболее стабильной из всех массивов суши.
\vs p059 2:2 \pc \bibemph{350\,000\,000} лет назад начался период великого потопа, охватившего все континенты, кроме центральной Азии. Массивы суши неоднократно покрывались водой; только прибрежные возвышенности оставались выше уровня этих мелководных, но многочисленных изменяющихся внутренних морей. Этот период характеризовался тремя крупными затоплениями, но еще до его конца континенты опять поднялись, общая площадь возникшей суши была на пятнадцать процентов больше, чем в настоящее время. Карибский регион был высоко поднят. В Европе этот период отчетливо не прослеживается, потому что колебания суши там были меньше, а вулканическая активность --- была более постоянной.
\vs p059 2:3 \pc \bibemph{340\,000\,000} лет назад произошло еще одно обширное погружение суши, не затронувшее лишь Азию и Австралию. Воды мировых океанов в целом соединились. Это была великая эпоха известняков, большая часть которых была образована известковыми водорослями.
\vs p059 2:4 Спустя несколько миллионов лет обширная площадь американских континентов и Европа начали подниматься из воды. В западном полушарии сохранился только рукав Тихого океана, затопивший Мексику и нынешний район Скалистых гор, но ближе к концу этой эпохи атлантические и тихоокеанские берега опять начали погружаться.
\vs p059 2:5 \pc \bibemph{330\,000\,000} лет назад начался отрезок времени относительной стабильности во всем мире, когда большая часть суши была опять над водой. Единственным исключением в этом царстве покоя на суше было извержение огромного североамериканского вулкана в восточном Кентукки, одно из величайших одиночных извержений, когда\hyp{}либо случавшихся в мире. Пепел этого вулкана покрыл пятьсот квадратных миль толщиной от пятнадцати до двадцати футов.
\vs p059 2:6 \pc \bibemph{320\,000\,000} лет назад произошел третий великий потоп этого периода. Кроме регионов, затопленных в предшествующие потопы, воды в это погружение залили во многих направлениях и другие районы Америки и Европы. Восточная часть Северной Америки и Западная Европа ушли под воду на 10\,000 --- 15\,000 футов.
\vs p059 2:7 \pc \bibemph{310\,000\,000} лет назад во всем мире, кроме южных частей Северной Америки, массивы суши опять значительно поднялись. Появилась Мексика, образовался Мексиканский залив, который с тех пор остается неизменным.
\vs p059 2:8 Жизнь в этот период продолжала развиваться. Мир опять становится спокойным и относительно мирным; климат --- мягким и ровным; наземные растения мигрируют все дальше от береговой кромки морей. Паттерны жизни хорошо развиты, хотя можно обнаружить только немногочисленные растительные окаменелости этих времен.
\vs p059 2:9 \pc Это была великая эпоха эволюции отдельных животных организмов: хотя многие коренные изменения, такие как переход от растений к животным, произошли ранее. Морская фауна развилась до уровня, когда ископаемые останки каждого типа жизни, более примитивного, чем позвоночные, уже можно найти в породах, отложенных в те времена. Но все эти животные были морскими. Еще не появились никакие наземные животные, за исключением нескольких типов червей, которые зарывались вдоль морских берегов; наземные растения еще не успели широко распространиться по всем континентам; содержание углекислого газа в воздухе было слишком велико, и организмы, дышащие воздухом, еще не могли существовать. На первоначальной стадии существование всех животных, кроме отдельных, самых примитивных, прямо или опосредованно зависит от растительной жизни.
\vs p059 2:10 Трилобиты по\hyp{}прежнему играли важную роль. Существовали десятки тысяч разновидностей этих мелких животных --- предшественников современных ракообразных. У некоторых трилобитов было от двадцати пяти до четырехсот маленьких глаз; у других глаза были редуцированы. К концу этого периода трилобиты доминировали в морях наряду с несколькими другими формами беспозвоночных. Но в начале следующего периода они полностью исчезли.
\vs p059 2:11 Широко распространены были известковые водоросли. Существовали тысячи видов ранних предков кораллов, морские черви многочисленных видов и много вымерших с тех пор разновидностей медуз. Появились кораллы и более поздние виды губок. Головоногие моллюски были хорошо развиты, и наутилус, осьминог, каракатица и кальмар --- их непосредственные современные представители.
\vs p059 2:12 Было много разновидностей животных с раковинами, но их раковины не были тогда приспособленными для защиты, как в последующие века. В водах древних морей обитали брюхоногие моллюски: одностворчатые багрянки, береговички и улитки. Двустворчатые брюхоногие в последующие миллионы лет сохранились почти такими же, как и в тот период, они объединяют мидий, сердцевидок, устриц и гребешков. Появились также организмы со створчатой раковиной, и эти плеченогие жили в тех древних водах почти так же, как и сейчас; у них даже были на створках замковые, бороздчатые и другие виды защитных приспособлений.
\vs p059 2:13 \pc Так кончается эволюционная история второго великого периода морской жизни, который известен вашим геологам как \bibemph{Ордовикский.}
\usection{3. Стадия второго великого потопа. Период кораллов --- эпоха плеченогих}
\vs p059 3:1 \bibemph{300\,000\,000} лет назад начался второй великий период погружения суши. Вторжения древних силурийских морей на юге и севере почти поглотили большую часть Европы и Северной Америки. Суша была не высоко поднята над морем, поэтому не происходило больших отложений около береговых линий. Моря кишели организмами с известковой раковиной, и из этих раковин на дне моря постепенно образовались очень толстые слои известняка. Это первое обширное отложение известняка, и оно покрывает практически всю Европу и Северную Америку, но выходит на поверхность земли только в нескольких местах. Толщина этой древней скальной породы в среднем приблизительно тысяча футов, но многие зоны этих отложений с тех пор были сильно деформированы наклонами, смещениями пластов, сдвигами, и многие из них превратились в кварц, глинистые сланцы и мрамор.
\vs p059 3:2 В скальных слоях этого периода не найдено вулканических расплавленных пород или лавы; следы лавовых потоков обнаружены только в районах огромных вулканов южной Европы, восточного Мена и в Квебеке. Вулканическая активность в основном окончилась. Это был пик морских отложений; возникновение гор или было слабым, или отсутствовало.
\vs p059 3:3 \pc \bibemph{290\,000\,000} лет назад море в основном отступило с континентов, и дно окружающих океанов опускалось. До момента очередного затопления массы суши мало изменились. На всех континентах начиналось раннее формирование гор, и высочайшими из этих поднятий коры стали Гималаи в Азии и огромные Каледонские горы, простирающиеся от Ирландии через Шотландию до Шпицбергена.
\vs p059 3:4 Именно в отложениях этой эпохи находятся основные месторождения газа, нефти, цинка и свинца; газ и нефть образовались из громадных скоплений растительных и животных останков, снесенных за время предшествующего затопления суши, тогда как минеральные отложения представляют собой осадки неподвижных водоемов. Многие отложения каменной соли относятся к этому периоду.
\vs p059 3:5 Трилобиты быстро исчезали, и главную роль в этот период стали играть крупные моллюски, или головоногие. Эти животные достигали пятнадцати футов в длину и до фута в диаметре и стали хозяевами моря. Этот вид животных появился \bibemph{внезапно} и стал доминировать в морской жизни.
\vs p059 3:6 В это время вулканическая активность была высока в европейском секторе. В течение миллионов и миллионов лет не было таких неистовых и обширных извержений вулканов, как тогда вокруг Средиземноморской впадины и, особенно, в районе Британских островов. Этот лавовый поток в районе Британских островов сейчас выглядит как перемежающиеся слои лавы и скал толщиной 25\,000 футов. Эти породы были отложены прерывающимися разливами лавы, которая растекалась по ложу мелководных морей, покрывая скальные отложения, и они все вместе впоследствии были высоко подняты над морем. Сильные землетрясения происходили в северной Европе, особенно в Шотландии.
\vs p059 3:7 Морской климат оставался мягким и ровным, и теплые моря омывали берега полярных земель. Вплоть до Северного полюса в этих отложениях можно найти плеченогих и других морских ископаемых. Множились брюхоногие, плеченогие, губки и рифообразующие кораллы.
\vs p059 3:8 Эта эпоха окончилась вторым наступлением Силурийских морей одновременно с еще одним слиянием южных и северных океанов. В жизни моря доминируют головоногие, ассоциированные формы жизни стремительно развиваются и образуют самостоятельные виды.
\vs p059 3:9 \pc \bibemph{280\,000\,000} лет назад после второго Силурийского наводнения континенты снова значительно поднялись. Скальные отложения этого затопления известны в Северной Америке как Ниагарские известняки, потому что по этим пластам сейчас протекает Ниагарский водопад. Этот скальный слой простирается от восточных гор до района долины Миссисипи, а на юге даже западнее его. Несколько слоев пересекают Канаду, часть Южной Америки, Австралию и большую часть Европы; средняя толщина этой Ниагарской свиты около шестисот футов. Во многих районах можно найти скопления обломочной породы, сланцевой глины и каменной соли, непосредственно покрывающие Ниагарские отложения. Это накопление вторичных оседаний. Соль оседала в больших лагунах, которые поочередно то открывались в море, то отрезались от него, происходило испарение воды с отложением соли вместе с остальными веществами, находившимися в растворе. В отдельных районах такие залегания каменной соли имеют толщину семьдесят футов.
\vs p059 3:10 Климат ровный и мягкий, и морские ископаемые откладывались в полярных районах. Но к концу этой эпохи моря становятся столь солеными, что выживают лишь немногие формы жизни.
\vs p059 3:11 Ближе к концу последнего Силурийского погружения широко распространяются иглокожие --- морские лилии --- это ясно видно из отложений криноидных известняков. Трилобиты почти исчезли, а моллюски продолжают царствовать в морях; очень сильно увеличивается формирование коралловых рифов. В эту эпоху в отдельных подходящих районах впервые появляются примитивные водные скорпионы. Вскоре и \bibemph{внезапно} появляются настоящие скорпионы\hyp{}существа, действительно дышащие воздухом.
\vs p059 3:12 Эти события заканчивают третий период морской жизни, покрывающий около двадцати пяти миллионов лет и известный вашим исследователям как \bibemph{Силурийский.}
\usection{4. Стадия великого подъема суши. Период наземной растительной жизни. Эпоха рыб}
\vs p059 4:1 В вековой схватке между сушей и водой, победа на относительно долгий период была за морем, но уже близилось время победы суши. Континенты дрейфовали в ограниченном пространстве и периодически все они по всему миру соединялись между собой тонкими перешейками и узкими полосками земли.
\vs p059 4:2 По мере поднятия суши после последнего силурийского затопления подходит к концу важный период в развитии мира и эволюции жизни. Это заря новой эпохи на земле. Голый и непривлекательный пейзаж прежних времен начинает сменяться на растительный, и вскоре появятся первые великолепные леса.
\vs p059 4:3 Морская жизнь этой эпохи была очень разнообразной, что определялось изоляцией различных видов, но позже происходили свободное смешение и ассоциация всех этих различных типов. Плеченогие рано прошли пик своего развития и стали исчезать, их сменили членистоногие, и впервые появились усоногие раки. Но самым великим событием было неожиданное появление группы рыб. Настала эпоха рыб, этот период мировой истории характеризовался \bibemph{позвоночными} животными.
\vs p059 4:4 \pc \bibemph{270\,000\,000} лет назад континенты уже полностью находились выше уровня воды. В течение предыдущих миллионов и миллионов лет столько суши одновременно не находилось над водой, во всей мировой истории это была одна из величайших эпох поднятия суши.
\vs p059 4:5 Спустя пять миллионов лет значительные пространства Северной и Южной Америки, Европы, Африки, северной Азии и Австралии снова были на короткое время затоплены, а Северную Америку в отдельные периоды времени затапливало почти полностью. И толщина образовавшихся в этот период слоев известняка возросла с 500 до 5\,000 футов. Эти многочисленные Девонские моря распространялись сначала в одном, затем в другом направлении, так что огромное арктическое североамериканское внутреннее море через северную Калифорнию соединилось протокой с Тихим океаном.
\vs p059 4:6 \pc \bibemph{260\,000\,000} лет назад, к концу этой эпохи опускания суши, Северная Америка была частично покрыта морями, которые одновременно соединялись с водами Тихого, Атлантического, Арктического океанов и Мексиканского залива. Отложения на этих последних стадиях первого Девонского потопа имеют толщину в среднем около тысячи футов. Коралловые рифы, характерные для этих времен, свидетельствуют о том, что внутренние моря были чистыми и мелкими. Такие коралловые отложения выходят на поверхность на берегах реки Огайо около Луисвилла, Кентукки, и их толщина около ста футов; существует около двухсот разновидностей этих отложений. Эти коралловые формации простираются через Канаду и северную Европу до арктических регионов.
\vs p059 4:7 Вслед за этими погружениями, во многих местах береговые линии значительно поднялись, так что более ранние отложения покрылись илом или сланцевыми глинами. Есть также пласт красного песчаника, который характерен для одного из девонских отложений, и этот красный слой простирается на большей части земной поверхности, его находят в Северной и Южной Америке, Европе, России, Китае, Африке и Австралии. Такие красные отложения указывают на засушливые или полузасушливые условия, но климат этой эпохи был все еще мягким и ровным.
\vs p059 4:8 В течение всего этого периода суша к юго\hyp{}востоку от острова Цинциннати оставалась значительно выше уровня воды. Но большая часть западной Европы, включая Британские острова, была затоплена. В Уэльсе, Германии и других местах в Европе Девонские скалы имеют в толщину 20\,000 футов.
\vs p059 4:9 \pc \bibemph{250\,000\,000} лет назад появились группы рыб, позвоночных, что стало одним из наиболее важных событий во всей эволюции до появления человека.
\vs p059 4:10 Членистоногие, или ракообразные, были предками первых позвоночных. Предшественниками группы рыб были два модифицированных членистоногих предка, один обладал длинным телом, соединяющим голову и хвост, другой же был бесчелюстной предрыбой без позвоночника. Но эти предварительные типы были быстро уничтожены, когда рыбы, первые позвоночные животного мира, \bibemph{неожиданно} появились с севера.
\vs p059 4:11 Многие из крупнейших ныне существующих рыб вышли из этой эпохи, отдельные разновидности, имеющие зубы, доходили до двадцати пяти --- тридцати футов в длину; современные акулы --- это выжившие потомки таких древних рыб. Легочные и панцирные рыбы достигли своего эволюционного пика, и уже до окончания этой эпохи рыбы приспособились как к пресным, так и к соленым водам.
\vs p059 4:12 Подлинные костные слои из рыбьих зубов и скелетов можно найти в отложениях, относящихся к концу этого периода; богатые ископаемыми пласты расположены вдоль берега Калифорнии, так как в этом районе много закрытых заливов Тихого океана вдавались глубоко в сушу.
\vs p059 4:13 Суша быстро покрывалась новыми видами наземной растительности. Прежде лишь несколько растений росло еще где\hyp{}нибудь на суше, не только у кромки воды. В это время во всех частях мира \bibemph{внезапно} возникло и стремительно распространялось по поверхности быстро поднимающейся суши плодовитое \bibemph{семейство папоротников.} Вскоре появились виды деревьев, имевших два фута в толщину и сорок футов в высоту; позднее образовались листья, но эти ранние разновидности имели только рудиментарную листву. Было много более мелких растений, но их ископаемые остатки не найдены, потому что их обычно уничтожали появившиеся еще раньше бактерии.
\vs p059 4:14 По мере того, как поднималась суша, Северная Америка оказалась соединенной с Европой сухопутными мостами, простирающимися до Гренландии. И сейчас в Гренландии под ледовым панцирем сохранились остатки ранних сухопутных растений.
\vs p059 4:15 \pc \bibemph{240\,000\,000} лет назад части суши как в Европе, так и в Северной Америке начали погружаться. Это оседание вызвало последний и наименее масштабный из Девонских потопов. Арктические моря опять распространились к югу на большую часть Северной Америки, Атлантика затопила большую часть Европы и западной Азии, а южная часть Тихого океана покрыла почти всю Индию. Это наводнение медленно наступало и так же медленно отступало. Кэтскилльские горы вдоль западного берега реки Гудзон --- крупнейший геологический памятник этой эпохи, который можно найти в Северной Америке.
\vs p059 4:16 \pc \bibemph{230\,000\,000} лет назад моря продолжали свое отступление. Большая часть Северной Америки оставалась над водой, сильная вулканическая активность имела место в районе Св. Лаврентия. Гора Роял вблизи Монреаля --- это эродированный кратер одного из этих вулканов. Отложения всей этой эпохи хорошо представлены в Аппалачских горах в Северной Америке, где река Саскеханна прорезала свою долину, обнажив эти последовательные слои, которые достигают 13\,000 футов в толщину.
\vs p059 4:17 \pc Поднятие континентов продолжалось, атмосфера обогащалась кислородом. Суша была покрыта обширными лесами из папоротников в сто футов в высоту и уникальными деревьями того времени, молчаливыми лесами; не было слышно ни звука, ни даже шороха листа, потому что у этих деревьев еще не было листьев.
\vs p059 4:18 \pc Так подошел к концу один из самых продолжительных периодов эволюции морской жизни, \bibemph{эпоха рыб.} Этот период мировой истории длился почти пятьдесят миллионов лет; Ваши исследователи назвали его \bibemph{Девонский.}
\usection{5. Стадия движения коры. Каменноугольный период папоротниковых лесов. Эпоха лягушек}
\vs p059 5:1 Появление рыб во время предшествующего периода стало вершиной эволюции морской жизни. С этого момента начала приобретать все большее значение эволюция сухопутной жизни. И этот период открывается условиями, почти идеальными для появления первых наземных животных.
\vs p059 5:2 \pc \bibemph{220\,000\,000} лет назад многие из континентальных областей суши, включая большую часть Северной Америки, находились выше уровня воды. Суша покрылась великолепной растительностью; это была подлинно \bibemph{эпоха папоротников.} Углекислый газ по\hyp{}прежнему присутствовал в атмосфере, но в уменьшающемся количестве.
\vs p059 5:3 Вскоре после этого центральная часть Северной Америки опять была затоплена, образовалось два огромных внутренних моря. И атлантические и тихоокеанские береговые нагорья были расположены сразу за современной береговой линией. Эти два моря вскоре слились, смешивая свои различные формы жизни, и объединение этих морских фаун положило начало быстрому повсеместному закату морской жизни и зарождению последующего периода сухопутной жизни.
\vs p059 5:4 \pc \bibemph{210\,000\,000} лет назад теплые воды арктических морей покрыли большую часть Северной Америки и Европы. Южные полярные воды затопили Южную Америку и Австралию, но Африка и Азия оставались высоко поднятыми.
\vs p059 5:5 Когда уровень моря был наибольшим, \bibemph{неожиданно} произошло новое событие в эволюции. Внезапно появились первые сухопутные животные. Было много видов этих животных, которые могли жить и на суше, и в воде. Эти дышащие воздухом амфибии произошли от членистоногих, чьи плавательные пузыри превратились в легкие.
\vs p059 5:6 Из морских соленых вод на сушу выползли наземные улитки, скорпионы и лягушки. И сейчас еще лягушки откладывают свои яйца в воду и их молодь --- головастики --- живут как маленькие рыбки. Этот период может быть точно охарактеризован как \bibemph{эпоха лягушек.}
\vs p059 5:7 Спустя очень короткое время впервые появились насекомые и наряду с пауками, скорпионами, тараканами, сверчками и саранчой вскоре распространились по континентам мира. Стрекозы достигали тридцати дюймов в размахе. Появилась тысяча видов тараканов, и некоторые достигали в длину четырех дюймов.
\vs p059 5:8 Две группы иглокожих были особенно хорошо развиты и по сути стали основными ископаемыми этой эпохи. Крупные, несущие панцирь акулы также были высоко развиты и доминировали в океане в течение более чем пяти миллионов лет. Климат был по\hyp{}прежнему мягким и ровным; морская жизнь изменялась мало. Развивались пресноводные рыбы, трилобиты были на пороге вымирания. Кораллы были редки, и большая часть известняков образована морскими лилиями. Лучшие строительные известняки отложились во время этой эпохи.
\vs p059 5:9 Воды многих внутренних морей были так сильно насыщены известью и другими минералами, что сильно препятствовало прогрессу и развитию многих видов морских существ. Со временем моря очистились в результате интенсивного отложения породы, в отдельных местах содержащей цинк и свинец.
\vs p059 5:10 Толщина отложений этой ранней Каменноугольной эпохи от 500 до 2\,000 футов, в них входят песчаники, сланцевые глины и известняки. Древнейшие пласты содержат ископаемые останки как сухопутных, так и морских животных и растений вместе с гравием и отложениями водоемов. В этих древних пластах практически нет рентабельного угля. Эти отложения во всей Европе очень похожи на те, что находятся в Северной Америке.
\vs p059 5:11 К концу этой эпохи суша Северной Америки начала подниматься. В этом процессе был короткий перерыв, когда моря опять заняли около половины своих предыдущих лож. Это было короткое затопление, и большая часть суши вскоре снова существенно поднялась над водой. Южная Америка по\hyp{}прежнему была соединена с Европой через Африку.
\vs p059 5:12 Эта эпоха стала свидетелем появления гор Вогез, Шварцвальд и Уральских. Остатки других и более древних гор можно найти по всей Великобритании и Европе.
\vs p059 5:13 \pc \bibemph{200\,000\,000} лет назад начались по\hyp{}настоящему активные стадии Каменноугольного периода. В течение двадцати миллионов лет перед этим происходило отложение ранних слоев каменного угля, но теперь началось более интенсивное формирование пластов угля. Длина эпохи реального отложения угля была немногим больше двадцати пяти миллионов лет.
\vs p059 5:14 Суша периодически поднималась и опускалась из\hyp{}за изменения уровня моря, вызванного активностью дна океанов. Эта нестабильность коры --- опускание и подъем суши --- в сочетании с изобильной растительностью прибрежных болот усилило образование отложений угля, что и послужило основанием назвать этот период \bibemph{Каменноугольным.} А климат по всему миру был по\hyp{}прежнему мягким.
\vs p059 5:15 Слои угля перемежаются со сланцевыми глинами, камнем и обломочной породой. Эти угольные пласты в центральных и восточных Соединенных Штатах изменяются по толщине от сорока до пятидесяти футов. Но многие из этих отложений были вымыты во время последующих поднятий суши. В отдельных районах Северной Америки и Европы несущие уголь пласты достигают 18\,000 футов в толщину.
\vs p059 5:16 Наличие корней деревьев, выросших в глине, подстилающей современные угольные пласты, свидетельствует, что уголь формировался точно там, где его сейчас находят. Уголь --- это сохранившиеся в воде и модифицированные давлением остатки ряда растений, растущих в трясине и на болотистых берегах в те отдаленные века. Угольные слои часто содержат и газ, и нефть. Торфяные пласты, остатки прошлой растительности, превратились бы в какой\hyp{}нибудь сорт угля, если бы были подвержены соответствующему давлению и температуре. Антрацит подвергался большему давлению и температуре, по сравнению с другими сортами угля.
\vs p059 5:17 В Северной Америке число слоев угля в различных пластах, которое говорит о числе опусканий и подъемов суши, колеблется от десяти в Иллинойсе, двадцати в Пенсильвании, тридцати пяти в Алабаме, до семидесяти пяти в Канаде. В угольных пластах находят как пресноводных, так и морских ископаемых животных.
\vs p059 5:18 В течение этой эпохи горы Северной и Южной Америки были активны, поднимались и Анды, и южные предшественники Скалистых гор. Огромные высокие прибрежные районы Атлантики и Тихого океана начали погружаться, и постепенно в результате эрозии и затопления береговая линия обоих океанов отодвинулась примерно до настоящего положения. Толщина отложения этого затопления в среднем тысяча футов.
\vs p059 5:19 \pc \bibemph{190\,000\,000} лет назад Северо\hyp{}Американское Каменноугольное море распространилось на запад и через район современных Скалистых гор, через северную Калифорнию соединилось с Тихим океаном. Отложение угля продолжалось по всей Америке и Европе, слой за слоем, по мере того как прибрежные земли поднимались и опускались в процессе периодических колебаний морского берега.
\vs p059 5:20 \pc \bibemph{180\,000\,000} лет назад Каменноугольный период, в течение которого по всему миру --- в Европе, Индии, Китае, северной Африке и Америке формировался уголь, закончился. В конце периода формирования угля Северная Америка к востоку от долины Миссисипи поднимается, и большая часть этой области с тех пор остается выше уровня моря. В этот период поднятия суши появляются современные горы Северной Америки, как в районе Аппалачей, так и на западе. Вулканы были активны на Аляске, в Калифорнии и в горообразующих районах Европы и Азии. Восточная Америка и западная Европа были связаны континентом Гренландии.
\vs p059 5:21 Поднятие суши начало изменять морской климат предшествующих эпох и заменять его менее мягким и более разнообразным континентальным климатом.
\vs p059 5:22 Растения этих времен размножались спорами, которые ветер мог разносить далеко и широко. Стволы деревьев Каменноугольного периода достигали обычно семи футов в диаметре и ста двадцати пяти футов в высоту. Современные папоротники являются настоящими реликтами этих прошедших эпох.
\vs p059 5:23 В целом, это были эпохи развития пресноводных организмов; в предшествующей морской жизни произошли лишь незначительные изменения. Но важной особенностью этого периода было \bibemph{внезапное} появление лягушек и их многих родственников. Характерными проявлениями жизни в эпоху угля были \bibemph{папоротники} и \bibemph{лягушки.}
\usection{6. Стадия смены климата. Период семенных растений. Эпоха биологического упадка}
\vs p059 6:1 Этот период стал концом кардинального эволюционного развития морской жизни и началом переходного периода, ведущего к последующим эпохам наземных животных.
\vs p059 6:2 Это была одна из эпох сильного обеднения жизни. Тысячи морских видов погибли, а на суше жизнь еще едва принялась. Это было время биологического упадка, эпоха, когда жизнь почти исчезла с поверхности земли и из глубин океанов. К окончанию продолжительной эры морской жизни на земле было более ста тысяч видов живых существ. К концу этого переходного периода выжило менее пятисот.
\vs p059 6:3 Особенности этого нового периода были связаны не столько с остыванием земной коры или длительным отсутствием вулканической деятельности, сколько с необычной комбинацией банальных и существовавших ранее причин --- сокращениями морей и возрастающим поднятием необъятных масс суши. Мягкий морской климат прежних времен исчезал, и быстро развивался более резкий континентальный тип погоды.
\vs p059 6:4 \pc \bibemph{170\,000\,000} лет назад на всей поверхности земли происходили огромные эволюционные изменения и приспособления. Суша поднималась по всему миру, и океанское ложе в то же время опускалось. Появились изолированные горные хребты. Восточная часть Северной Америки была уже высоко над уровнем моря, западная медленно поднималась. Континенты были покрыты большими и маленькими солеными озерами и многочисленными внутренними морями, которые соединялись с океанами узкими проливами. Толщина слоев этого промежуточного периода колеблется от 1000 до 7000 футов.
\vs p059 6:5 Поднятия суши сформировали многочисленные складки земной коры. Это было время поднятий континентов, за исключением исчезновения некоторых сухопутных мостов, включая континенты, которые так долго соединяли Южную Америку с Африкой и Северную Америку с Европой.
\vs p059 6:6 Постепенно по всему миру внутренние озера и моря высыхали. Начали возникать изолированные горы и местные ледники, особенно в южном полушарии, и во многих регионах ледниковые отложения этих локальных формаций льда можно найти среди отдельных верхних и позднейших отложений угля. Появилось два новых климатических фактора --- оледенение и сухость. Многие из высоких регионов суши стали сухими и бесплодными.
\vs p059 6:7 \pc Во время этих климатических изменений значительно изменились и наземные растения. Впервые появились \bibemph{семенные растения,} и они лучше обеспечивали пищей развившуюся позднее жизнь сухопутных животных. Насекомые претерпели кардинальное изменение. Чтобы удовлетворить потребности в период зимней спячки и во время засухи, возникли \bibemph{стадии покоя.}
\vs p059 6:8 \pc Из наземных животных лягушки, которые достигли расцвета в предшествующую эпоху, начали быстро вымирать, но они выжили потому, что могли долго жить даже в высыхающих лужах и прудах этих отдаленных и особенно тяжелых времен. Во время заката эпохи лягушек в Африке произошел первый шаг в эволюционном развитии от лягушек к рептилиям. И так как массы суши были все еще соединены между собой, это дорептильное, дышащее воздухом создание распространилось по всему свету. К этому времени атмосфера так изменилась, что стала замечательным образом поддерживать дыхание животных. Северная Америка вскоре после того, как там появились эти дорептильные лягушки, временно была изолирована, отрезана от Европы, Азии и Южной Америки.
\vs p059 6:9 Постепенное охлаждение океанских вод резко усиливало разрушение океанической жизни. Морские животные этих эпох нашли временное пристанище в трех благоприятных убежищах: в районе современного Мексиканского залива, в Гангском заливе в Индии и в Сицилийском заливе Средиземноморского бассейна. И именно из этих трех районов позднее вышли новые морские виды, приспособленные к неблагоприятным условиям, чтобы вновь наполнить моря.
\vs p059 6:10 \pc \bibemph{160\,000\,000} лет назад земля в основном была покрыта растительностью, способной поддерживать жизнь наземных животных и атмосфера стала идеальной для дыхания животных. Так заканчивается период сокращения морской жизни и тех испытаний биологически неблагоприятными условиями, которые устранили все формы жизни кроме тех, которые имели ценность для продолжения существования и которые, таким образом, получили право стать предками стремительно развивающейся и чрезвычайно многообразной жизни последующих эпох планетарной эволюции.
\vs p059 6:11 Окончание этого периода биологического упадка, известного вашим исследователям как \bibemph{Пермский,} означает также завершение длительной \bibemph{Палеозойской} эры --- четверти планетарной истории, продолжительностью двести пятьдесят миллионов лет.
\vs p059 6:12 Обширный океанический питомник жизни на Урантии выполнил свою задачу. В течение долгих эпох, пока суша не была приспособлена для поддержания жизни, до того как атмосфера стала содержать достаточное количество кислорода для поддержания более высоко организованных наземных животных, море породило и воспитывало раннюю жизнь планеты. Теперь, по мере того, как на земле начала разворачиваться вторая стадия эволюции, биологическое значение моря постепенно сокращалось.
\vsetoff
\vs p059 6:13 [Представлено Носителем Жизни Небадона, одним из первоначального Отряда, направленного на Урантию.]
