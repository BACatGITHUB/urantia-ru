\upaper{70}{Эволюция человеческих форм правления}
\author{Мелхиседек}
\vs p070 0:1 Как только человек частично разрешил проблему обеспечения существования, перед ним встала задача регулирования взаимоотношений между людьми. Для развития производства требовались закон, порядок и социальное согласие; для частной собственности необходимо было правительство.
\vs p070 0:2 В развивающемся мире противоречия естественны; мирное существование обеспечивается только какой\hyp{}нибудь регулирующей системой. Социальное регулирование неотделимо от социальной организации; объединения предполагают наличие контролирующей власти. Правительство вынуждено упорядочивать отношения племен, кланов, семейств и личностей.
\vs p070 0:3 Правительство --- плод бессознательного развития; оно развивается путем проб и ошибок. Оно имеет непреходящую ценность для выживания; поэтому становится традиционным. Анархия увеличивает хаос, и, чтобы устранить его, повсеместно и постепенно возникло правительство, относительный закон и порядок. Борьба за существование в буквальном смысле вынуждала человеческую расу идти по пути прогресса к цивилизации.
\usection{1. Происхождение войны}
\vs p070 1:1 Война является естественным состоянием, доставшимся развивающемуся человеку в наследство; мир --- это социальная мера, позволяющая оценить продвинутость цивилизации. До начала объединения развивающихся рас в общество человек был крайне эгоистичным, подозрительным и невероятно задиристым. Насилие --- это закон природы, враждебность --- автоматическая реакция детей природы, тогда как война --- это те же самые действия, осуществляемые коллективно. И если где\hyp{}либо и когда\hyp{}либо здание цивилизации начинает испытывать напряжение из\hyp{}за сложностей социального развития, всегда происходит мгновенный и разрушительный возврат к этим древним методам насильственного устранения вражды между взаимодействующими человеческими сообществами.
\vs p070 1:2 Война является зверской реакцией на недоразумения и раздражители; мир достигается цивилизованным решением всех таких спорных проблем. Воевали все сангикские расы, а также выродившиеся Адамиты и Нодиты. Андониты рано научились золотому правилу, и даже сегодня их потомки --- эскимосы живут, в основном, руководствуясь этим правилом; среди них силен обычай и во многом им не свойственны жестокие антагонизмы, ведущие к насилию.
\vs p070 1:3 Андон учил своих детей разрешать споры, ударяя палкой по дереву, одновременно его проклиная; побеждал тот, чья палка ломалась первой. Позднее Андониты разрешали споры, устраивая публичные представления, на которых спорившие высмеивали друг друга, а присутствующие аплодисментами определяли победителя.
\vs p070 1:4 Такой феномен, как война, не мог возникнуть до тех пор, пока общество не было достаточно развитым, чтобы осознать и на себе испытывать периоды мира, и, как следствие, выработать приемы ведения войны. Сама концепция войны требует определенного уровня организации.
\vs p070 1:5 С появлением социальных объединений личная вражда начинала подчиняться групповым чувствам, и это способствовало внутриплеменной стабильности за счет межплеменного мира. Миром, таким образом, прежде всего пользовались члены одной группы, или племени, которые всегда недолюбливали и ненавидели другую группу, чужаков. Древний человек считал добродетелью пролить кровь чужака.
\vs p070 1:6 Но даже и так не получалось вначале. Когда древние вожди старались уладить размолвку, они часто считали целесообразным по крайней мере раз в году разрешать в племени драку камнями. Клан делился на две группы, и они в течение дня сражались друг с другом. И это все проводилось только для того, чтобы развлечься; они любили драться по\hyp{}настоящему.
\vs p070 1:7 \P\ Феномен войны сохраняется потому, что человек произошел от животного, а все животные воинственны. Среди ранних причин войн были:
\vs p070 1:8 \ublistelem{1.}\bibnobreakspace \bibemph{Голод,} поиски пищи приводили к набегам. Недостаток земель часто служил причиной войн, и в этих сражениях древние мирные племена были практически истреблены.
\vs p070 1:9 \P\ \ublistelem{2.}\bibnobreakspace \bibemph{Дефицит женщин ---} стремление приобрести достаточное число домашних работников. Кража женщин всегда вызывала войну.
\vs p070 1:10 \P\ \ublistelem{3.}\bibnobreakspace \bibemph{Тщеславие ---} желание продемонстрировать племенную доблесть. Более развитые группы сражались, чтобы навязать свой образ жизни менее развитым.
\vs p070 1:11 \P\ \ublistelem{4.}\bibnobreakspace \bibemph{Рабы ---} потребность пополнить ряды рабочей силы.
\vs p070 1:12 \P\ \ublistelem{5.}\bibnobreakspace \bibemph{Месть} была мотивом для войны, если племя считало, что соседнее племя виновно в смерти их соплеменника. Траур продолжался до тех пор, пока домой не приносили голову врага. Война из\hyp{}за мести была обычным явлением вплоть до сравнительно поздних времен.
\vs p070 1:13 \P\ \ublistelem{6.}\bibnobreakspace \bibemph{Развлечение ---} в древние времена война рассматривалась молодежью как развлечение. Если для войны не находилось подходящего и достаточного предлога, а мир становился гнетущим, соседние племена сходились в полудружеской схватке, сражались, чтобы развлечься, насладиться мнимой битвой.
\vs p070 1:14 \P\ \ublistelem{7.}\bibnobreakspace \bibemph{Религия ---} желание обратить в свою веру. Все примитивные религии разрешали войну. Только в сравнительно недавние времена религия начала протестовать против войны. Древнее духовенство было, к сожалению, обычно связано с военной кастой. Одним из величайших за все века шагов к миру была попытка отделить церковь от государства.
\vs p070 1:15 \P\ Древние племена всегда начинали войну по повелению своих богов, по приказанию своих вождей или жрецов. Иудеи верили в такого «Бога сражений»; и повествование об их набеге на мидианитов является типичным описанием зверской жестокости древних племенных войн; это нападение, с убийством всех мужчин, а впоследствии и всех детей мужского пола и всех женщин, которые не были девственницами, соответствовало нравам вождей племени две сотни тысяч лет назад. И все это было проделано «во имя Господа Бога Израиля».
\vs p070 1:16 Написанное здесь является рассказом об эволюции общества --- естественном разрешении проблем рас, выборе человеком своей судьбы на земле. Божество не подстрекает к таким зверствам, хотя человек часто охотно перекладывает ответственность на своих богов.
\vs p070 1:17 \P\ Милосердие в войне медленно приходило к человечеству. Даже когда иудеями правила женщина, Дебора, сохранялась такая же самая массовая жестокость. Ее полководец после победы над неевреями повелел, чтобы «все войско пало от меча; не осталось никого».
\vs p070 1:18 Очень рано расы стали использовать отравленное оружие. Допускались все виды увечий. Саул без колебания потребовал сто крайних плотей филистимлян как приданое, которое Давид должен был отдать за его дочь Мельхолу.
\vs p070 1:19 Сначала войны велись между целыми племенами, но в более поздние времена, когда два человека из разных племен вступали в конфликт друг с другом, то вместо битвы между племенами, происходила дуэль между спорящими. Стало также обычным для противоборствующих войск ставить все на исход поединка между представителями, выбранными от каждой стороны, как в случае с Давидом и Голиафом.
\vs p070 1:20 Первым облагораживающим войну явлением было стремление взять пленных. Затем от военных действий были освобождены женщины, потом последовало признание нестроевых лиц. Вскоре появились касты военных и постоянные армии, чтобы справляться с постоянно усложняющейся техникой ведения войны. Таким воинам рано запрещали вступать в связь с женщинами, а женщины задолго до того прекратили сражаться, хотя всегда кормили и ухаживали за солдатами и побуждали их к бою.
\vs p070 1:21 Обычай объявления войны представлял собой огромный шаг вперед. Такие оповещения о намерении драться указывали на возникновение чувства порядочности, а за этим стали постепенно развиваться правила «цивилизованных» военных действий. Очень рано вошло в обычай не сражаться около религиозных мест, а позднее --- не драться в определенные святые дни. Затем пришло общее признание права убежища; политические беженцы получали защиту.
\vs p070 1:22 Так военные действия постепенно развивались от охоты первобытного человека до вполне упорядоченной системы у «цивилизованных» наций более поздних времен. Но мирные социальные отношения все еще очень медленно приходят на смену враждебным.
\usection{2. Социальная ценность войны}
\vs p070 2:1 В минувшие века жестокая война часто побуждала к социальным изменениям и облегчала претворение в жизнь новых идей, которые мирным, естественным путем проходили бы за десять тысяч лет. Страшной ценой, которая была уплачена за определенную пользу войны, было то, что общество временно отбрасывалось назад в дикость; приходилось отказываться от цивилизованного благоразумия. Война --- сильное лекарство, очень дорогое и очень опасное; хотя и часто исцеляющее от определенных социальных беспорядков, бывает, что оно и убивает пациента --- разрушает общество.
\vs p070 2:2 Постоянная потребность в национальной безопасности создает много новых и передовых социальных изменений. Общество и сегодня пользуется многими полезными нововведениями, которые первоначально возникли как исключительно военные, и даже обязано войне танцами, одним из ранних прототипов которых было строевое обучение.
\vs p070 2:3 \P\ Война была социально значима для исчезнувших цивилизаций, поскольку она:
\vs p070 2:4 \ublistelem{1.}\bibnobreakspace Побуждала к дисциплине, усиливала кооперацию.
\vs p070 2:5 \ublistelem{2.}\bibnobreakspace Высоко ценила мужество и отвагу.
\vs p070 2:6 \ublistelem{3.}\bibnobreakspace Воспитывала и укрепляла национальное самосознание.
\vs p070 2:7 \ublistelem{4.}\bibnobreakspace Уничтожала слабых и неприспособленных людей.
\vs p070 2:8 \ublistelem{5.}\bibnobreakspace Разрушала иллюзию о примитивном равенстве и селективно расслаивала общество.
\vs p070 2:9 \P\ Война имела определенную ценность для эволюции и отбора, но так же, как и от рабства, по мере медленного развития цивилизации от нее должны будут когда\hyp{}нибудь отказаться. Древние войны способствовали путешествиям и культурному обмену; сейчас этим целям лучше служат современные транспорт и связь. Древние войны укрепляли нации, но современные сражения разрушают цивилизованную культуру. Результатом древних военных действий было уничтожение неразвитых народов; результатом современного конфликта является избирательное уничтожение лучших представителей человечества. Древние войны способствовали организации и эффективности, но сейчас это стало задачами современной индустрии. На протяжении прошедших веков война была социальным фактором, который двигал цивилизацию вперед; этот результат сейчас в большей степени достигается честолюбием и изобретениями. Древние военные действия поддерживали концепцию Бога сражений, но современный человек считает, что Бог есть любовь. Война служила многим важным целям в прошлом, она была незаменимыми лесами в построении цивилизации, но быстро становится несостоятельной с точки зрения культуры --- неспособной приносить пользу обществу, в какой\hyp{}либо мере соизмеримую с сопровождающими ее ужасающими потерями.
\vs p070 2:10 Одно время медики верили в кровопускание как в лекарство от многих болезней, но с тех пор открыты лучшие средства лечения большинства заболеваний. Точно так же и международное «военное кровопускание» должно уступить место лучшим методам лечения болезней наций.
\vs p070 2:11 Нации на Урантии уже вступили в великую борьбу между националистическим милитаризмом и индустриализацией, и во многих отношениях этот конфликт аналогичен вековой борьбе между скотоводом\hyp{}охотником и земледельцем. Но чтобы индустриализация восторжествовала над милитаризмом, она должна избежать сопровождающих ее опасностей. Опасности зарождающейся индустрии на Урантии следующие:
\vs p070 2:12 \ublistelem{1.}\bibnobreakspace Сильный уклон в сторону материализма, духовная слепота.
\vs p070 2:13 \ublistelem{2.}\bibnobreakspace Поклонение могуществу богатства, искажение ценностей.
\vs p070 2:14 \ublistelem{3.}\bibnobreakspace Пороки роскоши, культурная незрелость.
\vs p070 2:15 \ublistelem{4.}\bibnobreakspace Возрастающая опасность лености, бесчувственность к служению.
\vs p070 2:16 \ublistelem{5.}\bibnobreakspace Рост нежелательной расовой терпимости, биологическое разрушение.
\vs p070 2:17 \ublistelem{6.}\bibnobreakspace Угроза стандартизированного индустриального рабства, стагнация личности. Труд облагораживает, но нудная работа останавливает развитие.
\vs p070 2:18 \P\ Милитаризм автократичен и жесток, он --- дикость. Он способствует социальной организации завоевателей, но разрушает покоренных. Индустриализация более цивилизована и должна проводиться так, чтобы способствовать инициативе и поощрять индивидуализм. Общество должно всеми способами воспитывать оригинальность.
\vs p070 2:19 Не делайте ошибки, прославляя войну; лучше присмотритесь, что она сделала для общества, чтобы вы смогли более наглядно представить себе, что необходимо найти ей заменители, чтобы продолжать развитие цивилизации. И если не найдется таких адекватных замещающих, можете быть уверены, что войны еще долго будут продолжаться.
\vs p070 2:20 Человек никогда не примет мир как единственное нормальное состояние жизни до тех пор, пока совершенно и окончательно не убедится, что мир лучше всего обеспечивает его материальное благополучие, и до тех пор, пока общество мудро не выработает мирные формы удовлетворения этой врожденной потребности --- периодически давать волю коллективному стремлению освободиться от постоянно накапливающихся агрессивных эмоций и энергий --- реакции, направленной на самосохранение человеческого вида.
\vs p070 2:21 Но, между прочим, следует отдать должное войне как школе опыта, которая вынуждала расу самонадеянных индивидуалистов подчиняться власти, в высшей степени сконцентрированной, --- главе племени. Старомодная война выбирала для руководства от природы великих людей, но современной войне это больше не свойственно. Чтобы выявить лидеров, общество должно теперь обратиться к мирным завоеваниям --- индустрии, науке и социальным достижениям.
\usection{3. Ранние человеческие сообщества}
\vs p070 3:1 В самом примитивном обществе \bibemph{орда ---} это все; даже дети являются общей собственностью. Развивающаяся семья заместила орду в воспитании детей, а появлявшиеся кланы и племена стали представлять собой социальные единицы.
\vs p070 3:2 Сексуальный голод и материнская любовь создали семью. Но настоящее правительство не появляется до тех пор, пока не начинают формироваться надсемейные группы. До возникновения семьи в орде главенствовали неофициально выбранные индивидуумы. Африканские бушмены никогда не поднимались выше этого примитивного уровня; у них в орде нет вождей.
\vs p070 3:3 \P\ Семьи стали объединяться кровными узами в кланы, объединения родственников, которые впоследствии превратились в племена, территориальные сообщества. Военные действия и внешние угрозы вынуждали родственные кланы организовываться в племена, а коммерция и торговля сплачивали эти ранние и примитивные союзы в одно целое, обеспечивая относительный мир внутри них.
\vs p070 3:4 Миру на Урантии гораздо больше содействуют международные торговые организации, чем вся сентиментальная софистика иллюзорного планирования мира. Торговым отношениям способствуют развитие языка, улучшенные методы коммуникации и усовершенствованный транспорт.
\vs p070 3:5 Отсутствие единого языка всегда препятствовало росту мирных сообществ, но деньги стали международным языком современной торговли. Единство современного общества в основном поддерживается индустриальным рынком. Прибыль является могущественным цивилизующим фактором, особенно, когда к ней прибавляется желание служить обществу.
\vs p070 3:6 \P\ В ранние века каждое племя было охвачено концентрически расходящимися кругами возрастающего страха и подозрений; поэтому тогда\hyp{}то и было общепринято убивать всех чужаков, позже --- превращать их в рабов. Древнее представление о дружбе означало усыновление кланом; и существовало поверье, что принадлежность к клану позволит пережить смерть, --- одна из самых ранних концепций вечной жизни.
\vs p070 3:7 Церемония усыновления состояла в выпивании крови друг друга. В некоторых группах вместо крови употреблялась слюна --- таково древнее происхождение принятого в обществе обычая поцеловаться. И все церемонии союза, будь то свадьба или усыновление, всегда заканчивались празднованием.
\vs p070 3:8 В более поздние времена кровь разбавляли красным вином, а впоследствии выпивалось только вино; церемония усыновления заключалась в касании чашами с вином и завершалась выпиванием напитка. Иудеи использовали видоизмененную форму этой церемонии усыновления. Их арабские предки применяли присягу, во время которой рука кандидата лежала на половых органах члена племени. Иудеи относились к принятым в племя с добротой и по\hyp{}братски. «Пришелец поселившийся у вас, да будет для вас то же, что ваш соплеменник; люби его, как себя».
\vs p070 3:9 После приема гостей людей связывала «гостевая дружба». Когда гости уезжали, разбивалась пополам тарелка, одна часть передавалась уезжающему и служила исчерпывающей рекомендацией третьему лицу, если оно позднее прибывало с визитом. Было обычаем для гостей отплачивать за гостеприимство, рассказывая истории о своих путешествиях и приключениях. Рассказчики древних времен стали столь популярны, что в конечном итоге им запретили появляться в сезоны охоты и сбора урожая.
\vs p070 3:10 Первые мирные договоры скреплялись «кровными узами». Посланцы мира от двух враждующих племен встречались, оказывали друг другу почести, а потом кололи кожу до крови; после этого они высасывали кровь друг друга и провозглашали мир.
\vs p070 3:11 Самые ранние мирные посольства состояли из делегаций мужчин, приводивших выбранных ими девственниц для сексуального удовлетворения бывших врагов; сексуальная тяга противопоставлялась жажде войны. Племя, которому была оказана такая честь, совершало ответный визит, предлагая своих девственниц; после чего надолго воцарялся мир. И вскоре были одобрены взаимные браки между семьями вождей.
\usection{4. Кланы и племена}
\vs p070 4:1 Первой мирной группой стала семья, затем клан, племя и позднее нация, которая в конечном итоге превратилась в современное территориальное государство. Тот факт, что современные мирные сообщества давно уже переросли кровные союзы и охватывают нации, очень обнадеживает, несмотря на то, что на Урантии продолжают тратиться большие деньги на подготовку к войнам.
\vs p070 4:2 Кланы были группами внутри племени, связанными кровными узами, и они существовали благодаря следующим общим факторам.
\vs p070 4:3 \P\ \ublistelem{1.}\bibnobreakspace Происхождение от общего предка.
\vs p070 4:4 \ublistelem{2.}\bibnobreakspace Верность общему религиозному тотему.
\vs p070 4:5 \ublistelem{3.}\bibnobreakspace Общение на одном диалекте.
\vs p070 4:6 \ublistelem{4.}\bibnobreakspace Проживание в одном месте.
\vs p070 4:7 \ublistelem{5.}\bibnobreakspace Страх перед одними и теми же врагами.
\vs p070 4:8 \ublistelem{6.}\bibnobreakspace Общий военный опыт.
\vs p070 4:9 \P\ Главы кланов всегда подчинялись вождю племени, а ранние племенные правительства были добровольным союзом кланов. Аборигены Австралии так и не создали племенную форму правительства.
\vs p070 4:10 Полномочия гражданских (невоенных) вождей кланов обычно передавались по материнской линии, военных вождей племен --- по отцовской линии. Свита вождей племен и первых королей состояла из глав кланов, было обычаем приглашать их на встречу с королем несколько раз в год. Это позволяло королю наблюдать за ними и лучше укреплять их сотрудничество. Кланы играли важную роль в местном самоуправлении, но они сильно сдерживали развитие крупных и сильных наций.
\usection{5. Начальные формы правления}
\vs p070 5:1 Каждый институт человечества возник в какой\hyp{}то момент, и гражданское правительство является продуктом прогрессивного развития точно так же, как брак, индустрия и религия. Из союзов древних кланов и первобытных племен человечество постепенно сформировало соподчиненные уровни управления, ведущие прямо к тем формам социальной и гражданской регуляции, которые характерны для второй трети двадцатого века.
\vs p070 5:2 С постепенным образованием семьи как ячейки общества в клановой организации, группировке кровно\hyp{}родственных семей, установились основы управления. Первым настоящим правительственным органом был \bibemph{совет старейшин.} Эта регулирующая группа состояла из старых людей, которые проявили себя каким\hyp{}либо действенным образом. Мудрость и опыт рано начали цениться даже варваром, и отсюда тянется длительный период доминирования старейшин. Эта олигархия старости постепенно переросла в патриархат.
\vs p070 5:3 В древнем совете старейшин присутствовали зачатки всех функций правительства: исполнительной, законодательной и судебной. Когда совет объяснял нравы своего времени, он был судом; когда устанавливал новые правила поведения в обществе, он был законодательным органом; по тому, как такие решения и указы проводились в жизнь, он был исполнительным. Глава совета был одним из предшественников вождя племени.
\vs p070 5:4 У некоторых племен были женские советы, и время от времени многие племена имели правителей\hyp{}женщин. Некоторые племена красного человека сохранили учение Онамоналонтона, следуя единогласному правлению «совета семи».
\vs p070 5:5 \P\ Человечество с трудом пришло к пониманию, что вопросы и мира, и войны нельзя решать путем общественных дебатов. Примитивные болтуны редко оказывались полезны. Раса рано познала, что армия под командованием нескольких лидеров из одного клана не имеет шансов противостоять армии под управлением одного человека. Война всегда создавала королей.
\vs p070 5:6 \P\ Сначала военные вожди избирались только для военной службы, и они отказывались от части своей власти в мирное время, когда власть сосретодачивалась на управлении гражданскими делами. Но постепенно они перестали делиться властью и в мирные периоды, стараясь продолжать править и между войнами. Они часто стремились вести одну войну за другой. Этих древних военных правителей мир не приводил в восторг.
\vs p070 5:7 В более поздние времена некоторых вождей выбирали для гражданской деятельности; их избирали за необычные физические или выдающиеся личные качества. У красного человека часто было два состава вождей --- сашемы, или гражданские вожди, и наследственные военные вожди. Гражданские правители были также судьями и учителями.
\vs p070 5:8 Некоторые древние сообщества управлялись знахарями, которые часто действовали как вожди. Один и тот же человек был и жрецом, и лекарем, и главой племени. Довольно часто ранние королевские знаки изначально являлись символами или эмблемами одежды жрецов.
\vs p070 5:9 Целый ряд таких этапов постепенно привел к образованию исполнительной ветви власти. Клановые и племенные советы продолжали существовать в качестве советников и как предшественники появившихся позднее законодательной и судебной ветвей. И сегодня в Африке в различных племенах еще существуют все эти формы примитивного правительства.
\usection{6. Монархическое правление}
\vs p070 6:1 Эффективное управление государством пришло только с появлением вождя, обладающего всей полнотой исполнительной власти. Человек осознал что эффективно управлять можно не путем внушения каких\hyp{}то идей, а только наделив личность властью.
\vs p070 6:2 Управление выросло из идеи о семейной власти или богатстве. Когда патриархальный царек стал настоящим королем, его иногда называли «отцом своего народа». Позднее считалось, что предками королей были герои. И еще позднее правление стало наследственным благодаря вере в божественное происхождение королей.
\vs p070 6:3 Наследственная королевская власть позволила избежать анархии, которая ранее приводила к хаосу в периоды между смертью короля и избранием преемника. Семья имела биологического главу; клан --- избранного природного лидера; племя и, позднее, государство не имело неоспоримого лидера, и это было дополнительным основанием сделать королевское правление наследственным. Идея королевских семей и аристократии также базировалась на обычаях «владения именем» в кланах.
\vs p070 6:4 Право наследования королей в конечном итоге стало восприниматься как нечто сверхъестественное, считалось, что корни королевского рода уходят в прошлое вплоть до времен материализованного штата Принца Калигастии. Так короли стали личностями\hyp{}фетишами, их ужасно боялись, и при дворе даже была принята специальная форма речи. Еще в недавние времена существовало поверье, что прикосновение королей исцелит болезнь, и некоторые народы Урантии по\hyp{}прежнему считают, что их правители имеют божественное происхождение.
\vs p070 6:5 Древний король\hyp{}фетиш часто содержался в изоляции; он считался особой слишком священной и на него можно было смотреть только в праздники и священные дни. Чтобы его олицетворять, обычно избирался представитель, таково происхождение премьер\hyp{}министров. Первым должностным лицом кабинета был распорядитель пищи; вскоре к нему присоединились и другие. Правители стали назначать представителей, ответственных за торговлю и религию; формирование кабинета было прямым шагом к деперсонализации исполнительной власти. Такие помощники древних королей сделались титулованной аристократией, дворянами, и жена короля постепенно обрела сан королевы, поскольку женщины стали пользоваться большим уважением.
\vs p070 6:6 \P\ Беспринципные правители добивались огромной власти с помощью ядов. Древняя придворная магия была жестока; враги короля вскоре умирали. Но даже самый деспотичный тиран был в чем\hyp{}то ограничен; его, по крайней мере, удерживал в определенных рамках постоянный страх перед убийством. Знахари, колдуны\hyp{}врачи и священники всегда были для королей могущественным сдерживающим фактором. Позднее землевладельцы, аристократия, ограничивали его власть. А время от времени кланы и племена просто поднимали восстания и свергали таких деспотов и тиранов. Свергнутым правителям, если их приговаривали к смерти, часто предлагали покончить жизнь самоубийством, что положило начало распространенному в древности обычаю совершать самоубийство в определенных обстоятельствах.
\usection{7. Первые клубы и тайные общества}
\vs p070 7:1 Кровное родство сформировало первые общественные группы; объединение расширило родственный клан. Взаимные браки были следующим шагом в укрупнении групп, а возникшее в результате объединенное племя стало первым настоящим государством. Следующим достижением в социальном развитии была эволюция религиозных культов и политических клубов. Впервые они появились как секретные сообщества и первоначально были исключительно религиозными; впоследствии стали выполнять функцию регулирования. Сначала это были мужские клубы; позднее появились и женские. Вскоре они разделились на два направления: социополитическое и религиозно\hyp{}мистическое.
\vs p070 7:2 \P\ Было много причин, почему эти общества были тайными, в частности:
\vs p070 7:3 \ublistelem{1.}\bibnobreakspace Страх навлечь на себя неудовольствие правителей из\hyp{}за нарушения некоторых табу.
\vs p070 7:4 \ublistelem{2.}\bibnobreakspace Стремление соблюдать религиозные церемонии меньшинства.
\vs p070 7:5 \ublistelem{3.}\bibnobreakspace Охрана ценного «духа», или секретов торговли.
\vs p070 7:6 \ublistelem{4.}\bibnobreakspace Использование определенных тайных заклинаний, или магии.
\vs p070 7:7 \P\ Сама закрытость таких обществ давала всем членам власть таинственности над остальной частью племени. Секретность взывала и к тщеславию, посвященные чувствовали себя аристократией общества своего времени. После инициации мальчики начинали охотиться с мужчинами, тогда как до этого они собирали овощи вместе с женщинами. И высшим унижением, позором племени, было не выдержать испытаний, связанных с достижением половой зрелости, и вследствие этого продолжать пребывать вне мужского жилища вместе с женщинами и детьми и считаться женоподобным. Кроме того, не прошедшим инициацию не разрешалось жениться.
\vs p070 7:8 \P\ Первобытные люди с раннего возраста учили своих подростков сексуальному воздержанию. Стало обычаем забирать мальчиков у родителей с момента зрелости до брака, их образование и обучение было вверено закрытым обществам мужчин. И одной из основных функций этих клубов был контроль за юношами, препятствовавший таким образом появлению незаконнорожденных детей.
\vs p070 7:9 Профессиональная проституция появилась тогда, когда эти мужские клубы стали платить за использование женщин из других племен. Но ранние группы были на редкость целомудренны.
\vs p070 7:10 Церемония инициации при достижении зрелости обычно растягивалась на пять лет. В эти церемонии входило множество самоистязаний и болезненных надрезов. В одной из таких секретных общин делали обрезание как часть обряда инициации. При достижении зрелости на тело стали наносить племенные знаки, что тоже входило в обряд инициации; татуировка появилась как знак принадлежности к обществу. Эти пытки, эти страшные лишения закаливали юношей, знакомили с реалиями жизни и ее неизбежными трудностями. Но появившиеся позднее атлетические игры и спортивные соревнования лучше служили этим же целям.
\vs p070 7:11 Но секретные общества старались поднять моральный уровень юношества; одной из основных задач церемоний при достижении половозрелости было стремление внушить мальчику, что он не должен домогаться чужих жен.
\vs p070 7:12 Вслед за годами суровой дисциплины и обучения и сразу перед женитьбой молодым мужчинам обычно предоставляли короткий период досуга и свободы, потом они возвращались, женились и на всю жизнь подчинялись племенным табу. И этот древний обычай дожил и до настоящего времени как глупое представление о необходимости «перебеситься».
\vs p070 7:13 \P\ Позже во многих племенах одобряли образование женских закрытых клубов, целью которых была подготовка девочек\hyp{}подростков к браку и материнству. После инициации девочки были готовы к браку, и им разрешалось принять участие в «выставке невест», выйти в свет тех дней. Рано возникли женские ордена с обетом безбрачия.
\vs p070 7:14 Вскоре появились открытые клубы, где группы холостых мужчин и незамужних женщин составляли свои отдельные организации. Эти ассоциации на самом деле можно считать первыми школами. И хотя мужские и женские клубы часто преследовали друг друга, отдельные развитые племена, особенно после контакта с учителями Даламатии, пробовали ввести совместное обучение, учредив для этого школы\hyp{}интернаты для обоих полов.
\vs p070 7:15 \P\ Закрытые общества помогли построению социальных каст, главным образом благодаря таинственному характеру своей инициации. Члены этих обществ сначала носили маски, чтобы отпугнуть любопытных от своих ритуалов оплакивания --- поклонения предкам. Позднее этот ритуал превратился в псевдоспиритический сеанс, на котором, как считалось, появятся духи. Древние общества «нового рождения» использовали знаки и разработали специальный тайный язык; они к тому же отказались от определенной пищи и напитков. Они действовали как ночная полиция и, помимо того, принимали участие в самой различной социальной деятельности.
\vs p070 7:16 Все закрытые общества обязывали давать клятву, поощряли доверие и учили хранить секреты. Эти ордена внушали страх и контролировали толпу; это были охранительные общества, практикующие самосуд. Они были первыми шпионами, когда племена находились в состоянии войны, и первой тайной полицией в периоды мира. Самое главное, они заставляли неправедных королей все время быть в напряжении. Чтобы противостоять им, короли основали собственную тайную полицию.
\vs p070 7:17 Из этих обществ вышли первые политические партии. Первым партийным правительством стали «сильные» против «слабых». В древние времена смена правления происходила только через гражданскую войну, что всецело доказывало --- что слабые стали сильными.
\vs p070 7:18 Эти клубы использовались торговцами, чтобы взымать долги, и правителями для сбора налогов. Налогообложение потребовало длительных усилий, одной из самых ранних его форм была десятина --- одна десятая часть охотничьей добычи или награбленного. Вначале налоги взымались, чтобы содержать королевский дом, но потом посчитали, что их легче собирать под видом пожертвований на содержание храмовой службы.
\vs p070 7:19 Постепенно эти закрытые общества переросли в первые благотворительные организации и позднее развились в самые ранние религиозные общества --- предшественники церквей. Наконец, некоторые из этих обществ стали межплеменными, первыми международными братствами.
\usection{8. Социальные классы}
\vs p070 8:1 Умственное и физическое неравенство человеческих существ стало причиной появления социальных классов. Единственными мирами без социальных слоев являются самые примитивные и самые развитые. Древняя цивилизация еще не подверглась социальному расслоению, а мир, установленный в свете и жизни, в значительной степени стирает такое деление человечества, столь характерное для всех промежуточных стадий развития.
\vs p070 8:2 По мере того как общество продвигается от дикости к варварству, люди, его составляющие, начинают группироваться в классы по следующим основным причинам:
\vs p070 8:3 \ublistelem{1.}\bibnobreakspace \bibemph{Естественная ---} знакомство, родство и брак; первые социальные различия основывались на половых различиях, возрасте и крови --- родстве с вождем.
\vs p070 8:4 \P\ \ublistelem{2.}\bibnobreakspace \bibemph{Личная ---} признание способности, стойкости, мастерства и мужества; вскоре последовало признание мастерского владения языком, знания и вообще интеллекта.
\vs p070 8:5 \P\ \ublistelem{3.}\bibnobreakspace \bibemph{Случайная ---} война или эмиграция приводила к разделению человеческих групп. На классовую эволюцию сильно влияло завоевание, отношение победителя к побежденному, рабство же привело к первому тотальному разделению общества на свободных и рабов.
\vs p070 8:6 \P\ \ublistelem{4.}\bibnobreakspace \bibemph{Экономическая ---} богатый и бедный. Богатство и владение рабами стало генетическим базисом одного из классов общества.
\vs p070 8:7 \P\ \ublistelem{5.}\bibnobreakspace \bibemph{Географическая ---} классы появились вслед за возникновением городских и сельских поселений. Крупные города и сельские поселения соответственно способствовали различиям между торговцем\hyp{}промышленником и скотоводом\hyp{}крестьянином, с их несхожими взглядами и привычками.
\vs p070 8:8 \P\ \ublistelem{6.}\bibnobreakspace \bibemph{Социальная ---} классы постепенно сформировались в соответствии с общепринятой оценкой социальной значимости различных групп. Среди самых ранних расслоений такого рода были разграничения между священниками\hyp{}учителями, правителями\hyp{}воинами, капиталистами\hyp{}торговцами, работниками и рабами. Раб никогда не мог стать капиталистом, а работнику иногда мог открыться путь в ряды капиталистов.
\vs p070 8:9 \P\ \ublistelem{7.}\bibnobreakspace \bibemph{Профессиональная ---} по мере того как множились профессии, начинали образовываться касты и гильдии. Работающие разделились на три группы: профессионалов, включающие знахарей; квалифицированных работников, и неквалифицированных работников.
\vs p070 8:10 \P\ \ublistelem{8.}\bibnobreakspace \bibemph{Религиозная ---} древние культовые касты создали собственные классы внутри кланов и племен, а благочестивость и мистицизм священников в течение долгого времени помогали им выступать как отдельная социальная группа.
\vs p070 8:11 \P\ \ublistelem{9.}\bibnobreakspace \bibemph{Расовая ---} наличие двух или более рас в пределах данной нации или территории обычно формировало цветные касты. Первоначальная кастовая система в Индии, так же как и в древнем Египте, возникла на базе деления по цвету кожи.
\vs p070 8:12 \P\ \ublistelem{10.}\bibnobreakspace \bibemph{Возрастная ---} юность и зрелость. В племенах мальчик находился под присмотром отца на протяжении всей жизни последнего, тогда как девочка оставалась на попечении матери только до замужества.
\vs p070 8:13 \P\ Гибкие и меняющиеся социальные классы незаменимы для развивающейся цивилизации, но когда \bibemph{класс} становится \bibemph{кастой,} когда социальные слои «окаменевают», повышение социальной стабильности оплачивается сокращением личной инициативы. Социальная каста решает проблему получения места в производстве, но при этом резко сокращается индивидуальное развитие, что в конечном итоге препятствует социальному сотрудничеству.
\vs p070 8:14 Классы в обществе, сформировавшись естественным путем, сохранятся до тех пор, пока человек постепенно не достигнет их эволюционного уравнивания путем разумного манипулирования биологическими, интеллектуальными и духовными ресурсами прогрессирующей цивилизации, такими как:
\vs p070 8:15 \ublistelem{1.}\bibnobreakspace Биологическое обновление расовых ветвей --- селективное устранение низших человеческих линий. Это послужит искоренению многих различий между людьми.
\vs p070 8:16 \ublistelem{2.}\bibnobreakspace Развитие умственных способностей путем образования, которое последует за таким биологическим улучшением.
\vs p070 8:17 \ublistelem{3.}\bibnobreakspace Расцвет религиозного чувства родства и братства смертных.
\vs p070 8:18 \P\ Но такие меры смогут принести настоящие плоды только в отдаленных тысячелетиях будущего, хотя значительные сдвиги немедленно последуют за разумным, мудрым и \bibemph{терпеливым} использованием этих ускоряющих факторов культурного прогресса. Религия является могучим рычагом, который поднимает цивилизацию из хаоса, но она беспомощна, если не опирается на здоровый и нормальный разум, прочно базирующийся на здоровой и нормальной наследственности.
\usection{9. Права человека}
\vs p070 9:1 Природа кроме жизни и мира, чтобы в нем жить, никаких других прав человеку не дарует. Природа даже не дарует права на жизнь, как можно заключить, представив себе, что скорее всего произойдет, когда безоружный человек встретится в первобытном лесу с голодным тигром. Основным даром общества человеку является безопасность.
\vs p070 9:2 \P\ Постепенно общество утверждает свои права и в настоящее время они следующие:
\vs p070 9:3 \ublistelem{1.}\bibnobreakspace Гарантия снабжения пищей.
\vs p070 9:4 \ublistelem{2.}\bibnobreakspace Военная защита --- безопасность благодаря боеготовности.
\vs p070 9:5 \ublistelem{3.}\bibnobreakspace Сохранение внутреннего мира --- предотвращение насилия над личностью и социальных беспорядков.
\vs p070 9:6 \ublistelem{4.}\bibnobreakspace Регулирование сексуальной жизни --- брак и институт семьи.
\vs p070 9:7 \ublistelem{5.}\bibnobreakspace Собственность --- право на владение.
\vs p070 9:8 \ublistelem{6.}\bibnobreakspace Поощрение личной и групповой конкуренции.
\vs p070 9:9 \ublistelem{7.}\bibnobreakspace Обеспечение образования и обучения молодежи.
\vs p070 9:10 \ublistelem{8.}\bibnobreakspace Содействие торговле и коммерции --- индустриальное развитие.
\vs p070 9:11 \ublistelem{9.}\bibnobreakspace Улучшение условий труда и его оплаты.
\vs p070 9:12 \ublistelem{10.}\bibnobreakspace Гарантия свободы вероисповедания для того, чтобы все остальные виды общественной деятельности могли бы стать духовно мотивированными и, следовательно, возвышенными.
\vs p070 9:13 \P\ Когда права древнее, чем знание об их происхождении, их часто называют \bibemph{естественными правами.} Но человеческие права не являются по\hyp{}настоящему естественными; они полностью социальны. Они относительны и постоянно изменяются, являясь ничем иным как правилами игры --- признанным регулированием отношений, управляющих постоянно изменяющимися феноменами человеческой конкуренции.
\vs p070 9:14 То, что может считаться правом в одну эпоху, может не считаться таковым в другую. Выживание большого числа людей с дефектами и дегенератов происходит не потому, что они имеют какое\hyp{}либо естественное право так обременять цивилизацию двадцатого века, но только потому, что нравы общества этого периода, само общество так решило.
\vs p070 9:15 В Средние века в Европе признавалось всего несколько прав личности; тогда каждый человек принадлежал кому\hyp{}нибудь еще и права часто сводились просто к привилегиям или были подарком государства или церкви. Но и отступление от такого подхода было в равной мере ошибочно, поскольку внушало веру, что все люди рождены равными.
\vs p070 9:16 Слабые и низшие всегда боролись за равные права; они всегда настаивали, чтобы государство вынуждало сильных и высших обеспечивать их потребности или другим образом компенсировать то неравенство, которое являлось естественным результатом их собственного безразличия и лености.
\vs p070 9:17 Но идеал равенства --- дитя цивилизации; его нет в природе. Даже культура сама по себе убедительно показывает присущее людям неравенство, вызванное их неравными способностями. Внезапная и неэволюционная реализация предполагаемого природного равенства быстро отбросит цивилизованного человека назад к грубым обычаям примитивных веков. Общество не может предоставить равных прав всем, но оно может обещать управлять различающимися правами каждого справедливо и беспристрастно. Задача и обязанность общества обеспечить дитя природы справедливой и мирной возможностью самоподдержания, участия в самовоспроизведении, в то же самое время получения некоторого самоудовлетворения --- этих трех составляющих человеческого счастья.
\usection{10. Эволюция справедливости}
\vs p070 10:1 Естественная справедливость --- это созданная человеком теория, а не реальность. В природе справедливость является чисто теоретической, полностью вымышленной. В природе есть лишь один вид справедливости --- неизбежная причинно\hyp{}следственная связь.
\vs p070 10:2 Справедливость, как она представляется человеку, означает обретение человеком его прав и, таким образом, является результатом прогрессивной эволюции. Концепция справедливости может быть существенной составляющей духовно одаренного разума, но в мирах космоса она не возникает полностью развитой сама по себе.
\vs p070 10:3 Примитивный человек приписывал все события личности. В случае смерти дикарь спрашивал не \bibemph{что} его убило, но \bibemph{кто.} Случайное убийство, таким образом, не признавалось, а при наказании за преступление мотив преступника совершенно не принимался во внимание; приговор выносился в соответствии с нанесенным ущербом.
\vs p070 10:4 \P\ В самом древнем примитивном обществе общественное мнение действовало непосредственно; служители закона были не нужны. В первобытной жизни не было личностного. Соседи человека отвечали за его поступки; поэтому они имели право вмешиваться в его дела. Общество базировалось на положении, при котором группа должна интересоваться и иметь некоторую степень контроля над поведением каждого человека.
\vs p070 10:5 Рано возникла вера в то, что духи осуществляют правосудие через знахарей и священников; появились первые расследователи преступлений и служители закона. Их ранние методы расследования преступления заключались в жестоких испытаниях ядом, огнем и болью. Такие зверства были ничем иным, как примитивным третейским судом; спор не обязательно разрешался справедливо. Например, когда обвиняемого приговаривали к испытанию ядом, и его при этом рвало, именно он объявлялся невиновным.
\vs p070 10:6 Ветхий Завет увековечил одно из таких жестоких испытаний --- проверку супружеской верности: если мужчина подозревал, что его жена ему неверна, он приводил ее к священнику и заявлял о своих подозрениях, после чего священник готовил состав из святой воды и мусора с пола храма. После должной церемонии, включающей пугающие проклятия, подозреваемую жену заставляли пить эту гадость. Если она была виновной, «вода, которая вызывает проклятие, войдет в нее и станет горькой, и ее чрево разбухнет, и ее бедра будут гнить, и женщина должна быть проклята среди ее народа». Если оказывалось, что какая\hyp{}нибудь женщина могла глотнуть этого отвратительного снадобья и не проявить симптомов физической болезни, она оправдывалась от подозрений, выдвинутых ее ревнивым супругом.
\vs p070 10:7 Эти отвратительные методы расследования преступления в то или иное время использовались почти всеми развивающимися племенами. Дуэль является современным пережитком суда через испытание.
\vs p070 10:8 Не следует удивляться, что три тысячи лет назад иудеи и другие полуцивилизованные племена использовали такие примитивные методы отправления правосудия, но изумляет то, что здравомыслящие люди впоследствии будут сохранять сведения о таких пережитках варварства на страницах священных писаний. Глубокое размышление позволяет прийти к четкому выводу, что никакое божественное существо никогда не давало смертному человеку таких несправедливых указаний относительно как расследования, так и вынесения судебного решения о предполагаемой супружеской неверности.
\vs p070 10:9 \P\ Общество рано усвоило принципы возмездия: око за око, жизнь за жизнь. Все развивающиеся племена признавали право кровной мести. Месть стала целью примитивной жизни, но с тех пор религия сильно изменила эти древние племенные обычаи. Учителя религии откровения всегда провозглашали «„Месть --- моя“, сказал Господь». Убийство из мести в древние времена в целом не так уж и отличалось от современных убийств с претензией на неписаный закон.
\vs p070 10:10 Самоубийство было обычной формой возмездия. Если некто не мог отомстить за себя при жизни, он лишал себя жизни с верой в то, что как призрак сможет вернуться и излить гнев на своего врага. И поскольку вера в это была общепринятой, угрозы самоубийства на пороге дома врага было обычно достаточно, чтобы заставить того пойти на мировую. Примитивный человек не очень дорожил жизнью; самоубийство из\hyp{}за пустяков было обычным делом, но учения Даламатинцев сильно смягчили распространенность этого обычая, а в более поздние времена досуг, комфорт, религия и философия сделали жизнь и более приятной, и более желанной. Однако голодовки --- это современный аналог такого древнего метода возмездия.
\vs p070 10:11 Одно из самых древних толкований развитого племенного закона рассматривало кровную вражду как дело, касающееся племени. Но, как ни странно, даже тогда человек мог убить свою жену и не понести никакого наказания, если полностью заплатил за нее. Современные эскимосы, однако, все еще оставляют право назначения и исполнения наказания за преступление, даже за убийство, пострадавшей семье.
\vs p070 10:12 Другим достижением было наложение штрафа как меры взыскания наказания за нарушение табу. Такие штрафы составляли первые общественные доходы. Обычай платить «деньги за кровь», заменяя кровную месть, также вошел в обиход. В виде компенсации обычно отдавали женщин или скот, это было задолго до введения настоящих штрафов --- денежной компенсации как наказания за преступление. И поскольку идея наказания базировалась по сути на компенсации, то все, включая человеческую жизнь, в конечном итоге получило цену, которую должно было уплатить за причиненный ущерб. Иудеи были первыми, кто отменил обычай уплаты «денег за кровь». Моисей учил: «не бери выкупа за душу убийцы, который повинен смерти, но его должно предать смерти».
\vs p070 10:13 \P\ Таким образом, правосудие сначала осуществлялось семьей, потом кланом и позднее племенем. Отправление подлинного правосудия началось с момента изъятия права на месть из рук частных лиц и родственных групп и передачи его в руки социальной группы, государства.
\vs p070 10:14 \P\ Наказание сожжением заживо было когда\hyp{}то очень распространено. Оно признавалось многими древними правителями, включая Хаммурапи и Моисея; последний приказывал, чтобы многие преступления, особенно серьезные сексуальные, наказывались сожжением на костре. Если «дочь священника» или другого знатного гражданина начинала заниматься проституцией, по обычаям иудеев надо было «сжечь ее огнем».
\vs p070 10:15 Измена --- «продажа» или предательство собственной племенной группы --- была первым государственным преступлением. Кража скота повсеместно наказывалась смертью на месте, и даже недавно кража лошадей наказывалась так же. Но с течением времени стало ясно, что суровость наказания как средство устрашения была менее важна по сравнению с неотвратимостью и быстротой.
\vs p070 10:16 Когда общество не наказывало преступления, групповое возмущение обычно отстаивало свои права через самосуд; предоставление убежища было способом избежать этого неожиданного гнева толпы. Линчевание и дуэль представляют собой нежелание индивидуума уступить свое право восстановить справедливость государству.
\usection{11. Законы и суды}
\vs p070 11:1 Провести строгое разграничение между нравами и законами так же трудно, как и точно определить момент рассвета, когда ночь сменяется днем. Нравы являются и законами, и полицейскими правилами в процессе становления. Существуя достаточно долго, нечеткие нравы обычно кристаллизуются в точные законы, конкретные правила и четко сформированные общественные договоры.
\vs p070 11:2 Закон всегда является сначала отрицающим и запрещающим; в развивающихся цивилизациях он постепенно становится утверждающим и направляющим. Древнее общество управлялось отрицанием, предоставляя индивидууму право жить, навязывая всем остальным правило «ты не должен убивать». Каждое предоставление прав или свободы индивидууму требует урезания свобод всех прочих, и это осуществляется с помощью табу, примитивного закона. Вся идея табу изначально является запретом, поскольку примитивное общество в своей организации было исключительно запрещающим, и древнее отправление правосудия состояло в обеспечении соблюдения табу. Но сначала эти законы применялись только к соплеменникам; в более поздние времена, например, иудеи свои отношения с неевреями определили уже отдельным этическим кодексом.
\vs p070 11:3 Клятва появилась в эпоху Даламатии как попытка сделать свидетельство более правдивым. Такие клятвы сводились к произнесению проклятия на собственную голову. До того ни один человек не стал бы свидетельствовать против своей родной группы.
\vs p070 11:4 \P\ Преступлением было оскорбление племенных нравов, грехом было нарушение табу, которые пользовались одобрением духов; существовала путаница в определении преступления и греха из\hyp{}за неспособности отделить их друг от друга.
\vs p070 11:5 Собственные интересы установили табу на убийство, общество утвердило его в качестве традиционных норм поведения, тогда как религия освятила этот обычай как моральный закон, и в результате такого тройного действия человеческая жизнь стала более безопасной и священной. Общество не могло бы существовать в древние времена, если бы права не были одобрены религией; предрассудок был моральной и общественной полицией на протяжении долгих веков эволюции. Все древние провозглашали, что их древние законы, табу, были даны их предкам богами.
\vs p070 11:6 Закон является кодифицированной записью длительного человеческого опыта, выкристаллизованного и узаконенного общественного мнения. Нравы были сырым материалом аккумулированного опыта, из которого позднее правящие умы сформулировали писаные законы. Древний судья не имел законов. Когда он выносил решение, то просто говорил: «Это обычай».
\vs p070 11:7 Ссылка на прецедент в судебных решениях представляет собой попытку судей адаптировать написанные законы к изменяющимся условиям общества. Это обеспечивает постепенную адаптацию к изменяющимися социальными условиями, в сочетании с впечатляющей непрерывностью традиции.
\vs p070 11:8 \P\ Имущественные споры разрешались многими путями, например:
\vs p070 11:9 \ublistelem{1.}\bibnobreakspace Уничтожением спорной собственности.
\vs p070 11:10 \ublistelem{2.}\bibnobreakspace Силой --- спорящие дрались между собой.
\vs p070 11:11 \ublistelem{3.}\bibnobreakspace Арбитражем --- решением третьей стороны.
\vs p070 11:12 \ublistelem{4.}\bibnobreakspace Обращение к старейшинам, позднее к судам.
\vs p070 11:13 \P\ Первые суды были кулачными боями, проводившимися по правилам; судьи были не более чем арбитрами или рефери. Они следили, чтобы драка велась в соответствии с принятыми правилами. Вступая в судебную тяжбу, каждая из сторон оставляла залог судье, чтобы оплатить расходы и штраф после победы одного над другим. «Сила все еще была правом». Позднее словесные прения заменили физические удары.
\vs p070 11:14 Вся идея примитивного правосудия заключалась не столько в том, чтобы быть справедливым, сколько в желании избавиться от спора и таким образом предупредить общественные беспорядки и частные проявления насилия. Но примитивный человек не так уж сильно обижался на то, что сегодня считалось бы несправедливостью; как должное воспринимался факт, что люди, обладающие властью, использовали ее на благо себе. Тем не менее, состояние любой цивилизации очень точно определяется основательностью и беспристрастностью ее судов и честностью судей.
\usection{12. Распределение гражданской власти}
\vs p070 12:1 В процессе эволюции системы управления огромные усилия были направлены на концентрацию власти. Руководители вселенной познали на опыте, что эволюционирующие народы населенных миров лучше всего управляются гражданским правительством представительного типа, при условии надлежащего баланса власти между хорошо скоординированными исполнительной, законодательной и судебной ветвями.
\vs p070 12:2 \P\ Если примитивная власть опиралась на силу, физическую мощь, то идеальное правительство является представительной системой, в которой лидерство основывается на способности; но во времена варварства в целом велось слишком много войн, а это не позволяло представительному правительству эффективно функционировать. В долгой борьбе между разделением власти и единством управления победил диктатор. Древние и нечетко очерченные права примитивного совета старейшин постепенно концентрировались в личности абсолютного монарха. После появления королей группы старейшин сохранились как квази\hyp{}законодательно\hyp{}судебные совещательные органы; позднее возникли законодательные органы с координирующим статусом, и в конечном итоге независимо от законодательных органов были основаны верховные суды.
\vs p070 12:3 Король был исполнителем нравов, изначального, или неписаного закона. Позднее он ввел в силу законодательные указы, выкристаллизованное общественное мнение. Народное собрание, как выразитель общественного мнения, хотя и возникало очень медленно, ознаменовало собой огромное социальное достижение.
\vs p070 12:4 Древние короли были сильно ограничены нравами --- традицией или общественным мнением. В недавние времена некоторые нации Урантии кодифицировали эти нравы в качестве документальной основы для системы управления.
\vs p070 12:5 \P\ Смертные Урантии имеют право на свободу; они должны создать свои системы управления, им следует принять свои конституции или другие хартии гражданской власти и исполнительной процедуры. И совершив это, им следует выбрать наиболее компетентных и достойных сограждан в качестве глав исполнительной власти. Для представителей законодательной ветви им необходимо выбирать только тех, кто обладает достаточным интеллектом и моралью, чтобы исполнять такие священные обязанности. В качестве судей высоких и верховных трибуналов должны быть выбраны только те, кто одарен природной способностью и приобрел мудрость благодаря своему богатому опыту.
\vs p070 12:6 Если люди хотят сохранить свою свободу, они должны, избрав свою хартию вольности, обеспечить возможность мудрой, разумной и безбоязненной ее интерпретации, чтобы предотвратить:
\vs p070 12:7 \ublistelem{1.}\bibnobreakspace Узурпацию незаконной власти либо исполнительной, либо законодательной ветвями.
\vs p070 12:8 \ublistelem{2.}\bibnobreakspace Махинации невежественных и суеверных агитаторов.
\vs p070 12:9 \ublistelem{3.}\bibnobreakspace Замедление научного прогресса.
\vs p070 12:10 \ublistelem{4.}\bibnobreakspace Застой из\hyp{}за доминирования посредственности.
\vs p070 12:11 \ublistelem{5.}\bibnobreakspace Доминирование порочных меньшинств.
\vs p070 12:12 \ublistelem{6.}\bibnobreakspace Контроль со стороны амбициозных и хитрых потенциальных диктаторов.
\vs p070 12:13 \ublistelem{7.}\bibnobreakspace Губительное разрушение, вызванное паникой.
\vs p070 12:14 \ublistelem{8.}\bibnobreakspace Эксплуатацию со стороны беспринципных людей.
\vs p070 12:15 \ublistelem{9.}\bibnobreakspace Порабощение граждан налогами со стороны государства.
\vs p070 12:16 \ublistelem{10.}\bibnobreakspace Нарушения социальной и экономической справедливости.
\vs p070 12:17 \ublistelem{11.}\bibnobreakspace Объединение церкви и государства.
\vs p070 12:18 \ublistelem{12.}\bibnobreakspace Потерю личной свободы.
\vs p070 12:19 \P\ Перечисленные условия составляют цели и задачи конституционных трибуналов, действующих как контрольные органы представительного правительства в развивающемся мире.
\vs p070 12:20 Усилия человечества, направленные на усовершенствование системы управления на Урантии, касаются формирования систем управления, адаптации к постоянно изменяющимся текущим потребностям, улучшения распределения власти в правительстве и затем выбора по\hyp{}настоящему мудрых лидеров\hyp{}правителей. Хотя и существует божественная и идеальная форма правления, она не может быть вам раскрыта; ее надо методично и усердно искать мужчинам и женщинам каждой планеты во вселенных времени и пространства.
\vs p070 12:21 [Представлено Мелхиседеком из Небадона.]
